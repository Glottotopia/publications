\section*{List of contributors}
\textbf{Ian Smith} holds a doctorate in Linguistics from Cornell University. He has taught at Monash University, the National University of Singapore, the University of Sydney, and at York University, where he is Associate Professor of Linguistics. His primary research field is language contact. The bulk of his work focuses on South Asian contact languages, particularly Sri Lanka Portuguese, Sri Lanka Malay, and Sourashtra. He has also published on English and Kugu Nganhcara. 

\textbf{Scott Paauw} is an Assistant Professor of Linguistics at the University of Rochester, where he has taught since 2005. He has a PhD in Linguistics from the University at Buffalo. He has worked on Sri Lanka Malay since 2003. His primary areas of interest are language contact, language description and documentation, pidgins and creoles, bilingual issues, historical linguistics, typology, applied linguistics and sociolinguistics. He is particularly interested in these topics as they relate to Malay/Indonesian, Austronesian languages, the languages of Papua, and the languages of South Asia. 

\textbf{Romola Rassool} is presently working towards the completion of her doctoral thesis at the Asia Institute, University of Melbourne, Australia. Her thesis, which is on sociolinguistic aspects of the Sri Lanka Malay language, is based on extensive fieldwork she undertook in the Western, Eastern, Central and Southern Provinces of Sri Lanka. She is presently on study leave from the University of Kelaniya, Sri Lanka, where she holds the position of senior lecturer in English as a Second Language at the English Language Teaching Unit.

\textbf{Sebastian Nordhoff} acquired his PhD from the University of Amsterdam with \em A Grammar of Upcountry Sri Lanka Malay\em. His main interest are language description and documentation, language contact, and language change. He is also the author of the grammar authoring platform galoes.org as well as of glottolog.org, a knowledgebase of references to linguistic literature and genealogical affiliations, which was constructed at his current workplace, the Max Planck Institute for Evolutionary Anthropology.

\textbf{Mohamed Jaffar}, a native Sri Lankan Malay, has steeped himself in the culture, lore and language of his community since his early years.  He counts himself among a  fortunate few, among Sri Lankan Malays, that can still decipher and read with a great deal of fluency old extant texts in Ceylon Malay `gundul', the modified Arabic script also known as `huruf Jawi'.
Jaffar is a founder member of the old \em Alliance Française de Ceylan \em and is a self-confessed Francophile.  While in Sri Lanka he worked for \em La Compagnie des Messageries Maritimes\em, and later on, up to his emigration to Canada in 1974, for  \em L'Union des Transports Aériens\em.  He became active in the \em Sri Lanka Malay Association of Toronto \em and served as its president. In June 2012 he graduated from Toronto’s York University with a specialist B.A. majoring in Linguistics and French.
His particular interests comprise reviving lost lexicon, devising a system of Romanized spelling that will prove user-friendly, and, eventually, a primer in the English medium for autodidacts.

\textbf{Peter Slomanson} received his Ph.D. in 2005 from the City University of
New York. He conducted post-doctoral fieldwork in Sri Lanka, funded by
the American Institute for Sri Lankan Studies, and subsequently taught
at the College of Staten Island. He has lectured since 2009 at the
University of Aarhus, in Denmark. Dr. Slomanson is also affiliated
with the Information Structuring and Typology research group of the
French \em Centre National de Recherche Scientifique\em. His research
involves modeling non-canonical contact language development, and the
roles of discourse-pragmatic accommodation and reanalysis in the
development of contact language and second language grammars.

\textbf{Peter Bakker} is a linguist at Aarhus University in Denmark. He is a graduate of the University of Amsterdam (1992), where he defended his dissertation on Michif, an endangered Cree-French mixed language in Canada. He has written extensively on language contact, especially on new languages that have emerged out of contact, and the social and historical reasons thereof. His publications range from languages created by twins to Basque-based pidgins in Canada, purism in mixed languages and Saramaccan creole of
Suriname. 