\chapter{Sri Lanka Malay: new findings on contacts}

\chapterauthor{Peter Bakker}{Aarhus University}

\section{Introduction}\label{bakker:sec:1}

The purpose of this paper is threefold. One is historical. It will shed a light on the vexing question of contacts and \textbf{intermarriages} between the Sri Lanka Malays and the Moors that have divided the students of Sri Lanka Malay (SLM). Evidence will be discussed on the basis of biological literature that sheds light on these questions, namely a previously uncited study on the molecular genetics of Sri Lankan populations (Section \ref{bakker:sec:2}).

The second vexing question is just as controversial. It has become known as the so-called Tamil bias. Some people, including myself, have suggested Tamil as a source for the typological conversion of some form of colloquial, more or less isolational Malay towards the rather agglutinative and verb-final form of Sri Lanka Malay (SLM) that we find today. Others, notably Ansaldo and Nordhoff, claim that Sinhala was an important source, if not necessarily the main source, of the changes in SLM. I will provide background material that will shed light on the question, both from historical, and linguistic vantage points. On the basis of historical sources, most of them not previously quoted in the discussion of SLM among linguists, I will show that the connections between the Malays and the Sinhalese were rather tense in most of the historical period (Section \ref{bakker:sec:3}). I will also show that, from a linguistic perspective, the influences from Tamil and Sinhala are compatible with early contact with Tamil and later contact with Sinhala (Section \ref{bakker:sec:4}). Sociolinguistically also, the evidence is overwhelming for contacts with Tamil rather than with Sinhala. All of these more or less historical questions are important for a third question, and that is the genesis of the SLM language, to be discussed in Section \ref{bakker:sec:6}. Before that, in Section \ref{bakker:sec:5}, I will discuss whether we can call SLM a creole language  – based on a historical and typological arguments. In order to be able to say anything about the genesis of the language, any answer should be informed by the historical data, even if they are limited. This scenario is not new, but it is based on the appropriate contemporary sources. The final section (\ref{bakker:sec:6}) will place SLM in a wider perspective.
   

\section{Malay intermarriage: genetic data on Sri Lanka Malays}\label{bakker:sec:2}
One point of controversy is whether or not there was no, some or considerable intermarriage between Tamil speakers (mostly Moors, of partial Arabic descent) and Sri Lanka Malays. See \citet{Slomansontv}. In the travel literature I have not encountered a great deal of discussion on that theme (see Section \ref{bakker:sec:3}). In this section I will to summarize an article (perhaps the only one on the subject that includes data on Sri Lanka Malays) dealing with intermarriage between the major population groups of Sri Lanka, based on human genetics data.

\citet{PapihaEtAl1996} gathered genetic data from 508 individuals representing the Sinhalese, Tamils, Moors, Burghers and Malays of Sri Lanka, and from each of these five groups at least 100 persons were investigated. For this project, the authors studied a large number of loci of genetic variation. They compared the groups with each other and with other groups such as the Portuguese, Dutch, Arabs, and Malays of Malaysia and Singapore, in order to ``clarify the range of genetic variation among the populations of Sri Lanka'' \citep[707]{PapihaEtAl1996}.

Before going into their data, it may be useful to briefly introduce the groups studied (see also \citet{Coperahewa2009} for sociolinguistic data on Sri Lanka languages). The Sinhalese are the largest community of Sri Lanka, with ca. 74\% of the population in 1981 (apparently the most recent census data available before the sampling of subjects at that time); they are mostly Buddhists and speak an Indo-Aryan language. The Tamils consist of two groups, the Sri Lanka Tamils (12.8\%) and Indian Tamils (5.4\%). The first group arrived in waves at an unspecified time, up to millennia ago, and the Indian Tamils arrived in recent centuries. The Tamils are mostly Hindus. Due to frequent intermarriage, the Papiha study groups two Tamil populations together. Tamil is a Dravidian language. The Moors comprise 7,1\% of the Sri Lankan population. They descend from Arab traders who came to Sri Lanka in the 9\textsuperscript{th} and 10\textsuperscript{th} centuries, and intermarried with local people. The native language of the Moors is a non-standard variety of Tamil. The Moors are Muslims. The Burghers are descendants of Dutch and Portuguese settlers who intermarried with Sinhalese (Tamils are not mentioned, but as Nordhoff, p.c., 2011, points out, Tamils as well), or locals who identified with them. They are Christians. The Malays came first from  Java and other parts of what is now Indonesia, and subsequently from Malaysia. They are Muslims. The Burghers and Malays comprise less than one percent of the population. The Veddahs, the aboriginal inhabitants of Sri Lanka, and other minor groups, were not studied (an earlier study had confirmed the genetic distinctiveness of the Veddas, \citet[708]{PapihaEtAl1996}; see also \citet{Stoudt1961}).

Papiha et al. conclude, confirming earlier studies, that the Tamils and Sinhalese are quite similar to each other genetically, and that the Moors, the Burghers and the Malays are dissimilar from the two major groups. They state that ``[o]verall, genetic heterogeneity among the five populations was statistically significant'', and also that there is ``significant variation among subpopulations'' (p. 726). The distinctiveness of the Malays with respective to the other groups suggests that there was not so much intermarriage of the Malays with other groups, even though Papiha et al. remark that ``The Malays have married Sinhalese Tamil and Muslim women'' (p. 711), but this statement does not seem to be based on the genetic research they report on. See Figure \ref{fig:bakker1}, which is Figure 3 from Papiha's article (1996:732). 

\begin{figure}%1
\includegraphics[width=\textwidth]{\imgpath bakker-img3}
\caption[Genetic distances between Sri Lankan populations]{A tree of genetic distances between Sri Lankan populations \citep[from][732]{PapihaEtAl1996}.}
\label{fig:bakker1}
\end{figure}


Papiha et al. have studied several frequencies of genetic signals that can be found in blood samples. Looking at their findings about the Malays in more detail, they found that the Malays completely lacked one type of haplotype (a set of distinctive genetic markers in DNA sequences), whose lack is ``a typical characteristic of most Asian populations of Southeast Asia'' \citep[719]{PapihaEtAl1996}. This suggests that the Sri Lanka Malays have remained relatively close to their pre-Sri Lanka Malay genetic profile. In fact, data show that ``the genetic composition of the Sri Lankan Malays has been preserved and still resembles the original populations.'' (p. 733-734). Another marker also ``indicates their affinity with Asian populations'' (p. 723) and the mean heterozygosity was lowest among the Malays, which suggests ``some degree of isolation'' and preferred marriage partners among their own group (p. 726). Among the Malays, ``the effect of admixture is not as pronounced'' (p. 733). For another marker, ``the Malays differed significantly from the Sinhalese'' (p. 720), and for yet another one the ``differences between the Malays and Tamils were also statistically significant'' (p. 724). For another marker, the Malays differed significantly from the Tamils and Burghers (p. 721), Elsewhere, the Malays differ significantly from all the other groups (p. 723-724), or from the Sinhalese and the Tamils (p. 725), or from the Moors (p. 725). 

On the basis of these data it is possible to estimate how endogamous the different Sri Lankan groups are. The Malays appeared to be the most endogamous, ``possibly resulting from social, professional and religious isolation'' (p. 728) of all the Sri Lanka subpopulations. An earlier study \citep[43]{Stoudt1961}, on the basis of measurement of physiological characteristics, had concluded the same: ``The Moors and Malays, as Muslims, are both endogamous groups and have virtually no marital and few extra-marital sexual relations with the other races of the island''. 

Figure \ref{fig:bakker1} showed the genetic differentiation between the five Sri Lankan groups.  \citet{PapihaEtAl1996} also compared them with groups from outside the island. When Arabs, Dutch, and Portuguese genetic data are included, the result is a tree as in Figure \ref{bakker:sec:2} (taken from \citet[734, Fig. 4]{PapihaEtAl1996}. Of the five groups, the Malays are the most genetically distant, then the Burghers (no doubt due to their European-Sri Lankan admixture), then the Moors and then the Tamils and Sinhalese. The presence of these other groups pulls the Malays and Burghers away from the other Sri Lankan populations, and closer to the Malaysian and European populations respectively, as one might expect.

\begin{figure}%2
\includegraphics[width=\textwidth]{\imgpath bakker-img4}
\caption[A tree of Sri Lanka populations with Europeans and Malays]{A tree of Sri Lanka populations with Europeans and Malays
\citep[from][734]{PapihaEtAl1996}.}
\label{fig:bakker2}
\end{figure}

When we try to interpret these data, we find that the Malays are the group with the least admixture with the other Sri Lankan groups. They are the most deviant of the Sri Lanka populations, and they show the closest genetic affinity with the ancestral population of (Malaysian) Malays. On the other hand, there is a clear chronological pattern in the degree of admixture: the two oldest populations, the Tamils and the Sinhalese, both present on the island since millennia, display the highest degree of admixture. The Moors (whose presence on the island began a thousand years ago) are the next, then the Burghers (whose ethnogenesis can be estimated to date from the early 1600s or earlier) and the Malays (whose collective presence on the island dates primarily from the late 1600s). The fact that the Malays show the lowest degree of admixture may be due to the fact that they were the most recent population to arrive at the island. 

In a sense, these data are inconclusive. Proponents of the claim that Malays and Moors intermarried may point out that the closest affinity of the Sri Lanka Malays among the island's population is with the Moors. Opponents may point out that the Malays show the least admixture of all the island populations, and maintain a strong East Asian genetic profile (i.e. Malaysian; no Javanese or Indonesian data were taken into consideration).

Even though intermarriage may have been relatively rare, this does not mean that the social, economic and religious contacts of the Sri Lanka Malays were exclusively in-group, exclusively with other Malays. In the next section we provide historical data about contacts with other groups.

\section[A historical survey of relations]{Malays, Tamils and Sinhalese: a historical survey of relations}\label{bakker:sec:3}
The genetic data discussed above suggest limited gene flow between Malays and other Sri Lankan groups. Who were the main contacts of the Sri Lankan Malays outside their own community? Some have claimed that these were Tamil, others that these were Sinhalese. Here I will provide data on contacts between the Sri Lanka Malays and the other Sri Lankan groups from contemporary sources. 

I did an informal survey of several dozen early books on Sri Lanka, in order to find out what was said about the languages of the Sri Lanka Malays. The Malays have always been a small minority on the island, and the chances that travelers or scientists actually encountered and identified them are small, and even less frequent are descriptions of the group, or remarks on their language, especially on the basis of informed, direct observation. Here I list all those that I have encountered, without being selective in a certain direction, distinguishing between the pre-British period, and the colonial period of the island, keeping the independent indigenous Kandyan Kingdom distinct.

\subsection{The Portuguese and Dutch period}%3.1
The oldest source on Malays and their languages in Ceylon/Sri Lanka mentioned in the linguistic literature is the Schweitzer quotation dating from 1670s, which mentions Ambonese, Sinhalese and Tamil marriage partners (e.g. \citet[14]{Nordhoff2009}, \citet{SmithEtAl2006cll}). This is the only source known to me in which Malay-Sinhalese marriages are mentioned. Schweitzer, a German in the service of the in Dutch, met a party of ``Ambonese'' in Sitiwaka (Dutch territory, at that time), whose wives are described as follows: ``The wives, who in part are Ambonese, in part Sinhalese and Malabarian may not say anything [against the stripping of their ornaments].''\footnote{`Malabarian' means Tamil in this case.} \citep[106]{Schweitzer1931}, [Nordhoff's translation from German (2009:14)] These Ambonese were polyglots: ``Beside their own Language, they generally speak Maleysh, Cingulaish, Portuguese and Dutch''.
% (1700: 323).
The Sinhalese were said to be ``mighty afraid'' of the Ambonese, which suggests a social distance between the groups.% \citep[323]{Schweitzer1700}. 

Schweitzer met Tamils ``under the influence of Hollanders''
% \citep[271]{Schweitzer1700},
(\citeauthor{Schweitzer1931} \citeyear{Schweitzer1931}),
and his party also used Tamil guides.
%  (p. 306).
He also mentions ``Cingulaish soldiers or inhabitants'',
% (p. 280),
including those fighting with the Dutch against the Kandyans.
%  (p.200, 298, 319).
Thus, Sinhalese soldiers fought both on the side of the Dutch and for the Kandyan King.
%  (p. 319).

\citet[26]{Powell1973} mentions Portuguese sources claiming that already in 1587 the King of Kandy and Sitawaka, used soldiers who were not Sinhalese but ``Javas, Caffirs and of other nationalities''. I have not been able to locate a contemporary source for this claim. In fact, the Sinhalese are frequently mentioned as not using arms, and as not engaging in trade. For their armies, they used non-Sinhalese forces. This habit may go back several centuries, as Marco Polo had observed the same in the 13\textsuperscript{th} century: ``The inhabitants of Ceylon are not fighting men but paltry and mean-spirited creatures. If they have the need of soldiers, they hire them abroad, especially Saracens.''\footnote{Two
 alternative translations: ``the people are averse to a military life, abject and timid, and when they have occasion to employ soldiers, they procure them from other countries in the vicinity of the Mahometans.'' (translation Emerson Tennent); ``The people of Seilan are no soldiers, but poor cowardly creatures. And when they have need of soldiers they get Saracen troops from foreign parts. (translation Henry Yule)
}
(Book 3, Chapter 14). These Saracens are likely to be Arabs, and probably ancestors of the Sri Lanka Moors. \citet{Tennent1859} quotes 16\textsuperscript{th} century Portuguese sources suggesting that the Moors combined Arabic and Tamil in their speech. With reference to the Sri Lanka Moors, Odoardo Barbosa ``describes their language as a mixture of Arabic and Malabar, and states that numbers of their co-religionists from the Indian coast resorted constantly to Ceylon, and established themselves there as traders.'' (p. 617). These Muslim immigrants from India may be the incentive for the Moors to shift to Tamil. 

Robert Knox spent 20 years in the Kandyan kingdom from 1660 to 1679, after the crew of the ship, on which he was a sailor, was captured. He left a detailed description of the island and its inhabitants, printed in London in 1681, and an account of his escape \citep{Winterbottom2009}. Knox does not mention the Malays in his writings, but as he appears familiar both with events around the Kandy court and every-day life in the countryside, it is likely that Malays were not a particularly prominent group, if not completely absent, in Kandy at the time. Knox spoke Portuguese (p. 171) and Sinhala, besides English, and he spoke Portuguese with the Tamils near Mannar, because he could not understand Tamil (p. 166-167). Knox met Tamils in the Kandyan kingdom who did not speak Sinhala (p. 159). This quotation may indicate a lack of bilingualism: ``The Language they [Tamils, P.B.] speak is peculiar to themselves, so that a \textit{Chingulays} cannot understand them, nor they a \textit{Chingulays}. `` (p. 175.). The Kandyan King at the time knew Sinhala and Portuguese (p. 136, 175), but not Tamil. 

Knox gives first hand descriptions not only of the Sinhalese, Moors and Tamils (p. 61), but also the Veddas and even the tiny Gypsy-like group of the Rodi or Rodiyas (``\textit{Roudeahs}'', p. 71), and Muslim Moorish beggars who have their own ``temple'' in Kandy (p. 187). All this suggests a keen interest in ethnic differences, and it makes the absence of references to Malays even more remarkable. 

{The King's armed forces were of diverse origin. The King's body guards include Africans, called Kaffirs: ``Next his own Person} {\textit{Negro's}} {watch.} He hath also a Guard of \textit{Cofferies} or \textit{Negro's,} in whom he imposeth more confidence, then in his own People.'' (p. 35). He also had ``white soldiers'', commanded by Dutch and Portuguese people (p. 187), ``\textit{Europ{\ae}ans}; making them his great Officers, accounting them more faithful and trusty than his own People'' (p. 187). It is not clear what the ratio of White, Black and indigenous soldiers was.

\citet{Wickremesekera2004} studied the military operations of the Kandyan kingdom. She showed that the kings used local conscripts (presumably mostly Sinhala-speaking, with some Tamil, if they reflect the general population of  Kandy) and foreign mercenaries. In and before the Portuguese period the latter were mostly from India, speakers of both Dravidian and Indic languages. Later also Europeans and Africans were pressed into service after being taken prisoner (p. 139-140). Malay troops in particular ``played a prominent role in the Kandyan army'' (p. 140).  In the early 1800s an estimated 400-500 Malays were observed with Kandyan troops by British spies, but foreigners made up only a fraction of the soldiers in battle (p. 140). Modest numbers of Tamils functioned as personal guards to the Kandyan kings between 1740 and 1815 (p.139). The Kandyan Malay soldiers could have met speakers of both Tamil and Sinhala.

\subsection{Malays in the coastal areas}\label{bakker:sec:32}
Until 1815, Europeans only exercised power over the coastal areas, while having diplomatic or military contacts with the kingdoms of the interiors. Here I offer initial remarks about language use and intermarriage from the colonial area, thereafter from the Kandyan Kingdom.

The oldest source I found is from a Dutch account of 1672, accessed through an English translation of the original Dutch \citep{Baldaeus19581959}. The author  spent ten years on the island, between 1656 and 1665. The book contains detailed accounts of the military conflicts between the Dutch and Portuguese. Javanese are mentioned several times. One assault described for 1656 mentions the employment of Javanese, ``Bandanezen'', ``Lascarijnen'' (local soldiers, probably Sinhalese) and ``Toepassen'' \citep[free indigenous Christians, or descendants of Portuguese fathers and indigenous mothers;][]{Veth1889} and Europeans (p. 225/116).\footnote{The
 page numbers refer to the 1958-9 translation of Baldaeus, followed by the original page in the Dutch 1672 edition.
}
Also earlier, during the Dutch siege on Colombo, mixed groups of soldiers had been used, consisting of  people from Banda (Dutch possessions in Indonesia), Javanese and Mardijkers\footnote{The
 ethnonym {\em Mardijkers} would be derived from the Malay word {\em merdeka} `free'. See e.g. \citet{Veth1889}.
 }
\citep[free indigenous people, probably from Dutch possessions;][110ff]{Veth1889} for transport of equipment, and ``Ceylonsche''  for bearing away the wounded (p. 148/71). Similar mixed groups are mentioned elsewhere (p. 147/71: Javanese, Bandanese, Mardijkers, Dutch; p. 162: Bandanese and Mardijkers);   Javanese (p. 226/116); Lascarijns from Matara, probably Sinhala speakers (p. 258/133); Dutch, Javanese and Lascarijns in 1653 (p. 275/144); Javanese and Lascarijns in 1650 (p. 279-80/145); Portuguese, Toepassen and Sinhalese (p. 280/146).  The account also makes clear that desertion was common and that many soldiers shifted their loyalties. This source suggests cooperation and contact between the Malays and Sinhalese under Dutch rule in the 1650s and 1660s.

The next source  is from 1797-1799. A Swiss regiment called \textit{Régiment De Meuron} consisted of mercenaries. At the time of their arrival in Sri Lanka they worked for the Dutch, but they soon shifted their allegiance to the British in 1795. In a book about the Regiment, referring to the late 1700s, it was claimed that the Malays and the Muslims (Moors) did not speak Sinhala, and also that the Malays treated the Sinhalese badly. This is a second hand source, and a source for this claim is not given \citep[155]{DeMeuron1982}.

Robert Percival, a captain in the British army, wrote a description of Sri Lanka, published in 1803, and a second edition appeared in 1805. He had arrived on the island in 1796, and spent several years on the coast and in the interior. Percival mentions intermarriage of the Malays with especially the Moors: 

\begin{quote}
The Malays of the various islands and settlements also differ among themselves, according to the habits and appearance of the nations among whom they are dispersed. Yet still they are all easily distinguished to be of the Malay race. For, although they intermarry with the Moors and other casts, particularly in Ceylon, and by this means acquire a much darker colour than is natural to a Malay.'' \citep[168]{Percival1805}
\end{quote}

The clothing of the Sri Lanka Malays was quite distinctive, but they had also adopted some habits from the Moors:

\begin{quote}
{``The Malays of a higher rank wear a wide Moorish coat or gown, which they call} {\textit{badjour,}} {not unlike our dressing-gowns. ({\dots}) The slipper or sandal in use among them, is the same with that worn by the Moors.''} {\citep[169]{Percival1805}}
\end{quote}

{According to Percival, ``Low Portuguese'' (presumably a pidginized or creolized variety) was the language spoken in communication between Europeans and Sinhalese, including} {in the army at the time, and Sinhala (``Ceylonese'') was not used, but knowledge of a little Tamil (``Malabar'') was necessary in the armed forces and in the domestic sphere.}

\begin{quote}
``The Ceylonese language is so harsh and disagreeable to an European that few or none ever attempt to speak it; nor indeed is it at all necessary. The officers of the regiments stationed here have little opportunity and little occasion to learn it, few or none of the natives of the island but who have some other dialect being in our domestic service. Some little smattering of the Moor and Malabar language is necessary to be able to speak to the black servants of that description. \textbf{The low Portuguese is the universal language spoken amongst the Cinglese in our settlements}, and indeed amongst all the natives who have any intercourse or connexion with Europeans; and it is also spoken by the Moor and Malabar servants.'' (\citet[203]{Percival1805}, my emphasis).
\end{quote}

Vicomte George Valentia, who traveled through the Orient in the early 1800s, visited Sri Lanka, and described \em inter alia \em the Muslim population. The Muslim Malays ``can be contrasted with the natural inhabitants of the country, with whom they have no connection'' \citep{Valentia1813}.\footnote{All
 translations are mine. Unfortunately I had no access to the original English text of Valentia. The complete text in French is given here. ``Il y a aussi deux castes de Musulmans, qui sont très nombreuses. La première est celle des \textit{lebbies} ou des marchands africains que les Hollandais considéraient comme étrangers, et qui étaient taxés chacun à vingt-quatre schellings par an, taxe que le gouvernement anglais a supprimée. Les lebbies sont actifs et industrieux. M. North leur avait donné un mufti, pour juger leurs procès; mais, comme ses confrères, il se laissait corrompre par des présens. En conséquence il a été destitué et ses fonctions de juge sont exercées par le gouverneur lui-même. La seconde classe de Musulmans est celle des Malais, qu'on peut diviser en princes, en militaires et en brigands, quoique cette dernière dénomination puisse sans injustice leur être appliquée à tous. Les princes sont des souverains déposés, soit de Java, soit des îles adjacentes, soit de la presqu'île de Malacca, que les Hollandais ont bannis à Ceylan. Les militaires sont à la solde des Anglais et font de très-bons soldats. Quoiqu'ils soient si sensibles à l'honneur, que leur courroux est fatal lorsqu'ils se croient outragés, ils se soumettent sans murmure aux punitions infligées en vertu des règlemens. On peut les opposer aux naturels du pays, avec lesquels ils ne contractent aucune liaison. Cependant, leur nombre n'est pas assez grand pour qu'ils soient dangereux. La conduite qu'ils ont tenue durant la. guerre de Candy leur a fait infiniment d'honneur. Ils n'ont abandonné leurs postes qu'après que les officiers anglais leur en ont eu donné l'exemple; et même leurs chefs ont préféré être mort à l'ignominie.'' 
}

In 1845, the Malays were associated with the Tamils and the Moors by the Danish-French geographer Malte-Brun: 

\begin{quote}
``These places are home to a population of 8,000-10,000 individuals, whose physical characteristics and religion are similar to those of the Arabs, whereas their language is derived from Malay. They are called Mopla\"is by the Tamils, and are subjected to a chief who considers himself a vasal of the British'' \footnote{``
 Ces lieux nourissent une population de 8 à 10.000 individus, que leur caractère physique et leur religion rapprochent des Arabes, tandis que leur langue dérive du malais. Ils sont appelés Mopla\"is par les Malabares, et soumis à un chef qui se reconnaissait vassal des Anglais.''
} (1845-1847: 316)
\end{quote}

In the mid 1800s, the Sri Lanka Malays served in a military expedition against the Sinhalese of the interior, under the command of the British. They had killed many Sinhalese, and naturally this had done considerable damage to the rapport between the two groups, which lasted for some time. Pearson mentions 2,000 Sinhalese refugees: 

\begin{quote}
Neither the Dutch nor the Portuguese had ever conquered the whole of the island, which was accomplished by the British in 1815. Since then there have been a few rebellions, which, however, were easily suppressed. During the last one, in 1848, some 2,000 up-country Sinhalese were put to flight by thirty Malays who wore the British uniforms, a proof that the ancient warlike spirit of the Kandyans is practically extinct (Pearson 1904; second hand source)
\end{quote}

A contemporary Russian observer describes the atrocities of the Malays against the Sinhalese in some detail: 

\begin{quote}
It is not too long ago that the British took possession of the interior of the island of Ceylon. They had a lot of trouble to succeed, because of the extreme thickness of the forests and the innumerable spider webs that fill these woodland mazes, where the British only advanced with a lot of difficulty, and where the Sinhalese kept hiding to attack them suddenly and destroy them. Therefore they employed the Malays, whom they brought from Java and who, like ferocious animals, have penetrated into those depths, tearing up the Sinhalese with their poisoned krises for the British.\footnote{Il 
 n'y a pas longtemps que les Anglais ont pris possession de l'intérieur de l'île de Ceylan, qui était inconnu , et ils ont eu beaucoup de peine à y réussir, à cause de l'extrême épaisseur des forêts et des innombrables toiles d'araignée qui remplissent ces dédales sylvestres, où les Anglais n'avan\c{c}aient qu'avec beaucoup de difficultés, et où se tenaient cachés les Cingalis pour tomber sur eux à I'improviste et les détruire. Alors on a employé les Malais, qu'on a fait venir de Java, et qui, comme des bêtes féroces, ont pénétré dans ces profondeurs, et, déchirant de leurs \textit{criss} empoisonnés les Cingalis, ont conquis ainsi l'intérieur de Ceylan pour les Anglais.
} 
\citep[27-28]{Soltykoff1853}
\end{quote}

Not surprisingly, it was also reported that the Sinhalese feared the Malays:


\begin{quote}
Since that time Malay regiments were created that are here. All these informations I have from the Malay who works for me. But, what is positive is that the Sinhalese, the indigenous inhabitants of Ceylon, dread the Malays extremely.'' \footnote{Depuis
 ce temps on a formé des régiments de Malais qui sont ici. Tous ces renseignements, c'est de mon Malais que je les tiens; mais, ce qu'il y a de positif, c'est que les Cingalis, habitants aborigènes de Ceylan, redoutent extrêmement les Malais.
}  
\citep[27-28]{Soltykoff1853}
\end{quote}


Even in the late 1800s, the Malay soldiers working for the British would not encounter any Sinhalese while on duty, only Europeans and Africans:

\begin{quote}
The troops stationed there by the British consist for two thirds of European soldiers. The remainder are Kafirs [Africans brought to Sri Lanka by the Portuguese, P.B.] and Malays. No Sinhalese was part of the army.\footnote{Les
 troupes que l'Angleterre y entretient sont pour les deux tiers composées de soldats européens; le reste est cafre ou \emph{\textmd{malais}}; aucun Cingalais ne fait partie de l'armée.
}\footnote{Sebastian
 Nordhoff (p.c.) points out that other indigenous groups such as the Tamils and the Veddas were likewise excluded from the British army. Indeed the Brits typically used soldiers from other parts of the empire, in this case mostly India and Malacca, not local ones.
}\citep[412]{Cotteau1889}
\end{quote}



In the late 1800s, Tamils and Malays were reported to be employed in transportation in the same location. \citet[65]{Bruyas1898} described ``voyageurs'', people who transport passengers between Trincomalee and Batticaloa, as belonging to two ethnic groups, Tamils and Malays, who are distinguishable by their clothing. 

Many observers, by the way, describe the different way of dressing of the Moors, Malays, Tamils, Portuguese, Sinhalese (e.g. \citet{Soltykoff1840}, \citet[65]{Bruyas1898}, \citet[89]{Madrolle1926}, writing about 1902). This shows that the ethnic groups valued their distinctness.

This selection of quotations, covering almost 250 years, point to two directions: In the Dutch period, at least in the period 1650-1667, Javanese, Bandanese, and other Malay speaking soldiers in the service of the Dutch were often associated with local soldiers, often Sinhala speakers. A century later, the Malays seem to have had almost exclusively contacts with Tamil speakers. The Malays associate with Moors and speakers of Tamil, sometimes  even marrying them, while no connections with the Sinhalese are mentioned, except in the Dutch period, which may have extra weight as it took place in the formative period. On the contrary, the Malays in the service of the British massacred Sinhalese, who after this hated the Malays. This suggests potential influence from both Sinhala and Tamil on SLM.


\subsection{The Malays in Kandy, 1800-1815}\label{bakker:sec:33}. 
All the sources in Section \ref{bakker:sec:32} relate to the coastal areas conquered by the Europeans -- first the Portuguese, then the Dutch and finally the British. The interior of the island was an independent kingdom, virtually impenetrable by the Portuguese and Dutch armies, but finally conquered by the British in through several ``wars'' (1803-1805, 1815, 1817-1818). But even before the British period, there were numerous contacts -- both diplomatic and military confrontations - between the Europeans and the Kandyans. Some were mentioned in \label{bakker:sec:31}.

SLM is also spoken in what used to be an independent kingdom, as is evident from  \citet{Nordhoff2009}, a description of the Kandy variety for instance. What was the language situation there before the conquest, and what in particular do we know about the Malays in the area before approximately 1850? 

Sri Wickrama, the last King of Kandy, used Tamil soldiers in the early 1800s \citep[88]{Powell1973},  by 1800 also Malay mercenaries (id.), and  by 1803 ``Kaffirs'' (p. 87, 89, 123, 147). The Kandyan chief communicated in Tamil with the British \citep[125]{Powell1973}. Also in the second Kandy war, an invasion on behalf of the Kandyan king was conducted by ``1,000 men, many of whom were Tamils or Malays'' \citep[204]{Powell1973}, and at a later stage ``only 200 Malabars, the same number of Kandyman militia and a few Moormen, Kaffirs and Malays'' (p. 215, 218), and another group of 50 Malabars (p. 219). Kandyan representatives wrote their names ``in a mixture of Sinhalese, Nagari [Indian] and Tamil scripts'' \citep[231]{Powell1973}. It seems that Tamils dominated the Kandyan army: in the 1815 convention ``all male people of the Malabar caste'' \citep[284]{Powell1973} were expelled from the Kandyan provinces, presumably to prevent a military action against the newly established British regime. 

All this seems to point to the predominance of Tamil speakers even in the army of the Kandyan king, and thus a more likely source language for the Malay soldiers fighting for the Kandyian king. 

Why the Sinhalese were ``as little interested in trade as they were in manual labour'' \citep[56,225]{Powell1973} may have to do with the presence of three other populations that  performed both functions: Tamil-speaking Hindus from Sri Lanka and South India, and Muslims of Malay and partial Arabic descent (Tamil-speaking Moors) (id.). The Moors are also regularly mentioned as important in trade. 

\subsection{The Malays and the Tamil and Sinhala languages}%3.4. 
These quotations (which have not been selected in order to promote a particular point of view in the discussion of the ``Tamil bias'') attest to the historical importance of contacts with Tamil speakers in the later period and with Sinhala speakers in the Dutch army. If there has been  further influence from the Sinhala language on Sri Lanka Malay, this must date from after 1900, perhaps mostly from after 1950. However, also in the 1600s Malays may have been in direct contact with Sinhalese women (one quotation) and before 1796 they were certainly enlisted in armies in which Sinhalese were also present. There is only one primary source on language contact, and that involves Sinhala. In trade, Tamil may have been more important, and certainly in religion, Sinhala played no role for the Malays.
We can conclude that the Malays preserved a clear separate identity, also in dress, from all other groups. This does not imply that the Sri Lanka Malays were a socially isolated group. Malays had historically more documented contact with Tamils and Moors than with Sinhala-speakers, with the possible exception of the Dutch period in the army. Religion, location, and employment patterns may all have played a role here. Tamil and Portuguese, not Sinhala, were the main contact languages of the Sri Lanka Malays until at least the early 20th century. In the army, Portuguese was probably a lingua franca, alongside the main language, English.

Religious allegiances must have been strong, especially when religions enforce conversion, or when members of a religious community expect all marriage partners to have the same religion. Both of these considerations apply in the case of Muslims, and as the Tamil Moors and Sri Lanka Malays share a religion, in fact one that requires marriages with Muslims only, it is to be expected that this form of contact existed. Ethnic bonds were also strong for the different groups.

% Nordhoff continues: ``This is far from being solid evidence for a close Malay-Moor relationship or linguistic influence from Tamil. Future research should be aware of the shakiness of these assumptions and take a more skeptical approach, like \citep{Ansaldo2005}.''

We could say that the future is here, and these results point to a Moorish Tamil direction as far as the historical facts are concerned with a considerable quantity of quotations on Malay-Moor contact, and some source mentioning Sinhalese -- though there is also some indirect evidence of contacts with the Sinhalese in the pre-British period. In the next section we will look at the linguistic and sociolinguistic data, and see that those also point to a historical Tamil influence, where Sinhala influence became important only recently.

\section{Linguistics: the Tamil bias -- or the Sinhala bias?}\label{bakker:sec:4}
No researcher on SLM would contend that the radical changes in the structure of SLM from SVO to SOV, from prepositional to postpositional, the development of case-marking, and so on, have developed spontaneously,  were  already in place in Indonesia, or were the result of a  process of drift. No doubt these processes of change were contact-induced. The same process took place in Sri Lanka Portuguese, with an almost identical outcome. The largely parallel changes in the two languages \citep[cf.][]{Bakker2006} must have taken place under the influence of the major Sri Lankan languages, and that is probably also uncontroversial. 

There are two matters that have been a subject of debate with regard to  contact. The first matter is, \textbf{when} did the radical changes take place; Was it in the 20\textsuperscript{th} century, or in the 18\textsuperscript{th} century? 
Alternatively, could the changes have take place gradually? The second matter is the question as to \textbf{which Sri Lankan language} triggered the changes that led to SLM differing  so radically from all other varieties of Malay? Was it the minority language, Tamil, or was it the majority language Sinhala? I will only discuss the second question here.

Many authors had routinely assumed that the major or only source of influence was Tamil, until this question was taken up by Ansaldo (2008, and earlier), who claimed that it was actually Sinhala, or at least more Sinhala than Tamil. \citet{Ansaldo2005ms,Ansaldo2008genesis} referred to an ostensible ``Tamil bias'' in the study of Sri Lanka Malay. \citet{Nordhoff2009} devotes section 2.7 to the plausibility of contact with the Tamil-speaking Muslim Moors and the Sinhala speaking Buddhists, providing a balanced overview of the different positions. 

Let us go review what Nordhoff, in his generally excellent book on Kandy SLM, has to say with respect to the respective potential source languages (also see \citet{Nordhoff2008msdiachronica} for an updated survey).

\subsection{Structural influences from Tamil or Sinhala}%4.1. 
\subsubsection{Loanwords}%4.1.1.
From which language are there more lexical borrowings into SLM? Nordhoff writes (2009:257):

\begin{quote}
There are many Tamil loanwords for basic vocabulary terms, like \textit{kattil} `bed', as well as many animal names (\textit{vanaati} `butterfly', \textit{vavval} `bat'). The Tamil loans are phonologically integrated in the SLM system, and also used by speakers who do not know Tamil.
\end{quote}

The fact that quite intimate and basic words are borrowed, plus words for local fauna, suggests that Tamil lexical influence is rather deep and old. The loans are also phonologically integrated, which is usually indicative of greater age. Plus, people who do not speak Tamil also use them. 

There are also loans from Sinhala. These are treated in a very different manner by SLM speakers:

\begin{quote}
The Sinhalese borrowings seem to be nonce borrowings, and the speakers are aware of the code-switch. (p. 163)
\end{quote}

Any Sinhala word can be code-switched, possibly not, or not always, phonologically integrated. Still, this is quite marginal, as Sinhala is stigmatized (S. Nordhoff, p.c.). Both the occurrence and its stigmatization are typical of code-switching contexts. However, no conclusion can be drawn with certainty about the age of this pattern. It is theoretically possible that speakers have been mixing Malay and Sinhala in this way for centuries, keeping Sinhala words phonologically distinct. In general, however, the lack of permanent loans and the apparent lack of phonological integration, suggest recent influence in a context involving expanded knowledge of Sinhala as a second language. \citet{Paauwtv} identified 71 words of Tamil origin and 19 of Sinhala origin. This also indicates a stronger lexical influence from Tamil than from Sinhala. The lexical data suggest early influence from Tamil, and late influence from Sinhala.

\subsubsection{Phonology}%4.1.2. 
With regard to phonology, Nordhoff discusses two aspects in relation to the source of contact. A number of specific points where Tamil influence had been suggested are discussed, and Nordhoff concludes that these do not hold:

\begin{quote}
In phonology, the phoneme inventories do not suggest Tamil influence, and the purported constraints on the occurrence of initial retroflexes do not hold either. (p. 60) 
\end{quote}

One of the properties of phonology is stress, and according to  
\citet[131]{Nordhoff2009}, SLM stress is hard to be determined, if it  
exists at all. This elusive nature of stress is shared with Tamil, but not  
with Sinhala, where stress is assigned based on weight and position of the  
syllables.

On the whole, however, there are influences from both languages: ``Phonology is a mix of Sinhala and Malay features'' (p. 60). Phonology thus appears inconclusive, or at least, there is not one clear source for the phonological properties of SLM.

\subsubsection{Semantics}%4.1.3. 
With regard to semantics, Nordhoff points to influence from both languages: ``\textit{Anthi} is very common. The conflation of future and habitual is of Tamil influence.'' (2009:294) and ``Tamil has a similar use of the numeral \textit{one} for indicating vagueness \citep[135, cited on p. 322]{Schiffman1999}.''  %(2009:322).
Sinhala is mentioned as an adstrate influence: both SLM and Sinhala have two existential forms, one for animate and one for inanimate subjects (the former being \textit{duuduk} in SLM). Nordhoff writes ``that the grammaticalization path of \textit{duuduk} can be explained by this [Sinhala] adstrate influence'' (2009:169). 

Within the noun phrase, a number of controversies have surfaced. The interplay of marking accusative and animacy has been attributed to Tamil (\citet{Smith2003timing}; see \citet[59]{Nordhoff2009}. Nordhoff claims that SLM falls between Sinhala and Tamil (p. 60). The identity of the ablative and instrumental cases reported for SLM is attributed to Sinhala by Ansaldo in an unpublished paper: ``\citet{Ansaldo2005ms} notes that this morpheme shows an ablative/instrumental syncretism, as also found in Sinhala.'' \citep[345]{Nordhoff2009}. Further the marking of (in)definiteness is obligatory in Sinhala, not in SLM, and is hence closer to Tamil according to Smith. Nordhoff contradicts this and claims that indefiniteness marking is obligatory, and hence closer to Sinhala.
 
In the verb, the ``infinitive combined with the interrogative clitic =\textit{si} is used to request permission'' in SLM and Sinhala, not in Tamil \citep[280]{Nordhoff2009}. This, however, appears to be present in Tamil as well (S. Nordhoff, p.c.). Hence, semantic data point in both directions, and are inconclusive.


\subsubsection{Morphology: plural marking} \label{sec:bakker:414}
According to Smith, number marking is optional in Tamil, obligatory in Sinhala, and SLM is like Tamil. Nordhoff claims that number marking is also optional in SLM, inherited from trade Malay. On the other hand, the ``reinforcing use of \textit{pada} parallels the use of \textit{-gal} in Tamil'' \citep[Mr. X, Mr. Y, Mr. Z \textit{pada}, ][324]{Nordhoff2009}.

\subsubsection{In short}\label{bakker:sec:415}
For phonology, there is no evidence pointing to a historical shift that must have been from Tamil (or Sinhala, for that matter) to Malay (but some features such as retroflex consonants and distinctive vowel length are definitely Lankan). The lexicon shows an early dominance of Tamil, and (late) twentieth century dominance of Sinhalese. 
As for semantics, there is evidence of more recent Sinhala influence in the  case   syncretism of SLM (this is a   pragmatically salient structure), and the existence of   divergent grammaticality judgments  on the irrealis  (p. 294).
 

For semantics, there is evidence of more recent Sinhala influence (case syncretism), e.g. pragmatically salient structures, and existence of divergent judgments of grammaticality (p. 294, 1st p. irr.). There is also Tamil influence, for instance in the merger of future and habitual. 

If we only had the linguistic data at our disposal, we would have to conclude that there is early influence from Tamil. Tamil was the first, or at least the most important, second language learned in (most of?) the SLM communities. This is also found in historical sources. Only later, perhaps as late as the 20\textsuperscript{th} century, was Sinhala added to the repertoire for the \em general \em community. Sinhala influence is mostly semantic (convergence), and probably recent. There is also textual evidence that support such a claim, as in example \xref{bakker:ex:1} \citep[257]{Nordhoff2009}: 

\ea\label{bakker:ex:1}
\gll Itthu muusing=ka cinggala \textbf{thraa}\\
\textsc{dist} time=\textsc{loc} Sinhalese \textsc{neg}\\
`At that time there was no Sinhala.'
\z

This is corroborated by the history of education among the Sri Lanka Malays \citep[cf.][]{Bichsel1989}. Schools for the Malays were established from 1812, where the curriculum included Malay and Tamil until the last quarter of the 19\textsuperscript{th} century \citep[96-99]{Hussainmiya1990}, and Sinhala became a language of instruction for them only in the 20\textsuperscript{th} century.

\subsection{Educational and religious history of the Sri Lanka Malays}%4.2. 
The Sri Lanka Malays had a strong connection with the armed forces throughout their history in the Dutch and British periods. There were military schools for Malay boys, from 1815-1860. \citet[96]{Hussainmiya1990} writes the following about the languages of instruction at these schools: ``The education provided was excellent and included Malay language, Tamil and English, but no Christian religious education''.

According to Bichsel-Stettler, by the end of the 19\textsuperscript{th} century: ``Generally, Malays either opted for Tamil-medium or, most often, for English-medium schools (\citet[24]{Bichsel1989}, also cited by \citet[26]{Nordhoff2009}). Sinhala-medium education was not chosen until the middle of the [20\textsuperscript{th}] century \citep[27f]{Bichsel1989}.

The Sri Lanka Malays were Muslims when they arrived in Sri Lanka. Malays had their own mosques from 1783 \citep[47]{Nordhoff2009}. Mosque services took place in Malay and Tamil \citep[124]{Hussainmiya1990}, according to the above-mentioned sources (which are not always specific as to places and periods), but Sinhala is never mentioned.

\subsection{Recent developments: From quadrilingualism to bilingualism}%4.3.
A recurrent theme in \citet{Nordhoff2009} and in the texts, is the pride in the multilingualism of the older generation, and regret with respect to language loss among the younger generations. The older people are generally quadrilingual (SLM, Tamil, Sinhala, English), whereas many young people are losing the connection with their Malay heritage and only speak Sinhala and English.

Nordhoff describes this as follows:

\begin{quote}
While the older generations have very good command of grammar, style and register in all four languages, this is not necessarily the case with the younger generation, who often only have full command of Sinhala and a local variety of English. Knowledge of Tamil is often absent, and full command of Sri Lanka Malay (which would permit to convey any message) is as well. \citep[32]{Nordhoff2009}
\end{quote}

For a sociolinguist, this would point to a situation of shift, away from Tamil and SLM, and towards English and Sinhala.

\subsection{Tamil or Sinhala?}%4.4. 
If we now compare the linguistic data presented in this and the preceding section, then everything (historical, educational, religious, structural, sociolinguistic) appears compatible with a scenario of early influence from Tamil, and later influence (from the mid-1900s?) from Sinhala. If Tamil were the most important L2 for the Malays, then the founder principle should predict, from the historical data, a much more significant impact from Tamil than from Sinhala, and that is indeed what we find. 

However, there is growing evidence for Sinhala knowledge by Malays in the Dutch period.
There is the Schweitzer observation, and additional indirect evidence of early Sinhala competence in that Malays often belonged to the same army untis as the Sinhalese in the Dutch period. 
However, Portuguese may have been their lingua franca \citep[346 and passim]{Baldaeus19581959}. Most of the Sinhala influence is found in areas typically found in current language contact \citep[cf.][]{Matras2009}. There may indeed be a ``Tamil bias'' in SLM studies, but rightfully so: it is the only language (aside from Malay itself) that is mentioned for education, outgroup contacts and religion from the late 1700s to the mid 1900s.


\section{Is SLM a creole?}\label{bakker:sec:5}
SLM and Sri Lanka Portuguese (SLP) are treated as falling under the rubric of creole studies \citep{Nordhofftvintro}. But are these languages creoles? Both languages underwent structural influence, but hardly any lexical influence from other languages, to the extent that they started to resemble one another structurally; both languages converged to a great extent with the local languages \citep[cf.][]{Bakker2006}. SLP was traditionally called a creole, SLM was not (at least not until recently, and in fact only by some of the linguists involved). Perhaps there is a Eurocentric bias here, in that linguists have been more aware of creolized Portuguese in South Asia than of Malay. 

If one takes a certain structural prototype when defining what a creole is, then SLM is very far from a prototypical creole. Derek Bickerton, for instance, who worked on structural properties of creoles, would not have considered SLM a creole, and probably no one else would, on the basis of its structural features. If creoles have SVO word order, no inflection and preverbal TMA \citep[292-293]{Seuren1998}, then SLM could not be classified as a creole, as only TMA is creole-like, remotely resembling prototypical creole TMA systems. SLM is, however, an SOV language with inflectional morphology, and therefore ought not to be classified as a creole. This is a structural-typological definition.

One can also take another point of departure. If one says that a creole language is a language that inherited, for historical reasons such as slavery or migration, only part of the lexicon and grammar of a language, which also underwent influence from other languages, then SLM would fall under that definition, and hence be classified as a creole. Thus, if one defines a creole as a language whose speakers developed a partly new grammar on top of the limited set of features inherited from the combination of lexifier and substrates, then SLM would qualify as a creole. In contrast to other creoles, however, the structural innovations in SLM are not mainly from unknown or unidentified sources, but partial copies of existing structures. The innovations in SLM are not creative solutions triggered by the communicative needs in situations of interethnic communication, but changes triggered by the frequent use of several languages, in a situation where grammatical norms are loosened \citep[cf.][]{Matras2009}. 

\citet{Nordhoff2009} avoids the term ``creole'' for SLM, whereas  \citet{Jayasuriya2002nusa} and  \citet{SmithEtAl2004} call the language a ``creole''. \citet{Robuchon2003} discusses creolization processes, but he does not consider the language to be a creole. \citet{Slomanson2006,Slomanson2008,Slomanson2011} does not consider the language to be a creole either. Also \citet{Ansaldo2008genesis} argues against the use of the label ``creole'' for SLM. 

In my view SLM is indeed a restructured language, but not a creole from a structural or historical point of view. At no point in its history was Malay in Sri Lanka  reduced lexically and grammatically to a medium of interethnic communication that one could call a pidgin or a basic variety (cf. \citet{BeckerEtAl2003} for discussion of the latter in the context of creole studies) -- which would be a necessary part of my definition of a creole language. 

In recent work, we have shown that creoles form a distinct type of language, with a number of properties that in combination set creoles apart from non-creoles \citep{BakkerEtAl2011}. If we take existing sets of features as formulated as typical for creoles as a point of departure, and we sample a set of non-creoles and compare them, then creoles are clearly different from the non-creoles -- despite the fact that no creole has all the relevant features. Similarly, if we take existing sets of features selected by typologists for the languages of the world, and add creoles to the set of languages, we get the same results: creoles stand out as a distinct group among the languages of the world (see Bakker et al. 2011 for details). The findings by    \citet{SzmrecsanyiEtAl2009}, based only on English varieties, also show that creoles are structurally distinct varieties, different from both first language varieties and second language varieties of English. Unpublished preliminary findings by Michael \citet{Cysouw2009apics}, based on a comparison of 46 features common between an early version of the \textit{Atlas of Pidgin and Creole Structures} APICS (which includes SLM) and the typological \textit{World Atlas of Language Structures} \citep[WALS;][]{WALS}, made him conclude that: ``APICS languages are clearly different from WALS languages''. In Cysouw's findings, SLM does not cluster with the creoles, but with the non-creole WALS languages.

We obtained similar results for SLM compared to a sample of creole and non-creole languages. For our study, we chose \em inter alia \em the dataset gathered for the \textit{Comparative Creole Syntax} project, published in \citet{HolmEtAlEd2007}. Those authors tried to cover a broad range of categories ($n=97$) that were reputed to be characteristic of creoles. To give an impression of the range of  features covered, here are the category headings, where different types were distinguished under each heading: 

\begin{itemize} 
\item the verb and stative/dynamic distinctions, 
\item tense-mood-aspect, 
\item complementizers, 
\item dependent clauses, 
\item negation, 
\item passive, 
\item adjectives as verbs, 
\item copula, 
\item noun phrase structure, 
\item the expression of possession, 
\item pronouns and case, 
\item coordinating conjunctions, 
\item prepositions, 
\item miscellaneous. 

\end{itemize}


In brief, they cover a wide range of morphological and syntactic features that  a language may or may not have. All features have been scored as either present or absent -- and not all of the features appear to be recurrent in all creoles. In fact, none of the 97 selected features are shared by all the sampled creoles, and a fair number of the features were only found in a minority of the sampled creoles. This shows the relative variety of the creoles (they are by no means a homogenous set). Apparently many creoles do not conform to the prototype that the (Atlantic-oriented) editors presumed to be typical for creoles.

\begin{table}%1
\begin{tabular}{llll}
abbreviation & name & lexifier & location\\
\hline
AN & Angolar  & Portuguese  & Gulf Guinea \\
BD & Berbice Dutch & Dutch & Caribbean \\
CV & Cape Verdean & Portuguese  & Gulf of Guinea\\
DM & Dominican & French & Caribbean\\
HA & Haitian & French & Caribbean\\
JA & Jamaican & English & Caribbean\\
KO & Korlai & Portuguese & South Asia\\
KR & Krio & English & West Africa\\
KR & Kriol & English & Guinea coast\\
NB & Nubi, Kinubi & Arabic & Interior Africa \\
ND & Ndyuka  & English & Caribbean \\
NG & Nagamese & Indic & South Asia\\
NH & Negerhollands & Dutch & Caribbean \\
PL & Palenquero & Spanish & South America\\
PP & Papiamentu & Span/Port & Caribbean\\
SC & Seselwa (Seychelles) & French & Indian Ocean\\
SL & Sri Lanka Malay & Malay & South Asia\\
TP & Tok Pisin & English & Pacific \\
ZM & Zamboangueno & Spanish & Pacific\\
\end{tabular}
 \caption{Abbreviations of creole names in Figure \ref{fig:bakker1}.}
\label{bakker:tab:1}
\end{table}

Holm and Patrick asked the contributors to their book to collect the same information on 18 different creoles, thus providing directly comparable data on 18 creoles. The syntactic categories studied by the contributors to Holm \& Patrick are clearly biased in the direction of a creole profile. Table \ref{bakker:tab:1} lists the contact languages (18 creoles, plus SLM), used in the comparison, and the abbreviations used, the respective lexifiers and the approximate location of the creoles. It is immediately clear that there is a large range of lexifiers (Arabic, Dutch, Indic, English, French, Portuguese, Spanish) and they span the globe. It is a balanced sample, with geographical, lexical and social differences, be it somewhat biased towards Caribbean creoles. 

\begin{figure}

\includegraphics[height=.7\textheight]{\imgpath bakker-img1.jpg}
 \caption[18 creoles and Sri Lanka Malay]{18 creoles and Sri Lanka Malay (SL). 
SLM is found at the bottom, along with the non-creoles.}
\label{fig:bakker3}
\end{figure}

Figure 3 shows a phylogenetic network comprising the 18 creoles as discussed in Holm \& Patrick, to which Sri Lanka Malay was added. The SLM scores were added by me, on the basis of Nordhoff's (2009) description. The longer the lines are that radiate from the center, the more deviant a language is. Transverse lines point to conflicting signals. The star-like Figure 1 suggests that, at least on the basis of the 97 morphosyntactic features selected by Holm \& Patrick, these creoles have just as much in common as they differ from one another, because of the roughly equal length of the spokes -- with the notable exception of SLM. 

Note that the creole languages cluster only loosely around lexifiers in that the Spanish and Portuguese creoles cluster, but otherwise there is no significant clustering of regions, age or type of creole (see Bakker et al. 2011 for details, where statistical tests confirm the findings). The odd one out is clearly SLM, with its much longer spoke than what we find for the other languages, all of them creoles, suggesting it does not fit in with the other languages. On the other hand, SLM does cluster with the other two South Asian languages, Korlai Creole Portuguese and Nagamese (also called ``Pidgin Assamese'', actually a creole). These three languages apparently share a number of properties, most likely being those that developed under the influence of Dravidian and Indo-Aryan languages.

\begin{table}%2
\begin{tabular}{llll}
 abbreviation & name & affiliation & location\\
\hline
AIN & Ainu & Isolate & East Asia/Japan\\
BRA & Brahui & Dravidian & South Asia/N. India\\
IND & Indonesian & Austronesian & Pacific, Indonesia\\
KOL & Yukhagir & Isolate & Asia, Siberia\\
KOY & Koyra Chiini & Nilosaharan, Songhay & Africa, Mali\\
MAN & Mandarin & Sino-Tibetan & Asia, China\\
MIN & Mina & Afro-Asiatic, Chadic & Africa, Cameroon\\
PIR & Pirah\~a & Mura, Amerind & S. America, Amazon\\
SLM & Sri Lanka Malay & mixed? & South Asia\\
\end{tabular}
 \caption{Abbreviations and affiliations of non-creoles.}
\label{bakker:tab:2}
\end{table}

What happens if we add more non-creoles to the creoles? We added eight non-creoles from eight phyla, after having extracted features from published grammatical descriptions. These are listed in Table \ref{bakker:tab:2}. All of these languages are spoken in regions far removed from one another, excluding direct contact or areal influence. From each family, we have selected a member that is most known for its more analytic profile (often regarded as a typical profile for creoles). This was done in order to skew the data towards creole-like structures. For instance, we chose from the Afro-Asiatic family the isolating Mina language rather than morphologically quite fusional Bedouin Arabic. Other non-creoles have been selected for their simplicity score in \citet{Parkvall2008}'s matrix, because of the supposed lesser complexity of creoles, thus further biasing the sample towards a creole profile. As a 9\textsuperscript{th} language we added SLM. The results can be seen in Figure \ref{fig:bakker4}.

\begin{figure}%4 
\includegraphics[height=.7\textheight]{\imgpath bakker-img2.jpg}
\caption{Sri Lanka Malay among creoles and noncreoles (creoles marked with initial C, and noncreoles in CAPS; slm = Sri Lanka Malay)
Sri Lanka Malay is found at the top left, among the non-creoles and not close to IND (=Indonesian). Out of the nine non-creoles, Sri Lanka Malay is typologically closest to Kolyma Yukhagir and Brahui.}
\label{fig:bakker4} 
\end{figure}

Figure \ref{fig:bakker4} shows two quite clear things. Creole languages, all starting with C- followed by the two letter code used in Table \ref{bakker:tab:1}, all cluster together, and the non-creoles, all marked with three capitals (see Table \ref{bakker:tab:2} for explanation), also cluster together. This means that creoles, at least on the basis of these 97 features, are very different from this selection of the languages of the world, intentionally skewed in the direction of some sort of creole profile. SLM appears among the non-creoles and relatively far from its lexifier Indonesian (IND). The language is most closely connected to two other continental Asian languages, Brahui (BRA, Dravidian, Pakistan) and Kolyma Yukaghir (KOL, isolate, Siberia). 

This corroborates one thing that has been suggested before by several researchers, namely that SLM is typologically far removed from creoles, and much closer to Dravidian  languages. If we can take the 97 creole features as a litmus test, then SLM does not qualify as a creole. From this structural-typological perspective, SLM is not a creole. Note that this is based on a predefined set of creoles and their typological profile.


\section{How did SLM come into being?}\label{bakker:sec:6}
SLM is interesting because it is such an extreme outcome of language contact -- but it is far from unique in the world. The combination of its Malay lexicon, with its Tamil-Sinhala structure, without the direct borrowing of morphemes, has inspired all those linguists and others who have investigated the language to speculate on its genesis. I will outline my view of that matter in this section.

Some simplification processes of Austronesian took place in the genesis of Proto-Malay from its Austronesian predecessor. When Malay developed into a language of interethnic communication in Southeast Asia, more simplification took place, leading to vernacular forms of Malay and pidginized forms such as the set of contact vernaculars called Bazaar Malay. Vernacular forms of Malay, including possibly pidginized varieties, were subsequently transplanted from insular South East Asia and Malaysia to Sri Lanka.

Once in Sri Lanka, the Malays also learned one or more of the local languages. Most or all of them also knew other languages, such as some form of Portuguese, Sinhala and Tamil, or one additional language, Tamil. Tamil was a more likely choice because of religious affiliations as well as the political, educational and military situation. This led to a situation in which both Malay and Tamil were used (in some cases perhaps also Sinhala, but historical evidence is scanty thus far, and linguistic evidence is thin). Most likely there was even a double-nested diglossia, where both languages also had their L and H varieties (colloquial Sinhala, formal Sinhala; vernacular Malay and Indonesian/Malaysian Malay). At some point in time, most likely rapidly, this multilingual situation led to a fairly radical convergence toward ``Lankan'', in which SLM developed into the language it is now.

The resulting language SLM shows, summarized in a simplified way, the following properties. At the level of the sentence, SLM has developed Lankan syntax: the language is verb-final. In the verb phrase we find preservation of the position of creole/vernacular Malay preverbal marking, not marking with stem modification (Ablaut, consonant changes) and verbal suffixes as in Tamil and Sinhala. It has been claimed that TMA marking in creoles shows remarkable similarities in the semantic and syntactic properties (some of which are also found in vernacular forms of Malay), but SLM deviates significantly from those semantic categories. The meanings of the vernacular forms of the TMA show a semantic merger with the Lankan categories. Under influence of the Lankan languages, the number of preverbal markers was expanded, to fit the same categories as found in Tamil.


In the noun phrase, the convergence towards Lankan languages was more radical. The noun phrases became noun-initial, i.e. instead of prepositions, the language developed postpositions, postclitics and case markers. The semantic categories found in the NP show much overlap with Lankan languages.
The result is a language that is semantically very close to the Lankan languages, and the sentential and NP structures are also syntactically very Lankan. Sinhala and English influence affected the language more recently, mostly in the lexical domain, and in discourse strategies, in the form of alignment of pragmatic systems along the line of \citet{Matras2009}.


\section{Conclusion} %7.
What happened with SLM is fairly unusual on a global scale, and it must have to be a special type of social situation in which this process took place. It is far from unique, though. It is reminiscent of processes elsewhere, which have been labeled convergence, metatypy and fusion, and such phenomena can be encountered in all sprachbund situations (linguistic areas). SLM (with SLP) may have attracted more attention since it contradicts an assumption that a linguistic area must consist of more than two languages, as for instance in \citet{Thomason2000areas}. Of course many languages are spoken in the Sri Lanka sprachbund (Tamil, Sinhala, Malay, Portuguese), but there is no evidence of widespread mutual multilingualism, sometimes considered a precondition for convergent areas.

Where do we find similar processes, in which a language preserves all of its lexicon, but models its structure on other languages, at the same time using native morphemes to develop semantic equivalents of categories in the other language? In all linguistic areas we find such processes, but their exact nature differs from place to place. Furthermore, linguistic areas rarely if ever show identical structures of two languages, and specific languages in a sprachbund typically share only a subset of the features deemed typical for the area.

Where do we find the more radical and overarching process as in SLM (and SLP)? One case that has been discussed in connection with creole studies is the famous Kupwar scenario, in which local Dravidian and Indic languages diverged from related varieties spoken elsewhere, and in which the languages involved came to share a large proportion of their respective grammars. \citet{Nadkarni1975} describes the influence of Dravidian on the Indo-Aryan language Konkani.

A case much more reminiscent of the SLM case is that of the Wutun language \citep{Li1983,Li1984,LeeSmithEtAl1996}, and also some of the other languages spoken in the same region that have converged towards an agglutinative and verb-final profile \citep{Wurm1996northchina}. Wutun is lexically Chinese, but as is true of the surrounding varieties of Tibetan and Mongolian, it is verb-final, has case markers and an agglutinative structure -- a development very reminiscent of what happened with SLM. 

Also in regions in which Austronesian and Papuan languages border on one another similar things have been reported, such as the Papuanized Takia languages (Ross 1996; other examples are given there as well, not all equally convincing). The Nonthaburi Malay variety in Thailand \citep{Tadmor1992,Tadmor1995,Tadmor2004} converged on Thai, and in so doing, became an isolating language. The Tariana Arawak language in Brazil converged in the direction of local Tucanoan structures \citep{Aikhenvald2002,Aikhenvald2003}. Such processes are not uncommon -- in fact, the whole principle of linguistic areality is based on convergent changes. \citet{Matras2009} provides an attractive framework for language contact, embedded in a functional and pragmatic perspective, in which bilingual speakers reduce their cognitive burden by replicating discourse patterns from a pragmatically dominant language into the home language.

A process affecting the entire grammatical system is rare, perhaps even nonexistent. Even the landmark examples such as Sri Lanka Malay, Kupwar languages and Takia retain a number of their original structures -- Sri Lanka Malay for instance retained the preverbal position of tense, mood and aspect, despite Tamil and Sinhala suffixation. A study comparing the more radical changes cross-linguistically is a desideratum for contact linguistics. 

