\chapter[The Lexical Sources of SLM]{The Lexical Sources of Sri Lanka Malay Revisited}

\chapterauthor{Scott Paauw}{University of Rochester}

In 2004, I wrote a paper entitled ``A Historical Analysis of the Lexical Sources of Sri Lanka Malay''in which I looked at the source languages which have contributed to the lexicon of Sri Lanka Malay (SLM), with special attention paid to the Malay varieties which have provided lexical items.\footnote{This
 paper is based on research funded in part by a grant from the United States National Science Foundation (NSF). I would like to thank the students who worked with me under this grant, Cihangir Okuyan and Christine Wingrove, as well as colleagues who have advised me on matters relating to Sri Lanka Malay: Ian Smith, Peter Slomanson, Sebastian Nordhoff, Peter Bakker,  B.A. Hussainmiya, and David Gil. I am especially grateful to the two linguists from the SLM community, Mohamed Jaffar and Romola Rasool, who have generously shared their knowledge.
} 
Since that time, our knowledge of the Sri Lanka Malay language has increased significantly,\footnote{The
 work of Peter Slomanson and Ian Smith, as well as Sebastian Nordhoff's grammar of up-country Sri Lanka Malay \citep{Nordhoff2009} have added a great deal to our knowledge of SLM. 
}
although there has been no further work done on the lexical origins of the language, and, sadly, no comprehensive dictionary or word list is yet available. During the past several years as well, our understanding of colloquial Malay varieties has also been improved.\footnote{David
 Gil's Malay/Indonesian Dialect Mapping Project has yielded some very informative results about colloquial Malay varieties. It is hoped that my dissertation \citep{Paauw2008} has been a valuable contribution to the field as well. 
}
The time is ripe, therefore, to reexamine the lexical sources of Sri Lanka Malay.

A brief summary of the socio-historical setting in which Sri Lanka Malay developed is helpful to give a background to the analysis presented in this paper. Most of what we know of the history of the Sri Lanka Malay community comes from the work of the historian (and community member) B.A. Hussainmiya. The following summary is based largely on Hussainmiya's work, and in particular \citet{Hussainmiya1987,Hussainmiya2008}. The Dutch colonial administration began bringing Indonesians to Sri Lanka in 1656, following the expulsion of the Portuguese. These immigrants consisted of political exiles, criminal deportees, soldiers and slaves. Political exiles consisted chiefly of Javanese aristocrats, but also included members of the elite from Sumatra, Maluku (the Moluccas), Makassar, and Timor. Soldiers made up a much larger contingent of immigrants, and came from Ambon, Banda, Bali, Java, Madura and from the Buginese and Malay areas \citep[24]{Adelaar1991}. A form of Vehicular Malay became the lingua franca of the community, and led to the community identifying itself as `Malay' despite its diverse origins. This identity was reinforced by later waves of immigration under the British administration beginning in 1796, which were recruited mainly from the Malay Peninsula.

Most of the immigrants were Muslim, and the community soon became identified with the Muslim religion. Immigrants who were not Muslims likely integrated into other communities and lost their Malay identity. Due to their shared religion, the Malays associated most closely with the Muslim Tamils (known in Sri Lanka as Moors). \citet[24]{Adelaar1991} explains the importance of this connection: 

\begin{quote}
Although from a cultural point of view they lost many of their traditional customs and practices due to their integration with the Moors, (with whom they have often intermarried), it is to them that the Malays owe the maintenance of their religious identity and possibly even their identity as a separate ethnic group.
\end{quote}

Although the Sri Lanka Malay language was frequently written during the 19\textsuperscript{th} century, and newspapers in the language were published, it ceased to be used as a written language early in the 20\textsuperscript{th} century. During the 19\textsuperscript{th} century, the SLM community was still in contact with the larger Malay-speaking world, and Standard Malay was used as a model for the written language, creating a diglossic situation. When Sri Lanka Malay was no longer written, and contact with other Malay speakers was lost, the colloquial variety of the language became the only variety which speakers knew.

We now have a much clearer picture of the nature of the SLM language today, thanks to the work of Nordhoff, Slomanson, Smith and others, although the picture of how it got to this point is largely unknown. It is clear that SLM is the result of extreme language contact between Malay and the other languages of Sri Lanka. The degree to which phonological, morphological and syntactic features have been influenced by the languages of Sri Lanka has been described quite well. It is useful, however, to have some idea of the language which was originally brought to Sri Lanka by the Indonesian immigrants, and for this purpose, an understanding of the diversity of Malay varieties is helpful.

The Malay language has been a lingua franca in the Indonesian archipelago and beyond for at least two thousand years. From its origin in western Borneo as a member of the Malayic branch of Malayo-Polynesian languages within the Austronesian family, it has spread to Sumatra, and to the Malay Peninsula, as well as to communities in Java and the islands of Eastern Indonesia. In addition to the traditional classification of Malay into ``literary Malay, lingua franca Malay, and `inherited Malay' '' \citep[673]{AdelaarEtAl1996}, it is essential to account for the divide between colloquial and formal varieties. In many areas of Indonesia, there exist two or three varieties which interact in a complicated diglossia. Table \ref{paauw:tab:malayicvarieties} shows the complicated nature of these sometimes very divergent varieties which exist under the label of `Malay'.

\begin{table}
\begin{tabular}{llp{5cm}}
\hline
\textbf{Type} &
\textbf{Sub-Type} &
\textbf{Languages}\\\hline
National Languages &
Standard Varieties &
Bahasa Indonesia

Bahasa Malaysia\\\hhline{~--}
 &
Colloquial Varieties &
Modern Colloquial Indonesian, Nusatenggara Jauh Indonesian, Riau Indonesian, many others\\\hline
``Inherited'' Varieties &
Malayic varieties &
\textit{Borneo: }``Dayak'', Banjar, Iban, Salako, others
\newline
\textit{Sumatra}: Minangkabau, Kerinci\\\hhline{~--}
 &
Group I 

(Malay varieties) &
\textit{Sumatra}: Middle Malay (Seraway, Besemah), Riau Malay, Medan (Deli) Malay, others
\newline
\textit{Malaya}: Trengganu Malay, Kedah Malay, Kelantan Malay, Penang Malay, others\\\hhline{~--}
 &
Group II 

(Malay varieties) &
\textit{Borneo}: Brunei Malay, Sarawak Malay
\newline
\textit{Maluku}: Bacan
\newline
\textit{Sumatra}: Palembang, Bangka, Beliton
\newline
\textit{Java}: Jakarta\\\hline
Contact Varieties &
East Indonesia
\newline
West Indonesia &
\textit{Established Communities}: Ambon, North Maluku, Banda, Manado, Kupang, Larantuka,  Papuan Malay
\newline
\textit{Regional Lingua Francas}: Makassar, Alor, Tual, Tenggara Timur Jauh, Aru, Tanimbar

Peranakan Malay (Java), Loloan (Bali)\\\hhline{~--}
 &
Malaysia &
Baba Malay, Malaysian Bazaar Malay\\\hhline{~--}
 &
Outside Indonesia &
Sri Lanka Malay, Nonthaburi Malay (Thailand), Melajoe Sini (Netherlands), Lugger Malay (Australia), Cocos Malay\\\hline
\end{tabular}
\caption{Malay(ic) Varieties.}
\label{paauw:tab:malayicvarieties}
\end{table}

Throughout Indonesia and Malaysia, the standard varieties exist alongside the colloquial versions of the national language. The colloquial varieties of the national languages are very different from the standard languages, featuring ``undressed'' morphology, different vocabulary, highly variable syntax, and largely context-dependent semantic interpretations. In addition, in areas which are traditionally Malay-speaking (the Malay Peninsula, parts of Sumatra and Borneo, and communities in Java, Bali and Eastern Indonesia), these two national varieties exist alongside ``inherited'' varieties of Malay in a complex ``triglossia''.

To complicate matters further, there are three major dialect regions in the Malay world, roughly equating to patterns of settlement. Group I varieties are those which left the homeland first, and are spoken in Central and Northern Sumatra and the Malay Peninsula (as well as in the Malaysian areas of Borneo). Group II varieties represent a later wave of emigration and are spoken in Southern Sumatra, Jakarta, Bacan Island in Maluku and in the area of the original Malay homeland in Western Borneo. The third group of Malay varieties are those which developed through trade in Eastern Indonesia, including the Malay varieties spoken in North Maluku, Manado, Ambon, Banda, Kupang and Indonesian Papua. These varieties share many features, and descend from a single language, which has been given the name Eastern Indonesia Trade Malay (EITM) \citep[298]{Paauw2008}.

Apart from the shared features of EITM described in \citet{Paauw2008}, there has been little information available on the three major dialect regions of Malay. Happily, that situation is improving. For the past 17 years, David Gil has been working on the Malay/Indonesian Dialect Mapping Project, cataloguing 300 linguistic features (lexical, phonological, and grammatical) used in the colloquial varieties of the national language  in locations throughout the Malay world. He has given names to the three varieties, and his terms will be used here for these varieties. The ``Group I'' languages are referred to as ``Malaka'', after the trading entrep\^ot in the region where those language varieties are spoken, which had significant historical importance. The ``Group II'' languages are ``Java'', representing the cultural and political center of those language varieties, while the varieties descended from EITM are known as ``Maluku'', for the spice islands which were the center of the Eastern Indonesia trade.

All three of these regions, Malaka, Java and Maluku, were intensively involved in trade, both within and outside of the archipelago, for hundreds of years. All three regions, therefore, employed a variety of Malay which could be termed Vehicular Malay. The question then is what form the Vehicular Malay took which brought the Malay language to Sri Lanka.

Three previous studies have attempted to address this question. The first observer to look at the question was K.A. Adelaar (1991), who collected a limited sample of data from two informants in Colombo, including a word list of 262 words. Based on this data, Adelaar identified 19 features in SLM which differed from Standard Malay (SM), some of which lined up with other varieties of colloquial Malay. From these features, Adelaar determined that the data he collected ``agree with Hussainmiya's (1987) account in-so-far as they reflect the Moluccan Malay, Javanese, Jakartanese (or Batavian) and Tamil components that made up the Sri Lanka Malay community. But they disagree as to Hussainmiya's implication that the basis of SLM was Jakartanese, and they also do not show a strong influence from Javanese.''Adelaar found that the varieties which had the most in common with SLM were Moluccan Malay varieties (North Moluccan Malay and Ambon Malay), Baba Malay (a creole spoken by Chinese in Malaysia and Singapore), Bazaar Malay (a pidgin used for interethnic communication in Malaysia and Singapore), and Jakartanese (in this case, the Betawi language of Jakarta, distinct from colloquial Jakarta Indonesian). 

Adelaar compiled the following list of ``the most striking differences between SM\footnote{Adelaar
 does not distinguish between the dialects of Malay in his analysis, and his use of the term ``Standard Malay''refers to a variety with features largely shared between the Malaka and Java regions (although not necessarily with the Maluku region), roughly comparable to the national language varieties of Indonesia and Malaysia. 
} 
and SLM'' \citep[25-26]{Adelaar1991}:

\begin{table}
\small
\begin{tabular}{p{2.5cm}p{4cm}p{4cm}}
\textbf{Standard Malay} &
\textbf{Sri Lanka Malay} &
\textbf{Malay varieties following SLM}\\\hline
1)*-\textit{h} {\textgreater} -\textit{h} &
*-\textit{h} {\textgreater} {\O} &
Moluccan, Bazaar, Baba, Jakartanese\\
2){}-{}-{}- &
retroflex series &
(Tamil and Sinhalese influence)\\
3) {}-{}-{}- &
(contrastive) vowel length &
(Tamil and Sinhalese influence)\\
4){}-{}-{}- &
(contrastive) consonant gemination &
(Tamil and Sinhalese influence)\\
5)\textit{{}-m/-n/-{\ng}} &
\textit{{}-{\ng}} &
Moluccan\\
6)\textit{{\textschwa}} &
\textit{i/u} or \textit{{\textschwa}} varying with \textit{i/u} &
Moluccan\\
7) retention of most of the inherited morphology &
loss of most of the inherited morphology &
Moluccan, Bazaar, Baba, Jakartanese\\
8) locative preposition + noun phrase  &
noun phrase + linker + locative preposition &
Bazaar\\
9) noun + determiner &
determiner + noun &
Moluccan, Bazaar, Baba Jakartanese\\
10) possessed + possessor &
possessor + linker + possessed &
Moluccan, Bazaar, Baba, elsewhere\\
11) noun + adjective &
adjective + noun &
(Tamil and Sinhalese influence)\\
12) prepositions &
postpositions &
(Tamil and Sinhalese influence)\\
13) subject-verb-object &
subject-object-verb &
(Tamil and Sinhalese influence)\\
14) \textit{ada} denoting existence of a noun &
\textit{ad{\textschwa}/ar{\textschwa}}: progressive aspect of a verb &
Moluccan, Bazaar, Baba\\
15) {}-{}-{}- &
negators: \textit{t{\textschwa}r-/tra} &
Moluccan, Baba\\
16) full tense-mood- aspect adverbials &
full and reduced tense- mood-aspect adverbials &
Moluccan, Baba\\
17) plural personal pronouns are independent lexemes &
plural personal pronouns are historically compound forms with *\textit{ora{\ng}} &
Moluccan, Bazaar, Baba\\
18) {}-{}-{}-\ \  &
1st and 2nd personal pronouns borrowed from Hokkien Chinese &
Bazaar, Baba, Jakartanese\\
19) {}-{}-{}- &
plural marker -\textit{pada} &
Jakartanese\\
\end{tabular}
\caption{Features collected by \citet{Adelaar1991}}
\label{paauw:tab:adelaar}
\end{table}

In my own study \citep{Paauw2004}, although I addressed Adelaar's conclusions, I was looking primarily at lexical data. Drawing from three word lists (including Adelaar's) with about 1300 discrete words, I found that 88.4\% of the lexicon consisted of words originating in Malay, 10\% were words originating in Sri Lanka, and 1.6\% were of unknown origin.\footnote{The
 total for words of Malay origin includes words borrowed into Malay from a variety of languages \textit{before} the Malay settlement in Sri Lanka. The total for words of Sri Lankan origin includes words from Tamil and Sinhala and also words borrowed from other languages \textit{after} the Malay settlement of Sri Lanka.
} 
Looking more closely at the words of Malay origin, I found that 63\% were words used in both Indonesian Malay (the Java and Maluku regions) and Peninsular Malay (the Malaka region), 11.7\% were words of Indonesian origin, and 0.8\% were words of Peninsular Malay origin (the remaining 21.7\% were loan words).  This indicated that the lexicon was much more heavily influenced by Indonesian Malay than by Peninsular Malay. I did not look at he Indonesian data in terms of origin in the Java or Maluku dialects.

The most recent study of the origins of the Vehicular Malay which lexified SLM is \citet{Gil2010}.  Through analysis of the 300+ lexical, phonological and grammatical features included in the Malay/Indonesian Dialect Mapping Project, Gil found that there was a significant eastern (Maluku) component, plus a smaller Java-centered component.\footnote{It 
 should be noted that Gil does not refer to these regions as ``dialects'', and his data is only drawn from colloquial varieties of the standard language (and not ``inherited'' varieties of Malay).
} 
Gil found that when comparing features shared by SLM and Maluku, but not found in Java or Malaka, there were eight features which he counted as showing a ``strong'' likelihood of showing a shared origin, ten features which he labeled as ``possible'' and six features marked as ``weak''. When compared with Java, SLM had two strong features, two possible features, and five weak features. Finally, when compared with Malaka, SLM had no strong features, two possible features, and five weak features. The comparisons suffered from a lack of data from SLM for many members of the feature set, yet the conclusions show a marked correspondence.

Gil's feature sets for strong correlations for SLM with Maluku and Java are given in tables \ref{paauw:tab:gilmaluku}-\ref{paauw:tab:giljava} (Gil's transcription of SLM data is kept, and may differ from transcriptions elsewhere in this paper):

\begin{table}

\begin{tabular}{p{5cm}p{2cm}p{2cm}p{2cm}}
\textbf{Feature} &
\textbf{Maluku} &
\textbf{SLM} &
\textbf{Corresp.}\\\hline
Word for `cockroach' &
kakarlak &
kakarlath{\footnotemark} &
strong\\
Word for `laugh' &
tharthaava &
tertawa &
strong\\
Coda of \textit{bodoh} `stupid' &
{}-k &
{}-k, {\O} &
strong\\
Different suprasegmental pattern for words with historical penultimate \textit{{\textschwa}} &
yes &
yes &
strong\\
Reduced form of \textsc{possessive} \textit{punya} &
pe, pu, pung &
pe, ppe &
strong\\
Short \~{} long pronoun alternation &
yes &
yes &
strong\\
\textsc{progressive} with \textit{ada} &
ada &
ara- &
strong\\
\end{tabular}
\caption{Gil's feature sets for strong correlations for SLM with Maluku.}
\label{paauw:tab:gilmaluku}
\end{table}

\footnotetext{The 
 Maluku and SLM forms are a loan word from Dutch. The Java form is {\em kecoa} while the Malaka form is \textit{lipas}
}

\begin{table}
\begin{tabular}{p{5cm}p{2cm}p{2cm}p{2cm}}
\textbf{Feature} &
\textbf{Java} &
\textbf{SLM} &
\textbf{Corresp.}\\\hline
Word for `ear' is \textit{kuping} &
yes &
yes &
strong\\\hline
Prenasalization in \textit{tangis} `cry' &
yes &
yes &
strong\\
\end{tabular}
\caption{Gil's feature sets for strong correlations for SLM with Java.}
\label{paauw:tab:giljava}
\end{table}

The results of the Adelaar and Gil studies indicate a clear connection between SLM and the Maluku varieties through primarily structural and phonological features. This is a promising area for future comparative work to determine the depth of the correspondence between SLM and Maluku.

The current study concentrates on the lexical inventory of SLM, and is not intended to discount in any way the importance of work on phonological and structural comparison of SLM with other Malay varieties. Important evidence can be gathered by looking at each area of a language. Examining a language's word inventory can inform us about its origins and changes in the lexicon of a language can tell us a great deal about language contact.

For this study, a word list of 1710 discrete headwords was used, drawn from four sources:

\begin{itemize}
\item A word list of approximately 1000 discrete entries collected by Bichsel-Stettler as part of her M.A. thesis in 1989 \citep{Bichsel1989}.

\item A word list of over 500 items drawn from recorded conversations collected by B.A. Hussainmiya in the late 1970s.

\item A word list of 262 items collected by K.A. Adelaar in 1987 from two informants and included in \citet{Adelaar1991}.

\item Saldin's \nocite{Saldin2007} \textit{Kamus Bahasa Melayu Sri Lanka 2007}, provided to our project by Sebastian Nordhoff, containing about 2000 entries, many of which are duplicates, and containing many of the items in Bichsel-Stettler's word list.
\end{itemize}

The lists were combined, and duplicate entries were removed, as well as place names and ethnonyms, and instances of obvious code-switching (in the transcribed conversations). The original transcriptions used in each document were kept, illustrating the wide variation used in transcribing SLM.

The words in the combined list were tagged for origin, as follows:\footnote{Dictionaries
 of Bahasa Indonesia, Bahasa Malaysia, and Tamil were used to identify word origins. Much valuable information on loan words was provided by Coope's (1976) \nocite{Coope1976} etymological dictionary of Malay.}

\textit{Malay origin:}
\begin{itemize}
\item Words shared between the three regions

\item Words exclusive to the Malaka region	

\item Words exclusive to the Maluku region

\item Words either exclusive to the Java region, or shared by the Java and Maluku varieties (but not found in the Malaka region); This category was necessitated by the lack of dictionaries for Maluku region varieties

\item Loan words which were part of the Malay language before it was brought to Sri Lanka (identified by language of origin)

\end{itemize}

\textit{Sri Lankan origin:}

\begin{itemize}
\item Words borrowed from Tamil

\item Words borrowed from Sinhala

\item Words borrowed from other languages after settlement in Sri Lanka (not found in other Malay varieties)

\item Words of unknown origin (not found in other Malay varieties)\footnote{There
 were 68 words of unknown origin. A number of these may originate in colloquial or regional varieties of Tamil or Sinhala, as the author's knowledge of these languages is not extensive (but does include a reading ability in Standard Tamil). Other words of unknown origin appear to be Austronesian in form, though not from any of the three regions of Malay. These may be borrowings from other languages contributed through the SLM community's diverse origins in Indonesia. 
}
\end{itemize}


The results of the study were remarkably similar to the 2004 study, despite a larger word list and an improved study methodology. The lexicon of Sri Lanka Malay is overwhelmingly of Malay origin (Table \ref{paauw:tab:originsoflexemes} totals include loan words).

\begin{table}
\centering
\begin{tabular}{lllll}
 &
\textbf{2004 Study} &
 &
\textbf{2012 Study} &
\\\hline
\textbf{Malay words} &
1179 &
88.4\% &
1526 &
89.2\%\\
\textbf{Sri Lankan words} &
133 &
10.0\% &
116 &
 6.8\%\\
\textbf{Unknown origin} &
21 &
1.6\% &
68 &
 4.0\%\\\hline
\textbf{Total} &
1333 &
 &
1710 &
\\
\end{tabular}
\caption{Origin of SLM lexemes in two studies.}
\label{paauw:tab:originsoflexemes}
\end{table}

The maintenance of such a high percentage of words of Malay origin illustrates that the lexicon is the chief linguistic marker of Sri Lanka Malay identity, even as the morphology and syntax of the language have undergone significant changes through convergence with the other languages of Sri Lanka. It is notable that SLM has a higher percentage of Malay-derived vocabulary than some Malay varieties in Indonesia, such as Ambon Malay (85.9\%) and Betawi (85\%).\footnote{Figures
 are from \citet[3]{Blust1988}. The figures represent cognates with Standard Indonesian for a 200-word list. 
}


Analysis of the 1526 words of Malay origin show that the majority are shared by all Malay dialects, but that the Java region has contributed a larger share of the vocabulary than the other varieties (note that the totals for Java include words which are shared between the Java and Maluku regions as well as words exclusive to the Java region).

\begin{table}
\centering
\begin{tabular}{lll}
 &
\textbf{Number of words} &
\textbf{Percentage of all words}\\\hline
Shared Malay origin &
918 &
53.7\%\\
Java region &
260 &
15.2\%\\
Maluku region &
12 &
0.7\%\\
Malaka region &
9 &
0.5\%\\
Loan words\footnotemark{} &
327 &
19.1\%\\\hline
\multicolumn{1}{l}{Total} &
\multicolumn{1}{l}{1526} &
\\
\end{tabular}
\caption{Importance of the three possible regions of origin for the lexicon.}
\label{paauw:tab:detailedoriginoflexemes}
\end{table}
\footnotetext{Loan
 words are from non-Austronesian languages. Eighteen identified loans from Austronesian languages (all are from Javanese and Sundanese) are included under Java region. 
}

These results show that there is a very strong Java region component, even if the fact that words counted as Java region include an unknown number of words shared with the Maluku region. This points to a strong influence of the Java region in the formation of Sri Lanka Malay. This evidence, along with the evidence presented by Adelaar and Gil emphasizing the influence of the Maluku region, leads to the obvious conclusion that these two regions each had a significant role in the formation of SLM. The historical facts support this conclusion, in that many of the immigrants who formed the Sri Lanka Malay community were from Eastern Indonesia, and thus probably spoke the Maluku dialect. Others, including most of the political exiles, were from Java and spoke the Java dialect (although almost certainly as a second language). Furthermore, all of the soldiers were recruited in Batavia, and thus had resided in a Java Malay-speaking milieu before leaving for Sri Lanka. These facts explain the role of both Java Malay and Maluku Malay in the formation of SLM. The SLM language was already established, and the Malay community had resided in Sri Lanka for nearly 150 years before the British colonial administration began bringing immigrants from the Malay Peninsula. Because these later immigrants encountered an established language in Sri Lanka, their Malaka dialect had little effect on SLM.

David Gil's Malay/Indonesian Dialect Mapping Project referred to earlier includes a number of lexical items which can be associated with the patterns of the three regions of Malay. Although \citet{Gil2010} does not address many of these in relation to Sri Lanka Malay, presumably because most items show a convergence with two varieties while Gil was concerned with items converging with a single variety, it is worthwhile to examine these lexical items and how SLM fits into the picture. This is particularly illuminating because these are lexical items selected specifically because they are associated with particular varieties and thus can provide information about the origins of SLM. Table \ref{paauw:tab:cognates} presents these items.

\begin{sidewaystable}
\begin{tabular}{llllll}
\textbf{Word} &
\textbf{Malaka} &
\textbf{Java} &
\textbf{Maluku} &
\textbf{SLM} &
\textbf{SLM aligns with}\\\hline
\textsc{passive} marker &
kena &
kena &
dapat &
kena &
Malaka, Java\\
`large' &
besar &
gede &
besar &
bessar &
Malaka, Maluku\\
`card' &
kad &
kartu &
kartu &
kartu &
Java, Maluku\\
`ear' &
telinga &
kuping &
telinga &
kuping &
Java\\
`easy' &
senang/muda &
gampang &
gampang &
gampang &
Java, Maluku\\
`shrimp paste' &
belacan &
terasi &
belacan/terasi &
belacang &
Malaka, Maluku\\
`give' &
bagi &
kasi &
kasi &
kasi &
Java, Maluku\\
`rose-apple' &
jambu &
jambu &
jawas &
jambu &
Malaka, Java\\
`meet' &
jumpa &
ketemu &
ketemu &
kutumung &
Java, Maluku\\
`pink' &
merah jambu &
pink &
merah muda &
merah jambu &
Malaka\\
`put' &
letak &
taro &
taro &
taro &
Java, Maluku\\
`look' &
tengok &
lihat &
lihat/lia &
liyat &
Java, Maluku\\
`shoe' &
kasut &
sepatu &
sepatu &
supatu &
Java, Maluku\\
`delicious' &
sedap &
enak &
sedap/enak &
enak &
Java, Maluku\\
`tree' &
pokok &
pohon &
pohon &
pohong &
Java, Maluku\\
`understand' &
faham &
ngerti &
mengerti &
mengerti/merti &
Maluku\\
`village' &
kampung &
desa &
kampung/negri &
kampong/nigiri &
Malaka, Maluku\\
`wall' &
dinding &
tembok &
dinding/tembok &
tembok &
Java, Maluku\\
1\textsc{s} and 2\textsc{s} pronouns from Chinese &
 &
gua, lu &
 &
go, lu &
Java\\
`when' &
bila &
kapan &
kapan &
kapan &
Java, Maluku\\
`navel' &
pusat &
pusar &
pusat/pusar &
pusar &
Java, Maluku\\
`ginger' &
halia &
jahe &
halia/jahe &
jaye &
Java, Maluku\\
`cassava' &
ubi kayu &
singkong &
kasbi &
ubi kayu &
Malaka\\
`cold' &
sejuk &
dingin &
dingin &
dingin &
Java, Maluku\\
\end{tabular}
\caption{Comparison of the vocabulary of the different varieties.}
\label{paauw:tab:cognates}
\end{sidewaystable}

Java and Maluku each have 18 correspondences, while Malaka only has 7, further underscoring the closer relationship between Java and Maluku with SLM. Three further items from Gil's feature set, the words for `car', `bicycle' and `bus' show correspondences between Malaka and SLM, but these can be explained by the fact that these concepts entered the language in the 19\textsuperscript{th} or early 20\textsuperscript{th} centuries, during a period when SLM and Malaka shared a colonial master, the British, and there was intensive contact between the two languages.

One area of particular interest when considering the origins of SLM is the pronoun set. Pronouns are usually quite stable in languages, and are often used in comparisons between languages to show genetic relatedness. However, the first and second person singular pronouns in many Indonesian languages are often unstable, and varieties of Malay are particularly notorious for variation in pronouns and, indeed, avoidance of first and second person singular pronouns altogether. Among the seven Eastern Indonesian Malay varieties described in \citet{Paauw2008}, each variety (with one exception) has different second person pronouns from each of the other varieties, although all are varieties of the Maluku region.

\begin{table}

\begin{tabular}{lll}
\textbf{Variety} &
\textbf{1}\textbf{\textsc{s}}\textbf{ Pronoun} &
\textbf{2}\textbf{\textsc{s}}\textbf{ Pronoun}\footnotemark{}\\\hline
Manado Malay, North Moluccan Malay &
kita &
ngana\\
Ambon Malay &
beta &
ose\\
Banda Malay &
beta &
pane\\
Kupang Malay &
beta &
lu\\
Larantuka Malay &
kita &
eNko, no (\textsc{masc}), oa (\textsc{fem})\\
Papua Malay &
say, kita &
kow, ko\\
\end{tabular}
\caption{1st and 2nd person pronouns in a number of varieties.}
\label{paauw:tab:pronouns}
\end{table}
\footnotetext{It
 is notable that the second person singular pronouns are borrowed in several cases: \textit{ngana} (from Ternate, a Papuan language), \textit{ose} (from Portuguese \textit{voce}), and \textit{lu} (from Hokkien). 
}

In addition to this pronominal variation, Manado Malay, North Moluccan Malay, Ambon Malay, Kupang Malay and Papua Malay have short forms of the pronouns in alternation with the long forms, as SLM does. This was one of the features Gil labeled as a strong correspondence between Maluku and SLM.

The plural pronouns in Maluku consist of the singular pronouns + \textit{orang} `person', a feature which also occurs in SLM. This method of plural pronoun formation is also found in contact varieties of the Malaka region (specifically the pidgin used for inter-ethnic communication known as Bazaar Malay and the creole Baba Malay, spoken by Chinese in Singapore and Malacca). In addition, the form in SLM which one would expect to exist as a 2\textsc{p }form, \textit{lorang} (derived from \textit{lu} + \textit{orang}), is a singular pronoun in SLM, alternating with \textit{lu}.

The alternating short and long pronoun forms, as well as the plural pronoun formation, point to an origin in Maluku for the SLM pronouns. However, the first and second person singular pronouns in SLM are from the Java region, including the Hokkien loans \textit{go} `1\textsc{s}' and \textit{lu} `2\textsc{s}', as well as another 1s pronoun, \textit{saya} (with the short form \textit{se} in SLM). This somewhat confusing picture points once again to an origin in both Maluku and Java for the SLM pronouns

\begin{table}

\begin{tabular}{lllll}
 &
\textbf{SLM} &
\textbf{Java} &
\textbf{Maluku} &
\textbf{Origin of SLM form}\\\hline
1\textsc{s} &
\textit{go}

\textit{saya/se} &
\textit{gua}

\textit{saya} &
\textit{kita}

\textit{beta} &
Java\\
2\textsc{s} &
\textit{lu}

\textit{lorang} &
\textit{lu}

\textit{kamu, kau} &
(various) &
Java\\
3\textsc{s} &
\textit{de} &
\textit{dia} &
\textit{de} &
Maluku\\
1\textsc{p} &
\textit{kitang}, \textit{serang} &
\textit{kita } &
\textit{katong}

\textit{(ki)torang} &
Maluku (\textit{kita} + \textit{orang})\\
2\textsc{p} &
\textit{lorang pada} &
(none) &
(various) &
Java and Maluku\\
3\textsc{p} &
\textit{derang} &
\textit{mereka} &
\textit{dorang} &
Maluku (\textit{de} + \textit{orang})\\
\end{tabular}
\caption{Comparison of pronouns  between SLM, Java, and Maluku.}
\label{paauw:tab:pronounsslmjavamaluku}
\end{table}

Another interesting area of convergence between SLM and Maluku is in the negator \textit{ter/tra}{}-, variants are which are found in most varieties of the Maluku region, but are largely unknown outside of the Maluku region.\footnote{There
 is some evidence that a negator with a form similar to \textit{ter-} exists in the Malaysian pidgin known as Bazaar Malay, and even evidence that a similar form was used in Java Malay in the 19\textsuperscript{th} century. However, today, negation with \textit{ter-} is only commonly found in Maluku and SLM. 
} 

The words which are categorized in the previous analysis of SLM word origins as of Sri Lankan origin are those which have been added to the language since the Malay settlement in Sri Lanka, and are not found in other Malay varieties. These include loan words from Tamil and Sinhala, as well as loans from other languages which have occurred since SLM was formed. These words of Sri Lankan origin are described in the list below:

\begin{table}
\centering
\begin{tabular}{lll}
 &
\textbf{Number} &
\textbf{Percentage of all words}\\\hline
Tamil &
71 &
 4.2\%\\
Sinhala &
19 &
 1.1\%\\
Other\footnotemark{} &
26 &
 1.5\%\\
Unknown origin\footnotemark{} &
68 &
 4.0\%\\\hline
Total &
184 &
10.8\%\\
\end{tabular}
\caption{Loan words from Sri Lankan languages in SLM.}
\label{paauw:tab:slloanwords}
\end{table}
\addtocounter{footnote}{-2}
\stepcounter{footnote}\footnotetext{These consist of words borrowed from English (18), Portuguese (4), Hindi (2), Dutch (1) and Arabic (1). }
\stepcounter{footnote}\footnotetext{These are words which are not of Malay origin. It is likely that many of them are loans from Tamil or Sinhala dialects. }
Although SLM has been notably resistant to borrowing, and retains a very large percentage of Malay words, some borrowing has occurred over the centuries, most often relating to the culture, flora and fauna of the Malays' new home. As the SLM community has historically maintained the closest ties with the Muslim Tamil community, it is not surprising that the largest number of loan words is from Tamil.

Although it may not directly relate to the development of SLM, it is interesting to note the language of origin of loan words in Malay which have become a part of SLM, as shown below.

\begin{table}
\centering
\begin{tabular}{ll}
\textbf{Language of Origin} &
\textbf{Number of Loan Words}\\\hline
Arabic &
150\\
Sanskrit &
 79\\
Portuguese &
 28\\
Dutch &
 21\\
Persian &
 16\\
Hindi/Urdu &
 15\\
Tamil &
 11\\
Hokkien &
 4\\
English &
 3\\\hline
 &
327\\
\end{tabular}
\caption{Loanwords from other languages in SLM.}
\label{paauw:tab:otherloanwords}
\end{table}

The preponderance of Arabic loan words is probably a reflection of the SLM community's strong degree of religious devotion. In other varieties of Malay, the percentage of loan words from Arabic is much lower.

In conclusion, the lexical evidence points to Java Malay as the most important source of the lexicon of Sri Lanka Malay. Other evidence, provided by Adelaar and Gil, uses phonological and structural data to identify the Maluku region as the primary source. However, there are problems with identifying Maluku as the source for SLM. While there are certain features (the pronoun set, the negator \textit{ter-}, a few lexical items) which could only have had their origin in Maluku, there are other features (word final nasal coalescence, schwa replacement) which only occur in some words in SLM and are not regular processes. In addition, if we consider both these features and another features of Maluku which doesn't occur in SLM, final stop deletion, and we posit Maluku as the source for SLM we are left with the impossible task of explaining how a variety of nasals appear in word final position if they had coalesced in the source, and how a variety of word final stops occur if they had been deleted in the source. Once nasals have merged or consonants have been deleted, they cannot somehow be replaced. Furthermore, there are features in SLM which can only have come from Java Malay, such as the plural marker \textit{pada}, which is only found in Java Malay among Malay varieties of the Indonesian archipelago. These features and the preponderance of vocabulary associated with Java Malay indicate that Java Malay \textit{must} have had a role in the formation of SLM. 

A much more reasoned approach, and one which is supported by the historical record, is that both the Java and Maluku regions had an important role in the formation of SLM. 
