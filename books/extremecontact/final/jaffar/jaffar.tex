
\chapter[SLM Serial Verb Constructions]{The Semantics of Serial Verb Constructions in Sri Lanka Malay}

\chapterauthor{Mohamed Jaffar}{York University, Toronto}


\section{Introduction}
The aim of this paper is to examine Serial Verbs\footnote{Serial
 Verbs (SV) and Serial Verb Constructions (SVC) are used interchangeably in this paper.
} 
in Sri Lanka Malay, at least the current many that I, as a native speaker, can readily tap into.\footnote{For
 helping me find the right idiomatic expression, I am indebted to Anton Fernando, Nirusha Perera and Iris Wickramasinghe (Sinhala); Zaneera Tegal, Sam Tegal, Zreena Jaffar and Zowriya Jayman (Sri Lanka Malay); Rafiya Cassim and Marhooma Samsudeen (Colombo Muslim Tamil), all native speakers of their respective languages. My grateful thanks go to Ian Smith for his advice in connexion with the morphosyntax of Dravidian languages and Sri Lanka Portuguese --- which helped me immensely in understanding analogues in my native Sri Lanka Malay ---and to Sebastian Nordhoff for engaging me in lively debate from which I learnt some valuable lessons in navigating the murky waters of terminology. Last but not least, I acknowledge a large debt of gratitude to Philippe Bourdin, linguist, mentor and friend, for his advice on matters requiring fine linguistic judgment, and for his suggestions regarding a few aspects of this paper.
} 
I shall refer often to the work of Sebastian \citet{Nordhoff2009,Nordhoff2012jpcl}, said work being the only one I know that has dealt with the subject of SLM Serial Verb Constructions in any significant detail. I examine the structure of SVCs (which, in SLM, is straightforward) and, above all, the semantic implications of SLM serial verb constructions. Wherever possible I have tried to place them in the framework of an areal typology, in order to see clearly the influences of the adstrate languages: Sinhala (S), Tamil (T), and Colombo Muslim Tamil (CMT),
a variety of Shonam, the language of Tamil-speaking Muslims. 

Sri Lanka Malay (SLM) is the language spoken by the descendants of the soldiery that came with the Dutch in the seventeenth century, drawn from a number of islands in the archipelago that comprised the Dutch East Indies.
There were also a few exiled Javanese rebel princes, and a few convicts of whom very little information has been recorded or very little is known even anecdotally. There was at one time a high level of language maintenance thanks to the efforts of the British regimental schools that imported teaching materials, mainly from Singapore, and encouraged the children of the Malay expatriates to learn to read Standard Malay and to write it in `\textit{gundul}', the name used in Sri Lanka for `\textit{huruf Jawi}', the Arabic script, suitably modified and used for writing Malay. Primers, copy-books to imitate the `\textit{gundul}' script, prayer-guides, Islamic catechisms and prayer-books, all of these, silver-fish-eaten and coming apart in their bindings, were in the collections of local families. Many have discarded them; many more have donated them to the National Museum. It is reasonable to assume therefore that there was a `language-in-the-making' with Malays intermarrying; mingling with the other communities of the Island; and leading the life of the uprooted as best they could. Over time, a spoken Malay would develop that would retain Malay lexis but re-invent itself morphosyntactically along the lines of Tamil and Sinhala. It is my contention that the influence of Tamil,\footnote{Each
 Tamil-speaking Muslim community in Sri Lanka has a dialect of its own depending on location (Colombo, Beruwala, Galle, Matara, Kandy, Nawalapitiya, Trincomalee, Batticaloa, and Jaffna, to name a few). It would appear that the community as a whole prefers the name `Muslim' to `Moor' in order to invite into its fold all persons professing the faith of Islam. Sri Lanka Malays, Muslims themselves, have stoutly resisted the invitation on the just grounds that they have a sufficiently clear linguistic identity to subsist on their own. 
} 
on balance, is likely to have been greater than that of Sinhala. My reasons for making that assumption are grounded in historical and sociolinguistic realities. As was to be expected, with the winding down of the regimental schools, the Malay teachers of language and religion found employment elsewhere. Itinerant Moor teachers rushed to fill the breach and entered `contracts' with householders to teach Qur'anic Arabic (and Arabic-Tamil) to their little children. Interestingly, side-by-side with writings in Malay --- the quality of which was suffering a steady decline --- the local Malays also produced works\footnote{These
 works were mainly religious in nature but a few devoted themselves to sorcery and witchcraft.
} 
in Arwi or Arabu-Thamul: the variety of Tamil used by Tamil-speaking Muslims, a knowledge of which was important for the community's religious needs. Many families saved the books in Arwi and gave away those written in Jawi for they no longer were able to read them.\footnote{Personal
 observations and books retrieved from old collections in my family.
} 

\section{Serial Verb Constructions in SLM}

The constructions I propose to examine meet the requirements for serial verb status as specified in \citet{Aikhenvald2006}: 

\begin{enumerate} 
\item they are a sequence of verbs acting together as a single predicate;
\item they are monoclausal and allow no markers of syntactic dependency on their components;
\item their intonational properties are those of a monoverbal clause; no intonation breaks or pause markers occur between the components of an SVC;
\item they share TAM, modality and polarity values;
\item together the verbs refer to one single event;
\item prototypical SVCs share at least one argument.
\end{enumerate}

\citet{Aikhenvald2006} also mentions a few other parameters defining the structure of serial verbs and, here again, SLM serial verbs fit unimpeded into the four categories given hereunder:

\begin{enumerate}
\item COMPOSITION: Asymmetrical and Symmetrical SVCs:

 \textit{Asymmetrical} SVCs may consist of a verb from a relatively unrestricted class and another from a semantically or grammatically closed class (
 \trs{thulis}{write} + \trs{(h)abis}{finish} = \textit{write} \textit{finish} (completive aspect))

 \textit{Symmetrical} SVs consist of two or more verbs chosen from a semantically and grammatically unrestricted class: (\trs{lari}{run} + \trs{pi}{go} = \textit{lari pi} = `go running')

\item CONTIGUITY versus NON-CONTIGUITY of the verb components: all SLM serial verbs studied here are consistently contiguous: (\trs{thulis ambel}{write take}; \trs{masak simpang}{cook keep}; \trs{làpas tharo}{let-go put}.\footnote{There
 being just one type of schwa in my dialect, I have chosen to depict it by /à/ whereas other contributors might have shown the schwa differently, as /à/, /ì/, and /ù/ respectively, depending on the perceived surface realizations of the schwa in certain idiolects.
 }) 

\item WORDHOOD OF COMPONENTS: all SLM serial verb components studied here are `stand-alone' phonological words that, in combinations, form single lexical entities: \trs{masak}{cook}; \trs{simpang}{keep}; \trs{thulis}{write}; \trs{ambel}{take}; \textit{làpas} `let go', `set free'; \trs{tharo}{put}

\item MARKING OF GRAMMATICAL CATEGORIES IN SV CATEGORIES --- such as, for instance, person, tense, aspect, modality, negation or valency changing --- may be marked just once per construction (`SINGLE MARKING') or can be marked on every component (`CONCORDANT MARKING'). 

 The SLM serial verbs studied here are consistently of the `single marking' variety:

\ea
\gll  su-masak simpang {\em vs} *su-masak su-simpang \\
 \textsc{past}-cook keep { } *\textsc{past}-cook \textsc{past}-keep\\
 `was cooked and set aside.'
\z

\end{enumerate}
\section{Verbal Compounds}%2

In describing the Verbal Compound Nordhoff has this to state: `The string \textit{kasithaau} in [\ref{jaffar:ex:1}] is parsed into one phonological word, which can be seen from the absence of a \textbf{long vowel}\footnote{Emphasis
 mine.} in \textit{kasi}' \citep[171]{Nordhoff2009}:

\ea\label{jaffar:ex:1}
%1
\gll 
Badulla Kandy Matale samma association={nang} \textbf{{masà-kasi-thaau}}\\
 Badulla Kandy Matale all association=\textsc{dat}  must-give-know\\
 `Badulla, Kandy, Matale, we must inform all associations.'
\z

Before we consider verbal compounds we have first to address the somewhat vexatious `long-vowel' story. \citet{Sutadhara1995} raised the issue of long and short vowels in SLM and posited the existence of a long penultimate vowel in a word as a mark of its phonological completeness, its wordhood, see \citet[18]{Sutadhara1995}.\footnote{


Sutadhara traces a linguistically plausible phonological surface representation path for the word 

\trs{paha}{thigh}, based on the application of some rules, from initial \textipa{paa} $\rightarrow $ laryngeal insertion \textipa{paha} $\rightarrow $stress assignment \textipa{paha} $\rightarrow$ vowel lengthening \textipa{'pa:ha} $\rightarrow$ paragogic glottal insertion \textipa{'pa:haP} terminating in a word \textit{paa.ha'} (long vowel in the penult, ending in glottal stop).
Native speakers of SLM regularly pronounce the word as \textipa{paha} (short vowels in both syllables) and would have difficulty in readily identifying \textit{paa.ha'}, as described by Sutadhara, with the SLM word for `thigh'.
}

The issue is whether this is in accord with the intuitions of native speakers. As a native speaker myself, and one who is in constant touch with native speakers of several varieties of SLM (Hambantota, Kandy, Badulla, Colombo), I am inclined to side with the theory of a stress-based pronunciation, as opposed to one of long and short vowels, in determining the `phonologically autonomous' status of compound verbs. Even then, the absence of a stressed vowel in the first verb (\textit{kasi} in \textit{kasi \textbf{thaa}u} in the example given above) may not be a defining element. How else could one explain a set of serial verbs, in everyday SLM use, that might lay a legitimate claim to lexicalization as a`single-word' despite being articulated clearly as separate phonological words: 
\textit{aja}\footnote{Sutadhara
 proposes a faulty etymology, \textit{(h)ajat}  (VM from Arabic, meaning `wish'- n.) leading to \textit{aja(t) bapi}  [\textipa{Paj@'ba:pIP}] \citep[26]{Sutadhara1995}.
 Irrefutably, the better choice for this serial verb pair is \textit{ajak} (VM `invite' –v.) producing \textit{aja(k) bapi}   \phonet{a:ja ba:pi}.  I have never heard this combination pronounced as suggested by Sutadhara, with a short vowel in \textit{aja} and a long vowel in \textit{bapi}  \phonet{aja.ba:pi}, among native speakers of SLM.    
} 

`invite' + \trs{ba'wa}{bring} resulting in 
\textit{\textbf{a}ja b\textbf{a}wa}, 
`escort someone toward the speaker'; 
and, also with 
\textit{bapi}\footnote{\textit{Bapi}
 is undoubtedly a grammaticalization of an older SVC: \trs{bawa}{bring} + \trs{pergi}{go} evolving into $\rightarrow $\textit{bawa}+\textit{pi} $\rightarrow $ \textit{ba+pi} $\rightarrow $ today's \textit{bapi} `take away'.
} 
`take away' in 
\trs{\textbf{a}ja \textbf{ba}pi}{escort someone away from the speaker}. The first verb (in diachrony, \textit{ajak}, from Vehicular Malay, `to invite, solicit' is no longer in autonomous use in SLM, and is no longer productive. There is no inkling of the possible meaning of that first verb in SLM collective memory today, and both speaker and hearer consider  
\textit{\textbf{a}ja bawa} (often contracted to \textit{aja ba}), and \textit{aja bapi}, to be single lexical entities. This is all the more evident in the serial verb construction \textit{pi}+\textit{aja bawa}:

\ea\label{jaffar:ex:2}
\gll Mina, chàpat bapa=nya pi aja bawa! \\
Mina, soon father=\textsc{acc}  go accompany bring!\\
`Mina, go fetch Dad quick!'
\z

Yet, these verbs are pronounced respectively  
\textit{\textbf{a}ja \textbf{ba}wa} and
\textit{\textbf{a}ja \textbf{ba}pi}, i.e. with equal stress in the penult of each contributing verb. The forms \textit{*aja\textbf{ba}wa}, and \textit{*aja\textbf{ba}pi}  ---where the stress is on the penult of only the second contributing verb, as it would rightly be had the two verbs been parsed as a single phonological word--- are both not attested in any variety of SLM that I am acquainted with. 

However, contrary to what I claimed in \citet{BourdinEtAl2010}, I now agree with \citet{Nordhoff2009} that both `\textit{kasithau}' and `\textit{kasikawing}' must be regarded as single verbs\footnote{He
 has put forward other convincing arguments save for those regarding the `long vowel'.
}
in their own right. To sum up, SLM possesses five SVCs that have undergone lexicalization, and there appears to be no likelihood of their increasing in number:

\begin{enumerate}
\item \textit{kasi thau}, `advise', `inform',
\item \textit{kasi kawing}, `give in marriage',
\item \textit{kasi kànal} `make known', `introduce', `present', as in one person to another, a rare but nonetheless attested variant of \textit{mà-kànalkang}, also \textit{kànalkang làpas},
\item \textit{aja bawa}, `accompany toward the speaker',
\item \textit{aja bapi} `accompany away from the speaker'.
\end{enumerate}

This is in accord with what \citet[51]{Aikhenvald2006} has to say about one-word serial verbs, `[they] tend to be restricted to a more limited set of verb roots. That is, if a language has one-word and multiple-word serial verbs, the former tend to be limited, and the latter productive.'

\section{Full verb serialization}%3

I would propose, for SVCs proper, the following general characterization:

\begin{quote} 
Two or more verbs operating in a series, mainly contiguously, to provide in a synergistic manner a shade of meaning that each verb is incapable of conveying on its own.
\end{quote}
 

In SLM (as in other languages) the class of SVCs proper is partitioned into two subsets: (a) the two verbs act in tandem and the resulting combination carries the sum of their lexical meanings, in which case we are dealing with {\em full verb serialization} (FVS), or (b) one of the two verbs undergoes some degree of semantic depletion in the process of serialization, in which case it is appropriate to call it a {\em vector verb} as it `points', metaphorically, the direction in which the main verb is semantically headed. 

% \subsection{Full Verb Serialization} %3.1

As each of the participating verbs contributes its lexical content to the meaning of the whole, the FVS type is more productive in SLM than the `vector verb' type where both the contributing verbs and resulting combinations belong to restricted classes. Mostly full verb serializations combine verbs of motion: 
\trs{lari}{run}, 
\trs{pi}{go}, 
\trs{dathang}{come}.

SVCs of the following type as noted by \citet{Nordhoff2009} are full verb serials\footnote{The
 SLM words shown here have been re-written in a spelling convention of my own for conformity's sake and vary from Nordhoff's in that they do not portray pronunciation, e.g. `long vowels' and doubled consonants.
} 
with one TAM marking:

\begin{enumerate}
\item  \textit{jalang pi} `walk go', `go walking'
\item  \textit{lari kàluling},\footnote{Nordhoff
 gives the meaning `go astray' as elicited by him among the up-country Malays. In my dialect, as in many others, \textit{lari kàluling} simply means `run about.'
} `run around'
\item \trs{lari dathang}{come running}
\item  \trs{cari dathang}{come looking for}
\item  \trs{ambel bapi}{take away}
\end{enumerate}

I shall give two examples here:


\ea\label{jaffar:ex:3}
\gll kàcil (k)anak-anak pada lapang=ka adà \textbf{lari} \textbf{kàluling}\\
 little child     \textsc{pl} field=\textsc{loc}  \textsc{nonpast} run around\\
 `The little children are running around on the green.'
\z


\ea\label{jaffar:ex:4}
\gll polis as=dathang yang-curi barang=pada=nya su-\textbf{ambel} \textbf{bapi}\\
 police \textsc{cp}-come \textsc{rel}-stole(n) goods=\textsc{pl}=\textsc{acc}  \textsc{past}-take take-away\\
 `The police came and took away the goods that were stolen.'
\z


\section{Vector Verbs}%4

I shall conform here to the definition in \citet{Nordhoff2009} which I think captures accurately the semantics of the linguistic operation performed by this type of serial verb. A verb in this construction, \textit{usually}\footnote{Cf.
 \textit{(à)mbath} further along which refutes this statement by being the first of the two verbs, contrary to the usual pattern.
} 
the second in the series, imparts some useful aspectual or other grammatical information, to complement or augment the meaning of the first verb. For SLM,
% \footnote{Again,
%  for the sake of conformity, the spellings are mine.
% } 
\citet[174]{Nordhoff2009} lists a number of these vector verbs, viz. 
\trs{ambel} {take}, for (self)-benefactive and ingressive; 
\trs{kasi}{give}, for benefactive; 
\trs{(h)abis}{finish}, for completive; 
\trs{tharo}{put}, for affective; 
\trs{simpang}{keep}, for continuative, prospective; 
\trs{duduk}{sit}, for progressive; 
\trs{kàna}{strike}, for adversative; 
and \trs{pukul}{hit}, for intensive-aggressive (see Section \ref{jaffar:sec:5} of this work a detailed discussion of this last-named).

\subsection{\trs{Ambel}{take}}
Verb combinations with \textit{ambel} as the vector have variously been described as having BENEFACTIVE, INGRESSIVE-INCEPTIVE-INCHOATIVE meanings. As a rule, though, the versatility of \textit{ambel} lies in its ability to impart to the verb-sequence a notion of `seizing or appropriating' (the general acception of `take'), thus rendering more lasting the activity described by the main verb. This is true of most of the \textit{ambel} compounds.

In order to highlight the nuances of meaning, I give below a few examples --- together with their equivalents in CMT and S --- firstly of an utterance employing the core verb unaided by \textit{ambel}, secondly followed by an SVC using the same core verb with \textit{ambel} as vector, e.g. \trs{thulis}{write}. However, with \trs{ambel}{take}, the combined serial verb `write take' conveys the BENEFACTIVE meaning `note down', `record'.

\ea
\ea\label{jaffar:ex:5}
\gll Amath telfon nombàr=nya su-\textbf{thulis} SLM\\
 Amath telephone number=\textsc{acc}  \textsc{past}-write \\
`Amath wrote the telephone number.'
\ex\label{jaffar:ex:6}
\gll Amath telfon nombàr=nya su-\textbf{thulis} \textbf{ambel}\\
Amath telephone number=\textsc{acc}  \textsc{past}-write take\\
`Amath wrote down the telephone number (recorded it)'
\z
\z


\ea
\ea\label{jaffar:ex:7}
\gll Amath telfon nomb{\E}r=e \textbf{e{\textrtaill}uzinaan} CMT\\
 Amath telephone number=\textsc{acc}  \textsc{past}-write\\
`Amath wrote the telephone number.'
\ex%8
\gll Amath telfon nomb{\E}r=e \textbf{e{\textrtaill}uzi} \textbf{kon{\dz}aan}\\
Amath telephone number=\textsc{acc}  \textsc{cp}\footnotemark{}.write \textsc{past}-take\\
`Amath wrote down the telephone number (recorded it)'
\z
\z
\footnotetext{\citet{Arden1891}
 calls it the Verbal or Adverbial Participle.
}

 

\ea
\ea\label{jaffar:ex:9}
\gll Amath {\textrtailt}{\ae}lifon angkay={\E} \textbf{liwwa} S\\
 Amath telephone number=\textsc{acc}  write.\textsc{past}\\
`Amath wrote the telephone number.'
\ex%10
\gll Amath {\textrtailt}{\ae}lifon angkay={\E} \textbf{liya} \textbf{gatta} \\
 Amath telephone number=\textsc{acc}  write.\textsc{cp} take.\textsc{past}\\
`Amath wrote down the telephone number (recorded it).'
\z
\z

The same processes produce similar differences in meaning with \trs{dàngar}{hear} and \trs{dàngar ambel}{hear take} = `heed, listen, pay attention to' and their equivalences in   CMT \textit{kee\textrtaill}/\textit{kee{\textrtailt}{\textrtailt}u ko} and Sinhala \textit{ahann{\E}}/\textit{aha gann{\E}}.

\ea
\ea\label{jaffar:ex:11}
\gll yang-bilang=nya anak thra \textbf{dàngar}  SLM\\
 \textsc{rel}-say=\textsc{acc}  child \textsc{neg} hear\\ 
\ex%12
\gll  {\textesh}enn-att=e pulle=kki \textbf{keekk{\E}} ille CMT\\
 say-\textsc{nmlz}=\textsc{acc}  child=\textsc{dat}  hear \textsc{neg}\\
\ex%13
\gll kiy{\E}pu-de lamaya={\textrtailt}{\E} \textbf{{\ae}hune} n{\ae}h{\ae} S: \\
 say-\textsc{nmlz} child=\textsc{dat}  hear \textsc{neg} \\
 `The child didn't hear what was said.'
\z
\z

\ea
\ea\label{jaffar:ex:14}
\gll yang-bilang=nya anak thàma(u) \textbf{dàngar} \textbf{ambel} SLM \\
 \textsc{rel}-say=\textsc{acc}  child \textsc{neg} want hear take\\
\ex%15
\gll {\textesh}enn-att=e pulle \textbf{kee{\textrtailt}{\textrtailt}uk} \textbf{kollaadu}  MT \\
 say-\textsc{nmlz}=\textsc{acc}  child  hear take \textsc{neg}\\
\ex%16
\gll lamaya kiy{\E}pu=de \textbf{aha} \textbf{gann{\E}} n{\ae}h{\ae}  S \\
 Child say.\textsc{ptcp}=\textsc{interr} hear take \textsc{neg}\\
`The child will not listen to what is told him.'
\z
\z

Occasionally there is a RECIPROCAL~meaning, for example:

\trs{pukul}{hit}/\trs{pukul ambel}{hit take; hit each other, fight}. The CMT rendering of this is \textit{a{\dz}i/a{\dz}icci.ko}; the Sinhala forms are \textit{gahann{\E}/gaha gann{\E}}.

\ea
\ea\label{jaffar:ex:17}
\gll ithu orang anjing=nya arà-\textbf{pukul}  SLM \\
 \textsc{dist} man dog=\textsc{acc}  \textsc{nonpast}-strike\\
\ex%18
\gll and{\E} manu{\textesh}an naaiy=e \textbf{a{\dz}ikkiraan} CMT \\
 \textsc{dist} man dog=\textsc{acc}  \textsc{nonpast}-strike\\
\ex%19
\gll ara minihaa balla=t{\E} \textbf{gahan{\E}vaa}  S \\
 \textsc{dist} man dog=\textsc{dat}  \textsc{nonpast}-strike\\
`That man is striking the dog'
\z
\z

\ea
\ea\label{jaffar:ex:20}
\gll (k)anak-anak jalang=ka arà-\textbf{pukul} \textbf{ambel} SLM \\
 child\~{}\textsc{red} 3\textsc{pl} street=\textsc{loc}  \textsc{nonpast}-hit take\\
\ex%21
\gll pullei-{\E}l theru=le \textbf{a{\dz}icci} \textbf{kolraanuv{\E}l}  CMT \\
 child-3\textsc{pl} street=\textsc{loc}  hit.\textsc{cp}.3p take \\
\ex%22
\gll lama-yin paar=e \textbf{gaha} \textbf{gann{\E}vaa}  S \\ 
 child-3\textsc{pl} street=\textsc{loc}  hit take\\
`The children are fighting each other in the street'
\z
\z

It seems perfectly legitimate to give here an account of the function of \textit{ambel} in terms of the TRANSITIVITY HYPOTHESIS propounded by \citet{HopperEtAl1980}. It appears that one fundamental property of \textit{ambel} would be to carry the semantics of the verb-combination to a higher order of transitivity by reason of some resulting characteristic such as VOLITIONALITY, INDIVIDUATION, AFFECTEDNESS, or AGENTIVITY:

In \xref{jaffar:ex:6}, \textit{thulis+ambel}: Amath is recording the number (deliberateness of the Agent, high degree of volition) and the number itself now takes on, so to speak, a status it did not possess prior to the act of its being written down. In other words, \textit{ambel} has a transitivizing effect both in terms of AGENTIVITY of the Subject and AFFECTEDNESS of the Object;

In \xref{jaffar:ex:11}, \textit{dàngar}, by itself, refers to a sensory event that is involuntary;

In \xref{jaffar:ex:14}, \textit{dàngar+ambel}, together the two verbs refer to an act of volition. The referent of the subject is not the passive recipient of a sensory impression but an Agent with a degree of control. So, much as with \textit{thulis}, \textit{ambel} acts as a transitivizer by enhancing the AGENTIVE and VOLITIONAL character of the process. 

In \xref{jaffar:ex:17}, \textit{pukul}, by itself, we have one Agent and one Patient.

In \xref{jaffar:ex:20}, \textit{pukul+ambel}, however, the addition of \textit{ambel} entails an increase in AGENTIVITY (the meaning of reciprocity that it contributes elevates both participants to the status of Agents). 

\subsection{\trs{Àmbath}{thrash, lash, whip}}
{\em Àmbath}\footnote{An
 entry for \textbf{àmbath} /{\E}mbat/ in \citet{StevensEtAl2004} reads as follows: `\textbf{embat mengembat 1} to lash/whip/thrash with a strip of bamboo, a piece of rope, etc.'
}
must qualify as the vector verb of choice in SLM to express the INTENSITY of an act described by the main `action'{}-verb:
\trs{(à)mbath}{`thrash'} + V = intensive, do V with vigour, intensity.

\ea\label{jaffar:ex:23}
\gll meja=ka yang-ada makanan=nya se \textbf{makang} \\
 table=\textsc{loc}  \textsc{rel}-be food=\textsc{acc}  1\textsc{sg} eat\\
`I ate the food that was on the table.'
\z


\ea\label{jaffar:ex:24}
\gll meja=ka yang-ada makanan=nya se \textbf{mbath} \textbf{makang} \footnotemark \\
 able=\textsc{loc}  \textsc{rel}-be food=\textsc{acc}  1\textsc{sg} strike eat\\
 `I ate up\footnotemark the food that was on the table.'
\z
\footnotetext{Breaking
 away from the general SLM pattern, the bleached verb is on the left of the main verb it modifies, a phenomenon not unusual in multiple verb constructions (cf. vector verb \textit{kàna}). I cite \citet{Nordhoff2009}: `4.1.6. Position of the bleached and the unbleached verb...the bleached verb can be on the left side or the right side, and the unbleached verb then occupies the other position.'
}
\footnotetext{`An
    action viewed from its endpoint, i.e. a telic action...in the telic sense. \textit{I ate it up}, the activity is viewed as completed, and is carried out in its entirety...' \citet{HopperEtAl1980}
  }

\ea\label{jaffar:ex:25}
\gll polis=dari sini dathang (s)aja, maling de \textbf{lari} \\
 police=\textsc{abl}  here arrive only thief 3\textsc{sg} run\\
`As soon as (they) from the police came here, the thief ran.'
\z

\ea\label{jaffar:ex:26}
\gll polis=dari sini dathang (s)aja, maling de \textbf{mbath} \textbf{lari} \\
 police=\textsc{abl}  here arrive only thief 3\textsc{sg} strike run\\
 `As soon as (they) from the police came here, the thief sped off (running).'
\z

\ea\label{jaffar:ex:27}
\gll ithu bothol sopi=nya de \textbf{minung} \\
 \textsc{dist} bottle arrack=\textsc{acc}  3\textsc{sg} drink \\
 `He drank the bottle of arrack.'
\z


\ea\label{jaffar:ex:28}
\gll ithu bothol sopi=nya de \textbf{mbath} \textbf{minung} \\
 \textsc{dist} bottle arrack=\textsc{acc}  3\textsc{sg} strike drink \\
 `He drank up the bottle of arrack.' 
\z

CMT and S do not have similar constructions, i.e. with an equivalent vector verb of intensity.

\subsection{\trs{Kàna}{be struck by}}
\em Kàna \em appears to be the veritable workhorse in vector-verb constructions. 
It must be noted that \textit{pukul} and \textit{kàna} are not interchangeable in their usage for, when it occurs in isolation, \textit{pukul}, the transitive verb, has the active meaning of `strike' as opposed to \textit{kàna}, the intransitive verb, which has the inherently passive meaning of `be struck':

\ea\label{jaffar:ex:29}
\ea
\gll ithu orang anjing=nya arà-\textbf{pukul}  SLM \\
 \textsc{dist} man dog=\textsc{acc}  \textsc{nonpast}-strike\\
\ex%30
\gll and{\E} manu{\textesh}an naaiy=e \textbf{a{\dz}ikkiraan} CMT \\
 \textsc{dist} man dog=\textsc{acc}  strike.\textsc{nonpast}\\
\ex%31
\gll ara minihaa balla=t{\E} \textbf{gahan{\E}vaa}  S \\
 \textsc{dist} man dog=\textsc{dat} strike.\textsc{nonpast}\\
`That man is striking the dog'
\z
\z

\ea\label{jaffar:ex:32}
\ea
\gll *ithu orang anjing=nya arà-\textbf{kàna}  SLM \\
 \textsc{dist} man dog=\textsc{acc}  \textsc{nonpast}- be.struck \\
\ex
\gll *and{\E}  manu{\textesh}{\E}n naaiy=e \textbf{pa{\dz}ukiraan}  CMT \\
 \textsc{dist} man dog=\textsc{acc}  be.struck\\
\ex
\gll *ee miniha balla={\textrtailt}{\E} \textbf{vadinavaa}  S \\
 \textsc{dist} man dog=\textsc{dat}  be.struck\\
 (`That man (be struck) the dog', ungrammatical because of the verb with an inherently passive meaning)
\z
\z

Also, \textit{kàna} has an extended meaning of `come in contact with': 

\ea\label{jaffar:ex:35}
\ea
\gll se=pe kaki pinthu=ka \textbf{arà-kàna}  SLM \\
 s1s=\textsc{poss} foot door=\textsc{loc}  \textsc{nonpast}- be.struck \\
\ex%36
\gll en{\dz}=e kaal kadavu=le \textbf{pa{\dz}udu}  CMT \\
 1\textsc{sg}=\textsc{poss} foot door=\textsc{loc}  be.struck \\
\ex%37
\gll ma=ge kakula dor=e \textbf{vadinavaa}  S \\
 1\textsc{sg}=\textsc{poss} foot door=\textsc{loc}  be.struck \\
 `My foot strikes (comes in contact with) the door.'
\z
\z



\ea\label{jaffar:ex:38}
\ea
\gll *se=pe  kaki pinthu=nya arà-\textbf{pukul}  SLM \\
 1\textsc{sg}=\textsc{poss} foot door=\textsc{acc}  \textsc{nonpast}- strike\\
\ex%39
\gll *en={\dz}e kaal kadav=e \textbf{a{\dz}ikkidu} CMT \\
1\textsc{sg}=\textsc{poss} foot door=\textsc{acc}  strike\\
\ex%40
\gll *ma=ge  kakula dor{\E}=t{\E} \textbf{gahanavaa}  S \\
 1\textsc{sg}=\textsc{poss} foot door=\textsc{dat}  strike\\
 (`My foot strikes/hits the door', ungrammatical because verb calls for a human Agent and `foot' does not qualify).
\z
\z 

SLM, in common with CMT, has no morphological device for Passive Voice. Muslim Tamil turns to an auxiliary verb \textit{pa{\dz}u} (which, coincidentally, has the same inherently passive meaning of `be struck' or `come in contact with' as does \textit{kàna} in SLM) with the difference that where CMT uses `infinitive + \textit{pa{\dz}u}' SLM employs `\textit{kàna} + V1' (core verb). Sinhala, on the other hand, has no passive formation; rather, it relies on the lexical choice of an intransitive or involitive verb to achieve the semantic equivalent of a passive \citep[39]{GairEtAl1997}.

As \citet[188]{Nordhoff2009} rightly observes: 
`It is therefore not the case that \textit{kìnna} [=\textit{kàna}] changes the syntactic status of arguments, as a passive construction would do. Rather, it contributes a semantic shade of meaning, very much in the way other vector verbs do.' 

\ea\label{jaffar:ex:41}
\ea
\gll th(r)a kà-thau-an=nang duith su-\textbf{kàna} \textbf{bayar}  SLM \\
 \textsc{neg}  \textsc{nmlz}-know-\textsc{nmlz}=\textsc{dat}  money \textsc{past}-be.struck pay\\
\ex%42
\gll teriyaam{\E} {\textesh}alli \textbf{ke{\textrtailt}{\textrtailt}ip} \textbf{pa{\textrtailt}{\textrtailt}ucci}  CMT \\
Unknowingly cash pay.\textsc{inf} be_struck.\textsc{past} \\
`Unwittingly the money was paid out.'
\z
\z


\ea\label{jaffar:ex:43}
\ea
\gll anging=nang jànela su-\textbf{kàna} \textbf{buka}  SLM \\
 wind=\textsc{dat}  window \textsc{past}-be.struck open\\
 \ex \label{jaffar:ex:44}
\gll kaattuk=ku jannal \textbf{torandup} \textbf{pa{\textrtailt}{\textrtailt}ucci} CMT \\
   wind=\textsc{dat}  window open.\textsc{inf} \textsc{past}-be.struck \\
 `The window got opened for (by) the breeze.' 
\z
\z

\textit{Kàna} also participates in idiomatic constructions, with some intransitive verbs, that have no MT or S equivalents:

\ea\label{jaffar:ex:45}
\gll inceia sama se=dang m-omong th(r)a \textbf{dapath} \\
 3\textsc{sg}.\textsc{hon} \textsc{comit} 1\textsc{sg}=\textsc{dat}  \textsc{inf}-speak \textsc{neg} get\\
 `I didn't get to speak with him' 
\z

\ea\label{jaffar:ex:46}
\gll inceia sama se=dang m-omong th(r)a \textbf{kàna} \textbf{dapath}, bukang!\\
 3\textsc{sg}.\textsc{hon} \textsc{comit} 1\textsc{sg}=\textsc{dat}  \textsc{inf}-speak \textsc{neg} be.struck get \textsc{expl} \\
 `I didn't get to speak with him, y'know !' (the circumstances just weren't propitious).
\z

\ea\label{jaffar:ex:47}
\gll kàmareng pagi siang\~{}siang se \textbf{bangung} \\
 yesterday morning early 1\textsc{sg} rise\\
 `I woke up early yesterday morning.'
\z


\ea\label{jaffar:ex:48}
\gll kàmareng pagi siang\~{}siang se \textbf{kàna} \textbf{bangung} \\
 yesterday morning early 1\textsc{sg} be struck rise\\
 `I woke up early yesterday morning (involuntarily).'
\z

\subsection{\textit{Simpang}, `keep', `preserve'}
{\em Simpang} usually conveys the meaning `set aside (for later) and can be seen as prospective:
\trs{Masak}{cook} + \trs{simpang}{keep} = `cook keep' (cook and set aside)

\ea\label{jaffar:ex:49}
\ea
\gll Àma, nasi sàdikith \textbf{masak} \textbf{simpang} sahar wakthu=nang! SLM \\
 Ma rice little cook keep sahar time=\textsc{dat}  \\
\ex%50
\gll Ummaa, {\textesh}oor kony{\E}m \textbf{aakki} \textbf{veyngg{\E}} sahar neeratt-ukku! CMT \\
 Ma rice little  cook keep sahar time-\textsc{dat}  \\
\ex%51
\gll Amma, sahar velaaw{\E}-{\textrtailt}{\E} bat {\textrtailt}ikak \textbf{uyaala} \textbf{tiyann{\E}}! S \\
 Ma sahar time-\textsc{dat} rice little cook keep\\
`Mum, steam some rice (and set it aside) for the \textit{sahar} (ritual Ramadhan meal)!
\z
\z

\subsection{\trs{Liath}{see}}
{\em Liath} conative, V+\textit{liath} mean `try to V', e.g.
\trs{pake}{wear} + \trs{liath}{see} = `wear see' (try on some garment):

\ea
\ea\label{jaffar:ex:52}
\gll ithu kemeja=nya \textbf{pake} \textbf{liath} SLM \\
 \textsc{dist} shirt=\textsc{acc}   wear see\\
\ex%53
\gll and{\E} {\textesh}a{\textrtailt}{\textrtailt}ey=e \textbf{u{\dz}uttu} \textbf{paarung{\E}} CMT \\
 \textsc{dist} shirt=\textsc{acc}   wear see\\
\ex%54
\gll ee kamise \textbf{{\ae}{\und}{\E}la} \textbf{balann{\E}} S \\
 \textsc{dist} shirt   wear see\\
 `Try on that shirt !'
\z
\z

\subsection{\trs{(H)abis}{finish}}
{\em (H)abis}, completive, conveys a meaning of V completed,

\trs{Makang}{eat} + \trs{(h)abis}{finish} = `eat finish', `finish eating':

\ea\label{jaffar:ex:55}
\ea
\gll nyari sore kithang siang\~{}siang su-\textbf{makang} \textbf{habis} SLM \\
 today evening 1\textsc{pl}    early\~{}\textsc{red} \textsc{past}-eat finish\\
\ex%56
\gll ind{\E} andikk=i neerattood{\E} naang{\E} \textbf{tindu}              \textbf{mu{\dz}icci{\textrtailt}{\textrtailt}}=oom CMT \\
 this evening=\textsc{dat}  early     1p  eat.\textsc{cp}  finish.\textsc{past}=1p\\
\ex%57
\gll ad{\E} h{\ae}{\und}{\ae}{\ae}we api veelapaing \textbf{kaala} \textbf{ivarayi} S \\
 today evening 1\textsc{pl} early eat.\textsc{cp}  finish\\
`We finished eating (dinner) very early this evening.'
\z\trs{(H)abis}{finish}
\z

\subsection{\trs{Kasi}{give}}
{\em Kasi}, benefactive, conveys the notion of V being benefactive to the patient:

\ea
\ea\label{jaffar:ex:58}
\gll the athu cangker me-\textbf{thuang} \textbf{kasi}=si? SLM \\
 tea one cup \textsc{inf}-pour give=\textsc{interr}\\
\ex%59
\gll tee oru koope \textbf{uutti} \textbf{tara}=vaa? CMT \\
 tea one cup pour give=\textsc{interr}\\
\ex%60
\gll tee kooppe=yak \textbf{hadala} \textbf{denn{\E}}=d{\E} S \\
 tea cup=3\textsc{sg} make give=\textsc{interr}\\
 `May I pour (you) a cup of tea?'
\z
\z

\subsection{\trs{Tharo}{put}}
{\em Thaaro} adds the meaning of detrimentally affective to Patient,

\ea\label{jaffar:ex:61}
\ea
\gll ithu pohong jàrok=nya derang su-\textbf{pothong} \textbf{tharo} SLM \\
 \textsc{dist} tree lime=\textsc{acc}  3\textsc{pl} \textsc{past}-cut put\\
\ex%62
\gll and{\E} dee{\textesh}ikka maratt=e \textbf{vetti} \textbf{poo-{\textrtailt}{\textrtailt}u{\textrtailt}aa={\ng}g{\E}} CMT \\
 \textsc{dist} lime tree=\textsc{acc}  cut put-\textsc{past}=3\textsc{pl}\\
\ex%63
\gll ara dehi gaha \textbf{kapala} \textbf{d{\ae}mma} S \\
 \textsc{dist} lime tree  cut.\textsc{cp} put.\textsc{past}\\
 `They cut down that lime tree.'
\z
\z

\subsection{\trs{Duduk}{sit}}
{\em Duduk}, durative, conveys an ongoing aspect of V:

\ea\label{jaffar:ex:64}
\gll nene=nya arà-masak waktu cucu=pada arà-\textbf{maeng} \textbf{duduk} SLM \\
 grandmother=\textsc{poss} \textsc{nonpast}-cook time grandchildren=\textsc{pl} \textsc{nonpast}-play sit\\
`While their grandmother was cooking, the children were playing.'
 [CMT and S do not have an equivalent SVC with `sit' as a vector verb].
\z

\subsection{Verb reduplication}
Two more examples of Serial Verb Construction, actually a process of reduplication to indicate an ongoing process, bear reporting and these appear to be calqued on Sinhala:

\subsubsection{CONTINUATIVE or PROGRESSIVE reduplicated verb}
 \textit{nangis}\textit{${\sim}$}\textit{nangis}: `weep.weep' to mean `keep on weeping'

\ea
\ea\label{jaffar:ex:65}
\gll Mina kamar=ka \textbf{nangis${\sim}$nangis} arà-duduk  SLM \\
 Mina room=\textsc{loc}  weep${\sim}$weep \textsc{nonpast}- sit\\
\ex%66
\gll Mina kaamar=e \textbf{an{\dz}{\E}${\sim}$an{\dz}aa} inn{\E}vaa  S \\
 Mina room=\textsc{loc}  weep${\sim}$weep \textsc{nonpast}- sit\\
`Mina is weeping (keeping on weeping) in the room.'
\z
\z

It must be noted here that Bahasa Indonesia has both \textit{bertangis${\sim}$tangisan} and \textit{menangis${\sim}$nangis} to convey the continuative-progressive sense of `keeping on weeping' --- and it is quite possible that SLM has carried this construction over from its forbear, Vehicular Malay. 

\subsubsection{PROGRESSIVE reduplicated SVC with \textit{ambel}}

\ea
\ea\label{jaffar:ex:67}
\gll \textbf{lari} \textbf{ambel${\sim}$lari} \textbf{ambel} bas=ka as-lompat de su-naik SLM \\
 run take${\sim}$run take bus=\textsc{loc}  \textsc{cp}-jump 3\textsc{sg} \textsc{past}-climb\\
\ex \label{jaffar:ex:68}
\gll \textbf{oo{\dz}i} \textbf{kon{\dz}u${\sim}$oo{\dz}i} \textbf{kon{\dz}u} paanji bas=le eeri{\textrtailt}aan  CMT \\
 run take${\sim}$run take jump bas=\textsc{loc}  climb\\
\ex%69
\gll \textbf{duvaa} \textbf{g{\E}n{\E}${\sim}$duvaa} \textbf{g{\E}n{\E}} p{\ae}nala bas=ekee n{\ae}gga  S \\
 Run take${\sim}$run take jumped bus=\textsc{loc}  climb\\ 
 `He ran and (while running) hopped on the bus'
\z
\z

\subsection{The case of \textit{pukul}: vector verb or full verb serialization?}\label{jaffar:sec:5}

\citet[186]{Nordhoff2009} classifies \textit{pukul} as a vector verb, as in \xref{jaffar:ex:70} \textit{buvang-puukul} and \xref{jaffar:ex:71} \textit{bale-king puukul}, and states that it is `indicative of aggressive or vigorous activity.' This meaning, however, is not attested in the regular speech of Malays in Colombo and suburbs, Hambantota, and Badulla, which suggests that we may be dealing with a form of idiolectal usage.

\ea\label{jaffar:ex:70}
\gll 
[Incayang=pe\footnotemark{} kàpaala=ka anà-aada] thoppi=dering moonyeth pada=nang su-\textbf{buvang} \textbf{puukul}\\
3\textsc{sg}.\textsc{polite}=\textsc{poss} head=\textsc{loc}  \textsc{past}-exist hat=\textsc{abl}  monkey \textsc{pl}=\textsc{dat}  \textsc{past}-throw hit\\
`He took the hat from his head and violently threw it at the monkeys.'
\z

\footnotetext{Items 
 \xref{jaffar:ex:70} and \xref{jaffar:ex:71} are taken from \citet[186]{Nordhoff2009} where they bear the numbers (83) and (84) respectively.
} 
A careful parsing of \xref{jaffar:ex:70} will show that the meaning is one of `\textbf{striking} by means of throwing' (\textit{buang+pukul}, typically a full verb serialization), and not one of violent throwing.

\ea\label{jaffar:ex:71}
\gll Ithu=kapang ithu moonyeth pada=le [anà-maayeng duuduk thoppi] pada=dering inni oorang=nang su-\textbf{bale-king} \textbf{puukul} \\
 dist=when \textsc{dist} monkey \textsc{pl}=\textsc{addit} \textsc{past}-play sit hat \textsc{pl}=\textsc{abl}  \textsc{neg} man=\textsc{dat}  \textsc{past}-return-\textsc{caus} hit\\
 `Then the monkeys threw back the hats with which they had been playing.'
\z

Again, in \xref{jaffar:ex:71} above, \textit{bale-king puukul} (\textit{balek-king pukul}) is clearly `\textbf{strike} back', not `throw back violently'.\footnote{The 
 verb \trs{buang}{throw} is nowhere in the example even though the author notes that for \textit{puukul} as a vector verb `the only instances are related to throwing an item,' (\textit{idem}).
}
Both verbs contribute to the semantics of the statements in \xref{jaffar:ex:70} and \xref{jaffar:ex:71} with their individual literal meanings, and these examples with \textit{pukul} must perforce be read as full verb serializations (FVS). Note that the productivity of \textit{pukul} as a vector is severely constrained for otherwise we should have expected to encounter *\textit{tholak pukul} `push violently', *\trs{thàndang pukul}{kick violently}; *\trs{tharek pukul}{pull violently}; and even *\trs{pukul pukul}{hit violently}. 

\section{Conclusion}
Sri Lanka Malay serial verb constructions are prototypical in that they are a sequence of verbs acting together as a single predicate without any overt marker of coordination, subordination or syntactic dependency of any sort; they operate in a series, mainly contiguously, to provide in a synergistic manner a shade of meaning that each verb is incapable of conveying on its own; consistent with the conditions suggested in \citet{Aikhenvald2006}, they describe what can be conceptualized as a single event; they are monoclausal; their intonational properties are those of a monoverbal clause; they have just one tense, aspect and polarity value; and, additionally, they share all arguments. In multi-word constructions (as opposed to one-word or compound serial verbs) they are of the `single marking' variety, i.e. they are marked just once and not on every component verb. Verbal combinations involving \textit{ambel, (à)mbath} and \textit{kàna} have all undergone a process of lexicalization and this is evident from the fact that their meaning is not compositional. All the other verbal sequences we have examined carry meanings that are essentially compositional in the sense that each of the two verbal elements retains its original semantics. Nearly all of the SVCs examined in this paper seem to be calqued on corresponding constructions in Colombo Muslim Tamil, and sometimes on those of Sinhala, suggesting a notable adstrate influence in the evolution of the language. Nonetheless, there are a few constructions --- especially in the `reduplicating' variety \xref{jaffar:ex:68} and in the \textit{kàna} `pseudo-Passive' variety [cf. \xref{jaffar:ex:44} and \xref{jaffar:ex:46}] --- where there is reason to postulate a Vehicular Malay origin. 
 


% 
% {\centering\itshape
% Mohamed Jaffar, York University, Toronto
% \par}
% 
% York University Glendon Campus\newline
% 2275 Bayview Avenue\newline
% Toronto, Ontario\newline
% M4N 3M6 \newline
% Canada
% 
% Email: mjaffar@yorku.ca

