

\chapter[Power and privilege in SLM maintenance]{Issues of power and privilege in the maintenance of Sri Lanka Malay: A sociolinguistic analysis}
\chapterauthor{Romola Rassool}{University of Melbourne \& University of Kelaniya}

% \begin{abstract}
% Although there has been a general consensus among the Sri Lanka Malay community that the Sri Lanka Malay (SLM) language is facing a degree of endangerment, its members are divided as to what measures should be taken to arrest the issue of language loss. While some segments of the community are of the opinion that the SLM language must be taught, promoted, and thus strengthened, others are of the opinion that Malaysian Malay or Indonesian should be taught as a means of revitalising Sri Lanka Malay by converging it with a more `standard' variety. 
% 
% This paper is a critical analysis of the notion that the SLM language is `endangered'. In particular, it is an examination of how the notion of `endangerment' is used to by various agents to justify a range of language revitalisation efforts, each with its own political agenda. With regard to the members of the SLM community who have advocated a move towards a more `standard' language, I maintain that such advocacy stems from a politics of disempowerment and a desire to perpetuate the hegemonic situation that already exists between the urban, educated, affluent minority of the community and the rural majority. In this regard I ask if the larger SLM community is not doubly disadvantaged: on the one hand by the Malays' minority political status in Sri Lanka and, on the other, by members of its own urban community. 
% 
% \end{abstract}

\section{Introduction}\label{rassool:sec:1}

The Sri Lanka Malays are a minority community who form 0.3\% of the total population of Sri Lanka.\footnote{I 
 would like to express my gratitude to Michael Ewing, Sander Adelaar, the editor of this volume, Kaushalya Perera and Dinali Fernando for their comments on earlier versions of this paper. I am also thankful to Eileen Dane and Naushad Rassool for their comments on various aspects of the Sri Lanka Malay community and to Harsha Vithaanaarachchi for helping me with formatting and citation issues. 
}
 At present there are 55,353\footnote{This
  number is arrived at by adding the 54,782 quoted in the census of 2001 (\url{http://www.statistics.gov.lk/PopHouSat/PDF/Population/p9p8 Ethnicity.pdf}) and the 327 in the Trincomalee district, 163 in the Ampara district, and 81 in the Batticoloa district reported in the preliminary reports of the census conducted by the Department of Census and Statistics of Sri Lanka in 2007. The report on the census of 2001 did not contain information regarding the population of these districts due to the fact that travel to those districts was not possible because of the inter-ethnic conflict that prevailed in the north and east of Sri Lanka during that period. However, in my own visit to Trincomalee I observed that the number of Malays there is probably well over the figure stated in the census and my observation was confirmed by many Malay residents of the district. It is possible that the numbers in the other districts are similarly understated. Also, these are the figures pertaining to the section of the Sri Lankan population that has identified themselves as Malay by ethnicity and is not a reflection of the number of people claiming proficiency in SLM.
}
Malays in Sri Lanka and they are spread mostly in the Western, Central, and Southern provinces of the country. They trace their origins to the Malays who migrated to Sri Lanka (then Ceylon) primarily during the period of Dutch colonial expansion. The lexifier of their language, Sri Lanka Malay (SLM), is Indonesian in nature, while its syntax, morphology and phonology have been considerably influenced by Sinhala and Tamil, the major languages of Sri Lanka. Since the present state of the SLM language is seen by many individuals in the community as a consequence of Sri Lanka's language policy of the mid 20\textsuperscript{th} century, I provide a brief summary of the political changes of that period and how they affected the SLM language. 

Though Sri Lanka was colonised by the Portuguese (1505 -- 1656), the Dutch (1656 -- 1796) and the British (1796 -- 1948), it was not until British rule that inter-ethnic tensions between the majority Sinhalese population and the minority populations emerged to the foreground. Much of the tension centred round the Sinhalese and the Tamil population, the largest minority ethnic group. Feeling that the Tamils were advantaged during British rule, in the lead-up to and after the declaration of independence in 1948, the Sinhalese government introduced policies that were perceived as privileging the Sinhalese people. Much of the political discourse of the time surrounded the issue of language. In 1956 the Official Language Act (popularly known as the `Sinhala-only' Act) was introduced, declaring Sinhala the sole official language. Though this act was subsequently amended through various provisions and Sinhala and Tamil are both declared national languages with English as the `link' language, it is commonly believed that this unequal language policy led to the inter-ethnic rivalry that is still in evidence between the two largest ethnic groups in Sri Lanka. These tensions climaxed in a protracted conflict between the Liberation Tigers of Tamil Eelam (LTTE) and the security forces of Sri Lanka over demands by the Tamil Tigers for a separate homeland in the north of Sri Lanka, which ended in 2009 when the LTTE was militarily defeated by the armed forces. 

The Sri Lanka Malays felt that they suffered a setback due to the new language policy. Prior to the 1956 Act, they had been assured of jobs due to their proficiency in English. The 1956 Act has also been seen as the indirect cause of the decline of the SLM language. According to \citet{Saldin2001}, when English was the medium of instruction in schools, Malay was the language spoken in the home. When the medium of instruction became Sinhala or Tamil, Malay parents felt they needed to speak to their children in English, in order to ensure that the children knew English, which was (and still is) seen as crucial for employment and upward mobility. Therefore, in a sense, the Official Languages Act of 1956 is considered to be responsible for Malay ceasing to be the language of the home \citep{AnsaldoEtAl2006,Bichsel1989,LimEtAl2007,Saldin2001}. 

The notion that the Sri Lanka Malay language is `endangered' has been unquestioned since it was first introduced in the late 1980s and early 1990s \citep{Hussainmiya1987,Saldin2001} and is now taken as a given by members of the community and linguists alike. While there are aspects of a language endangerment situation, the basis of this classification has not been adequately critiqued in the literature. The language shift witnessed in urban centres such as Colombo and Kandy is cited as evidence for this language loss \citep{Ansaldo2008genesis,AnsaldoEtAl2006,LimEtAl2007}. This perception of language loss has led many individuals from within the community and from the linguistic community overseas to take an interest in revitalising the language. One of the proposed remedies has been to teach Malay to members of the community \citep{Saldin2001,Saldin2000}. But the community is in a quandary as to which variety should be taught: Sri Lanka Malay or one of the `standard' forms of Malay as used in Malaysia and Indonesia?

Within the context of the above discussion, this paper aims to discuss the implications of using the notion of `endangerment' by members of the community as well other interested parties as a means of justifying the teaching of `standard' Malay, and to highlight the ideologies that underlie such a move. It also examines the views of a group of five urban Malay youth on which variety of Malay they think should be taught. 

While there is a growing body of literature regarding the SLM language, a quick survey of the writings reveals that the sociolinguistic aspects of the language remain seriously understudied. The first attempt to examine the phonology of the language was undertaken by \citet{Tapovanaye1986} who later examined the segmental phonology and vowel length of SLM (1995).\nocite{Tapovanaye1995} \citet{Bichsel1989} undertook a historical analysis of the linguistic development of SLM and \citet{Kekulawala1979} examined the kinship terms used in SLM in relation to language universals. Adelaar traced the roots of SLM in \citet{Adelaar1991} and \citet{AdelaarEtAl1996} classified the language spoken by the ancestors of the SLM community as a Pidgin-Malay Derived (PMD) variety. The dialect situation of SLM is discussed by \citet{Robuchon2003}. \citet{Smith2003timing} presented evidence from nominal inflection for Tamil as the major substrate language in SLM. \citet{Slomanson2003} examined the verb system of SLM and concluded that the verb in SLM has not converged with Tamil and Sinhala the same way the nominal system has. 
%I believe Ansaldo 2008, 2009, 2011 has also said something about the origins of SLM grammar, might be worth acknowledging? It does also have sociolinguistic implications{\dots}
\citet{SmithEtAl2004} concluded that the lexifier of SLM is Indonesian in nature while the influence of the local languages is strongest in syntax and weakest in phonology
%Add reference to list (2004) and (2006)
In 2006, Smith and Paauw  \nocite{SmithEtAl2006cll} argued that Tamil is the main source language for the TMA categories of SLM, citing linguistic as well as sociohistoric evidence. \citet{Slomanson2008lingua} discussed the morphosyntactic complexity of the SLM verb. Recently, the various aspects of the grammar of Upcountry Malay have received detailed attention from \citet{Nordhoff2009}. The only work of a sociolinguistic nature has been undertaken by \citet{AnsaldoEtAl2006}, \citet{LimEtAl2006,LimEtAl2007}, and \citet{Ansaldo2008genesis}, and will be discussed in detail in this paper. In addition, \citet{Ansaldo2009book} has discussed the multifaceted nature of the linguistic repertoire as well as the identity of the Sri Lanka Malays and suggested that notions such as language `shift' and `death' may not apply well to multilingual communities such as the Sri Lanka Malays. %
%Add to ref list Ans (2009)
Given that in present-day Sri Lanka SLM arguably attracts the most academic interest and is in the most dynamic state of all languages, it is important that scholarly attention be paid to the sociolinguistic aspects of the language. This is especially important since inter-communal and intra-communal power differentials affect the ideologies that underpin language maintenance initiatives, as will be discussed in this paper. 

The nexus between language ideologies, identities, and power has been examined by many linguists \citep{BlackledgeEtAl2004,Heller1995,Heller1999,Hornberger1998, IrvineEtAl2000,Skutnabb-KangasEtAl1998}. \citet{Bourdieu1977,Bourdieu1982,Bourdieu1991} views linguistic diversity as inextricably linked to the production and reproduction of social inequality. He also views linguistic practices as a form of symbolic capital, convertible into economic and social capital, and distributed unequally within any given speech community. Influenced largely by Bourdieu's notions, poststructuralist linguistic anthropologists have examined the complex 
interplay between language ideologies and power structures. Approaching the field with the consensus that ``the exercise of power, in modern society, is increasingly achieved through ideology, and more particularly through the ideological workings of language'' \citep[2]{Fairclough1989}, linguists have focused on studying the ``contestable, socially positioned, and [political]'' nature of language ideologies  \citep[382]{HillEtAl1992}. Informed by these views, I examine the current situation of the Sri Lanka Malay language and the political underpinnings of various language revitalisation initiatives. 

In the next section, I discuss the notion of language endangerment, how it has been applied to the SL Malay context to justify various political stances, and the ideologies behind such justifications. 

\section{The formulation of SLM as an `endangered' language} %2

Over the last two decades, scholarly attention has been paid to the alarming loss of languages \citep{Abley2003,Crystal2000,Dalby2003,Florey2005,Kinkade1991,Krauss1992, NettleEtAl2000,RobinsEtAl1991,Wurm1996unesco}. Language shift in its simplest terms can be viewed as loss of speakers and domains of use \citep{Romaine2006}. Language maintenance occurs when the community collectively decides to continue using the language or languages it has traditionally used \citep{Fasold1984}. Language revitalisation involves efforts to impart new vigour in a language already in use through increasing the language's domains, often entailing increased institutional power \citep{Paulston1994}. 

Highlighting the political nature of language loss, \citep[37]{May2004} states that ``[l]anguage loss is not only, perhaps not even primarily, a linguistic issue -- it has much more to do with power, prejudice, (unequal) competition and, in many cases, overt discrimination and subordination''. So too is the case with language revitalisation, as the case of Sri Lanka Malay, discussed below, illustrates.  

A few members of the SLM community  \citep{Saldin2000,Saldin2001,Saldin2007dico, Thaliph2005} and non-Malay linguists \citep{Ansaldo2008genesis,AnsaldoEtAl2006, LimEtAl2006,LimEtAl2007} have focused on the issue of language vitality with regard to the SLM community. These discussions have been based on the assumption that Sri Lanka Malay is `endangered' \citep{Saldin2001,AnsaldoEtAl2006,LimEtAl2007}. The political nature of this formulation of endangerment and its consequences deserves further analysis. 

In the discussion on what form language revitalization efforts should take, the voice from the community that is most frequently heard is that of B. D. K. Saldin (2001, 2000, 2007)\nocite{Saldin2000,Saldin2001,Saldin2007dico}. A member of the educated urban middle-class, Saldin has for many years advocated an alignment with `standard' Malay. For instance, in \citet[xii]{Saldin2007} he states:

\begin{quote}
Assuming that both the expertise and funding are available, it would perhaps be ideal to revive the literary Malay, which is not being spoken in the Malay world. However, if we do this, then the vast majority of Malays are bound to feel alienated because `standard' Malay is unintelligible to most Sri Lankan Malays{\dots}
\end{quote}

This statement illustrates Saldin's awareness that the `standard' variety of Malay is `unintelligible' to the majority of the Sri Lanka Malays. It then becomes necessary to ask why a respected elder of the community advocates the teaching of `standard' Malay, when, by his own admission, it is ``a foreign tongue'' to most Sri Lanka Malays (2007: xii).

A close examination of the ideologies that underpin Saldin's writings reveals his negative stereotyping of the members of the community. He has positioned the Sri Lankan Malays as `indifferent' to the plight of their language. He implies that it is this perceived `indifference' that is responsible for the endangerment of the language and the erosion of their identity. Wishing to draw the readers' attention to the fact that Malays should speak their language in order to be distinguished from their co-religionists, the Moors, \citet[vii]{Saldin2001} states: 

\begin{quote}
The Malays in Sri Lanka have been lamenting the fact that their identity is being glossed over and being submerged by that of the Moors, and that they are being referred to by the all encompassing term of Muslim. My personal view is that the Malays have only themselves to blame for this. 
\end{quote}

Saldin has also stated that the Sri Lanka Malay community ``may not be bothered'' about the future of the language: 

\begin{quote}
Since the majority of Malays are more concerned with the problems of survival, they may not be bothered about doing something that would not bring them any economic gains. Therefore if one leaves it to the silent majority to revive a language nothing will come out of it. Some sort of guidance will not be out of place for those interested in preventing our language from dying altogether. \citep[vii]{Saldin2000}. 
\end{quote}

This quote reflects Saldin's attitude to many salient issues:

\begin{enumerate}
\item While acknowledging Saldin's view that the majority of the Malays have economic burdens which prevent them from focusing on issues such as language, it is necessary to point out that there exists an attitude of `talking down' in Saldin's concern for the future of the language.  In the above quote, Saldin is also giving his judgment that, if left to the community to decide, ``nothing will come out of it.'' Yet, it is important that the community be consulted in the matter of the0 future of their language. 
\item His view that the rural Malay community may not want to engage with their urban counterparts in working towards a solution to language loss is impressionistic and not backed by empirical research. 
\end{enumerate}
The above quotes illustrate instances of what \citet{Skutnabb-Kangas1998} discusses as means by which groups are rendered invisible and are indeed invalidated through the labels used to characterise them. The practice of blaming the members of the community for what Saldin perceives as lack of commitment cannot have positive effects on the way the problem is perceived by the community and cannot be seen as encouraging -- it is more likely to be disempowering. This opinion and its politics of disempowerment can have grave implications on a minority community of speakers of whom many believe they speak a `broken' or `sub-standard' variety of Malay.  

This review of Saldin's writings reveals the ``partial'' and ``interest-laden'' \citep{HillEtAl1992} nature of his language ideology. It is also evidence for Kroskrity's (2000) \nocite{Kroskrity2000} observation that language ideologies are constructed in the interests of a particular group. By promoting an alignment with Malaysian Malay, Saldin is serving the interests of the urban SL Malays who are most privileged by such a move. Ideologically, the members of the urban minority favour this move because it enables them to feel linked to the larger ``\textit{rumpun Melayu}'' (Malay stock) and more concretely, it means they can communicate in Malaysian Malay when they travel to Malaysia and Indonesia, as they frequently do.

The community's efforts to teach Standard Malay to its members must also be viewed in terms of the ideologies of the Malaysian government through its high commission in Colombo. The high commission conducted two courses in `standard' Malay for members of the Malay community (in 2002 and 2003) and the six students who fared best in the courses were trained as language teachers in Malaysia. They were then expected to teach `standard' Malay to their respective sub-communities in Sri Lanka. The next project to teach `standard' Malay was in 2008 when a three-month course was conducted by the Malaysian high commission in Colombo. These initiatives were welcomed and appreciated by the community without any critical inquiry into the possible ideological motives behind such moves and without questioning the possible consequences to the Sri Lanka Malay language. The language programmes and trips to Malaysia were made possible through the offices of the \em Gabungan Persatuan Penulis Nasional Malaysia \em (GAPENA): the Federation of National Writers' Association of Malaysia, which was led by Tan Sri Prof Ismail Hussein. Ismail Hussein's views on the \textit{Dunia Melayu}, which, in his own words, was considered his `hobby', illustrate his ideologies:

\begin{quote}
My obsession is to build a Global Malay Tribe (\textit{Suku Melayu Dunia}). To unify Bangsa Melayu throughout the world. I have a dream that the Melayu people whose total population counts as many as 350 million at present and who reside in hundred thousands of islands, could build a single association and fraternity{\dots} (Irwan Kelana, Warta GAPENA, January 2001, cited in \citet[34-35]{Hisao2010}
\end{quote}


 This political view, with its goal of uniting the Malays under a common banner, is an indicator of the motives behind the teaching initiatives -- to create a sense of identity with and loyalty to the Malaysian state. When he was criticised for investing resources that were better spent in developing Malaysian arts and literature, his response reveals his long-term goals: 

\begin{quote}
When our country is hit by a crisis, then all the relationships and cooperations we have constructed and conducted with the international Malay societies will prove to be very relevant and important{\dots} We have seen how the relationships and cooperations with the wide \textit{Dunia Melayu} [Malay world] could be a key for the survival of the Malay language and literature in this country itself.'' (Warta GAPENA, July 2003 cited in \citet[34]{Hisao2010}.
\end{quote}

These views suggest a possible agenda behind the seemingly `helpful' and `benevolent' decision to teach Standard Malay to the Malay community of Sri Lanka. The Malays of Sri Lanka are wooed by the Malaysian government, which chose to conduct the second GAPENA conference in Colombo (1985), arranges periodic visits by representatives of the Malaysian government to the Malay Club in Colombo where grants are given for various community projects, finances occasional trips to Malaysia for members of the urban segment of the community to attend conferences and seminars at the expense of the Malaysian state etc. The fact that the SL Malay community's vulnerable minority status and its weak political position are possibly being exploited to propagate a Pan-Malay ideology is something the community is either ignorant of or chooses to overlook. Highly relevant to this argument is the fact that the beneficiaries of the `generosity' of the Malaysians are the most forceful advocates of teaching Standard Malay. Some members of the community, when they have an occasion to address the Malaysian or Indonesian officials, have appealed to them to help their Malay brethren, thereby probably justifying the `reaching out' the Malaysians do through their linguistic and cultural activities. For instance, in his address to the Malaysians at the International Symposium on Malayo-Polynesians in the Commonwealth held in Malaysia in 1998, one member of the SLM community asked 

\begin{quote}
[our] motherlands -- Indonesia and Malaysia -- to direct their \textit{benevolent} thoughts (faidat-ul-fikr) towards their \textit{hapless prot\'eg\'es} -- the Sri Lanka Malays -- in initiating \textit{beneficial actions} (amal-ul-khair) for maintaining \textit{good feelings of humanity} (barakat-ul-insan) and effective cultural rapport{\dots}'' (Thaliph1998, p. 5), (italics mine)
\end{quote}

Such ideas, which are expressed in a language appealing to the Malaysians' religious sense (through the use of Arabic terminology) as well as emotional sense (as highlighted in the italicised terms), place the Malaysians unquestioningly in a position of power and cast the SL Malays as their `powerless' brethren. 

In my interviews with the Malays from various parts of the country, I only heard one opinion that was openly critical of the Malaysian initiatives: 

\begin{quote}
I understand the reason why the Malaysian High Commission would be keen on conducting these classes and trying to teach us Malays to study their language: because of [their idea of making] Bahasa Melayu, Bahasa Dunia (INT/WP/12).
\end{quote}

This opinion, also from a member of the urban elite, stands in stark contrast to the views of the rest of the community, although it is quite possible that there may be similar opinions which are not stated publicly. While I acknowledge my own subjective reading of the events and comments that are described in this section, what needs to be emphasised is that there is little critical analysis of the power dynamics that underlie what form language revitalization efforts should take. From a critical ideological perspective, teaching `standard' Malay to the community is a tacit acknowledgment and affirmation of the unequal power distribution between the two communities. Interestingly, I was informed by the member of the community who was liaising between the Malaysian officials and the SLM community when the classes were being organised that all offers to help ceased when the Malaysians understood that the SLM community was at a crossroads and wondering whether it should continue to teach `standard' Malay or promote SLM in its stead. 

The Indonesian government does not seem to have similar motives when it offers to help maintain the Malay language in Sri Lanka through its representatives in its embassy in Colombo. The Indonesian language classes that are conducted by the embassy are open to all Sri Lankans, regardless of ethnic origin, unlike the classes held by their Malaysian counterparts, which are exclusively for Sri Lanka Malays. The difference in ideologies has been commented on by \citet{Tirtosudarmo2004} who states: ``[w]hen it comes to the notion of `Malayness'{\dots}the Malaysians are much more assertive than the Indonesians'' (2004, p. 2). 

Having explored the ideologies behind the views espoused by Saldin and the initiatives of the Malaysian high commission, I turn to an examination of how researchers into the SLM language have perceived the language loss. 

\citet{Ansaldo2008genesis} states that SLM varieties ``are currently endangered as they are no longer spoken by the younger generation, with one exception, the community in Kirinda'' (2008, p. 14) and he states that this is due to the fact that the younger generation is abandoning the language because of their desire to converge with the dominant languages of the country and thus avoid being disadvantaged.  With particular reference to the situation in Colombo, \citet{LimEtAl2006} state that ``SLM in [the Colombo Malay] community is no longer a home language for the younger generation and is thus considered an endangered language'' (2006, p. 3). As is evident from these statements, the criterion seems to be the maintenance of the language by the younger generation. However, \citet{Himmelmann2005endangered} points out the need to differentiate between the symptoms and causes of endangerment, and it seems that in this context, the reduction in the number of children speaking SLM could be a symptom for which the causes are to be found elsewhere. Himmelmann states that language endangerment may be defined as ``a rapid decline in the number and quality of domains in which a given language is used'' (2005, p. 3). Therefore, I argue that the notion of `endangerment' should be problematised to include an examination of domains of use, attitudes to the language, and possibly more criteria.  

Ansaldo presents his assessment of the vitality of the language in an overview of the SLM speech communities, reproduced in Table \ref{rassool:tab:1}. 

\begin{table}
\begin{tabular}{p{2cm}p{4cm}p{4cm}}
Community &Characteristics &Vitality\\
Colombo &
Middle-upper class; often bi- or trilingual (Tamil/ Sinhala) standardising in Malay; restricted use of SLM; English fairly fluent. &
Endangered; no SLM in younger generation.\\
Slave Island (Colombo) &
Lower class; most Tamil influenced; bi- or trilingual; no English. &
Very endangered; use of SLM discouraged.\\
Kandy (and Hill Country)  &
Similar to Colombo community; weak standardisation forces. &
Endangered.\\
Hambantota &
Traditionally heavy Sinhalese-speaking area; low-middle class, often trilingual; limited English. &
Mildly endangered.\\
Kirinda &
Lower class; good trilingual competence in middle-younger generations; English limited to a few individuals. &
Fully vital; mother tongue even in present generation.\\
\end{tabular}
\caption{SLM speech communities \citep[from][]{Ansaldo2008genesis}.}%Table 1: 
\label{rassool:tab:1}
\end{table}

Generalisations based on locations and age groups are a necessary and useful point of departure for studies of this nature. However, they need to be questioned in order to avoid essentialism and to move towards a more nuanced view of the current situation.  While acknowledging the potential helpfulness of social class as a category along which endangerment is determined, it is useful to delve deeper into what actually constitutes social class. Similarly, the employment of labels such as `endangered' and `very endangered' can be problematic when they are used without adequate definition. In summary, it is felt that the categorisation of the SLM speech communities and their features can benefit from a more critical and detailed examination. 

Having reviewed the opinions of the urban minority, the Malaysian officials, and the researchers, it is necessary to reflect on how the ideologies of these `powerful' agents potentially affect the larger SLM community. 

The majority of the Sri Lankan Malays possess little socioeconomic power and live away from the urban centres of the country. Generally speaking, they are also less proficient in English and, in a country where English has huge social and economic capital, have lower social standing as a result. Using Bourdieu's (1977, 1982, 1991) \nocite{Bourdieu1977,Bourdieu1982,Bourdieu1991} notion that linguistic practices are a form of symbolic capital, which is convertible into economic and social capital, it can be pointed out that the majority of the SLM population are doubly disadvantaged: on the one hand, they are part of a minority community whose linguistic (and other) interests are largely overlooked by the central government, and on the other, they are dominated by a more socioeconomically powerful minority within their own community.

Bourdieu's model of symbolic domination rests on his notion that the dominated group is complicit in the misrecognition, or valorisation, of one language or variety as an inherently better form than another. It is possible that, after many years of having their variety of Malay devalued by the dominant segment of the Sri Lanka Malay society, the dominated group has misrecognised their form of Malay as inferior to `standard' Malay. 

The nexus between language and power becomes crucial to the question of the language ideologies of the SLM community. \citet{Heller1982} links language and power in two significant ways. On the one hand, language is seen as part of the processes of social action and interaction and in particular, as a way in which people influence others. On the other, it is a symbolic resource which may be tied to the ability to gain access to, and exercise, power. In the case of the SL Malay community, the dominant minority exert power over the rest of the community through two languages: the first is English, the knowledge of which is seen as symbolically opening the doors to privilege and wealth, and the second is `standard' Malay, the knowledge of which connects the users to the larger Malay world of the East. 

It has been observed that ``in the most restrictive formulations of this connection, ideology is always the tool, property, or practice of dominant social groups; the cultural conceptions and practices of subordinate groups are, by definition, nonideolgical'' \citep[7]{Woolard1998}. I believe that the field of SLM studies can benefit from a questioning of this position as it is possible that subordinate groups might have an alternative ideology, but it does not get voiced due to the entrenched dominance of the more powerful minority. \citet{BucholtzEtAl2004} state that ``speakers who resist, subvert, or otherwise challenge existing linguistic and social norms are vital to the theoretical understanding of identity as the outcome of agency{\dots}'' (p. 373).

\section{The language--identity nexus in the SLM context} %2

Recent work in language and identity in the fields of sociolinguistics and linguistic anthropology has emphasised the need for the examination of identity as a social and linguistic construction \citep{BucholtzEtAl2004,Joseph2004}. There has also been emphasis on the multiple nature of identity -- for instance, \citet{Joseph2004} posits that one's identity shifts according to the context one finds oneself in and also that it there are many versions of one's identity based on other people's construction of it. 

While some linguists view language and ethnic identity as inextricably linked \citep{Fishman1991,Joseph2004,Pozzetta1991}, more recent work has questioned the assumed relationship between the two. Within sociolinguistics, sociology, and the anthropology of ethnicity, there is a growing consensus that language is at most only a contingent factor of one's ethnicity. \citet{Edwards1985,Edwards1994,Edwards2001} and \citet{Eastman1984} assert that language is often only a surface or secondary characteristic of ethnicity. \citet{Barth1969} cautions us against assuming a simple one-to-one relationship between ethnic units and cultural similarities or differences such as language. In other words, the specific linguistic community decides how salient language is as a marker of ethnicity: ``it is the \textit{perceived} usefulness of these cultural attributes in maintaining ethnic boundaries which is central'' \citep[40]{May2004}. The above notion seems to suggest that there is nothing intrinsic to one's ethnicity and thus specifically rejects any automatic link between ethnicity and language. This view is also in keeping with the postmodern/poststructuralist trend of viewing all forms of identity as multiple, shifting, contingent, and hybrid. But that does not mean it will never be a significant or constitutive factor of identity. In theory, language may well be just one of many markers of identity. In practice, it is often much more than that since the language-identity link encompasses significant dimensions that are both cultural and political. One's individual and social identities, and their complex interconnections, are inevitably mediated in and through particular languages. The political dimension is significant because languages formally (and informally) become associated with particular ethnic and national groups. 

My inquiries into the role of language as a marker of ethnic identity within the SLM community have been framed by the academic debates mentioned above. During the course of the fieldwork I conducted for my doctoral thesis from January to July 2010, I learned that all segments of the SLM community perceive that language is intrinsic to one's identity. This might possibly explain how the language has survived this long. Another reason for the community's attachment to the language could be the heightened awareness of ethnicity (and language as a marker of ethnic identity) which came about due to the inter-ethnic tensions and conflict in the island state. The reasons for the conflict are also largely though not exclusively language-based: since the implementation of the Official Language Policy in 1956, the language policies of Sri Lanka are perceived by minority communities as privileging the Sinhalese.  It has been observed that ``language, functioning as a marker of identity and ethnic group interest, has played a major catalytic role in the generation of barriers to national unity and the peaceful development of post-independence Sri Lanka'' \citep[116]{Dharmadasa2007}. 

The present day Sri Lanka Malay community has organised itself into many ``Malay Associations'' under the Sri Lanka Malay Confederation (SLAMAC), and the maintenance of the SL Malay language is high on the agenda of these organisations. Many associations have conducted language classes in the past (Mabole Malay Association, Sri Lanka Malay Association, Kandy Malay association, to name a few) and they have chosen to teach Malaysian Malay, usually by teachers who were trained by the Malaysian government. However, for some of the newer organisations, it is a pressing issue which variety of Malay should be taught. For instance, the Women's Association of Sri Lanka Malays (WASLAM), the newest  social, cultural and religious organisation formed by Malay women (established on 4th April 2010), has stated in its constitution that one of its goals is ``to propagate the use of the Malay language''. The chairperson of the committee for Islam and Malay language states: ``it has not been specified in the WASLAM constitution that Malay language shall be `Sri Lanka Malay language' but, as a matter of policy, the Board of Management will go along with Sri Lanka Malay'' (p.c).  WASLAM is also in the process of organising a seminar ``to arrive at a consensus as to which kind of Malay should be encouraged'' (ibid). This is evidence of a more questioning attitude to the issue of language maintenance than before. 

The issue of how Sri Lankan Malay identity is being (re)negotiated through the community's linguistic preferences has been the subject of some of Ansaldo and Lim's academic writing \citep{Ansaldo2008genesis,AnsaldoEtAl2006,LimEtAl2007}. While acknowledging the varied nature of the linguistic repertoire and the communicative practices of the Sri Lanka Malay community as a whole, \citet{LimEtAl2007} discuss the fact that the Malays of Colombo in particular are facing a situation of language endangerment. This observation has been made previously by many others including \citet{Saldin2000,Saldin2001,Saldin2007}, \citet{Hussainmiya1987,Hussainmiya1990,Hussainmiya2008} and \citet{SmithEtAl2004}. It is an accurate reflection of only one segment of the Colombo Malay population, albeit the most powerful one. My research has shown that in suburban areas around Colombo -- for instance, Wattala, Hunupitiya, Mabole, Makola, Kolonnawa, and Mattegoda -- at the very least -- there are sub-communities of Malays who \textit{do} speak SLM at home. This observation should lead us to reassess our decision to classify sub-communities and their linguistic range solely by geographical location and move towards an inquiry of social class as at least an additional (if not primary) determiner of language maintenance or shift, a process already begun by \citet{Ansaldo2008genesis}. The definition of what constitutes a social class is notoriously problematic, but one that academics need to grapple with if they wish to arrive at a more nuanced understanding of perceived language loss. It must be emphasised that what is being advocated here is not a rejection of geographical location as a determinant  of language vitality but a recommendation that geographical location (including the urban-rural divide) and social class (including but not limited to wealth, education, employment, access to cultural capital) be combined when determining the language vitality of specific sub-communities. 

This leads me to the issue of who provides the linguists with the data that their writings are based on. The awareness that the `gate-keepers' of a community are usually members of a more powerful segment of the community has been observed. Therefore the choice of community members the linguist has access to can be influenced by the ``partial'' and ``interest-laden'' \citep{HillEtAl1992} view of the `gate-keeper', leading even the most well-meaning scholars to make generalisations based on the minute sub-section of the community they have access to. As a result, what the linguists might advocate could be in the interests of that sub-section of the community and not necessarily be in the best interests of the community as a whole. 

I wish to illustrate the above point using Lim and Ansaldo's (2007) assessment of the language loss witnessed among the Colombo Malays. Based on their observations of this sub-community, \citet{LimEtAl2007} state: 

\begin{quote}
it is with Malaysia that the SLM community align themselves, both in terms of language and identity, and the choice in the revitalization process is consequently not for Sri Lanka Malay but for Malaysia's Standard Malay.'' \citep[233]{LimEtAl2007}
\end{quote}


This is a synecdochic situation where a part of the community (the Colombo Malays) is made to represent the whole (all Sri Lanka Malays) and can be illustrated as in Figure \ref{rassool:fig:1}

\begin{figure}

Perceived language loss (among urban Malays) $\Longrightarrow$    Desire to revitalise language $\Longrightarrow$   Desire to learn/teach Malaysian Malay 	$\Longrightarrow$	Identity re-alignment (for all Malays) 
 
\caption{Path from perceived language loss to identity re-alignment.}
\label{rassool:fig:1}
\end{figure}


The academic argument supporting re-alignment with Malaysian Malay is summarised below as it plays a crucial role in intellectualising the shift and thus possibly serving to justify it. \citet{LimEtAl2006} suggest that the acquisition of a more global identity by the SLM community be viewed as positive agency by researchers. This is done by citing arguments of linguistic citizenship, where ``the community is served by its linguistic resources -- which comprise negotiable, multiple, diverse and shifting identities -- and is not restrained by its language'' (2006, p. 6). Lim and Ansaldo state that, viewed from a linguistic human rights paradigm, the case of SLM would be viewed as one of loss of language diversity and forsaking of a unique identity. But they point out that the linguistic human rights paradigm has been criticized in recent years for ``endors[ing] an ethno-linguistic stereotyping in the form of monolingual and uniform identities'' (2006, p. 5). They also state that the linguistic human rights paradigm forces groups of speakers to work actively to differentiate themselves by claiming unique linkages of language and identity so as to gain political leverage in the competition for scarce resources. Alternatively, Lim and Ansaldo recommend that the case of SLM be viewed from a linguistic citizenship perspective, where language is viewed at the same time as a semiotic resource for the (re)construction of agency and self-representation, an economic resource, and site of political and economic struggle, a global resource to address local{}--global concerns, and an intimate resource as the foundation of respect for difference on a global level.

Influenced by current discourses in linguistic citizenship \citep{FreelandEtAl2004,May2004,StroudEtAl2004}, \citet{LimEtAl2007} propose a new interpretation of the traditional notion of `language shift' -- instead of the traditional idea that the shift away from a mother tongue would lead to the loss of a crucial part of the community's identity, Lim and Ansaldo suggest that this shift be viewed more as positive agency on the part of the community. Arguing from a stance which states that the identity of the SL Malay is defined by being multilingual, \citet{LimEtAl2007} also state that a potential shift from SLM to Standard Malay ``does not make a qualitative difference to the Malays' multilingual repertoire, nor to the identity they have'' (p. 225). They are of the view that Standard Malay still fulfills the same function as that of SLM, of identifying them as Malay (both in SL and in the wider Malay world) and that this can be viewed as identity alignment coming full circle: this is the region that constitutes the SL Malays' origins and the SL Malays of the previous generations ``always used to talk about going home'' (Salma Suhood Peiris, 2006, cited in \citet[225]{LimEtAl2007}). 

Aside from the observation that this argument is based on what is viewed only in one sub-community (which was discussed earlier), I wish to comment on some of the researchers' opinions. 

Though there is the perception that there could be economic benefits from learning `standard' Malay, this is a largely misplaced notion: there have been no employment opportunities created as a result of ties with the Malaysian government and even if such openings were to exist in the future, they are most likely to benefit only the sub-community `closest' to the Malaysians -- the urban, Colombo-based Malays. As stated in Section \ref{rassool:sec:1}, any other opportunities that have been granted (scholarships, attendance at seminars and conferences held in Malaysia etc) have also been enjoyed by the urban Malays. In my interactions with the rural Malays, a few participants showed me that they were aware of the fact that  such `perks' were enjoyed by their urban counterparts, and that they were resentful of it. 

Considering that community members who are most likely to acquire `standard' Malay will be from the urban minority, Ansaldo and Lim's recommendation would exacerbate an already existing class division within the community. By making the members of the urban middle class who are proficient in `standard' Malay the `authorities' of the language, such a move can only reinforce the power dynamics that already exist within the community.

Lim and Ansaldo state that a shift to `standard' Malay ``does not make a qualitative difference to the Malays' multilingual repertoire'' because the community's identity is defined being multilingual and adding one more language will not change that.  In contrast, I am of the opinion that such an alignment \textit{will} make a difference to ``the identity they have'' by adopting a language that is not used in speech or in writing by anyone in Sri Lanka, save the proponents of such alignment (who tend to identify with Malaysian Malays). In order to illustrate this point, I draw on the findings of a focus group discussion conducted in 2010. 

\section{Alternative views} %3

In the next section I provide excerpts from a focus group discussion that was conducted as part of my fieldwork amongst five Malay youth from the suburbs of Colombo whose ages ranged from 19 to 23.  They can be considered as representing an urban middle class youth segment of the Malay population in terms of the access they have to economic, social and cultural capital \citep{Bourdieu1977}. While three of the participants are working at junior management level in the corporate sector, two are studying in private university colleges which are affiliated to universities overseas. The discussion was conducted in English as all five participants speak English as their home language. The purpose of highlighting the views of these members of the community is to examine the range of opinions that exists within one supposedly similar demographic. Further, it sheds light on how young people of the community view the link between their language and their identity and I feel that the discussion on the future of the language will benefit from including the opinions of youth members of the community. 

\begin{quote}
 I am very comfortable with Sri Lankan Malay and that ought to be the language we learn first, mainly because we are currently living in Sri Lanka and there is a need to speak Sri Lankan Malay more than any other form of Malay. 
{\itshape(FG/ WP 05)}
\end{quote}


\begin{quote}
You don't want to learn Malaysian or Indonesian Malay -- you want to learn our own Malay -- Sri Lankan Malay. Sri Lankan Malay personally for me is a unique language compared to Malaysian Malay and Indonesian Malay.
{\itshape
(FG/ WP 01)}
\end{quote}

These two comments underscore the perception among the youth that their identity as Sri Lankans is what defines them and is opposed to the emphasis on Pan-Malay identity that is witnessed especially among the older generation. This may be because, unlike the older Malays who lived through the change in language policy and have a sense of being increasingly marginalized politically, the Malays of the 18 -- 25 age group have grown up with an identification of Sri Lanka as home. I believe that the sense of patriotism that has been advocated implicitly and explicitly in the public sphere in the wake of the conflict between the Liberation Tigers of Tamil Eelam (LTTE) and the Sri Lankan Armed Forces could also be one of the reasons for the importance given to a Sri Lankan identity. In this regard, it is useful to recall the statement made by Sri Lankan president Mahinda Rajapakse after the military defeat of the LTTE in May 2009: 


\begin{quote} 
We [have removed] the word minorities from our vocabulary three years ago. No longer are the Tamils, Muslims, Burghers, Malays and any others minorities. There are only two peoples in this country. One is the people that love this country. The other comprises the small groups that have no love for the land of their birth. Those who do not love the country are now a lesser group.''\footnote{This 
 statement by the president has also been interpreted as a denial of the rights of the minority communities of Sri Lanka. Some critics have voiced concerns that this speech may justify the continued domination of the country by the Sinhalese, and be thus ``evidence of a majoritarian mindset'' \citep{Ismail2009}.
}
\citep{Rajapakse2009}
\end{quote}

 Influenced by this and similar political rhetoric superficially promoting `social cohesion' and `inclusiveness', it is possible that the present generation of Sri Lankan youth are (re)negotiating their ethnic and linguistic identities in relation to their national identity and their wish to align themselves with a Sri Lankan form of Malay is a reflection of this evolution.  

Providing a variation of the same theme is this opinion by another member of the focus group:

\begin{quote}
We should be proud of the language we speak which is Sri Lankan Malay and not Malaysian or Indonesian Malay, because our grandparents they've spoken Sri Lankan Malay why not we continue the same language?
 \textit{(FG/WP 02)}
\end{quote}

In contrast to many members of the community who have harked back to Indonesia and Malaysia as `our motherlands' \citep[5]{Thaliph1998}, this participant feels his grandparents, and through them his history, is rooted in Sri Lanka. It is possible that to the present generation, the fact that the Sri Lankan Malays originated from what is present day Indonesia and Malaysia may mean less than the fact that all their associations with the past are rooted in Sri Lanka.

These opinions have to be contrasted with those of older urban Malays who state that they feel embarrassed when they speak SLM. But this embarrassment is only when they speak with Malaysian or Indonesian dignitaries they meet in Colombo or Malaysians or Indonesians they meet during their travels to these two countries. For instance, a member of the urban Malay community who has regular contact with Malaysians and Indonesians stated:

\begin{quote} 
I feel a little ashamed to -- I'm being frank with you -- I feel a little ashamed because I feel the language is not{\dots}, even though it is okay for us, we communicate.
{\itshape  (IN/WP/09)}
\end{quote}

Later in the interview, the same participant states:

\begin{quote}
\dots in the presence of Indonesians and Malaysians or Malays, I feel a little uncomfortable because I think the language is not up to {\dots} maybe it is not grammatical because I have seen people when they hear us speaking, the smiles. You get what I mean? You get this amused look which is not an amusement to us. I'm being very frank.
{\itshape (IN/WP/09)}
\end{quote}

The hesitation to actually verbalise the extent of his embarrassment and the actual reasons for it (as seen by the incomplete sentences) could be due to a perception that it might be `disloyal' to criticize SLM but could also be an unwillingness to engage with me as the interviewer on the perceived `incorrectness' of SLM in relation to Standard Malay. This notion that SLM may not be a `proper' language is commonly held among many segments of the community as it is among minority or marginalized linguistic communities everywhere. Linked to this is the idea that SLM is a `creole' with the layman's understanding of the word `creole' as signifying a `lesser' language.  One opinion which had a significant impact on the way the language is viewed by the SLM community was expressed in 1986: 

\begin{quote}
The present SLM, although perceived by the local Malays as `Bahasa Melayu', is but a heavily creolised language and therefore widely divergent from the standard Malay spoken in the Malay peninsula and the archipelago. \citep[153]{Hussainmiya1987}
\end{quote}

This classification and the accompanying implication of speaking a `corrupt' or `broken' variety of the language has been detrimental to the way the language has been viewed by the community and ``has a significant impact on the type of [language] shift that may occur as well as its speed{\dots}'' \citep[220]{LimEtAl2007}. As illustrated in the case cited above, this perception exists in the mind of the community members as a rationale for moving towards a `better' language.

Returning to the opinions of the participants of the focus group, one is made aware of how multilingualism is perceived by the youth. 

\begin{quote} 
And even if you do learn Standard Malay I don't think it would be a hindrance to Sri Lankan Malay -- it's kind of like knowing how to speak English, Sinhala and Malay. 
{\itshape
(FG/ WP 05)}
\end{quote}

This quote reflects the participant's attitude towards the acquisition of multiple languages. It possibly highlights the reason that has made the Malays be regarded as one of the most multilingual communities in Sri Lanka \citep{LimEtAl2007, Saldin2001,Vijaycharya2004}: their actual multilingualism as well as their positive attitude towards learning many languages.  This participant's opinion also reveals that he views Standard Malay and SLM as two separate languages, which is why he speaks of them alongside English and Sinhala.

\begin{quote}
{We don}{'}{t really have the materials to learn our own language}{ -- }{I mean there are no text books. No kind of dictionary of any sort. The only way we can learn is from parents, grandparents}{ -- }{people who already know the language}{{\dots}} {it would be better if we had some kind of materials for Sri Lankan Malay.} 

{\itshape
(FG/ WP 01)}

\end{quote}

Similar opinions are frequently heard among Malays from all age groups and sub-communities. This is possibly due to the perception that SLM is a `lesser' variety because it does not exist in written form. Regarding this view, it is useful to cite Kloss' (1978) \nocite{KlossEtAl1978} discussion of the factors that determine if a given idiom qualifies as a `language'. He states that a variety can be termed a `language' if it demonstrates considerable `Abstand' and `Ausbau'. Abstand or structural difference is the amount of disparity the variety shows in relation to other `languages' under which it may be subsumed. Kloss defines Ausbau as ``language by development'' and points out that the `language' in question needs to have developed into ``tools for qualified purposes or spheres of application'' \citep[25]{KlossEtAl1978}. Joseph has included ``publication, education and other functions associated with the more prestigious realms of Western culture'' (2004, p. 3) as some of the factors that help create Ausbau. Applying this principle to the Sri Lanka Malay context, it is clear that SLM can be considered a language different from `standard' Malay by the Abstand factor because it has now been established that in terms of structure SLM has diverged considerably from the Malay varieties to which in traces its origins \citep{Adelaar1991}. However, SLM does not demonstrate sufficient Ausbau to qualify as a language in its own right because it lacks the `official' status of having developed tools for education and communication. This is why it is heartening to note that members of the community and linguists are uniting to work towards developing teaching and literary material in SLM. One example of such an effort was the organisation of a seminar entitled ``Symposium on the Mother Language of the Sri Lanka Malays'' in Colombo, Sri Lanka in May 2011.\footnote{More 
 details of this seminar can be viewed at \url{https://sites.google.com/site/symposiumonslmalay/} 
}
At this forum, linguists working on various aspects of the SLM language were invited by the Women's Association of Sri Lanka Malays (WASLAM) and the Confederation of Sri Lanka Malays (COSLAM) to present their opinions on the preservation of the spoken and written varieties of SLM. As a follow-up to this initiative, WASLAM is currently in the process of working with other Malay organisations throughout the island towards formulating means of developing teaching and literary resources in SLM. Further, a short children's story has been translated into SLM and its narration has been uploaded onto YouTube.\footnote{
 \url{http://www.youtube.com/watch?v=VuVBjrIVjk8} 
}
It is hoped that these and similar efforts will be successful in revitalizing a variety of Malay which is viewed as a crucial marker of identity by the members of the community and is of value to academic discussions surrounding extreme language contact. 

\section{Summary}

The Sri Lanka Malay community has managed to retain its language for the last 350 years despite facing many challenges. The fact that SLM has survived these pressures from other larger, more economically strong, more socially acceptable languages points to the crucial role played by language in defining identity for this community. This was confirmed by most of the participants of my study who felt that language was a crucial marker of their ethnic identity.

However, the members of the community seem to be divided as to which identity should be given priority -- the Pan-Malay identity or the Sri Lanka Malay identity, and this difference of opinion drives much of the current debate surrounding SLM. While some members of the urban majority feel that `standard' Malay should be taught, I have pointed out the power dynamics that possibly underpin such a discourse. Using data from a focus group discussion with young Malay participants, I also discussed the views of the younger generation regarding language revitalization and point out that the youth seem to desire a more Sri Lankan identity and hence show signs of wanting to preserve Sri Lanka Malay. 

\nocite{DeumertEtAlEd2006}
