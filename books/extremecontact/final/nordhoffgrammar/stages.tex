\documentclass[a4paper,12pt]{article}
\usepackage[utf8]{inputenc}
\usepackage{ipashortcuts,lingsty}
\usepackage{gb4e,tipa}

\usepackage[authoryear]{natbib}
%opening
\title{The genesis of Sri Lanka Malay as a multi-layered process}
\author{Sebastian Nordhoff}

\begin{document}

\maketitle
  
\section{Introduction}
Sri Lanka Malay differs remarkably from all other varieties of Malay. This fact has been attributed to a variety of causes and processes, some of which are mutually exclusive.
\begin{itemize}
 \item \textbf{creolization}: Uncontrolled acquisition of Malay by Moorish wives, entailing nativization. Time frame: probably before 1800 \citep{Smith2003timing, SmithEtAl2004, SmithEtAl2006cll, Slomanson2006cll}
 \item \textbf{rapid convergence}: SLM rapidly changed its structure within few generations (Bakker 1995 and passim\nocite{Bakker1995}). Time frame: either 1650-1800 OR 1873-1970.
 \item \textbf{metatypy} \citep{Bakker2006}: Time frame: 1656-2010
 \item \textbf{ganging up} \citep{Ansaldo2008genesis,Ansaldo2009}: time frame: 1656-2010, presupposes constant prestige relations
\end{itemize}

These analyses share the common principle that the shape of the language at point $t_0$ was influenced by an event or scenario $E$ which lead to the language having a different shape at $t_1$. $L_{t0}$ is Vehicular Malay, $L_{t1}$ is modern Sri Lanka Malay, and $E$ is any of the processes mentioned just above.


\section{Different stages in scenarios of language change}
The study of displaced population shows that there are normally several interlocking and overlapping events. e.g.

\begin{itemize}
 \item \citet{Trudgill1986} analyzes the genesis of new dialects as follows:

\begin{tabular}{llll}
 I & adult migrants & first generation & rudimentary levelling\\
II & first native-born speakers & second generation & extreme variability and further levelling\\
III& \multicolumn{2}{c}{subsequent generations} & focussing, levelling, and reallocation \citep[200]{TrudgKers2005}
\end{tabular}

 \item \citet{Meshtrie1993} applies Trudgill's scenario to South African Hindi, where he posits the same sequence of events

 \item \citet{Siegel} shows that the same holds true of displaced populations in Oceania (Melanesia, Fiji, Hawai'i)

 \item \citet{Siegel2008} further notes that there is remarkable agreement between substratists and superstratists about the fact that there are at least two stages in Creole formation, adaptation (an L2 process by adults) and nativization (an L1 process by children). He shows that Bickerton, Chaudenson, and Mufwene agree on that point and have all three written passages to that effect. 
\end{itemize}  

The following stages can be distinguished for Sri Lanka Malay:
 
\begin{table}[h]
\begin{tabular}{lllll}
  0 & Pidgin formation 		 	& Malay 	& Indonesia 		& $<$1600 -1800\\
  1 & Dialect Levelling/koineization    & Malay 	& (Batavia,) Sri Lanka 	& 1650 - 1800\\
  2 & Substrate Reinforcement 		& Malay+Lankan	& Sri Lanka 		& 1650 - today\\ 
  3 & ???                     		& Lankan	& Sri Lanka 		& ?? \\
  4 & convergence towards Sinhala 	& Sinhala	& Sri Lanka 		& 1950-today\\ 
\end{tabular}
\end{table}

\begin{table}
\begin{tabular}{p{8cm}p{8cm}} 
\vspace{-6cm}
\fbox{
\parbox{7cm}{ 
    loss of m\E-\\
    loss of di-\\
    loss of -in\\
    h$\to\emptyset$ 
}}$_{0}$

\fbox{
\parbox{7cm}{ 
  N punya/pe N\\
  pada\\
  oomong\\
  kulluth\\
  C\E{}C\#\\
  [+high]$\to$[+mid]/\_C\#\\
  ada $\to$ TAM\\
  thraa\\
  pl pronouns in orang\\
  Chinese pronouns go/lu\\
  dropping of schwa\\
  (s)atu\\
  Serial verbs\\ 
}
}$_{1}$
\fbox{ 
\parbox{7cm}{ 
  prenasalization\\
  dental vs. retroflex\\
  gemination\\
  vowel length\\
  n$\to$\ng/\_\#\\
  vector verbs\\
  existentials duuduk/aada\\
  kànà-\\
  zero nominalization\\
  DEM N\\
  GEN N\\
  QUANT N\\
  relator nouns\\
  terms of address, pronoun avoidance\\
  encliticization of TAM adverbs\\
  SOV\\ 
}
}$_{2}$
&
\fbox{ 
\fbox{ 
\parbox{7cm}{ 
  morphologicization\\ 
  d$\to$\dentd{}/\#\_\\
  w$\to$\V\\ 
  EVID\\ 
  QUOT\\
  CASE\\
  INF\\
  CONJ PTCPL\\
  PRES PTCPL\\ 
  indefinite pronouns \\
  rigid word classes\\
  coordinating clitics\\ 
  negation pattern \\
  SOV\\
  postpostions\\
  RELC N\\
  STD DAT N MORE ADJ\\
  Tamil loans\\
  s$\to$\tsh\\
  tail head linkage\\ 
}
}}$_{3}$

\fbox{ 
\parbox{7cm}{ 
  INDEF\\
  ADJ N\\
  simultaneous arà-\\
  non-nominative subjects 
}
}$_{4}$


\fbox{ 
\parbox{7cm}{ 
  copula\\
  \unj{}\\
  DEIC+X\\
  -da\\
  jang-\\
}
}$_{indp.}$
\end{tabular}
\caption{features discussed and stages assigned}
\end{table}

\begin{itemize}
 \item 37 can be explained  
 \item 18 remain in stage 3   
 \item These 18 items form a test case for different scenarios of genesis
\end{itemize}


\section{Stages}
\subsection{Stage 0: Piding formation}
\begin{itemize}
 \item pidgin formation
 \item loss of voice morphology (\em m\E{}N-, di-, -in\em)
    \begin{itemize}
    \item found in nearly all Pidgin Derived Malay varieties \citep{Adelaar1991,AdelaarEtAl1996,Paauw2004,Paauw2008phd}
    \end{itemize}
 \item h$\to\emptyset$
    \begin{itemize}
    \item also widely attested in PDMs \citep{Adelaar1991,AdelaarEtAl1996,Paauw2004,Paauw2008phd}
    \end{itemize}
\end{itemize}

\subsection{Stage 1: dialect levelling}
 \begin{itemize}
  \item many features of SLM are present in one or the other of the Indonesian source dialects
  \item mainly Moluccas and Jakarta
  \item \citet{Adelaar1991,AdelaarEtAl1996,Paauw2004,Paauw2008phd}
 \end{itemize}

\begin{itemize}
 \item possessive construction P'OR LINKER P'UM
    \begin{itemize}
    \item LINKER is punya or a reduction thereof, e.g. pe \citep{Adelaar1991,Paauw2004,Paauw2008phd}
    \end{itemize} 
 \item plural word \em pada \em from Jakarta \citep{Adelaar1991,Paauw2004} 
 \item \trs{oomong}{to speak} from Javanese via Jakarta \citep{Paauw2004}
 \item \trs{kulluth}{broom} and some other lexemes from Javanese via Jakarta
 \item schwa in final syllables e.g. \phontrs{\dentt a:n@m}{plant} from Jakarta \citep{	Adelaar1985,Nordhoff2009phd}
 \item lowering of high vowels in final syllables \citep{Paauw2004,Paauw2008phd,Nordhoff2009phd} 
 \item the existential \em ada \em is used to mark tense or aspect, from Moluccan \citep{Adelaar1991}  
 \item negator \em thraa \em from the Moluccas \citep{Adelaar1991,Paauw2004}
 \item plural pronouns based on sg+\trs{orang}{man}, common in many PDMs \citep{Adelaar1991,Paauw2004}
 \item Chinese pronouns \trs{go}{1s},\trs{lu}{2s} from Jakarta \citep{Adelaar1991,Paauw2004}
 \item Dropping of schwa in initial syllables from Jakarta (Uri Tadmor, p.c.)
 \begin{itemize}
  \item The retention of schwa in this position was also found in Sri Lanka, cf. Sinhala
 \trs{kerabuvaa}{Ohrring} \citep{Gunasekara1891} vs. SLM \em kraabu$<$kerabu\em.
 \end{itemize}
 \item Schwa$\to$a (Mol.) $\to$ a: (SLM) \citep{Paauw2004}
 \item dropping of /s-/ in \trs{satu}{one} \citep{Paauw2004}
 \item Serial Verbs \citep{Paauw2004,Nordhofffcmvc}
 \end{itemize}

\subsection{Stage 2: substrate reinforcement}
\begin{itemize}
 \item features marginally found in Indonesian varieties receive a boost upon contact with the Lankan languages
 \item phonemicization of the dialectally different syllabification of NC clusters. Some Malay dialects syllabified N.C, while others syllabified .NC (Uri Tadmor, p.c.). Under influence from Sinhala, SLM now has a phonemic contrast between N.C and $^N$C, e.g. \trs{am.bel}{take} vs. \trs{gaa.mbar}{picture} \citep{Tapovanaye1995,ApoussidouEtAl2008,Nordhoff2009phd}
  \item the subphonemic contrast between dental \dentt{} and alveolar d was extended to cover both dental and alveolar voiced and voiceless stops under influence from Sinhala and Tamil \citep{Bichsel,Tapovanaye1995,Nordhoff2009phd}. This had already taken place in the early 19th century since we find manuscripts where two different d's are graphemically distinguished	in the Arabic script employed \citep{Hussainmiya1987}
  \item gemination of consonants after schwa as found as a subphonemic contrast in some Malay varieties was phonemicized in Sri Lanka under influence from Sinhala and Tamil. Indonesian words of Indian origin like \trs{topi}{hat} or \trs{kapal}{ship}, which had lost the original geminate consonant from Hindi/Tamil in Indonesia, reinstated the geminate in Sri Lanka under influence from Tamil (\em kappal, \dentt oppi \em) or Sinhala (\em \dentt oppiya\em). This leads to near-minimal pairs like \trs{kappal}{ship}, \trs{kaapang}{when}; \trs{thoppi}{hat}, \trs{soopi}{liquor}; \trs{dìkkath}{vicinity}, \trs{ikkang}{fish}, \trs{thiikam}{stab}.
  \begin{itemize}
    \item This gemination can be analyzed as the result of a bimoraic foot \citep{ApoussidouEtAl2008,Nordhoff2009phd}/quadrimoraic word \citep{Tapovanaye1995}.
    \item Sinhala also has a bimoraic foot structure \citep{Letterman1993}, but the correlates thereof a completely different from what we find in SLM
    \item Tamil has no bimoraic foot structure. It is even debatable whether Tamil has stress or any other sign of foot structure \citep{Keane2001}
    \item It is unclear how exactly geminated consonants became part of SLM grammar
  \end{itemize}
  \item subphonemic quantity contrasts in PDM penultimate syllables became `more phonemic' in Sri Lanka under influence of Sinhala and Tamil, where vowel quantity is phonemic.
  \begin{itemize}
   \item There is an interplay between consonant gemination and vowel length, and one can be used to predict the other. Vowel length does not have a high functional load. There are some isolated cases where vowel length has to be stored lexically, e.g. \trs{thurus}{straight} vs. \trs{thuurung}{descend}. Another word with a lexically stored long vowel is \trs{kaarthu}{quarter of an hour}
  \end{itemize}
  \item Velarization of final nasals as found in the Moluccas \citep{Adelaar1991,Paauw2004} was reinforced by the same process found in Sinhala
  \item the existential use of \trs{duduk}{sit} to mean `to live, reside' found in Indonesia was reinforced by Sinhala \trs{innavaa}{exist.\textsc{anim}}, which also has a historical meaning of `to sit' \citep{Nordhoff2010ismil}
  \item the involitive construction with \em kena \em was reinforced by the Sinhala involitive construction (3rd conjugation) \citep{Nordhoff2010ismil}
  \item the possibility to use clauses as NPs common in Indonesia was reinforced by the Sinhala zero-adclausal nominalization \citep{Nordhoff2010ismil}
  \item the optional order DEM N found in PDMs was reinforced by DEM N in Tamil and Sinhala \citep{Adelaar1991,Paauw2004,Paauw2008phd}
  \item the optional order QUANT N found in PDMs was reinforced by QUANT N in Tamil  \citep{Adelaar1991,Paauw2004,Paauw2008phd}
  \item linker nouns/relator nouns \citep{Adelaar1991,SmithEtAl2004,Nordhoff2009phd} are found in some Indonesian varieties of Malay\citep{Paauw2004}, as well as in Sinhala and Tamil.
  \item  encliticization of TAM adverbs \citep{Adelaar1991} is found in some Indonesian varieties of Malay and are a precursor of the inflectional expression of TAM in SLM, as in Sinhala and Tamil.
  \item the general liberal attitude to phrase structure in Indonesian varieties of Malay suggests that the occurrence of occasional SOV word order cannot be excluded. This occasional SOV word order would dramatically be reinforced by Sinhala and Tamil.
\end{itemize}    

\subsection{Stage 4: convergence towards Sinhala}
\begin{itemize}
 \item I skip stage 3, to be taken up again below
 \item After independence of Sri Lanka from the UK (1948), educational policies became nationalist and favoured Sinhala
 \item As a consequence, Sinhala exerts influence on the other languages of the island
 \item It is reasonable to expect that SLM today is more Sinhala-like than what it was before
 \item The following features are probably due to recent Sinhalaization, or have at least got a boost in the recent past
 \item Extensive use of the indefinite marker \em hatthu\em
 \begin{itemize}
  \item also doubling of the indefiniteness marker in certain constructions following a Sinhalese model \citep{Nordhoff2010ismil}
 \end{itemize}
  \item dominance of the order ADJ N in younger speakers where older speakers also have N ADJ
  \item use of the non-past marker \em arà \em in past participle context on the model of Sinhala \em -navaa\em-forms
  \item instrumental and accusative subjects \citep{Nordhoff2010ismil}
\end{itemize} 
 
\subsection{Stage 3: the remainder}
\begin{itemize}
 \item After discussing Stages 0, 1, 2, and 4, we have a set of elements which cannot be attributed to any of the mentioned stages. 
 \item These are clear influences from Lankan languages, which can however not be attributed to the recent past
 \begin{itemize}
  \item Tamil influence
  \item `slow' Sinhala influence which could not have completed within 60 years. 
 \end{itemize}
 \item These features are due to one or more processes other than the ones mentioned.
 \item Candidate processes: Creolization, (rapid) convergence, metatypy, independent language change.
\end{itemize}

\subsubsection{Independent developments}
\begin{itemize}
 \item the following changes are probably not due to language contact at all
 \begin{itemize}
  \item development of a copula \citep{Nordhofffccopula}
  \item t$\to$c/\#\_ \citep{Nordhoff2009phd}
 \end{itemize}
 \item the following four changes are only due to indirect language contact \citep{NordhoffEtAl2007alt}
 \begin{itemize}
  \item development of a past tense suffix \em -da\em
  \item development of the negator \em jama/jang\em
  \item development of the phonemic prenasalized voiced palatal stop \unj
  \item use of deictics+enclitics to structure discourse
 \end{itemize}
\end{itemize}

\subsubsection{Language contact phenomena}

\begin{itemize}
 \item `morphologicization': SLM makes more use of bound morphology and clitics than any other variety of Malay
 \item TAM-affixes attach at the left side, rather than on the right side, what would be expected under language contact \citep{Nordhoff2006cll}
 \item dental articulation of initial /d/, rather than alveolar, in some lexemes \citep{Smith2003timing}
 \item w$\to$\V \citep{Bichsel,Smith2003timing}
 \item grammaticalized evidential marking analogous to Sinhala and Tamil \citep{SmithEtAl2006cll,Nordhoff2009phd}
 \item development of a quotative analogous to Sinhala and Tamil \citep{Nordhoff2009phd}
 \item development of case 
 \item development of an infinitive marker \citep{Nordhoff2009phd}
 \item development of a conjunctive participle \citep{Nordhoff2009phd}
 \item development of a simultaneous participle through reduplication \citep{Nordhoff2009phd}
 \item indefinite pronouns based on interrogative pronouns \citep{Nordhoff2009phd}
 \item crystallization of three distinct word classes \citep{Nordhofffcjoat}, as opposed to the absence of clear word class distinction in PDM \citep{Paauw2004}
 \item coordinating clitics (X=CLT Y=CLT) similar to what is found in Sinhala and Tamil
 \item complexification of the negation pattern, distinguishing predication type, tense, clause type \citep{Slomanson2008lingua,Nordhoff2009phd} \citep{Slomanson2008lingua,Nordhoff2009phd}
 \item SOV
 \item Postpositions
 \item RELC N
 \item STD=DAT N more ADJ 
 \item Tamil loans
 \begin{itemize}
  \item \trs{dhaatha}{elder sister} ,\trs{mavol}{daugther},\trs{maven}{son} 
  \item \trs{kattil}{bed}, \trs{kuure}{roof}, \trs{kusini}{kitchen},\trs{marakari}{vegetables}
  \item \trs{karcel}{problem},\trs{konnyong}{a bit}
  \item \trs{nandu}{crab}, \trs{selendi}{spider}
  \item s$\to$c/\#\_ (\em cippi,cinggalaa,cerappu\em, cf. Tamil \phonet{sip:i,siNgalaar,serap:u})\kuckn
 \end{itemize}
 \item Tail-head linkage \citep{Nordhoff2009phd}
\end{itemize}                  
 
\section{Intermediate summary}
\begin{itemize}
 \item There are many structures which have to be attributed to stages 0 (pidginization) and 1 (dialect levelling). Others most likely arose in stages 2 (substrate reinforcement) or 4 (convergence towards Sinhala), although they might as well have taken place in stage 3. 
 \item This already shows that the genesis of Sri Lanka Malay was not a one-event process
\end{itemize}

\section{What to do with stage 3?}  
\begin{itemize} 
 \item In the chain of events which lead to the formation of modern Sri Lanka Malay, can we say what kind of process is responsible for the changes we observe in  stage 3?
 \item are we dealing with language shift from Tamil to SLM and associated phenomena (Smith, this conference)?
 \item or are we dealing with language maintenance and convergence towards Sinhala/Tamil along the lines proposed by Ansaldo (paper distributed at conference)?
 \item the call is difficult to make, but a comparison with a clear case of language shift and a clear case of language maintenance can be instructive
 \item what happens if Lankan populations shifts to an SVO language with little morphology? 
  \begin{itemize}
  \item Take a look at Sri Lankan English and see whether that corresponds to what we find in SLM
  \end{itemize}
 \item what happens if marginal Malay population retains their language in an environment dominated by other linguistic structures?  
  \begin{itemize}
  \item Take a look at Nonthaburi Malay (Thailand) \citep{Tadmor1995phd} and see whether what we find is comparable to what we find in SLM
  \end{itemize}
  \item restrict comparison to stage 3 items, since other explanations can be adduced for elements in the other stages.
\end{itemize}
 



\section{Conclusion}
\begin{itemize}
 \item Differences between Traditional Malay and SLM have to be attributed to a variety of sources and processes
 \item One-size-fits-all approach will not work
 \item plurality of processes, similar to what has  been argued for Surinamese Creole by \citet{Migge2008}
 \item future discussion of the genesis of SLM should be explicit as to
   \begin{itemize}
    \item the feature  under discussion
    \item the time frame relevant for the development of this feature
    \item the processes at work for that particular feature
   \end{itemize}

\end{itemize}
\end{document}
