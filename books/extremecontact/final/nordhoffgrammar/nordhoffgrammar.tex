 
\chapter{Synchronic grammar of Sri Lanka Malay}

\chapterauthor{Sebastian Nordhoff}{Max Planck Institute for Evolutionary Anthropology}

 
\section{Introduction} 
This paper presents a synchronic sketch of current Sri Lanka Malay as spoken in Kandy and surroundings. It serves as a basic introduction to the language, and as a backdrop for the diachronic chapters, which discuss the development of this language in more detail. In order not to preempt the historical papers, diachronic observations are limited to a minimum in this introduction. The breadth and depth of this sketch are limited by the available space. For more elaboration of the grammatical domains mentioned, the reader is referred to \citet{Nordhoff2009}.


\section{Phonology}

\subsection{Consonants}
Sri Lanka Malay has 24 native consonants and 3 consonants only used in loanwords (see Table \ref{tab:SLMConsonantPhonemes}). The most notable sounds are the sounds traditionally referred to as `retroflexes'. These are in opposition to the dentals, which is unique in the Malay world. The term `retroflex' is chosen here in analogy to a similar phonological distinction found in Tamil and Sinhala, even if the SLM articulation is more to the front than what we find in the other languages. Example \xref{ex:minimalpairsstops} gives (near-)minimal pairs.

\ea\label{ex:minimalpairsstops}
\gll baa\dentt ok baa\tz ok \dentd aa\dentt a\ng{} \dz aapur \\
     cough coconut.shell come oven \\ 
\z

The voiceless dental variant is far more common than the voiced dental variant, while for the retroflex consonants this is reversed, with the voiced stop much more common than the voiceless stop.
Table \ref{tab:dentalretroflex} gives a breakdown of the occurrences of these phonemes in a subset of my corpus. This skewed distribution is due to the asymmetric realization of /t/ and /d/ in the Malay varieties which gave rise to SLM: /t/ is normally realized as dental, while /d/ is realized as alveolar. This was served as a default articulation in SLM.



\begin{table}
    \centering
        \begin{tabular}{lcccccc}
                 & labial & dental & retroflex    & palatal          & velar & glottal\\
        \hline
        stops    &&&&&&\\
        ~~~voiceless& p   & \dentt{} & t        &    c        & k   &   \\
        ~~~voiced   & b   & \dentd{} & d        &   \J          & g   &   \\
	~~~prenasalized&\mb& 	     & \nd         &  \nJ        & \nG &   \\
	nasals      & m   &         &  n           & \ny	  & \ng &   \\
       fricatives  & (f) &          &    s (z) (\textesh)    &   &     & h \\
	 approximants & \V  &          &       &   j         &     &   \\
	 liquids &   &          &  r ~~ l      &            &     &   \\
        \end{tabular}
    \caption[SLM consonant phonemes]{SLM consonant phonemes, including approximants. Palatal stops are phonetically affricated. Parentheses indicate phonemes only found in loanwords.}
    \label{tab:SLMConsonantPhonemes}
\end{table}



The second remarkable feature is the presence of a phonemic series of prenasalized stops. These count as one segment in phonotactics. Near minimal pairs, with indication of syllabification, are given below.

\ea
\gll baa.pa baa.be\dentt{} braa.\umb a\dentt \\
     father tripes spread \\
\z

Note that the long vowel in \phonem{braa{\umb}a{\dentt}} is synchronically not an instance of phonetic compensatory lengthening  \citep{WetzelsEtAl1986}, since there are words like \trs{ambel}{take}, which do not show a long vowel in front of the sequence NC. There is independent evidence for vowel lengthening in open penultimate syllables, so that an account based on different syllabifications of \phonet{am.bel} and \phonet{bra:.\umb a\dentt{}} can provide a satisfactory explanation.


\begin{table}
\centering
\begin{tabular}{lrrrr|rr}
		& \multicolumn{2}{c}{dental}&\multicolumn{2}{c}{apical}\\
voiceless 	& 186& (64\%)		& 18 &(6\%)		& 182 & (70\%)\\
voiced 		& 11 &(4\%)		& 77& (26\%)		& 77 & (30\%)\\
\hline
		& 197 &(68\%)		& 95 &(32\%)		& 292 & (100\%)
\end{tabular}
\caption{Frequencies of dental and retroflex stops.}
\label{tab:dentalretroflex}
\end{table}
\subsection{Vowels}
SLM has a six-vowel system (Table \ref{tab:SLMVowelPhonemes}).
The five full vowels /a,e,i,o,u/ are easily established and present no problem for the analysis. It can be noted, however, that /e/ and /o/ are remarkably less frequent than /i/ and /u/ which are in turn less frequent than /a/, in a ratio of about 1:1:3:3:9.

% 1952 258 676 184 622
% 3692
% 52.8 7.0 18.3 5.0 16.8
    \begin{table}[!h]
    \centering
        \begin{tabular}{rccccc}
        %            & front &   & central &   & back \\
                &  i    &  	 &        	 &  		 &   u   \\
                &       &  e &  \textipa{@}       &  o &      \\
                &       &   &    a   &   &      \\
        \end{tabular}
    \caption{SLM vowel phonemes.}
    \label{tab:SLMVowelPhonemes}
  \end{table}

The status of schwa presents more problems for the analysis. The first descriptions \citep{Hussainmiya1987,Bichsel1989} note that historical schwa is realized as [i] or [u] and question its phonemic status. There is also a good deal of variation with regard to the realization as [i] or [u]. The analysis of \citet{Tapovanaye1986,Tapovanaye1995}, elaborated by \citet{Nordhoff2009}, is able to shed light on this: there is a phonemic schwa, which is realized as \phonet{@} in the final syllable as well as in the antepenultimate. In the penultimate syllable, schwa is raised to \phonet{I} or \phonet{U}  to make up for a lack in moraic weight (see below). The raised allophone of schwa depends on a variety of circumstances, such as phonological environment, but also idiolect. Presence of labials or [u] favours \phonet{U}, presence of [i] or palatals/velars favours \phonet{I}. Cases in point are \phontrs{\dz{}UppaN}{before} and \phontrs{\dz{}Ikka\dentt}{vicinity}. Certain speakers raise \phonem{@} even further and arrive at realization like \phonet{i,u}, where  the sound is indistinguishable from realizations of \phonem{i,u}, explaining the initial confusion as to the status of \phonem{@}. Independent of phonetic realization, the moraic structure of the word nearly always allows us to ascertain whether we are dealing with an underlying schwa (which triggers gemination of the following consonant in most contexts) or with an underlying /i/ or /u/. 


\subsection{Syllable structure}
The SLM syllable structure is

\ea (s)(C)(L)V(X) \z

where \em s \em is extrasyllabic; 
C is any consonant, except for {\ng} or {\tz} in the onset; 
L is a liquid or a glide;
and X is either the second part of a long vowel or a coda  consonant. 
The coda consonant on the penultimate is typically either the first part of a geminate or a nasal homorganic to a following stop. Exceptions to this rule are \textipa{k} in \phontrs{sEksa}{problem} and \textipa{r}  in  \phontrs{{\dentt}IrbaN}{fly}.

\subsection{Structure of the phonological word}

The phonological word typically consists of two syllables. There are only a handful of monosyllabic content words like \trs{\dentt ee}{tea}, \trs{pon}{bride}, \trs{pii}{go}. Trisyllabic words are very often derived from a disyllabic stem, most commonly be the nominalizer \em -an\em. A case in point is \trs{makanan}{food} from \trs{maakang}{eat}. Tetrasyllabic words are normally compounds of two disyllabic stems, such as \trs{anak klaaki}{child'+`male'=`boy}.

The penultimate of disyllables is lengthened when open. This can be explained by extrametricality of the final syllable and a bimoraic foot requirement, as shown in \xref{ex:wordstructure}, which presents the prototypical structure of the SLM word.

\ea \label{ex:wordstructure} $(\mu\mu)<\sigma>$ \z

I will now discuss how the schema in \xref{ex:wordstructure} applies to SLM lexemes and how it explains certain phonological phenomena, most notably vowel lengthening, vowel raising, and gemination.

The simplest case is a word which already has sufficient material to build a bimoraic foot without the final syllable in the underlying form. A case in point would be \phonemtrs{ban\dentt{}u}{help}, which can be parsed without problems, as shown in \xref{ex:phon:wordstructure:banthu}.

\ea \label{ex:phon:wordstructure:banthu} $(ba_{\mu}n_{\mu})<\dentt{}u>$ \z

\phonem{@} is not able to project a mora in SLM, and would lead to a defective foot. In such cases, underlying \phonem{@} is realized as \phonet{I} or \phonet{U} when a mora is required. This is shown for the lexeme \phonemtrs{m@n\dentt{}a}{raw}.

\ea *$m$\E$n_{\mu}<\dentt{}a>$ \z
\ea \label{ex:raisedschwa} $(m$\I$_{\mu}n_{\mu})<\dentt{}a>$ \z

If there is no coda in the lexeme, the vowel is lengthened if it is  
\phonem{a},
\phonem{e},
\phonem{i},
\phonem{o},
\phonem{u}. 
The extra length of the vowel provides the mora required for the bimoraic foot. This is exemplified with the word \phonemtrs{maka\ng{}}{eat}

\ea
\ea *$ma_{\mu}<ka${\ng}$>$ 
\ex $(ma_{\mu}:_{\mu})<ka$\ng{}$>$ 
\z
\z

In case the underlying vowel of the penult is \phonem{@}, two things happen: first, the vowel is raised as in \xref{ex:raisedschwa}. Second, the onset of the final syllable is geminated, thereby yielding a coda for the penultimate. This is shown for the lexeme \phonemtrs{p@rang{}}{war}. Note that b. and c. contain operations which add one mora, but only when they are combined as in d. is there sufficient material to build a foot.

\ea\ea *$p$\E$<ra$\ng{}$>$ 
\ex *$p$\I_{\mu}$<ra$\ng{}$>$ 
\ex *$p$\E$r_{\mu}<ra$\ng{}$>$ 
\ex $(p$\I$_{\mu}r_{\mu})<ra$\ng{}$>$ 
\z\z
 
As for what concerns trisyllables, this typically contain more segmental material to begin with, so that mora-generating processes are less often required. A typical example is \phonemtrs{makanan}{food}, which parses as follows:

\ea $(ma_{\mu}ka_{\mu})<nan>$ \z

As compared to \trs{maakang}{eat} discussed above, there is no need for vowel lengthening since there are already two moras available. Trisyllables only undergo vowel-lengthening if the first syllable contains a schwa.\footnote{Schwa
 is not found in the second syllables of trisyllables.
} 
A case in point is \phonemtrs{k@pala}{head}. In order to create an additional mora for the creation of the bimoraic foot, the vowel-lengthening strategy is chosen. 

\ea\ea *$k$\E$pa_{\mu}<la>$
\ex $k$\E$(pa_{\mu}a_{\mu})<la>$\z\z

Raising \phonem{@} to \phonet{I} or \phonet{U} would be another theoretical possibility, but this is not employed in the antepenultimate syllable. A theoretical explanation for this is found in \citet{ApoussidouEtAl2009}.

Phonological parsing becomes more unstable when dealing with compound words, which are the only ones to have four or more syllables. In phonologically integrated compounds, only one long vowel can be found. This is illustrated for \phonemtrs{kaca ma\dentt{}a}{spectacles}

\ea$(ka_{\mu}ca_{\mu})(ma_{\mu}a_{\mu})<$\dentt{}$a>$\z

Not all compounds are phonologically integrated however, and parsings of the type $(\mu\mu)<\sigma>(\mu\mu)<\sigma>$ can also be found.


\subsection{Orthography}
The orthography of SLM used in this article is based on the Indonesian and Malaysian standard with the following additions and modifications:
\begin{itemize}
 \item `Retroflex' stops are represented by plain consonants \graphem{d,t}.
 \item Dental stops are represented by adding an \graphem{h}, yielding \graphem{dh,th}.
 \item Unraised schwa is written \graphem{à}; raised schwa is written \graphem{ì} or \graphem{ù}, depending on the feature [back].
 \item Vowel length is represented by doubling the vowels. 
 \item Consonant length is represented by doubling the consonants.
 \item For the digraphs \graphem{dh, th, ny, ng} only the first part is doubled when geminated, yielding \graphem{ddh,\footnote{\graphem{ddh} 
    is a theoretical possibility, although no word with this sequence has been found as of yet.}
 tth,nny,nng}.
 \item Prenasalized consonants may optionally be indicated with a breve over the nasal part if this is needed for the topic under discussion \graphem{\umb, \und, \unJ, \ung}. In most cases, this is not done.
\end{itemize}

The differences between RUMI and SLM conventions can be exemplified with the following cognates:

\begin{enumerate}
 \item Indonesian \trs{tembak}{shoot} is rendered in SLM as \graphem{thee\umb ak}, with an \graphem{h} for the dental stop, a long vowel, and a prenasalized stop.
 \item Indonesian \trs{banyak}{many} is rendered in SLM as \graphem{bannyak} with two \graphem{n} indicating the geminate status of \phonem{\ny}. The sequence \graphem{nyny} is dispreferred as it could lead to the pronunciation as \phonet{nini} by speakers familiar with English orthography.
\end{enumerate}


\subsection{Intonation}
SLM has three main intonation contours: 
declarative \xref{ex:phon:int:declarative}, 
progredient \xref{ex:phon:int:progredient}, and
interrogative \xref{ex:phon:int:interrogative}.
SLM declarative sentences show a fall in intonation towards the end, starting at the beginning of the last lexical word. Interrogatives show a rise in pitch towards the end. The most salient feature of SLM intonation is the steep LH\% pattern at the end of subordinate clauses (progredient). The following three examples illustrate these patterns.

The assertive contour is characterized by a fall from H to a low  tone target. The exact position of H needs more research, but appears to be towards the beginning of the last lexical word of the assertion.


\ea \label{ex:phon:int:declarative}
 \includegraphics[width=0.5\textwidth]{\imgpath kithamperuuma}
\glll    ~   ~         H~~~~L\\
	 ini kitham-pe ruuma \\
     \textsc{prox} \textsc{1pl=poss} house  \\
    `This is our house.'
\z


The progredient contour has a low tone on the beginning of the last syllable and a high boundary tone, leading to a steep rise.


\ea \label{ex:phon:int:progredient}
 \includegraphics[width=0.5\textwidth]{\imgpath ciinggal}
\glll 	~ ~	~	~~L	H	~~~~~~~L	H	~	  ~~~~~~L 	H	~	H 		L\\
       pìirang=na an-daathang ooram pada $~\mid~$ s-pìrrang $~\mid~$  deran=na as-banthu $~\mid~$  siini=jo su-cii\u n\u ggal $~\mid$ \\
      war=\textsc{dat} \textsc{past}-come man \textsc{pl} ~ \textsc{cp}-wage.war ~ 3=\textsc{dat} \textsc{cp}-help ~ here=\textsc{emph} \textsc{past}-settle  \\
    `The men who had come, after having waged war and helped him, settled down.' 
\z


The interrogative contour resembles the progredient contour in having a steep rise, but the preceding dip is either less pronounced or non-existent.


\ea \label{ex:phon:int:interrogative}
 \includegraphics[width=0.5\textwidth]{\imgpath gheenaaapeyangarabiilangnew}
\glll       ~L  		~  ~     ~        H \\
	{\em ghee} =na aapeyang biilang $\mid$ \\
    { } =\textsc{dat} what say \\
    `How do you say for ``Ghee''?'
\z


\section{Word classes}

\subsection{Main word classes}
In contradistinction to most other varieties of Malay, SLM has solid word class distinctions, most prominently between nouns, verbs, and adjectives \citep{Nordhofffcjoat}. A complication in this is that adjectives can convert to nouns or verbs and thus conceal their true nature to the superficial observer. \trs{Bìssar}{big} can be used as a noun to mean `the big one' or as a verb `to become big'. It is still possible to tell the adjectival nature of \em bìssar\em, though: neither nouns nor verbs can be used in all the three meanings given above. It is thus the flexibility of the adjectives which singles them out. See \citet{Nordhofffcjoat} for a fuller discussion.

Nouns and verbs can be identified by their (im)possibility to combine with verbal morphology as listed below in Section \ref{sec:verbalmorphology}.

Two subclasses are of special interest: Within the verbs, we can single out two existentials: \em aada\em, which can combine with any referent, and \em duuduk\em, which can only be combined with animate referents. 

\ea 
    \ea
	\gll ruuma aada/*arà-duuduk\\
	house exist/\textsc{nonpast}-exist.\textsc{anim}\\
	`There are houses.'
    \ex
	\gll moonyeth aada/arà-duuduk\\
	monkey exist/\textsc{nonpast}-exist.\textsc{anim}\\
	`There are monkeys.'

    \z
\z

Among the nouns, a special subclass is the so-called `relator nouns' which indicate a relation between two referents, most often spatial, although other relations are also found. An example is given in \xref{ex:relatornouns}.

\ea\label{ex:relatornouns}
\gll Ruuma=pe duppang pohong aada  \\
    house=\textsc{poss} front tree exist   \\
    `There are trees in front of the house.' (K081103eli02)
\z

In the example above, the relator noun is linked to the ground by the possessive enclitic \em =pe\em. This is the most common case, but absence of any linker, as well as the use of the dative marker \em =nang \em instead of \em =pe \em are also found. The dative marker is especially used for temporal relations. For a relator noun like \trs{blaakang}{behind/after}, the spatial meaning emerges when combined with \em =pe \em whereas the temporal meaning is obtained by using \em =nang\em.

\ea \label{ex:form:reln:pe:space}
\gll Kandi=ka {\em Malay} {\em mosque}$_{space}$\textbf{=pe} \textbf{blaakang}=ka incayang=pe zihaarath aada. \\
     Kandi=\textsc{loc} Malay mosque=\textsc{poss} back=\textsc{loc} 3\textsc{sg}.\textsc{polite}=\textsc{poss} shrine exist. \\
    `In Kandy behind the Malay Mosque, there is his shrine.'
\z

\ea\label{ex:form:reln:blaakang:nang:time}
\gll Dr Draaman duuva thaaun$_{time}$=\textbf{nang} \textbf{blaakang} incayang su-mnii\u n\u ggal. \\
 Dr Draaman two year=\textsc{dat} after 3\textsc{sg}.\textsc{polite} \textsc{past}-die\\
`After two years, Dr Draman died.' (K051213nar08)
\z


% \begin{table}
% \begin{center}
% \centering
% \begin{tabular}{llll}
% 				& N 	& V  & ADJ \\
% \hline
% verbal prefixes 		& -- 	& +  & + \\
% causativizer \em -king\em	& (--) 	& +  & + \\
% nominalizer \em -an \em		& (--) 	& +  & + \\
% deictics 			& + 	& -- & + \\
% plural marker \em pada \em	& (+) 	& -- & + \\
% superlative marker \em anà-\em 	& -- 	& -- & +
% \end{tabular}
% \caption{Diagnostics for word class membership in SLM.}
% \end{center}
% \end{table}

\subsection{Pronouns}
\subsection{Personal pronouns}
SLM distinguishes two numbers, three persons, and two levels of politeness in pronouns (Table \ref{tab:personalpronouns}). First and second person singular impolite pronouns are originally Chinese loans and have pejorative connotations in other Malay varieties as well. The plural pronouns all share the formative \em -ang\em, which grammaticalized from \trs{oorang}{man, people}. The plural pronouns can additionally be marked by the plural word \em pada \em but this is optional.

SLM pronouns can combine with enclitic case markers in a quite regular way. There are two slight twists: the monosyllabic pronouns \em see, goo, luu, \em and \em dee \em shorten the vowel when combining with a case marker. For the same pronouns, the dative allomorph is \em =dang \em instead of \em =nang\em, and the possessive allomorph is \em =ppe \em instead of \em =pe\em.
An overview of SLM pronouns are given in Table \ref{tab:personalpronouns}. 


\begin{sidewaystable}
    \centering
	\begin{center}
	\begin{tabular}{llllp{2cm}p{3cm}}
	pronoun &  =yang [\textsc{acc}] & =nang [\textsc{dat}] &=pe [\textsc{poss}]	& meaning & origin\\
\hline
	goo 	& goyang   	& godang & goppe 	& 1s.\textsc{familiar} (Southern dialect)& Hokkien \citep{Adelaar1991,Ansaldo2005ms} \\
	see 	& seeyang 	& sedang & seppe	& 1s.\textsc{polite} \\
	luu 	& lu(u)yang	& ludang & luppe	& \textsc{2s.familiar} & Hokkien \citep{Adelaar1991,Ansaldo2005ms} \\
	dee 	& deeyang 	& dedang & deppe 	& \textsc{3s.inanimate}/ \textsc{3s.familiar} & \\%(eeya)
	diya	& diyayang	& diyanang & diyape	& \textsc{3s} \\%= dee (meeya)
	siaanu  & siaanuyang   	& siaanunang & siaanupe	& 3s.\textsc{proximal} \\
	incayang  & incayangyang & incayangnang	& incayangpe & 3\textsc{sg}.\textsc{polite} & \em encik ia \em + addition of velar nasal \citep{Slomanson2008ismil} \\
	spaaman & spaamanyang  	& spaamannang 	& spaamanpe & 3\textsc{sg}.\textsc{polite} & uncle?? (Std, Malay, Javanese \em paman \em \citep[141]{Adelaar1985})\\
	kithang & kithangyang  	& kithannang 	& kithampe &1\textsc{pl} & \lt *kita orang (`1\textsc{pl} man')\\
	lorang{\footnotemark}  	& lorangyang 	& lorangnang &lorampe	& 2\textsc{pl}/\textsc{2s.polite} & \lt *lu orang \\
	derang  &  derangyang	& derangnang 	& derampe & 3\textsc{pl} &\lt *de orang, neutral\\
	incayang pada & incayang padayang 	& incayang padanang 	& incayang padape & \textsc{3pl.polite} & incayang + \trs{pada}{\textsc{pl}} \\
	\end{tabular}
	\end{center}
	\caption[Pronouns]{Pronouns.  The etymology of the plural pronouns includes a contraction of \trs{orang}{man} to \em rang \em \citep{AdelaarEtAl1996,Adelaar2005struct,Adelaar1991}.%0,212,32
 	}
	\label{tab:personalpronouns}
\end{sidewaystable}

\footnotetext{\citet[32]{Adelaar1991} has \em lurang\em.}

\subsubsection{Interrogative pronouns}\label{sec:interrogativepronouns}
The basic SLM pronouns are given in Table \ref{tab:interrogativepronouns}. They can be combined with postpositions to yield a wider array of meanings, e.g. \trs{saapa+pe}{who+\textsc{poss}}=`whose'.

\begin{table}[p]
\begin{center}
 \begin{tabular}{llllll}
  aapa    & `what' & mana   & `where', `which' &      craapa  & `how'  \\
  saapa   & `who'  & kaapang & `when'         &      dhraapa & `how many' \\ 
          &      &         &              &      kànaapa & `why' \\
 \end{tabular}
\end{center}
\caption{SLM interrogative pronouns.}
\label{tab:interrogativepronouns}
\end{table}

\subsection{Indefinite pronouns}
Indefinite pronouns are formed by combining a suitable interrogative pronoun with one of the following enclitics

\begin{itemize}
 \item additive \em =le\em: every- % (everyone, everywhere, always)
 \item similitive \em =ke\em: any- %(anyone, anywhere, at anytime)
 \item indeterminate \em =so\em: some- %(someone, somewhere, sometime)
 \item negative \em =pon\em: negation% (no one, nowhere, never)
\end{itemize}

Examples include 
\trs{saapale}{everyone}
\trs{manake}{anywhere}
\trs{dhraapso}{however many}
\trs{kaapangpon}{never}.


\subsection{Deictics}
SLM distinguishes two distances in its deictic system, proximal and distal. The base forms \em in(n)i \em and \em it(t)hu \em are transparently related to a number of adverbs of location, source, manner and amount, as shown in Table \ref{tab:deictics}.

\begin{table}[p]
\begin{center}
% use packages: array
\begin{tabular}{llllll}
base form & place  	& source & manner& amount& gloss \\ %location&
\hline
ini 	& siini 	& sindari& giini & sgiini & proximal \\ %siinika &
ithu 	& siithu 	& sithari& gitthu& sgiithu& distal \\ %siithuka&
\end{tabular}
\end{center}
\caption{Deictics.}
\label{tab:deictics}
\end{table}

\subsection{Numerals}
The numerals from one to twenty are given in Table \ref{tab:numerals}. Productive numeral affixes are \trs{blas}{-teen}, \trs{-pulu}{-ty} and \trs{kà-}{\textsc{ord}}.

\begin{table}[p]
\begin{center}
\begin{tabular}{llcllcll} 
satthu,hatthu 	& 1 	&& subblas	& 11 	&& duvapulsatthu	& 21\\
duuva         	& 2 	&& dooblas 	& 12 	&& duvalpulduuva 	& 22	\\
thiiga        	& 3 && thigablas 	& 13 	&& thigapulu 	& 30	\\
(ù)mpath 	& 4 	&& ùmpathblas 	& 14 	&& sraathus 	& 100	\\
liima 		& 5 	&& limablas 	& 15 	&& duuva raathus & 200	\\
(ì)nnam		& 6 	&& ìnnamblas 	& 16 	&& thiiga raathus & 300	\\
thuuju 		& 7 && thujublas 	& 17 	&& sriibu 	& 1,000	\\
dhlaapan 	& 8	&& dhlapamblas 	& 18 	&& duuva riibu 	& 2,000	\\
s(m)biilan	& 9 	&& s(m)bilamblas& 19 	&& thiiga riibu 	& 3,000	\\
spuulu 		& 10 	&& duvapulu 	& 20 	&& lakh \phonet{l\ae k} & 100,000
\end{tabular}
\caption{SLM numerals.}
\label{tab:numerals}
\end{center}
\end{table}

\subsection{Modals}
A very small but important word class are the modals. There are five of them:
\begin{enumerate}
 \item \trs{boole}{can}
 \item \trs{thàrboole}{cannot}
 \item \trs{(kà)maau(van)}{want}
 \item \trs{thàr(kà)mauvan}{\textsc{neg}.want}
 \item \trs{thussa}{\textsc{neg}.want}
\end{enumerate}

The modals can be used on their own to form a full utterance (e.g. \trs{Boole.}{It is possible.}). The complement of the modal takes the infinitive (e.g. \trs{Ini mà-miinung boole.}{It is possible to drink this.}). If the participant to whom the modal applies shall be indicated, it takes the dative marker \em =nang \em (e.g. \trs{Lorang=nang ini mà-miinung boole}{You can drink this}).



\section{Morphology}
\subsection{Nominal morphology}\label{sec:nominalmorphology}
The SLM noun takes very few affixes. The causativizer \em -king \em can marginally be combined with it, e.g. \trs{kafan}{shroud}$\to$\trs{kafanking}{enshroud}. Furthermore, the suffix \em -an\em, which is normally a nominalizer, can also be combined with nouns in certain cases, which are lexicalized. Examples are \trs{rajahan}{government} from \trs{raaja}{king} or \trs{thumpathan}{seat in parliament} from \trs{thumpath}{place}. Last but not least, kin terms can take the suffix \em -yang\em, which means that the kin relation does not involve either speaker or addressee. An example would be \trs{bapayang}{X's father}.


Most nominal categories are expressed by enclitics rather than suffixes. The clitic status is clear from the fact that material can intervene between a noun and these markers, most typically an adjective. Concepts covered by enclitics are for instance the plural (\em pada\em), the indefiniteness marker \em hatthu\em, which can attach on either side, and the enclitic postpositions which indicate semantic roles and can be seen as case markers. Table  \ref{tab:casemarkers} gives an overview of those.

\begin{table}[p]
    \centering
 \begin{tabular}{lllll}
form & meaning && form & meaning\\
  \hline
\zero{}	   & nominative          && =kapang   & `when' \\
=yang	   & accusative          && =sangke   & `until' \\ 
=nang	   & dative              && =thingka  & `during, while' \\
=ka	   & locative            && =apa      & `after' \\ 
=dering	   & ablative 		 && =sikin    & `because (of)' \\
=pe	   & possessive	         && =lanthran & `because (of)' \\
=(sà)saama & comitative          && =subbath  & `because (of)' \\
 \end{tabular}
\caption{Case marking postpositions.}
\label{tab:casemarkers}
\end{table}

\subsubsection{The indefiniteness marker \em hatthu\em}
The  indefiniteness marker \em hatthu \em can be realized as \em hat(t)hu \em or \em at(t)hu, \em and can be either proclitic or enclitic. Occasionally, speakers use both the pro- and the enclitic, which yields double morphosyntactic encoding of semantic content. Care must be taken to distinguish the free numeral \trs{sat(t)hu}{one}  from the clitic indefinite marker, especially because the numeral has two allomorphs  \em hatthu \em and \em atthu\em, which are homophonous to the two forms of the clitic. However, the form with initial \em s- \em is not possible for the clitic, so that numerals can often be identified in that manner.

Example  \xref{ex:hatthu:atthuhatthu} shows the different realizations of the clitic. In this section, the cliticized nature of \em hatthu \em will be indicated by an equal sign (=). This is not done in other sections for reasons of legibility.


\ea \label{ex:hatthu:atthuhatthu}
\gll Se \textbf{atthu}=aade,  se \textbf{hatthu}=aade. \\
     1\textsc{sg} \textsc{indef}=younger.sibling 1\textsc{sg} \textsc{indef}=younger.sibling  \\
    `I am a younger sibling, I am a younger sibling.'  (K061120nar01)
\z


The following example shows the use of the numeral \em satthu, \em identifiable by the \em s-\em. This \em s- \em could not occur in the example above, but the two forms used in \xref{ex:hatthu:atthuhatthu} could  also occur in \xref{ex:hatthu:satthu}.


\ea \label{ex:hatthu:satthu}
\gll Mlaayu=dring   \textbf{satthu} oorang=jo    se. \\
      Malay=\textsc{abl} one man=\textsc{emph} 1\textsc{sg} \\
    `I am one of those Malays.'  (K060108nar02)
\z


The following examples show the proclitic use \xref{ex:hatthu:proclitic}, the enclitic use \xref{ex:hatthu:enclitic}, and the doubled use \xref{ex:hatthu:preenclitic1} when referring to the concept of story.


\ea \label{ex:hatthu:proclitic}
\gll Hathu=oorang=pe muuluth=dering \textbf{hathu}=criitha kal-dhaathang. \\
     \textsc{indef}=man=\textsc{poss} mouth=\textsc{abl} \textsc{indef}=story when-come  \\
    `When a story comes out of a man's mouth.'  (B060115prs15)
\z




\ea \label{ex:hatthu:enclitic}
\gll Giini criitha=\textbf{hatthu}=le aada. \\
     like.this story=\textsc{indef}=\textsc{addit} exist  \\
    `There is also a story like that.'  (K051206nar07)
\z




\ea \label{ex:hatthu:preenclitic1}
\gll Itthu \textbf{atthu}={\em story}=\textbf{atthu}. \\
       \textsc{dist} \textsc{indef}=story=\textsc{indef}\\
    `This is a story.'  (B060115nar05)
\z



Next to its function of marking indefiniteness, \em hatthu \em is also used to integrate English loanwords, a calque of the Sinhala form \trs{eka}{one} \citep{NitzEtAl2010}. The following example shows a loanword integrated by \em hatthu\em. Crucially, we are dealing with a \em definite \em context here, showing that \em hatthu \em is not used in its primary function here.

\ea  
\gll \textbf{Inni}     {\em mock}       {\em wedding}=\textbf{hatthu}  mas-gijja. \\
      \textsc{prox} mock wedding=\textsc{indef}  must-make \\
    `I have to do this mock wedding.'  (K060116nar10)
\z

\subsubsection{The plural marker \em pada\em}
\em Pada \em is a cliticized plural word \citep{Dryer1989plural}. \citet[14]{Ansaldo2005ms} gives \em pada \em as a suffix, but the fact that it can combine with hosts from a variety of word classes, like  nouns, adjectives, quantifiers, numerals and pronouns and even headless relative clauses suggests that it is a clitic. The clitic nature of \em pada \em is furthermore underscored by \xref{ex:pada:metaling}, where we find metalinguistic material intervening between \em pada \em and its host.
 
\ea \label{ex:pada:metaling}
\gll Derang samma jaau \textbf{uudik} -- {\em village} {\em area} --  \textbf{pada}=nang   su-pii. \\
      3\textsc{pl} all far village {} {} {} {} \textsc{pl}=\textsc{dat} \textsc{past}-go \\
    `They all went to remote villages.' (K051222nar04)
\z 

The most common use of \em pada \em is on nouns, as in \xref{ex:pada:nouns}

\ea \label{ex:pada:nouns}
\gll Suda skaarang    kitham=pe  aanak \textbf{pada}    laayeng pukurjan \textbf{pada}    arà-girja. \\
     thus now \textsc{1pl}=\textsc{poss} child \textsc{pl} other work \textsc{pl} \textsc{non.past}-make. \\
    `So now our children work in other professions.' (K051222nar05)
\z

Nominalized clauses can also host {\em pada}. Note that there is no overt signalling of the derivation.

\ea\label{ex:form:pada:headless2}
   \gll [[[Neene       pada] anà-biilang] \zero{} pada]=jo  itthu. \\
     grandmother \textsc{pl}  \textsc{past}-say  { }  \textsc{pl}=\textsc{emph} \textsc{dist} \\
`What the grandmothers said was this.' (K051206nar03)
\z

\em Pada \em optionally combines with plural pronouns \xref{ex:pada:pron}  and quantifiers \xref{ex:pada:quant}. Since these are all inherently already plural,  \em pada \em only serves to emphasize that fact.

\ea \label{ex:pada:pron}
\gll Itthu=nam blaakang=jo, \textbf{kitham} \textbf{pada} anà-bìssar. \\
 \textsc{dist} after=\textsc{emph} 1\textsc{pl} \textsc{pl} \textsc{past}-big\\
`After that, we grew up.' (K060108nar02)
\z
 
\ea \label{ex:pada:quant}
\gll \textbf{Spaaru} \textbf{pada} bannya baae=nang anthi-duuduk. \\
     some \textsc{pl} very good=\textsc{dat} \textsc{irr}-stay \\
    `Some are well off.'  (K061122nar01)
\z

This emphasizing function can also be used in enumerations, as in \xref{ex:pada:enum}.

\ea \label{ex:pada:enum}
\gll \textbf{Mr} \textbf{Dole=pe} \textbf{Mr} \textbf{Samath=pe} \textbf{Mr} \textbf{Yusu} \textbf{pada}=le {\em interview}=nya thraa. \\
Mr Dole=\textsc{poss} Mr Samath=\textsc{poss} Mr Yusu \textsc{pl}=\textsc{addit} interview=\textsc{dat} \textsc{neg}\\
`Mr Dole, Mr Samath and Mr Yusu were not selected for the interview.' (K060116nar05)
\z

\subsubsection{The semantics of \em pada \em and \em hatthu\em}\label{sec:morph:Thesemanticsofpadaandhatthu}

\em Pada \em  might actually be better analyzed as expressing \em collective nominal aspect, \em rather than nominal number. Collective aspects signals `that the set consists of multiple individual entities which together form a collective' \citep[102]{Rijkhoff2002}. This collective interpretation of a set is supported by \xref{ex:pada:enum}, where a group of three gentlemen should give interviews. The three gentlemen are named and coordinated, but then the plural marker is added.
It is clear that there is only one specimen of each of the named persons, and the group only exists once, so \em pada \em cannot indicate cardinality greater than one in the strict sense. Rather, it emphasizes that we are dealing with a collectivity, the group is not seen as monolithic, but as composed of several members (`multiple individual entities'), and the cardinality of the members is greater than one. This interpretation as collective actually fits well with the optionality of \em pada \em according to Rijkhoff's presentation of \em set nouns\em.

A logical extension of the analysis of \em pada \em as a `collective aspect marker' would be to analyze  \em hatthu \em (called `indefiniteness marker' here) as a `singulative aspect marker', indicating that the set is conceptualized as a whole, and not as `consisting of multiple individual entities'. SLM \trs{ruuma}{house} would then be transnumeral, \trs{ruuma pada}{house \textsc{collective}} would mean `the concept ``house'' interpreted as consisting of multiple entities', and \trs{hatthu ruuma}{\textsc{singulative} house} would mean `the concept ``house'' interpreted as consisting of a singleton entity'. This `singularizing' function of \em hatthu \em finds support in example \xref{ex:ptcpt:mod:pl:hatthu}, where the concept of `parents', which is ontologically necessarily of cardinality greater than one, is modified with \em hatthu\em, indicating that it should be conceptualized holistically, and that the internal constituency of the concept does not matter.


 
\ea \label{ex:ptcpt:mod:pl:hatthu}
   \gll Kithang samma \textbf{hatthu} \textbf{umma}+\textbf{baapa}=pe      aanak pada, kithang samma sudaara pada. \\
    1\textsc{pl}     all   \textsc{indef}   mother+father=\textsc{poss} child \textsc{pl}   1\textsc{pl}     all   brother \textsc{pl} \\
`We are all the same parents' children, we are all brothers' (B060115cvs01)
\z


A reanalysis of the indefinite article as singulative has been proposed for Turkish by \citet{Schroeder1999}, and can be used to explain the uncommon order of the `indefinite article' \em bir \em in this language \citep[319]{Rijkhoff2002}. In Turkish, \em bir \em can intervene between an adjective and a noun as in \trs{me\c shur bir \c sair}{famous a poet}. If \em bir \em is an indefinite article, this order  would violate a universal  that the order ADJ INDEF N does not exist \citep{Greenberg1963,Hawkins1994}. If it is analyzed as a nominal aspect marker, on the other hand, the universal would not apply. 
% The interesting thing is that SLM shows the same structure as Turkish. An example would be \trs{bàrnaama hatthu oorang}{famous a man}. 

\subsection{Verbal morphology}\label{sec:verbalmorphology}
SLM has limited derivational morphology for verbs, but a comparatively large array of inflectional affixes as far as Malay varieties go.

\subsubsection{Derivational morphology}
Verbs can take the nominalizer \em -an \em and the causativizer \em -king\em.


\ea\label{ex:constr:deriv:an1}
\gll Se=ppe laayeng \textbf{omong-an} samma see anà-biilang. \\
      1\textsc{sg}=\textsc{poss} other speak-\textsc{nmlz} all 1\textsc{sg} \textsc{past}-say \\
    `I had said everything in my other speech.' (K061127nar03)
\z

\ea\label{ex:constr:deriv:king:verb}
\gll Baaye meera caaya kapang-jaadi, \textbf{thurung-king}. \\ % bf
     good red colour when-become, descend-\textsc{caus}  \\
    `When  [the food] has  turned to a nice rose colour, remove (it) [from the fire].'  (K060103rec02)
\z

Both can attach to any verb regardless of transitivity. Furthermore, an `involitive verb'  can be derived with the prefix \em kànà-\em. This implies lack of control on the part of the agent as shown in \xref{ex:kana:nom} and \xref{ex:kana:dat}.


\ea \label{ex:kana:nom}
\gll Se naasi arà-maakang. \\ %bf
     1\textsc{sg} rice \textsc{nonpast}-eat  \\
    `I eat rice.' (K081104eli03)
\z


\ea\label{ex:kana:dat}
\gll Se=dang naasi arà-\textbf{kànà}-maakang. \\
     \textsc{1s=dat} rice \textsc{nonpast}-\textsc{invol}-eat  \\
    `I eat rice without wanting it/I compulsively eat rice/I was forced to eat rice by something beyond my control.' (K081104eli03)
\z


\subsubsection{Inflectional morphology}
Inflection is preverbal in SLM (with the exception of imperatives) \citep{SmithEtAl2004,Slomanson2006cll}. Table \ref{tab:verbalaffixes} lists the forms. Verbs have only one preverbal slot for inflection. This means that some meanings cannot be combined in morphology. This is most notably the case for the debitive \em mas- \em and the habilitive \em bole-\em, which block the expression of tense on the same verb. The same is true of the prefix \trs{kapang-}{when}, which means any of \em when X verbs, when X verbed, when X will verb\em.

\begin{table}
    \centering
 \begin{tabular}{lllll}
arà-  & nonpast 	&  thàrà-   & past negative \\ 
anà-  & past 		&  thama-   & nonpast negative\\
su-   & past 		&  thus-   & want.not\\ 
mas-  & debitive 	&   jamà-    & subordinate negative \\
anthi-& irrealis  	&   kal-    & conditional   \\
mà-   & infintive  	&   kapang-  & when \\ 
asà-  & \multirow{2}{*}{\parbox{2cm}{conjunctive\\ 
		      participle} }  &  marà- & adhortative  \\ 
      &                &            & \\
\hline
bole= & can & & \\
\hline 
-la & imperative & -de & impolite imperative\\
 \end{tabular}
\caption{Verbal affixes in SLM.}
\label{tab:verbalaffixes}
\end{table}

For the glosses of these prefixes, some explanation is in order: \em arà- \em is often referred to as present, but it can also refer to the future as in \xref{ex:ara:fut}, so that `nonpast' is the more precise gloss.


\ea \label{ex:ara:fut}
\gll Laskalli \textbf{arà-maakang}. \\
 other.time \textsc{nonpast}-eat\\
`Next time you will eat.' (B060115cvs16)
\z


This contrasts with the past morphemes \em su- \em and \em anà-\em. These have the same temporal value, but differ in information structure. As a broad generalization, \em su- \em can only be used in pragmatically neutral contexts (predicate focus or sentence focus), whereas \em anà- \em is required with argument focus, although it can be used in neutral contexts as well. Example \xref{ex:morph:v:ana:argfoc} illustrates \em anà- \em  in an argument focus context, a constituent question in this case. Example \xref{ex:morph:v:ana:decl} shows that in a normal declarative sentence, both prefixes can be used


\ea\label{ex:morph:v:ana:argfoc}
\gll \textbf{Mana} binaathang lorang=yang \textbf{anà/*su-}giigith. \\
     which animal 2\textsc{pl}=\textsc{acc} \textsc{past}-bite  \\
    `Which animal bit you?' (K081105eli02)
\z



\ea\label{ex:morph:v:ana:decl}
\gll Itthu binaathan lorang=yang \textbf{anà-/su}-giigith. \\
     \textsc{dist} animal 2\textsc{pl}=\textsc{acc} \textsc{past}-/\textsc{past}-bite  \\
    `That animal bit you.' (K081105eli02)
\z


The morpheme \em anthi- \em is often glossed as `future', but can also be used for the apodosis of conditionals. This suggests that it rather has a modal value of `irrealis', encompassing situations which have not happened yet, regardless of whether one thinks that they are likely to happen (future) or not (irrealis). 

\ea\label{ex:form:anthi:apod:nokalu}
\gll Cinggala blaajar katha biilang thingka, ithu=kapang=jo gaaji athi-livath-king. \\
     Sinhala learn \textsc{quot} say when \textsc{dist}=when=\textsc{emph} salary \textsc{irr}-much-\textsc{caus}  \\
    `When I say learning Sinhala, that means that your salary increases then.' (K051222nar06)
\z

\subsubsection{Periphrases}

SLM does not make use of nominal reduplication, however, verbal reduplication is used to inflect verbs for present participle status, a calque of a Sinhala structure.


\ea\label{ex:redupl:cp}
\gll Kancil \textbf{lompath}\~{}\textbf{lompath} arà-laari. \\
     rabbit jump\~{}\textsc{red}      \textsc{nonpast}-run \\
    `The rabbit runs away jumping.'  (K081104eli06)
\z


Besides the morphologically marked tenses, SLM has a periphrastic tense, the perfect. This is formed by the so-called conjunctive participle  of a verb and the existential \em aada \em in the affirmative. In the negative, \em thraa \em replaces \em aada\em.


\ea\label{ex:perfect:asa}
\gll Itthu=le kitham=pe mlaayu pada=jo  itthu thumpath samma \textbf{asà-}kaasi \textbf{aada}. \\
but 1\textsc{pl}=\textsc{poss} Malay \textsc{pl}=\textsc{emph} \textsc{dist} place all \textsc{cp}-give exist \\
`Therefore it was our Malays who have  all given away those lands.' (B060115cvs04)
\z



\ea\label{ex:constr:pred:perf:neg}
\gll Hatthu dhaatha \textbf{asà-kaaving} \textbf{thraa}. \\
      one elder.sister \textsc{cp}-marry \textsc{neg} \\
    `One elder sister has not married.' (K061019prs01)
\z



SLM has two types of verb serialization, which are very similar to each other: Vector Verb Serialization and Motion Verb Serialization \citep{Nordhofffcsvc,Jaffartv}. Vector Verb Serialization follows a general South Asian model of marking aspectual and attitudinal values by a second verb following the primary verb.  In this case, this second verb loses its literal meaning and expresses aspectual, adverbal or attitudinal shades of meaning.


\ea\label{ex:constr:pred:vector:ambel}
\gll Kanabisan=ka=jo duva oorang=le \textbf{anà-thaau} \textbf{ambel} [Andare duva oorang=yang=le asà-enco-kang aada] katha. \\
      last=\textsc{loc}=\textsc{emph} two man=\textsc{addit} \textsc{past}-know take Andare two man=\textsc{acc}=\textsc{addit} \textsc{cp}-fool-\textsc{caus} exist \textsc{quot} \\
    `At the very end, both women understood that Andare had fooled both of them.' (K070000wrt05)
\z


A list of these vector verbs and their aspectual values is given in Table \ref{tab:vectorverbs}. In contradistinction to the comparable structures in most South Asian languages, none of the verbs is morphologically marked as being subordinate to the other; in most Indian languages which have a similar construction, the first verb is either a participle or an infinitive.

\begin{table}
    \centering
 \begin{tabular}{lll|lll}
 verb   & literal \newline meaning & aspectual \newline meaning    & verb & literal \newline meaning      &  aspectual \newline meaning  \\
\hline
 ambel   & take & inchoative  & duuduk & sit    &  progressive\\
 thaaro  & put  & hostilitive &  abbis & finish &  completive\\
 simpang & keep & prospective  \\       
 \end{tabular}
\caption{SLM vector verbs and their values.}
\label{tab:vectorverbs}
\end{table}

The second serial verb construction involves two motion verbs, broadly construed. The first denotes the manner of motion and the second, the path. The path is typically expressed by \trs{pii}{go} or \trs{dhaathang}{come}, but more complex paths like \trs{kluuling}{roam} are also possible.


\ea\label{ex:constr:pred:v:svc:pii}
\gll Hathu haari, hathu oorang thoppi mà-juval=nang kampong=dering kampong=nang su-\textbf{jaalang} \textbf{pii}. \\
     \textsc{indef} day \textsc{indef} man hat \textsc{inf}-sell=\textsc{dat} village=\textsc{abl} village=\textsc{dat} \textsc{past}-walk go  \\
    `One day, a man went and walked from village to village to sell hats.'  (K070000wrt01)
\z      



\ea\label{ex:constr:pred:v:svc:kluuling}
\gll Aanak su-\textbf{laari} \textbf{kluuling}. \\
       child \textsc{past}-run roam\\
    `The child went astray.' (K061019sng01)
\z


\subsection{Adjectival morphology}
Adjectives can take the causativizer \em -king\em, the nominalizer \em -an\em, and the superlative marker \em anà- \em (homophonous with the past tense marker). Adjectives can furthermore convert to nouns or verbs and then have the entire inflectional potential of those classes.

\subsection{Coordinating clitics}
Coordinating clitics play an important role in SLM morphosyntax. They can attach to NPs, Verbs, or clauses and are used to express a number of coordinations, quantifications and referential statuses. Their meaning varies depending on whether they are used on one host or on multiple hosts. A special meaning is obtained when combining them with interrogative pronouns. See Table \ref{tab:coordinatingclitics}.

\begin{table}
 \begin{center}
\begin{tabular}{lllll}
 & simple use & multiple use & indefinite pronoun & gloss\\
\hline
=le & X also & X and Y & everyone & additive\\
=si & X ? & X or Y ? & n/a & interrogative\\
=so & n/a & X or Y & someone & indeterminate\\
=ke & like X & X or Y & anyone & similitive\\
=pon & not any X & neither X nor Y & not anyone & any
 \end{tabular}
  \end{center}
\caption{Coordinating clitics.}
\label{tab:coordinatingclitics}
\end{table}
 
\subsection{Discourse clitics}
SLM has an ubiquitous `emphatic' clitic \em =jo\em, which is used for various senses of highlighting, e.g.
argument focus \xref{ex:jo:argfoc},  
contrastive topic \xref{ex:jo:contrtop},
and specificational/identificational predication \xref{ex:jo:speficicational}, 
among others.

% 
% \ea \label{ex:jo:argfoc:andare}
% \gll Suda kanabisan=ka raaja Andare=yang mà-enco-king asà-pii, \textbf{raaja=jo} su-jaadi enco. \\
%       thus last=\textsc{loc} king Andare=\textsc{acc} \textsc{inf}-fool-\textsc{caus} \textsc{cp}-go king=\textsc{emph} \textsc{past}-become fool \\
%     `So finally the king had tried to make a fool out of Andare, but it was the king who turned out to be the fool.'  (K070000wrt02)
% \z
% }

 
\ea\label{ex:jo:argfoc}
\gll \textbf{Guunung=ka=jo} kithang arà-duuduk; \textbf{guunung=nang=jo} kithang arà-pii. \\
      mountain=\textsc{loc}=\textsc{emph} 1\textsc{pl} \textsc{nonpast}-stay mountain=\textsc{dat}=\textsc{emph} 1\textsc{pl} \textsc{nonpast}-go \\
    `It is in the hills that we live, it is to the hills that we go.' (B060115prs01)
\z
 
\ea \label{ex:jo:contrtop}
\gll [Pon=pe \textbf{ruuma=ka=jo} thaama duuva thiiga aari athi-duuduk. \\
      bride=\textsc{poss} house=\textsc{loc}=\textsc{emph} earlier two three day \textsc{irr}-stay \\
    `As for at the bride's house, then, they would stay two, three days back then.'  (K061122nar01)
\z


 \ea\label{ex:jo:speficicational}
   \gll \textbf{Itthu}    \textbf{oorang} \textbf{pada=jo}       kithang. \\
    \textsc{dist} man \textsc{pl}=\textsc{emph} 1\textsc{pl} \\
`Those people are us' (K051222nar03)
\z



A less common morpheme is \em =jona\em, which is glossed as `phatic' here. It has about the same domain of usage as \em y'know \em in English, i.e. it is used to express that the speaker expects the hearer to be already aware of the propositional content conveyed.


\ea\label{ex:jona}
\gll Punnu mlaayu pada kàthaama {\em English}=\textbf{jona} anthi-oomong. \\
       many Malay \textsc{pl} earlier English=\textsc{phat} \textsc{irr}=speak \\
    `Many Malays would speak English, y'know,  in former times.'  (K051222nar06)
\z


\section{Syntax}
As a general rule, SLM is a right-headed language. The verb is normally the last element of the sentence, adpositions follow nouns, and nominal heads follow their modifiers. There are exceptions to all these rules but the overall tendency is very clear in all domains.

\subsection{Syntax of the NP}
The SLM NP is generally right-headed, although exceptions occur. Some nominal modifiers, like deictics, possessors, quantifiers, and relative clauses always precede the NP. Examples for this are given in \xref{ex:np:poss} to \xref{ex:np:quant}.


\ea\label{ex:np:poss}
\gll [\textbf{Se=ppe}     naama] Mohomed Imran Salim. \\ % bf
     1\textsc{sg}=\textsc{poss} name Mohomed Imran Salim  \\
    `My name is Mohomed Imran Salim.' (K060108nar01)
\z



\ea\label{ex:np:deic}
\gll [\textbf{Ithu} jaalang]=ka  mà-pii thàràboole. \\
      \textsc{dist} road=\textsc{loc} \textsc{inf}-go cannot \\
    `You could not take that road.'  (B060115nar05)
\z



\ea\label{ex:np:relc}
\gll [[Seelon=nang dhaathang aada \zero]$_{RELC}$ mlaayu] oorang ikkang. \\ % bf
 Ceylon-\textsc{dat} come exist { } Malay man fish\\
`The Malays who came to Sri Lanka were fishermen.' (K060108nar02)
\z


 
\ea\label{ex:np:quant}
\gll Inni     sudaari=pe   femili=ka    [\textbf{bannyak} oorang] tsunami=dang     s-puukul su-pii. \\
      \textsc{prox} sister=\textsc{poss} family=\textsc{loc} many people tsunami=\textsc{dat} \textsc{cp}-hit \textsc{past}-go\\
    `In this sister's family, many people were swept away by the tsunami.' (B060115nar02)
\z


It is possible for quantifiers to float outside of their NP. This is shown in the following example for \trs{samma}{all}.


\ea\label{ex:np:quant:float}
\gll [Kafan kaayeng pada] \textbf{samma} asà-ambel. \\
     shroud cloth \textsc{pl} all \textsc{cp}-take  \\
    `Having taken all the tissue for the shroud, ... .'  (B060115nar05)
\z


For some other modifiers, prenominal position is still the norm, but postnominal position is also allowed. This is true for numerals \xref{ex:np:num}, 
the indefiniteness marker \em hatthu \em \xref{ex:np:indef}, 
and adjectives \xref{ex:np:adj}.

\ea\label{ex:np:num}
    \ea
	\gll \textbf{Duva-pulu}    \textbf{ìnnam} \textbf{riibu}    \textbf{ùmpath}  \textbf{raathus} \textbf{lima-pulu}    \textbf{duuva} {\em votes}  incayang=nang    anà-daapath. \\ % bf
	two-ty six thousand four hundred five-ty two votes 3\textsc{sg}.\textsc{polite}=\textsc{dat} \textsc{past}-get\\
	    `He got 26,452 votes.' (N061031nar01)
    \ex
	\gll [Panthas  rooja   kumbang pohong  komplok]  \textbf{duuva} asà-jaadi su-aada. \\
	    beautiful rose flower tree bush two \textsc{cp}-grow \textsc{past}-exist \\
	    `Two beautiful rose bushes had grown.'  (K070000wrt04)
    \z
\z

\ea\label{ex:np:indef}
    \ea
	\gll \textbf{Hatthu} avuliya aada kitham=pe ruuma dìkkath. \\
	    \textsc{indef} saint exist 1\textsc{pl}=\textsc{poss}  house vicinity \\
	    `There is a saint close to our house.' (K060108nar02)
    \ex
	\gll See avuliya \textbf{atthu} su-jaadi. \\
	    1\textsc{sg} saint \textsc{indef} \textsc{past}-become \\
	    `I have become a saint.'  (K051220nar01)
    \z
\z

\ea\label{ex:np:adj}
    \ea
	\gll \textbf{Mìntha}$_{ADJ}$ \textbf{daaging}$_{N}$=yang cuuci. \\ % bf
	    raw beef=\textsc{acc} wash \\
	    `Wash the raw beef.'

{\dots}
%     \ex
% 	\gll asà-cuuci laada=le gaaram=le bathu giling-an=ka  giiling. \\ % bf
% 	    \textsc{cp}-wash pepper=\textsc{addit} salt=\textsc{addit} stone grind=\textsc{nmlz}=\textsc{loc} grind \\
% 	    `Having washed it, grind salt and pepper in a grinding stone.'
    \ex
	\gll asà-giiling \textbf{daaging}$_{N}$ \textbf{mùntha}$_{ADJ}$=yang baathu=ka asà-thaaro, giccak. \\ % bf
	    \textsc{cp}-grind beef raw=\textsc{acc} stone=\textsc{loc} \textsc{cp}-put smash  \\
	    `Having ground and having put the raw beef on a stone, smash it.' (K060103rec02)
     \z
\z

 There is one nominal modifier  which can only occur at the very right edge. This is the plural marker \em pada\em.

\ea
\gll [Spaaru oorang \textbf{pada}] su-pii. \\
     some man \textsc{pl} \textsc{past}-go  \\
    `Some men left.'  (B060115nar01)
\z

Furthermore, NPs can contain two or more nouns, which can be analysed as compounds or adnominal modification. These sequence are either left-headed like {\em oorang ikkang} `man'+`fish'=`fisherman' or right-headed like {\em paadi+pohong\em} `unharvested rice' + `tree' = `rice plant'.

The order of elements in the NP is schematized as below. All elements are optional. This also applies to the head noun. 

\footnotesize
\cbx[\label{cb:np:prepostnom}]{ 
$\oldstylenums{1}$
$\left\{\begin{array}{l} \rm DEIC\\\rm POSS\end{array}\right\}$
RELC
$\left\{\begin{array}{l} \rm POSS\\\rm DEIC\end{array}\right\}$
$\oldstylenums{1}$
QUANT
$\left\{\begin{array}{l} \rm NUM\\\rm ADJ\\\end{array}\right\}$*
$\oldstylenums{1}$
N
$\oldstylenums{1}$
\textbf{Noun}
$\begin{array}{c}\oldstylenums{1}\\\rm N (PL) \\\rm ADJ (PL)\\\rm QUANT (PL) \\\rm NUM (PL) \end{array}$}{NP}
\normalsize

This reads as follows: elements to the left precede elements to the right. Elements in vertical alignment offer a choice of any one of the vertically aligned elements, but not more than none.  The asterisk indicates that there maybe several instances of this modifier. $\oldstylenums{1}$ represents the position of the indefiniteness marker, which is particularly interesting and is discussed in more detail in the following section.

\subsection{The position of the indefiniteness marker}\label{sec:hatthu}
The indefiniteness marker \em hatthu \em has some very peculiar morphosyntactic properties. First, it can occur before \xref{ex:positionhatthu:adjhatthun}, after \xref{ex:positionhatthu:hatthuadjn}, or on both sides of the noun \xref{ex:positionhatthu:hatthunhatthu}. Second, its position is normally between prenominal modifiers and the head noun, violating the principle of scope. This is found e.g. for the sequence of adjective and noun in \xref{ex:positionhatthu:adjhatthun}, which is interrupted by \em hatthu\em.

\ea\label{ex:positionhatthu:adjhatthun}
\gll Itthu [bannyak [laama]$_{ADJ}$ [hathu]$_{INDEF}$ [ruuma]$_{N}$]. \\ % bf
      \textsc{dist} very old \textsc{indef} house \\
    `That one was a very old house.'  (K070000wrt04)
\z
 

\ea\label{ex:positionhatthu:hatthuadjn}
\gll Hathu muusing=ka ...  [hathu]$_{INDEF}$  [kiccil]$_{ADJ}$ [ruuma]$_{N}$ su-aada \\
     \textsc{indef} time=\textsc{loc} ... \textsc{indef} small house \textsc{past}-exist  \\
    `Once upon a time, there was a small house.'  (K07000wrt04)
\z

\ea\label{ex:positionhatthu:hatthunhatthu}
\gll Sithu=ka hathu maccan hathu  duuduk aada. \\
     there=\textsc{loc} \textsc{indef} tiger \textsc{indef} stay exist  \\
    `A tiger stayed there.'  (B060115nar05)
\z

The `interrupting' position of \em hatthu \em is also attested with N+N sequences.

\ea
\gll Ini pohong atthas=ka \textbf{moonyeth} \textbf{hathu}=\textbf{kavanan} su-aada. \\
     \textsc{prox} tree top=\textsc{loc}  monkey \textsc{indef}=group \textsc{past}-exist\\
    `On top of this tree was a group of monkeys.'   (K070000wrt01)
\z


\subsection{Simple declarative clause}
The simple declarative clause is verb final and has SOV word order.\footnote{As
 usual in word order typology, S and O have semantic definitions.  The label `SOV' is used for convenience here. Neither subject nor object have straightforward definitions in SLM. The descriptor `NP NP V' would be more appropriate than `SOV', but  preference is given here to the more familiar term `SOV'.} The order in the preverbal field is free, but due to frequent dropping of elements, the question of the relative ordering of elements is normally moot since only one NP is realized. 

An example of a simple declarative clause is given below.

% 
% \ea\label{ex:clause:decl:ppdrop2}
% \gll [Samma thumpath]$_{top}$=\zero{}  mlaayu aada. \\
%      every place Malay exist  \\
%     `In every place, there are Malays.'  (K051222nar04)
% \z      
% }\\ 



\ea\label{ex:semrole:pat:theme:zero}
\gll Se=ppe    baapa  incayang=nang    ummas=\zero{} su-kaasi. \\ % bf
      \textsc{1s=poss} father 3\textsc{sg}.\textsc{polite}=\textsc{dat} gold \textsc{past}-give\\
    `My father gave him gold.'  (K070000wrt04)
\z

 

It is possible to find NPs after the verb. The postverbal position  can be occupied by 
 heavy elements \xref{ex:clause:decl:right:heavy}, 
 spatial arguments of location or goal \xref{ex:clause:decl:right:local1}, and 
 elements in focus \xref{ex:clause:decl:right:foc2}. 

\ea\label{ex:clause:decl:right:heavy}
\gll Kitha=nang$_{NP}$ maau$_{PRED}$ [\textbf{kitham=pe} \textbf{mlaayu} \textbf{lorang} \textbf{blaajar} \textbf{lorang=pe} \textbf{mlaayu} \textbf{kitham} \textbf{blaajar}]$_{NP}$. \\
 1\textsc{pl}=\textsc{dat} want 1\textsc{pl}=\textsc{poss} Malay 2\textsc{pl} learn 2\textsc{pl}=\textsc{poss} Malay 1\textsc{pl} learn\\
`We want that you learn our Malay and that we learn your Malay.' (K060116nar02)
\z

\ea\label{ex:clause:decl:right:local1}
\gll see anà-laaher     \textbf{Navalapitiya=ka}$_{loc}$ \\
     1s \textsc{past}-be.born Nawalapitiya=\textsc{loc}  \\
    `I was born in Nawalapitiya.' (K051201nar01)
\z

\ea\label{ex:clause:decl:right:foc2}
\gll Hindu arà-maakang \textbf{kambing}. \\
 Hindu \textsc{nonpast}-eat goat\\
`Hindus eat GOAT.' (K060112nar01)
\z

\subsection{Copular clauses}
One  noteworthy morphosyntactic difference between SLM and other varieties of Malay is the presence of a copula, \em (asà)dhaathang(apa)\em. This copula derives from the conjunctive participle of the verb \trs{dhaathang}{come}, and probably went from a resultative meaning similar to English \em a dream come true \em to  a stative meaning mainly used for predicates of group membership. A fuller account can be found in \citet{Nordhoff2011copula}. The three forms given below are in free variation.

\ea
\gll Se=ppe    naama \textbf{asà}dhaathang  Cintha Sinthani. \\
     1\textsc{sg}=\textsc{poss} name \textsc{copula} Chintha Sinthani. \\
    `My name is Chintha Sinthani.' (B060115prs04)
\z
 
\ea
\gll Se=ppe    baapa  dhaathang\textbf{apa}  Jinaan Samath. \\
     1\textsc{sg}=\textsc{poss} father \textsc{copula} Jinaan Samath. \\
    `My father was Jinaan Samath.' (N060113nar03)
\z

\ea
\gll {\em Estate}=pe {\em field} {\em officer} \textbf{asà}dhaathang\textbf{apa}  kithang=pe     kaake\\
     estate=\textsc{poss} field officer \textsc{copula} 1\textsc{pl}=\textsc{poss} grandfather. \\
    `The estate field officer was our grandfather.' (N060113nar03)
\z

\subsection{Relative clause}
Relative clauses are preposed and not signalled by morphology. There is no verbal marker of relative status nor is there a relative pronoun. Relative clauses are not restricted in any way as compared to main clauses. Any morphology, any tense, and any predication type may be used in relative clauses. 

The following examples show relative clauses where the relativized element is
the agent \xref{ex:relc:ag},
a place \xref{ex:relc:loc},
a point in time \xref{ex:relc:time}, and
a possessor \xref{ex:relc:poss}.

\ea\label{ex:relc:ag}
\gll thoppi arà-daagang    oorang \\
     hat \textsc{nonpast}-trade man  \\
    `The hat seller' (K070000wrt01)
\z


\ea\label{ex:relc:loc}
   \gll Siithu [nya-duuduk    thumpath]=ka baapa  su-jaatho. \\ % bf
     there \textsc{past}-stay place=\textsc{loc} father \textsc{past}-fall  \\
    `There, at the place he was staying, my father fell.' (K051205nar05)
\z

\ea\label{ex:relc:time}
\gll Suda [puthri=le biini=le arà-caa{\und}a haari]=le su-dhaathang. \\
      so queen=\textsc{addit} wife=\textsc{addit} \textsc{simult}-meet day=\textsc{addit} \textsc{past}-come \\
    `So then the day came when the wife and the queen were to meet.'  (K070000wrt05)
\z      

\ea\label{ex:relc:poss}
\gll [Laaki anà-mnii\u n\u ggal hathu pompang]. \\ % bf
     husband \textsc{past}-die \textsc{indef} woman  \\
    `A woman whose husband had died.'  (K070000wrt04)
\z      


SLM also has the possibility to form so-called `headless relative clauses'. The relative clause in \xref{ex:hrelc:relc} and the headless relative clause in \xref{ex:hrelc:hrelc} share the same structure, but the head is missing in \xref{ex:hrelc:hrelc}.
 
\ea\label{ex:hrelc:relc}
\gll [Lorang anà-maasak ikkang] eenak. \\
     2\textsc{pl} \textsc{past}-cook fish tasty  \\
    `The fish you cooked is tasty.'  (K081105eli02)
\z
 
\ea\label{ex:hrelc:hrelc}
\gll [Lorang anà-maasak \zero] eenak. \\
     2\textsc{pl} \textsc{past}-cook { } tasty  \\
    `What you cooked is tasty.'  (K081105eli02)
\z

\subsection{Argument clause}
Clauses used as arguments of e.g. perception verbs do not take any special morphological marking. There is no complementizer, no special verb form, etc. The subordinate clause is simply put in the place that another NP would have in the predication. Argument clauses can take case enclitics like any other NP. This is shown below for the dative.


\ea\label{ex:clauses:arg:main}
    \ea
	\gll \textbf{lorang} \textbf{suurath=yang} \textbf{mlaayu=dering} \textbf{anà-thuulis}\\ % bf
	2\textsc{pl} letter=\textsc{acc} Malay=\textsc{abl} past=write \\
	    `You wrote the letter in Malay.'   
    \ex
	\gll Kitham=pe baapa su-biilang [[\textbf{lorang} \textbf{suurath=yang} \textbf{mlaayu=dering} \textbf{anà-thuulis}]$_{CLAUSE}$=nang bannyak arà-suuka]. \\ % bf
	    1\textsc{pl}=\textsc{poss} father \textsc{past}-say 2\textsc{pl} letter=\textsc{acc} Malay=\textsc{abl} past=write=\textsc{dat} much \textsc{simult}-like \\
	    `Daddy said that he liked very much that you wrote the letter in Malay.'  (Letter 26.06.2007)
    \z
\z

 

% In some cases, the argument clause takes the accusative marker \em =yang\em, even if it is not a patient semantically nor is there any reason to assume that it is a direct object. The morpheme \em =yang \em then signals the status as argument clause. It is debatable whether this use is really an instance of the accusative morpheme, or whether we are dealing with a homophonous morpheme here.
% 
% \ea yang \ea

Utterance verbs like \trs{biilang}{say} or \trs{butharak}{shout} use the quotative \em katha \em to mark their arguments. These arguments are not clauses, but full utterances. This can be seen from the fact that non-clausal utterances like \trs{iiya}{yes} or onomatopoetic expressions like \em dam dam dam \em can also be used with \em katha\em.


\ea\label{ex:form:katha:psych1}
\gll Skarang biini arà-iingath  [\textbf{puthri} \textbf{thuuli} \textbf{katha}]; Puthri arà-iingath [\textbf{biini} \textbf{thuuli} \textbf{katha}].\\
   now wife \textsc{nonpast}-think queen deaf \textsc{quot} queen \textsc{nonpast}-think wife deaf \textsc{quot} \\
    `Now the wife thought the queen was deaf, and the queen thought the wife was deaf.'  (K070000wrt05)
\z



\ea \label{ex:katha:interj1}
\gll \textbf{{\em Yes}}  \textbf{katha} m-biilang. \\
  yes \textsc{quot} \textsc{past}-say\\
`He said ``yes''.' (K060116nar11)
\z



\ea \label{ex:katha:interj2}
\gll Aanak pompang duuva=nang [\textbf{slaamath}] \textbf{katha} su-biilang. \\
     child girl two=\textsc{dat} goodbye \textsc{quot} \textsc{past}-say  \\
    `He said ``Goodbye'' to the two girls.'  (K070000wrt04)
\z




\ea \label{ex:katha:onom1}
\gll \textbf{Dam} \textbf{dam} \textbf{dam} \textbf{katha} su-aada. \\
     dam dam dam \textsc{quot} \textsc{past}-exist  \\
    `(The rain) went like ``dam dam dam''.'  (overheard)
\z


Purposive clauses are formed with the infinitive \em mà- \em and the dative marker \em =nang\em. They can precede the matrix verb \xref{ex:cls:purp:pre} but are often dislocated because of their heaviness \xref{ex:cls:purp:post}.

\ea\label{ex:cls:purp:pre}
\gll Derang [dìkkath=ka aada laapang]=nang [mà-maayeng]$_{purp}$=nang su-pii. \\ % bf
     3\textsc{pl} vicinity=\textsc{loc} exist ground=\textsc{dat} \textsc{inf}-play=\textsc{dat} \textsc{past}-go  \\
    `They went to play on the ground nearby.'  (K070000wrt04)
\z 

\ea\label{ex:cls:purp:post}
\gll Itthu   {\em cave}=nang kithang=le pii aada [mà-liyath]$_{purp}$=nang. \\ % bf
 \textsc{dist} cave=\textsc{dat} 1\textsc{pl}=\textsc{addit} go exist \textsc{inf}-look=\textsc{dat}    \\
    `We have also gone to that cave to have a look.'   (K051206nar02)
\z

\subsection{Interrogative clause}
\subsubsection{Polar questions}
Polar questions are normally marked by the interrogative clitic \em =si \em and a particular intonation contour. Normally, both of these signs are present, but one only can also be sufficient. The interrogative clitic attaches to the verb when the whole sentence is questioned.


\ea\label{ex:cl:interr:cl:v}
\gll Se=pe uumur masà-biilan=\textbf{si}? \\
 1=\textsc{poss} age must-tell=\textsc{interr}\\
`(Do I) have to tell my age?' (B060115prs01)
\z


The clitic attaches to an NP or other element if only this portion of the sentence is questioned. In this case, the choice of past tense morphemes is restricted to \em anà-\em; \em su- \em is not possible. 


\ea\label{ex:form:suana:question:polarconst}
\gll \textbf{Daging baabi=si} *su-/anà- bìlli??? \\
	pork=\textsc{interr} \textit{su-anà-} buy  \\
    `Did you buy PORK???'  (K081105eli02)
\z


It is possible to attach \em =si \em to more than one element, resulting in an alternative question.


\ea\label{ex:si:alternative}
\gll Piisang=si maa\u n\u gga=si maau? \\
     plantain=\textsc{interr} mango=\textsc{interr} want  \\
    `Is it plantain or mango that you want?'  (K081105eli02)
\z


\subsubsection{Content Questions}
Content questions are formed by putting a suitable interrogative pronoun (see Section \ref{sec:interrogativepronouns}) at the place of the questioned element. This is thus \em in situ\em, although interrogative pronouns also frequently occur in initial position. Since this position could also be occupied by normal NPs, it can still be analyzed as \em in situ\em.


\ea\label{ex:cl:interr:wh:initial1}
\gll \textbf{Mana} nigiri=ka arà-duuduk? \\
 which country=\textsc{loc} \textsc{nonpast}-stay\\
`In which country do you live?' (B060115cvs16)
\z



\subsection{Imperative clause}
Imperative clauses are formed by either attaching one of the suffixes \trs{-la}{\textsc{imp.polite}} and \trs{-de}{\textsc{imp.impolite}}, by using the preverbal particle \em mari\em, or both. 



\ea \label{ex:form:affix:la:double}
    \ea
	\gll Saayang se=ppe thuan \textbf{mari} laari-\textbf{la}. \\
	    love 1\textsc{sg}=\textsc{poss} sir come.\textsc{imp} run-\textsc{imp} \\
	    `Come my beloved gentleman, come running here.'
    \ex
	\gll See=samma kumpul \textbf{mari} thaa\u ndak-\textbf{la}. \\
	    1\textsc{sg}=\textsc{comit} gather come.\textsc{imp} dance-\textsc{imp}  \\
	    `Come and dance with me.' (N061124sng01)
    \z
\z


Additionally, \em mari(-la) \em used without a verb means `Come!' The use of a bare verb is also possible \xref{ex:imp:bare}.
 
\ea\label{ex:imp:bare}
\gll Aajuth thaakuth=ka su-naangis, ``See=yang \zero-luppas-\zero''. \\ % bf
     dwarf fear=\textsc{loc} \textsc{past}-cry ~~1\textsc{sg}=\textsc{acc} leave  \\
    `The dwarf screamed in fear: ``Leave me!'' '  (K070000wrt04)
\z 

Adhortatives are formed with the prefix \em marà\em, etymologically related to \em mari\em.

\ea
\gll Kitham \textbf{marà}-maayeng. \\
     1\textsc{pl} \textsc{adhort}-play  \\
    `Let's play.' (K081104eli06)
\z


Negative imperatives are formed with the verbal prefix \em jamà-\em.

\ea
\gll Hatthu=le \textbf{jamà}-gijja, baapa, ruuma=ka duuduk! \\ % bf
      \textsc{indef}=\textsc{addit} \textsc{neg.imp}-do father house=\textsc{loc} stay \\
    `Don't do anything, daddy, stay at home!'  (B060115nar04)
\z     


\section{Cases}
Case relations are realized as enclitic postpositions in SLM. They can attach to a noun, a pronoun, a quantifier, a deictic, a numeral, or an argument clause. There is a many-to-many relationship between case postpositions and semantic roles. The most frequent morphemes are 
\trs{=nang}{\textsc{dat}},
\trs{=yang}{\textsc{acc}},
\trs{=ka}{\textsc{loc}},
\trs{=dering}{\textsc{abl}}, and
\trs{=pe}{\textsc{poss}}. 
In addition, there are a number of temporal and causal postpositions, which are also enclitics, but are much less frequent than the morphemes just mentioned. 


%  This is summarized in Table \ref{tab:postpositionssemantics}.
% 
% \begin{table}
% \centering
% \begin{center}
% % use packages: array
% \begin{tabular}{cccccc}
% sduuduk		&	=dering 	&	=\zero{} 	&	=yang	& 	=nang 	& 	=ka 		\\
% 		&\framebox[2cm]{INSTR}	&  \multicolumn{3}{c}{\framebox[6.8cm]{PAT}} 		&\framebox[2cm]{LOC}	\\
% 		&\multicolumn{2}{c}{\framebox[4cm]{AGENT}}	&  		&\multicolumn{2}{c}{\framebox[4.4cm]{GOAL}}\\
% \multicolumn{2}{c}{\framebox[4cm]{SRC}}&\multicolumn{2}{c}{\framebox[4.4cm]{THEME}}& \framebox[2cm]{REC}  		\\
% 		&\multicolumn{2}{c}{\framebox[4cm]{TEMP DOMAIN}}&		& \framebox[2cm]{BEN}	 		\\
% 		&\framebox[2cm]{PATH} 	&\framebox[1.2cm]{ROLE}	& 		& \framebox[2cm]{PURP}  		\\
% 		&\framebox[2cm]{~SET DOM.\begin{picture}(0,0)(0,0)
% 					\put(5,6){\line(1,0){227}}
% 					\put(5,4){\line(1,0){227}}
% 					\end{picture}
% 					}&			&		& 		&\framebox[2cm]{SET DOM.}\\
% 		&\multicolumn{2}{c}{\framebox[4.4cm]{DURATION}}	& 		& \framebox[2cm]{EXP}    		\\
% 		&			&			& 		& \framebox[2cm]{PORTION} 		\\
% 		&			&			& 		& \framebox[2cm]{VALUE} 		\\
% \end{tabular}
% \end{center}
% \caption[Repartition of semantic roles on morphemes]{Repartition of semantic roles on morphemes. \em =nang \em is used to express eight different semantic roles, \em =dering \em is used for seven, while the other morphemes have lower numbers of semantic roles they can express. Eight semantic roles can be expressed by more than one morpheme. The two morphemes which can be used for the role \textsc{set domain} are linked by a line because for typographical reasons they could not be made contiguous. The semantic roles of \textsc{comitative}, \textsc{role}, and \textsc{reason} are left out of the graphic.}
% \label{tab:postpositionssemantics}
% \end{table}


\section{Valency}
\begin{table}
  \begin{tabular}{l|lll|l}
  one-place & \multicolumn{3}{c}{two-place}       & three-place \\
  \hline
  NOM V     & NOM NOM   & DAT NOM   & INSTR NOM   & NOM NOM DAT \\
  DAT V     & NOM ACC   & DAT ACC   & INSTR DAT   & NOM ACC DAT \\
  ACC V     & NOM DAT   & DAT DAT   & INSTR ACC   & NOM NOM ABL \\
  INSTR V   & NOM INSTR & DAT INSTR & INSTR INSTR & NOM ACC ABL \\
  \end{tabular}
\caption{Overview of case frames for predicates of different arity.}
\label{tab:ng:valency}
\end{table}

SLM does not have very solid valency distinctions. Addition and deletion of arguments are constrained by semantic plausibility, but only marginally by the lexical class of a lexeme. Typical case frames are given in Table \ref{tab:ng:valency}.

\subsection{One-place predicates}
SLM is a role-dominated language. The notions of `subject' and `object' are not needed to describe its syntax. The semantic role of a referent is directly reflected in morphosyntax through the case marker selected (See Section \ref{sec:nominalmorphology}). One-place-predicates can have their arguments marked with either nominative, accusative, dative, or instrumental. The discussion of case frames will therefore be restricted to these four cases here, setting aside spatial and temporal cases, among others. 
As for one-place predicates, nominative is the most common case \xref{ex:pred:argstr:1:zero:actor}. 
Dative is assigned for experiencers \xref{ex:pred:argstr:1:zero:dative:exp}, or if a modal is present in the clause \xref{ex:pred:argstr:1:zero:dative:modal}. 
Accusative-marking is very restricted in one-place predicates and has only been found on one verb \trs{thìnggalam}{sink} \xref{ex:pred:argstr:1:acc:titanic}. 
Finally, the instrumental is not conditioned by semantic role, but by the nature of the referent: if it is an institution, such as a government or a committee, use the instrumental \xref{ex:pred:argstr:1:instr}, otherwise, the postposition is assigned according to semantic role.



\ea\label{ex:pred:argstr:1:zero:actor}
\gll Itthukapang      Tony Hassan={\zero}{\footnotemark} su-pii. \\ % bf
      then Tony Hassan \textsc{past}-go \\
    `Then Tony Hassan left.' (K060116nar09)
\z

\footnotetext{I use {\zero} to explicitly indicate the absence of a postposition.}
 
 

\ea\label{ex:pred:argstr:1:zero:dative:exp}
\gll Go=\textbf{dang}    karang bannyak thàràsìggar. \\
     1s.\textsc{familiar}=\textsc{dat} now very sick  \\
    `I am now very sick.' (B060115nar04)
\z
 




\ea\label{ex:pred:argstr:1:zero:dative:modal}
\gll   Kithang=\textbf{nang}   \el{}    {\em two} {\em o'clock}=ke=sangke  bole=duuduk. \\
      1\textsc{pl}=\textsc{dat} { }    two o'clock=\textsc{simil}=until can-stay  \\
    `We can stay up until two o' clock.' (K061026rcp04)
\z



\ea\label{ex:pred:argstr:1:acc:titanic}
\gll {\em Titanic} kappal=\textbf{yang} su-thìnggalam. \\
     Titanic ship=\textsc{acc} \textsc{past}-sink  \\
    `The ship ``Titanic'' sank.' (K081104eli05)
\z

\ea\label{ex:pred:argstr:1:instr}
\gll {\em Police}=\textbf{dering} su-dhaathang. \\
     police=\textsc{abl} \textsc{past}-come  \\
    `The police came.' (K081105eli02)
\z


\subsection{Two-place predicates}\label{sec:argstr:Two-placepredicates}
With two place predicates, we have to distinguish the actor argument and the undergoer argument. The actor will normally be either in the nominative (most verbs) or in the dative (experiencer verbs and modals). It will be in the instrumental under the same conditions that hold for one-place predicates.

The non-actor argument will normally be in the accusative,
although the accusative marker \em =yang \em is often dropped \xref{ex:pred:argstr:2:zeroacc}. 
Recipients and beneficiaries \xref{ex:pred:argstr:2:zerodat:banthu}, 
as well as some rare patients (of the verbs \trs{puukul}{hit}, \trs{thiikam}{stab} \trs{theembak}{shoot}) will also be in the dative \xref{ex:pred:argstr:2:zerodat:puukul}. 
The person being addressed in questions appear in the instrumental. 


In case of experiencer verbs, the stimulus will normally be in the nominative.
In some cases, most notably \trs{thaau}{know}, the stimulus/theme can also be in the accusative.

If a modal is present in a clause with a verb which already assigns dative to the non-actor, both actor and non-actor will be marked with the dative, giving rise to ambiguity.




\ea \label{ex:pred:argstr:2:zeroacc}
\gll  Ithukapang       lorang=pe     leher=(\textbf{yang})   kithang=\zero{} athi-poothong. \\
      then 2\textsc{pl}=\textsc{poss} neck=\textsc{acc} 1\textsc{pl} \textsc{irr}-cut \\
    `Then we will cut your neck.' (K051213nar06)
\z




\ea\label{ex:pred:argstr:2:zerodat:banthu}
\gll Derang pada=\zero{} arà-banthu cinggala  raaja=\textbf{nang}. \\
     3\textsc{pl} \textsc{pl} \textsc{nonpast}-help Sinhala king=\textsc{dat}  \\
    `They help the Sinhalese king.' (K051206nar03)
\z



\ea \label{ex:pred:argstr:2:zerodat:puukul}
\gll   Rose-red=\zero{} buurung=\textbf{nang}   su-puukul. \\
      Rose-red bird=\textsc{dat} \textsc{past}-hit \\
    `Rose-red hit the bird.' (K070000wrt04)
\z


\xref{ex:pred:argstr:2:zerodat:puukul} shows one possibility of assigning zero and dative to two arguments. The actor is \zero-marked while the undergoer receives dative marking. Another possibility, which exists for experiencer verbs, is to mark the experiencer  with the dative \em =nang \em and the stimulus  with zero. This is shown in \xref{ex:pred:argstr:2:zerodat:dinggar} (and is a common South Asian construction  \citep[159ff]{Masica1976}).



\ea \label{ex:pred:argstr:2:zerodat:dinggar}
\gll [svaara hatthu]=\zero{}  derang=\textbf{nang}   su-dìnngar. \\
     noise \textsc{indef} 3\textsc{pl}=\textsc{dat} \textsc{past}-hear  \\
    `They heard a noise.' (K070000wrt04)
\z


Institutional actors take the instrumental as discussed above. The other argument can be marked with \em =yang \em \xref{ex:pred:argstr:2:instracc}, or bear no marking \xref{ex:pred:argstr:2:instrzero}.

 


\ea \label{ex:pred:argstr:2:instracc}
\gll See=\textbf{yang} {\em police}\textbf{=dering} nya-preksa. \\
     1\textsc{sg}=\textsc{acc} police=\textsc{abl} \textsc{past}-enquire  \\
    `I was questioned by the police.' (K051213nar01)
\z

 

\ea \label{ex:pred:argstr:2:instrzero}
\gll {\em British}  {\em Government}=\textbf{dering}   {\em Malaysia} Indonesia,  [inni nigiri pada]=\zero{}    samma peegang. \\
    British Government=\textsc{abl}  Malaysia Indonesia \textsc{prox} country \textsc{pl}  all catch\\
   `The British Government captured Malaysia and Indonesia,  those countries.' (K051213nar06)
\z

With modal proclitics, the actor can receive dative marking. In \xref{ex:pred:argstr:2:zerodat:baaca}, \trs{kithang}{we} receives dative marking and the theme of reading, \trs{mulbar}{Tamil}, is zero-marked.


\ea \label{ex:pred:argstr:2:zerodat:baaca}
\gll Kithang=\textbf{nang} baaye=nang mulbar=\zero{} bole=baaca. \\
      1\textsc{pl}=\textsc{dat} good=\textsc{dat} Tamil can=read \\
    `We can read Tamil well.'  (K051222nar06)
\z      


The presence of a  modal proclitic does not preclude accusative marking on the undergoer. This is found for instance in \xref{ex:pred:argstr:2:accdat:aathi}.



\ea \label{ex:pred:argstr:2:accdat:aathi}
\gll   aathi=\textbf{yang} sajja hatthu oorang=\textbf{nang} bole=ambel. \\
        liver=\textsc{acc} only one man=\textsc{dat} can-take \\
    `Only one person can take the liver.'
\z

 

% 
% \ea \label{ex:pred:argstr:2:accdat:army}
% \gll Se\textbf{dang} aada  ini      {\em army} pada=\textbf{yang}  mà-salba-kang=nang. \\
%      \textsc{1s=dat} exist \textsc{prox} soldier \textsc{pl}=\textsc{acc} \textsc{inf}-escape-\textsc{caus}=\textsc{dat}  \\
%     `I had to save these soldiers.' (K051213nar01)
% \z
% }

When using a verb which normally assigns the dative case to the undergoer  a modal proclitic results in  both arguments being  marked with the dative.


\ea
\gll Se=\textbf{dang} Farook=\textbf{nang} bole=puukul. \\
     \textsc{1s=dat} Farook=\textsc{dat} can=hit  \\
    `I can hit Farook.' (K081104eli05)
\z


In these cases, the actor is normally associated with the leftmost argument, while the undergoer is the other argument.  

% 
% \ea\label{ex:pred:argstr:2:wo:dative:pointing}
% \gll Ini kaaka=nang itthu kaaka=nang bole=puukul. \\
%      \textsc{prox} elder.brother=\textsc{dat} \textsc{dist} elder.brother=\textsc{dat} can=hit  \\
%     `This brother can hit that brother.'
%     `That brother can hit this brother.' (K081104eli05)
% \z


To sum up, two-place predicates normally have zero-marked actors, and undergoers are either marked for accusative or dative. In special cases, actors can be marked for instrumental or dative.

% G051222nar01.txt:\tx se=dang se=ppe    biinile      thiiga
% G051222nar01.txt:\tx aanak pada araduuduk


\subsection{Three-place predicates}\label{sec:argstr:Three-placepredicates}
Three-place predicates are typically predicates of transfer, i.e. giving and taking. As such, they include an agent, a theme, and a goal (or source). The agent is typically in the nominative \xref{ex:argstr:3:zzd:ummas}-\xref{ex:argstr:3:zzd:jaithan}, although modals in the sentence can change this to dative \xref{ex:pred:argstr:3:ddz}, and institutional agents trigger instrumental marking\xref{ex:argstr:3:instr}.  The theme is either unmarked (=nominative) \xref{ex:argstr:3:zzd:ummas}-\xref{ex:pred:argstr:3:ddz} or in the accusative \xref{ex:argstr:3:instr}. Recipients, beneficiaries and goals are marked by the dative, and sources by the ablative. The following examples illustrate this


\ea \label{ex:argstr:3:zzd:ummas}
\gll Se=ppe    baapa=\zero{}  incayang=\textbf{nang}    ummas=\zero{} su-kaasi. \\
      1\textsc{sg}=\textsc{poss} father 3\textsc{sg}.\textsc{polite}=\textsc{dat} gold \textsc{past}-give\\
    `My father gave him gold.'  (K070000wrt04)
\z      


\ea \label{ex:argstr:3:zzd:jaithan}
\gll Derang=pe umma=\zero{}   derang=\textbf{nang}  [jaithan=\zero=le,  jaarong  pukurjan=\zero=le]      su-aajar. \\
     3\textsc{pl}=\textsc{poss}  mother 3\textsc{pl}=\textsc{dat} sewing=\textsc{addit} needle work=\textsc{addit} \textsc{past}-teach \\
    `Their mother taught them sewing and needle work.' (K070000wrt04)
\z




\ea \label{ex:pred:argstr:3:ddz}
\gll Kithang=\textbf{nang} miskin pada=nang duvith={\zero} bole=kaasi. \\
     1\textsc{pl}=\textsc{dat} poor \textsc{pl}=\textsc{dat} money can=give  \\
    `We can give money to the poor.' (K081104eli05)
\z



\ea \label{ex:argstr:3:instr}
\gll {\em Police}=\textbf{dering} see=\textbf{yang} {\em remand}=\textbf{nang} su-kiiring. \\
     police=\textsc{abl} 1\textsc{sg}=\textsc{acc} remand=\textsc{dat} \textsc{past}-send  \\
    `The police sent me into custody.' (K081105eli02)
\z



\subsection{Summary of valency structure}\label{sec:argstr:Summaryofargumentrstructure}
To sum up the distribution of zero, accusative, dative and instrumental, on the roles of S, A, P and R (=Recipient), the following can be said:\footnote{The Sinhala facts are very similar to this. See \citet{Gair1976sinhalasubject,Gair1991infl,Henadeerage2002}.}

\begin{itemize}
 \item The dative marker can be found on R and P. Additionally, it can be found on S and A if they are experiencers. Furthermore, modal proclitics can assign the dative to S or A
 \item The accusative marker can be found on P and in rare instances on S
 \item The instrumental marker can be found on S and A when they are institutional
 \item Zero can be found on S, A and P. Zero is never found on R.
\end{itemize}

This distribution can be illustrated as in Figure \ref{fig:argstr}.


\begin{figure}
 \centering
 \includegraphics[height=.3\textheight]{\imgpath sap-slm}
 \caption[Coding of semantic roles in SLM]{The coding of semantic roles in SLM. The instrumental \em =dering \em can code S and A, but is marginal, as seen by the small portion of the circles it covers. The accusative \em =yang \em can be used for P, where it is common, and for S, where it is marginal. This difference in frequency is represented by the different sizes of the covered surface. The dative marker \em =nang \em can be used for S, A and P, as can zero. For expository reasons, dative and zero are not distinguished in the illustration. Zero is much more frequent than the dative in all three roles. R, finally, can only be marked by the dative.}
 \label{fig:argstr}
\end{figure}

\section{Negation}
SLM has a highly developed negation system, which distinguishes predicate type, tense, and syntactic status, as well as a number of lexicalized forms.  The first important distinction is between verbal negators, nonverbal negators, and existential negators. \em Thraa \em is used to negate existence and \em bukang \em is used to negate any other nonverbal predicates. Within the verbal predicates, there are a number of further choices. If the predicate is found in a subordinate clause which would take the infinitive or the conjunctive participle in the affirmative, \em jamà- \em is used to negate this predicate. If the predicate is in another type of clause, the choice of negator depends on the time reference. For past reference, \em thàra- \em is used, for nonpast reference, \em thama-\em. A special case is the negation of the perfect tense, which is \em V thraa\em, instead of the affirmative \em V aada\em.

Adjectives fall into two lexical classes. The larger one is negated by postverbal \em thraa\em. The smaller one is negated by the prefix \em thàrà-\em. This prefix is used for all tenses. This contrasts with the use of the same prefix with verbs, where it encodes past reference.

The lexicalized negations include \trs{thàrboole}{cannot} for \trs{boole}{can} and \trs{thussa}{want not} for \trs{maau}{want}. The other meaning of \em maau, \em `need', is negated by \em thàrkamauvan\em. 

When answering questions in the negative, \em thraa \em is used as well. These facts are summarized in Table \ref{tab:negation}.


\begin{table}
\begin{center}
% use packages: array
\begin{tabular}{r|c|c|c|c|}
 & past & perfect & present & future \\\hline
\hline 
fin. verbal clause & thàrà-V & V thraa &  \multicolumn{2}{|c|}{thama-V}  \\\hline
inf. verbal clause &  \multicolumn{4}{|c|}{jamà}  \\\hline
existential & \multicolumn{4}{|c|}{thraa}  \\\hline
nominal &  \multicolumn{3}{|c|}{bukang} & thama-jaadi/bukang \\\hline
adjectival1 &  \multicolumn{3}{|c|}{ADJ thraa} & thama-ADJ \\\hline
adjectival2 &  \multicolumn{3}{|c|}{thàrà-ADJ} & thama-ADJ \\\hline
circumstantial &  \multicolumn{3}{|c|}{bukang} &  \\\hline
% locational & \multicolumn{4}{|c|}{thraa\footnotemark}  \\\hline
animate locational & thàràduuduk & duuduk thraa & \multicolumn{2}{|c|}{thama-duuduk}\\\hline
~(ka)maau(van) & \multicolumn{4}{|c|}{thàrkamauvan/thussa}  \\\hline
~boole & \multicolumn{4}{|c|}{thàrboole}   \\ \hline
\hline
 constituent neg.  &  \multicolumn{4}{|c|}{bukang} \\
\hline
\end{tabular}
\end{center}
\caption[Negation patterns for various predicate types and tenses]{Negation patterns for various predicate types and tenses.}
\label{tab:negation}
\end{table}

\section{Discourse}
A very salient feature of SLM discourse is the frequent dropping of participants. It is possible to drop any argument of a clause. So, \trs{Sukaasi.}{Gave.} is a full sentence with the meaning `X gave Y to Z', where X, Y, and Z are instantiated with referents from the common ground between speaker and hearer.

Another salient feature is the frequent use of conjunctive participle clauses to indicate sequences of events. In a chain of events, all but the last are marked by the conjunctive participle \em asà-\em. The last event then receives the normal tense marking, thereby establishing  reference time. 


\ea\label{ex:form:asa:clausechain}
\gll Oorang pada \textbf{asà-pìrrang}, derang=nang \textbf{asà-banthu}, siini=jo su-cii\u n\u ggal. \\
 man \textsc{pl} \textsc{cp}-wage.war 3\textsc{pl}=\textsc{dat} \textsc{cp}-help here=\textsc{emph} \textsc{past}-settle\\
`The men, having waged war, having helped them,  settled down right here.' (K051222nar03)
\z



Finally, the distal deictic \em itthu \em can combine with a number of clitics. These combinations are then used as a variety of connectors, e.g. 
\begin{itemize}
 \item \trs{itthu}{\textsc{dist}} + \trs{nang}{\textsc{dat}} = \trs{itthunam}{then}
 \item \trs{itthu}{\textsc{dist}} + \trs{kapang}{when} = \trs{i(tthu)ka(pa)ng}{then} \xref{ex:beyond:link:ikang} 
 \item \trs{itthu}{\textsc{dist}} + \trs{le}{\textsc{addit}} = \trs{itthule}{but} \xref{ex:beyond:link:itthule}
\end{itemize}



\ea\label{ex:beyond:link:ikang}
\ea 
\gll Kandi=ka {\em Malay} {\em mosque}=pe blaakang=ka incayang=pe zihaarath aada. \\ % bf
     Kandi=\textsc{loc} Malay mosque=\textsc{poss} back=\textsc{loc} 3\textsc{sg}.\textsc{polite}=\textsc{poss} shrine exist  \\
    `In Kandy behind the Malay Mosque, there is his shrine.' 
\ex
\gll \textbf{I(tthu)ka(pa)ng} derang=pe sudara pompang=jo aada Hanthane=ka. \\
     then 3\textsc{sg}=\textsc{poss} sibling sister=\textsc{emph} exist Hantane=\textsc{loc} \\
    `Then, his sister is in Hantane.'    (K060108nar02)
    
\z    
\z  



\ea\label{ex:beyond:link:itthule}
\gll Spaaru mlaayu pada arà-oomong \textbf{itthule} mulbar arà-oomong. \\
     some Malay \textsc{pl} \textsc{nonpast}-speak but Tamil \textsc{nonpast}-speak  \\
    `(Only) few Malays speak (Malay), but they speak Tamil.'  (G051222nar04)
\z      



\nocite{Gair1998}