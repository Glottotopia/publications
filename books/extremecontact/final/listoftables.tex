\section*{List of tables}
1.1 Comparison of a prototypical Creole setting with the Sri Lankan setting\\
\\\\
2.1 SLM consonant phonemes\\
2.2 Frequencies of dental and retroflex stops\\
2.3 SLM vowel phonemes\\
2.4 Pronouns\\
2.5 SLM interrogative pronouns\\
2.6 Deictics\\
2.7 SLM numerals\\
2.8 Case marking postpositions\\
2.9 Verbal affixes in SLM\\
2.10 SLM vector verbs and their values\\
2.11 Coordinating clitics\\
2.12 Overview of case frames for predicates of different arity\\
2.13 Negation patterns for various predicate types and tenses\\
\\\\
3.1 Abbreviations of creole names in Figure \ref {fig:bakker1}\\
3.2 Abbreviations and affiliations of non-creoles\\
\\\\ 
5.1 SLM speech communities\\
\\\\
6.1 Malay(ic) Varieties\\
6.2 Features collected by Adelaar 1991\\
6.3 Gil's feature sets for strong correlations for SLM with Maluku\\
6.4 Gil's feature sets for strong correlations for SLM with Java\\
6.5 Origin of SLM lexemes in two studies\\
6.6 Importance of the three possible regions of origin for the lexicon\\
6.7 Comparison of the vocabulary of the different varieties\\
6.8 1st and 2nd person pronouns in a number of varieties\\
6.9 Comparison of pronouns between SLM, Java, and Maluku\\
6.10 Loan words from Sri Lankan languages in SLM\\
6.11 Loanwords from other languages in SLM\\
\\\\
7.1 The nine features typical of the SSA\\
\\\\
8.1 The sources of Sourashtra tense-mood-aspect markers\\
8.2 The sources of Sri Lanka Portuguese tense-mood-aspect markers\\
8.3 Summary of verbal noun functions in the five languages\\
\\\\ 
10.1 The different stages of language contact in Sri Lanka Malay\\
10.2 `Sri Lankan' features in a variety of linguistic domains and the most likely point of their emergence\\
10.3 `Sri Lankan' features in phonology\\
10.4 `Sri Lankan' features in morphology\\
10.5 `Sri Lankan' features in syntax\\
10.6 `Sri Lankan' features in semantics\\
10.7 `Sri Lankan' feature in discourse\\
10.8 `Sri Lankan' lexical features\\
10.9 Stage 0 features\\
10.10 Stage 1 features\\
10.11 Stage 2 features\\
10.12 Stage 3 features\\
10.13 Stage 4 features\\
10.14 Stage 5 features\\
10.15 The three phases of dialect levelling according to  Trudgill 1986\\
