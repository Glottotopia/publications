\chapter[South-South Asian Linguistic Area]{Sri Lankan Languages in the South-South Asia Linguistic Area: Sinhala and Sri Lanka Malay}

\chapterauthor{James W. Gair}{Cornell University}

\section{Introduction and Background}
Murray Emeneau defined a linguistic area as `àn area which includes languages of more than one family but showing traits in common which are found not to belong to the other members of (at least) one of the families'', and South Asia has long been recognized as such an area, beginning with the pioneering work of Jules Bloch.\footnote{This
  paper is based on a presentation given at the Workshop on Language contact in and around the Indian Ocean, University of Amsterdam, November 26, 2009. I would like to express my appreciation the participants and especially to Sebastian Nordhoff, Kees Hengeveld, and Umberto Ansaldo for encouragement and comments.}
 As Emeneau characterized that area in his influential 1956 paper: \nocite{Emeneau1956} ``The end result of the borrowings is that the languages of the two families, Indo-Aryan and Dravidian, seem in more respects more akin to one another than Indo-Aryan does to the other Indo-European languages.'' \citep[16]{Emeneau1956}, and its characteristics were further defined in the work of a number of scholars such as Colin Masica and others.
Within a linguistic area, there always exists the possibility for sub-areas, and within the South Asia one there clearly appears to be a southern subarea (SSLA), which I have attempted elsewhere to begin to enumerate \citep{Gair1994} and will repeat here while proposing some extensions. Geographically, this area coincides essentially with the South and South-Central Dravidian Language area, extended to bordering languages, including some island languages. Within it there exist a number of Non-Dravidian languages, Indo-Aryan and otherwise. Most if not all of its defining features are characteristic of both South and South-Central Dravidian and thus their appearance in the non-Dravidian languages in the sub-area may confidently be claimed to have Southern Dravidian as its ultimate source.\footnote{In
  what follows, I will use the term ``Southern Dravidian'' for both the South and South-Central languages, unless further specified, since the difference is irrelevant here.
}
The island Indo-Aryan languages are Sinhala and Dhivehi (Maldivian), the national languages of Sri Lanka and the Republic of the Maldives respectively. The group also includes the language of the island of Minicoy, where it is known as Mahl (or Mahal), but that is linguistically a dialect of Dhivehi. Together, they form a southern branch of Indo-Aryan, which can be referred to as ``Southern Insular Indo-Aryan'' (SIIA).\footnote{This
  term is an amplification of the useful term ``Insular Indo-Aryan'' originated by Sonia Fritz \citep{Fritz2002}, extended slightly to indicate more fully the geographical location and to emphasize the separation from the mainland IA languages.
}
Sinhala and Dhivehi are now not mutually intelligible, but in addition to sharing the features characteristic of the general South-South-Asian area, they exhibit others specific to SIIA, such as the development of prenasalized stops (``half nasals``) and indefinite affixes derived from the number ``one'', and thus clearly form a sub-family within IA. 
Within the southern subarea there exist other Indo-Aryan languages isolated from their northern kin. These other Indo-Aryan isolates are later comers to the area, by migration or political-cultural importation. Among them are Sourashtra, dealt with by Ian Smith \citep{Smithtv}, centered in Madurai, resulting from migration in the 16\textsuperscript{th} century CE, and Dakkhini Hindi/Urdu, chiefly in Hyderabad, dating from a somewhat earlier time and the result of Islamic rule in areas of the Deccan.\footnote{There
  is at least one other Indo-Aryan isolate in Vaagri Boli, (among other names) spoken by generally semi-nomadic groups in Tamil Nadu, Maharashtra and Karnataka, but I have too little data on it to include it here.
} 
Marathi, bordering the area on the northwest, also requires some mention. Typologically as well as geographically, it is a ``border language'', sharing some features with its Dravidian neighbors sometimes as alternates, but also remaining fundamentally akin to its north Indian Indo-Aryan relatives. Sri Lankan Portuguese, described by Ian Smith \citep{Smithtv}, is another Indo-European isolate, though not, of course, Indo-Aryan. 
Languages of Sri Lanka, by and large, reflect the characteristics, largely syntactic, that characterize the area. The most prominent of these languages is Sinhala, the others being Sri Lankan Tamil, with several dialects, Sri Lanka Portuguese, and Sri Lankan Malay. For Sri Lanka Malay, we now have extensive available information, thanks to a flourishing of research and description by several scholars, including the recent dissertation of Sebastian Nordhoff. In relation to language contact, SL Portuguese and SL Malay are of special interest for at least two reasons. Unlike Sinhala and Dhivehi, as non Indo-Aryan languages, they did not, at the time of importation, share the typological characteristics common to the wider South Asian area, setting aside any prior creolization involving the Indian subcontinent. Also, the time depth of contact is relatively short, compared to Sinhala/Dhivehi, for which contact extended for millennia, and this has led to interesting work concerning the necessary time span for creolization, in Nordhoff as well as in previous work by Ansaldo, Bakker, Paauw, Smith, Slomanson, and others. 
This paper is devoted primarily to Sinhala and SL Malay, with the aim of demonstrating the degree to which the latter has incorporated the features defining the SSLA subarea. Sinhala examples are thus presented essentially for purposes of comparison so as to provide a reference point for the subarea features as they occur in SL Malay. As stated earlier, the defining features of the subarea are essentially if not completely South Dravidian in origin. I have exemplified those features in Dravidian  languages elsewhere (esp. 1994 and 2009) \nocite{Gair1994,Gair2009} and will not repeat that material here. They can, however, be taken as a background assumption throughout to be present as in those languages as well as those exemplified here. Other than Sinhala, the other major contact language for Malay was some variety or varieties of Sri Lankan Tamil\footnote{Sri
  Lankan Tamil itself includes several varieties, and they differ in various respects from the Tamil of the mainland \citep{Suseendirarajah2008}. However, on the basis of available evidence, they all share the areal features here. For the best studied Jaffna variety see especially \citet{GairEtAl1978} and \citet{Suseendirarajah1993}.
} 
and it has been argued to be the primary one. My not including examples from Sri Lankan Tamil and restricting myself here to Sinhala does not imply that the SL Malay features are the result of direct Sinhala influence, and should emphatically not be taken as reflecting a conclusion to that question. It does however, seem clear from SL Malay evidence in the works cited here that it has absorbed features from both Tamil and Sinhala, whatever the relative weighting. I do not include SL Portuguese here partly for reasons of time and space, but also because I do not have extensive enough information at his time.
Peter Bakker \nocite{Bakker2006} has, in fact, raised the question of a Sri Lankan \textit{Sprachbund}, and made a number of interesting important observations particularly regarding these relatively recent comers. However, there is a terminological and empirical question in regard to the term if it implies a sprachbund territorially limited to the island. If we look only at languages resident in the island and extend this to the Maldives and Minicoy, ignoring the subcontinent, the term is apt, but some of the features cited are characteristic of the languages of Southern South Asia generally and others are shared with the wider South Asian area. Thus the proposed sprachbund would in fact constitute at best a sub-area of the South-South Asia one, sketched out below. Also, if Tamil is included, that automatically invokes an extension to South India, unless one restricted it to features of Sri Lankan Tamil not found on the mainland, and that would involve several local varieties distributed geographically and culturally. Thus, in order to justify the existence of such an island-limited sprachbund it would be necessary to find features limited to Sri Lanka, including Sri Lankan Tamil as well as the other languages. This is, of course, a real, if so far unaddressed possibility. Bakker is to be thanked for raising it, and Nordoff, for one (p.c.), is investigating it further. 

\section{The Defining Features}\label{gair:sec:2.2}
Earlier, I referred to the proposed set of nine features that characterized the South-South Asian area. These are given in Table \ref{gair:tab:ninefeatures} below.\footnote{One
  reviewer pointed out that many of these features are not limited to SSLA languages but are commonly found in head final languages. This is indeed true, and not limited to South Asian languages. The point within the universe of discourse here, however, is that they form a constellation that distinguishes them within the larger South Asian environment, while also sharing the common features of that larger area of which they form a part, while noting that the main initial  source of the traits here was the southern Dravidian languages. The same reviewer also raised the interesting observation that many of them are found in languages of the northeast, especially some in Nepal as well as in some Munda languages, stating that (s)he found and  it ``quite intriguing that these traits seem to continue throughout the eastern part of the subcontinent right up to Nepal''. This raises the question as to whether these are separate developments or are in some way connected with SSLA. Given that the traits at issue are indeed not uncommon in head final languages, the former is plausible, and there is the added possibility of a different subarea. It might be added that some of the features, such as the `say' complementizer, are found in Bengali, and in fact Dravidian influence has been posited there (\citet{Klaiman1977}, and see \citet{Bayer2001}).  This invites further research involving the cultural and linguistic history of the area, but that is clearly beyond the confines of the present paper.
} 
It should be recalled that these are as a set defining  features of the South-South Asia Linguistic area, which in turn forms a part of a larger South Asian one, and thus shares those features as well. In the case of Sinhala and Dhivehi, they have not surprisingly been inherited from their mainland source. This is not of course the case for the non-SA imports SL Malay, or SL Portuguese, which  as stated at the beginning were typologically very different from the South Asian ones. My concern in this paper is with the extent to which SL Malay participated in the SSLA sprachbund, and features of that language not specific to that subarea have not been dealt with. In investigating contact induced typological change in SL Malay, however, it is obvious that those more general features are of equal importance to those specified here, since they were all part of the donor languages. As one crucial example of such a vitally important feature, we may note basic SOV and right-headedness, and in fact the very strong right-headedness that is characteristic of SS Asia and underlies several of the features here. In relation to Sri Lankan Malay, one interesting aspect is that this did not extend to the morphological level, though it does pervade the syntax. This also appears to be the case with SL Portuguese, as noted by Ian Smith, who states in relation to SL Portuguese and Sourashtra ``Structurally, Sourashtra was closer to Tamil, again because of a closer starting position; SLP showed more accommodation to Tamil, but had not modified the structure of some closely bound elements such as verbal prefixes'' \citep[408]{Smith2001}.\nocite{Smith2001} This raises some interesting questions as to variable resistance of aspects of the grammar to contact induced change, but that is beyond our concerns here.

% \subsection{South-South Asian Areal Features}
\begin{table}
\begin{tabular}{p{1cm}p{8cm}}
\ref{gair:sec:3.1} & Question marker appears at the end of sentence (postverbal) as the unmarked location, but may also occur on questioned sentence-internal constituents)\\
\ref{gair:sec:3.2} &  Subordinate Clauses marked at the end, by a verbal affix or a conjunctive form of some kind, rather than by initial conjunctions (which are rare or missing altogether except for sentence adverbs)\\
\ref{gair:sec:3.3} & Preposed Relative Clauses (Adjectival Sentences) as the only or main alternative.\\
\ref{gair:sec:3.4} & Correlatives use a WH rather than a correlative form of the Indo-Aryan type and are generally restricted to indefinite or conditional contexts and commonly employ a sentence particle (dubitative or question) on the subordinate clause.\\
\ref{gair:sec:3.5} &  Sentence-Final quotative from `say'\\
\ref{gair:sec:3.6} &  Sentences may be nominalized without genitivization (or deletion) of subject, by employing a sentence-final form or verbal affix.\\
\ref{gair:sec:3.7} &  Focused (Nominal Cleft) Sentences, including those with rightward focus.\\
\ref{gair:sec:3.10} &  Negatives:\\
\textbullet &  Negative varies with type of main clause (Verbal, Equational, Existential).\\
\textbullet & Negative verbs in subordinate clauses.\\
\textbullet &  Cleft sentences negated like nominal equational ones.\\
\ref{gair:sec:3.8} &  Conjunctive participles may occur with overt lexical subjects, not co-indexed with main subject (or agent). [Extent yet to be determined]\\
\hline
\ref{gair:sec:3.9} & A sentence-final reportative or hearsay particle.\\ 
%   & Marking of questioned constituents by the question marker with obligatory clefting.\\
%   &  (Semi-) obligatory clefting of interrogative (WH) forms ('who, what, where,' etc.)\\
\end{tabular}
\caption[The nine features typical of the SSA]{The nine features typical of the SSA, plus one additional feature, with references to the sections where they are discussed.}
\label{gair:tab:ninefeatures}
\end{table}
 
A sentence-final reportative or hearsay particle is found in all of the SSLA languages, but it is not, as far as I have been able to determine, found outside that region  in the northern Indo-Aryan languages, with the exception of Marathi, which has the form \textit{mhaNe}, \citep[19]{Bashir2006} 
and  Marathi is a kind of ``border'' South-South language, sharing some of the SSLA features, but retaining its general northern IA character.

The distributions of these reportative forms with regard to their use on constituents as well as in sentence final position varies across the languages, and in the case of Sinhala at least, it is intimately integrated with the system of focus and clefting \citep{Gair1986,Gair1994,Gair1997,Gair2009}.
Another likely possibility is the existence of coordinating or additive particles or affixes cited by \citet[143]{Bakker2006} such as Tamil --\textit{um}, Sinhala \textit{da}, SL Malay \textit{le(y)}.  SL Portuguese \textit{ta:m} has also  been cited. However, it remains to be established that non-southern Indian languages do not have this feature.
Emphatic particles have also been mentioned for Sri Lanka \citep[as in][142-3]{Bakker2006}, but in this case it appears to be characteristic of the wider South Asia area, with such forms as Hindi \textit{bhii}. 
% There are also features shared by only a subset of the languages that includes languages from both India and Sri Lanka and the Maldives

It should be noted that the all of the features that I have listed are syntactic. The literature on Malay and SL Portuguese, however, presents a number of relevant phonological and morphological features that invite wider investigation. For example, to cite one new contribution, Nordhoff notes the development of prenasalized stops in upcountry SL Malay, an uncommon change that puts it in a set with Sinhala and Dhivehi and clearly indicates Sinhala influence since it is lacking in the other languages. I will not attempt to pursue the non-syntactic features further here, since it represents a separate and in part new investigation, though one that is much to be desired.

\section[Instantiation of the features]{The diagnostic features as evidenced in the Sri Lankan languages, especially Sinhala and SL Malay}
In what follows (\ref{gair:sec:3.1} through \ref{gair:sec:3.10}), I will go through the features individually, illustrating with Sinhala and SL Malay. As stated earlier, my not including Tamil in no way represents a position on the discussion prominent in the literature on the relative extent to which Sinhala and Tamil influence affected the changes in SL Malay that brought it strikingly into alignment with the other SSLA languages. SL Malay language is of special interest to me since I had not considered it earlier as a member of the set, a consideration now made possible by the available scholarship on the language and greatly facilitated by Sebastian Nordhoff's detailed and comprehensive grammar. In what follows, Malay examples can be assumed to be taken from that grammar unless otherwise noted.
% \footnote{In
%   the SL Malay examples, I gave retained Nordhoff's conventions for indicating grammatical categories; thus, for example,  ``PL'' for plural, while the Sinhala examples retain my own notations; i.e.,``pl.''
% } 
Their sources are cited by page and example number.
% Throughout, examples will be numbered consecutively within sections. 

\subsection[Sentence-final question marker]{Question marker at end of sentence (postverbal) as the unmarked location, but may also occur on questioned sentence-internal constituents}\label{gair:sec:3.1} 
This is found in both Sinhala and Sri Lanka Malay, as illustrated in \xref{ex3.1.1} through \xref{ex3.1.4}. 
 


\ea\label{ex3.1.1}Sinhala ``Simple'' Sentence Question
\gll ee minihaa  iiye  gunapaala-{\dott}a salli  dunnaa=da?\\
  that  man  yesterday Gunapala-\textsc{dat} money gave=\textsc{interr}\\
  `Did that man give Gunapala money yesterday?'
\z



\ea\label{ex3.1.2} Sinhala Constituent (Focused) Sentence Question
\gll  ee minihaa=da  iiye  gunapaala-{\dott}a  salli dunn-e?  \\
 that man=\textsc{interr}  yesterday Gunapaala-\textsc{dat} money gave-\textsc{foc}  \\
 `Was it that man who gave Gunapala money yesterday?'
\z




\ea\label{ex3.1.3} SL Malay ``Simple'' Sentence Question
\gll   Sebastian pùddas arà-maakang=si\\
  Sebastian spicy \textsc{nonpast}-eat=\textsc{interr}\\
'  Do you eat spicy food, Sebastian?' (361, ex.339)
\z




\ea\label{ex3.1.4} SL Malay Questioned Constituent
\gll  Daging baabi=si  anà-billi?\\
  meat pork=\textsc{interr}   \textsc{past}-buy\\
  `Did you buy \textsc{PORK}?  (274, ex. 16)
\z


Sinhala has a special property here, in that the questioned constituent with \textit{da} requires the focusing affix on the verb, as part of a general system of focus and clefting. This is also the case in some SSLA Dravidian languages, but not in Tamil. SL Malay does not have a focusing system  employing a focusing form of the verb, so it is not surprising that it lacks this feature \citep{Slomansontv}. However, there is an intriguing pattern that bears a resemblance to Sinhala in relation to the verb affixes that co-occur with constituent questions. This will be dealt with in section \ref{gair:sec:3.7} on focus structures.

\subsection[Clause final complementizers]{Subordinate Clauses marked at the end, by a verbal affix or a conjunctive form of some kind, rather than by initial conjunctions (which are rare or missing altogether except for sentence adverbs)}\label{gair:sec:3.2} 
 One example of this is the conjunctive participle, illustrated in \xref{ex3.2.1} for Sinhala. 


\ea\label{ex3.2.1} 
\gll siri [k{\ae}{\ae}ma kaalaa] gedara giyaa\\
Siri  food eat-\textsc{cp} home  went\\
 `Siri ate and went home.'
\z


SL Malay has a form of this as well, as in \xref{ex3.2.2}:


\ea\label{ex3.2.2} 
\gll Samma oorang school=nang asà-pii arà-blaajar cingalaa\\
 all man school=\textsc{dat} \textsc{cp}-go \textsc{nonpast}-learn Sinhala\\
 `Everybody goes to school and learns Sinhala'  (242, ex.26)
\z


The conjunctive participle as such is a feature shared by the SSLA languages, but it is not a feature marking that subarea. Rather, as has been noted by numerous scholars, it is an areal feature of South Asia as a whole \citep{Emeneau1956,Masica1976}.\footnote{See
  \citet{Subbaraoinpress}, section 7.2 for a general account covering four major families.
} 
That does not, of course diminish its importance as a striking and important development in SL Malay, and like other features aligned with South Asia in general, it was a product of contact with languages within the subarea. 
Though the conjunctive particle as such is not a defining characteristic of the South-South Asia area, it is in those languages one exponent of a general SSLA pattern by which all markers of subordination, including the conjunctive participle, are at the right margin, and may be affixal, clitics, or other subordinating forms. This distinguishes these languages from the North Indian ones, which though SOV, are `mixed' in that regard, having left-marginal conjunctions and complementizers as well as conjunctive participles.\footnote{See
  \citet{Bayer2001} for a description and analysis of ``mixed'', i.e., double complementizer languages in South Asia.
} 
Also, conjunctive participles in SSLA languages are commonly not restricted to a single function, but occur in a variety of roles. In Sinhala, for example, they occur also as a participle in periphrastic perfect constructions, and may even occur as a main verb \citep{Gair2005}.
SL Malay is in this respect in general accord with the overall SSLA pattern. However, it differs from the other languages in the morphological alignment of the subordinating element(s), employing a prefix \textit{asà-} rather than a suffix. As only one other example, of this parallelism in function with a difference in morphological realization, there is an SL Malay temporal prefix \textit{kapang} (also appearing as \textit{kaN}, \textit{kal} and \textit{ka}), illustrated in \xref{ex3.2.3}: 


\ea\label{ex3.2.3} 
\gll Mosque=nang \textbf{kapang-pii} samma ooran=nang go athi-kaasi\\
   mosque=\textsc{dat}  when-go   all man=\textsc{dat} 1s.\textsc{familiar}      \textsc{irr}-give\\
    `When I go to the mosque, I give to everybody'. (386, ex.451)
\z


  This invites comparison to the Sinhala form \textit{ko{\dott}a} `when'', which follows the verb, but has a similar subordinating function and much the same sense:


\ea\label{ex3.2.4} 
\gll  pansala-{\dott}a  yana-ko{\dott}a  mama  h{\ae}ma-ken-eku-{\dott}a  k{\ae}{\ae}ma de-nawaa\\
  temple-\textsc{dat} go-when I every-one-\textsc{indef}-\textsc{dat} food give-\textsc{prs}\\
  `When I go to the temple, I give food to everyone.'
\z

\subsection{Preposed Relative Clauses (Adjectival Sentences or Participial Relatives) as the main or only alternative. }\label{gair:sec:3.3}

Sinhala and Tamil relative expressions are formed by preposing a clause with a relativizing, or adjectival, tensed verb form to the head, and the pattern is essentially the same in the other SSLA languages. 
SL Malay relative clauses, like Sinhala and the other South-South Asian Languages also precede the head, and this is the exclusive pattern \citep[370]{Nordhoff2009}. Thus  \xref{ex3.3.1}:


\ea\label{ex3.3.1} 
\gll  [Seelong=nang dhaataang aada {\O}] Mlaayu oorang ikkang\\
 Ceylon-\textsc{dat}  come exist  {} Malay  man  fish \\
 `The Malays who came to Sri Lanka were fishermen.'  (518, ex.61)
\z


Sentence \xref{ex3.3.1} also includes a perfect formed with an existential verb, another parallel to in Sinhala and Tamil (but also found in the North Indian languages). In Sinhala and Tamil, as well as in in the other SSLA languages, there are no TAM restrictions. Thus for Sinhala, parallel to \xref{ex3.3.1}: 


\ea\label{ex3.3.2} 
\gll  lankaawa-{\dott}a {\ae}wi-lla hi{\dott}a-pu  minissu govita{\ng} k{\E}ra-pu aya.\\
 Lanka-\textsc{dat} come-\textsc{cp} exist$\setmin$\textsc{past}-\textsc{rel}  men cultivation do-\textsc{past.rel}  people.  \\
'The people who came to Lanka were folk engaged in agriculture.'
\z

The nominal head in SL Malay can have a wide range of semantic roles within the clause, and there is no restriction to specific grammatical relations, such as subject or object (Nordhoff 523ff). This is also the case in Sinhala, Tamil, and the other SSLA languages \citep[Section 9.1.2)]{Subbaraoinpress}\footnote{As
  \citep[section 9.5.1.2]{Subbaraoinpress} asserts, any subcategorized element can be relativized in all Dravidian languages [also in Sinhala], though some elements such as non-subcategorized comitatives cannot be.  Such seems to be the case in Sinhala regarding some forms with postpositions. This might be expected from the fact that the head will naturally take the case required by the matrix sentence but it also seems to be a pragmatic function of interpretability. 
} 
in contradistinction to restrictions on preposed participial modifiers in many other languages including the North Indian IA ones such as Hindi. Thus \xref{ex3.3.3}-\xref{ex3.3.5} in Sinhala illustrate heads linked to subject, object and indirect object: 


\ea\label{ex3.3.3} 
\gll  [siri  gunapaala-{\dott}a  dunn-a]  pota\\
  Siri  Gunapala-\textsc{dat} give.\textsc{past}-\textsc{adj}  book \\
   `The book that Siri  gave Gunapala'
\z

\ea\label{ex3.3.4} 
  \em [gunapaala-{\dott}a pota dunna]  siri  \\ % no gll
'(the) Siri who gave the book to Gunapala' \\
\z

\ea\label{ex3.3.5} 
\em  [siri pota dunna]  gunapaala  \\ % no gll
   `(the) Gunapala to whom Siri gave the book'
\z


Note also that there is no change of case on constituents in the relative clause in either language, which occur in the same case that they would have in an independent sentence, and that this includes the subject. Thus there is no genitivization like that in northern IA languages like Hindi, or in some familiar European ones.
For SL Malay, \xref{ex3.3.1} illustrated a subject/agent. Theme/object and recipient are shown in \xref{ex3.3.6} and \xref{ex3.3.7}:
 

\ea\label{ex3.3.6}
\gll  [Kirras  pinthu=nang  arà-thatti  hathu svaara] su-dìnngar\\
strong door=\textsc{dat} \textsc{simult}-hammer \textsc{indef} noise \textsc{past}-hear\\
`They heard a noise of hard hammering at the door.' (441, ex.105)
\z




\ea\label{ex3.3.7} 
\gll  [Se duvith anà-kaasi oorang] su-iilang\\
  1s=\textsc{dat} money \textsc{past}-give  man \textsc{past}-disappeared\\
  `The man I gave money to disappeared.' (525, ex. 93)
\z


As Nordhoff notes, (528), SL Malay differs from Sinhala, Tamil, and other SSLA languages, in that it has not developed a relative participle and there is no overt marking for relativization. Nevertheless, it is clear that the shift of relative clauses to a right-headed pattern is a major typological one, and a striking result of contact with Sri Lankan languages. 
One striking feature of SL Malay not found in Sinhala, Tamil, or the other languages as far as I am aware, is the existence of genuinely headless relatives, i.e., zero or null headed ones, in which there is no overt form in any position representing a head. Nordhoff gives a number of undoubted examples, of which \xref{ex3.3.8} is one:


\ea\label{ex3.3.8} 
\gll  [Lorang anà-maasak {\O}] eenak\\
  2\textsc{pl}  \textsc{past}-cook {} tasty\\
  `What you cooked is tasty.' (453, ex. 163)
\z


Sinhala does not have this structure, nor as far as I am aware, do any of the Dravidian languages, Rather, the equivalent would have a pronominal form expressed, as in \xref{ex3.3.9}:


\ea\label{ex3.3.9} 
\gll   oyaa  iww-a  eewa  raha-yi\\
  you cook{$\backslash$}\textsc{past}-\textsc{rel} 3\textsc{pl}.\textsc{pron} \textsc{inanim} tasty-\textsc{pred} \\
  `What (`Those) you cooked is tasty.'
\z


The plural pronominal in \xref{ex3.3.9}, \textit{eewa}, can have indefinite or mass reference, but could also be specifically referential, so that the meaning could be `Those that you cooked are tasty.'
In Sinhala, the pronominal form is also marked for number, and if one thing is specifically referred to, a singular pronoun such as \textit{eeka} `that one' or \textit{eka}, the inanimate number `one', can occur as head, as can \textit{deka} `two', or any other number  under the appropriate circumstances:


\ea\label{ex3.3.10} 
\gll  oyaa iww-a  eeka   rahayi\\
  you cook{$\backslash$}\textsc{past}-\textsc{rel} 3\textsc{sg}.\textsc{pron} tasty-\textsc{pred} \\
  `That one you cooked is/was tasty.'
\z
 


\ea\label{ex3.3.11} 
\gll oyaa iww-a  eka   rahayi\\
  you cook{$\backslash$}\textsc{past}-\textsc{rel} 3\textsc{sg}.\textsc{pron} tasty-\textsc{pred} \\
  `The one you cooked is/was  tasty.\footnotemark'
\z\footnotetext{Clauses
  with \textit{eka} can lead to ambiguity, since action nominalized clauses of this form are of the same shape, but do not necessarily have a gap co-indexed with \textit{eka}. Thus \xref{ex3.3.10} could also have the meaning `your cooking (the activity) is tasty' which is ruled out pragmatically in this case. Cf. Exx. (\ref{ex3.6.6}-\ref{ex3.6.7}).
}

 Number can also be expressed with SL Malay headless clauses. As Nordhoff remarks, ``NPs based on headless relative clauses can only be modified by the plural marker \textit{pada}..., but by nothing else.'' (396). This is illustrated in \xref{ex3.3.11}:


\ea\label{ex3.3.12} 
\gll  [Seelon=nang anà-dhaatang {\O}] pada mlaayu pada.\\
  Ceylon=\textsc{dat} \textsc{past}-come {}  \textsc{pl} Malay  \textsc{pl}\\
  `Those who had come to Ceylon were the Malays'  (454, ex. 166)
\z


 The parallel here is interesting, and especially the occurrence of \textit{pada} attached to the entire clause, but apparently relating to the null element.

\subsection[Correlatives with interrogative+clitic]{Correlatives use a WH rather than a correlative form of the Indo-Aryan type and are generally restricted to indefinite or conditional contexts and commonly occur with a sentence particle (dubitative or question) on the subordinate clause.}\label{gair:sec:3.4}
 This is apparently not a feature of SLM, which is not surprising, since the model is not found in either of the Tamil or Sinhala adstrates in the varieties to which the Malay speakers would have been exposed.
 Sinhala does have a correlative structure, but it is characteristic of the Literary variety and has features like the Dravidian languages. It uses a correlative pronoun derived from the Old (and Middle) Indo Aryan \textit{ya-}  forms, rather than an interrogative as in Dravidian, but like Southern Dravidian it is generally confined to the ``who-ever, what-ever'' sense, and the correlative clause occurs with the question particle \textit{da} or with the conditional (``if'') form \textit{nam}.

\subsection{Sentence-Final quotative from ``say''}\label{gair:sec:3.5}

The SSLA languages employ a sentence/clause final quotative affix or particle to mark indirect or direct speech as well as related functions such as thought or intention. It is characteristically derived from `say' but in a frozen form or one from a `say' verb no longer in active use, as with Tamil \textit{enRu/(e)NNa}. In Sinhala, the quotative is \textit{kiyalaa}, which is homonymous with the conjunctive particle of \textit{kiyanawaa,} but it has a separate range of distribution specific to the quotative use. That range is partially exemplified in \xref{ex3.5.1}, \xref{ex3.5.2}, and \xref{ex3.5.3}:


\ea\label{ex3.5.1} 
\gll  
{}[siri  iiye  aawaa  kiyalaa]  gunapaala kiwwa\\
   Siri yesterday came  \textsc{quot}  Gunapala  say{$\backslash$}\textsc{past} \\
   `Gunapala said that Siri came yesterday.'
\z




\ea\label{ex3.5.2} 
\gll  ma-{\dott}a siri kiyalaa kiya-nawaa\\
I-\textsc{dat} Siri \textsc{quot} call-\textsc{prs} \\
'They call me ``Siri''.'
\z




\ea\label{ex3.5.3} 
\gll raksaawa hoyaa-ganna kiyalaa kola{\umb}a-{\dott}a giyaa.\\
  job  seek-find  \textsc{quot} Colombo-\textsc{dat} go-\textsc{past} \\
  (I) went to Colombo in order to find a job.' (i.e., with the  intention)
\z
 
SL Malay has an equivalent form, with much the same range of functions, partially illustrated in \xref{ex3.5.4}, \xref{ex3.5.5}, and \xref{ex3.5.6} It is not derived from a ``say'' verb as such, but rather from a Malay word for ``word'' but it is clearly modeled on the Sinhala/Tamil form  (Nordhoff 396). \xref{ex3.5.4} through \xref{ex3.5.6} are examples:



\ea\label{ex3.5.4} 
\gll  Se=pe oorang thuuva pada anà-biilang [kitham pada   Malaysia=dering anà-dhaathang] katha\\
Is=\textsc{poss} man old \textsc{pl} \textsc{past}-say  1\textsc{pl}    \textsc{pl} Malaysia=\textsc{abl} \textsc{past}-come \textsc{quot}  \\
'My ancestors told me that we had come from Malaysia.' (401, ex 465)
\z


 


\ea\label{ex3.5.5} 
\gll  [Aashik=nang hathu soldier mà-jaadi suuka]=si katha  arà-caanya. \\
 Ashik=\textsc{dat} \textsc{indef} soldier \textsc{inf}-become like=\textsc{interr} \textsc{quot} \textsc{nonpast}-ask\\
 `He asks if you want to become a soldier, Ashik.' (393, ex..480)
\z




\ea\label{ex3.5.6} 
\gll   See=yang  Tony katha arà-panggel\\
 I=\textsc{acc}  Tony   \textsc{quot}  \textsc{nonpast}-call\\
 `I am called ``Tony''.'  (390, ex. 466)
\z


\subsection[Nominalization without genitives]{Sentences may be nominalized without genitivization (or deletion) of subject, by employing a sentence-final form or verbal affix.}\label{gair:sec:3.6}
In the Southern Dravidian languages, the general pattern has a relative clause of the type discussed in in \ref{gair:sec:3.6} headed by a pronominal form, characteristically 3\textsuperscript{rd}. person neuter or some other non-agreeing form, as in \xref{ex3.6.1} from Tamil, where the pronominal form appears as affixed to the relativizing form of the verb.



\ea\label{ex3.6.1} 
\gll  \textbf{avan} \textbf{va-nt-atu}  nallatu\\
  he  came-\textsc{rel}-\textsc{nmlz} good\\
  `It's good that he came.'
\z


In essence, the nominalizing form nominalizes the entire sentence, producing an action nominal. These differ from relative clauses modifying a pronominal head in that there is no necessary gap within the sentence; i.e., they are complete, aside from the ever-present possibility of ellipsis in these languages. However, if the sentence does include a gap (i.e., a null pro-form), it will not be co-indexed with the head, but will generally have external reference.
In present day Sinhala, the pronominal form is \textit{eka}, the inanimate numeral `one', as in \xref{ex3.6.2} and \xref{ex3.6.3}. \textit{Eka} is also used to adapt foreign, especially English, loans, as in \textit{kaar-eka} 'the car'.

\ea\label{ex3.6.2} 
\gll [Silva mahattayaa ma-{\dott}a eeka kiww-a-eka] {\ae}tta \\
Silva  gentleman  I-\textsc{dat} that  say$\setmin$\textsc{past}-\textsc{adj}-\textsc{eka} truth\\
    `It is true that Mister Silva said that to me.'
\z

\ea\label{ex3.6.3} 
\gll \textit{eyaa} \textit{aaw-a-eka}  ho{\und}a-yi.\\
  he  come$\setmin$past-\textsc{adj}-\textit{eka}  good-\textsc{pred} \\
  `It is good that he came.'
\z


In all of these, the entire sentence is intact and in the same form as an independent equivalent except for the relativizing (adjectival) verbal affix and the pronominal form or \textit{eka}. There is no change in internal case marking. Thus in \xref{ex3.6.2} and \xref{ex3.6.3}, the subjects are in the nominative, as in an independent sentence equivalent. Thus also, in \xref{ex3.6.4} the dative case marking of the subject, required by the verb \textit{teerenawaa}, is retained as is that of the direct object  (required by the verb) in \xref{ex3.6.5}:



\ea\label{ex3.6.4} 
\gll  lamayaa-{\dott}a teer-un-a-eka ho{\und}a-yi.\\
  child-\textsc{dat} understand-\textsc{past}-\textsc{adj}-\textit{eka} good-\textsc{pred} \\
  `It is good that the child understood.'
\z



\ea\label{ex3.6.5} 
\gll   [noonaa  lamayaa-{\dott}a  banin-a-eka]  puduma-yi\\
  lady  child-\textsc{dat} scold-\textsc{adj}-\textit{eka}  surprising-\textsc{pred} \\
  `It is surprising that the lady scolded the child.'
\z


\textit{Eka} can also function as a co-indexed pronoun, so that given the possibility of null pronouns in these languages there is the possibility of homonymous sentences, as in \xref{ex3.6.6}, in which \textit{eka} is the co-indexed neuter singular relative clause head co-indexed with a gap, but which can also have the structure and reading in \xref{ex3.6.7}, in which \textit{eka} is  the sentence nominalizer  and the direct object is represented by a null indefinite pronoun.


\ea\label{ex3.6.6} 
\gll  gunapaala ammaa-{\dott}a ({\O})\textsubscript{i} dunn-a-eka\textsubscript{i}  ma{\ng} d{\ae}kkaa\\
  Gunapala  mother-\textsc{dat} {} give{$\setmin$}\textsc{past}-\textsc{adj}-one.\textsc{inanim} I saw\\
  `I saw what (the one) Gunapala gave mother.'
\z


or:


\ea\label{ex3.6.7} 
\gll gunapaala ammaa-{\dott}a ({\O})\textsubscript{indef} dunn-a-eka  ma{\ng} d{\ae}kkaa\\
  gunapaala mother-\textsc{dat}  {}  give{$\setmin$}\textsc{past}-\textsc{adj}-\textit{eka}  I saw\\
  `I saw that Gunapala gave (something) to mother.'
\z


 Malay nominalized sentences exist, but they are simpler than those in Sinhala and the other SSLA languages, since they lack any overt nominalizing element. As Nordhoff reports: ``in SLM, clauses can be used as noun phrases as they are. No further morphological flagging of this use is necessary'' (450). 
 They may also occur with case postpositions, as in \xref{ex3.6.8}:


\ea\label{ex3.6.8} 
\gll Suda butthul suuka [[asà-dhaatang]$_{CLS}$]$_{NP}$=nang.\\
  this very  like \textsc{cp}-come=\textsc{dat}\\
  `So, I am very pleased that you have come.'  (452, ex.158)
\z


 In \xref{ex3.6.8} above, the verb was in the conjunctive participle form, but an uninflected verb is possible \xref{ex3.6.9}:


\ea\label{ex3.6.9} 
\gll [Manis-an maakang]\textsubscript{CLS}]\textsubscript{NP}]=nang go suuka bannyak.\\
  Sweet-\textsc{nmlz} eat=\textsc{dat}  1\textsc{s.familiar} like much\\
  `I like very much to eat sweets.'  (452, ex.159)
\z


 In \xref{ex3.6.8} and \xref{ex3.6.9}, there is no overt subject in the nominalized clause, but SL Malay, as in the other SSLA languages, the subject may be overt, with no change of case, as in  \xref{ex3.6.10}:


\ea\label{ex3.6.10} 
\gll  kitham=pe baapa su-biilang [[lorang suurath=yang  mlaayu=dering anà-thuulis]\textsubscript{CLAUSE}=nang bannyak arà-suuka].\\
  1\textsc{pl}=\textsc{poss} father \textsc{past}-say 2\textsc{pl}  letter=\textsc{acc} Malay=\textsc{abl}    \textsc{past}-write=\textsc{dat} much \textsc{simult}-like\\
    `Daddy said that he liked very much that you wrote the letter in Malay.'  (450, ex.150)
\z


 SL Malay does have a nominalizing suffix \textit{-an}, but this appears primarily to serve to create nominal forms from verbs, such as the noun \textit{ajar-an} `teach(ing)  from the verb \textit{aajar} `teach', and not to nominalize sentences \xref{ex3.6.11}.


\ea\label{ex3.6.11} 
\gll  Lorang=nang see=yang ingath-an=si?\\
  2\textsc{pl}=\textsc{dat}  1s=\textsc{acc}  think-\textsc{nmlz}=\textsc{interr}\\
  `Do you have thoughts on me/are you thinking of me?   (512, ex. 36)
\z


 As pointed out earlier in Section \ref{gair:sec:3.3}, SL Malay has true headless relatives, in which there is no form expressing a head, and these  could be considered, from their distribution and ability to occur with case, to be a kind of nominalized sentence.

\subsection{Focused (Nominal Cleft) Sentences, including those  with rightward focus}\label{gair:sec:3.7}
 The basic pattern for the Southern Dravidian languages is for the presupposition to be expressed in a nominalized clause headed by a pronominal form, characteristically 3rd person neuter.\footnote{While
  the general pattern is clear, there are variations across languages. Thus Tulu and Dhivehi utilize focus affixes on the verb, but they are not clearly pronominal \citep{Somashekar1999,CainEtAl2000}. Such focused sentences are also found in Dakkhini Hindi, and Sri Lanka Portuguese shows a form of this structure as well.
}
 It thus resembles or is identical to a nominalized sentence of the kind described in Section \ref{gair:sec:3.6}, but will contain a gap linked to the focused element. The focused element commonly follows the presupposition, but different languages may also allow focus \textit{in} \textit{situ} or in other orders.\footnote{Where \label{gair:fn:12}
  in some of the languages including Sinhala, the focused element occurs internally, as in situ, there will of course not be a gap co-indexed with a rightward form, but the focused element is marked in some fashion, commonly a clitic such as an interrogative or emphatic one.
}  
The basic pattern for rightward focus is thus as in \xref{ex3.7.1}, where the content of \textsc{nmlz} varies from language to language and  where XP\textsubscript{i} represents the focused element. It is exemplified for Sri Lankan (Jaffna) Tamil in \xref{ex3.7.2}:


\ea \label{ex3.7.1} 
  {}[\-\textsubscript{S}[\textsubscript{S}{\dots}{\O}\textsubscript{i} \dots \textsc{v}-\textsc{tns}-\textsc{rel}-\textsc{nmlz}]  XP\textsubscript{i} ] 
\z



\ea\label{ex3.7.2} 
\gll  naan poonatu  yaa{\lbar}ppaan{\dotn}at-ukku\\
  I  go-\textsc{past}-\textsc{nmlz} Jaffna-\textsc{dat} \\
  `It was to Jaffna that I went.'  (Sri Lankan Tamil)
\z

 The pattern was borrowed into Sinhala at least by the eighth or ninth century, and was subsequently elaborated to become an integral feature that intersects with other patterns such as negation and both WH and yes-no interrogatives, and it is of very frequent occurrence in discourse.\footnote{I
  have dealt with this at length in several places, beginning with \citet{Gair1970}, For the history, see especially \citet{Gair1986}, and I will not repeat that here. A similar elaboration of the interaction with structures such as WH questions occurred apparently independently in Malayalam, but not in Tamil.}  In Sinhala, the verb is not marked by a pronominal form, but by a focusing (sometimes called ``Emphatic``) affix \textit{e(e).} That affix, however, does indeed derive historically from a 3\textsuperscript{rd} person masculine/neuter pronominal affix, although it is now specialized for focus/clefts and one kind of negation. Colloquial Sinhala examples are given in \xref{ex3.7.3} and \xref{ex3.7.4}. As \xref{ex3.7.5} shows, Sinhala allows items to be focused \textit{in} \textit{situ}, especially when marked by one of a set of focus-inducing forms, including the question particle and the reportative clitic \textit{lu} described in Section \ref{gair:sec:3.9} below among others. This characteristic is shared to a varying degree by some of the SSLA languages, and was referred to in Section \ref{gair:sec:2.2} above as such.


\ea\label{ex3.7.3} 
\gll mama giy-ee gama-{\dott}a \\
  1s go$\setmin$\textsc{past}-\textsc{foc} village-\textsc{dat} \\
  `It is to the village that I went.'
\z




\ea\label{ex3.7.4} 
\gll mama kiyeww-e \textbf{ee} \textbf{pota} \\
  I read$\setmin$\textsc{past}-\textsc{foc} that book \\
   `It was that book that I read.'
\z




\ea\label{ex3.7.5} 
\gll  eyaa \textbf{he{\dott}a}=\textbf{da} kola{\umb}a yan-ne?\\
  (s)he tomorrow=\textsc{interr} Colombo go.\textsc{prs}-\textsc{foc} \\
  `Is it tomorrow that (s)he is going to Colombo?'
\z


Despite the extensive information provided by Nordhoff, and his inclusion of work specifically on focus, it is not yet clear to me whether focusing sentences of this general type or a variant are present in SL Malay, but there some interesting and suggestive hints as to the possibility that this might be the case.
SL Malay constituents can be focused by attaching the emphatic clitic \textit{jo}, as in \xref{ex3.7.6}  (Nordhoff 379).


\ea\label{ex3.7.6} 
\gll [itthu katha]\textsubscript{UTT}=jo Mahinda arà-biilang.\\
  \textsc{dist} \textsc{quot}=\textsc{emph} Mahinda \textsc{nonpast}-say\\
   `That's what Mahinda [Rajapaksa, President of Sri Lanka] is saying.'  (379, ex.420)
\z


A Sinhala equivalent could be formed using the emphatic form \textit{tamayi}, which is one of several forms requiring clefting, i.e., the focusing affix \textit{e(e)} on the verb:\footnote{\textit{tamayi}
  itself is a complex form composed of an emphatic/reflexive form \textit{tama} and the predicate/emphatic marker -\textit{yi}. The former does not induce the focus affix on the verb, the latter does when occurring on a constituent other than the verb.
}


\ea\label{ex3.7.7} 
\gll  eeka tamayi mahinda kiyan-ne\\
  that \textsc{emph} Mahinda say.\textsc{prs}-\textsc{foc} \\
  `That's what Mahinda is saying.'
\z


One would naturally not expect similar verb marking in SL Malay, since it lacks the verb marking form. Tamil would not mark the verb for cleft function in a parallel sentence, but would simply use the emphatic/reflexive clitic \textit{taan} unless special focusing was required.\footnote{Sinhala
  has another emphatic clitic \textit{-ma}, which does not induce the verb marking, but would have a different sense, emphasizing the identity of Mahinda, rather than his participant function in the action. SL Malay \textit{-jo} seems to encompass that function as well.} 
 Although the focusing verb form is not found in SL Malay, as Nordhoff points out (p.275) there is a intriguing pattern of co-occurrence of past affixes with focus that is reminiscent of the Sinhala situation. There are two past affixes \textit{su-} and \textit{anà-}. The differences between them are subtle, and they are interchangeable in many contexts (ibid). However, in past questions where a constituent is emphasized by the question particle, (Nordhoff's polar questions) \textit{su-} is ruled out, as in \xref{ex3.7.8}: 


\ea\label{ex3.7.8} 
\gll  \textbf{Daging} \textbf{baabi=si} *su/anà-billi???\\
  meat pork=\textsc{interr} *su/anà-buy\\
  `Did you buy \textsc{PORK}? (274, ex.16)
\z


As Nordhoff also points out, the occurrence of the question particle \textit{si }here rules out the possibility of \textit{su}-, but \textit{si} can occur in simple past questions:


\ea\label{ex3.7.9} 
\gll   Itthu binaathan lorang=yang  anà-/su-giigith=si\\
  \textsc{dist} animal 2\textsc{pl}=\textsc{acc} \textsc{past}/\textsc{past}-bite=\textsc{interr}\\
  `Did that animal bite you?' (275, ex.19)
\z


This non-occurrence of \textit{su-} also to be the case with Content (Wh) questions (3.7.10), which require the focusing affix in Sinhala:


\ea\label{ex3.7.10} 
\gll  mana binaathang lorang=yang anà/*su-giigith?\\
  which animal 2\textsc{pl}=\textsc{acc} past-bite\\
  `Which animal bit you?' (274, ex.17)\footnotemark
\z
\footnotetext{The
  difference in the final consonant in {\em binaathan(g)} between this and other examples (cf. \xref{ex3.7.10}) may be due to a typo.
}

 Interestingly, unlike Sinhala, the question particle apparently does not occur on WH forms in SL Malay, and this as also the case in Tamil.
  In the pluperfect, however, \textit{su-} is the only form that can be used, and \textit{anà }is ruled out. (Ibid.)
Nordhoff notes also that it is possible that \textit{anà }could have a specification for clauses with argument focus, since it often occurs with the emphatic marker \textit{jo}, as in \xref{ex3.7.11}: (Ibid)
 

\ea\label{ex3.7.11}
\gll TV=ka=jo anà-kuthumung\\
  TV=\textsc{loc}=\textsc{emph} \textsc{past}-see\\
  `It was on \textsc{TV} that we saw it.' (275, ex.21)
\z


Nordhoff states:
\begin{quote}
 This distinction according to information structure would also make sense from a contact language perspective, since Sinhala has an `emphatic' verb form used in focal contexts \citep{Gair1985calque}
 \end{quote}
 However, as he goes on to say:
\begin{quote}
 Things seem to be more complex in SLM than in Sinhala, though. While in Sinhala, the use of the emphatic [our `focusing'] form [of the verb] is obligatory in focal contexts, this is not the case with \textit{anà-} in SLM. True, most argument focus constructions with past reference have \textit{anà-}, but there are some examples where we find \textit{su-} as well''. (p.240)
\end{quote}

At the very least, this invites further investigation and explanation.
Nordhoff also states that focus can also be indicated by rightwards extraposition, as in \xref{ex3.7.12}:


\ea\label{ex3.7.12} 
\gll Itthu=nang blaakang su-dhaatang [Hambanthota mlaayu pada]\\
  \textsc{dist}-\textsc{dat} after  \textsc{past}-come Hambantota Malay  \textsc{pl}\\
  `After that came the Hambantota Malays.'  (691, ex. 29)
\z


 \xref{ex3.7.13} and \xref{ex3.7.14} seem to be similar, though they were not singled out as examples of focus:


\ea\label{ex3.7.13} 
\gll [Seelon=nang anà-dhaatang {\O}] pada mlaayu pada.\\
  Ceylon=\textsc{dat} \textsc{past}-come {}  \textsc{pl} Malay  \textsc{pl}\\
  `Those who had come to Ceylon were the Malays'  (454, ex. 166)
\z




\ea\label{ex3.7.14} 
\gll Itthu vakthu [kithang=nang nya-aada {\O}] asàdhaathang ini JVP katha hathu problem\\
 \textsc{dist} Time 1\textsc{pl}=\textsc{dat}  \textsc{past}-exist {}] \textsc{copula} \textsc{prox} JVP \textsc{quot} \textsc{indef} problem\\
 `What we had at that time was the so-called JVP problem.' (454, ex. 168)
\z


 As mentioned in Section \ref{gair:sec:3.6} above, the form of the presupposition in Dravidian clefts is essentially that of an action nominalized sentence of the kind described in \ref{gair:sec:3.6},  
but it includes a gap (null item) co-indexed with the focused element. SL Malay nominalized sentences and relative expressions do not, however, have overt morphological marking.\footnote{See
  sections \ref{gair:sec:3.3} and \ref{gair:sec:3.6}. \citet[147]{Slomanson2006cll} states that Colombo SL Malay has a suffix -\textit{nya}, which nominalizes sentences, as in (his 21b) 
  \ea 
  \gll [Ali ara-pi-nya] sE ara-liyat\\ 
   Ali \textsc{prs}-go-\textsc{nmlz} 1s \textsc{prs}-see\\
  `I see Ali going.'
  \z
  This does in fact resemble Sinhala and Tamil. I cannot find this form in Nordhoff, except as a dative, as in ex. 110, p.531, and as an accusative, in the same example, however, so that it may represent a difference in dialect. It may also be an allomorph of a different form. See \citet[529]{Nordhoff2009} on this form.
} 
Thus if there were to be developed a cleft/focus construction parallel to that of Sinhala and the other languages as a result of contact, we could not expect it to have the same morphological characteristics, given their absence in the language. SL Malay does, however have true headless relatives. Thus if we were to speculate, the existence and nature of these headless relatives would seem to make them a likely candidate for expressing presuppositions, since they have an empty slot, parallel to the gap in the Sinhala and Tamil presuppositions (the {\O} in examples \xref{ex3.7.13} and \xref{ex3.7.14}). Indeed, example \xref{ex3.7.12}, presented by Nordhoff as an example of focus, does seem to be analyzable as containing such a clause, and correspondingly, \xref{ex3.7.13} and \xref{ex3.7.14}, which exemplified headless relatives, seem to involve something like the focus in \xref{ex3.7.12}. Thus we might very tentatively conclude that  cleft / focused structures on the SSLA pattern do exist in that language, even if in inchoate form. Establishing this would require investigation of their role in discourse and relation to other sentence types, but in any case, SL Malay seems to have all of the building blocks for such a development to take place.
 In a footnote (p.690, fn.1), Nordhoff points to the existence of one speaker who has a kind of pseudo-cleft construction involving the copula \textit{a(bbi)sdhaatang}, which is shown in the following two examples \xref{ex3.7.15} and \xref{ex3.7.16}.


\ea\label{ex3.7.15} 
\gll See anà-pass.out] abbisdhaathang University of Peradeniya=ka.\\
1s \textsc{past}-graduate \textsc{copula} University of Peradeniya=\textsc{loc}\\
Where I graduated was the University of Peradeniya.' (690, fn. 1)
\z




\ea\label{ex3.7.16} 
\ea
\gll [Itthu arà-kirja]  abbisdhaatang,\\
  \textsc{dist} \textsc{nonpast}-make \textsc{copula}\\
  `How you make it is'
\ex
\gll thullor asà-ambel=apa baaye=nang asà-puukul=apa\\
   egg \textsc{cp}-take=after  good-\textsc{dat}  \textsc{cp}-hit=after\\
   `You take eggs and beat them well, and...'  (690, fn. 2)
\z
\z


These really do resemble Sinhala clefts, though the role of \textit{abbisdhaathang}, not clearly having a counterpart there, is puzzling.  Nordhoff remarks that the construction seems to be idiolectal, but in any case it shows that the development of a SSLA type cleft/focus structure under apparently obvious contact is a real possibility, and in part realized, at least in this one case.
 Clearly this is another inviting topic for further exploration.

\subsection{Negation} \label{gair:sec:3.10}
 There are a number of points in which SL Malay negative distribution bears resemblance to Sinhala, as well as to Tamil. The details are complex, given that each of the languages has a complex system of negation involving several forms and their interaction with other factors such as sentence type, aspect, and, tense, a well as with independent versus subordinate status (for SL Malay, see the helpful survey and table given by \citet{Nordhoffgrammartv}). A full investigation would also require a consideration of Tamil and the other Dravidian languages that time considerations did not allow me to undertake, so I shall not take this up in detail here, but save it for later and separate treatment. We might note, however a few points in passing. 
 In both Sinhala and SL Malay, negation varies with the type of clause, and there are different negators for verbal, equational, attributive and existential predicates. In Sinhala, NP predicates negate with \textit{nemeyi} (also \textit{nevi} and \textit{neveyi}), as in \xref{ex3.10.1}, and in SL Malay with \textit{bukang}, as in \xref{ex3.10.2}.


\ea\label{ex3.10.1} 
\gll mee pota puskola potak nemeyi.\\
 this book ola.leaf book \textsc{neg.nonv} \\
 `This is not an ola leaf manuscript.'
\z




\ea\label{ex3.10.2} 
\gll Deram Islam oorang bukang (SL Malay)\\
 3\textsc{pl} Islam man \textsc{neg.nonv}\\
 `They were not Muslims.' (259, ex.406)
\z


 Adjectival predicates in Sinhala negate with \textit{n{\ae}{\ae}}, as in \xref{ex3.10.3}, and in SL Malay with \textit{thraa,} as in \xref{ex3.10.4}, or the prefix  \textit{thàrà-}, as in \xref{ex3.10.5}. 


\ea\label{ex3.10.3} 
\gll mee  pota  ho{\und}a n{\ae}{\ae} (Sinhala)\\
 This book.\textsc{def}  good  \textsc{neg} \\
 `This book is not good.' 
\z



\ea\label{ex3.10.4} 
\gll Itthu muusing gampang thraa. (SL Malay)\\
 \textsc{dist} time easy \textsc{neg}\\
 `It was not easy back then.' (258, ex.302)
\z




\ea\label{ex3.10.5} 
\gll Itthu thàrà-baae\\
 this \textsc{neg}-good.' \\
 `This is not good.'  (297, ex.99)
\z


 Nordhoff (p.224) remarks that ``The exact conditions that trigger one or the other adjectival negation [in SL Malay] are unclear.``
 Nordhoff correctly notes, in a footnote (p.221, fn 27), that Sinhala has two negators with similar uses to SL Malay \textit{thraa} and \textit{bukang}, further noting that Indian Tamil has lost the distinction though it is retained in Jaffna Tamil. It would be interesting to find whether the distinction holds in any other variety of Sri Lanka Tamil that might have been in contact with SL Malay.
Independent non-focused Sinhala verbal sentences in all tenses, negate with \textit{n{\ae}{\ae}}: 
 

\ea\label{ex3.10.6}
\gll mama ada w{\ae}{\dotd}a k{\E}ranne n{\ae}{\ae}\\
  I today work do \textsc{neg} \\
  `I am not working today',
\z


 In SL Malay, the negation of verbal sentences varies with tense and aspect. In the non-past, the form is \textit{thuma-} or \textit{thumau-},  \xref{ex3.10.7} and in the past, \textit{thàrà-} \xref{ex3.10.8}:


\ea\label{ex3.10.7} 
\gll Go kaapang=le saala thamau-gijja.\\
 1\textsc{s.familiar} when-\textsc{addit} wrong \textsc{neg.nonpast}-do\\
 `I never do wrong.'   (299, ex. 107)
\z


 

\ea\label{ex3.10.8}
\gll Mister Yussuf thàrà-siggar=le\\
 Mister  Yussuf \textsc{neg}-well=\textsc{addit}\\
 `Mister Yussuf was also unwell.'  (297, ex.100)
\z


This is not the case in Sinhala.
 There are other points of resemblance, such as that existential sentences negate by replacing the existential verb with a negative form, and that the perfect also negates by replacing the existential light verb, but I will not pursue those here, except for one set to be dealt with below in connection with focus and constituent negation.
 The list of SSLA features also included the appearance of negative verbs in subordinate clauses.  Sinhala has a negative verbal prefix \textit{no-}, which may be used with non-finite or dependent verbs, including the focusing verb form and the finite verb if it is in an embedded sentence. Comparison with SL Tamil or Malayalam in regard to the negative verb or prefix is especially  difficult owing to the complexity of all of the  systems, and I will not pursue it here, but the occurrence of the \textit{no}-  prefix on focusing verbs in Sinhala relates to the negation of cleft/focusing sentences, and it is illustrated in \xref{ex3.10.9}. The effect is essentially to exclude the focused element from the positive equivalent of the presupposition. Sinhala focused sentences can also negate using \textit{n{\ae}tte}, a focusing form of \textit{n{\ae}{\ae}}, as in \xref{ex3.10.10}.\footnote{Sinhala
  negation is a rich subsystem involving several forms with complex distributional characteristics. An account of the forms and semantics is given in \citet{DeAbrew1981}, and some elaboration is presented in \citet{FoleyEtAl1993}.}


\ea\label{ex3.10.9} 
\gll mama no-ka-nn-e  harakmas witara-yi\\
 I  \textsc{neg}-eat-pres-foc beef only-\textsc{pred}\\
 `It's only beef that I don't eat.'
\z




\ea\label{ex3.10.10} 
\gll minihaa  ya-nn-e n{\ae}-tte gam{\E}-{\dott}{\E}\\
 man  go-\textsc{foc} not-\textsc{foc}  village-\textsc{foc}\\
 `It's to the village that the man does not go.' \citep{DeAbrew1981}
\z


 Sentences negating the focus i.e, clefted sentences with constituent negation, negate with with \textit{neveyi/nemeyi}, as in \xref{ex3.10.11}. 


\ea\label{ex3.10.11} 
\gll iiye gunapaala-{\dott}a salli dunne ee minihaa nemeyi\\
  yesterday Gunapala-\textsc{dat} money gave-\textsc{foc} that man \textsc{neg.nonv}\\
 `It was not that man who gave Gunapala money yesterday.'
\z


Nominal predicate sentences also negate with \textit{nemeyi}, as stated earlier:


\ea\label{ex3.10.12} 
\gll mee pota puskola potak nemeyi\\
 this book olaleaf book-\textsc{indef} \textsc{neg.nonv} \\
 `This is not an ola leaf manuscript.'(=\ref{ex3.10.1})
\z


Here, there is a striking similarity with SL Malay. Nordhoff states (p.588) that \textit{bukang} is used to negate both nominal predicates and constituents, as in \xref{ex3.10.13} and \xref{ex3.10.14}.


\ea\label{ex3.10.13} 
\gll Deram Islam oorang bukang (SL Malay)\\
  3\textsc{pl} Islam man \textsc{neg.nonv}\\
  `They were not Muslims.' (=\ref{ex3.10.2})
\z




\ea\label{ex3.10.14} 
\gll Thaangang=dering bukang kaaki=dering masà-maayeng\\
  hand=\textsc{instr} \textsc{neg.nonv}  foot=\textsc{instr}  must-play\\
  `You must not play with the hands, but with the feet.' (299, ex.408)
\z


The list of features relevant to the SSLA area included one saying that cleft sentences negated like nominal equational ones (2.1.8.3),\footnote{This
  actually referred to the negation of the focus, since there are other forms of negation in Sinhala clefts, as illustrated earlier in \xref{ex3.10.9} and \xref{ex3.10.10}.} 
and despite the possible absence of marked cleft sentences of the Dravidian-Sinhala type in SL Malay, the parallel just mentioned and illustrated in \xref{ex3.10.14} is suggestive, and appears worthy of investigation in relation to a previously unnoted effect of contact on negation.



\subsection{Conjunctive Participles}\label{gair:sec:3.8}


Conjunctive participles may occur with overt lexical subjects, not co-indexed with main subject (or agent). 
 In Sinhala example \xref{ex3.8.1}, the conjunctive participle clause has an overt subject not coreferential with that of the main clause, and it is in the same case that it would have in an independent sentence:\footnote{See
  especially \citet{Gair2005} and \citet{McFaddenEtAl2010} for case and coreference in relation to finiteness.
}


\ea\label{ex3.8.1} 
\gll ammaa gihil-laa ma-{\dott}a seerama gedara-w{\ae}{\dotd}a k{\E}ra-nna oona unaa\\
     mother go-\textsc{cp} I-\textsc{dat} all  house-work  do-\textsc{inf}  necessary become.\textsc{past} \\
   `With Mother gone, I had to do all the housework.'  
\z

 In SL Malay, the same possibility holds, as noted by Nordhoff (p.464), who provides the following example: 


\ea\label{ex3.8.2} 
\ea
\gll Go asà-nii{\ung}al,\\
  1\textsc{s.familiar} \textsc{cp}-die\\
  `I having died'
\ex  
\gll Alla go=nya  asà-dhaathang,\\
  Allah  1\textsc{s.familiar}=\textsc{dat} \textsc{cp}-come\\
  `Allah having come towards me.'
\ex  
\gll kuburan asà-gaali,\\
  grave \textsc{cp}-dig\\
  `The grave having been dug'
\ex 
\gll Go=nya  kubur-king!\\
  1\textsc{s.familiar}=\textsc{acc} bury-\textsc{caus}\\
  `Bury me!'
'I die and Allah comes for me and the grave will be dug and they will have
 me buried.' (539, ex110a)
\z
\z



\subsection{Reportative/hearsay particle }\label{gair:sec:3.9}
 A form commonly referred to as a reportative or hearsay particle or affix is a feature of the SSLA languages, including SL Malay.\footnote{Malayalam
  may be an exception here, \citet[14]{Bashir2006} remarks that ``in Malayalam evidentiality distinctions are not morphologically encoded'', but she does give an example with a verbal noun with the evidential function:
  \ea
   \gll R{\={a}}man-re acchan -{\={\i}} v{\={\i}}Tunirmmiccu k{\={e}}TTu\\
    Raman-\textsc{gen} father(\textsc{nmlz}) this house build(VERBAL NOUN)\\
    `Raman's father built this house' (Speaker has learned this from a third party).
    \z
} 
It may also have evidential force indicating knowledge not directly known. Nordhoff refers to it an evidential clitic  (pp. 337ff), and that is certainly an important aspect of its character. For Sinhala speakers, is commonly taken as having the English translation equivalent `it seems', though the hearsay implication is stronger in Sinhala, which generally carries the sense that someone has said/heard it. The unmarked position for its occurrence is sentence final, but in some languages, as in Sinhala, it may occur on internal constituents as well. Other languages within the general South Asia area may, of course have forms with similar functions, but they are generally of a different form and location. Marathi and Dakkhini Hindi, however, do have such forms, clearly as a result of Dravidian influence. A Sinhala example is given in \xref{ex3.9.1}.


\ea\label{ex3.9.1} 
\gll poliisiy-e{\ng} gunapaala-{\dott}a hariya{\dott}a g{\ae}huwaa=lu.\\
   police-\textsc{instr}  Gunapala-\textsc{dat} really hit$\setmin$\textsc{past}=\textsc{evid}\\
  `The police really beat Gunapala, they say/it seems.\\ 
\z

(This example also illustrates a SLM feature shared with Sinhala that is not in the list; the use of the instrumental form in ``corporate'' subjects.)
 The equivalent in SL Malay is given by Nordhoff as \textit{kiyang}, or \textit{keyang}, and as he notes, others have given \textit{kanyang} (p. 337).
  Like Sinhala \textit{lu}, \textit{kiyang} can mark hearsay or information for which the speaker does not take responsibility. Nordhoff gives several examples with somewhat different implications, including \xref{ex3.9.2}:


\ea\label{ex3.9.2} 
\gll Seelong Airport=yang duva-pulu-umpath vakhtu=le asà-bukka arà-simpang kiyang\\
 Ceylon Airport=\textsc{acc} two-ty-four  hour=\textsc{addit} \textsc{cp}-open  \textsc{nonpast}-stay \textsc{evid}\\
 `The Ceylon Airport is open 24h, it seems.' (388 ex. 458)
\z


 SL Malay \textit{kiyang}, or its variants, have generally been described as a sentence final form that must attach to the predicate, as is said to be the case with Tamil \textit{--aam}, on which it is said to be modeled, as in \citet[142]{Bakker2006} or \citet[175]{SmithEtAl2006cll}.\footnote{Whether
  it is the case that -\textit{aam} must occur on the last constituent or the predicate, is not entirely clear. The account in \citet[214]{GairEtAl1978}, which calls it a `quotative marker' also has ``usually'' \citet[148]{Suseendirarajah1993} for Jaffna Tamil, simply says that it occurs ``finally''. \citet[151]{Schiffman1999} however, says that -\textit{aam} can be added to various constituents to indicate that the speaker does not claim responsibility for the veracity of the statement, but merely reports something. He states that It is usually added to the last constituent of the sentence, but adds that ``it can also occur somewhat idiomatically (or ironically) in other places in a sentence, e.g. with reduplicated noun phrases'';
  \ea  
  \gll periya ivar-aam ivaru\\ 
	big this.man.\textsc{hon}-\textsc{evid} this.man.\textsc{hon}\\
       `Well, la-de-da, get a load of him.' (i.e., he thinks he's hot stuff.)
   \z
}
In Sinhala, \textit{lu} commonly appears sentence final, with scope over the entire sentence, but it may also occur on internal constituents.  It is a focus-marking or more precisely, perhaps, a `Cleft-linked' or `Cleft-requiring' form, along with a number of others. That is, its appearance on an internal constituent requires the cleft/focus construction described in  \ref{gair:sec:3.7} as requiring a special verb form. As such, it is part of a complex system that is very much integrated into Sinhala structure, and which I have described elsewhere (cf. footnote \ref{gair:fn:12}).
Nordhoff notes that it is difficult to elicit \textit{kiyang}, and that is quite understandable, given its semantics, discourse function, and specificity. It is, as he points out optional, to which he might have added that it is triggered by the discourse situation and the wider context, which are difficult to induce. He does provide, however, examples of sentences elicited using Sinhala (Exx. 455-457, p. 338) that show a distribution like Sinhala. They are given here as \xref{ex3.9.2bis}-\xref{ex3.9.4}, and I have provided a separate item-by item gloss for Sinhala. Note that (\ref{ex3.9.3}b) does not accord with the characterization referred to above of \textit{kiyang} as necessarily sentence final.\footnote{This
  was pointed out by one reviewer, for which I am grateful. In this case, of course, the placement could simply be calquing the eliciting form (\ref{ex3.9.2bis}a), but there is a possibility that there is a parallel with the characterization of the placement of --\textit{aam} as in the preceding footnote. In any case this is beyond the scope of this paper and invites further research.
}



\ea\label{ex3.9.2bis} 
\ea
\gll Haturaa balahatkaaray-e{\ng} lamun ba{\und}avaagannavaa=lu  (Sinhala)\\
  enemy force-\textsc{instr}     children recruit=\textsc{evid} \\
  `The enemy is recruiting children, it seems.'\footnotemark (387, ex. 455)
%TODO pl???
\ex
\gll Satthuru paksa aanak pada arà-kumpulkang kiyang.  (SL Malay)\\
  enemy   force children \textsc{pl} \textsc{nonpast}-recruit \textsc{evid}\\
  `The enemy is recruiting children, it seems.' (387, ex. 455)
\z
\z
\footnotetext{I
  would take this as `The enemy is recruiting children by (illegal or improper) force.'}



\ea\label{ex3.9.3} 
\ea
\gll Haturaa=lu balahatkaaray-e{\ng} lamun ba{\und}avaaga-n-ne. (Sinhala)\\
  enemy-\textsc{evid} force-\textsc{instr} children recruit-\textsc{prs}-\textsc{foc} \\
'The enemy, it seems, is recruiting children.' (387, ex 456)
\ex
\gll Satthuru=kiyang paksa aanak pada arà-kumpulkang. (SL Malay)\\
  enemy=\textsc{evid} force child \textsc{pl} \textsc{nonpast}-recruit  \\
   The enemy, it seems, is recruiting children.' (387, ex. 456)
\z
\z

 

\ea\label{ex3.9.4} 
\ea
\gll Haturaa balahatkaaray-e{\ng} lamun=lu ba{\und}avaaga-n-ne. (Sinhala)\\
  enemy  force-\textsc{instr} children=\textsc{evid} recruit-\textsc{prs}-\textsc{foc}  \\
'The enemy is recruiting CHILDREN, it seems.'  (387, ex. 457)
\ex
\gll Satthuru paksa aanak pada=kiyang arà-kumpulkang. (SL Malay)\\
  enemy       force child \textsc{pl}=\textsc{evid} \textsc{nonpast}-recruit  \\
   `The enemy is recruiting CHILDREN, it seems.' (387, ex. 477)
\z
\z

Nordhoff presents these with a due cautionary statement since they were elicited using Sinhala \textit{lu}, but if they do in fact represent sentences that could occur in natural SL Malay discourse, they are further striking evidence of detailed and complex assimilation to Sinhala.\footnote{Nordhoff
  notes that unlike in Sinhala \textit{kiyang} does not trigger a special verb form, but appears with different past forms. However, the Sinhala verb form he refers to is not specific to \textit{lu}, but is a central part of the cleft/focus system mentioned earlier, and it does not appear to have  a SL Malay equivalent, so that its non-appearance here is not surprising.}

\section{Summary}
It is clear that SL Malay, in the time since its arrival on the island of Lanka, has adopted through contact a number of features characteristic of the languages of that island, whether Sinhala or Tamil, most likely both. These Sri Lankan languages, along with Dhivehi, in turn are part of a larger South-South Asia linguistic area, the features of which are essentially South Dravidian in origin. However, It should be recalled that the South-South Asia linguistic area, in turn, forms a part of a larger South Asian one, and thus shares the relevant general features of that larger area as well.\footnote{An
  extensive presentation of syntactic typological characteristics of languages of India, and some extent wider South Asia, from four major families, covering both similarities and differences, is given in \citet{Subbaraoinpress} and in part in his contribution in \citet{KachruEtAl2008}.
} 
In the case of Sinhala and Dhivehi, of course, those features were carried over as inheritances from their mainland source. This is not the case for the non-SA imports SL Malay, or SL Portuguese, which, as stated at the beginning, were typologically very different from the South Asian ones, and are especially interesting for that reason. My concern here was with the extent to which SL Malay participated in the SSLA sprachbund, and features not specific to that subarea have not been dealt with directly. In investigating contact induced typological change in SL Malay however, it is obvious that those more general features are of equal importance with those specified here, since they were all characteristic of the donor languages. 
 As one crucial example of such a vitally important feature, we may note basic SOV and right-headedness, and in fact the strong right-headedness that is characteristic of SS Asia and underlies several of the features here. One interesting aspect here is that this did not extend to the morphological level in SL Malay, though it does pervade the syntax. This was noted here earlier in relation to the conjunctive participle and temporal prefixes developed under influence from post-element forms. Something similar also appears to be the case with SL Portuguese, as noted by Ian Smith, who states in relation to SL Portuguese and Sourashtra \citep[408]{Smith2001}: ``Structurally, Sourashtra was closer to Tamil, again because of a closer starting position; SLP showed more accommodation to Tamil, but had not modified the structure of some closely bound elements such as verbal prefixes.'' This raises some interesting questions as to the variable resistance of aspects of the grammar to contact induced change, but that is beyond our concern here, which was simply to illustrate the extent to which SL Malay conforms to the SSLA pattern.
 In the case of Sinhala, and most likely Dhivehi, the Dravidian contact spanned millennia. For SL Malay, as for SL Portuguese, the time span was shorter, encompassing a few centuries at most (see papers in this volume for views on this for Malay). Thanks to the work of Sebastian Nordhoff and others represented here, SL Malay is now an unusually well documented language in regard to both its structure and its history and can serve as an unusually clear and important example of contact induced change resulting in a major typological shift in branching direction from a strongly right branching language to an essentially left branching one as part of a linguistic subarea in a relatively short span of time. 
