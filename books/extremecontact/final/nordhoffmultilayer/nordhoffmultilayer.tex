 
\chapter[The genesis of SLM as a multi-layered process]{The genesis of Sri Lanka Malay as a multi-layered process} 
 
\chapterauthor{Sebastian Nordhoff}{Max Planck Institute for Evolutionary Anthropology}
  
\section{Introduction}
Sri Lanka Malay differs remarkably from all other varieties of Malay. The most notable differences are morphological complexity and word order, but more subtle changes, like paragogic velar nasals are also found. In this paper, I take stock of the features which can be seen as defining for Sri Lanka Malay and try to trace them to the input languages Malay, Tamil, and Sinhala. I furthermore try to establish the most likely point in time where the change took place as well as the process at work.

The set of `Sri Lankan' features includes features which can be considered peculiar by the casual observer familiar with the most prominent Malay varieties, Bahasa Malaysia and Bahasa Indonesia. Some of these features are in fact not innovations on Sri Lankan soil, but predate the arrival of the Malays in Sri Lanka and are also found in basilectal varieties of Malay. Some of these features have occasionally been claimed to be of Tamil or Sinhala influence because Standard Malay does not show them. By including them in the set of features to investigate, we can sort out which of the `Sri Lankan' features are indeed Sri Lankan, and which are retentions from basilectal varieties of Malay.

Language change is the result of speakers' adaptation to the society which surrounds them. As such, social variables like demographics, prestige, political situation, and legislation influence the speech of speakers, and as a result, language change. There is also a residue of internal change, or drift, which will be discussed where relevant. In the Sri Lankan context, however, non-internal language change is by far more important.

In order to arrive at a useful classification of the `Sri Lankan' features, we must first take a look into the social and demographic history of the Sri Lankan Malays. We can distinguish 5 periods based on the following landmark events:

\begin{enumerate}
 \item The arrival of the Dutch in Ceylon (1656)
 \item The eviction of the Dutch by the British and the creation of the Malay regiment (around 1800)
 \item The disbandment of the regiment (1873)
 \item The independence of Ceylon from the UK (1948) and the establishment of nationalist language policies favouring Sinhala (1956)
\end{enumerate}

The Dutch had captured parts of the Indonesian archipelago in the early 17\textsuperscript{th} century and had set up their capital in Batavia on the major island of Java. In order to continue their conquest, they recruited native soldiers from Java and other islands, who would eventually make it to Sri Lanka in 1656. These recruits came from all over the archipelago, but the Eastern islands of the Moluccas were overrepresented \citep{Paauwtv}. The recruits came from a variety of ethnic backgrounds and spoke widely diverse languages. Around Batavia, so-called \em kampungs \em were set up, which were ethnically quite homogeneous. For purposes of inter-ethnic communication, the different groups used Malay as a lingua franca. Malay had had a long history as lingua franca in the archipelago, even in areas where no native Malays were found. Different varieties of this trade language existed, with substrate influence giving rise to diverging structures. This difference is still discernible today in contemporary daughter languages, spoken in different location in the East of Indonesia \citep{Paauw2008phd}. 

These lingua franca versions of Malay are very different in structure from what is known in most of the academic world about Malay/Indonesian, and some of the peculiar features of Sri Lanka Malay are actually shared with them, so that these features need not necessarily have their origin in Sri Lanka. The structures of SLM which can be attributed to its origin as an offshoot of the trade varieties of Malay are an interesting area of research, but are only marginally relevant to the language's fate on Sri Lanka. I will classify them as `Stage 0'. Stage 0 covers everything before the 17\textsuperscript{th} century.

In Batavia, speakers with different version of the lingua franca met, and eventually mingled. The precise time of the mingling is difficult to establish, but it is sure that as soon as the ships to Sri Lanka set sail, ethnic heterogeneity was the norm. This is comparable to other displaced population with dialectal diversity, e.g. Indians in South Africa \citep{Mesthrie1993} or British in New Zealand \citep{Trudgill2004}. Documentation of the events in  South Africa and New Zealand can inform us about the processes likely to have taken place in the Malay context. These processes generally go under the rubric of `dialect levelling'. Dialect levelling probably already took place back in the Jakartan kampungs, but gained in importance as a process when the recruits reached Sri Lanka. The period of dialect levelling was finished when Sri Lanka Malay had established itself as a focussed variety with comparatively little internal variation. There is agreement that the point of stablization predates the arrival of recruits from the Malay peninsula, starting 1819. Peninsular Malay has had close to no impact on SLM, suggesting that the new arrivals adapted to the local standard, which must have been established at that time \citep[compare the `Founder Principle'][]{Mufwene1996}. This period of dialect levelling is Stage 1. It starts with the establishment of kampungs in Batavia, and ends before the Malaysians arrived, giving it the approximate time span of about 1640-1810. 

In Sri Lanka, the Malays started to interact with the local population. This gave rise to at least three different processes, which are more difficult to pin down than the ones discussed above. The easiest one is the most recent, convergence towards Sinhala. After the independence of Ceylon, the Sri Lankan government enacted a number of laws favouring Sinhala in public life. This led to a sociolinguistic shift with Sinhala gaining prestige and domains while the other languages receded. This applies not only to Sri Lanka Malay, but also to English and the Moor variety of Tamil. The timing of this process is easy to link to the establishment of said laws in the 1950s. The processes at work are convergence and attrition, and the period extends to the present. I will catalogue this process as `Stage 4'.

The remaining stages 2 and 3 are the most difficult to analyse since they overlap to a large extent. Stage 2 covers the reinforcing of marginal Malay structures by Sinhala and Tamil (`Substrate reinforcement'); Stage 3 covers the imposition of novel Lankan structures upon Sri Lanka Malay, which were not found in any of the Trade Malay varieties.
These two periods basically began with the first contact between Malays and Lankans and continue up to this day. The process at work in Stage 2 was substrate reinforment; as for Stage 3, various proposals have been made: creolization \citep{SmithEtAl2004}, convergence \citep{Bakker2006}, or metatypy \citep{Ansaldo2008genesis}. See Table \ref{tab:stages}. It is one goal of this paper to disentangle various claims about language contact, language change, demographics, sociolinguistics and processes at work by sorting the `Sri Lankan' features into the different stages established. Anything which is a result of stages 0, 1, 2, or 4 cannot be used as an argument for the process at work in Stage 3 \citep[cf.][]{Smithtv,Slomansontv}. Note that I also include a Stage 5 as a category for developments which are internal and are not due to any kind language change, while at the same time not being shared with other varieties of Malay either.  

\begin{table}[h]
\begin{tabular}{lp{3cm}lll}
stage & process & influence & place & time\\
\hline 
  0 & Pre-SLM	 		 	& Malay 	& Indonesia 		& $<$1600 -1800\\\\
  1 & Dialect Levelling/koineization    & Malay 	& (Batavia,) Sri Lanka 	& 1650 - 1800\\\\
  2 & Substrate Reinforcement 		& Malay+Lankan	& Sri Lanka 		& 1650 - today\\\\
  3 & ???                     		& Lankan	& Sri Lanka 		& ?? \\\\
  4 & convergence towards Sinhala 	& Sinhala	& Sri Lanka 		& 1950-today\\\\
  5 & internal development 		&		& Sri Lanka 		& 1650-today \\\\	
\end{tabular}
\caption{The different stages of language contact in Sri Lanka Malay.}
\label{tab:stages}
\end{table}

I will now discuss the `Sri Lankan' features and their classification.
  

\section{What happened}

In this paper I will investigate 60 features which distinguish SLM from other varieties of Malay. Most of the features are clearly Sri Lankan, but some of the features are common to spoken varieties of Malay and distinguish them from the written standard language. These features are included for the benefit of the readers familiar with Bahasa Indonesia/Bahasa Malaysia, but not with the wealth of the spoken varieties.

The following examples show the drastic nature of the changes

% \ea
% \ea
% \gll Derang pada \textbf{baaye=nang}   mlaayu arà-oomong \textsc{SLM} \\
% \textsc{3pl} \textsc{pl} good=\textsc{dat} Malay \textsc{nonpast}-speak\\
% \ex 
% \gll Dia bicara bahasa melayu lancar sekali \textsc{Colloquial Indonesian}\\
% 3 speak language Malay flow straight \\
% `They speak good Malay.' (p.c. David Gil)
% \z
% \z 
% 
% \ea 
% \ea
% \gll batu\~{}batu itu dia bawa  dari Ceylon ke negara\~{}negara yang lain \\
%      stone\~{}pl dist 3 bring from Ceylon to country\~{}pl rel other \\
% `Those stones, he brought them from Ceylon to other countries.' (p.c. David Gil)
% \ex
% \gll Itthu    baathu=yang    incayang Seelong=dering laayeng nigiri=nang asà-baapi. \\
%  \textsc{dist} stone=\textsc{acc} \textsc{3s.polite} Ceylon-\textsc{abl} other country=\textsc{dat} \textsc{cp}-bring\\
% `Those stones, he brought them from Ceylon to other countries.' (K060103nar01)
%  
% \z  
% \z  
% 
\ea
  \ea 
    \gll bapak saya kasi  dia emas  \textsc{Colloquial Indonesian}\\
	father 1\textsc{sg} give  3\textsc{sg} gold\\
	`My father gave him gold.' (p.c. David Gil) 
  \ex
    \gll Se=ppe    baapa incayang=nang    ummas su-kaasi \textsc{SLM} \\
	\textsc{1sg}=\textsc{poss} father \textsc{3s}.\textsc{polite}=\textsc{dat} gold \textsc{past}-give\\
`My father gave him gold.' (K070000wrt04)
  \z
\z    

We note a change in word order, long vowels and consonants, and the development of bound morphology. A list of  60 `Sri Lankan' features is given in  Table \ref{tab:overview}.


\begin{table}[p]
\centering
\begin{tabular}{p{8cm}p{8cm}} %outer table
\begin{tabular}{ll}
\textbf{phonology}  & stage\\ 
\hline
loss of initial {\em h-} 		&  1\\
loss of initial {\em s-} in {\em satu} 	&   1\\
phonemic schwa      		&  1\\
existence of schwa in final syllables	&  1\\
dropping of schwa in initial syllable &   1\\
lowering of high vowels in final syllables &   1\\
velarization of final nasals	&   1\\ 
phonemic prenasalization mb,nd,ng&   2\\
consonant gemination 		&   2\\
vowel length			&  2\\ 
dental/retroflex distinction 	&   3\\
\phonet{v}$\to$\phonet{V}	&   3\\ 
\unJ 				&   5\\
paragogic \ng 			&   5\\
% s$\to$c in Tamil loans 		&   5\\
/\#t/$\to$/\#c/ 			&   5\\
% d$\to$\dentd 			& independent development& 5\\ 
\\	
\textbf{morphology} \\ 			
\hline
no  \textit{m\E{}N-} &  0\\ 
no  \textit{di}- 	&  0\\ 
% no -\textit{in} 	&  0\\   
plural pronouns in orang&  1\\ 
chinese pronouns 	&  1\\ 
TAM markers based on the
 existential \em ada\em &  1\\ 
enclitic TAM adverbs 	&  1\\ 
possessive marker =pe 	&  1\\  
plural marker pada 	&  1 \\ 
negator thraa 		&   1 \\   
% \hline 
%  kànà-  			& substrate reinforcement (Sinhala) 			& 2    \\  
% zero nominalization 	& subtrate reinforcement (Sinhala) 			& 2\\ 
  morphologicization   	&  3   \\ 
 rigid word classes  	&  3    \\
 coordinating clitics  	&  3 \\
 indefinite pronouns   	&  3    \\
 discourse markers based on DEIC+X   & 3 \\
 negation pattern   	&  3 \\ 
CASE &  3 \\
INF &  3 \\
CP &  3 \\
PRES PTPL &  3 \\
Impolite imperative  &  3 \\
copula  		&   3 \\
existentials duuduk/aada&  2,3,4\\
 INDEF  		&  3,4 \\ 
 jamà- 			&  5 \\
\end{tabular}
% \caption{Morphology}
% \end{table} 
 
& %outer table cell boundary

% \begin{table}
\begin{tabular}{ll} 
\textbf{syntax}& stage\\ 
\hline
 serial verbs   &  2 \\
 vector verbs  &   2 \\
 QUANT N     &   2 \\ 
 DEM N    &  2 \\
 GEN N   &  2 \\
 relator nouns  &   2 \\ 
 SOV  &  3 \\
 RELC N    &  3 \\
 ADJ N      &  3 \\
 STD DAT N MORE ADJ  &  3 \\ 
 postpositions  &  3 \\  
\\
\textbf{semantics} \\ 
\hline
EVID &  3 \\
QUOT &  3 \\
tense rather than aspect & 3 \\
dative subjects &  3 \\ 
non-nominative subjects &   3,4 \\
 arà- &   3,4 \\ 
\\
\textbf{discourse} \\ 
\hline 
% terms of address & archaism & \\
% pronoun avoidance & archaism & \\
tail-head linkage & 3\\ 
\\
\textbf{lexicon} \\ 
\hline
Jakartan words like \em oomong\em & 1 \\
Javanese words like \em kulluth\em & 1\\
Tamil loans like \em kattil\em & 3 \\
\\ \\ \\ \\ \\  %for vtop alignment
% \end{tabular}
% \caption{Lexicon}
\end{tabular}
\end{tabular} %outer table
\caption{`Sri Lankan' features in a variety of linguistic domains and the most likely point of their emergence.}
\label{tab:overview}
\end{table}
 
\newpage
\subsection{Phonology}
The phonological features which can be taken to be Sri Lankan are given in Table \ref{tab:phonology}.

\begin{table}[h!] 
\begin{tabular}{p{4cm}p{4cm}p{2cm}l}
loss of initial {\em h-} 		& shared with other Trade Malay varieties & \tiny \citet{Adelaar1991,AdelaarEtAl1996,Paauw2004,Paauw2008phd} 	& 1\\
loss of initial {\em s-} in {\em satu} 	& shared with other Trade Malay varieties & \tiny \citet{Paauw2004}	& 1\\
phonemic schwa      		& archaism                 &                          	& 1\\
existence of schwa in final syllables	& archaism from Jakarta  &  \tiny \citet{Adelaar1985,Nordhoff2009}                      & 1\\
dropping of schwa in initial syllable & Jakarta  &  \tiny Uri Tadmor, p.c.  & 1\\
lowering of high vowels in final syllables	& shared with other Trade Malay varieties   &\tiny \citet{Paauw2004,Paauw2008phd,Nordhoff2009} & 1\\
velarization of final nasals	& shared with other Trade Malay varieties     	 &	& 1\\
\hline
phonemic prenasalization \umb,\und,\ung& substrate reinforcement (Sinhala) & \tiny \citet[118]{Nordhoff2009}& 2\\
consonant gemination 		& substrate reinforcement   & \tiny \citet[119]{Nordhoff2009}                      	& 2\\
vowel length			& substrate reinforcement & \tiny \citet[117]{Nordhoff2009} 				& 2\\
\hline
dental/retroflex distinction 	& \textbf{candidate} & \tiny \citet[117-118]{Nordhoff2009}				& 3\\
\phonet{v}$\to$\phonet{V}	& \textbf{candidate} & \tiny \citet{SmithEtAl2004}	 	& 3\\
\hline
phonemic prenasalization  \unJ 	 & independent development  &\tiny \citet[120]{Nordhoff2009}                     	& 5\\
paragogic \ng 			& independent development & \tiny \citet[120]{Nordhoff2009}				& 5\\
% s$\to$c in Tamil loans 		& independent development				& 5\\
/\#t/$\to$/\#c/ 			& independent development  &\tiny \citet[120]{Nordhoff2009}		& 5\\
% d$\to$\dentd 			& independent development				& 5\\
\end{tabular}
\caption{`Sri Lankan' features in phonology.}
\label{tab:phonology}
\end{table}

Next to a couple of features which are common in other varieties of Malay as well, and which are therefore regarded as preceding the language's arrival in Sri Lanka, we find some salient phonological differences, namely length distinctions in consonants and vowels and the tautosyllabic analysis of certain NC clusters, giving rise to phonemic prenasalized stops. These phonological features are all also found in the Lankan languages\footnote{Tamil has no prenasalization.} and are among the one most often cited for the peculiar status of Sri Lanka Malay. At a closer look, however, it emerges that these structures are not at all novel in Sri Lanka Malay. Many Malay dialects have a regular subphonemic lengthening of the penultimate syllable, which is the only place where vowel quantity distinctions are found in Sri Lanka Malay. We are thus dealing with a phonemicization of a subphonemic contrast through language contact, not with the development of a completely novel feature. Similar things can be said about gemination. Gemination is found in some Malay dialects following a schwa. This is also the case in Sri Lanka Malay, where the overwhelming majority of geminate consonants follow a schwa.\footnote{There 
 are a couple of exceptions like \trs{appi}{fire} or \trs{ikkang}{fish}.
}
The process of gemination of consonants after schwa already present in some forms of Trade Malay was regularized and expanded in Sri Lanka, certainly under influence from Sinhala and to a lesser degree from Tamil, but it is not a phenomenon completely alien to a Malay language. A similar argument can be made for the prenasalized stops. Some dialects of Malay syllabify NC sequences as .NC, others as N.C. Speakers from both types were brought to Sri Lanka, where lexemes from different dialects made it into the languages. This phonemicization was without doubt helped by Sinhala having a phonemic distinction between tautosyllabic and heterosyllabic NC sequences, but Malay was not a blank slate when Sinhala started to exert its influence in phonology.

Things are different for the dental/retroflex distinction. SLM expanded the original /\dentt,d/ set to a fourfold distinction of /\dentt,\tz,\dentd,\dz/. There are no known dialects with \tz{} or \dentd, so that this development is one of the few instances in SLM phonology where we are dealing with a clear Stage 3 process. Another clear instance of a Stage 3 process is the change in articulation of the labiodental from fricative \phonet{v} to approximant \phonet{V}. 

There are some other phonological features of Sri Lanka Malay, which constitute independent developments and are not due to language contact.  These are the development of phonemic \phonem{\unJ}, the addition of a paragogic velar nasal to some historically coda-less lexemes (e.g. \trs{buunung}{kill}$<$\em bunu(h)\em) and  the change from initial  \phonem{t} to \phonem{c} in some lexemes.

For the main concern of this paper, the features of stage 3, we can only retain two items: the development of a dental/retroflex distinction and the change in articulation of the labiodental consonant.

   
\subsection{Morphology}
The morphological features which can be taken to be Sri Lankan are given in Table \ref{tab:morphology}.

\begin{table}[h!]
\centering
\begin{tabular}{p{4cm}p{4cm}p{2cm}l}
no \emph{m\E{}N}- & shared with other Trade Malay varieties & \tiny \citet{Adelaar1991,AdelaarEtAl1996,Paauw2004,Paauw2008phd} 	& 0\\ 
no \emph{di}- 	& shared with other Trade Malay varieties  &\tiny \citet{Adelaar1991,AdelaarEtAl1996,Paauw2004,Paauw2008phd} 	& 0\\ 
% no -in 	& shared with other Trade Malay varieties \tiny \citet{Adelaar1991,AdelaarEtAl1996,Paauw2004,Paauw2008phd} 	& 0\\ 
\hline
plural pronouns in orang& shared with other Trade Malay varieties &  \tiny \citet{Adelaar1991,Paauw2004}	& 1\\ 
chinese pronouns 	& shared with other Trade Malay varieties &\tiny \citet{Adelaar1991,Paauw2004}	& 1\\ 
TAM markers based on the
 existential \em ada\em & shared with other Trade Malay varieties, mainly Moluccan  &\tiny \citet{Adelaar1991}   	& 1\\ 
enclitic TAM adverbs 	& shared with other Trade Malay varieties  &\tiny \citet{Adelaar1991}  	& 1\\ 
possessive marker =\emph{pe} 	& shared with other Trade Malay varieties & \tiny \citet{Adelaar1991,Paauw2004,Paauw2008phd}	& 1\\  
plural marker \emph{pada} 	& innovation from Jakarta  & \tiny \citet{Adelaar1991,Paauw2004} 				& 1 \\ 
negator \emph{thraa} 		& innovation from the Moluccas  &\tiny \citet{Adelaar1991,Paauw2004}			&  1 \\   
% \hline 
%  kànà-  			& substrate reinforcement (Sinhala) 			& 2    \\  
% zero nominalization 	& subtrate reinforcement (Sinhala) 			& 2\\
\hline
  morphologicization   	& \textbf{candidate} & & 3   \\ 
 rigid word classes  	& \textbf{candidate} & \tiny \citet{Nordhofffcjoat} & 3    \\
 coordinating clitics  	& \textbf{candidate} & \tiny \citet[314-330]{Nordhoff2009}& 3 \\
 indefinite pronouns   	& \textbf{candidate} & \tiny \citet[391-392]{Nordhoff2009}& 3    \\
 discourse markers based on DEIC+X   & \textbf{candidate} &\tiny \citet[359-361]{Nordhoff2009}		&  3 \\
 negation pattern   	& \textbf{candidate}  &\tiny \citet[588-590]{Nordhoff2009} & 3 \\
  INF & \textbf{candidate} &\tiny \citet[244-245]{Nordhoff2009} &  3 \\
  CP & \textbf{candidate}  &\tiny \citet[241-243]{Nordhoff2009}&  3 \\
  PRES PTPL & \textbf{candidate}  &\tiny \citet[358-359]{Nordhoff2009}&  3 \\
  CASE & \textbf{candidate} & \tiny \citet[285-313]{Nordhoff2009}&  3 \\
  Impolite imperative &  \textbf{candidate}  &\tiny \citet[265-266]{Nordhoff2009} & 3 \\
\hline
existentials \emph{duuduk}/\emph{aada}& substrate reinforcement, convergence towards Sinhala & \tiny \citet[142-146]{Nordhoff2009} & 2,3,4\\	
 INDEF  		& \textbf{candidate}, Sinhala influence & \tiny \citet[278-282]{Nordhoff2009}& 3,4 \\
\hline
 copula  		& independent development  &\tiny \citet{Nordhoff2011copula} &  5 \\
 \emph{jamà}- 			& independent development & \tiny \citet[245-247]{Nordhoff2009} &  5 \\
\end{tabular}
\caption{`Sri Lankan' features in morphology.}
\label{tab:morphology}
\end{table}

In the domain of morphology, we note that the absence of voice morphology (\em m\E{}N-, di-\em) is not at all particular to Sri Lanka Malay, but actually the norm for basilectal varieties of Malay and found in all dialects of Trade Malay.
The particular pronouns employed, which feature Chinese roots for the impolite 1\textsuperscript{st} and 2\textsuperscript{nd} person as well as a plural formative \em rang \em $<$ \trs{orang}{man} are also comparable to what is found elsewhere in the Malay world. There are two cases of morphological elaboration: the grammaticalization of adverbs and the existential \em ada \em to yield TAM clitics. These also have their origin in Indonesia, although the grammaticalization has spreaded further in SLM than in other varieties. The possessive marker \em punya\~{}pe\em, the plural marker \em pada \em and the negator \em thraa \em are salient characteristics of Sri Lanka Malay, but these three features are shared with other offshoots of Trade Malay and do not constitute a case of language contact in Sri Lanka.

Where we find language contact is in the morphologicization of the language. No other Malay variety comes close to Sri Lanka Malay in the extent of bound morphology. Where many grammatical categories are optional in other varieties of Malay, and realized by particles or adverbs, in Sri Lanka Malay, we find a lot of obligatory categories, like tense or case, realized as bound forms. There is furthermore the formation of nonfinite categories such as an infinitive and two participles. This greater availability of bound material led to the creation of rigid word classes \citep{Nordhofffcjoat}, which is very untypical of a Malay variety. Beyond the domain of the word, we also find a number of devices which are clearly the result of language contact, e.g. the Coordinating Clitics, which are also used in the formation of indefinite pronouns and some discourse markers. These clitics have clear counterparts in Sinhala and Tamil, which also use them for the formation of pronouns and discourse markers. They are not found elsewhere in the Malay world, so that we are dealing with a clear case of language contact here. A final case of language contact in morphology is the negation  pattern, where the past/nonpast-split in negation morphemes is a Tamil influence in Sri Lanka Malay.

Sri Lanka Malay has two existentials with a split along animacy. This is due to influence from Sinhala, but a root in Malay dialects cannot be ruled out. Here, and in other cases of Sinhala influence, it is difficult to establish whether we are dealing with a recent influence from Sinhala, or whether this change predates the political changes of the 20\textsuperscript{th} century. The same is true of the grammaticalization of the indefiniteness marker \em hatthu\em, which can be recent or an older phenomenon.

In the domain of morphology, finally, we can mention two internal developments which are not due to language contact. The first is the subordinate negator \em jamà- \em \citep[\em ja\ng- \em in ][]{Slomanson2008lingua}, the second one is the emergence of a copula \citep{Nordhoff2011copula}. Neither of these developments has a good model in either Malay, Sinhala, or Tamil.\footnote{Tamil \em -aama \em and to a lesser extent Sinhala \em noo \em provide some imperfect approximations of \em jamà-\em.}

\subsection{Syntax}
The syntactic features which can be taken to be Sri Lankan are given in Table \ref{tab:syntax}.

\begin{table}[h!]
\centering
\begin{tabular}{p{4cm}p{4cm}p{2cm}l}
 serial verbs   & substrate reinforcement  &\tiny \citet{Paauw2004,Nordhofffcsvc} &  2 \\
 vector verbs  & substrate reinforcement  &  \tiny \citet{Paauw2004,Nordhofffcsvc} &  2 \\
 QUANT N     & substrate reinforcement & \tiny \citet{Adelaar1991,Paauw2004,Paauw2008phd} &  2 \\ 
 DEM N    & substrate reinforcement & \tiny \citet{Adelaar1991,Paauw2004,Paauw2008phd}  &  2 \\
 GEN N   & substrate reinforcement &  & 2 \\
 relator nouns  &  substrate reinforcement  &\tiny \citet{Paauw2004} &  2 \\
\hline
 SOV  & \textbf{candidate} & \tiny \citet{Adelaar1991}&  3 \\
 RELC N    & \textbf{candidate}  &\tiny \citet{Adelaar1991}&  3 \\
 ADJ N      & \textbf{candidate} & \tiny \citet{Adelaar1991}&  3 \\
 STD DAT N MORE ADJ  & \textbf{candidate}  &\tiny \citet[582-583]{Nordhoff2009} &  3 \\ 
 postpositions  & \textbf{candidate} & \tiny \citet{Adelaar1991}&  3 \\ 
\end{tabular}
\caption{`Sri Lankan' features in syntax.}
\label{tab:syntax}
\end{table} 

In the domain of syntax, we find two different types of serial verb constructions, which are modelled mainly on Tamil structures, but have some resemblances to certain Trade Malay varieties \citep{Nordhofffcsvc,Jaffartv}. We are thus dealing with substrate reinforcement. The bulk of the other features have to do with word order, where Sri Lanka Malay has a clear preference for head-final structures. These features are split as to whether they are also found in some Trade Malay varieties, or whether they are pure influence from Sinhala and/or Tamil. The last feature, relator nouns to specify spatial and temporal relations like \trs{duppang}{in front of, before} and \trs{blaakang}{after} are again a clear case of influence from both Sinhala and Tamil, a Stage 3 process.


\newpage
\subsection{Semantics}
The semantic features which can be taken to be Sri Lankan are given in Table \ref{tab:semantics}.

\begin{table}[h!]
\centering
\begin{tabular}{p{4cm}p{4cm}p{2cm}l}
\hline
QUOT & \textbf{candidate} & \tiny \citet[340-346]{Nordhoff2009}&  3 \\
EVID & \textbf{candidate}  &\tiny \citet[337-340]{Nordhoff2009}&  3 \\
tense rather than aspect &   \textbf{candidate} & &  3 \\
dative subjects & \textbf{candidate}  &\tiny \citet[421-430]{Nordhoff2009}& 3 \\
\hline
accusative and instrumental subjects & \textbf{candidate}, Sinhala Influence & \tiny \citet[421-430]{Nordhoff2009} &  3,4 \\
 \emph{arà}- & substrate reinforcement, Sinhala Influence & \tiny \citet[253]{Nordhoff2009} &  3,4 \\
\end{tabular}
\caption{`Sri Lankan' features in semantics.}
\label{tab:semantics}
\end{table}

It is in the domain of semantics that we find some clear instances of Stage 3 processes. Sri Lanka Malay has a number of grammatical categories completely absent from other Malay varieties, but found in both Sinhala and Tamil. These include  evidentiality and the quotative.  
In a different domain, Sri Lanka Malay gives preference to the encoding of tense, rather than aspect, what is found in other Malay varieties. This mirrors what we find in Sinhala and Tamil. The possibility to use dative subjects is due to influence from Tamil and Sinhala, while the use of instrumental and accusative subjects is due to influence from Sinhala alone.

Next to the clear Stage 3 processes, there are again a number of features which could be due to early influence from Sinhala (Stage 3), or to late influence (Stage 4). These are the use of instrumental and accusative subjects as well as the semantics of the nonpast marker \em arà- \em, which can be used in subordinate contexts with past reference next to its more common usage in present or future contexts in both Sinhala and SLM.


\subsection{Discourse}
The discourse feature which can be taken to be Sri Lankan are given in Table \ref{tab:discourse}.

\begin{table}[h!]
\centering
\begin{tabular}{p{4cm}p{4cm}p{2cm}l}
tail-head linkage & \textbf{candidate} & \tiny \citet[474-475]{Nordhoff2009} &    3\\
\end{tabular}
\caption{`Sri Lankan' feature in discourse.}
\label{tab:discourse}
\end{table}

SLM discourse is organized through tail-head linkage, like in Sinhala and Tamil. This is a feature not found in the relevant varieties of Malay, so that we are dealing with a Stage 3 process.


\subsection{Lexicon}
The lexical features which can be taken to be Sri Lankan are given in Table \ref{tab:lexicon}.

\begin{table}[h!]
\centering
\begin{tabular}{p{4cm}p{4cm}p{2cm}l}
\emph{oomong} etc & vocabulary from Jakarta &   &  1 \\
\emph{kulluth} etc & vocabulary from Javanese &   &  1 \\
\hline
\emph{kattil} etc & vocabulary from Tamil & & 3\\
\end{tabular}
\caption{`Sri Lankan' lexical features.}
\label{tab:lexicon}
\end{table}

Sri Lanka Malay shows a number of vocabulary items from Jakarta, which are obviously a retention. Some other vocabulary items are of Javanese origin, but without a clear link with Jakarta. These are also retentions. SLM has borrowed from Tamil, mainly animal names, but also some very basic items like \trs{kattil}{bed}, \trs{marakari}{vegetables} or \trs{dhaatha}{elder sister}. These borrowings belong in Stage 3. There are  no noteworthy borrowings from  Sinhala.

In the following, I will discuss what happened during the various stages.

\section{When did it happen}
Having charted the linguistic domains of `Sri Lankan' features, I now turn to sorting these features into the different stages. 

\subsection{Summary of Stage 0 features}

\begin{table}[h!]
\centering
\begin{tabular}{cc} 
no \emph{m\E{}N}- & 
no \emph{di}- 	 	\\ 
\end{tabular}
\caption{Stage 0 features.} 
\end{table}

Stage 0 refers to features of Sri Lanka Malay which distinguish it from the modern standard languages, but are in fact shared with all other offshoots of Trade Malay. These features are limited to the absence of voice morphology.
They are of no particular interest to the study of language contact.

\subsection{Summary of Stage 1 features}

\begin{table}[h!]
\centering
\begin{tabular}{p{4cm}p{4cm}p{4cm}} 
loss of initial \emph{h}- 		& plural pronouns in \emph{orang}	& \emph{oomong} etc\\
loss of initial \emph{s}- in \emph{satu} 	& chinese pronouns  	& \emph{kulluth} etc\\
lowering of high vowels in final syllables & TAM markers based on the
 existential \em ada\em  &  \\
velarization of final nasals	& enclitic TAM adverbs  &  \\
phonemic schwa      		&  possessive marker =\emph{pe}   &  \\
\multirow{2}{*}{existence of schwa in final syllables}	&   plural marker pada                       &  \\
& negator \emph{thraa} & \\ 
\end{tabular}
\caption{Stage 1 features.} 
\end{table}

Stage 1 features stem from different dialects of Trade Malay, which met in Batavia and then Sri Lanka and whose mixture formed the very first instance of a Sri Lankan variety of Malay in the 17\textsuperscript{th} century. Some features can be traced to a specific area, while others are found in several areas, but not all \citep{Paauwtv}. The features we find today are those which made it through the dialect levelling process; an equally important number of features did not make it into Sri Lanka Malay and was lost on the way.

The extent of dialect levelling is quite strong in phonology and morphology, but mainly in pronouns and particles, not so much in bound morphology. The lexicon also shows signs of dialect levelling, but syntax and semantics are completely absent. In these areas, Sri Lanka Malay either retains a general Trade Malay pattern (Stage 0) or   adapts to the Lankan model (Stages 2, 3, and 4).


\subsection{Summary of Stage 2 features}

\begin{table}[h!]
\centering
\begin{tabular}{p{4cm}p{4cm}p{4cm}} 
retroflexes& serial verbs  & QUANT N  \\
consonant gemination 		& vector verbs         & DEM N \\
vowel length			&  relator nouns         & GEN N \\ 
phonemic prenasalization \umb{}, \und{}, \ung{} & & \\
\end{tabular}
\caption{Stage 2 features.} 
\end{table}

The features of Stage 2 were marginally present in Trade Malay varieties, but received a boost upon arrival in Ceylon because the Lankan languages made frequent use of them.
Substrate reinforcement is found in the quantity contrasts of vowels and consonants, and in prenasalization. Furthermore, verb serialization and head-final structures which were present in some Trade Malay varieties gained in importance upon contact with Tamil and Sinhala \citep{Nordhofffcsvc}.

\subsection{Summary of Stage 3 features}


\begin{table}[h!]
\centering
\begin{tabular}{p{2cm}p{2cm}p{2cm}p{2.5cm}p{2.5cm}} 
  SOV  			 & EVID 		&   coordinating clitics     & morphologicization  &v$\to$\V  \\
  RELC N    		& QUOT		&  indefinite pronouns      & rigid word classes & tail-head linkage \\
  ADJ N      		& CASE 		&  DEIC +  X    	& impolite imperative  & \emph{kattil} etc  \\
  STD DAT N MORE ADJ  	& INF 		&   copula\footnotemark{}				& 	& \\ 
  postpositions  	& CP 		&   				& 	&   \\ 
  		      & PRES PTPL 	&   				& 	& \\ 
			& DAT SUBJ &   				& 	& \\   
                        & ACC/INSTR SUBJ& 	&    
\end{tabular}
\caption{Stage 3 features.} 
\end{table}
\footnotetext{The development of the copula \em asàdhaatahg \em  was analyzed as an independent development in \citet{Nordhoff2011copula}. Newer research, however, shows that Tamil (\em vantu\em) and Sinhala (\em ävillaa\em) models do exist for this form. }

Stage 3 contains the features which have the most relevance for theories of language contact and change. These features are neither plain nor modified retentions from earlier varieties, and they cannot be attributed to independent developments either. Some of the features in this stage could also be attributed to Stage 4, recent influence from Sinhala.

Stage 3 contains two large groups of features, one relating to word order, the other to grammatical categories. 
A smaller number of features pertain to the emergence of clitics and word formation processes involving them. 
The last noteworthy category contains processes referring to the development of morphology. The last column is the wastebasket category. 
The composition and relevance of these categories will be discussed in detail below, but it can already be noted here that these features do not fall within the area of features typically associated with creolization.

\subsection{Summary of Stage 4 features}

\begin{table}[h!]
\centering
\begin{tabular}{p{2.5cm}p{2.5cm}p{2.5cm}p{2.5cm}} 
existentials\newline \emph{duuduk}/\emph{aada}&   INDEF  		&  ACC/INSTR SUBJ &   semantics of the nonpast form \emph{arà}  \\
\end{tabular}
\caption{Stage 4 features.} 
\end{table}

Stage 4 includes features which are due to recent influence from Sinhala. For all of those features, an earlier date than the 20\textsuperscript{th} century is also possible, in which case we would be dealing with stage 3, or even Stage 2 in the case of the existentials \citep{Nordhoff2012sinhalainfluence}.

The features with Sinhala influence all draw upon existing material and only change the use a little bit. They do not involve huge changes. The difference in existentials involves the reinterpretation of one verb. The grammaticalization of the indefinite article renders a formerly optional feature obligatory. The non-nominative subjects are a straight copy of Sinhala verb categorization, facilitated by the fact that ablative and accusative are available as cases anyway. The extension of nonpast to subordinate past context finally is also easy to accomplish since the nonpast form is already available and the new meaning can easily be added to its range of functions. 

Given the superficial nature of the changes through Sinhala influence, these could easily be very recent, although an earlier origin cannot be ruled out.


\subsection{Summary of Stage 5 features}

\begin{table}[h!]
\centering
\begin{tabular}{p{2cm}p{2cm}p{2cm}p{2cm}} 
\unJ 	  & paragogic \ng & \#t$\to$\#c 				& \emph{jamà}	 \\
\end{tabular}
\caption{Stage 5 features.} 
\end{table}

As for independent developments (Stage 5), we find three phonological features, the development of a new clause pattern, the copula construction, and a new verbal prefix. All in all, this  is a mixed bunch which could have occurred like that in any other language without external influence. 

 



\section{Why did it happen}
In this section, I will survey in more details the processes which led to the emergence of the typical features of Sri Lanka Malay.
 

\subsection{Stage 0: general Trade Malay features}
The process  which led to Stage 0 features in SLM is trivial: we are dealing with retention.



\subsection{Stage 1: dialect levelling}
In Stage 1, which started in Batavia and continued in Sri Lanka until the Sri Lankan variety of Malay emerged as a more or less unified language, the different dialects of Trade Malay were in contact. In the beginning, variation was high and coherence low, but as mutual accommodation progressed, divergence diminished and some features crystallized. This process is not unique to Sri Lanka; it is also found in other communities of displaced people. \citet{Trudgill1986, Trudgill2004} analyses the general processes which take place when an internally diverse community is displaced to a foreign shore. Trudgill restricts his theory to the formation of new dialects by dialect mixture in a community isolated from other speakers of the same language. His example of choice is New Zealand, but the definition corresponds to the Sri Lankan situtation as well. He distinguishes three phases:

\begin{table}[h!]
\centering
\begin{tabular}{lllp{3.5cm}}
 I & adult migrants & first generation & rudimentary levelling\\
II & first native-born speakers & second generation & extreme variability and further levelling\\
III& \multicolumn{2}{c}{subsequent generations} & focussing, levelling, and reallocation %\citep[200]{TrudgKers2005}
\end{tabular}
\caption{The three phases of dialect levelling according to \citet{Trudgill1986}.}
\label{tab:dialectlevelling}
\end{table}

In the first phase, adults of linguistically diverse backgrounds find themselves in foreign territory. In order to arrive at successful communication, they avoid the most divergent features of their dialects, which pose the greatest problems to communication. In phase II, children are born to the first generation, who grow up in a very variable linguistic environment where widely diverging forms from a wide variety of dialects are found. The children are exposed to varieties from speakers outside of their immediate family and their own speech eventually integrates some of the non-family members' speech. Through the generations (phase III), redundant constructions are eliminated. The constructions eliminated and retained normally do not all come from the same dialect, and are not predictable on a deterministic basis, although factors like articulatory ease, salience, and integration into other parts of the relevant linguistic subsystem do play a role. Based on these factors, it is possible to make probabilistic predictions about which features are more likely to be retained on a statistic level \citep{Trudgill2004}. Constructions which are replaced by constructions from other dialects either fall into disuse, or they are reallocated to other functions \citep{Trudgill1986}. Most of the constructions only have communicative functions, but some become markers of identity \citep{LePageEtAl1985} as the displaced community acquires a sense of identity distinct from the mother country population.  \citet{Trudgill2004} provides evidence from recordings of speakers born in the late 19\textsuperscript{th} century in New Zealand, which show the levelling proceeding through the generations.  Trudgill's findings are quantitatively evaluated with a mathematical model in \citet{BaxterEtAl2009trudg}, who reject Trudgill's reliance on frequency factors alone, but confirm the overall validity of the scenario.


\citet{Mesthrie1993} applies Trudgill's scenario to South African Bhojpuri, where he posits the same sequence of events. The British recruited indentured labourers from India. The Indians all spoke different dialects from the Bhojpuri-Hindi-continuum. In South Africa, the dialects first lost the most extravagant features and were then levelled, similar to what happened in New Zealand. The South African case provides another parallel to what happened in Sri Lanka: initial dialect diversity is reduced and a common norm emerges.


% \citet{Siegel2008} further notes that there is remarkable agreement between substratists and superstratists about the fact that there are at least two stages in Creole formation, adaptation (an L2 process by adults) and nativization (an L1 process by children). He shows that Bickerton, Chaudenson, and Mufwene agree on that point and have all three written passages to that effect.  

Sri Lanka Malay shows features from Jakarta (plural marker \em pada\em, dative/allative \em nang \em, retention of schwa in final syllables, dropping of schwa in penultimates, lexical features like \trs{oomong}{speak} rather than \em becara\em) and from the Moluccas (negator \em thraa\em, velarization of final nasals, lowering of high vowels in final syllables). Gil (this volume) provides further evidence that Sri Lanka Malay is in some respects like Jakartan varieties, in others, like Moluccan varieties, and yet in others, like both.\footnote{Malaccan/Malaysian 
 features are not found at all in Sri Lanka Malay.
} 
The combination of features from Jakarta and the Moluccas is quite arbitrary and involves features from phonology, morphology and syntax from both areas without any obvious principle. The selection appears to be quite random, reflecting the general ethnical heterogeneity of the immigrant population, where no one subgroup could impose their dialect. 
An area were we can find phonological reflexes of different dialects is the treatment of schwa. Some varieties of Trade Malay have lost schwa and merged it with /a/ or /i/ in the cases which interest us here. Sri Lanka Malay retains schwa, but there are at least two cases where a lexeme has been taken from a non-schwa variety. This can be seen from the fact that these lexemes have a long vowel, which is never possible in SLM if they underlying vowel is schwa. This shows that the underlying vowel is not schwa in these cases, and that the lexemes must come from a dialect were the distinction between schwa on the one hand and /a/ or /i/ on the other hand was lost. The first case of this is the word \trs{baalai}{nonsense}. The cognate word is \em b\E{}lai \em in schwa-varieties, but \em balai \em in varieties which have lost schwa. Given that the SLM lengthens the vowel, the lexeme must have entered SLM as /balai/, with a full /a/. This contrasts with lexemes which entered SLM with a schwa. These have geminate consonants instead of long vowels, next to raising of schwa. An example is the word for `to split' which is \em belah \em in Indonesia, but \em bìlla \em in Sri Lanka.

The mixing of dialects is followed by regularization. Sri Lanka Malay has generally dropped schwa in antepenultimates, so that the historical from \trs{c\E{}lana}{trousers} is now pronounced \em claana\em. There is some evidence that this process, which is common in present day Jakarta (Uri Tadmor, p.c.), was not fully completed in all of the varieties which entered Sri Lanka. The word for a certain type of ear jewelry is \em kerabu \em in Indonesia. It is found in Sinhala as \em kerabuva \em \citep{Gunasekara1891}, with the schwa intact (and an addition of the suffix \em -va\em, which need not concern us here). However in moder SLM, the same lexeme is \em kraabu\em, without the schwa. Obviously, Sinhala borrowed that lexemes at a point in time where the pronunication with schwa was still current. The subsequent change in SLM did not affect the Sinhala lexeme, which preserves the former state. This shows that, next to the Jakartan dialects with schwa dropping, there were some dialects without schwa dropping in Sri Lanka, but their treatment of schwa was ironed out in the regularization process taking place within the local-born generations.
  
\subsection{Stage 2: substrate reinforcement}
Jeff Siegel observed in a number of papers that in language contact, some marginal structures of one language become more prominent upon contact with another language.
In Melanesia, the pidgin used on plantations in the late 19\textsuperscript{th} century by workers of different origin was brought back to the different homelands of the workers (New Guinea, Solomons Islands, Vanuatu). \citet[349-350]{Siegel1998substrate} shows that there was initially a good deal of variation in Melanesian Pidgin. This variation was reduced when the workers went back to their home regions and used Melanesian Pidgin as a lingua franca for interethnic communication. In the three different major regions, the erstwhile plantation pidgin changed into different directions. In New Guinea, the features matching the local  Western Oceanic languages were retained where variability existed, and gave rise to Tok Pisin. In the Solomon islands, variability was disposed of by retaining features which matched the local
languages of the Southeast Solomonic subgroup, giving rise to Solomons Pijin. And in Vanuatu, the features which matched with features of the North-Central Vanuatu subgroup of Oceanic were retained, giving rise to Bislama. The initial variability found in a diverse population was thus reduced (`levelled'), but the levelling we find here is not purely internal, as was the case in New Zealand or South Africa. Rather, it is influenced by the other languages of the linguistic ecology. If there is a choice among alternative expressions for the same content, speakers tend to prefer the alternative which is also found in other languages they know. This is the case for Melanesian Pidgin, and it is also the case for the elimination of the high initial variability of Sri Lanka Malay. The language of the immigrants had variable order for the placement of the possessor. Some dialects prefered prenominal possessors, others postnominal possessors. On Lankan soil, both Tamil and Sinhala only admit prenominal possessors. The initial variability found in the Malay speech community was then reduced and the feature retained was the one which matched the other languages of the area, the `substrate' so to speak. The reduction of variability to align with a contact language where this variability does not exist is not restricted to Melanesia or Sri Lanka: \citet{Siegel1987,Siegel2000lis} shows this for Fiji and Hawai'i, \citet{Singler1988} for Liberian English.
 

For Sri Lanka Malay, this pattern is widely attested. Given the large dialectal variation of the immigrants' languages, there was often at least one variety where a feature was found that `fit' within the Lankan model. The dialect levelling process is non-deterministic, as discussed above, but a similar structure in another language the Malays had command of (Sinhala or Tamil), facilitated the retention of a feature from a Malay variety. 

Syntactic features like serial verbs and relator nouns fall into this category. They are not a very prominent aspect of Malay languages in general, and are not found across the board. Since these features play an important role in Tamil (and in Sinhala, as far as relator nouns are concerned), the Malay varieties which did have them had a greater impact in the dialect levelling process as far as these features are concerned, and the fledgling use of these constructions in some Malay varieties became well-established through contact with Tamil and Sinhala, which made other speakers realize the existence and potential usefulness of these constructions.

Another aspect of substrate reinforcement is the elimination of certain variants from the dialect pool. Trade Malay varieties allow nominal modification to the left and to the right for a number of modifiers. Sinhala and Tamil generally only allow prenominal position.\footnote{Sinhala 
 (and to a lesser extent Tamil) numerals and quantifiers are an exception to this.
} 
Sri Lanka Malay is not as rigid as Sinhala or Tamil, but is clearly more a right-headed language, even in the nominal domain, than any other variety of Malay. This order of constituents, however, did not come out of the blue, but could build on tendencies already present in Trade Malay varieties, which were reinforced through the speakers' frequent dealing with languages where this order was obligatory.

A clear instance of substrate reinforcement is the velarization of final nasals. This is a feature of Moluccan varieties of Trade Malay, but is also found in Sinhala. The presence of a language in the ecology with a similar phonological rule helped the Moluccan feature to be retained in this domain.

Substrate reinforcement can further be observed in some lexemes, where phonological features from different dialects were retained. The clearest of these cases is prenasalization. NC clusters are syllabified as N.C in some varieties of Malay, while others have .NC \citep{Tapovanaye1995,ApoussidouEtAl2008,Nordhoff2009}. In Sri Lanka we find both, e.g. \trs{am.bel}{take} and \trs{gaa.mbar}{picture} or \trs{an.jing}{dog} and \trs{baa.njir}{flood}.\footnote{The 
 long vowel in the prenasalized words is a consequence of the penult losing its coda as a consequence of the different syllabification. See \citet[120ff]{Nordhoff2009} for more detailed discussion.
}

Gemination of consonants after schwa as found as a subphonemic contrast in some Malay varieties was phonemicized in Sri Lanka under influence from Sinhala and Tamil. Indonesian words of Indian origin like \trs{topi}{hat} (from Hindi) or \trs{kapal}{ship} (from Tamil), which had lost the original geminate consonant in Indonesia, reinstated the geminate in Sri Lanka under influence from Tamil (\em kappal, {\dentt}oppi\em) or Sinhala (\em {\dentt}oppiya\em). This leads to near-minimal pairs like \trs{kappal}{ship}, \trs{kaapang}{when} or  \trs{thoppi}{hat}, \trs{soopi}{liquor}. Once this phonological contrast was established, the formerly subphonemic contrast between \phontrs{ku:mis}{moustache} and \phontrs{kUmmis}{Thursday} or \phontrs{\dentt i:kam}{stab} and \phontrs{\dentt Ikkam}{press} was also analysed as phonemic. At the same time, the above pairs can be used to explain the difference in vowel length in Sri Lanka Malay\footnote{In many cases, a long/short consonant can be explained by a short/long vowel, or the other way round. There are, however, some cases where the long consonant must be specified lexically (\trs{appi}{fire} vs. \trs{baapi}{bring}), as well as there are cases where the length of the vowel must be stored in the lexicon  (\trs{thurus}{straight} vs. \trs{thuurung}{descend}).
} 

%   \begin{itemize}
%     \item This gemination can be analyzed as the result of a bimoraic foot \citep{ApoussidouEtAl2008,Nordhoff2009}/quadrimoraic word \citep{Tapovanaye1995}.
%     \item Sinhala also has a bimoraic foot structure \citep{Letterman1993}, but the correlates thereof a completely different from what we find in SLM
%     \item Tamil has no bimoraic foot structure. It is even debatable whether Tamil has stress or any other sign of foot structure \citep{Keane2001}
%     \item It is unclear how exactly geminated consonants became part of SLM grammar
%   \end{itemize}

The issue of the dental/retroflex distinction is more difficult. Malay varieties generally have a dental /\dentt/ and an alveolar /d/. What has to be explained is the emergence of the postalveolar/retroflex /\tz/ and the dental /\dentd/. The voiceless `retroflex' stop /\tz/ is found in loanwords from Tamil and in native words in front of a back vowel e.g. \trs{baa[\tz]ok}{coconut shell} or \trs{paa[\tz]ok}{hiss} or \trs{oo[\tz]ak}{brain}. From an articulatory point of view, it is plausible to assume that the backness of the vowel caused the preceding consonant to be articulated further back, yielding a subphonemic contrast. This contrast phonemicized upon the entry of loanwords from Tamil, where /\dentt/ and /\tz/ had to be kept distinct, e.g. \trs{ka[\tz\tz]il}{bed} and \trs{a[\dentt\dentt]e}{leech}. It is unclear, however, whether this can be seen as a case of substrate reinforcement, since the subphonemic contrast we hypothesized above had no function in Trade Malay, and its functional usefulness could therefore not be reinforced. 

The development of /\dentd/ is more difficult to explain. This phoneme is only found in initial position in a dozen words or so. It is much more restricted than /\tz/. \citet{Smith2003timing} tries to explain this phoneme through the adoption of a Tamil constraint against retroflex onsets, but fails to notice that Tamil has an equally important constraint against voiced onsets, so that /\#\dentd/ is an unlikely outcome of contact with Tamil. Sinhala has words starting with /\dentd/ as well as words starting with /\dz/, but there seems to be no reasons to assume that the split of Malay /\#d/ into /\#\dentd/ and /\#\dz/ had anything to do with Sinhala. What we can say is that this distinction was probably already made in the early 19\textsuperscript{th} century  since there are manuscripts where two different d's are graphemically distinguished	in the Arabic script employed \citep{Hussainmiya1987}. It is tempting to see the split of the voiced stop as a result of language contact, but closer analysis reveals that the explanation is far from obvious \citep{Smith2012jlc,Nordhoff2012jlcsmith}.

%   \item the existential use of \trs{duduk}{sit} to mean `to live, reside' found in Indonesia was reinforced by Sinhala \trs{innavaa}{exist.\textsc{anim}}, which also has a historical meaning of `to sit' \citep{Nordhoff2010ismil}
%   \item the involitive construction with \em kena \em was reinforced by the Sinhala involitive construction (3\textsuperscript{rd} conjugation) \citep{Nordhoff2010ismil}
%   \item the possibility to use clauses as NPs common in Indonesia was reinforced by the Sinhala zero-adclausal nominalization \citep{Nordhoff2010ismil}  
%   \item  encliticization of TAM adverbs \citep{Adelaar1991} is found in some Indonesian varieties of Malay and are a precursor of the inflectional expression of TAM in SLM, as in Sinhala and Tamil.
%   \item the general liberal attitude to phrase structure in Indonesian varieties of Malay suggests that the occurrence of occasional SOV word order cannot be excluded. This occasional SOV word order would dramatically be reinforced by Sinhala and Tamil. 


\subsection{Stage 3: where the problems lie}\label{sec:disc:stage3}

After discarding everything which is either an early development (see above) or possibly a very recent change (see below), we are left with a set of 22 features which are the result of the crucial Stage 3 and can help elucidate the process at work. Was it creolization, convergence, metatypy, or `ganging up'?

At this point, it is worthwhile to review the two main theories which have been advanced for the genesis of Sri Lanka Malay. Ian Smith and colleagues have defended the view that Sri Lanka Malay is an instance of creolization. See Smith's contribution to this volume for an overview of the argument. While in the better known cases of creolization, the displaced population acquires the socially dominant language, in this case, the local population is argued to have acquired the language of the displaced population. More specifically, Tamil speaking women belonging to the Muslim group of the Moors married Malay soldiers and tried to acquire their husband's language. In doing so, they transfered structures of their native Tamil language to their variety of Malay. This tamilized Malay was passed on to the children of the union, who nativized this new variety.

This contrasts with the theory of `rapid convergence' advanced by \citet{Bakker1995nl,Bakker2000convergence, Bakker2000rapid,Bakker2006}. In his view, Sri Lanka Malay changed within a very short time span, one or two generations. \citet{Bakker2006} also offers `metatypy' as an explanation: similar to what is found in some other heavy contact settings, the grammatical structures of the language of wider communication is imposed on the in-group language, but the in-group language retains its vocabulary as a marker of identity.

This metatypic explanation is also supported by \citet{Ansaldo2008genesis,Ansaldo2009book}. Ansaldo additionally adds that Malay converged towards Sinhala and Tamil at the same time since those languages are grammatically very similar. The double weight of this alliance exerts a greater influence on Malay than any of the two would have had individually.

I have argued above that several processes must be kept apart in the genesis of Sri Lanka Malay. The stage which can give us answers about the processes of language change of interest here is stage 3. The earlier stages have no or only disputable contact influence, the later stages do not fall into the formative period. Taking a look at stage 3, we find, a number of word order features, and a number of grammaticalizing semantic domains.  Furthermore, we are dealing with some features which have to do with the disconnection of phonological word and morphological word, i.e. the morphologicization of the language. Another set is related to Lankan style clitics.

\subsubsection{Creolization}
The term `creolization' has received a lot of interpretations (see Bakker, this volume). The Bickertonian view of Creolization as the breakdown of a communicative system and its successive reinvention clearly does not apply to Sri Lanka Malay, since there are many Malay features which are retained, showing that there was no break in transmission.
This shows at the same time that Sri Lanka Malay is not a `young' language \citep[cf][]{AnsaldoEtAl2009age}, however defined, but of course the bulk of its morphosyntactic structure is of quite recent origin.

% 
% If `creolization' does not exist, and all languages which share a lexifier are just dialects of the same proto-language, in the sense of \citet{Chaudenson} or \citet{Mufwene}, one wonders why Sri Lanka Malay has so many features not found anywhere else in the Malay world, but curiously found in Sinhala and/or Tamil.

If `creolization' means the transmission of first language features into a target language (see Smith, this volume), it is difficult to see how Sri Lanka Malay fits the bill since the features transfered are not those which would have been expected under this assumption
 
None of the features mentioned in Section \ref{sec:disc:stage3} looks like anything which has been proposed for creole formation. Head-final word order is only rarely ever found in creoles, and has occasionally be argued to be an unlikely outcome of creolization, although this is due to the fact that the settings investigated included relatively few head-final input languages to begin with. Languages traditionally analysed as creoles typically involve a reduction of morphology, so that the increase in morphology in SLM and the semantic domains which need not be expressed morphologically do not really fit the bill and would rather suggest an origin different from what theories of creolization suggest. The development of word classes is also not something which is typically associated with creole formation. 

Second language learners are known to reduce the grammatical features not necessary for communication. The propositional content of a message does not depend on evidentiality, for instance. If imperfect learning was important for Sri Lanka Malay, we would expect some difficult features of Trade Malay (if there are any) to be imperfectly, or not at all,  acquired by Tamilophones. Imperfect learning does generally not suggest that learners invent new categories in the target language, on the other hand. German learners of English, for instance, are not known to invent gender or case when learning English, even if those features are important in their native language. Russian learners of English do not create aspect, Slovenians do not create duals. Germans, Russians and Slovenians simply strip their languages of their morphosyntactic complications instead of transposing them to English.

Nevertheless, it is true that we often do find structures in Creoles which are due to transfer of L1 features. Crucially, this only happens when the target language is used as a medium of interethnic communication, either between populations who do not share another language (e.g. Tok Pisin) or in clear power relations (e.g. between slaves and masters). Neither is found in Sri Lanka. Sri Lanka Malay was never used in communications which involved only non-Malays, and Malays never enjoyed the power and prestige to enforce their dialect upon the people who addressed them in a way similar to the European powers.

The setting where non-Malays spoke Malay is when they married into a Malay family (see Slomanson, this volume). But in this setting, the target language is not used in the way (inter-ethnic communication) which is required in most theories of Creole formation for substrate effects to take place.



% It is true that L2-learners do transfer structures when they do not have enough exposure to L1 speech. This structures can crystallize when the L2 as a medium of interethnic communication serving as a neutral code for speakers of a diverse linguistic background. Examples would be English in international organizations, or English as an official language of India. This does obviously not apply to the SLM case.
% The second setting in which a medium of interethnic communication is used is when there are clear power relation and differences in prestige between the languages. A slave would have to address the master in the colonial language, even if his fellow slaves all shared the same language. This setting does not apply to Sri Lanka either. There is no reason to assume than Trade Malay, a soldiers' jargon, enjoyed any higher prestige than Tamil, a language with a literary tradition of several millenia. Furthermore, the Malays did not dominate the Moors socially. It is true that the Malays were soldiers, but they were not in the same power relations to the Moors as the plantation masters were to the slaves. Given the sociolinguistic setting we find in Sri Lanka, both in earlier times and today, the most likely thing for a Moor would be to address a Malay in English (or another colonial language in former periods), or in Tamil.
%
% In order for L2 effects to take place in Moorish acquisition of Malay, one must make sure that 1) Moors wanted to acquire Malay, 2) they were not exposed to a sufficiently high quantity of L1 Malay speech, and 3) they used Malay to communicate with non-Malays, so that their L2 variety could stabilize instead of gradually converging towards the target language. In most cases, Moors had no reason to acquire Malay. Many Moors would have been  completely ignorant of the existence of this immigrant group (remember that the Moors outnumbered Malays 200:1). Those Moors who had an interest of learning Malays were those who married into a Malay family. But in those cases, condition 2) would not be met: when in a Malay family, the exposure to Malay speech would certainly be high enough to iron out substrate influence, especially since Trade Malay is a language which does not pose significant problems in acquisition (straightforward phonology, no morphology, no grammatical relations, no subcategorization etc).



% 
% 
%  This is for instance the case with Indian English, which, despite having native speakers, is mainly an L2. The existence of certain structures in Indian English which are carried over from the L1 of the speakers is undeniable. But this is due to the fact that Indian English is a medium of inter-ethnic communication. The same is true for English, French and Portuguese in the better known Creole-settings. Speakers from a linguistically diverse background had to learn a new language, the colonial language, in order to communicate. Such was not the case in Sri Lanka. Malay was by no means a medium of inter-ethnic communication. Sri Lanka Malay did not evolve because Sinhalese in Tamils did not share a common code and fell back on a third language. Furthermore, it is not the case that the Malays had high prestige and that the Moors had to speak in Malay to them, analogous to an African slave who had to address his master in the colonial English. Since the Malays were multilingual by necessity, the most normal thing for a Moor would have been to address a Malay in Tamil or English, as is still the case today.
% 
% It is true that there are a number of substrate features in Creole languages which do not involve simplification, e.g. the transitive marker \em -im \em in Tok Pisin.





\subsubsection{Metatypy}
As for (rapid) convergence and metatypy in Bakker's sense, I will treat them together here since the morphosyntactic argumentation will be the same. The speed of development relies on sociohistorical information and cannot be evaluated based on morphosyntactic considerations alone. \citet{SmithEtAl2006cll} and \citet{Nordhoff2009} have shown that some sociohistorical key assumption for Bakker's theory of rapidity are mistaken, and this argument will not be repeated here.

As for the compatibility of the morphosyntactic features with the theories, other instances of metatypy involve changes in word order (Takia/Waskia in PNG \citep{Ross1996,Ross1997,Ross2001,Ross2003diagnosing,Ross2007}, Cappadocian Greek in Turkey \citep{Dawkins1916}), so that the SLM facts jibe well with this theory. The development of new obligatory grammatical theories is also often observed in situations of metatypy or convergence. \citet[117ff]{Aikhenvald2002lc} for instance observed that the Arawakan language Tariana developed obligatory evidential marking through contact with Tucano (Tucanoan), a language which was established as a language interethnic communication by missionaries. Crucially, there is no indication that a significant number of Tucano speakers tried to acquire Tariana in the relevant period. Rather, Tariana speakers were familiar with the language of wider communication Tucano and found the evidentiality status of a proposition an important aspect  \citep[296]{Aikhenvald2004evid}, which they expected to be conveyed in their native language Tariana as well.\footnote{To be fair, speakers of Tucanoan languages also use evidentials when they speak Portuguese \citep[298]{Aikhenvald2004evid}, so that the transfer of L1$>$L2 is a possibility as well. It is clear, however, that the status of Sri Lanka Malay with regard to Tamil matches more closely Tariana's relation to Tucanoan than Tariana's relation to Portuguese.}

Similar things can be said about volitionality. In Sinhala, and to a lesser extent Tamil, volition is an important parameter of the grammar. Sinhala sentences typically indicate whether the action was performed with volition or not. This constant exposure to the parameter of volition led to the Malays' expecting this information to be conveyed; failure to do so would result in an infelicitous message and communication problems.

\citet{Nordhofffcjoat} argues that cognitive entrenchment of semantic entity classes with the prototypical acts of reference (for objects, yielding nouns) and predication (for actions, yielding verbs) through contact with Sinhala and Tamil is responsible for the development of rigid word classes in SLM.


\citet{Ansaldo2008genesis, Ansaldo2009book} argues that the structure of SLM can best be explained by metatypy with frequency effects, i.e. a shift away from Malay structures is more likely if Sinhala and Tamil have the same structure (e.g. dative) and less likely if their structure is different (e.g. accusative). For the list of Stage 3 features compiled above, this generalization seems to hold. The only item where SLM did not converge towards Sinhala and Tamil at the same time is the present participle formation by reduplication, which is found in Sinhala, but only marginally in Tamil.



%   \item \trs{dhaatha}{elder sister} ,\trs{mavol}{daugther},\trs{maven}{son}
%   \item \trs{kattil}{bed}, \trs{kuure}{roof}, \trs{kusini}{kitchen},\trs{marakari}{vegetables}
%   \item \trs{karcel}{problem},\trs{konnyong}{a bit}
%   \item \trs{nandu}{crab}, \trs{selendi}{spider}



\subsection{Stage 4: convergence towards Sinhala/attrition}
Probably already since 1873, but surely since 1956, Sri Lanka Malay has been  losing ground in the Sri Lankan ecology. In 1873, the Malay Regiment, which provided a privileged and sheltered place for Malay interaction and cultural life, was disbanded. The former soldiers found work in police and fire stations across the country, but the concentration of Malays in their new work places was far less than it was before.\footnote{This 
 development actually precedes the disbandment of the Regiment by a dozen years or so.
}
In 1956, the so-called `Sinhala Only' law changed the linguistic landscape of Sri Lanka. The Malays, who had up to then used Malay at home and English, and to a lesser degree Tamil, outside the home, had to become proficient in that language. In order to keep the economic advantage provided by English, the home language shifted from Malay to English, and Malay started losing domains. This process continues up to today, and there are many Malay children whose first language is Sinhala and who have only limited command of Malay.

The increased bilingualism in Sinhala is likely to have increased Sinhala influence on Malay, and it is probable that present-day SLM is more Sinhala-like than SLM was a hundred years ago. On the other hand, the extent of recent Sinhala influence should not be exaggerated: people born in the 1930s, who had acquired Malay before the nationalist policies were instated, speak pretty much the same way as younger fluent speakers, suggesting that the Sinhala structures in SLM were already present before 1950.

The processes at play here are convergence, but also attrition, as found for instance in Gaelic speaking communities in Scotland \citep{Dorian1989}. Since Malay is used less and less, and the domains of use become fewer and fewer, children do not acquire a full command of the language and do not know their ways in Malay in certain situations. In those situations, they draw on their general linguistic competence to find a way to convey the message. This general linguistic competence is based on Sinhala, so that in ad hoc formations, it is Sinhala, rather than Malay, which provides the communicative strategies. 

A good candidates for this process is the indefinite article. While it cannot be excluded that its use in Malay goes back to very early times, it is equally plausible that it is a recent development. Due to loss of competence in Malay \em hatthu \em was used every time when the speaker would have used \em ek\~{}ak \em in Sinhala. Using \em hatthu \em was not ungrammatical before, but it was not required. Through the use of Sinhala on a nearly permanent basis, this pattern became entrenched, and the knowledge about the optionality of \em hatthu \em disappeared. This change is very local and has no consequences on other parts of the grammar, so that it can very easily be quite recent.

The use of instrumental and accusative subjects is also quite local and could be recent. The accusative and the instrumental are of course much older, but their use with subjects could be due to Sinhala structures gaining importance. It is very easy to add a postposition to an NP, and has no repercussion elsewhere in the grammar, so that this change would not need a lot of time to take place.

Another change which can be explained by attrition is the dominance of the order ADJ N in younger speakers where older speakers also have N ADJ. The continuous exposure to the head final structure in Sinhala led to the dismissal of the historically dominant head-initial structure.
 

\subsection{Stage 5: independent developments}
Languages change through contact, but of course internal developments are also common. The fact that Sri Lanka Malay is a contact language does not mean that it is somehow immune to internal change. Since for the features listed for this stage contact is not a viable explanation, they have to be analyzed as independent developments and are of no further interest for the topic of this paper beyond the point made in the beginning of this paragraph

 


\section{Conclusion}
In this paper, I have shown that the current shape of Sri Lanka Malay is the result of at least 6 different stages. SLM can thus not be categorized into a handy drawer without being explicit about what particular stage one is talking. In order to come to grips with the structure of this language, a holistic approach must be taken, and the development of a language must be understood as a continuous rather than a punctual process. At different points in time, different processes are at work, and different demographic, social, and political circumstances lead to different behaviour by the speakers, resulting in different types of language change \citep[cf.][]{Migge1998jpcl,Migge2003,MiggeEtAl2008}.

The stage of most interest for creole studies is stage three. About a third of the `Sri Lankan' features are found in this stage. In this paper I have shown that the phenomenona of stage three are probably not the result of creole formation, but of metatypy or convergence.

