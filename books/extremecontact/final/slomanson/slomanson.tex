  
\chapter[Known, inferable, and discoverable in SLM research]{Known, inferable, and discoverable in Sri Lankan Malay research}
\chapterauthor{Peter Slomanson}{Aarhus University}

\section{Introduction}%1

A considerable amount of investigation has been carried out over the course of the preceding decade on the history of Sri Lankan Malay, its speakers, the grammar and lexicon of their language, and on its current prospects for survival (\citet{Ansaldo2008genesis,Ansaldo2011a,LimEtAl2006,LimEtAl2007,AnsaldoEtAl2009,Nordhoff2009,Slomanson2006cll,Slomanson2008lingua,Slomanson2009,Slomanson2011,SmithEtAl2004,SmithEtAl2006cll} \textit{inter alia}).\footnote{I
  am indebted to many people for the content of this paper, but in particular to B.A. Hussainmiya, who hosted me on a research visit to Brunei in the summer of 2008, providing me with extensive historical documentation and immeasurably increasing my understanding of Sri Lankan Malay history. I am also extremely grateful to the honourable Al Haj Muzni Ameer for permitting me to consult the marriage records discussed in this paper. I am naturally indebted to all the native speakers from whom I have learned over the years, but my understanding of the grammar of Sri Lankan Malay owes most to Mohamed Jaffar of Colombo and Toronto, and to Mohamed Thawfeek Mohamed Rihan of Kirinda. Mohamed and Rihan are not just excellent linguistic and cultural informants, but true friends. For discussion of the material in this paper, I would also like to thank Sebastian Nordhoff, Scott Paauw, and Romola Rassool, who are indispensable.
} 
In my own work, I have focused on morphosyntactic description and explanation, with special emphasis on comparison and contrast with the other languages of Sri Lanka. I have also worked on plausible explanations of various aspects of the language's development, based on fieldwork data collected in different communities over the course of several years. In order to explain how Sri Lankan Malay assumed its modern form, both grammatical and sociohistorical investigation are indispensable. We continue to make progress on a number of important questions, not least of which on the matter of glottogenesis and subsequent development. This is important because the grammar of the language differs so profoundly from the range of Malay dialects spoken in the ancestral areas of the Indonesian archipelago, and because the answers have the potential to increase the understanding in linguistic science of the nature of grammatical change under contact conditions.\footnote{Unlike
  my other publications on Sri Lankan Malay, this is not a paper whose goal is grammatical description, analysis, and explanation, but rather it is a paper on the field of Sri Lankan Malay studies itself, emphasizing linguistics and other research that contributes to our understanding of linguistic matters. The content of the paper is intended to be viewed as corrective, with respect to misconceptions, and programmatic with respect to future work in the field. Since it is not a grammar paper, I have included only one, hopefully illuminating, data example.
}

In section two, I will focus primarily on what is known and knowable, presenting points that bear on a lingering controversy over the external context that gave rise to the language, emphasizing this matter because misconceptions about its external history can give rise to misguided approaches to its internal development. In section three, dealing with what is inferable, I will discuss the need to construct plausible diachronic scenarios in the absence of actual attestations. In section four, I will discuss what is discoverable, including new synchronic information,  lexical information, as well as a broad range of sociolinguistic phenomena, including information on patterns of variation and change, as well as macrosociolinguistic information about when and how the language is used, and by whom.

\section{What is known?}%2

\subsection{External facts}%2.1

I will briefly and selectively discuss the history of Malay settlement in Sri Lanka. I will then discuss contact with Sri Lankan Muslim Tamil speakers (also known as Moors), a matter which has become a highly resolvable bone of contention. I will frequently use the term Shonam, the native glottonym for Sri Lankan Muslim Tamil, a variety which is historically distinct from other varieties of Tamil spoken in Sri Lanka. The controversy in question concerns the relative extent to which Shonam and colloquial Sinhala, respectively, have influenced the development of Sri Lankan Malay.

\subsubsection{Settlement history}%2.1.1

In some of my historical comments, I will use the term ``Indonesian'', meant to be understood here as ``coming from the region that includes the territories of what are now Indonesia, Malaysia, Singapore, and Brunei'', in order to avoid the ethnic connotations of the term ``Malay''. The Dutch colonial period in Sri Lankan history began in the mid-seventeenth century. We know that prior to that period there had been contacts between the Indonesian world and Sri Lanka, but that the actual modern Sri Lankan Malay community came into existence at the beginning of the Dutch period in Sri Lanka's history. We know that there were different social categories of migrants, and that their status under the Dutch administration was not the same. In spite of the fact that the ethnic Javanese among the migrants were unwilling political exiles, they represented a privileged segment of the community. Far more of the migrants were or would become soldiers, however, and throughout the period preceding the twentieth century, the Malay presence in Sri Lanka had an active military character. The decline in the economic stability and cultural autonomy of the Malay community in the late British period are attributed to the loss of this military role \citep{Hussainmiya1987,Hussainmiya1990}.

The military character of Malay history in Sri Lanka is part of what helps us to know more about the community's history than we would otherwise be able to know. If there is anything that colonial powers leave records of, it is their military goals and policies, if not the negative consequences of those goals and policies. The Netherlands East Indies Company (VOC), in addition to using coastal Sri Lanka and the South African Cape of Good Hope as places of exile for politically subversive individuals in Indonesia, also strategically used Indonesia as a source of military and other labor, including slave labor. This was based on a policy of maintaining relatively cordial relationships with indigenous elites and local populations in the colonial territories themselves. The Dutch refrained from enslaving and indenturing the Sinhalese in Sri Lanka, the Khoikhoi in South Africa, and the Javanese and Sundanese on Java in Indonesia \citep{Ward2009}. The local indigenous population were not to see local people being subjected to undesirable treatment at home. This policy resulted from political calculation, and is the colonial policy that is most responsible for the Malay presence in Sri Lanka.

\subsubsection{History with respect to the so-called Tamil bias}%2.1.2

Now we come to what is known, but which has recently been rendered controversial. This is the part of the historical picture that has to do with the relationship of the Malay community to the Moorish community, the much larger community of Sri Lankan Muslims, whose historic ethnic language, Shonam, is a contact variety of Tamil, which was written until the early 20th century in modified Arabic orthography, known as Arwi (also a term for Shonam, when written using this script, as is Arabu-Thamul or Arabic-Tamil, in which ``Arabic'' refers primarily to the orthography).

In the linguistic literature on Sri Lankan Malay, the authors Ansaldo and Nordhoff, respectively, have explicitly objected to an ostensible Tamil bias in Sri Lankan Malay linguistic studies. The phrase ``the Tamil bias'' is associated with \citet{Ansaldo2008genesis}, but Nordhoff, in his 2009 University of Amsterdam Ph.D. dissertation investigated the claim in detail, examining the external arguments for a closer connection to Tamil (Shonam) than to Sinhala. Ansaldo places great emphasis on the significance of Malay-Moor intermarriage in Smith, Paauw \& Hussainmiya (2004), going so far as to claim that the existence of this type of intermarriage is little attested and that the number of such marriages that there have been is neglible. This claim, essentially the criticism that the existing literature lacks evidence for one of its points of departure, relies on support from Ansaldo's own investigative work. I will review this approach and the findings that it yielded. I will subsequently address what I find to be missing from the discussion, as well as further points made in \citet{Ansaldo2008genesis,Ansaldo2011a} with which I disagree, based on evidence that I have found in the course of my research, and in that of the historian B.A. Hussainmiya.

\citet{Nordhoff2009} also takes issue with the idea that Malay-Moor intermarriage provided the external context for major grammatical change in Sri Lankan Malay. However in contrast with Ansaldo, he does not contest the fact that Malay-Moor intermarriage is a frequent occurrence and has been in the past. While Nordhoff has done part of the work of challenging Ansaldo's claim about intermarriage, by discussing such sources as native genealogical investigation that demonstrate the frequency of Malay-Moor intermarriage, I will continue to pursue the matter with additional evidence. I am doing this because the intermarriage issue is a major part of the claim that there is a ``Tamil bias'' that is unmotivated by available evidence, and because this claim continues to be repeated in print \citep{Ansaldo2011a}. Nordhoff concurs with Ansaldo in rejecting the view that Malay-Moor intermarriage could have created the conditions for glottogenesis, and concludes that there is little room, based on external evidence, for treating Shonam as of greater importance than colloquial Sinhala in the divergence of Sri Lankan Malay from vehicular Malay.\footnote{This
  term, which was coined by Ian Smith, was introduced into the literature on Sri Lankan Malay by Smith and Paauw, to refer collectively to the vernacular intercommunal Malay varieties of Indonesia. These were the varieties that were introduced to Sri Lanka in the seventeenth century. Contrasts between varieties spoken are likely to have been leveled through koineization, although we have no direct evidence of this process.}
I will present new external arguments for the case that a Malay-Moor symbiosis is the context in which linguistic change took place prior to Sri Lankan independence in the mid-twentieth century. With respect to Ansaldo's position, the best counter-evidence is nineteenth and twentieth century Muslim marriage records. With respect to Nordhoff's position, the existence of bilingual religious literature, as well as Shonam literature written by ethnic Malays, suggests a close relationship between ordinary Malays and Moors, but also between Malays and the Shonam language itself. This should demonstrate that, rather than an a priori bias, the case for the Shonam role in SLM glottogenesis is based on evidence.

I will divide this discussion into three categories: (a) intermarriage, (b) residential patterns, and (c) religious and cultural life.

\paragraph{Intermarriage and the ``Tamil bias''}%2.1.2 (a)

Living in close proximity and sharing a demanding devotional culture was conducive to intermarriage between the Moors and the Malays, regardless of frequently divergent occupational paths. In the earliest period, a poor ratio of male to female migrants must have led to considerable intermarriage, although we do not have statistics for this. Mixed marriages do not necessarily lead to a non-Malay identity or a loss of language loyalty. In the modern period, we find many examples of a pattern in which individuals raised in Malay/Moorish homes have been raised bilingually. The one parent, one language approach that happens to be favored by modern linguists seems to have been adhered to as a matter of course. The position adopted by Smith, Paauw and Hussainmiya is that Moorish mothers spoke L2 Malay which assumed target variety status for children.

The comments in this paragraph might well be categorized as ``inferable'', however I will include them here as a follow-up to the topic addressed in the preceding paragraph. I take Smith, Paauw, and Hussainmiya's position to be plausible. However it is more likely, if bilingual homes in an earlier period were actively bilingual, that there was convergence in these homes, so that after the earliest stage, the Malay that was spoken natively began to be more and more strongly influenced by L2 varieties. Children will have heard a considerable amount of L2-influenced Malay. Given the status of Shonam as a vehicular language for the dissemination of Islamic knowledge, the L2-influenced Malay of native Shonam speakers is unlikely to have been stigmatized, particularly in the absence of a normative Malay standard. What we now think of as standard Malay is based on a literary variety that had little general currency throughout the Indonesian archipelago, whereas vehicular Malay varieties have been widely spoken as auxiliary languages for centuries. Migrants to Sri Lanka are consequently more likely to have conformed to (new) local vernacular norms, beginning with pragmatic norms, in the absence of a normative Malay variety.

As early as the Dutch period, 1658-1798, members of socially prominent Malay families married into elite Moorish families (B.A. Hussainmiya, p.c.). As a rule, however, the Malays preferred endogamous marriages within their own community, and this is not particularly surprising. Marriages to Moors were far from taboo, as we can see from evidence in the \textit{thombos}\footnote{These
  are collection of household records, listing all household members. The practice predates the Dutch colonial administration. The records of greatest interest, all of which are written in Dutch, are numbered 1/3990 and 1/3991.
} maintained by the Dutch colonial authorities, \em pace \em claims made in \citet{Ansaldo2008genesis} and \citet{Ansaldo2011a}, and still farther from rare, as can be seen from voluminous evidence from the British period. It must be borne in mind that as compared to the number of such intermarriages between these two minority communities, marriages between the Malays and the majority Sinhalese were rare events. It is the available documentation from marriage registries maintained by both communities that reveals the frequency of Malay-Moor intermarriage.\footnote{Naturally,
  census materials and marriage records include no information on physical characteristics. It is worth mentioning though that, depending on the community, a significant proportion of Malay people appear far more classically Sri Lankan than Indonesian. This is particularly true in Kirinda, a community that is somewhat isolated from other Malay communities, and was founded at the beginning of the nineteenth century. This suggests that extensive intermarriage is not a recent development, although it may have occurred less in some communities than in others.

  At the beginning of the 19th century, Percival, a British military officer, remarked that

  ``Although they (Malays) intermarry with the Moors and other castes (sic) particularly in Ceylon and by this means acquire a much darker colour than is natural to a Malay; still their characteristic features are strikingly predominant.'' \citep[115]{Percival1803}.
}

What follows is a series of citations from \citet{Ansaldo2008genesis} with respect to the so-called Tamil bias.

\begin{quote}
 Bakker's claim of 'heavy Tamil pressure' may perhaps have been present in the S(ri) L(ankan) P(ortuguese) community, but there is absolutely no historical evidence that this occurred in the Sri Lankan Malay communities. 
\end{quote}

\begin{quote}
  The most specific claim regarding the creolization of Sri Lankan Malay is that it developed as a result of intermarriage between Malay men and (Tamil) Moor women \citep[e.g.][]{SmithEtAl2004}; this view is based primarily on the historical observations of \citet{Hussainmiya1987,Hussainmiya1990} regarding the records of marriage under the Dutch (thombos) which, according to him, show several cases of intermarriage between SLM and Tamil Moors.  
\end{quote}

\begin{quote}
 Unless the Tombo [sic] in Hussaimiya's [sic] possession reveal completely different data from the ones of the National Archives, it is safest to discount Hussaimiya's observations. 
\end{quote}

Hussainmiya has not claimed to own Dutch period records and manuscripts that are not in the possession of the Sri Lankan National Archives. His comments are based on what is found in the archives. It may be the case that Ansaldo saw different records from those that Hussainmiya, a professional historian, examined.

\begin{quote}
 While sharing a common religion may have played a role in individual marriages between SLM and Tamil Moors, there is no historical evidence to lend support to a claim of diffuse intermarriage between these two communities, especially of such a magnitude that could conceivably lead to restructuring of the vernacular.
\end{quote}


If by ``diffuse'', the author means \textit{extensive} intermarriage, there is in fact ample evidence of extensive intermarriage. While the number of records for the Dutch period is not great, those we have found provide evidence for Malay-Moor intermarriages, and no evidence of Malay-Sinhala intermarriages, although it was noted in an account by VOC employee \citet{Schweitzer1680} that these did take place \citep[11]{Nordhoff2009}.

Nordhoff makes a very cogent point, that ``the Dutch had just pushed the Moors away from the colonial cities'' \citep[37]{Nordhoff2009}. The extent of Dutch antipathy to the coastal Moors was great, because they represented highly skilled commercial competition, consequently the colonial authorities would eventually attempt to ban Moors from living in Colombo altogether. Portuguese antipathy to the coastal Moors had also been great, primarily for theological reasons, and nevertheless Colombo was largely a Muslim city at the end of the Portuguese period. The Dutch did not begin attempts to exclude the Moors from Colombo as early as they did from Galle however, and given the extent of Moorish presence in Colombo, it would have been difficult to exclude them entirely. I agree with Nordhoff that this detail weakens the case for extensive contact with Moors in at least part of the Dutch period, and that this contributes to a kind of historical quandary. It is quite clear that the history of Malay-Moor interaction is a complex one about which more needs to be learned. Ansaldo (2008, see quotation on page 8 of this paper), in his investigation of stated attitudes to intermarriage in certain circles, has come across what one can call an antipathy to a Moorish connection in certain urban Malay circles, although there is no evidence that that antipathy has any historical depth.\footnote{See
  further discussion of this antipathy later in this paper.
} 
\footnote{In
  my own work, I have naturally dealt far more extensively with the linguistic evidence for a Shonam model than with the historical details. I think this follows from the linguist's desire for straightforward historical answers, in order to get on with the business of linguistic research. Be that as it may, history in the real world is not always as cooperative as we might like, and this forces us, as contact linguists interested in diachrony, to get involved in complex questions that we would normally leave to historians.
}

The Malays were a very small Muslim population living in a foreign environment, maintaining an ethnic language in the absence of normative authority. With respect to external criteria alone, given simultaneous language maintenance, apparent domestic bilingualism in the periods for which we have evidence, and most significantly, the status of Shonam as a language of Islamic process and practice, it would be surprising if the grammatical organization of Shonam had not been replicated in Sri Lankan Malay, although the \textit{extent} of this replication continues to surprise observers. It is ultimately the interaction of external factors, rather than a single factor, that has lent itself to the outcome. The competing diachronic narrative proposed by Ansaldo is based on what he views as feature competition, of a type in which all definable surface properties of grammars are regarded as features. My objection to this rests on the implausibility of the external context (ecology, in Ansaldo's terms) and on the fact that the success of one feature or another has no particular trigger, either linguistic or sociolinguistic. This detracts from the potential explanatory value of Ansaldo's approach. Feature competition seems to be a restatement of the incontestable observation that Sri Lankan Malay owes some of its properties to Shonam and some to Sinhala. (I take what little unambiguous Sinhala influence that there is to be adstratal, i.e late, however I will put this matter aside for the moment.) What motivates success? If this is not stated, then this approach has no explanatory value.  



\begin{quote}
  As \citet{Hussainmiya1987} is the only work to directly, though briefly, address this issue, it is important to verify the claim in a precise manner. My investigation of the Dutch \textit{tombos} referred to in that work, for the period 1678-1919, in the National Archives at The Hague (microfilm copies of the Colombo archives) and the National Archives of Sri Lanka in Colombo yielded the following results:

 1.The records for the period up to 1796 are damaged by water, making parts of the entries illegible. The most revealing information for identification here are the signatures of the parties. There is, however, hardly any information of ethnic group, which makes it difficult to identify Malay/Indonesian and Moors given that both groups share the practice of adopting Arabic names. In a particularly interesting section in the \textit{tombos }dedicated to mixed marriages \citep[cf.][]{Hussainmiya1987}, only five of 238 entries clearly refer to individuals of Javanese origin: of these, two records refer to Javanese-Moor marriage, one to a Javanese-Javanese marriage, and the remaining two are unclear.

 2.The following period until 1919, albeit under British rule and therefore less interesting for our claim, shows a more structured archiving system where indication of race is given. Where legible, this reveals still a majority of Western marriages, a growing number of marriages between Eurasians and Burghers (locally born of Dutch/Western heritage), and between Burghers. There are two clear entries involving Malays, one married to a Eurasian (between 1867-1897), and one to a Burgher (1885-1897). From 1897 onwards, race is clearly specified; of 196 entries, only one is Malay.\footnote{The
  preceding sections 1 and 2 from \citet{Ansaldo2008genesis} are cited uncritically in \citep[42-43]{Nordhoff2009}. 
 }

\end{quote}

Of course Ansaldo is correct that historical claims require evidence, although his evidence ought to be retrievable, and this includes negative evidence (i.e. that documentation is \textit{not} available). In fact, the documentation in question is available.

The implicit assumption that religious identity could have been a peripheral consideration in Malay marriage arrangements is difficult to reconcile with knowledge of the way other Muslim communities in the region are organized. A strong tendency for Malays to marry Moors can be demonstrated for at least the nineteenth century onwards, due to the preservation of \textit{kaduthams},\footnote{The
  term \textit{kadutham} is a Tamil word literally meaning ``letter''. It has specialized meaning in Sri Lanka however, where it refers to a Muslim marriage certificate that is issued at the time of marriage. It was originally kept by the issuer, known as a \textit{khatib}. The \textit{khatib} is affiliated with a particular mosque, and responsible for its catchment area. These records were kept well into the twentieth century, and the text continued to be written in an orthography derived from that used for writing Arabic. Eventually, \textit{khatibs} began to keep marriage records in English that had (and continue to have) the character of logs, with a row for each new marriage.
} 
which provide clear evidence that Sri Lankan Malays who remained within the Malay community (i.e. who remained Muslims) have married Moors in great numbers, and that they have not married Sinhala converts to Islam in similar numbers. Since the imperative that Muslims marry Muslims has been well-documented in the ancestral Indonesian communities as well as in Sri Lankan Muslim communities, it would seem that the burden of proof rests with Ansaldo to demonstrate that communal religious culture and identity has \textit{not} been a primary consideration in Malay marriage arrangements, and has not been sufficient to privilege Malay-Moor marriages over Malay-Sinhala marriages. (The fact that Malays have married non-Malays is clear from the documentary record, so there is no question of complete ethnolinguistic endogamy.)

The evidence presented in statement 2 (see above) in \citet{Ansaldo2008genesis} follows from a misconception. There were in fact no \textit{thombos} in ``the period until 1919'', subsequent to 1796, the last year of the Dutch colonial administration. Under British colonial administration, Muslim marriage records were kept in mosques. Let us suppose that the content of statement 2 were accurate. It is not clear to me what could be meant by ``this reveals still a majority of Western marriages''. A potential explanation for the fact that Ansaldo only found ``two clear entries involving Malays, one to a ``Eurasian'', likely a Burgher, and the other to a Burgher (Burghers are Christian), is that the records that Ansaldo is describing are church records, such as those kept in Wolvendahl church in Colombo, and the ``two clear entries'' there do not involve Muslim Malays. The usefulness of this information is questionable, in contrast with the data to be found in mosques, where records of marriages involving Muslims is ordinarily kept, and has been since the beginning of the British period at the end of the eighteenth century.

With respect to demography and its effects, \citet{Ansaldo2008genesis} assumes that the language of the majority was necessarily acquired or influential, and that all communicative interaction is of equal\textbf{ }significance and intensity.

\begin{quote}
 Even if we were to accept the claim that Tamil might have been more closely involved in the evolution of SLM due to the fact that the language of the religious texts and practices would have been Tamil \citep{Hussainmiya1987}, \textit{the sheer numerical and social predominance of Sinhala in Sri Lankan society cannot be ignored} (emphasis mine, PS), and for this reason we should consider the two languages as, at least, adstrates of similar significance in the evolution of SLM.  
\end{quote}


Historians and historical sociolinguists will find evidence for the relative density and multiplexity of Malay-Moor networks in documentation collected by Hussainmiya. As it stands, the evidence for close Malay-Sinhala interaction is largely limited to military service in the Kandyan kingdom in the Dutch period and similar episodes. This involved a minority of Malays, since the original colonial function of the many Malay soldiers brought by the Dutch to Sri Lanka was to serve as non-native \textit{adversaries} of the Kandyan Sinhalese (see Bakker, this volume), so those who aligned themselves with the Sinhalese were those who managed to flee Dutch control. In the British period, by which time the Kandyan kingdom had come under colonial control, Malays were deliberately organized into all-Malay regiments \citep{Hussainmiya1990,Hussainmiya2008}, the purpose of which, again, was essentially to insure the submission of the Sinhalese.

Returning to the matter of intermarriage, Malays, as we have seen, frequently married Moors during the British colonial period and subsequent to it. In purely quantitative terms, the available evidence for Malay-Moor intermarriage during the preceding Dutch period, found in the \textit{thombos}, is less striking. It is quite clear however, based as it is on instances of explicit ethnic labeling (with the terms \textit{Javaan} (''Javanese'') and \textit{Moorman} (''Moor''), as well as on onomastic evidence. For example, some of the Moorish family names end in -\textit{poelle}, from Tamil -\textit{pillai. }By contrast, there are no Javanese or Malay marriages to Sinhalese in the \textit{thombos}. This is not to say that they never occurred, but the evidence for them has not been found in those records. It is for this reason that the burden of evidence rests with anyone wanting to show that there was extensive Malay-Sinhala intermarriage or that Malay-Sinhala marriage was not much less likely than Malay-Moor intermarriage. By contrast, documentation of Malay-Moor intermarriage (\textit{kaduthams} and Muslim marriage registers) is extremely extensive, whereas there is comparatively little evidence for any significant number of Malay-Sinhala intermarriages.\footnote{Members
  of non-Muslim ethnic groups have converted to Islam to marry Malays, however I have found no evidence that this was ever a frequent occurrence. The \textit{kadutham }records to which I have had access distinguish between Malay-Malay marriages and Malay-Moor marriages. The officiating \textit{khatib }would be aware of the fact that one Muslim partner was of Sinhala origin, or at least that s/he was neither Malay nor Moor, unless the partner was attempting to conceal this fact.
} Marriage records in mosques include the original name of Sinhalese converts to Islam, which is one reason that we can see in the records when a Malay married a Sinhalese person. The original name will always be unmistakable, though such marriages are far less frequent than marriages to Moors. If marriage to a Sinhalese convert was not viewed positively, marriage to a non-convert was tantamount to apostasy. Even if evidence were found that such unions were extensive, those were not unions that would have been able to contribute to accumulating changes in the form of SLM, since the new family would have been forced to weaken or sever existing network ties, with little potential for renewing ties with a Malay community. It is necessary to bear the cultural-ideological ``ecology'' in mind, in addition to the demography. While Hindus do not believe in one god, Buddhists technically do not believe in a god at all. It follows that though the extent of Malay-Sinhala interaction may have varied across periods and locations, it is most unlikely that this weakening of boundaries would extend to marriage, which would be the last boundary. None of this applies to Moors, irrespective of any cultural and occupational differences between the Malays and the Moors.

Ansaldo takes for granted that since Sri Lanka is a multi-ethnic society, this means that social, commercial, and other networks are \textit{necessarily} characterized by comparable relative levels of interaction across ethnic lines, so the presence of a greater number of Sinhala speakers means that Malays must have interacted with more Sinhala speakers or ``at least'' with as many Sinhala speakers as they did with Moorish Tamil speakers, and outside of religious matters, they must have done so in much the same ways. This is a post-industrial western view of social organization which bears little resemblance to the historical ecology of Sri Lankan communities.

It is ultimately inconclusive to introduce anecdotal evidence into such a discussion, however Ansaldo's sample of fifty families in which there is almost no intermarriage between Malays and Moors is problematic for a number of reasons.  In the first place, the socioeconomic class of the correspondents is a potentially relevant variable. Ansaldo has claimed that his oral history evidence was collected among the ``older generations''. What is the significance of this fact?  In order for a claim with respect to intermarriage to be conclusive, we would expect to see a systematic survey carried out in a range of communities, certainly including substantially more than fifty families, in order to ascertain what Malay attitudes toward Malay-Moor intermarriage might be in the present period, and to find out what proportion of families interviewed have at least one Moorish member.  The paragraph referred to above reads as follows,

\begin{quote}
 In addition to the historical record, clear evidence against a solid Tamil influence in the development of the SLM community comes from oral history recorded in three different SLM communities: in Kirinda, Colombo and Kandy, of approximately 50 families interviewed in total, only two revealed genealogies including Moor-Malay intermarriage.  Most families report that marrying outside the SLM community was considered taboo and only allowed in extreme cases.  It is only in the present generations that weddings outside the community start being allowed.  Moreover, the Moors appear to have had very low status in the eye of the SLM communities and, their low status may be seen as a counteracting force to the hypothetical appeal of religious affinity.  Moreover, in at least one community - Kirinda - intermarriage with Sinhalese is well attested in the history of several families. 
\end{quote}

In order for a pattern of ethnolinguistic endogamy to tell us anything significant about a general tendency or reluctance to marry members of another Muslim ethnic group (in this case Moors), members of the first ethnolinguistic group (Malays, in this case) must have access to potential out-group partners. Given the demography of the village and of the adjacent rural district, results from Kirinda can only skew the outcome of an investigation of this kind, and not in a way that can contribute to answering the question Ansaldo asks (i.e. to what extent Malays are prepared to marry Moors) which logically \textit{presupposes} that the Malays in question have access to potential Moorish partners. In fact virtually the only Moors present in the village are teachers in the village school who reside elsewhere, and members of the community are not aware that this was ever different. Consequently Ansaldo's Kirinda evidence cannot have the same status as genealogical evidence from elsewhere on the island. The Kirinda community was founded over two hundred years ago, in a then unpopulated rural coastal area, and as Ansaldo himself has written, the community has been rather isolated.\footnote{The
  area is of high symbolic importance in Sinhala Buddhist history, and there is a Buddhist temple on the water. Sinhala village life in the immediately surrounding area is historically rather recent however.
} What opportunity would its members have to marry Moors, without either settling far from the village or bringing in spouses from far afield? Most other Malay areas, by contrast, have historically been mixed Malay-Moor areas. In the present, the surrounding area is far less sparsely populated than in previous generations, however the present-day surrounding communities are entirely Sinhala-speaking and Buddhist.  If Ansaldo had asked the same question in the heavily Muslim town of Hambantota, he would have found a substantial number of Malay-Moor marriages.\footnote{One
  of my consultants in the village, a young woman who is a schoolteacher in the village, is engaged to marry a Moor. She went to school in Hambantota, and in that sense, her personal network extends beyond the village. This seems to be completely unproblematic. The village is quite pious, and increasingly traditional, due to the influence of itinerant Moorish preachers whose Friday message in Shonam is spread by loudspeaker from the village mosque. It is difficult to imagine how marriage to a Buddhist would be received, and asking how it would be received would be quite awkward.
}

Only a large-scale study to obtain quantitative data would provide complete closure in this matter, however social history, and for that matter anthropology, do not require quantitative data, but rather intensive engagement with communities and plausible interpretation of qualitative evidence. Almost every Malay family I have met in Sri Lanka, in a range of urban and rural communities over the course of eight years of active involvement, one calendar year of field research, and many additional research visits, contains at least one Moorish member. This is minimally the basis for a hypothesis that if there is any Malay-Moor marriage taboo at all, it is an exceedingly weak one. The families of Kirinda are the exception to this otherwise ubiquitous pattern, for the reasons stated above. Aside from recent marriage patterns, the core of Ansaldo's claim that there has been minimal or no intermarriage with Moors is also not born out by the \textit{historical} evidence, which is the variety of evidence at the core of his claim (because this is most relevant to the claim that a Tamil variety was the primary model language for the grammatical changes that took place in Malay as it came to be spoken in Sri Lanka).  I have already referred to the \textit{kadutham} evidence. Ansaldo examined two sources. The first source examined by Ansaldo consists of evidence from the \textit{thombos} (as discussed earlier). The second source, as we have seen, is British, according to Ansaldo's description.    With respect to evidence in the \textit{thombos}, \citet{Ansaldo2008genesis} states

\begin{quote}
  The first occurrence of a claim about Malay-Moor intermarriage occurs in \citet{Hussainmiya1987}: ``a number of'' these marriages are reported, next to Malay-Ambonese/Malabarese/Sinhalese unions recorded in the Dutch Tombos. The same work however suggests that SLM may be influenced by Sinhala, Tamil or both.
\end{quote}


I have seen no evidence of Malay-Ambonese/Malabarese/Sinhalese unions in the Dutch \textit{thombos} that I have examined myself in Colombo. This claim by Ansaldo may have been precipitated by the following well-known historical comment by the Dutch East Indies Company employee \citet{Schweitzer1680}, as discussed in \citet{Nordhoff2009}.

\begin{quote}
 The wives [of the Malays], which in part are Amboinese, in part Singulayans [Sinhalese], 
 and Malabarians [South Indians] may say nothing against [the stripping of their ornaments].
\end{quote}


With respect to another claim from \citet{Ansaldo2008genesis}, cited above (1), there is no section of the \textit{thombos }''dedicated to mixed marriages''. The Dutch did often add the designations \textit{Javaan} (''Javanese'') and \textit{Moorman} (''Moor'') when listing the members of households in particular areas in Colombo. These entries need to be read carefully because they listed unrelated household members such as servants as well. This is part of the argument for bilingual Malay-Shonam homes, although it was not made in Smith et al. The concept of a modern nuclear family is anachronistic in this discussion. In my own investigation, I found no water damage, although the florid Dutch script is easier to read in some records than in others. For records in which there is no explicit ethnic designation, it is still fairly easy to discern ethnicity. During the relevant period (1) there were more Javanese than there would be later, (2) the Javanese were likely to appear in the \textit{thombos}, because of their status relative to mere soldiers, (3) Javanese family names were not Arabic names and are in fact quite distinctive, and (4) Moor and Malay names are distinguishable in a number of ways. For example, \textit{Lebbe} (''priest'') appears again and again, as does \textit{Marikar}.\footnote{Nordhoff
  (2009:43) states ``I am not aware of any names exclusively borne by Moors.''
} These are associated with Moors only. Most usefully, in marital records from the Dutch period to the present, Malay titles appear again and again, particularly \textit{Thuan}. \textit{Thuan} was never born by Moors, and the female title \textit{Gnei} was never born by Moors either. The words themselves are Malay words (\textit{tuan} and \textit{nyi}, respectively in vehicular Malay). While it is true that there are Arabic names that can obscure ethnic origins, these are relatively infrequent, the list of family names exclusively born by Malays is quite long, and there are plenty of onomastic markers of Moorish identity as well.\footnote{The
  word is \textit{nyi} in Indonesian spelling, and was previously a title for any woman, although the word has since taken on a number of narrower meanings.
} 
\footnote{Ansaldo
  may have been influenced by his field experiences in Kirinda, where the names born by Malays are Arabic ones. However, against the general Sri Lankan Malay pattern, the names are concatenated in a specifically Moorish way. For example, Mohamed Thawfeek Mohamed Rihan, a consultant and close friend,  bears the given name Rihan. The preceding names are the name of Rihan's father. Rihan's brothers and sisters were all named in accordance with this pattern. This is the Moorish pattern and most certainly does not attest to any aversion to Moorish cultural traits.
}

Ansaldo comments further that

\begin{quote}
 While necessarily brief, the report above of the contents of the Tombos shows that there is hardly any reason to comment on the nature of intermarriage of the Malays in general, and less even to make specific claims about the origins of the parties.  While sharing a common religion may have played a role in individual marriages between SLM and Tamil Moors, there is no historical evidence to lend support to a claim of diffuse intermarriage between these two communities, especially of such a magnitude that could conceivably lead to restructuring of the vernacular.
\end{quote}

Recall the following comments by Ansaldo, repeated here for convenience.

\begin{quote}
  My investigation of the Dutch Tombos referred to in that work \citep{Hussainmiya1987}, for the period 1678-1919, in the National Archives at The Hague (microfilm copies of the Colombo archives) and the National Archives of Sri Lanka in Colombo yielded the following results:
\end{quote}

There were no \textit{thombos} kept in 1919, since that was over a century after the British took control of Sri Lanka from the Dutch, and as is well known, the \textit{thombos }ended with the Dutch. The relevant microfilmed \textit{thombo} documents are not accessible at the National Archives at the Hague. The function of the microfilming project is to store back-up copies in the Netherlands, and there are no plans to make these accessible to researchers.

\citet{Ansaldo2008genesis} states with respect to ostensible documentation from the British colonial period,

\begin{quote} 
 The following period until 1919, albeit under British rule and therefore less interesting for our claim, shows a more structured archiving system where indication of race is given.  Where legible, this reveals still a majority of Western marriages, a growing number of marriages between Eurasians and Burghers (locally born of Dutch/Western heritage), and between Burghers.  There are two clear entries involving Malays, one married to a Eurasian (between 1867-1897), and one to a Burgher (1885-1897).  From 1897 onwards, race is clearly specified; of 196 entries only one is Malay.

\end{quote}

In the first place, had there been British period \textit{thombos}, it is not clear why the British period should be less interesting.\footnote{\citet{Ansaldo2011b}
 refers to \citet{Slomanson2011} as ``criticizing archival data in support of a Tamil bias'', In fact, I stated that British period thombo data on marriages do not exist, and therefore are not found in archives at all.
} 
In the second place, it is surprising that Ansaldo implies that only people of whole or partial European origin were married in the relevant period, or that only records of the marriages of such people were kept.  In fact, the keeping of marriage records during the British period was communalized. This means that in order to find out who was marrying whom in Muslim Sri Lanka, it is necessary to consult the \textit{khatib} for records. These were typically kept in Malay or Shonam in modified Arabic script, as I have described. The often extreme extent of Malay-Moor intermarriage in the nineteenth and twentieth centuries is of great relevance in this discussion, because of the high incidence of bilingualism and language maintenance in Malay-Moor homes, yielding ethnically-mixed offspring, bilingual in Moorish Tamil and Sri Lankan Malay. 
% [The following passage repeats what has been said before and should be deleted in the interest of readability]
% \tiny 
% Note that there are factors that fall outside the realm of scientific investigation, but that are strongly suggestive of intermarriage prior to the nineteenth century. The people of Kirinda are predominantly southern South Asian in appearance, and unlike many upcountry Malays, few actually give the physical impression of Indonesian ethnic origin. The significance of this lies in the fact that Kirinda was settled by Malays just after the beginning of the British period. It has been a Malay village for over two hundred years, and historically was an isolated one. It is not logical to claim that Malay-Moor intermarriage is historically taboo, and at the same time claim that there was intermarriage with Sinhala people (who are Buddhist), even in Kirinda, which had been an isolated Malay settlement. Intermarriages that preceded settlement in Kirinda were far more likely to have involved Malay-Moor unions than Malay-Sinhala unions, for non-trivial theological reasons.  As for the \textit{kaduthams}, they were kept until at least the first two decades of the twentieth century, after which register records in English began to be kept. These are available for inspection, although only in the mosques at which the records were produced in the first place. As we have seen, Ansaldo claimed that for any evidence of Muslim marriages identified by him, it would be difficult to determine the ethnic identities of the bride and the groom, due to the practice of taking Arabic names. In addition to names that are simply Arabic, and therefore ethnically ambiguous, there are many surnames that are distinctively Sri Lankan Malay, there are many unmistakably Javanese names among Sri Lankan Malays, and there are names that were only borne by Moors. Additionally, Sri Lankan Malays had, and continue to have, distinctive titles, including \textit{Tuan} for men and \textit{Gnei }for women. These titles appear in many marriage records. 
\normalsize
With respect to both to the \textit{kaduthams} and the registers, there is explicit evidence of ethnicity.  In the registers (as opposed to the older \textit{kaduthams}) that are kept at the Wekanda Jumma mosque in Slave Island, the practice has been to maintain separate records for Malay-Malay marriages and for marriages involving non-Malays. The fact that Malay-Malay marriages are a category of its own reflects the fact that the mosque is a Malay mosque in a Malay area in Colombo. The records of mixed marriages are most interesting, since they contain many marriages that are indisputably marriages to Malays, based on the onomastic evidence, and this is true regardless of the year. A representative Moorish register page from 1990 (Figure 3) contains eighteen marriages.\footnote{Although
  in several cases, the ethnic identities are unmistakable, as in the entries in which the bride's name includes the SLM title \textit{gnei}, the onomastic evidence was reviewed by Mohamed Jaffar, who grew up bilingual in SLM and Shonam, in an ethnically mixed neighborhood in Colombo.
} 
\footnote{This
  does not represent all the Moorish marriages for that year.
} 
Of those eighteen marriages, eight are to Malays. This is only a representative sample, as there are several pages of records for that year. There are two marriages of Moors to converted Sinhalese.\footnote{There
  is no evidence for marriages of Malays to Sinhalese, although this does occur infrequently.
}  
I am assembling statistics for a range of years representing several periods.

The function of the above reference to a subpart of one year is that it demonstrates that there is no lack of evidence for Malay-Moor marriages or for a cultural taboo that would have prevented them from occurring. Had there been such a taboo, that would have weakened the external (socio-cultural) basis for investigating a variety of linguistic change. That is the variety specifically motivated by Malay-Shonam bilingualism and replication of a range of structural characteristics found in the Shonam language. What the evidence compiled thus far suggests is that there was (and continues to be) a \textit{preference} for marriages to Malays within the Malay community, but this preference has not precluded marriages to other Muslims. The intra-ethnic preference is most likely to be evident in the records maintained in a Malay mosque in a Malay neighborhood. However even at this Malay mosque, we find evidence for a large number of marriages of Malays with Moors. With respect to the ostensible taboo, it is likely that class needs to be considered as a variable. Questionnaires demonstrating the presence of a taboo were distributed at the Colombo Malay Club. This renders the results questionable because the club's members are socioeconomically removed from the general Malay community. It is the professional middle class of Colombo Malays that increasingly sought to distinguish itself from the Moors in the previous century, partly due to the goal of retaining an independent voice in political life \citep[16-18]{Hussainmiya1987}. The professional class constitute an elite with collective interests that are to some extent removed from those of the general Sri Lankan Malay community, and its views therefore not be treated a priori as representative of views held by the Sri Lankan Malay community as a whole. In addition, antipathy to the Moorish community in this group is the product of twentieth century communal politics, rather than events predating that period. 


\begin{figure}
\includegraphics[width=\textwidth]{\imgpath slomanson-img1.jpg}
\caption{A Malay \textit{kadutham} from 1919.}%1
\end{figure}
 
\begin{figure}
\includegraphics[width=\textwidth]{\imgpath slomanson-img2.jpg}
\caption{A page from a Malay marriage register.} %2
\end{figure}
 
\begin{figure}
\includegraphics[width=\textwidth]{\imgpath slomanson-img3.jpg}
\caption{A page from a Moorish marriage register in a Malay mosque, in which almost half of the marriages are Moor-Malay marriages.}%3
\end{figure}
 

\paragraph{Residential patterns}%2.1.2 (b)

The concept of the ethnically-specified settlement within and adjacent to an urban area was familiar to Malays prior to migration to Sri Lanka, as it characterized the spatial organization of Batavian (i.e. Jakartan) districts (\textit{kampungs}) on Java. Even the settlement of Malays in urban areas within Sinhala-majority regions does not guarantee that some degree of spatial proximity to Sinhala people could play a glottogenetic role. According to Shahul Hasbullah (p.c.), a demographer affiliated with the University of Peradeniya, a comparable pattern was in existence historically even in predominantly Moorish communities, where the designation Muslim, not an official category as it became in the twentieth century, was nevertheless a defining category for Muslims themselves, and this contributed heavily to where Moors chose to reside, in a way that language did not. Malay residential patterns in most areas have been characterized historically by spatial proximity to Moors and extensive social and cultural interaction with them. After the disbanding of the Malay regiment in the late 19th century, some Malay families fanned out to new areas especially in the hill country, to work as overseers and security personnel in the newly opened plantations of rubber, coffee and later tea estates. This fact in itself is significant, with respect to the Malay relationship to varieties of Tamil. Malays easily performed an indispensable middleman function between essentially monolingual English-speaking plantation owners and monolingual Tamil-speaking plantation workers, since the Malays spoke a Tamil variety (Shonam), although they were not ethnically Tamil themselves, and since they were frequently fluent in English as well.

\paragraph{Religious and cultural life}%2.1.2 (c)

In a number of communities, Malays founded their own mosques, in which Malay-speaking religious officials were to officiate, so that Malay could be used in administering religious rites. This fact is cited in \citet{Nordhoff2009}. However Malay mosques did not exist in Sri Lanka prior to the end of the Dutch colonial period (almost a century and a half into the Malay settlement in Sri Lanka), and these have never been exclusively Malay institutions, but rather were shared (B.A. Hussainmiya, p.c., Al Haj Muzni Ameer, p.c.). Malays and Moors developed a kind of symbiotic relationship, which only began to break down in the last century, under the pressures of ethnopolitical communalism. Despite the fact that most Moors did not speak Malay, the two groups shared a common religious culture, a fact that held greater cultural significance than the fact that their occupational lives often diverged.\footnote{On
  the east coast and in other Sinhala-majority areas, Moors have been associated historically with commerce, whereas Malays were most closely associated with military activity, plantation supervision, and police work.
} 
This is very different from following a single confessional tradition in modern western terms, since the extent to which daily life revolves around the organization of spiritual life and its strictures was far greater than what is familiar to relatively secularized people nowadays. Both groups belonged to the Sh\=afi'{\=\i} sect and their written religious literature used similar Arabic-based orthographic systems. The Malay religious texts were written in Jawi and the Moorish texts were written in Arabic-Tamil, also referred to as \textit{Arwi} \citep{Shuayb1993}.  Even the content of the religious texts was frequently identical, some being direct translations from Arabic-Tamil to Malay written in Jawi orthography, and vice versa.

It is significant that the Tamil Muslim-Indonesian religious and cultural symbiosis that we find evidence for in Sri Lanka extends beyond Sri Lanka and it may be that Muslim Tamil culture was familiar to the Indonesian immigrants prior to their arrival in Sri Lanka. The culture that became known as ``Moorish'' in Sri Lanka was not restricted to that island, but flourished and spread as the seafaring mercantile culture of Tamil-speaking Muslims of Arab and South Indian descent. Coastal southern India (''Malabar''), coastal Sri Lanka, and coastal area in western Indonesia and Malaya were its focal areas. In the Indonesian context, evidence exists supporting Muslim-Tamil and Indonesian (\textit{inter alia} Malay and Javanese) interactions, gleaned from close readings of textual sources in Javanese, Tamil and Malay. In the Sri Lankan context, Malays and Moors shared religious and literary texts. This is a matter for which we have hard evidence, as many of the texts still exist.\footnote{A
  not insignificant number of these are in the possession of B.A. Hussainmiya, who may compile and publish an annotated collection of these texts.
} 
A recent dissertation by the literary historian Ronit Ricci \citep{Ricci2006} describes the importance of literary links between the Javanese and Malay world on the one hand and the Tamil-speaking Muslim world in southern India and Sri Lanka on the other, based on common religious texts belonging to Tamil-speaking Muslims, the Malays and the Javanese in the Indonesian archipelago.\footnote{The
  footnotes on the following page are Ricci's.
}

\begin{quote}
 The coasts of Southeast India and Indonesia were part of the Indian Ocean's commercial network that was permeated -- beginning in the fifteenth century -- by an Islamic ethos, where goods and shared texts and values crossed the seas carried by Muslim merchants, pilgrims, soldiers and scholars, and where coastal towns, which functioned as important trade centers and ports, developed into major centers of Islamic learning and culture.\footnote{On
  the relationship between trade and Islam in these regions see, for example, Andre Wink, {\textquotedblleft}'\textit{Al-Hind}'. India and Indonesia in the Islamic World Economy, c. 700-1800 A.D.{\textquotedblright} in \textit{India} \textit{and Indonesia During the Ancien Regime }(Leiden: Brill, 1989) 48-49, and Kenneth McPherson, \textit{The} \textit{Indian Ocean: A History of People and the Sea }(New Delhi: Oxford University Press, 1993) 76-78.
}

The Muslims of South India and the archipelago shared a variety of relationships: from at least as early as the seventeenth century they had a shared set of pilgrimage sites, some of which are still popular today. Well known in South India is the lineage of the seventeenth century sufi mystic sheikh Sadaqatullah of Kayalpattnam, whose tomb continues to attract devotees from Malaysia and Indonesia; Muslims on both shores maintained mutual trade contacts, with the Nagore-Acheh route becoming one of the most profitable in the eighteenth century network. The Marakkaiyar trading clans of the Coromandel coast, claiming Arab seafarers and traders as their ancestors, had well established ties with the Muslim ports of the archipelago, exporting tobacco, cotton textiles, gems and pearls; men of the Tamil coastal towns traveled to Southeast Asia, sometimes remaining there for months at a time; members of the two communities even inter-married, the Marakkaiyars preferring intermarriage with the Muslims of the archipelago over marriage with the lower strata of Tamil Muslim society.

The \textit{madhab} (school of Islamic law) followed by Javanese and South Indian Muslims living along the coast is one and the same (Shafi'i) contacts in the sphere of Islamic education appear to have been strong, with similar institutions emerging in Tamil Nadu, Sumatra and Java; Indonesian pilgrims on their way to Arabia used to stop in the Maldives; in the eighteenth century a Coromandel mosque existed in Batavia, while in the early nineteenth century an approximate 2\% of Batavia's population were {\textquotedblleft}Moormen{\textquotedblright}, natives of the Coromandel and Malabar coasts; under colonial auspices contacts  {}-- whether through trade or the deployment, employment or exile of subjects. 
\citep[12-13]{Ricci2006}
\end{quote}

\begin{quote}
A Tamil tradition credits Umar Wali of Kayalpattnam with establishing Islamic schools in Sumatra, where he spent fourteen years in the mid eighteenth century. When Shu'ayb visited the region in 1978 he noted the similarities in curriculum between the contemporary \textit{pesantren} of northern Sumatra and the earlier \textit{madrasah} schools of Tamil Nadu and Sri Lanka, some of which no longer taught certain previously shared texts on account of North Indian influences on Muslim education in the south.

Religious teachers often traveled in a quest to disseminate their knowledge and religious convictions to others, expanding the geographical and cultural limits of the cosmopolis. It is known from the \textit{Sejarah} \textit{Melayu} ({\textquotedblleft}Malay Annals{\textquotedblright}) that Tamil Muslim teachers were influential in the Malay regions in the fifteenth century. The {\textquotedblleft}Annals{\textquotedblright} also claim -- as does the \textit{Hikayat} \textit{Raja-Raja Pasai} ({\textquotedblleft}Book of the Kings of Pasai{\textquotedblright}) -  that the apostles of Islam reached  Malay shores from the Coromandel coast.\footnote{Robson, \textit{Java} 262.} Shuayb discusses at length the deeds of the above-mentioned Umar Wali, a Tamil `saint' who spent years in the forests of Sumatra, propagating Islam\footnote{Shu'ayb 502.}; Bayly mentions a Tamil \textit{pir} from Vethalai who, while meditating in a Sumatran jungle, encountered and overcame a fierce elephant. In gratitude the  sultan granted him his daughter in marriage and nominated him as successor to the Achehnese sultanate\footnote{Bayly, \textit{Islam} 155.}; Javanese nobles and their retinues, exiled to Sri Lanka in the seventeenth and eighteenth centuries, brought with them -- if not in written certainly in oral form -- stories and traditions which were eventually shared with other Muslims on the island.\footnote{Hussainmiya, \textit{Orang} \textit{Regimen }38-42.} Although the historical accuracy of some of these mentions cannot always be determined with certainty such traditions attest to a sustained memory of participation in promoting Arabicized networks of language, literature and learning that connected Muslims across the region. \citep[388-389]{Ricci2006}
\end{quote}

The best evidence for a close religious and cultural relationship, a symbiosis between the Malay and Moorish communities in Sri Lanka, is the texts that exemplify the literary life of the relevant communities, just as the work of Ricci demonstrates connections between the Javanese Muslim world and the Tamil-speaking Muslim networks generally. While we may note to our consternation that there is too little direct evidence in literature of what \textit{spoken} Sri Lankan Malay looked like prior to the twentieth century, there is nevertheless extensive direct literary evidence for the Malay-Moorish symbiosis.\footnote{For a methodological critique of recent claims denying Malay-Moor intermarriage, see \citet{Rassooltv}.} This is discussed briefly in the introduction to \citet{Hussainmiya2008}.

\begin{quote}
 Still more important is the fact that the texts [ i.e. those associated with the Malay literary tradition ] revealed the strong links between the Malay community and the Tamil-speaking Sri Lankan Moors. Both shared many features of a literary tradition. They read and exchanged kitabs (in the sense of Islamic religious works) and other literary manuscripts. The Moorish texts were written in Arabic-Tamil, employing an adapted Arabic orthography similar to the one in which Malay was written. So Malays easily read Arabic-Tamil, as they knew Tamil, and were familiar with the script. Interestingly, the Malays in Sri Lanka pioneered the publication of trilingual instruction booklets in Arabic, Malay and Arabic-Tamil. They even published Arabic-Tamil newspapers such as Ajaib as-Sailan and Ummai.\footnote{Ummai
  (''The Truth'') is the first Arabic-Tamil (i.e. Shonam/Arwi) newspaper in Sri Lanka. It was edited and published by a Sri Lankan Malay.}
The Malay Jawi newspaper Alamat Langkapuri (1869-1870) carried letters and notices in Arabic-Tamil. Moreover, it was not uncommon to find works of Malay and Arabu-Tamil literature bound in the same codex of manuscripts owned by the Malays. They used texts on medicine, magic, and religious formulas written in Arabic-Tamil. Apart from the need to use Arabic liturgical works such as Subhana Maulud (eulogy on the Prophet Muhammad) a Malay intellectual such as Baba Ounus Saldin translated Arabic-Tamil works such as Gnanamani Malai into Malay for wide use among his colleagues. There were erudite Tamil poets among the Malays who wrote their works in Tamil. Poets such as Jumaron Tungku Usmand, who lived close to the Indian Tamil labourers in the hill country plantation districts, composed Malay songs in romanised Malay by employing folk Tamil literary forms of Kummi and Temmangu.
\end{quote}
 
\begin{figure}
\includegraphics[width=\textwidth]{\imgpath slomanson-img4.jpg}
\caption[Glossary page from a nineteenth century text used to teach Arabic in Qur'anic schools attended by Malays]{Scan of a single glossary page from a nineteenth century text used to teach Arabic in Qur'anic schools attended by Malays. From right to left, the first column is in Arabic, the second is in Malay, and the third is in ``Arabic-Tamil'' (an alternative name, as is Arwi, for Shonam written in Arabic script). Courtesy of B.A. Hussainmiya.}
\end{figure}
 



According to the Sri Lankan Malay linguist Mohamed Jaffar (p.c.), the deep cultural and spiritual connection between Sri Lankan Malay and Shonam in the oldest generation of Malays in Sri Lanka, even in Malay-Malay marriages, needs to be understood by linguists who may assume that Shonam is merely one equal component in the linguistic experience of Malay Muslims in Sri Lanka, as opposed to an intimate part of domestic life. Jaffar speaks of ``the prevalence of Shonam in their lives, all of it connected with religious practices: the recitation of \textit{doa} \textit{'aashura}, a supplication in the first month of the Islamic calendar, \textit{muharram}, in memory of the martyrdom of the Prophet's grandson Husayn ibn 'Ali at the battle of Karbala the recitation of the \textit{thali }\textit{faatiha, }a song in praise of the Prophet's daughter F\'athima, in both Arabic and \textit{Arwi} (\textit{Arabu-Thamul}) authored by Sayyid Muhammad ibn Ahmad Lebbai `\'alim-al-Q\'ahiri-al Kirk\'ari, great-grandfather of Shu'ayb Alim; and several other publications, usually Arabic-\textit{Shonam} bilingual, numerous to bear mention here, however, easily obtainable. As recently as in 1963 to be precise, the late Mr Saifuddin J Aniff-Doray --- a Sri Lanka Malay notable and one-time Principal of Zahira College, Colombo --- translated into English from \textit{Arabu}{}-\textit{Thamul} the voluminous ``Fat-aud Dayy\=an'' (Fata al-Dayyaan) of Sayyid Muhammad ibn Ahmad Lebbai mentioned above. The evidence and arguments that I have presented in this section is meant to demonstrate that exposure to involvement of Malays in the religious and literary culture of Tamil-speaking Muslims is not restricted to Sri Lanka, but was an Indian Ocean regional phenomenon. Not surprisingly, this involvement was perpetuated in  Sri Lanka, a context that needs to be understood in order to see that the linguistic behavior of Moors is likely to have been prestigious within Malay networks. It is also necessary to appreciate this in order to see that intensive interaction between Malays and Moors was greater than a husband, wife, and child domestic bilingualism configuration, but rather extended outward from the home to involve those Malays who did not marry Moors. Shonam was in fact a process language for religious practice among Malay in Sri Lanka, as it continues to be in the most conservative communities, including the all-Malay village of Kirinda.

\section{What is inferable in SLM research?}%3

Much of the diachronic analysis of the grammar of Sri Lankan Malay is inferable, and it is fortunate that this is so. In Malay vernaculars in general, including contact Malay varieties, not just in Sri Lanka, investigating the internal history of a variety is rendered more difficult by a highly diglossic culture. We have few attestations for Malay in Sri Lanka. What we do have is associated with the nineteenth and early twentieth centuries, and the Jawi texts from that period are written in what is essentially literary Malay. The British colonial government encouraged Malay literary activities in order to create a culturally favorable climate for the military personnel that were being recruited in Malaysia. The diglossia that characterizes Sri Lankan Malay literary culture during this period is also characteristic of Tamil and Sinhala literary culture. The written varieties of those languages have since been disseminated across the island as a result of the universal vernacular education policies of successive post-independence Sri Lankan governments. While literacy in at least one of the two official languages has been rendered nearly universal, a tremendous achievement for the country, neither of the major Muslim languages, Shonam and SLM, are written any longer, and the number of people able to read the surviving texts is small.\footnote{Note
  that Shonam or \textit{Arwi} texts were not simply literary Tamil texts in Arabic-derived script. This means that those texts have the potential to tell us more about the characteristics of Shonam, whereas Malay texts were written in literary Malay that was certainly not simply the formal register for the forms of Malay spoken in Sri Lanka, but rather a variety that was quite far removed from ordinary usage. 
} 
Since we do not have corpora or even isolated attestations of early periods in the history of vernacular Sri Lankan Malay, we have to depend on plausible argumentation, in order to reconstruct linguistic events in the language's development, whether in morphosyntax or another grammatical component. Linguists are familiar with this problem from other unwritten vernaculars, certainly including contact varieties. It is only a minority of the world's languages whose internal history can be examined on the basis of extensive corpora, but this fact ought not to dissuade us from investigating diachrony.
I have worked, among other topics, on the development of tense and finiteness contrasts in Sri Lankan Malay and the extension of the finiteness contrast to the negation system. The presence of a finiteness contrast is at first glance a surprising development in the grammar of Sri Lankan Malay, since there is no such contrast in vehicular Malay varieties. Moreover, although the function of these contrasts varies cross-linguistically, they do not encode semantic contrasts that are ordinarily likely to be restored in contact languages that do not have them, or that have lost them. My argumentation follows from the principle that grammatical organization that would ordinarily change only very slowly, may change rapidly where a bilingual minority has experienced a radical shift in discourse culture, and considerable interaction with second language acquirers speaking a typologically discordant language. An important function of finiteness in the Sri Lankan sprachbund is to demonstrate the relative status of the most recent predicate in a temporally asymmetrical sequence (representing a sequence of events that do not occur simultaneously). This is the conjunctive participle construction, that conveys the sense of ``Having done A\textsuperscript{x}, having (subsequently) done B\textsuperscript{y}, I did C\textsuperscript{z}'', where x and y are participles, and z is a finite matrix verb. As we can see from the conjunctive participle construction and its subordinate (adjoined) relationship to finite matrix predicates, Sri Lankan Malay has developed cross-clausal assymetries that are not transparently functional, and that consequently require explanation. In the shift from vernacular Malay culture to Lankan culture, the change in information structure conventions happens to align with profound typological difference. We may ask rhetorically why, given continued access to L1 Malay, would speakers restructure their vernacular to such an extent that it began to take on the grammatical characteristics of a Dravidian language? Syntax under conditions of stable bilingualism changes very slowly in response to contact, although surface configurations that do not violate the constraints of the existing grammar may become increasingly frequent in response to pragmatic changes, leading to reanalysis. This is not qualitatively different from a process we find taking place to a greater or lesser extent in linguistic change generally, but the process is accelerated substantially. This is not the same as creolization in the older and more specific technical sense of pidginization with massive deflection, followed by replacement through grammaticalization. Some of the changes in Sri Lankan Malay, such as the post-verbal distribution of aspect markers in finite clauses and the postposing of prepositions can in no sense be characterized as processes of replacement or accretion of missing grammatical phenomena.  
It is somewhat unexpected that a radical contact language would develop new phonologically-dependent morphology (whether affixes or clitics) to mark case contrasts, given the fact that syntactic and prosodic processes can overtly mark the thematic relationships that case morphology typically marks. However the function of the morphology is clear, since it encodes relationships between arguments of predicates that are abstractly present in all natural languages, although these relationships are encoded differently in the vehicular Malay lexifier varieties.\footnote{Descriptive
  linguists and others who object to this view will not object to the view that the logical relationships between arguments of a predicate are perceptually present in a predictable feature of human cognition, whether or not the formal linguistic system operationalizes the cognitive universal.
} 
The sociolinguistic scenario in Sri Lanka, and the acquisition of Malay by speakers of a language with robust case morphology, provides the conditions for what took place.\footnote{It
  is important to note that a strikingly similar pattern of nominal complexification took place in the development of Sri Lankan Portuguese. It would be implausible to claim that second language acquisition did not contribute to grammatical change in that case, since not only does Portuguese, like vehicular Malay, not feature non-pronominal case morphology, but the speakers of Sri Lankan Portuguese, still a cohesive community in parts of Tamil-speaking eastern Sri Lanka, are of southern South Asian appearance.
} 
The sentence in (1) exemplifies functional and morphological change in Sri Lankan Malay, since none of the functional affixes and clitics we see in bold have immediate formal analogues in vehicular Malay, although they are all derived from free-standing closed class morphology in that language (and in Javanese, in the case of \textit{na}, based on \textit{nang}).

\ea
\gll Buk=\textbf{yang} \textbf{e}-baca,  Miflal=\textbf{na}  m{\textschwa}sigit=\textbf{ring} tumman attu=\textbf{na} ittu=\textbf{yang} \textbf{m}\textbf{{\textschwa}-kasi=na} \textbf{si-k{\textschwa}mauan}.\\
     book=\textsc{acc} \textsc{ptcp}-give Miflal=\textsc{dat} mosque=\textsc{abl}             friend  \textsc{det}=\textsc{dat} \textsc{det}=\textsc{acc} \textsc{inf}-give=\textsc{dat} \textsc{past}-want\\
 ``Having read the book, Miflal wanted to give it to a friend from the mosque.''
\z


The presence of a finiteness contrast, straightforwardly visible in the presence of an infinitival prefix, constitutes complexification that requires plausible diachronic modeling. With respect to finiteness, the term complexification refers not so much to the phonological dependence of functional material on a lexical host, a rather subjective criterion for defining a gradable and ambiguous term, but to the accretion of functional contrasts not present in the language previously. 

\section{What is discoverable in Sri Lankan Malay research?}%4

The most detailed investigation to date of the grammar of Sri Lankan Malay is \citet{Nordhoff2009}, which is quite extensive and broad in its scope. The author sensibly focused on one variety (the ``upcountry'' variety spoken in Kandy) rather than attempting to accommodate all of the dialect variation in one grammar. Much remains to be said about the other varieties, although they are mutually intelligible, and have much in common. Much remains to be said about the variety investigated by Nordhoff as well, as we would expect with any language whose grammar has only recently been extensively investigated. Although Nordhoff was not the first to write on the phonology of SLM (see also Tapovanaye 1986, 1995 and Bichsel-Stettler 1989), he has asked significant questions, for example on pre-nasalization and segmental boundary perception, basing his investigation on acoustic analysis. His conclusions have in turn been challenged in recent work by Ian Smith. The segmental phonology of SLM is an area in need of greater investigation, but there is also evidence for changes in the language's intonational phonology during the course of the twentieth century, although this topic has not yet been systematically investigated.  There is also a range of open questions and controversies concerning the present grammar (particularly in syntax and morphosyntax) and concerning the present lexicon of the language, which is strongly in flux. In Kirinda, for example, in spite of the cross-generational vitality of the language, generalized trilingualism, and frequent contact with monolingual Sinhala speakers, Shonam and Tamil vocabulary is replacing older Malay lexical items. The same source lexical borrowing source is now less likely in communities outside the southeast, where much of the younger generation is being educated in Sinhala, rather than in (any variety of) Tamil.
There has so far been no systematic investigation of sociolinguistic variation in the language, although variation is extensive and apparent even to casual observers \citep[cf][]{Rassooltv}. This variation should be investigated quantitatively based on social categories that are meaningful within the SLM communities, and the sociology of the language. Getting at those social categories requires detailed ethnographic research. Given the extent of anthropological interest in Sri Lanka, it is surprising that no anthropologist, to my knowledge, has taken an interest in its Malay communities. Returning to the matter of grammatical investigation, it would be an obvious mistake to assume that there is any less to be discovered in synchronic linguistic research on SLM than there is in any other language. There are two related factors, which ought to stimulate further linguistic research. Informally, linguist observers who acknowledged the profound divergence of SLM from vehicular Malay varieties in the Indonesian archipelago had assumed that SLM consisted of a bare lexical inventory superimposed upon Tamil (U. Tadmor, p.c.). Recent research on the grammar of SLM has implicitly and explicitly demonstrated that SLM is highly divergent as a Malay variety, and that it remains an autonomous linguistic system, rather than an ethnically emblematic lexical inventory, analogous with, for example, Anglo-Romani. Given that this autonomous system is by all accounts much more Lankan than it is Malay, in all but its lexical content, it follows that we have many more questions to investigate with respect to microparametric contrasts with the Lankan languages that have influenced its development. Why has replication brought this much convergence, but no more?
Published research on Sinhala has been particularly rich in investigations of information structure phenomena (Gair 1986, 1998, Kariyakarawana 1998 et al), partly because focus is a complex process in Sinhala grammar with morphological as well as syntactic reflexes. For those linguists who believe that there is as much reason to investigate diachronic influence from Sinhala as there is to investigate diachronic influence from Shonam (Ansaldo and Nordhoff), it is worth noting that SLM has not grammaticalized the focus of (non-clausal) constituents under the influence of Sinhala, although focus morphology on verbs is one of the most striking and well-investigated features of Sinhala grammar. To what extent have Sinhala focus phenomena been replicated without recourse to specialized morphology in the grammar of SLM? In my own case, I have rejected the external case for Sinhala influence prior to Sri Lankan independence and I have also shown how SLM aligns more closely with Shonam than with Sinhala. This conclusion applies to glottogenesis and subsequent development. However changing social conditions have led to adstratal influence from Sinhala from the mid-twentieth century onwards, and the effects of this influence are likely to increase over time. To what extent can we identify Sinhala influence in the SLM focus system? How does this compare with Shonam focus specifically? Note that far too little is known about differences between the grammar of Shonam and that of Tamil varieties spoken by Sri Lankan Hindus, Christians, and others. References to ``Tamil'' are not helpful, since general Sri Lankan Tamil is of little relevance in the study of SLM.
While there is still a (very) small number of SLM-speaking Malays living in the Eastern Province, which is predominantly Tamil-speaking, most Malays are in greater contact with speakers of Sinhala than ever before, and due to the educational system, the mass media, and the breakdown of ethnic boundaries, Sinhala is rapidly becoming the dominant vernacular in the majority of youth peer networks, including those of Moors, Malays, Burghers and others. The current picture is quite close to Ansaldo's scenario [of Sinhala dominance], even if the historical picture deviates from it. To what extent is this affecting the grammar of the language? To what extent is SLM in urban areas losing sociolinguistic domains, and what effect has this had on its lexical inventory? Is this accompanied by the loss of grammatical contrasts as we see for endangered languages spoken elsewhere, or is it the case that because the dominant language, in this case Sinhala, has a range of contrasts encoded with inflectional morphology, that we find \textit{new} contrasts? \citet{Ansaldo2005ms} claimed that Kirinda Malay had developed an accusative-dative split where there was previously none, but he had unfortunately been misled by glosses provided by B.A. Hussainmiya (a historian, not a linguist) in field recordings transcribed for Ian Smith.\footnote{While
  Ansaldo may have been misled in that case, he also repeatedly (a) misrepresents the research that I have published on the grammar of Sri Lankan Malay, and (b) makes misleading statements that mask its existence. In a list of researchers judged guilty of some measure of ``Tamil bias'', Ansaldo \citet[377-378]{Ansaldo2011a} lists me as a weak offender. (''Slomanson 2006 argues for convergence between SLM, Sinhala and Tamil, though Malay-Tamil bilingualism is invoked to explain some aspects of tense-marking.'') I do not recognize my work in that description. He later summarizes his list of biased researchers with a reference to Smith \citet{Smith2003timing}. ``The most serious attempt to date to argue for Tamil as a primary ``substrate'' \citep[i.e.][]{Smith2003timing} is a claim which, crucially, failed to find any evidence.'' Not only have I published research providing grammatical evidence for the primary role of Shonam, but Ansaldo was present at \citet{Slomanson2007} on the Dravidian (i.e. Shonam) character of the Sri Lankan Malay negation system.
} 
Although this was not a genuine example of case split, since all modern Sri Lankan Malay varieties investigated have a robust accusative-dative split, Ansaldo was right to assume that changes in the language's case system, including possible elaboration, are still possible. Given changes in the sociocultural configurations in which SLM speakers now find themselves and given the absence of a written norm, the whole matter of variation and change merits much more extensive investigation. The rate of change may be more rapid than the rate of change in demographically dominant languages whose development is mediated by a written tradition and its associated conventions.  One of the standard strategies for investigating variation is to employ apparent time methodology, and to do so before too much time passes, since the passing of time leads to the loss of potential informants, with predictable information loss as its result. There are many elderly members of the community who are still alive, but it is important to start working with them now. There are SLM speakers in their nineties in some communities. In others, such as Kirinda, there are no individuals who have reached the age of eighty (M.T.M. Rihan, p.c.). This also leads to a loss of culture and historical memory, since not everything is being passed on. If we look at a community such as Kirinda, we see the effects of cultural change quite starkly. This is because none of what we see there is attributable to language loss as such. The entire younger generation is completely fluent in SLM. Nevertheless, the lexical inventory is changing rapidly, with Shonam and Tamil words, and to a much lesser extent, Sinhala and English ones, replacing Malay ones. The most common informant response to a question about the acceptability of a particular etymologically Malay word is ``That is what old people say'' and the counterpart used by young people is generally Tamil or Shonam. This is not surprising, since contacts with the SLM world beyond Hambantota are few, education is Tamil-medium, half of the teachers are Moors, Shonam/Tamil is the medium of Friday religious instruction that is broadcast over loudspeakers, and Tamil-medium television blares in every home. Although Moors do not reside in the village, some village young people attend Hambantota schools, where they become integrated into Moorish peer networks, in which the ordinary language is Shonam. Discovering what is happening linguistically in this age cohort (late teens and early twenties) is a matter of some importance, because this is the cohort that will be raising the next generation. Another area of sociolinguistic inquiry that is indispensable for investigating variation and change is network analysis. My historical claim with respect to the Shonam versus Sinhala question is that Malay-Moorish networks were densest, however there is no way to demonstrate this with mathematical precision, because the network members are no longer alive, and because there has been too little documentary evidence discovered. For present-day youth networks, there is no reason for us not to investigate.

\section{Conclusion}%5

I have provided external arguments in support of the view that Shonam was the primary Lankan influence on Sri Lankan Malay in its development since the beginning of the Dutch period in the island's history. I have reviewed comments on external argumentation in the literature, and have challenged the comments that I found to detract from our understanding of the social processes that contributed to grammatical change. I have also discussed the importance of inferable diachronic information that we obtain through plausible hypothesis formation. Lastly, I have made suggestions for future research that will advance our understanding of the trajectory of linguistic and sociolinguistic change in Sri Lankan Malay.
