

\chapter{Clauses}\label{sec:clauses}
We have discussed how morphemes can be combined into words and how words can be combined into NPs, PPs and predicates. We will now address the next higher unit and discuss how several phrases can be combined to form a clause.

A clause must obligatorily have a predicate (PRED), and it can have one or more nominal phrases (NPs) which encode the arguments of the predicate. These NPs often carry postpositions (POSTP). Additional material can also be put into the clause  in the form of adjuncts. All clauses are normally predicate-final (with some exceptions for relative clauses), so that the normal structure of the clause is as follows (The Kleene star * marks that an element can be present 0, 1, or more times).



\cbx{
\NP* PRED
}{CLS}


\section{Main clauses}\label{sec:cls:Mainclauses}
The main clause normally consists of a predicate in final position, and a number of adjuncts and arguments to the left. The order of those arguments and adjuncts is free.\footnote{This freedom of argument order in the preverbal field is an areal feature \citep{ButtEtAl1994}.} As argued in \formref{sec:gramrel}, SLM grammar has no privileged argument, like `subject' or `object'. Rather, all the arguments have the same prominence. This can be represented in the following Tree:\footnote{The trees follow Givón's (2001) \nocite{Givon2001a,Givon2001b} argument for the usefulness of trees based in the style of \citet{Chomsky1957,Chomsky1965} to represent constituency, hierarchy, category labels, and linear order, in an elegant way. Generative grammar has used, and modified, trees and tree structure in a number of ways since then, but these more recent types of trees would not serve the illustrative purpose they are intended for here any better than their more traditional counterparts. This holds true regardless of whether one finds the theoretical arguments for the more modern trees compelling or not.}

\ea \label{ex:clause:tree:intro}
\Tree   [.S  NP NP NP ... PRED ]
\z

This tree actually resembles Hale's analysis of Japanese clause structure\footnote{This analogy between SLM and Japanese is also found in the domain of relative clauses \formref{sec:cls:Relativeclause}.} \citep{Hale1982japconf}, in which he denies the existence of a VP in Japanese and argues that Japanese is non-configurational. Mohanan's (1982) analysis of Malayalam, a sister language of Tamil, \nocite{Mohanan1982} comes to the same conclusion. \citet[177]{Lehmann1989} finally states for Tamil:\footnote{Non-configurational analysis for Sinhala have also been proposed by \citet{Gair1983} and \citet{Henadeerage2002}.}

\begin{quote}
We assume that the verb and all its argument NPs, as well as adverbial adjuncts, are immediate constituents of the sentence. This means that in Tamil there is no syntactic bond \el{} which affects the verb and the object NPs but not the subject NP. That is to say, there is no verb phrase (VP) constituent in Tamil. Subjects, objects, and the verb are all immediate constituents of S.
\end{quote}

Just like Japanese, Malayalam, or Tamil,  SLM seems to lack  rigid constituency within the clause, with the exception of the position of the verb, which must be the last or second-to-last element (see below) in the clause.

In SLM, it is very common for arguments to be dropped if they are inferable from context. It appears that Sri Lanka Malay is similar to Late Archaic Chinese in this respect, for which  \citet{Li1997zero} said: ``What demands explanation is not zero-anaphora \el{} but the appearance of a referential expression, whether a pronoun or a full fledged noun phrase.'' While this chapter will discuss many clauses with overt arguments, it should be borne in mind that the normal case for Sri Lanka Malay is to have most of the arguments dropped. This is discussed in more detail in Chapter \funcref{sec:Informationflow}.



As for clauses, we can distinguish a number of subtypes, discussed in the   sections given:

\begin{itemize}
 \item Declarative clause \formref{sec:cls:Declarativeclause}
\item Copular clause \formref{sec:cls:Copularclause}
\item Two types of interrogative clauses \formref{sec:cls:Interrogativeclauseclitic}, \formref{sec:cls:InterrogativeclauseWH}
\item Imperative clause \formref{sec:cls:Imperativeclause}.
\end{itemize} 

\subsection{Declarative clause}\label{sec:cls:Declarativeclause}
The declarative clause is the most simple type. It consists of between zero and three arguments followed by a predicative phrase. See the section on predicate types for examples \formref{sec:pred}.


\cbx{\NP* PRED}{CLS}

The order of the arguments is free, but the leftmost argument is normally the topic.\footnote{See \citet[62]{Bayer2004} and references therein for a more extensive list of preferred orders in languages with free word order.} An example of this is given in \xref{ex:clause:decl:topic}. The case marker of the topic is occasionally dropped as in   \xref{ex:clause:decl:ppdrop1}\xref{ex:clause:decl:ppdrop2}.
 
\xbox{16}{
\ea\label{ex:clause:decl:topic}
\gll [[Ini kaaving=nang aada] haarath-saarath pada]$_{top}$ kitham=pe bannyak oram pada arà-kijja. \\
     \textsc{prox} wedding=\textsc{dat} exist   traditions \textsc{pl} 1p=\textsc{poss} many people \textsc{pl} \textsc{non.past}-make\\
       `The traditions that were at this wedding, many of our people follow them.'  (K061122nar01)
\z      
}\\     
 

\xbox{16}{
\ea\label{ex:clause:decl:ppdrop1}
\gll [Luvar  nigiri]$_{top}$=\zero{}    kithang=nang   mà-pii    thàrà-suuka. \\
 outside country \textsc{1pl}=\textsc{dat} \textsc{inf}-go \textsc{neg}-like\\
`We do not want to go abroad.' (K051222nar04)
\z
}


\xbox{16}{
\ea\label{ex:clause:decl:ppdrop2}
\gll [Samma thumpath]$_{top}$=\zero{}  mlaayu aada. \\
     every place Malay exist  \\
    `In every place, there are Malays.'  (K051222nar04)
\z      
}\\ 
 

%\xbox{16}{
%\ea\label{ex:constr:unreferenced}
%\gll Se=dang se=ppe biini arà-iingath. \\
%      \textsc{1s=dat} \textsc{1s}=\textsc{poss} wife pro-think \\
%    `I am thinking of my wife.'  (nosource)
%\z      
%}\\ 



Sometimes, material is found to the right of the predicate (cf. \citet[137ff.]{Slomanson2006cll}, \citet[14]{Ansaldo2005ms}). This is frequently done for arguments encoding location and goal.

\cbx{NP(=POSTP)_{TOPIC} \NP* PRED PP$_{loc}$}{CLS}


%\xbox{16}{
%\ea\label{ex:constr:unreferenced}
%\gll Itthu    baathu=yang    incayang Seelong=dering             \textbf{laayeng} \textbf{nigiri=nang} asà-baapi. \\
% \textsc{dist} stone=\textsc{acc} \textsc{3s.polite} Ceylon-\textsc{abl} other country=\textsc{dat} \textsc{cp}-bring\\
%`He brought those stones from Ceylon to other countries.' (nosource)
%\z
%}

Example \xref{ex:clause:decl:right:local1} shows a local argument following the predicate \trs{anà-laaher}{was born}.
%, while \xref{ex:clause:decl:right:local2} shows two local arguments   following the predicate.

 


\xbox{14}{
\ea\label{ex:clause:decl:right:local1}
\gll see anà-laaher     Navalapitiya=ka$_{loc}$ \\
     1s \textsc{past}-be.born Nawalapitiya=\textsc{loc}  \\
    `I was born in Nawalapitiya.' (K051201nar01)
\z
} \\

% \xbox{16}{
% \ea\label{ex:clause:decl:right:local2}
% \gll Kithan       nya-pii \textbf{Anuradhapura=dang}, \textbf{Katunaayaka}     \textbf{sduuduk}. \\
%  \textsc{1pl} \textsc{past}-go Anuradhapura=\textsc{dat} Katunayaka from\\
% `We went to Anuradhapura, from Katunayaka.' (K051206nar16)
% \z there is a tendency to have topical arguments at the leftmost position. The frequent initial position of pronouns and human arguments is a consequence of this. It is very common for arguments to be dropped if they are inferable from context. It appears tha
% }

When right dislocation of non-spatial arguments occurs, these are often in focus.\footnote{This is a  common focalization device in Sinhala and Tamil, which entails additional morphological changes in these languages, not mirrored in SLM \citep{Gair1985calque}. Right dislocation without morphological changes is less common but does exist in Tamil \citep{Herring1994} and Sinhala \citep{Paolillo1997}.}

\cb{NP(=POSTP)_{TOPIC} \NP* PRED \NP$_{foc}$}

In the following three examples, the argument following the predicate is in contrastive focus to other referents mentioned before in discourse (other ethnic groups, other food, other languages).

\xbox{16}{
\ea\label{ex:clause:decl:right:foc1}
\gll Nni Peradeniya jaalang=ka samma n-aada \textbf{mlaayu}. \\
 \textsc{prox} Peradeniya road=\textsc{loc} all \textsc{past}-exist Malay\\
`Everybody in this Peradeniya Rd was Malay.' (K051222nar04)
\z
}

\xbox{16}{
\ea\label{ex:clause:decl:right:foc2}
\gll Hindu arà-maakang \textbf{kambing}. \\
 Hindu \textsc{non.past}-eat goat\\
`Hindus eat GOAT.' (K060112nar01)
\z
}

\xbox{16}{
\ea\label{ex:clause:decl:right:foc3}
\gll Itthukapang kitham arà-blaajar \textbf{mulbar}. \\
then \textsc{1pl} \textsc{non.past}-learn Tamil \\
`Then we learn TAMIL.' (K051213nar02)
\z
}


Also, very heavy NPs can be dislocated to the right:

\xbox{16}{
\ea\label{ex:clause:decl:right:heavy}
\gll Kitha=nang$_{NP}$ maau$_{PRED}$ [\textbf{kitham=pe} \textbf{mlaayu} \textbf{lorang} \textbf{blaajar} \textbf{lorang=pe} \textbf{mlaayu} \textbf{kitham} \textbf{blaajar}]$_{NP}$. \\
 \textsc{1pl}=\textsc{dat} want \textsc{1pl}=\textsc{poss} Malay \textsc{2pl} learn \textsc{2pl}=\textsc{poss} Malay \textsc{1pl} learn\\
`We want that you learn our Malay and that we learn your Malay.' (K060116nar02)
\z
}

%soocera padanang annajuuval



%\xbox{16}{
%\ea\label{ex:constr:unreferenced}
%\gll Kuttumu aada nni      CD-ya. \\
% see exist \textsc{prox} CD-yang\\
%`.' (nosource)
%\z
%}

But also arguments fulfilling none of the conditions mentioned above can be found after the verb, as shown in the following four examples.

\xbox{16}{
\ea\label{ex:clause:decl:right:ordinary1}
\gll Suda hathu  {\em week}=nang   duuva skali  arà-dhaathang    \textbf{{\em daughter}}. \\
     so one week=\textsc{dat} two time \textsc{non.past}-come daughter  \\
    `Thus my daughter comes twice a week.' (K051201nar01)
\z
} \\

 \xbox{16}{
\ea\label{ex:clause:decl:right:ordinary2}
   \gll Derang pada arà-mintha    \textbf{nigiri}. \\
    3     \textsc{pl} \textsc{non.past}-ask country\\
`They are asking for a country of their own.' (K051206nar12)
\z
}

\xbox{16}{
\ea\label{ex:clause:decl:right:ordinary3}
\gll Itthu=nang      blaakang su-dhaathang     \textbf{Hambanthota} \textbf{mlaayu} \textbf{pada}. \\
      \textsc{dist}=\textsc{dat} after \textsc{past}-come Hambantota Malay \textsc{pl} \\
    `After that came the Hambantota Malays.' (K051206nar14)
\z
} \\
 
\xbox{16}{
\ea\label{ex:clause:decl:right:ordinary4}
\gll Itthu muusing  Islam igaama  nya-aajar kaasi \textbf{Jaapna}  \textbf{Hindu} \textbf{{\em teacher}}. \\
      \textsc{dist} time Islam religion \textsc{past}-teach give Jaffna Hindu teacher \\
    `At that time, those who taught Islamic religion were Hindu teachers from Jaffna.' (K051213nar03)
\z
} \\

It could be possible to analyze the last three examples as sentences similar to an English pseudo-cleft. The first part would be a headless relative clause (`What they are asking for',  `Who came then',  `Who taught Islam'), and the second part the instantiation thereof (`a country', `Hambantota Malaya', `Jaffna teachers'.). This structure is common in Tamil and could be at the origin of a similar structure in Sinhala as well \citep{Gair1985calque}. More research on the intonation contours of these sentences is needed to ascertain whether right dislocation and pseudo-clefts can be distinguished by their intonation.

If we summarize our findings about argument positions we can say that an arbitrary number of arguments can occur before the verb, but only one can occur after the verb.


\cbx{
\NP* PRED $\left(\NP\right)$
}{CLS}

This resembles very much the structure of the NP as given in \formref{sec:nppp:Thefinalstructureofthenounphrase}. The amount and order of elements preceding the head is quite free, while following the head, there is only one position, which has some restrictions to it. As for the noun phrase, these are more absolute (no deictic, possessive or relative clause can ever be used after the noun), while for the clause, they are more lax, more like tendencies (non-spatial arguments tend not to occur after the verb). Similar to the representation of the NP \xref{ex:np-pp:tree:final} in \formref{sec:nppp:TheSLMNPasappositional}, we can represent the SLM declarative clause as in \xref{ex:clause:tree:final}.

\ea \label{ex:clause:tree:final}
\Tree   [.S
	 [.pre  NP NP {...}	 ]
	   [.pred ]
	   [.post NP ] 
	]
\z




\subsection{Copular clause}\label{sec:cls:Copularclause}
Some predicates can be supported by the copula \em (asà)dhaathang(apa) \em \formref{sec:wc:Copula}. This is most often done for predicates of naming \xref{ex:cl:copula:name} or of class-membership, either in an ethnic group \xref{ex:cl:copula:class:ethn}, a profession \xref{ex:cl:copula:class:profession} or kin \xref{ex:cl:copula:class:kin}.  English material in the sentence seems to favour the use of the copula.

\cb{NP COPULA name}
\cb{NP COPULA class}


\xbox{16}{
\ea\label{ex:cl:copula:name}
\gll Se=ppe    baapa  dhaathangapa \textbf{Jinaan} \textbf{Samath}. \\
     \textsc{1s}=\textsc{poss} father \textsc{copula} Jinaan Samath  \\
    `My father was Jinaan Samath.' (N060113nar03)
\z
} \\


\xbox{16}{
\ea\label{ex:cl:copula:class:ethn}
\ea
\gll Se=ppe    {\em daughter-in-{\em law}}=pe {\em mother} asàdhaathang \textbf{bingaali}. \\
      \textsc{1s}=\textsc{poss} daughter-in-law=\textsc{poss} mother \textsc{copula} Bengali \\
    `My daughter-in-law's mother is Bengali.'   
\ex
\gll Ithukapang       {\em daughter-in-{\em law}}=pe     {\em father} asàdhaathang \textbf{mlaayu}. \\
      then daughter-in-law=\textsc{poss} father \textsc{copula} Malay \\
    `Then my daughter-in-law's father is Malay.' (K051206nar08)
\z
\z
} \\


\xbox{16}{
\ea\label{ex:cl:copula:class:profession}
\gll Umma=pe       baapa  dhaathangapa  hathu  \textbf{{\em inspector}}          \textbf{\em of}  \textbf{{\em police}}. \\
     mother=\textsc{poss} father \textsc{copula} \textsc{indef} inspector of police  \\
    `My mother's father was an inspector of police.' (N060113nar03)
\z
} \\



\xbox{16}{
\ea\label{ex:cl:copula:class:kin}
\gll Baapa=pe      umma   asàdhaathang  \textbf{kaake=pe}           \textbf{aade}. \\
    father=\textsc{poss} mother \textsc{copula} grandfather=\textsc{poss} younger.sibling  \\
    `My paternal grandmother was my grandfather's younger sister.' (K051205nar05)
\z
} \\

Note that the use of the copula is optional in all these cases; it is more common to hear sentences without the copula. The following examples give sentences contrasting with the examples above in the absence of the copula.


\xbox{16}{
\ea\label{ex:cl:copula:name:contr}
\gll Sudaara sudaari se=ppe naama \textbf{Wahida} \textbf{Jamaldiin}. \\
     brother sister \textsc{1s}=\textsc{poss} name Wahida Jamaldeen  \\
    `Brothers and sisters, my name is Wahida Jamaldeen.' (B060115prs05)
\z
} \\


\xbox{16}{
\ea\label{ex:cl:copula:class:ethn:contr}
\gll Sindbad  {\em the}  {\em Sailor}     \textbf{hatthu} \textbf{Muslim}. \\
 Sindbad the Sailor \textsc{indef} Muslim \\
`Sindbad the Sailor was a Moor, he was not a Malay.' (K060103nar01)
\z
}


\xbox{16}{
\ea\label{ex:cl:copula:class:prof:contr}
\gll \zero{}  karang \textbf{{\em Dialog}} \textbf{\em GSM}=ka   \textbf{{\em junior}} \textbf{{\em executive}} \textbf{hatthu}. \\
       { } now Dialog GSM=\textsc{loc} junior executive \textsc{indef}\\
    `She is now junior executive at Dialog GSM [phone company].'
\z
} \\
 
\xbox{16}{
\ea\label{ex:cl:copula:class:kin:contr}
\gll Se=ppe    neene       itthu  \textbf{kaake=pe}           \textbf{aade}. \\
     \textsc{1s}=\textsc{poss} grandmother \textsc{dist} grandfather=\textsc{poss} younger.sibling  \\
    `My grandmother is that grandfather's younger sibling.' (K051205nar05)
\z
} \\

There are occasionally other semantic classes introduced by the copula (see \formref{sec:wc:Copula} for a discussion).

Copular clauses are often used for equational predicates. This is especially true for the naming use, where we assert that referent X is the same as referent Y. We do not assert of any predicate that it is true of my father; rather, we assert that two referents are identical.

Because equation is a symmetric predications (\em My father is John \em and \em John is my father \em have the same truth values), inversion of the positions of X and Y does not change the truth value. However, information structure changes, as in \xref{ex:cl:copula:swap1} and \xref{ex:cl:copula:swap2}. In both examples, the referent first mentioned is topical, while the other one is the comment on the topic.
 

\xbox{16}{
\ea\label{ex:cl:copula:swap1}
\gll [Se=ppe    kaake]       asàdhaathang [{\em estate} {\em tea} {\em factory} {\em officer}]. \\ % bf
      \textsc{1s}=\textsc{poss} grandfather \textsc{copula} estate tea factory officer \\
    `My grandfather was the estate tea factory officer.' (K060108nar02)
\z
} \\

\xbox{16}{
\ea\label{ex:cl:copula:swap2}
\gll [{\em Estate}=pe  {\em field} {\em officer}] asàdhaathangapa  [kithang=pe     kaake]. \\ % bf
     estate=\textsc{poss} field officer \textsc{copula} \textsc{1pl}=\textsc{poss} grandfather  \\
    `The estate field officer was our grandfather.' (N060113nar03)
\z
} \\

On very rare occasions, the order of the copula and the two NPs can be mixed up, as in \xref{ex:cl:copula:chaos}.

\xbox{16}{
\ea\label{ex:cl:copula:chaos}
\gll asàdhaathangapa incayang  {\em army} {\em captain}. \\ % bf
     \textsc{copula} \textsc{3s.polite} army captain  \\
    `He was an army captain.' (B060115cvs04)
\z
} \\


\subsection{Interrogative clause, clitic}\label{sec:cls:Interrogativeclauseclitic}
A third clause type is formed by adding the interrogative clitic \em =si \em to the portion of a declarative clause one wishes to question,  normally the predicate. This is shown for verbal predicates in \xref{ex:cl:interr:cl:v} and for a nominal predicate in \xref{ex:cl:interr:cl:n}. Questioning an argument is shown in \xref{ex:cl:interr:const}. Right dislocation does not seem to be possible in interrogative clauses.

\cbx{CLS=si}{CLS}
\cbx{\NP NP=si PRED}{CLS}


\xbox{16}{
\ea\label{ex:cl:interr:cl:v}
\gll Se=pe uumur masà-biilan=\textbf{si}? \\
 1=\textsc{poss} age must-tell=\textsc{interr}\\
`(Do I) have to tell my age?' (B060115prs01)
\z
}.

\xbox{16}{
\ea\label{ex:cl:interr:cl:n}
\gll Lorang=nang see=yang ingath-an=\textbf{si}? \\
     \textsc{2pl}=\textsc{dat} \textsc{1s}=\textsc{acc} think-\textsc{nmlzr}=\textsc{interr} \\
    `Do you have thoughts on me/are you thinking of me?'  (K070000wrt04)
\z      
}\\ 

\xbox{16}{
\ea\label{ex:cl:interr:const}
\gll \textbf{Daging baabi=si} anà-bìlli? \\
     pork=\textsc{interr} \textsc{past}-buy  \\
    `Did you buy PORK?'  (K081105eli02) 
\z      
}\\  
 

This is also the Sinhala and Tamil way of forming interrogative clauses.


\subsection{Interrogative clause, WH}\label{sec:cls:InterrogativeclauseWH}
A fourth clause type is the interrogative clause involving an interrogative pronoun.
The interrogative pronoun substitutes the queried element. It is normally found \em in situ\em, but since the word order to the left of the verb is quite free anyway, this  also means that \em in situ \em position cannot be distinguished from initial position. Technically speaking, this is just a special case of the declarative clause, where one or several NPs are instantiated by interrogative pronouns, instead of nouns, pronouns, etc.


\cbx{\NP* WH(=POSTP) \NP* PRED}{CLS}

The following three examples show the use of a WH-pronoun in initial position.

\xbox{16}{
\ea\label{ex:cl:interr:wh:initial1}
\gll \textbf{Mana} nigiri=ka arà-duuduk? \\
 which country=\textsc{loc} \textsc{non.past}-stay\\
`In which country do you live?' (B060115cvs16)
\z
}


\xbox{16}{
\ea\label{ex:cl:interr:wh:initial2}
\gll \textbf{Saapa}  m-maati? \\
 who \textsc{past}-die\\
`Who died?' (K051213nar07)
\z
}

\xbox{16}{
\ea\label{ex:cl:interr:wh:initial3}
\gll \textbf{Aapa}   arà-biilang    itthu? \\
      what \textsc{non.past}-say \textsc{dist} \\
    `So, what does that mean?'  (K051206nar04)
\z      
}\\ 

Non-initial position of the WH-pronoun is found in the following two examples.

\xbox{16}{
\ea\label{ex:cl:interr:wh:situ1}
\gll Lorang naama kapang-biilang, [baapa \textbf{saapa} umma \textbf{saapa}] katha. \\
      \textsc{2pl} name when-say father who mother who  \textsc{quot}\\
    `When you tell your name, (also tell) who (are) (your) parents.'  (B060115nar04)
\z      
}\\ 

% \xbox{16}{
% \ea\label{ex:cl:interr:wh:situ2}
% \gll Itthu=nang      blaakang \textbf{aapa} nya-gijja. \\
%  \textsc{dist}=\textsc{dat} after   what \textsc{past}-do\\
% `Then, what did we do?' (K051206nar07)
% \z
% }


\xbox{16}{
\ea\label{ex:cl:interr:wh:situ3}
\gll [Incayang=pe naama \textbf{aapa}], {\em sir}=pe naama? \\
      \textsc{3s.polite}=\textsc{poss} name what sir=\textsc{poss} name \\
    `What's his name, the teacher's name?' (K060103cvs01)
\z
} \\
 


The interrogative pronoun can be doubled. This indicates that an exhaustive list is expected as the answer. A single item will not do. Examples \xref{ex:cls:interr:aapaanabilli} and \xref{ex:cls:interr:apaaapa} illustrate this.

\xbox{16}{
\ea \label{ex:cls:interr:aapaanabilli}
\gll Aapa anà-bìlli? \\
      what \textsc{past}-buy\\
    `What  did you buy?'   (K081103eli02)
\z
}\\


\xbox{16}{
\ea \label{ex:cls:interr:apaaapa}
\gll Aapa\~{}aapa anà-bìlli. \\
      what\~{}\textsc{red} \textsc{past}-buy\\
    `What all did you buy.'  (K081103eli02)
\z
}\\



%  \xbox{16}{
% \ea \label{ex:interr:apaaapa:naturalistic}
% \gll \textbf{Aapa} \textbf{aapa} kitham Kandi=pe {\em cultural} {\em show} atthu=le thaaro? \\
%  what what \textsc{1pl} Kandy=\textsc{poss} cultural show one=\textsc{addit} put\\
% `What did we also put on a Kandy Cultural show?' (K060116nar11)
% \z
% }
% 
% 

\subsection{Imperative clause}\label{sec:cls:Imperativeclause}
The fifth main clause type is the imperative clause. It consists of between 0 and 2 expressed arguments, none of them agent, and a verb at the right edge. The verb optionally carries the particle \em mari \em to its left or the imperative suffixes \em -la \em or \em -de \em to its right, or both. The verb cannot carry any further TAM marking.

\cb{ \NP* (\textit{mari}) V $\left(\begin{array}{r}-la\\-de\end{array} \right)$}

Examples \xref{ex:cl:imp:nothing1} and \xref{ex:cl:imp:nothing2} show the use of a bare verb in the imperative clause,  \xref{ex:cl:imp:mari} shows the use of \em mari\em. Examples \xref{ex:cl:imp:la} and  \xref{ex:cl:imp:de} show the use of the suffixes while \xref{ex:cl:imp:marila} has both \em mari \em and \em -la\em.

\xbox{16}{
\ea\label{ex:cl:imp:nothing1}
\gll Aajuth thaakuth=ka su-naangis, ``See=yang \zero-luppas-\zero''. \\ % bf
     dwarf fear=\textsc{loc} \textsc{past}-cry ~~\textsc{1s}=\textsc{acc} leave  \\
    `The dwarf screamed in fear: ``Leave me!'' '  (K070000wrt04)
\z      
}\\

\xbox{16}{
\ea\label{ex:cl:imp:nothing2}
   \gll Binthan {\em auntie}=ka    \zero-caanya-\zero{}, binthan {\em auntie}=yang   konnyong \zero-panggel-\zero{}. \\ % bf
    Binthan auntie=\textsc{loc} ask, Binthan auntie=\textsc{acc} few call \\
`Ask auntie Binthan, call Binthan auntie' (K060116nar06)
\z
}
 
\xbox{16}{
\ea\label{ex:cl:imp:mari}
\gll \textbf{Mari} maakang. \\
      come eat \\
    `Eat!'  (B060115rcp02)
\z      
}\\ 

\xbox{16}{
\ea\label{ex:cl:imp:la}
\gll Allah \textbf{diyath-la} inni pompang pada dhaathang aada. \\
      Allah watch-\textsc{imp} \textsc{prox} female \textsc{pl} come exist \\
    `Almighty, see, these women have come!' (K061019nar02)
\z
} \\


\xbox{16}{
\ea\label{ex:cl:imp:de}
\gll Pii! ... pii!! ... pii-\textbf{de}!!! \\
     go { }   go   { }  go-\textsc{imp.impolite}  \\
    `Go! Go now! Bugger off!!!'  (not on recordings)
\z
} \\

\xbox{16}{
\ea\label{ex:cl:imp:marila}
\ea
\gll Saayang se=ppe thuan \textbf{mari} laari-\textbf{la}. \\
      love \textsc{1s}=\textsc{poss} sir come.\textsc{imp} run-\textsc{imp} \\
    `Come my beloved gentleman, come here.'
\ex
\gll See=samma kumpul \textbf{mari} thaa\u ndak-\textbf{la}. \\
     \textsc{1s}=\textsc{comit} gather come.\textsc{imp} dance-\textsc{imp}  \\
    `Come and dance with me.' (N061124sng01)
\z
\z
} \\



The imperative prefixes \trs{marà-}{\textsc{adhort}} and \trs{jamà-}{\textsc{neg.imp}} \citep[cf.][]{Slomanson2008lingua} are used in a different construction. In this construction, mentioning of the agent (normally \trs{kithang}{we}) is possible.\footnote{The modal particle \em thussa \em can be used for prohibitions as well, but this is not an imperative clause syntactically speaking. The speech act of prohibitions is discussed in \funcref{sec:pragm:Requestingaction}.}

\cb{(\textit{kithang}) \NP* $\begin{array}{r}mar\grave{a}-\\jam\grave{a}- \end{array}$ V}

\xbox{16}{
\ea\label{ex:cl:imp:mara}
\gll Kitham \textbf{marà}-maayeng. \\
     \textsc{1pl} \textsc{adhort}-play  \\
    `Let's play.' (K081104eli06)
\z
} \\


\xbox{16}{
\ea\label{ex:cl:imp:neg:jama}
\gll Hatthu=le \textbf{jamà}-gijja baapa ruuma=ka duuduk. \\ % bf
      \textsc{indef}=\textsc{addit} \textsc{neg.imp}-do father house=\textsc{loc} stay \\
    `Don't do anything, daddy, stay at home!'  (B060115nar04)
\z      
}\\



The adhortative combined with the interrogative particle \em =si \em forms an adhortative like English \em shall we?\em

\cb{ \NP* \textit{marà}-V=\textit{si}}

\xbox{16}{
\ea\label{ex:cl:imp:adhort}
\gll Marà-pii=si. \\
     \textsc{adhort}-go=\textsc{interr}  \\
    `Shall we go?'  (K081105eli02)
\z      
}\\ 


% \xbox{16}{
% \ea
% \gll *maripiisi/ *nyaanyi la si. \\
%        \\
%     `.' (nosource)
% \z
% } \\


\section{Relative clause}\label{sec:cls:Relativeclause}
The relative clause has the same word order as the declarative main clause and precedes its head. Both head-internal and head-external analysis are possible for the SLM relative clause. Given that material can intervene between the relative clause and the head noun \formref{sec:nppp:Relativeorderintheprenominalfield}, a head-external analysis is preferred here.

\cb[\label{cb:form:relc:intro}]{
... 
$
	\left[
		\NP* 
		PRED
	\right]_{RELC}
$ 
NP(=POSTP)$_{main}$ ... PRED$_{main}$
}

Relative clauses are only indicated by position.\footnote{\citet{SmithEtAl2004} note that relative clauses ``are headed by a verbal adjective'', yet none of the examples they give has a form glossed as `verbal adjective'; it appears that all verbs in the examples they give are simply finite verbs.} Any declarative main clause can be turned into a relative clause by putting it before a nominal. That nominal can have any semantic role.\footnote{This is thus very different from other Western Austronesian languages, where only subjects can be relativized \citep[161]{Himmelmann2005typochar}.} This simple principle covers all there is to say about relative clauses, still the different possibilities will be discussed in detail.

The Relative Clause Construction is used for relativization properly speaking (\em The photo that John left surprised me\em), but also for fact clauses (\em The fact that John left surprised me\em). These two constructions are semantically different in that in the former, we are dealing with a first order entity  (an individual x,  \em photo\em) modified by a proposition, whereas in the latter, we are dealing with a third order entity (a proposition X, \em fact \em \citep[8]{Hengeveld1992nvpttd}, whose content is given by the fact clause \citep[cf.][46]{Lehmann1984}. While these two types are semantically different, this semantic difference is not mirrored in SLM syntax: both use the construction given in \xref{cb:form:relc:intro}.\footnote{See \citet{Matsumoto1997} for a comparable analysis of Japanese relative clauses and fact clauses.}

We will first discuss the occurrence of different TAM-markers in relative clauses, to show that they are fully finite\footnote{Unlike Sinhala and Tamil, which have non-finite relative clauses.} \formref{sec:cls:TAMintherelativeclause}. We will then turn to different predicate types in relative clauses \formref{sec:cls:Predicatetypesintherelativeclause} and finally discuss the different semantic roles on which one can relativize \formref{sec:cls:Semanticrolesintherelativeclause}.
Relative clauses can also occur without a head they modify. These headless relative clauses have been treated in \formref{sec:nppp:Headlessrelativeclauses}.


\subsection{TAM in the relative clause}\label{sec:cls:TAMintherelativeclause}
There are no restrictions on the TAM markers which can appear in the relative clause. The following sections give examples of relative clauses in the different tenses.

\subsubsection{Past \em anà-\em}\label{sec:cls:pastana}

The past tense prefix \em anà- \em can be found in relative clauses.

\xbox{16}{
\ea\label{ex:cl:relc:ana1}
\gll [Incayang=pe kàpaala=ka \textbf{anà}-aada] thoppi=dering moonyeth pada=nang su-buvang puukul. \\
      \textsc{3s.polite}=\textsc{poss} head=\textsc{loc} \textsc{past}-exist hat=\textsc{abl} monkey \textsc{pl}=\textsc{dat} \textsc{past}-throw hit \\
    `He took the hat from his head and violently threw it  at the monkeys.'  (K070000wrt01)
\z      
}\\


% \xbox{16}{
% \ea\label{ex:cl:relc:ana3}
% \gll Bannyak thuuva oorang nya-blaajar oorang. \\
%      much old man \textsc{past}-learn man  \\
%     `An old man, an educated man.' (K060116nar07)
% \z
% } \\


\subsubsection{Past \em su-\em}\label{sec:cls:su}


The past tense prefix \em su- \em can be found in relative clauses.

\xbox{16}{
\ea\label{ex:cl:relc:su}
\gll [Ruuma duuva subala=ka   \textbf{su}-aada      rooja pohong  komplok duuva]=yang   asà-baa=apa   mliige=pe     duuva subla=ka su-thaanàm. \\
      house two side=\textsc{loc} \textsc{past}-exist rose tree bush two=\textsc{acc} \textsc{cp}-bring=after palace=\textsc{poss} two side=\textsc{loc} \textsc{past}-plant \\
    `The rose bushes that stood at both sides of the house were brought and planted on both sides of the palace.' (K070000wrt04)
\z
} \\

% 
% \xbox{16}{
% \ea\label{ex:cl:relc:unreferenced}
% \gll Sa-{\em mix} {\em salad}. \\
%       \textsc{past}-mix salad \\
%     `A mixed salad.' (B060115nar03)
% \z
% } \\



 

% \xbox{16}{
% \ea\label{ex:cl:relc:unreferenced}
% \gll [Aanak raaja=pe perkathahan=yang Snow-white=nang=le Rose-red=nang=le sukahan=dering \textbf{su}-punnu hathu hidopan] su-thunjiking. \\
%       child king=\textsc{poss} word=\textsc{acc} Snow-white=\textsc{dat}=\textsc{addit} Rose-red=\textsc{dat}=\textsc{addit} desire=\textsc{abl} \textsc{past}-full \textsc{indef} dwelling \textsc{past}-show \\
%     `The prince showed them place (to live) which fulfilled the prince's promise to Snow-White and Rose-Red's pleasure.' (K070000wrt04)(test)
% \z
% } \\
 




\subsubsection{Perfect with \em aada\em}\label{sec:cls:perfectwithaada}
The perfect tense can be found in the relative clause.

\xbox{16}{
\ea\label{ex:cl:relc:perf:aada1}
\gll [Seelon=nang dhaathang \textbf{aada}] mlaayu oorang ikkang. \\
 Ceylon-\textsc{dat} come exist Malay man fish\\
`The Malays who came to Sri Lanka were fishermen.' (K060108nar02)
\z
}



 \xbox{16}{
\ea\label{ex:cl:relc:perf:aada2}
   \gll Itthu    asàdhaathang [baaye=nang vaasil-king \textbf{aada}]  {\em dagger} hatthu. \\
     \textsc{dist} \textsc{copula} good=\textsc{dat} blessed-\textsc{caus} exist dagger \textsc{indef} \\
`That was a well blessed dagger' (K051206nar02,K081105eli02)
\z
}

\subsubsection{Non-past \em arà\em}
\em Arà- \em in its non-past reading can be found in the relative clause.

\xbox{14}{
\ea
\gll thoppi arà-daagang    oorang \\
     hat \textsc{non.past}-trade man  \\
    `The hat seller' (K070000wrt01)
\z
} \\ 

 

\subsubsection{Simultaneous \em arà-\em}\label{sec:cls:ara}
The simultaneous reading of \em arà \em is also found.

\xbox{16}{
\ea\label{ex:cl:relc:ara:simult1}
\gll Suda [puthri=le biini=le \textbf{arà}-caa\u nda haari]=le su-dhaathang. \\
      so queen=\textsc{addit} wife=\textsc{addit} \textsc{simult}-meet day=\textsc{addit} \textsc{past}-come \\
    `So then the day came when the wife and the queen were to meet.'  (K070000wrt05)
\z      
}\\


This event is located in the past, yet \em arà- \em is used, which can then not have the non-past meaning, but rather the meaning of `simultaneous to the time of the matrix sentence' i.e. the coming of the day and the meeting are treated as referring to the same time frame.

The same thing can be said about the next example, where the hearing and the knocking are conceived of as simultaneous. It is not possible for the knocking to refer to non-past, since it is impossible to have heard something in the past which had not yet occurred. Therefore, the non-past reading of \em arà- \em is not an option here.


\xbox{16}{
\ea\label{ex:cl:relc:ara:simult2}
\gll [Kìrras pinthu=nang \textbf{arà}-thatti hathu svaara] su-dìnngar. \\
     strong door=\textsc{dat} \textsc{simult}-hammer \textsc{indef} noise] \textsc{past}-hear \\
    `They heard a noise of hard hammering at the door.'  (K070000wrt04)
\z      
}\\ 

\subsubsection{Irrealis \em anthi-\em}\label{sec:cls:anthi}
The irrealis is not found often in relative clauses. The following is the only example, where this prefix is used in its epistemic reading.

\xbox{16}{
\ea
\gll  [Kithang=nang duppang dhaathang \textbf{athi}-aada {\em exiles} pada]=jo anà-baa.\\
      \textsc{1pl} before come \textsc{irr}-exist exiles \textsc{pl}=\textsc{emph}  \textsc{past}-bring \\
    `The exiles, who would have come before us, were brought.' (K060108nar02)

\z
}

\subsubsection{Conjunctive participle \em asà-\em}\label{sec:cls:asa}
The conjunctive participle prefix can be used on its own in  relative clauses when it expresses perfect tense, as in \xref{ex:cl:relc:asa:main}. In this sentence, the existential \em aada\em, which expresses perfect tense together with \em asà-, \em is dropped. \em Asà- \em can also be used on non-final verbs if there are more verbs in the relative clause \xref{ex:cl:relc:asa:chain}.



\xbox{16}{
\ea\label{ex:cl:relc:asa:main}
\gll [Tony Hassan {\em uncle}=nang \textbf{asà}-kaasi (aada) duvith] athi-mintha ambel=si? \\
     Tony Hassan uncle=\textsc{dat} \textsc{cp}-give (exist) money \textsc{irr}-ask take=\textsc{interr}  \\
    `Shall I ask for the money you gave to uncle Tony Hassan?' (K071011eml01)
\z
} \\


\xbox{16}{
\ea\label{ex:cl:relc:asa:chain}
\gll [Banthu-an \textbf{asà}-mintha arà-naangis] svaara hatthu derang=nang su-dìnngar. \\
      help-\textsc{nmlzr} \textsc{cp}-beg \textsc{simult}-cry sound \textsc{indef} \textsc{3pl}=\textsc{dat} \textsc{past}-hear\\
    `They heard a sound of crying and begging for help.'  (K070000wrt04)
\z      
}\\ 
 


\subsubsection{Infinitive/purposive \em mà-\em}\label{sec:cls:ma}

The infinitive prefix \em mà- \em can be used for relative clauses of purpose \xref{ex:cl:relc:ma:nonang1}\xref{ex:cl:relc:ma:nonang2}. The dative marker \em =nang \em can optionally be present in the relative clause \xref{ex:cl:relc:ma:nang}.

\xbox{16}{
\ea\label{ex:cl:relc:ma:nonang1}
\gll Lorang se=dang [\textbf{mà}-hiidop] thumpath kala-kaasi, ...  \\
     \textsc{2pl} \textsc{1s=dat} \textsc{inf}-stay place if-give, ...  \\
    `If you give me a place to stay, ...'  (K070000wrt04)
\z      
}\\

\xbox{16}{
\ea\label{ex:cl:relc:ma:nonang2}
\gll Kithang lorang=nang baaye mliiga athi-kaasi,   [\textbf{mà}-kaaving] panthas pompang pada athi-kaasi,    duvith athi-kaasi. \\
      \textsc{1pl} \textsc{2pl}=\textsc{dat} good palace \textsc{irr}-give  \textsc{inf}-marry beautiful female \textsc{pl} \textsc{irr}-give  money \textsc{irr}-give \\
    `We will give you nice palaces, we will give you beautiful girls to marry, we will give you money.' (K051213nar06)
\z
} \\

  
\xbox{16}{
\ea\label{ex:cl:relc:ma:nang}
\gll Itthu=nang aada  [{\em divorce} \textbf{mà}-kijja=nang] hatthu prentha oorang. \\
      \textsc{dist}=\textsc{dat} exist divorce \textsc{inf}-make=\textsc{dat} \textsc{indef} law man \\
    `For that there is a lawyer to make the divorce.'  (K061122nar01)
\z      
}\\ 

\subsubsection{Negative past \em thàrà-\em}\label{sec:cls:negative}
The negated past tense prefix \em thàra- \em can also be used in relative clauses.

\xbox{16}{
\ea\label{ex:cl:relc:thara}
\gll Derang pada panggel=nang blaakang [\textbf{thàrà}-dhaathang oorang pada]=nang nya-force-kang kiyang \\
  \textsc{3pl} \textsc{pl} call=\textsc{dat} after \textsc{neg.past}-come man \textsc{pl}=\textsc{dat} \textsc{past}-force-\textsc{caus} \textsc{evid}    \\
    `After they had called (them), (they) apparently forced the people who had not come (to join).'  (K051206nar07)
\z      
}\\ 

\subsection{Predicate types in the relative clause}\label{sec:cls:Predicatetypesintherelativeclause}
All predicate types can be found in relative clauses.

\subsubsection{Verbal predicate}\label{sec:cls:verbal}
A simple example of a verbal relative clause is \xref{ex:cl:relc:verbal}.

\xbox{16}{
\ea\label{ex:cl:relc:verbal}
\gll Itthu [se arà-\textbf{kirijja}] mosthor=jo. \\
 \textsc{dist} \textsc{1s} \textsc{non.past}-make manner=\textsc{emph}\\
`That's the way I do it.' (B060115rcp01)
\z
}   

\subsubsection{Modal predicate}\label{sec:cls:modal}
Modal predicates can be used in relative clauses. This is especially frequent for \trs{boole}{can}, as given in \xref{ex:cl:relc:modal1}.

 
\xbox{16}{
\ea\label{ex:cl:relc:modal1}
\gll [Deram pada \el{} baae=nang pìrrang mà-kijja \textbf{boole}] oorang. \\
 \textsc{3pl} \textsc{pl} { } good=\textsc{dat} war \textsc{inf}-make can man\\
`They were men	 who were able to fight well.' (K051213nar06)
\z
}

Normally, the head noun takes the case the matrix clause requires, as in \xref{ex:cl:relc:modal1} (where it is zero), but it also occurs that the head noun takes the case required by the relative clause instead of the one required by the matrix clause. In \xref{ex:cl:relc:modal2}, the matrix clause would not assign any case to its only argument.  \em Boole \em in the relative clause normally assigns dative, which is then also marked on the head noun by the postposition \em =nang\em.

\xbox{16}{
\ea\label{ex:cl:relc:modal2}
\gll [Itthu    baaye mosthor=nang \textbf{bole}=kirja    oorang mlaayu]=nang sajja=jo. \\
      \textsc{dist} good manner=\textsc{dat} can-make man Malay=\textsc{dat} only=\textsc{emph} \\
    `The Malays are the only ones who can do it in the right way.' (K061026rcp01)
\z
} \\
 
The negation \em thàrboole \em can also be used in a relative clause.

\xbox{16}{
\ea\label{ex:cl:relc:modal:therboole}
\ea
\gll [Boole oorang   pada] samma dhaathang. \\
     can man \textsc{pl} all come   \\
    `The people who could came.' 
\ex
\gll [\textbf{Thàràboole}   oorang pada] su-biilang:     kithang=nang   mà-dhaathang    thàràboole. \\
      cannot man \textsc{pl} \textsc{past}-say \textsc{1pl}=\textsc{dat} \textsc{inf}-come cannot \\
    `The people who could not come said: ``We cannot come''.' (K051206nar07)
\z
\z
} \\

% 
%  \xbox{16}{
%  \ea\label{ex:cl:relc:unreferenced}
%    \gll {\em Malaysia}=ka    anà-duuduk        mosthor, [anà-boole    mosthor], kithang itthu=yang   itthu    mosthor=nang   arà-baapi. \\
%     Malaysia=\textsc{loc} \textsc{past}-exist.\textsc{anim} manner \textsc{past}-can manner \textsc{1pl} \textsc{dist}=\textsc{acc} \textsc{dist} manner=\textsc{dat} \textsc{non.past}-bring \\
% `We have developed our own customs from the ones of the people in Malaysia' (K060108nar02)
% \z
% }

Besides \em boole\em, \em maau \em has also been found occurring in a relative clause.



\xbox{16}{
\ea\label{ex:cl:relc:modal:mau}
\gll [Derang=nang \textbf{maau} mosthor] baalas katha nya-biilang. \\
      \textsc{3pl}=\textsc{dat} want manner answer \textsc{quot} \textsc{past}-say \\
    I gave them the answer in the way they wanted ' (K051213nar01)
\z
} \\




\subsubsection{Nominal and adjectival predicates}\label{sec:cls:nominalandadjectival}

Relative clauses based on nominal predicate clauses and adjectival predicate clauses are formally indistinguishable from nouns modified by a bare noun \xref{ex:cl:relc:nominal} or a bare adjective \xref{ex:cl:relc:adjectival} instead of a a clause.

\xbox{16}{
\ea\label{ex:cl:relc:nominal}
\gll [Moonyeth]$_{N/RELC}$  hathu kavanan su-aada. \\ % bf
     monkey \textsc{indef} group \textsc{past}-exist\\
    `There was a   monkey group.'\\
    `There was a group which consisted of monkeys.'  (K070000wrt01)
\z      
}\\ 


\xbox{16}{
\ea\label{ex:cl:relc:adjectival}
\gll Ini [laama]$_{ADJ/RELC}$ {\em car} pada=jo kithang arà-baapi. \\ % bf
      \textsc{prox} old car \textsc{pl}=\textsc{emph} \textsc{1pl} \textsc{non.past}-bring \\
    `It is these old cars we take [to Iraq].' \\
    `It is these cars which are old that we take [to Iraq].' (K051206nar19)
\z      
}\\ 

% The relative clause can be distinguished from a nominal premodification if there are arguments to it, as in \xref{ex:relc:pred:nom:kamauvan}, where \trs{se=dang}{1s=dat} is an argument of \trs{kamauvan}{desire/want}.
% 
% \xbox{16}{
% \ea\label{ex:relc:pred:nom:kamauvan}
% \gll Lorang [se=dang kamauvan pada]=yang gijja kaasi. \\ % bf
%      \textsc{2pl} \textsc{1s=dat} desire \textsc{pl}=\textsc{acc} make give  \\
%     `Please fulfill my wishes.'  (K051220nar01)
% \z      
% }\\
%  

\subsubsection{Circumstantial predicates}\label{sec:cls:circumstantial}
% \xbox{16}{
% \ea\label{ex:cl:relc:circ}
% \gll [[\textbf{Sithu=ka}     \textbf{aada}]  bìssar oorang pada]=yang   asà-attack-kang     mail=nya    asà-cuuri [\textbf{{\em mail}=ka}]    duvith arà-baapi. \\
%      there=\textsc{loc} exist big man \textsc{pl}=\textsc{acc} \textsc{cp}-attack-\textsc{caus} mail=\textsc{acc} \textsc{cp}-steal mail=\textsc{loc} money \textsc{non.past}-bring  \\
%     `He attacked the leaders who were there and stole the mail and took away the money in the mail' (K051206nar02)
% \z
% } \\
%  
Circumstantial predicates are normally found  with overt marking of the existential, as in the following two examples.

\xbox{16}{
\ea\label{ex:cl:relc:circ:aada1}
\gll [Incayang=pe kàpaala=ka \textbf{anà-aada}] thoppi=dering moonyeth pada=nang su-buvang puukul. \\
      \textsc{3s.polite}=\textsc{poss} head=\textsc{loc} \textsc{past}-exist hat=\textsc{abl} monkey \textsc{pl}=\textsc{dat} \textsc{past}-throw hit \\
    `He took the hat which was on his head and violently threw it  at the monkeys.'  (K070000wrt01)
\z      
}\\ 


\xbox{16}{
\ea\label{ex:cl:relc:circ:aada2}
\gll Derang [dìkkath=ka \textbf{aada}] laapang=nang mà-maayeng=nang su-pii. \\
     \textsc{3pl} vicinity=\textsc{loc} exist ground=\textsc{dat} \textsc{inf}-play=\textsc{dat} \textsc{past}-go  \\
    `They went to play on the ground which was nearby.'  (K070000wrt04)
\z      
}\\ 


%K051206nar04.txt:  pìrrang=nang     baae  aada oorang pada=jo


\subsubsection{Relative clauses based on utterances}\label{sec:cls:Utterance}
Besides clauses, it is also possible to use utterances to premodify a noun. The following examples show this for a short reported string \xref{ex:cl:relc:utterance:name1}\xref{ex:cl:relc:utterance:name2} and a long reported string giving the content of \trs{habbar}{news} \xref{ex:cl:relc:utterance:clause}.

\xbox{16}{
 \ea\label{ex:cl:relc:utterance:name1}
   \gll Baava=ka  [Kaasim katha] hatthu {\em family}. \\
     bottom=\textsc{loc} Kaasim \textsc{quot} \textsc{indef} family \\
`Below there is a family called ``Kaasim''.' (K051206nar06)
\z
}

\xbox{16}{
 \ea\label{ex:cl:relc:utterance:name2}
\gll Itthukang     anà-aada      [Mr  Janson  katha] hathu   oorang. \\
     then \textsc{past}-exist Mr Janson \textsc{quot} one man. \\
    `Then there was a certain Mr Janson.' (K051206nar04)
\z
} \\

\xbox{16}{
\ea\label{ex:cl:relc:utterance:clause}
\gll Se=dang habbar, [[ini      laama {\em car} pada samma inni     {\em suicide} {\em bombers} asà-dhaathang {\em car}=yang, aapa,    arà-{\em paavicci}     katha] habbar]. \\
      \textsc{1s=dat} news \textsc{prox} old cat \textsc{pl} all \textsc{prox} suicide bombers \textsc{cp}-come car=\textsc{acc} what \textsc{non.past}-use(Sinh.) \textsc{quot} news \\
    `I have information, information ``these suicide bombers come and take all these old cars and, what, and use them.'' ' (K051206nar19)
\z
} \\

\subsection{Semantic roles in the relative clause}\label{sec:cls:Semanticrolesintherelativeclause}
There is no restriction on the semantic roles that can be relativized on. This will be shown for the different semantic roles in the individual sections below.

\subsubsection{Agent}\label{sec:cls:Agent}
The verb \trs{dhaathang}{come} subcategorizes for Agent.

\xbox{16}{
\ea\label{ex:cl:relc:agent}
\gll [Seelon=nang dhaathang aada] mlaayu oorang ikkang. \\ % bf
 Ceylon-\textsc{dat} come exist Malay man fish\\
`The Malays who came to Sri Lanka were fishermen.' (K060108nar02)
\z
}


%\xbox{16}{
%\ea\label{ex:cl:relc:unreferenced}
%\gll [Banthu-an asà-mintha arà-naangis] svaara hatthu derang=nang su-dìnngar. \\
%      help-\textsc{nmlzr} \textsc{cp}-beg \textsc{non.past}-cry sound \textsc{indef} \textsc{3pl}=\textsc{dat} \textsc{past}-hear\\
%    `They heard a sound of crying and begging for help.'  (K070000wrt04)
%\z      
%}\\ 


\subsubsection{Patient}\label{sec:cls:Patient}
Patients of both intransitive clauses (\trs{mnii\u n\u ggal}{die} in \xref{ex:cl:relc:patient:intr}) and transitive clauses (\trs{baa}{bring} in \xref{ex:cl:relc:patient:tr1}\xref{ex:cl:relc:patient:tr2}) can be relativized on.

\xbox{16}{
\ea\label{ex:cl:relc:patient:intr}
\gll [{\em North}-pe    pìrrang-ka mnii\u n\u ggal mlaayu pada]=nang bìssar hatthu {\em religious} {\em function}. \\ % bf
north=\textsc{poss} war=\textsc{poss} die Malay \textsc{pl}=\textsc{dat} big \textsc{indef} religious function\\
`A religious function for the Malays who had died in the Northern war.' (K060116nar11)
\z
}
 


\xbox{16}{
\ea\label{ex:cl:relc:patient:tr1}
\ea 
\gll Ka-duuva     anà-dhaathang    {\em slaves}  pada, \\ % bf
     \textsc{ord}-two \textsc{past}-come slaves \textsc{pl}  \\
    `The second to come were slaves,'  
\ex
\gll [{\em soldier} pada        na-baa     oorang pada]. \\ % bf
     soldier \textsc{pl} \textsc{past}-bring man \textsc{pl}  \\
    `men brought by soldiers.'
\z
\z
}


\xbox{16}{
\ea\label{ex:cl:relc:patient:tr2}
\gll [[Itthu  mà-jaaga=nang] anà-baa mlaayu]=dring   satthu oorang=jo    se. \\ % bf
 \textsc{dist} \textsc{inf}-protect=\textsc{dat} \textsc{past}-bring Malay]=\textsc{abl} one man=\textsc{emph} 1s\\
`One of the Malay men brought to protect that (man) is me.' (K060108nar02)
\z 
}\\   
 

% \xbox{16}{
% \ea\label{ex:cl:relc:patient:tr2}
% \glll [[hathu_{NUM_i} duuri pohong]_{NP}=nang [puuthi   paanjang jee\u n\u ggoth]_{NP}=yang anà-kànà-daapath kìnna]_{RElC} hathu_{NUM_j} kiccil_{ADJ} jillek_{ADJ} Aajuth_{HEAD} hatthu_{NUM_j} yang su-kuthumung. \\
%       [{} {} {} {} {} {} {} {} {}]_{RELC} NUM ADJ ADJ N NUM {} {}\\
%       \textsc{indef} thorn tree=\textsc{dat} white long beard=\textsc{acc} r\textsc{past}-\textsc{invol}-get befall \textsc{indef} small ugly dwarf indef]=\textsc{acc} past see\\
%     `They saw a small ugly dwarf who had his beard got stuck in a thorn tree.'  (K070000wrt04)
% \z      
% }\\ 
% 


\subsubsection{Theme}\label{sec:cls:Theme}
For the semantic role of theme, relativization is possible for all the different shades of meaning, be they an item located at some place as in \xref{ex:cl:relc:theme:location}, a stimulus as in \xref{ex:cl:relc:theme:stimulus}, or  a concept known \xref{ex:cl:relc:theme:concept}.

\xbox{16}{
\ea\label{ex:cl:relc:theme:location}
   \gll Incayang  [ini      Seelong=ka  anà-aada    lakuan   baathu] asà-caari. \\ % bf
    \textsc{3s.polite} \textsc{prox} Seelon=\textsc{loc} \textsc{past}-exist wealth stone \textsc{cp}-find \\
`He was looking for the gems present in Ceylon.' (K060103nar01,K081105eli02)
\z
}

%  \xbox{16}{
%  \ea\label{ex:cl:relc:theme:theme2}
%    \gll [Spaaman anà-nii\u n\u ggal thumpath]=nang=le        [Passara   katha arà-biilang    nigiri]=nang=le dìkkath. \\ % bf
%          \textsc{3s} \textsc{past}-die place=\textsc{dat}=\textsc{addit} Passara \textsc{quot} \textsc{non.past}-say country=\textsc{dat}=\textsc{addit} vicinity\\
%  	`The place where he died is close to the village called Passara.' (B060115nar05,K081105eli02)
% \z
% }


\xbox{16}{
\ea\label{ex:cl:relc:theme:stimulus}
\gll [Kìrras pinthu=nang arà-thatti hathu svaara] su-dìnngar. \\ % bf
     strong door=\textsc{dat} \textsc{simult}-hammer \textsc{indef} noise] \textsc{past}-hear \\
    `They heard a noise of hard hammering at the door.'  (K070000wrt04)
\z      
}\\ 




\xbox{16}{
\ea\label{ex:cl:relc:theme:concept}
\gll [Se=dang   thaau mosthor]=nang karang=nang   ka-dhlaapan     {\em generation} arà-pii. \\ % bf
     \textsc{1s=dat} know way=\textsc{dat} now=\textsc{dat} \textsc{ord}-eight generation \textsc{non.past}-go  \\
    `As far as I know, we are now in the eighth generation.'  (K060108nar02)
\z      
}\\ 

% \xbox{16}{
% \ea\label{ex:cl:relc:theme:exist1}
% \gll Derang [ini kaaving=nang aada] haarath.saarath pada, kitham=pe bannyak ooram pada arà-kijja. \\ % bf
%      \textsc{3pl} \textsc{prox} wedding=\textsc{dat} exist   traditions \textsc{pl} 1p=\textsc{poss} many people \textsc{pl} \textsc{non.past}-make\\
%        `The traditions that were at this wedding, many of our people follow them.'   (K061122nar01)
% \z      
% }\\
  
% \xbox{16}{
% \ea\label{ex:cl:relc:theme:exist2}
% \gll Suda karang [kithang=nang   aada  {\em problem}] dhaathangapa kithang=pe     aanak  pada mlaayu thama-oomong. \\ % bf
%       thus now \textsc{1pl}=\textsc{dat} exist problem \textsc{copula} \textsc{1pl}=\textsc{poss} child \textsc{pl} Malay \textsc{neg.nonpast}-speak \\
%     `So the problem we are having now is that our children do not speak Malay.' (G051222nar01)
% \z
% } \\
 
 
% 
% \xbox{16}{
% \ea\label{ex:cl:relc:unreferenced}
% \gll [Andare kanabisan=nang anà-mintha] hathu raaja=ke asà-paake=apa kampong=nang mà-pii maau katha. \\
%     Andare last=\textsc{dat} \textsc{past}-ask \textsc{indef} king=\textsc{simil} \textsc{cp}-dress=after village=\textsc{dat} \textsc{inf}-go want \textsc{quot}   \\
%     `What Andare wanted as a last wish, was to go to the village dressed up as a king.'  (K070000wrt03)
% \z      
% }\\ 


% 
% \xbox{16}{
% \ea\label{ex:cl:relc:unreferenced}
% \gll [Sithu=ka     aada]$_{RELC}$  [bìssar]$_{ADJ}$ oorang$_{N}$ pada=yang   asà-attack-kang     mail=nya    asà-cuuri {\em mail}=ka    duvith arà-baapi. \\
%      there=\textsc{loc} exist big man \textsc{pl}=\textsc{acc} \textsc{cp}-attack-\textsc{caus} mail=\textsc{acc} \textsc{cp}-steal mail=\textsc{loc} money \textsc{non.past}-bring  \\
%     `He attacked the leaders who were there and stole the mail and took away the money in the mail' (K051206nar02)
% \z
% } \\


\subsubsection{Experiencer}\label{sec:cls:Experiencer}
The person experiencing the ability is relativized on in \xref{ex:cl:relc:exp}.

 
\xbox{16}{
\ea\label{ex:cl:relc:exp}
\gll Deram pada ... [baae=nang pìrrang mà-kijja boole oorang] \\ % bf
 \textsc{3pl} { } \textsc{pl} good=\textsc{dat} war \textsc{inf}-make can man\\
`They were men who were able to fight well.' (K051213nar06)
\z
}
    
\subsubsection{Recipient}\label{sec:cls:Recipient}
The recipient of the verb \trs{kaasi}{give} is relativized on in \xref{ex:cl:relc:rec}.

\xbox{16}{
\ea\label{ex:cl:relc:rec}
\gll [Se duvith anà-kaasi oorang] su-iilang. \\
     \textsc{1s=dat} money  \textsc{past}-give man \textsc{past}-disappear  \\
    `The man I gave money to disappeared.'  (K081105eli02)
\z
}\\ 

\subsubsection{Possessor}\label{sec:cls:Possessor}
The husband formerly `possessed' by the woman is relativized on in \xref{ex:cl:relc:poss}.

\xbox{16}{
\ea\label{ex:cl:relc:poss}
\gll [Laaki anà-mnii\u n\u ggal hathu pompang]. \\ % bf
     husband \textsc{past}-die \textsc{indef} woman  \\
    `A woman whose husband had died.'  (K070000wrt04)
\z      
}\\ 


\subsubsection{Location}\label{sec:cls:Location}
Locations can be relativized on. In \xref{ex:cl:relc:loc1}, the location of staying is at the same time the location of falling. The locative argument \trs{thumpath}{place} is found as the semantic role relativized on. The same is true of \em thumpath \em in \xref{ex:cl:relc:loc2}.

\xbox{16}{
\ea\label{ex:cl:relc:loc1}
   \gll Siithu [nya-duuduk    thumpath]=ka baapa  su-jaatho. \\ % bf
     there \textsc{past}-stay place=\textsc{loc} father \textsc{past}-fall  \\
    `There, at the place he was staying, my father fell.' (K051205nar05)
\z
} 

\xbox{16}{
 \ea\label{ex:cl:relc:loc2}
   \gll [Spaaman anà-nii\u n\u ggal thumpath]=nang=le        [Passara   katha arà-biilang    nigiri]=nang=le dìkkath. \\ % bf
         \textsc{3s} \textsc{past}-die place=\textsc{dat}=\textsc{addit} Passara \textsc{quot} \textsc{non.past}-say country=\textsc{dat}=\textsc{addit} vicinity\\
 	`The place where he died is close to the village called Passara.' (B060115nar05)
\z
}

Locations can also be relativized on in headless relative clauses \formref{sec:nppp:Headlessrelativeclauses}, as in \xref{ex:cl:relc:loc3}, where the place of graduating (\em passing out \em in Sri Lanka) is relativized upon.

\xbox{16}{
\ea\label{ex:cl:relc:loc3}
\gll [See anà-{\em pass.out} \zero{}] abbisdhaathang       {\em University} of Peradeniya=ka\\ % bf
      \textsc{1s} \textsc{past}-graduate { } \textsc{copula} University of Peradeniya=\textsc{loc} \\
    `Where I graduated was the University of Peradeniya.' (K061026prs01)
\z
} \\

\subsubsection{Time}\label{sec:cls:Time}
Points in time can be relativized on, such as \trs{haari}{day} in \xref{ex:cl:relc:time1}, or \trs{thaaun}{year} in \xref{ex:cl:relc:time2}.

\xbox{16}{
\ea\label{ex:cl:relc:time1}
\gll Suda [puthri=le biini=le arà-caa\u nda haari]=le su-dhaathang. \\ % bf
      so queen=\textsc{addit} wife=\textsc{addit} \textsc{simult}-meet day=\textsc{addit} \textsc{past}-come \\
    `So then the day came when the wife and the prince were to meet.'  (K070000wrt05)
\z      
}\\ 

  
\xbox{16}{
\ea\label{ex:cl:relc:time2}
\gll Itthu thaaun=jo [Mahathma Gandhi arà-buunu thaaun]. \\ % bf
dist year=\textsc{emph} Mahathma Gandhi \textsc{non.past}-kill year \\
`That year was the year that Mahathma Gandhi was killed.' (K051213nar02)
\z
}


% \xbox{16}{
% \ea\label{ex:cl:relc:time3}
% \gll Se=ppe [nya-laaher {\em date}] duuva duuva 1960. \\ % bf
%      \textsc{1s}=\textsc{poss} \textsc{past}-date two two 1960  \\
%     `My birthday is 2-2-1960.' (K061019prs01)
% \z
% } \\

\subsubsection{Instrument}\label{sec:cls:Instrument}
The following sentences shows that the instrument for collecting can be found in a relative clause.


\xbox{16}{
\ea\label{ex:cl:relc:instr}
\gll Incayang=ka [[bannyak panthas ummas baarang pada=le   bathu inthan pada=le anà-punnu-kang]$_{RELC}$ bìssar beecek caaya hathu {\em bag}$_{head noun}$] su-aada. \\ % bf
       \textsc{3s}=\textsc{loc} many beautiful gold good \textsc{pl}=\textsc{addit} stone value \textsc{pl}=\textsc{addit} \textsc{past}-lot-\textsc{caus} big mud colour \textsc{indef} bag \textsc{past}-exist\\
    `With him, he had a big brown bag with which he collected very beautiful golden items and jewellery.'  (K070000wrt04)
\z      
}\\ 

\subsubsection{Manner}\label{sec:cls:Manner}
The relativized argument can have the role of Manner.

\xbox{16}{
\ea\label{ex:cl:relc:manner}
\gll Itthu [se arà-kirija mosthor]=jo. \\ % bf
 \textsc{dist} \textsc{1s} \textsc{non.past}-make manner=\textsc{emph}\\
`That's the way I do it.' (B060115rcp01)
\z
}   


\subsubsection{Purpose}\label{sec:cls:Purpose}
If the relativized role is Purpose, the relative clause will be in the infinitive.

\xbox{16}{
\ea\label{ex:cl:relc:purpose1}
\gll Derang pada=nang [itthu mà-kumpul athu mosthor] thraa. \\ % bf
      \textsc{3pl} \textsc{pl}=\textsc{dat} \textsc{dist} \textsc{inf}-add \textsc{indef} way  \textsc{neg}\\
    `There is no way for them to collect all this.'  (G051222nar01)
\z      
}\\ 


\xbox{16}{
\ea\label{ex:cl:relc:purpose2}
\gll Itthu=nang aada  [{\em divorce} mà-kijja=nang hatthu prentha oorang]. \\ % bf
      \textsc{dist}=\textsc{dat} exist divorce \textsc{inf}-make=\textsc{dat} \textsc{indef} law man \\
    `Then there is a lawyer to make the divorce.'  (K061122nar01)
\z      
}\\

\subsection{The Relative Clause Construction used as a complement}\label{sec:cls:Fact}
Besides the use of the Relative Clause Construction to modify referents, as illustrated above, this construction can also be used to indicate the complement of nouns. In \xref{ex:cl:relc:fact}, the Relative Clause Construction is used for the complement of \trs{habbar}{news}.

\xbox{16}{
\ea\label{ex:cl:relc:fact}
\gll Se=dang habbar, [ini  laama {\em car} pada samma inni {\em suicide} bombers asà-dhaathang {\em car}=yang, aapa,   arà-paavicci     katha habbar]. \\ % bf
      \textsc{1s=dat} news \textsc{prox} old cat \textsc{pl} all \textsc{prox} suicide bombers \textsc{cp}-come car=\textsc{acc} what \textsc{non.past}-use \textsc{quot} news \\
    `I have information, information that these suicide bombers come and take all these old cars and, what, and use them.' (K051206nar19)
\z
} \\

The semantic distinction between, say, \xref{ex:cl:relc:purpose2} and \xref{ex:cl:relc:fact} is that in \xref{ex:cl:relc:purpose2} the subordinate clause modifies and restricts the reference of \trs{prentha oorang}{lawyer}. In \xref{ex:cl:relc:fact}, the subordinate clause gives the \em content \em of the news; the subordinate clause does not restrict the reference of \trs{habbar}{news} to those news which take cars and use them. Note that in modifying Relative Clause Construction, there is one argument which subordinate clause and main clause share, e.g. \trs{prentha oorang}{lawyer} in \xref{ex:cl:relc:purpose2}. This is not the case for \trs{habbar}{news} in \xref{ex:cl:relc:fact}, which is not an argument of the subordinate clause. Furthermore, all semantic arguments of the subordinate clause are syntactically present in \xref{ex:cl:relc:fact}, (agent \em suicide bombers\em, theme \em car\em), while in \xref{ex:cl:relc:purpose2}, the argument \trs{prentha oorang}{lawyer} is semantically present in both clauses, but syntactically, only in the main clause.

Next to finite clauses as \xref{ex:cl:relc:fact}, clauses containing the infinitive marker \em mà- \em can also be used as complement clauses.

\xbox{16}{
\ea\label{ex:cl:relc:ma:shareone}
\gll [Inni     mlaayu pada Sri  Lanka=nang    \textbf{mà}-dhaathang=nang {\em reason}]  aapa   katha arà-biilang. \\
     \textsc{prox} Malay \textsc{pl} Sri Lanka=\textsc{dat} \textsc{inf}-come=\textsc{dat} {} what \textsc{quot} \textsc{non.past}-tell  \\
    `(I) will tell (you now) what is the reason why these Malays came to Sri Lanka.'  (N060113nar01)
\z      
}\\

 
%K060116nar10  itthuka    aada hamma mlaayu kaaving  aapacare jaalang katha

\subsection{Discussion}
Typologically speaking,  SLM is of the Japanese type \citep[50f]{Lehmann1984}, which means that it has preposed relative clauses, and uses the same verb form in relative clause and main clause. Interestingly, SLM is  not of the Dravidian type \citep[70f]{Lehmann1984} like Sinhala or Tamil, which do  prepose relative clause, but use a  participle to distinguish the relative clause from the main clause. SLM has emulated many of the grammatical categories of Sinhala and/or Tamil, but it has not developed a relative participle as of now.

\citet{Slomanson2006cll}, based on Saldin (1996) (first edition of \citet{Saldin2001}), assumes a relative pronoun \em nya\em. The sentence provided for exemplification, however, suggests that we are not dealing with a relative pronoun, but with an allomorph of the past tense marker \em anà-\em. This sentence is given below:


\xbox{14}{
\ea
\gll   [[Jalan ka nya lari] aya\ng]] baisikal atu ka t\E rbuntur su mati-mati\\
       road \textsc{p} \textsc{rel} run chicken bicyle \textsc{det} \textsc{p} struck \textsc{past} die\\
    `The chicken which ran along the road was struck by a bicycle and died.' \citep[153]{Slomanson2006cll} (original orthography)
\z
} \\

There are a number of oddities in this sentences, e.g. the reduplication of the final verb, which I will not go into here. What is important is that the string \em nya lari aya\ng{} \em could as well be analyzed as \em nya-lari aya\ng\em, where \em nya-  \em would be an allomorph of the past tense prefix \em anà-\em, as discussed in \formref{sec:morph:ana-}. This is thus a normal finite past tense prefix, and not a relative pronoun. 


\section{Conjunctive participle clause}\label{sec:cls:Conjunctiveparticipleclause}

For sequence of events, a hypotactical construction exists, which consists in using the conjunctive participle (infinite) in all but the last clause.  The finite clause has to be the last event.\footnote{These clause chains \citep{Longacre1985combi} are a typical feature of South Asia \citep{Emeneau1971dravidindoaryan}.}

\cb{ 
$
	\left[ 
		\NP *
	 	\left\{\begin{array}{c}as\grave{a}-\\jam\grave{a}-\end{array}\right\}-V
	\right]
$* MAIN CLAUSE
}

The subordinate clauses may only have verbal predicates. This is normally no drawback since all [+dynamic] states-of-affairs are coded by verbs, and the occurrence of [-dynamic] states-of-affairs in a sequence is limited. If another state-of-affair is to be used in a sequence, it will get a dynamic reading (see \funcref{sec:func:Events} for strategies how to do this).

Example \xref{ex:cl:cp:intro} shows a typical instance of this construction. There are three clauses, of which the first two contain a verb marked by \em asà- \em, while the last one contains a verb in another tense. Note that in this case, the referent \trs{oorang pada}{the people} is introduced in the first clause, while in English, it would be introduced in the matrix clause, in this case the last one.


\xbox{16}{
\ea\label{ex:cl:cp:intro}
 \ea 
 \gll Oorang pada \textbf{asà-}pìrrang, \\
	 man \textsc{pl} \textsc{cp}-wage.war  \\
	`After having waged war'
	\ex
 \gll derang=nang \textbf{asà-}banthu, \\
	\textsc{3pl}=\textsc{dat} \textsc{cp}-help\\
	`and after having helped them'
	\ex
	\gll siini=jo se-cii\u n\u ggal. \\ % bf
	here=\textsc{emph} \textsc{past}-settle\\
	`the people settled down right here.' (K051222nar03)
	\z
\z
}

If one of the subordinate clauses is in the negative, \em jamà- \em \formref{sec:morph:jama-} is used instead of \em asà-\em.
 
\xbox{16}{
\ea\label{ex:cl:cp:jama}
\ea 
\gll Liivath aayer \textbf{jamà}-jaadi=\textbf{nang} \\
     much water \textsc{neg.nonfin}-become=\textsc{dat}  \\
    `Without putting too much water (=having not put too much water)'  
\ex
\gll itthu aayer=yang hathu blaangan=nang luppas. \\ % bf
     \textsc{dist} water=\textsc{acc} \textsc{indef} amount=\textsc{dat} leave  \\
    `leave that water for a while.' (K060103rec01)
\z
\z
} \\

There are no restrictions on the roles that the arguments of the conjoined clauses may have. Other languages allow conjunctive participle constructions only if the subject (in those languages) are identical. This is the case for most South Asian languages \citep[108]{Masica1976}, but not for SLM  as \xref{ex:goasniinggal} shows:\footnote{\citet[58]{Gair1976sinhalasubject} shows that this is not the case for Sinhala either, although \citet{GairEtAl1989Sinhalaacquisition} return to speaking of `subjects' in Sinhala, relevant for this construction. However, the examples used in the latter paper include dative marking on the purported subjects, so that `topical argument' seems to be a more relevant characterization than morphosyntactic subject.}

\xbox{16}{
\ea\label{ex:goasniinggal}
\ea
\gll Go \textbf{asà}-nii\u n\u ggal, \\
\textsc{1s.familiar} \textsc{cp}-die \\
`I having died'
\ex 
\gll Alla  go=nya   \textbf{asà}-dhaathang, \\
Allah \textsc{1s.familiar}=\textsc{dat} \textsc{cp}-come\\
``Allah having come towards me''
\ex 
\gll Kuburan     \textbf{asà}-gaali, \\
 grave \textsc{cp}-dig \\
`The grave having been dug'
\ex 
\gll Go=nya   kubur-king!    \\ % bf
      \textsc{1s.familiar}=\textsc{acc} bury-\textsc{caus}   \\
      Bury me!' \\
    `I  die and Allah  comes for me and the grave will be dug and they will have me buried.' (B060115nar05)
\z
\z
} \\

The only argument of the first clause is \trs{go}{I}, while the agent/`subject' of the second one is \em Allah\em. The arguments of the first and the second clause are not identical. One could argue that the restriction on coordination only holds for zero-anaphora. This is also not the case in SLM, as the third and fourth clause show. If this restriction held, \em Allah \em would be the subject of these clauses, but this is obviously not the case.


Another example of reference changing between the \em asà\em-clauses and the main clause is \xref{ex:cl:cp:ds}, where the first two lines are about the parents and the third one is about the children. Nevertheless, all but the last clause are marked by \em asà-\em.


\xbox{16}{
\ea\label{ex:cl:cp:ds}
\ea
\gll Nni      nigiri=ka=jo    \zero$_{i(ag)}$ kitham=pe     [aanak buva pada]$_j$=yang   asà-simpang. \\ % bf
      \textsc{prox} country=\textsc{loc}=\textsc{emph} { } \textsc{1pl}=\textsc{poss} child fruit \textsc{pl}=\textsc{acc} \textsc{cp}-keep \\
    `\textbf{We} have raised our children in this country and'
\ex
\gll \zero$_i$ \zero$_j$ inni     {\em schools} pada=nang   asà-kiiring, samma asà-kirja. \\ % bf
     { } { }  \textsc{prox} schools \textsc{pl}=\textsc{dat} \textsc{cp}-send all \textsc{cp}-make\\
    `\textbf{We} have send them to the schools here and done all that and'
\ex
\gll Karang \zero$_j$ asà-blaajar, \zero$_j$   pukurjan asà-kirja   ambel. \\ % bf
       now { } \textsc{cp}-learn { } work \textsc{cp}-make take \\
    `now \textbf{they} have learned and \textbf{they} have taken up work'
\ex
\gll Skarang \zero$_{i,j}$ siini=jo  arà-duuduk. \\ % bf
      now { } here=\textsc{emph} \textsc{non.past}-live \\
    `and now \textbf{we} live here.'  (K051222nar04)
\z
\z
} \\

\section{Purposive clauses}\label{sec:cls:Purposiveclauses}
The purposive clause resembles the main clause, with the exception that the verb is in the infinitive and that the purposive clause cannot stand on its own. It must have a matrix clause it attaches to. Purposive clauses are often additionally marked with \em =nang\em, but this is optional \citep[cf.][139f]{Slomanson2006cll}.

\cb{$\left[\NP* \left\{\begin{array}{l}m\grave{a}-\\jam\grave{a}-\end{array}\right\} V\right]$(\textit{=nang}) MAINCLAUSE}
\cb{MAINCLAUSE $\left[\NP* \left\{\begin{array}{l}m\grave{a}-\\jam\grave{a}-\end{array}\right\}V\right]$(\textit{=nang})}

The purposive clause can be center-embedded in the main clause as in \xref{ex:subord:purp:center} or follow the predicate of the main clause as in \xref{ex:subord:purp:right}.

\xbox{16}{
\ea\label{ex:subord:purp:center}
\gll Derang [dìkkath=ka aada laapang]=nang [mà-maayeng]$_{purp}$=nang su-pii. \\ % bf
     \textsc{3pl} vicinity=\textsc{loc} exist ground=\textsc{dat} \textsc{inf}-play=\textsc{dat} \textsc{past}-go  \\
    `They went to play on the ground nearby.'  (K070000wrt04)
\z      
}\\ 

\xbox{16}{
\ea\label{ex:subord:purp:right}
\gll Itthu   {\em cave}=nang kithang=le pii aada [mà-liyath]$_{purp}$=nang. \\ % bf
 \textsc{dist} cave=\textsc{dat} \textsc{1pl}=\textsc{addit} go exist \textsc{inf}-look=\textsc{dat}    \\
    `We have also gone to that cave to have a look.'   (K051206nar02)
\z
}\\

% The use of   \em jamà- \em to yield negative purposive clauses found by \citet{Slomanson} is dubious in the Upcountry, as shown in \xref{ex:subord:purp:jama}.
% 
% 
% \xbox{16}{
% \ea\label{ex:subord:purp:jama}
% \gll ??se incayang=nang duvith anà-kaasi jamà-oomong katha. \\
%       \textsc{1s} \textsc{3s.polite}=\textsc{dat} money \textsc{past}-give \textsc{neg.inf}-talk \textsc{quot} \\
%     `I gave him money so that he would not talk.' (K081103eli04)
% \z
% } \\


\section{Subordinate interrogative clauses}\label{sec:cls:Subordinateinterrogativeclauses}
Interrogative pronouns can replace NPs in subordinates just as they can in main clauses. This is done if the speaker reports a question as in \xref{ex:subq:repq1}\xref{ex:subq:repq2}, or if he confesses his ignorance as in \xref{ex:subq:ign1}\xref{ex:subq:ign2}, or if he indicates who knows the answer \xref{ex:subq:answer}.

\xbox{16}{
\ea\label{ex:subq:repq1}
\gll Andare raaja=ka su-caanya [inni mà-kìrring simpang aada \textbf{aapa=yang}] katha. \\
     Andare king=\textsc{loc} \textsc{past}-ask \textsc{prox} \textsc{inf}-dry keep exist what=\textsc{acc} \textsc{quot} \\
   `Andare inquired from the King  what was it that was left [on the mat] to dry.'  (K070000wrt02)
\z      
}\\

\xbox{16}{
\ea\label{ex:subq:repq2}
\gll Incayang  arà-caari     [inni     avuliya \textbf{aapa} \textbf{mosthor}]  katha. \\
      \textsc{3s} \textsc{non.past}-search \textsc{prox} saint what manner \textsc{quot} \\
    `He is looking for how that saint was.' (B060115cvs04)
\z
} \\


\xbox{16}{
\ea\label{ex:subq:ign1}
\gll [Ithu \textbf{maana\~{}maana} thumpath] katha kithang=nang buthul=nang mà-biilang thàrboole. \\
     \textsc{dist} which\~{}\textsc{red} place \textsc{quot} \textsc{1pl}=\textsc{dat} correct=\textsc{dat} \textsc{inf}-say cannot  \\
    `We cannot tell you exactly which places those were.'  (K060108nar02)
\z      
}\\ 

\xbox{16}{
\ea\label{ex:subq:ign2}
\gll Se=ppe oorang pada [see \textbf{saapa}] katha thàrà-thaau subbath see=yang su-uubar. \\
     \textsc{1s}=\textsc{poss} man \textsc{pl} \textsc{1s} who \textsc{quot} \textsc{neg}-know because \textsc{1s}=\textsc{acc} \textsc{past}-chase\\
    `Because my folks did not know who I was, they chased me.'  (K070000wrt04)
\z      
}\\ 


\xbox{16}{
\ea\label{ex:subq:answer}
\gll Incayang=nang thaau itthu=pe mosthor=atthas punnu, cinggala \textbf{aapacara} anà-banthu katha. \\
      \textsc{3s.polite}=\textsc{dat} know \textsc{dist}=\textsc{poss} way=about  much Sinhala how \textsc{past}-help \textsc{quot} \\
    `He know a lot about that, about how the Sinhalese helped the Malays.'  (K051206nar04)
\z      
}\\



The five examples above contain content questions, which are marked by \em katha\em. \em Katha \em can be missing, as in the following example.


\xbox{16}{
\ea\label{ex:subq:nokatha}
\gll Derang thàrà-thaau [\textbf{aapacara} anà-jaadi] \zero{}. \\
     \textsc{3pl} \textsc{neg}-know  how \textsc{past}-happen { } \\
    `They did not know how it happened.'  (K051213nar01)
\z      
}\\ 



It is also possible to have truly embedded questions, which are marked by \em =so. \em This is possible for content questions as in as in \xref{ex:subq:so:content} and for polar questions \xref{ex:subq:so:polar}.

\xbox{16}{
\ea\label{ex:subq:so:content}
\ea
\gll Itthu see arà-gijja mosthor=jo. \\ % bf
      \textsc{dist} \textsc{1s} \textsc{non.past}-make way=\textsc{emph}  \\
    `That's the way I do it.'
\ex
\gll See ini arà-biilang. \\ % bf
      \textsc{1s} \textsc{prox} \textsc{non.past}-say \\
    `That's what I am saying.'
\ex
\gll [Laayeng oorang pada \textbf{aapacara} \textbf{arà-gijja=so}] thàrà-thaau. \\
      other man \textsc{pl} how \textsc{non.past}-make=\textsc{undet} \textsc{neg}-know  \\
    `I do not know how other people do it.'  (B060115rcp02)
\z
\z
}\\ 

\xbox{16}{
\ea\label{ex:subq:so:polar}
\gll [Inni samma anthi-oomong=\textbf{so}] thàrà-thaau. \\
     \textsc{prox} all \textsc{irr}-say=\textsc{undet} \textsc{neg}-know  \\
    `I do not know what they say about all that.'  (G051222nar02)
\z      
}\\ 


In this function, \em =so \em competes with \em =si\em, which is found in   \xref{ex:subq:si1} and \xref{ex:subq:si2}. 

\xbox{16}{
\ea\label{ex:subq:si1}
\gll [Aashik=nang hathu {\em soldier} mà-jaadi suuka]=\textbf{si} \textbf{katha} arà-caanya. \\
     Aashik=\textsc{dat} \textsc{indef} soldier \textsc{inf}-become like=\textsc{interr} \textsc{quot} \textsc{non.past}-ask  \\
    `He asks if you want to become a soldier, Ashik.' (B060115prs10)
\z
} \\


\xbox{16}{
\ea\label{ex:subq:si2}
\gll Incalla   [lai     thaau sudaara sudaari pada]=ka    bole=caanya    ambel [[nya-gijja    lai     saapa=kee  aada]=\textbf{si}    \textbf{katha}]. \\
      Hopefully other know brother sister \textsc{pl}=\textsc{loc} can-ask take \textsc{past}-make other who=\textsc{simil} exist=\textsc{interr} \textsc{quot} \\
    `Hopefully, you can enquire from another person you know whether there is someone else who did something.' (N061031nar01)
\z
} \\

Note that in these two examples the quotative \em katha \em is found, so that the clause containing \em =si \em can be analyzed as a reported main clause. In \xref{ex:subq:so:content} and \xref{ex:subq:so:polar}, on the other hand, \em katha \em is not found, so that we are not dealing with a reported string, but rather with a true subordinate clause. The utterance containing \em =si \em thus has its own illocution (question), which is reported in the matrix clause with assertive illocution. There are thus two utterances, and two illocutions.

When \em =so \em is used on the other hand, we are not dealing with an utterance, but only with a clause, which cannot have illocutionary force. This can also be seen from the absence of \em katha\em, which can only attach to utterances, but not to clauses. The \em =si\em- construction  thus contains two utterances, whereas the \em =so\em-construction only contains one.

The difference can be captured in the following two patterns.

\cbx{
\fbox{  \NP* \fbox{\fbox{\NP* V}_{cls}=\textit{si}}_{utt} ~\textit{katha} \NP*	 V}_{cls}}
{utt}
 
\cbx{
\fbox{ \NP* \fbox{\NP* V=\textit{so}}_{cls} \NP*  V}_{cls}}
{utt}

\section{Supraordination}\label{sec:cls:supraordination}
An interesting feature of SLM syntax is the possibility of the matrix clause occurring \em within \em the embedded clause. I will call this structure `supra-ordination'. This is very often done with exlamations of surprise with the aim to make the adressee aware of the exceptional nature of the content (mirativity). An answer is normally not expected. A typographical means to render this meaning in English is the combined use of the question mark and the exclamation mark as in \em Do you know what my boss just said !?!?\em. In the source language, I will indicate supraordination by inverted brackets ]...[ as in the following examples.

\xbox{16}{
\ea\label{ex:cl:supraord:rhet1}
\gll 
Suda inni moonyeth pada ]\textbf{aapa} \textbf{thaau=si}[ anà-gijja  !\\
 so \textsc{prox} monkey \textsc{pl} what know=\textsc{interr} \textsc{past}-do      \\
    `Do you know what these monkeys then did!?'  (K070000wrt01)
\z      
}\\ 

\xbox{16}{
\ea\label{ex:cl:supraord:rhet2}
\gll Suda derang pada siini derang itthu    oorang ]\textbf{aapa} \textbf{thaau=si}[   anà-gijja. \\
      thus \textsc{3pl} \textsc{pl} here \textsc{3pl} \textsc{dist} man what know=\textsc{interr} \textsc{past}-make \\
    `Do you know what these men did!?' (K051206nar07)
\z
} \\


\xbox{16}{
\ea\label{ex:cl:supraord:rhet3}
\gll Itthu=nang      blaakang derang pada ]\textbf{aapa} \textbf{thaau}[ anà-gijja? \\
     \textsc{dist}=\textsc{dat} after \textsc{3pl} \textsc{pl} what know \textsc{past}-make   \\
    `After that do you know what they did?' (K051206nar15)
\z
} \\
 

In the preceding examples, we are dealing with rhetorical questions where the answer is not provided yet. These sentences have a verb in the final position. In \xref{ex:cl:supraord:norhet}, things are a little bit different in that at the end of the utterance all the content is provided, i.e. the hearer does not have to try to find the answer to the rhetorical question because the answer (\trs{jiimath}{talisman}) is already provided.

 \xbox{16}{
\ea\label{ex:cl:supraord:norhet}
   \gll Kithang arà-thaaro ]\textbf{aapa} \textbf{thaau=si}[ jiimath. \\
   \textsc{1pl} \textsc{non.past}-put what know=\textsc{interr} talisman \\
`We put you know what? A talisman!' (K051206nar02)
\z
}


% \xbox{16}{
% \ea\label{ex:cl:supraord}
% \gll Itthu=nang      aapa diya   asà-dhaathang. \\
%       \textsc{dist}=\textsc{dat} what see \textsc{cp}-come \\
%     `See what happened then!' (K051206nar02)
% \z
% } \\
%  


% 
% 
% K061127nar03.trs:karang liyath
% K061127nar03.trs:kithang pe tsunami atthas ka ini aapa arà jaadi katha


\section{The position of adjuncts}\label{sec:cls:Thepositionofadjuncts}
Adjuncts are formed either by adverbs or by postpositional phrases. They can occur anywhere before, between or after NPs and predicates. The following schema illustrates this for a main clause with right-dislocation.

\cb{ (ADJCT) NP (ADJCT) NP (ADJCT) PRED (ADJCT) NP (ADJCT))}



The following three examples show the use of the adjunct \trs{karang}{now} in initial, medial and final position.

\xbox{16}{
\ea\label{ex:cl:adjct:karang:initial}
\gll \textbf{Karang} ini kitham=pe nigiri su-jaadi. \\
 now \textsc{prox} \textsc{1pl}=\textsc{poss} country \textsc{past}-become\\
`This (country) has now  became our country.' (K0512222nar04)
\z
}


\xbox{16}{
\ea\label{ex:cl:adjct:karang:medial}
\gll Go=dang            \textbf{karang} konnyong  thàràsìggar. \\
     1\textsc{s.familiar}=dat now little ill  \\
    `I am a bit sick these days.'  (B060115nar04)
\z      
}\\ 



\xbox{16}{
\ea\label{ex:cl:adjct:karang:final}
\gll {\em Associations}  pada Bahasa Indonesia  Bahasa Malaysia=le       {\em introduce}-kang   aada \textbf{karang}. \\
  associations \textsc{pl} Bahasa Indonesia Bahasa Malaysia=\textsc{addit} introduce-\textsc{caus} exist now    \\
    `Associations have now introduced Bahasa Indonesia and Bahasa Malaysia.' (G051222nar03)
\z      
}\\ 


\section{Reported speech}\label{sec:cls:Reportedspeech}
Reported speech is indicated by the particle \em katha \em which is put at the end of the reported string. \em Katha \em is used regardless of the  reported string being a clause, a subclausal unit, or an utterance.
Example \xref{ex:reportedspeech:clause} shows a reported clause, \xref{ex:reportedspeech:interj} shows the use of \em katha \em on an interjection, which does not meet the criteria to be a clause. See \formref{sec:morph:katha} for more discussion and examples.


\xbox{16}{
\ea\label{ex:reportedspeech:clause}
\gll Se=ppe      oorang thuuva pada    anà-biilang [kitham pada {\em Malaysia}=dering    anà-dhaathang]$_{CLS}$    katha. \\ % bf
 \textsc{1s}=\textsc{poss} man old \textsc{pl} \textsc{past}-say \textsc{1pl} \textsc{pl} Malaysia=\textsc{abl} \textsc{past}-come  \textsc{quot}\\
`My elders told me that we had come from Malaysia.' (K060108nar02)
\z
}

\xbox{16}{
\ea\label{ex:reportedspeech:interj}
\gll [{\em Yes}]$_{INTERJ}$  katha m-biilang. \\ % bf
 Yes \textsc{quot} \textsc{past}-say\\
`(X) said ``yes''.' (K060116nar11)
\z
}

The reported utterance can precede \xref{ex:cl:report:precede} or follow \xref{ex:cl:report:follow1}\xref{ex:cl:report:follow2} the clause containing the verb of utterance or cognizance.

\cb{
  \NP*
	$
	\left\{   
		\begin{array}{l}
					\rm V_{say}\\\rm	V_{cognize}
		\end{array}
	\right\}$											[		UTTERANCE		\textit{katha}	]
}

\cb{
  \NP*	[		UTTERANCE		\textit{katha}	] (NP)
	$
	\left\{
		\begin{array}{l}
					\rm V_{say}\\\rm	V_{cognize}
		\end{array}
	\right\}$										
}



\xbox{16}{
\ea\label{ex:cl:report:precede}
\gll [Luu=nya jadi-kang rabbu saapa lu=ppe nabi pada saapa katha] biilang. \\ % bf
      \textsc{2s.familiar}=\textsc{acc} become-\textsc{caus} prophet who \textsc{2s}=\textsc{poss} prophet \textsc{pl} who \textsc{quot} say \\
    `Say who the prophet is who made you, who are your prophets.'
\z
} \\

\xbox{16}{
\ea\label{ex:cl:report:follow1}
\gll Biilang [\zero{} se=ppe]    katha. \\ % bf
     say {} \textsc{1s}=\textsc{poss} \textsc{quot}  \\
    `Say that you are mine.' (K061123sng03)
\z
} \\ 

\xbox{16}{
\ea\label{ex:cl:report:follow2}
\gll Spaaru oorang pada    arà-biilang    [\zero{} Seelong=nang  {\em English} anà-aaji     baa]   katha. \\ % bf
     some man \textsc{pl} \textsc{non.past}-say { } Ceylon=\textsc{dat} English \textsc{past}-bring bring \textsc{quot}  \\
    `Some people say that it was the English who brought (us, the Malays,) to Sri Lanka.' (K060108nar02)
\z
} \\




While \em katha \em operates on the level of the utterance, and not on the level of the clause, deictic reference is still readjusted to the present speech situation. This means that tenses and pronouns change between the original string and the reported string. For instance, in \xref{ex:cl:report:follow1}, a fragment taken from a song, the loved one is asked to say that he belongs to the speaker, as conveyed by the translation `Say that you are \textbf{mine}'. Here, the deictic reference is readjusted since the literal string would have been `I am \textbf{yours}.' as in `Say: ``I am \textbf{yours}.'' ' Another example is \xref{ex:cl:report:deic:lu}, where in a double-embedded utterance, the child is supposed to say `Your mother has died', but the intended reading is of course that the child's mother had died, not the father's. This is obvious from the use of the familiar second person pronoun \em luu\em, which a father can use when addressing his child, but which would be impossible if the child was to address the father.

\xbox{16}{
\ea\label{ex:cl:report:deic:lu}
\ea
\gll Andare aanak=nang su-biilang:. \\ % bf
      Andare child=\textsc{dat} \textsc{past}-say \\
     `Andare said to his child:'
\ex
\gll Aanak. \\
     child  \\
    ` ``Son, '' '
\ex
\gll `\textbf{lu}=ppe        umma su-maathi'     katha  bitharak=apa  asà-naangis   mari. \\
      \textsc{2s}=\textsc{poss} mother \textsc{past}-die \textsc{quot} scream=after \textsc{cp}-weep come.\textsc{imp} \\
    ` ``come and cry and weep `Your (=my) mother has died!' '' (to fool the king)'
\z
\z
} \\

The deictic center is also readjusted in the temporal domain. In example \xref{ex:cl:report:deic:temp} the speaker reports expressing her being pleased. The literal string would have been `We are pleased' but in reported speech, this is changed to `We were pleased', because at the time of speaking, the event has already passed.


\xbox{16}{
\ea\label{ex:cl:report:deic:temp}
\gll [Suda kithang=le \textbf{su}-suuka]     katha anà-biilang. \\ % bf
     thus \textsc{1pl}=\textsc{addit} \textsc{past}-like \textsc{quot} \textsc{past}-say  \\
    `So we said that we also liked it.' (K051222nar01)
\z
} \\

However, this is not always the case. In \xref{ex:cl:report:deic:temp:no}, a narrative about past events (identifiable by the past marker \em su-\em) contains an embedded quotation in non-past tense (\em arà-\em). Still, \em arà- \em in this sentence could also be analyzed as simultaneous tense \formref{sec:morph:ara-}, so that it is difficult to decide whether temporal readjustment does indeed not take place.


\xbox{16}{
\ea\label{ex:cl:report:deic:temp:no}
\gll Andare ruuma=nang asà-pii biini=nang \textbf{su}-biilang [puthri=nang kuuping \textbf{arà}-dìnngar kuurang katha]. \\
    Andare house=\textsc{dat} \textsc{cp}-go wife=\textsc{dat} \textsc{past}-say queen=\textsc{dat} ear \textsc{simult}-hear little \textsc{quot}  \\
    `Andare went home and said to his wife that the queen did not hear well.' (K070000wrt05)
\z
} \\


Furthermore, imperative clauses in direct speech are rendered as infinitival clauses in reported speech.


\xbox{16}{
\ea
\gll Oorang padanang   \textbf{mà-}dhaathang katha asà-biilang. \\ % bf
      ma \textsc{pl}=\textsc{dat} \textsc{inf}-come \textsc{quot} \textsc{cp}-say \\
    `Having told the people to come ...' (B060115cvs01)
\z
} \\
 

 
% \xbox{16}{
% \ea\label{ex:constr:unreferenced}
% \gll Karang [inni     aapa  n-jaari]       katha arà-biilang. \\
%  now \textsc{prox} what \textsc{past}-become \textsc{quot} \textsc{non.past}-say\\
% `Now I will tell you what became of them.' (K060116nar10)
% \z
% }





Sinhala and Tamil use their quotative constructions in very similar situations, but in both languages, the quotative is homophonous to the conjunctive participle of the word meaning `to say'. This is not the case in SLM, where \trs{asà-biilang}{\textsc{cp}-say} is different from \em katha\em.


%command
%    se Farook  n bilang V kata
%    se sepe aanakna  subiila  Mekkana  (*se)  (m )pii kata
%        son goes
%    se sepe aankànà subiila  Mekkana  anti/ar  pii kata
%        father goes
%promise
%    se Farro a promiski  V kata




%B060115rcp02.txt: laeng       oorang pada apcara  kijja
%K051220nar02.txt:  biilang baae  apcara thaaro mandi     appa
%K060116nar10  karang inni     aapa  njaari       katha  arà-biilang
%K060116nar10  itthuka       aada hamma mlaayu kaaving  aapacare jaalang katha



\section{Agreement}\label{sec:cls:Agreement}
Although SLM does not show agreement on the verb as such, we are presently witnessing an incipient grammaticalization process which uses resumptive pronouns just before or after the verb if the antecedent is not immediately adjacent.\footnote{Whether this already counts as agreement is subject to debate and theoretical orientation \citep[cf.][99f.]{Corbett2006}.}
This might very well evolve into an agreement system.\footnote{The mechanism has been described by \citet{Givon1976tpga}.} As of now, however, there is no general agreement system to be found in SLM.



\xbox{16}{
\ea\label{ex:cl:agr:spaaman}
\gll \textbf{Spaaman} avuliya su-jaadi=\textbf{spaaman}. \\
 \textsc{3s.polite} saint \textsc{past}-become=\textsc{3s.polite}\\
`He became a saint.' (B060115nar05)
\z
} 

The source for this pattern is possibly Tamil, where the agreement suffix is very often very similar to the pronoun (\trs{n\textbf{iinga\dotl} poor\textbf{inga\dotl}}{You(pl.) are going.}) and etymologically related. However, some speakers use resumptive pronouns before the verb \xref{ex:cl:agr:incayang}-\xref{ex:cl:agr:derang}, not after the verb as Tamil influence would predict.


\xbox{16}{
\ea\label{ex:cl:agr:incayang}
\gll \textbf{Dr} \textbf{Draaman} duuva thaaun=nang blaakang \textbf{incayang}=su-mnii\u n\u ggal. \\
 Dr Draaman two year=\textsc{dat} after \textsc{3s.polite}=\textsc{past}-die\\
`After two years, Dr Draman died.' (K051213nar08)
\z
}

\xbox{16}{
\ea\label{ex:cl:agr:deram}
\gll Itthunam       \textbf{samma} \textbf{baae}  \textbf{oorang} \textbf{pada} inni     nigiri=ka=jo        \textbf{deram}=sa-duuduk. \\
therefore all good man \textsc{pl} \textsc{prox} country=\textsc{loc}=\textsc{emph} \textsc{3pl}=\textsc{past}-stay \\
`It is because of that that all good men settled down in this country.' (N060113nar01)
\z
}

\xbox{16}{
\ea\label{ex:cl:agr:derang}
\gll Suda \textbf{deram}  inni     {\em political} {\em promise}  hatthu \textbf{derang}=eng-kaasi     {\em 1958}=ka. \\
       so \textsc{3pl} \textsc{prox} political promise \textsc{indef} \textsc{3pl}=\textsc{past}-give 1958=\textsc{loc}\\
    `So they made this political promise in 1958.'  (N060113nar02)
\z      
}\\ 

As of now, the instances of this pattern are isolated, so that we cannot speak of SLM having a generalized agreement system.
