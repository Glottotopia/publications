\chapter{The noun phrase}\label{sec:form:ConstructionsNP}
In SLM, a nominal phrase  can be built around any one of the following:
Nouns \formref{sec:nppp:Nounphrasesbasedonanoun},
personal pronouns \formref{sec:nppp:Nounphrasesbasedonapersonalpronoun},
interrogative pronouns \formref{sec:nppp:Nounphrasesbasedoninterrogativepronouns},
deictics \formref{sec:nppp:Nounphrasesbasedonadeictic}, and
numerals \formref{sec:nppp:Nounphrasesbasedonanumeral/quantifier}.
A special case are reciprocal noun phrases \formref{sec:nppp:Reciprocalnounphrases}.

Besides NPs headed by words as given above, NPs can also be headed by higher order syntactic structures, namely
postpositional phrases \formref{sec:nppp:Nounphrasesbasedonapostpositionalphrase} and
clauses \formref{sec:nppp:Nounphrasesbasedonaclause}.
These possibilities will be discussed in turn, before I argue in the last section \formref{sec:nppp:TheSLMNPasappositional} that the SLM NP is non-configurational and has an appositional structure.

\section{Noun phrases based on a noun}\label{sec:nppp:Nounphrasesbasedonanoun}
The noun is a very common base for constructing a noun phrase to be used in a clause. The noun can be modified by a number of other items:
adjectives,
nouns,
deictics,
quantifiers,
numerals,
the indefinite article,
possessors,
relative clauses,
and the plural marker.
We will discuss NPs involving only one of these items in turn, before we turn to the order of the elements in more complex noun phrases in the post- \formref{sec:nppp:Relativeorderofpostnominalmodifiers} and prenominal domain \formref{sec:nppp:Relativeorderintheprenominalfield}. 

\begin{table}[t]
	\centering
		\begin{tabular}{rclc}
			left 	&   	& right & far right\\
			\hline
			INDEF 	&     	& \textsc{INDEF} & PL	\\
			ADJ   	&       & (ADJ)	&	\\
			N   	& Noun 	& (N) 	&	\\
			QUANT  	& 	& QUANT &	\\
			NUM  	& 	& NUM 	&	\\
			DEIC  	& 	& 	&	\\
			POSS  	& 	& 	&	\\
			RELC  	& 	& 	&	\\
		\end{tabular}
	\caption[Position of elements within the NP]{Position of elements within the NP. The general occurrence of modifiers to the left of the head word is noted by \citet{Adelaar1991, Adelaar2005struct, SmithEtAl2004}. Elements in parentheses are generally morphological compounds on the level of the word, rather than  syntactic postmodifications on the level of the NP, although this distinction is fuzzy.}
	\label{tab:PositionOfSeveralElementsWithinTheNP}
\end{table}


Table \ref{tab:PositionOfSeveralElementsWithinTheNP} gives an overview over the position of these elements within the NP.
The postnominal field offers less possibilities than the prenominal field. Fewer classes are represented there, and stacking of modifiers is not possible in the postnominal field. This differs from the prenominal field, where members of more classes can be found, and where modifiers can be stacked. Another important feature of the SLM NP is the free position of the indefinite article (symbolized by $\downarrow$), which can occur at several places and multiple times in the same NP. The full structure of the NP is given as an orientation in \xref{cb:np:preemptstructure}.

\cbx[\label{cb:np:preemptstructure}]{
$\downarrow$
$\left\{\begin{array}{l} \rm DEIC\\\rm POSS\end{array}\right\}$
RELC
$\left\{\begin{array}{l} \rm POSS\\\rm DEIC\end{array}\right\}$
$\downarrow$
QUANT
$\left\{\begin{array}{l} \rm NUM\\\rm ADJ\\\end{array}\right\}$*
$\downarrow$
N
$\downarrow$
\textbf{Noun}
$\begin{array}{c}\downarrow\\\rm (N)  \\\rm (ADJ) \\\rm QUANT \\\rm NUM \end{array}$
PL}{NP}

In the following, we will discuss modifications of a noun by different elements \formref{sec:nppp:NPscontaininganadjective}-\formref{sec:nppp:NPscontaininginterrogativepronounsusedforuniversalquantification}, then the relative order of postnominal modifiers \formref{sec:nppp:Relativeorderofpostnominalmodifiers}, then the relative order of prenominal modifiers \formref{sec:nppp:Relativeorderintheprenominalfield}, before we turn to the position of the indefinite article in \formref{sec:nppp:Thepositionoftheindefinitemodifier}. The full structure of the NP is discussed in more theoretical detail in \formref{sec:nppp:Thefinalstructureofthenounphrase}.



\subsection{NPs containing an adjective}\label{sec:nppp:NPscontaininganadjective}

NPs containing an adjective are very common. The following two examples show several NPs with prenominal adjectival modification.

\xbox{17}{
\ea\label{ex:constr:NP:ADJN1}
\gll \textbf{Laayeng}$_{ADJ}^\curvearrowright$    \textbf{nigiri}$_{N}$=pe      soojor    pada=nang  \textbf{baae}$_{ADJ}^\curvearrowright$   \textbf{lakuvan}$_{N}$=nang    anà-juuval. \\
      different country=\textsc{poss} European \textsc{pl}=\textsc{dat} good price=\textsc{dat} \textsc{past}-sell \\
    `(He) sold (it) to the Europeans from the other countries for a good price.'  (K060103nar01)
\z
}

\xbox{16}{
\ea\label{ex:constr:NP:ADJN2}
\gll \textbf{Baaru}$_{ADJ}^\curvearrowright$ \textbf{oorang}$_{N}$ pada masà-thaaro. \\
 new man \textsc{pl} must-put\\
`(We) must put new people.' (K060116nar11)
\z
}


The order of the adjective and the noun in the NP has been described by  \citet{Adelaar1991,Jayasuriya2002,SmithRH} and \citet{Slomanson2006cll}, who all list prenominal adjectives as the only possibility. This ADJ N order is the most common one, but the inverse order is also possible, but less often heard.
The ordering of adjective and noun is not lexically specified nor dependent on the speaker, as \xref{ex:constr:NP:NADJ:double} shows. In this example, raw beef is being referred to twice, first in the order ADJ N, then in the order N ADJ.\footnote{Also note that this example shows the different realization of raised schwa (\em ì \em in the first clause, \em ù \em in the last one).}

\xbox{16}{
\ea\label{ex:constr:NP:NADJ:double}
\ea
\gll \textbf{Mìntha}$_{ADJ}$ \textbf{daaging}$_{N}$=yang cuuci. \\ % bf
      raw beef=\textsc{acc} wash \\
    `Wash the raw beef.'
\ex
\gll asà-cuuci laada=le gaaram=le bathu giling-an=ka  giiling. \\ % bf
      \textsc{cp}-wash pepper=\textsc{addit} salt=\textsc{addit} stone grind=\textsc{nmlzr}=\textsc{loc} grind \\
    `Having washed it, grind salt and pepper in a grinding stone.'
\ex
\gll asà-giiling \textbf{daaging}$_{N}$ \textbf{mùntha}$_{ADJ}$=yang baathu=ka asà-thaaro, giccak. \\ % bf
     \textsc{cp}-grind beef raw=\textsc{acc} stone=\textsc{loc} \textsc{cp}-put smash  \\
    `Having ground and having put the raw beef on a stone, smash it.' (K060103rec02)
\z
\z
}


Given the low frequency of N+ADJ strings and the possibility to exlain them on the level of morphology, all N+ADJ strings are treated as compounds \formref{sec:wofo:Compounding} rather than phrase formation for  ease of exposition.

% \xbox{16}{
%  \ea\label{ex:constr:NP:unreferenced}
%    \gll Go=ppe     naama Badulla buulath thaau, lorang Mr.  Mahamud  katha kala-biilang  \\
%     \textsc{1s}=\textsc{poss} name Badulla     whole   know  \textsc{2pl} Mr Mahamud \textsc{quot} if say\\
% `Whole Badulla knows my name if you say Mr Mahamud' (B060115nar04)
% \z
% }


\subsection{NPs containing another noun}\label{sec:nppp:NPscontaininganothernoun}

While many N+N sequences can be analyzed as compounds on the  word level, this compounding analysis is not possible if other material intervenes, as can be the case with the indefinite article \em atthu\em. This article can separate the modifier from its head noun, as in \xref{ex:constr:NP:N:atthukavanan} and \xref{ex:constr:NP:N:atthupiingir}.

\xbox{16}{
\ea\label{ex:constr:NP:N:atthukavanan}
\gll Ini pohong atthas=ka \textbf{moonyeth} \textbf{hathu}=\textbf{kavanan} su-aada. \\
     \textsc{prox} tree top=\textsc{loc}  monkey \textsc{indef}=group \textsc{past}-exist\\
    `On top of this tree was a group of monkeys.'   (K070000wrt01)
\z
}

\xbox{16}{
\ea\label{ex:constr:NP:N:atthupiingir}
\gll [Ini oorang \el{} caape subbath] [[\textbf{jaalang} \textbf{hathu}=\textbf{pii\u n\u ggir}]=ka anà-aada hathu pohong] baava=ka su-see\u nder. \\
    \textsc{prox} man \el{} tired because road \textsc{indef}=border=\textsc{loc} \textsc{past}-exist.inanim \textsc{indef} tree down=\textsc{loc} \textsc{past}-rest   \\
    `Because he was tired, this man sat down under a tree which stood at the side of the street.'  (K070000wrt01)
\z
}

This separation of modifier and head indicates that we are not dealing with a morphological process on the word level, but with a syntactic process on the level of the NP. \trs{Moonyeth}{monkey} modifies \trs{kavanan}{group} in \xref{ex:constr:NP:N:atthukavanan} in  very much the same way as the adjective \trs{lai}{more} modifies \em kavanan \em in \xref{ex:constr:NP:N:laihathu}.


\xbox{16}{
\ea\label{ex:constr:NP:N:laihathu}
\gll Ikang Seelon=nang \textbf{lai} \textbf{hathu=kavanan} anà-dhaathang. \\
      then Ceylon=\textsc{dat} other \textsc{indef}=group \textsc{past}-come \\
    `Then, (yet) another group came to Sri Lanka.'  (K060108nar02)
\z
}

This parallel structure indicates that nouns can premodify other nouns on the syntactic level. Because of this, nominal premodification of nouns is always analyzed as syntactic in this description, while postmodification is treated as a morphological operation \formref{sec:wofo:Compoundsinvolvingtwonouns}.

An example for nominal premodification without the indefinite article is given in \xref{ex:constr:NP:N:noindef}.

\xbox{16}{
\ea\label{ex:constr:NP:N:noindef}
\gll Aapcara=ke       incayang   ini  \textbf{ciina} \textbf{oorang}  Islam=nang   asà-dhaathang  asà-kaaving=apa  karang mà-siigith=nang  arà-pii. \\ % bf
     how=\textsc{simil} \textsc{3s.polite} \textsc{prox} China man Islam=\textsc{dat} \textsc{cp}-come \textsc{cp}-marry=after now mosque=\textsc{dat} \textsc{non.past}-go  \\
    `Somehow he, this Chinaman converted to Islam and married and now goes to the mosque.' (K051220nar01)
\z
}

%
% Example \xref{ex:constr:NP:N:topo} shows a noun modified by a proper noun, a toponym. This is arguably also done on the level of the phrase.
%
% \xbox{16}{
% \ea\label{ex:constr:NP:N:topo}
% \gll Kitham  arà-mirthi-kang Kluu\u mbu$\curvearrowright$  {\em confederation}=nang. \\ % bf
%      \textsc{1pl} \textsc{non.past}-understand-\textsc{caus} Colombo confederation=\textsc{dat} \\
%     `We make the Colombo confederation understand.' (K060116nar23)
% \z
% }

Another instance of NPs containing two nouns is comparison. In \xref{ex:constr:NP:N:ke}, \em soldier \em is modified by \trs{baapake}{like daddy}. The intervening similative clitic \em =ke \em makes it impossible to treat this as a morphological pattern on the level of the noun; rather, we are dealing with a construction on the level of the noun phrase.


\xbox{16}{
\ea\label{ex:constr:NP:N:ke}
\gll Se=dang [[baapa]$_N$=ke {\em soldier}$_N$]$_{NP}$ mà-jaadi suuka. \\ % bf
     \textsc{1s=dat} father=\textsc{simil} soldier \textsc{inf}-become like  \\
    `I want to become a soldier like daddy.' (B060115prs10)
\z
}

Postnominal modification with nouns  (\trs{orang ikkang}{man'+ `fish'= fisherman}) is also possible, but is treated as a morphological phenomenon in this grammar, not as a syntactic phenomenon.
%
%
% \xbox{16}{
% \ea\label{ex:constr:NP:unreferenced}
% \gll Derang=nang maau mosthor baalas katha nya-biilang. \\
%       \textsc{3pl}=\textsc{dat} want manner answer \textsc{quot} \textsc{past}-say \\
%     I gave them the answer in the way they wanted ' (K051213nar01)
% \z
% }




\subsection{NPs containing a quantifier}\label{sec:nppp:NPscontainingaquantifier}

This type of NP is are also common. The quantifier is normally preposed \xref{ex:constr:NP:QUANTN}, but can also be floated. This is especially true for \trs{samma}{every} \xref{ex:constr:NP:floatQUANT1}-\xref{ex:constr:NP:floatQUANT3}.

\xbox{16}{
\ea\label{ex:constr:NP:QUANTN}
\gll Inni     sudaari=pe   femili=ka    \textbf{bannyak}$^\curvearrowright$ \textbf{oorang} tsunami=dang     spuukul su-pii. \\
      \textsc{prox} sister=\textsc{poss} family=\textsc{loc} many people tsunami=\textsc{dat} \textsc{cp}-hit \textsc{past}-go\\
    `In this sister's family, many people were swept away by the tsunami.' (B060115nar02)
\z
}

In \xref{ex:constr:NP:floatQUANT1}-\xref{ex:constr:NP:floatQUANT2a}, the quantifier occurs after the plural marker \em pada\em, which is the rightmost element in any NP (see below, \formref{sec:nppp:NPscontainingthepluralmarker}). This is an indication that the quantifier occurs outside of the NP over which it has scope.


\xbox{16}{
\ea\label{ex:constr:NP:floatQUANT1}
\gll [Go=ppe     aanak pada] \textbf{samma} baaye. \\
     \textsc{1s.familiar}=\textsc{poss} child \textsc{pl} all good  \\
    `My children are all good.' (B060115cvs13)
\z
}


\xbox{16}{
\ea\label{ex:constr:NP:floatQUANT2}
\gll [Kafan kaayeng pada] \textbf{samma} asà-ambel. \\
     shroud cloth \textsc{pl} all \textsc{cp}-take  \\
    `Having taken all the tissue for the shroud, ... .'  (B060115nar05)
\z
}

\xbox{16}{
\ea\label{ex:constr:NP:floatQUANT2a}
\gll [Kithang=pe     oorang thuuva  pada] \textbf{bannyak} dhaathang aada. \\
     \textsc{1pl}=\textsc{poss} man old \textsc{pl} many come exist  \\
    `Our ancestors came in great quantities.' (K060108nar02)
\z
}




In \xref{ex:constr:NP:floatQUANT3} the quantifier occurs even after the predicate, clearly outside the NP over which it has scope.

\xbox{16}{
\ea\label{ex:constr:NP:floatQUANT3}
\gll Kitham=pe {\em association} itthu vatthu \textbf{duvith} thraa \textbf{bannyak}. \\
 \textsc{1pl}=\textsc{poss} association \textsc{dist} time money \textsc{neg} much\\
`At that time, our association did not have much money.' (K060116nar01)
\z
}

\subsection{NPs containing numerals}\label{sec:nppp:NPscontainingnumerals}

Numerals are commonly found in NPs. The numeral normally precedes the noun, as in \xref{ex:constr:NP:NUMN1}\xref{ex:constr:NP:NUMN2}, but may also follow on rare occasions as in \xref{ex:constr:NP:NNUM1}.

\xbox{16}{
\ea\label{ex:constr:NP:NUMN1}
\gll \textbf{Thiiga} oorang, \textbf{thiiga} oorang=le, \textbf{thiiga} oorang pada=jo itthu ini {\em volleyball} arà-{\em play}-king=kee. \\
 three man, three man=\textsc{addit}, three man \textsc{pl}=\textsc{emph} \textsc{dist} \textsc{prox} volleyball \textsc{non.past}-play-\textsc{caus}=\textsc{simil}       \\
    `Three men, three persons, three people play it, like volleyball.'  (N060113nar05)
\z
}


\xbox{16}{
\ea\label{ex:constr:NP:NUMN2}
\gll \textbf{Duva-pulu}    \textbf{ìnnam} \textbf{riibu}    \textbf{ùmpath}  \textbf{raathus} \textbf{lima-pulu}    \textbf{duuva} {\em votes}  incayang=nang    anà-daapath. \\ % bf
 two-ty six thousand four hundred five-ty two votes \textsc{3s.polite}=\textsc{dat} \textsc{past}-get\\
    `He got 26,452 votes.' (N061031nar01)
\z
}


\xbox{16}{
\ea\label{ex:constr:NP:NNUM1}
\gll [Panthas  rooja   kumbang pohong  komplok] $\curvearrowleft$duuva asà-jaadi su-aada. \\
      beautiful rose flower tree bush two \textsc{cp}-grow \textsc{past}-exist \\
    `Two beautiful rose bushes had grown.'  (K070000wrt04)
\z
}

% \xbox{16}{
%  \ea\label{ex:constr:NP:NNUM1}
%    \gll Se=dang aade pada  \textbf{umpath} arà-duuduk. \\
%     \textsc{1s=dat} younger.sibling \textsc{pl} four \textsc{non.past}-exist.\textsc{anim}\\
% `I have four younger siblings' (K060108nar01,K081105eli02)
% \z
% }


%
% \xbox{16}{
% \ea\label{ex:constr:NP:NNUM3}
% \gll aanak pada duuva aada. \\
%       child \textsc{pl} two exist \\
%     `There are two children.' (B060115prs14)
% \z
% }
%
%
%
% \xbox{16}{
% \ea\label{ex:constr:NP:unreferenced}
% \gll {\em school}=ka  kithang  mulbar aanak pada  cinggala aanak pada  sraani pada mlaayu pada samma hatthu  samma  hatthu=nang=jo anà-duuduk. \\
%      School=\textsc{loc} \textsc{1pl}  Tamil child \textsc{pl} Sinhala child \textsc{pl} Burgher  \textsc{pl} Malay \textsc{pl} all one all one=\textsc{dat}=\textsc{emph} \textsc{past}-stay\\
%     `In school, the Tamil children, the Sinhalese children, Burghers and Malays, we all stayed in one class.' (K051213nar03)
% \z
% }

% In Sinhala, numerals have to follow the noun. This is also an option in (Jaffna) Tamil \citep{GairIsolate}; generally speaking, Tamil prefers preposing numerals, though.

\subsection{NPs containing the indefinite article}\label{sec:nppp:NPscontainingtheindefinitearticle}

If the indefinite article is present within the NP, it can either
 be preposed \xref{ex:constr:NP:hatthu:preposed1}\xref{ex:constr:NP:hatthu:preposed2},
 postposed \xref{ex:constr:NP:hatthu:postposed1}\xref{ex:constr:NP:hatthu:postposed2},
or both pre- and postposed
\xref{ex:constr:NP:hatthu:prepost1}\xref{ex:constr:NP:hatthu:prepost2}, as the following six examples illustrate.



\xbox{16}{
\ea\label{ex:constr:NP:hatthu:preposed1}
\gll \textbf{Hatthu} avuliya aada kitham=pe ruuma dìkkath. \\
      \textsc{indef} saint exist \textsc{1pl}=\textsc{poss}  house vicinity \\
    `There is a saint close to our house.' (K060108nar02)
\z
}


\xbox{16}{
\ea\label{ex:constr:NP:hatthu:preposed2}
\gll \textbf{Hathu} haari, \textbf{hathu}=oorang [thoppi mà-juval]=nang kampong=dering kampong=nang su-jaalang pii. \\
     \textsc{indef} day \textsc{indef}=man hat \textsc{inf}-sell=\textsc{dat} village=\textsc{abl} village=\textsc{dat} \textsc{past}-walk go  \\
    `One day, a man walked from village to village to sell hats.'  (K070000wrt01)
\z
}



\xbox{16}{
\ea\label{ex:constr:NP:hatthu:postposed1}
\gll See avuliya \textbf{atthu} su-jaadi. \\
      \textsc{1s} saint \textsc{indef} \textsc{past}-become \\
    `I have become a saint.'  (K051220nar01)
\z
}

\xbox{16}{
\ea\label{ex:constr:NP:hatthu:postposed2}
\gll Mà-blaajar=nang see anà-kiiring se=ppe maama \textbf{hatthu}=pe ruuma=nang. \\
     \textsc{inf}-learn=\textsc{dat} \textsc{1s} \textsc{past}-send \textsc{1s}=\textsc{poss} uncle \textsc{indef}=\textsc{poss} house=\textsc{dat}  \\
    `I was sent to an uncle of mine's to study.' (K051213nar02)
\z
}


\xbox{16}{
\ea\label{ex:constr:NP:hatthu:prepost1}
\gll Sithu=ka \textbf{hathu} maccan \textbf{hathu}  duuduk aada. \\
     there=\textsc{loc} \textsc{indef} tiger \textsc{indef} stay exist  \\
    `A tiger stayed there.'  (B060115nar05)
\z
}


\xbox{16}{
\ea\label{ex:constr:NP:hatthu:prepost2}
\gll \textbf{Hathu} kaaving \textbf{hatthu}=nang   kapang-pii \\
      \textsc{indef} wedding \textsc{indef}=\textsc{dat} when-go \\
    `When we go to a wedding' (G051222nar04)
\z
}


%\xbox{16}{
%\ea\label{ex:constr:NP:atthu:atthuNatthu}
%\gll Kitham=pe   \textbf{atthu}=3-{\em tonner}=\textbf{atthu} aada, duppang=ka. \\
%      \textsc{1pl}=\textsc{poss} \textsc{indef}=3-tonner=\textsc{indef} exist, front=\textsc{loc} \\
%    `There was a three-tonner of ours at the front.'  (K051206nar16)
%\z
%}


% \xbox{16}{
%  \ea\label{ex:constr:NP:unreferenced}
%    \gll Kithang=nang   hathu  {\em job} hatthu mà-ambel=nang      kithang=nang   hathu  {\em application} mà-sign  kamauvan vakthu. \\
%     \textsc{1pl}=\textsc{dat} \textsc{indef} job \textsc{indef} \textsc{inf}-take=\textsc{dat} \textsc{1pl}=\textsc{dat} \textsc{indef} application \textsc{inf}-sign want time\\
% `When we want to take a job, when we want to sign an application' (K051206nar12)
% \z
% }

Within one idiolect, the positioning of \em (h)at(t)hu \em can vary. In the following stretch of discourse, we find two instances of nouns with following \em hatthu \em and one instance of pre-  and postnominal \em hat(t)hu\em.

\xbox{16}{
\ea\label{ex:constr:NP:hatthu:double}
\gll Kaaving \textbf{hatthu}=nang arà-pii vakthu,  jalang-an \textbf{hatthu} arà-pii vakthu,  \textbf{hathu} mayyeth \textbf{hatthu} arà-mnaaji. \\
     wedding \textsc{indef} \textsc{non.past}-go time walk-\textsc{nmlzr} \textsc{indef} \textsc{non.past}-go time \textsc{indef} corpse \textsc{indef} \textsc{non.past}-pray  \\
    `When we go to a wedding, when we go on a trip, when we recite for a dead person.' (K051213nar06)
\z
}

\em Hat(t)hu \em is actually found quite often more than once in an NP. This is often once preposed and once postposed, but double occurrences in the prenominal field can also be found. This is the case in \xref{ex:constr:NP:hatthu:double:pre}, where the indefiniteness marker for \trs{makanan}{food} occurs three times.

\xbox{16}{
\ea\label{ex:constr:NP:hatthu:double:pre}
\gll Itthu=le [\textbf{hathu}  [mlaayu oorang pada=pe]$_{poss}$ \textbf{hathu} [baae]$_{ADJ}$ \textbf{hathu} makanan]$_{NP}$=jo. \\
     \textsc{dist}=\textsc{addit} \textsc{indef} Malay man \textsc{pl}=\textsc{poss} \textsc{indef} good \textsc{indef} food=\textsc{emph}  \\
    `That is also a good dish of the Malays.' (K061026rcp02)
\z
}

This flexibility of the position of the indefiniteness marker suggests that it does not have a fixed position in the NP but can occur between any two constituents in the NP. Whereas the possessor, the adjective, the relative clause etc. have more or less predictible positions, the indefiniteness marker defies such predictions. This will be developed in more detail below \formref{sec:nppp:Thepositionoftheindefinitemodifier}.

%
% \xbox{16}{
% \ea\label{ex:constr:NP:unreferenced}
% \gll Itthu butthul hathu bìrrath,  bìrrath hathu makanan. \\
%       \textsc{dist} very \textsc{indef} heavy heavy \textsc{indef} food \\
%     `That is a very heavy meal.' (K061026rcp04)
% \z
% }

\subsection{NPs containing deictics}\label{sec:nppp:NPscontainingdeictics}
Deictics always precede the noun (\citet[29]{Adelaar1991}, \citet[214]{Adelaar2005struct}, \citet[137]{Slomanson2006cll}). The following two examples show this for the proximal deictic \em in(n)i \em and for the distal deictic \em it(t)hu\em.


\xbox{16}{
\ea\label{ex:constr:NP:deic:ini}
\gll \textbf{Inni} maccan su-baavung kiyang. \\
      \textsc{prox} tiger \textsc{past}-rise \textsc{evid} \\
    `The tiger apparently got up.'  (B060115nar05)
\z
}

\xbox{16}{
\ea\label{ex:constr:NP:deic:itthu}
\gll \textbf{Ithu} jaalang=ka  mà-pii thàràboole. \\
      \textsc{dist} road=\textsc{loc} \textsc{inf}-go cannot \\
    `You could not take that road.'  (B060115nar05)
\z
}



\subsection{NPs containing possessors}\label{sec:nppp:NPscontainingpossessors}
Possessors always precede the head noun \citep{Adelaar1991,Jayasuriya2002,Slomanson2006cll}. Examples \xref{ex:constr:NP:poss:simple} and \xref{ex:constr:NP:poss:triple} shows this for simple possession, while \xref{ex:constr:NP:poss:double} shows more complex recursive possessive relationships.

\xbox{16}{
\ea\label{ex:constr:NP:poss:simple}
\gll [Se=ppe$_{POSS}$    naama$_N$]$_{NP}$ Mohomed Imran Salim. \\ % bf
     \textsc{1s}=\textsc{poss} name Mohomed Imran Salim  \\
    `My name is Mohomed Imran Salim.' (K060108nar01)
\z
}

\xbox{16}{
\ea\label{ex:constr:NP:poss:triple}
\gll [Incayang=\textbf{pe}       wife]=le,       [{\em wife}=\textbf{pe}  baapa]=le  [masigith=\textbf{pe}  bìssar] mas-panggel. \\ % bf
\textsc{3s.polite}=\textsc{poss}   w.=\textsc{addit} w=\textsc{poss}  father=\textsc{addit} mosque=\textsc{poss} chief must-call \\
    `His wife and her father had to call the mosque's head.'  (K051220nar01)
\z
}

\xbox{16}{
\ea\label{ex:constr:NP:poss:double}
\gll [[Kithang=\textbf{pe} baapa]=\textbf{pe} naama] Mahamud. \\ % bf
     \textsc{1pl}=\textsc{poss} father=\textsc{poss} name Mahamud  \\
    `Our father's name is Mahamud.' (B060115nar03)
\z
}

%
% \xbox{16}{
% \ea\label{ex:constr:NP:unreferenced}
% \gll Itthu    abbisdhaathang       hathu {\em traditional} {\em food} hatthu oorang mlaayu=pe. \\
%       \textsc{dist} \textsc{copula} \textsc{indef} { } { }  \textsc{indef} man Malay=\textsc{poss} \\
%     `This is a traditional food of the Malays.' (K061026rcp01)
% \z
% } \\

\subsection{NPs containing locations}\label{sec:nppp:NPscontaininglocations}
These are rare. Normally, the location is put in a relative clause as in \xref{ex:constr:NP:loc}.


\xbox{16}{
\ea\label{ex:constr:NP:loc}
\gll Meeja=ka *(aada) maa\u n\u gga=yang kaasi. \\
     table=\textsc{loc} exist mango=\textsc{acc} give  \\
    `Give me the mango (which is) on the table' (K081118eli01)
\z
}
%
% One could argue that in \xref{ex:constr:NP:loc:dubious}, there is also a location involved, which modifies the head noun \em association\em. Since there is no locative marker \em =ka \em present, this argument is somewhat doubtful, but still the location would precede the head noun.
%
%
% \xbox{16}{
% \ea\label{ex:constr:NP:loc:dubious}
% \gll Kitham  arà-mirthi-kang Kluu\u mbu$\curvearrowright$  {\em confederation}=nang. \\ % bf
%      \textsc{1pl} \textsc{non.past}-understand-\textsc{caus} Colombo c.=\textsc{dat} \\
%     `We make the Colombo confederation understand.' (K060116nar23)
% \z
% } \\

\subsection{NPs containing relative clauses}\label{sec:nppp:NPscontainingrelativeclauses}
Relative clause always precede the head noun. Example \xref{ex:constr:NP:relc} illustrates this. The topic of relative clauses will be treated in more detail in \formref{sec:cls:Relativeclause}.

\xbox{16}{
\ea\label{ex:constr:NP:relc}
\gll [Seelon=nang dhaathang aada \zero] mlaayu oorang ikkang. \\ % bf
 Ceylon-\textsc{dat} come exist { } Malay man fish\\
`The Malays who came to Sri Lanka were fishermen.' (K060108nar02)
\z
}

\subsection{NPs containing the plural marker}\label{sec:nppp:NPscontainingthepluralmarker}
The plural marker is always at the rightmost position.
Example \xref{ex:constr:NP:pada} illustrates this.

\xbox{16}{
\ea\label{ex:constr:NP:pada}
\gll [Spaaru oorang \textbf{pada}] su-pii. \\
     some man \textsc{pl} \textsc{past}-go  \\
    `Some men left.'  (B060115nar01)
\z
}

It is impossible to have anything pertaining to the NP after \em pada \em with the exception of floated quantifiers \formref{sec:nppp:NPscontainingaquantifier}, of which one example is repeated here for convenience.

\xbox{16}{
\ea\label{ex:constr:NP:pada:float}
\gll Go=ppe     aanak \textbf{pada} \textbf{samma} baaye. \\
     \textsc{1s.familiar}=\textsc{poss} child \textsc{pl} all good  \\
    `My children are all good.' (B060115cvs13)
\z
}


\subsection{NPs containing indefinite expressions}\label{sec:nppp:NPscontainingindefiniteexpressions}
There is no indefinite pronoun strictly speaking in SLM, but the combination of an interrogative pronoun with the clitic  \em =ke \em can be used for that end.  This construction precedes the noun.

\xbox{16}{
\ea\label{ex:constr:NP:indefpron}
\gll {\em Important} {\em occasion} pada kala aada, \textbf{aapa=ke} \textbf{{\em festival}} pada laayeng  {\em wedding} {\em ceremony}, [...]  ithu vakthu kithang arà-kirja \\
     important occasion \textsc{pl} if exist what=\textsc{simil} festival \textsc{pl} other wedding ceremony \el{} \textsc{dist} time \textsc{1pl} \textsc{non.past}-make  \\
    `If there is an important occasion, some festival or otherwise a wedding ceremony, we will prepare it.' (K061026rcp01,K081105eli02)
\z
}


%
% \xbox{16}{
% \ea\label{ex:constr:NP:unreferenced}
% \gll 1978     {\em exam} gijja=nang  blaakang incayang    suda aapa=ke hathu  pukurjan magirja    arà-diyath duudu. \\
%       1978 exam make=\textsc{dat} after \textsc{3s.polite} thus what=\textsc{simil} \textsc{indef} work \textsc{inf}-make \textsc{non.past}-see exist.\textsc{anim} \\
%     `After having done the 1978 exam, he will try find some work to do.' (K051222nar08)
% \z
% } \\

\subsection{NPs containing interrogative pronouns used for universal quantification}\label{sec:nppp:NPscontaininginterrogativepronounsusedforuniversalquantification}
An interrogative pronoun can occur in the left field of an NP if the additive clitic \em =le \em is present at the right edge of the NP. If a postposition is used on the NP, then the additive clitic attaches to the right of that postposition, and technically speaking is outside of the NP itself. This combination of interrogative pronoun and additive clitic yields a universally quantified NP
\footnote{This construction never carries the meaning of \em whichever\em. For this meaning, the WH\~{}WH ... =\em so \em construction is used \formref{sec:nppp:NPscontaininginterrogativepronounsusedforuniversalquantification}.}

\cb{WH N(=POSTP)=\textit{le}}

The use in an NP is given in \xref{ex:constr:NP:interr:le:NP}, the use of a locative PP in \xref{ex:constr:NP:interr:le:PP}.

\xbox{16}{
\ea\label{ex:constr:NP:interr:le:NP}
\gll Skarang \textbf{maana} aari\textbf{=le} atthu atthu oorang=yang arà-buunung. \\
     now which day=\textsc{addit} one one man=\textsc{acc} \textsc{past}-kill   \\
    `Now people kill each other every day.'  (K051206nar11)
\z
}

\xbox{16}{
\ea\label{ex:constr:NP:interr:le:PP}
\gll \textbf{Maana} thaahun=ka=\textbf{le} ini rooja pohong komplok duuva kumbang=dering arà-punnu. \\
      which year=\textsc{loc}=\textsc{addit} \textsc{prox} rose tree bush two flower=\textsc{abl} \textsc{non.past}-push \\
    `Every year, these two rose bushes grow flowers.'  (K070000wrt04)
\z
}


% \xbox{16}{
% \ea\label{ex:constr:UQ:maana3}
% \gll Dee maana aari=le      asà-dhaathang, thingaari vakthu=nang   kalthraa maalang vakthu=nang. \\
%      3 which day=\textsc{addit} \textsc{cp}-come noon time=\textsc{dat} otherwise night time=\textsc{dat}  \\
%     `He came every day, at noon or otherwise during the night and (attacked).' (K051206nar02)
% \z
% } \\




%  \xbox{16}{
% \ea\label{ex:constr:UQ:maana5}
%    \gll Karang Kluu\u mbu {\em life} katha maana siyang=le       busy. \\
%     now    Colombo life \textsc{quot}  which morning=\textsc{addit} busy \\
% `Now life in Colombo is  busy all the mornings' (B060115cvs01)
% \z
% }


If the additive clitic attaches to a predicate rather than to the NP/PP, then the universal quantification only holds for the subset of entities which happen to have a positive truth value in the proposition formed by the predicate and its arguments. An example for this is \xref{ex:constr:NP:interr:le:pred:intro}. In this example, the proposed connection does not hold for any Malay. It only holds for the Malays which the addressee happens to meet.

\xbox{16}{
\ea\label{ex:constr:NP:interr:le:pred:intro}
\gll Lai     \textbf{saapa} mlaayu kuthumung=\textbf{le} aapa=ke      {\em connection} hatthu aada. \\
     other who Malay see=\textsc{addit} what=\textsc{simil} connection \textsc{indef} exist \\
    `Any other Malay you see, there will always be some kind of connection.' (K051206nar07,K081105eli02)
\z
}

Another example of this structure is \xref{ex:constr:NP:interr:le:pred:dhraapa}.

\xbox{16}{
\ea\label{ex:constr:NP:interr:le:pred:dhraapa}
   \gll Daalang=ka  pii=nang blaakang, \textbf{dhraapa}   puukul=\textbf{le}   thama-kuthumung. \\
     inside=\textsc{loc} go=\textsc{dat} after how.many hit=\textsc{addit} \textsc{neg.nonpast}-see \\
`After going inside, how ever much he hit them, one would not see.' (K051206nar02,K081105eli02)
\z
}

This structure can be formalized as follows, where reference of N$_i$ is restricted by the predicate V$_i$ before application of the universal quantifier.

\cb{$\left[WH~N_i(=POSTP)~\NP*~V_i=\textit{le}\right]~PRED$}

It is possible to use other interrogative pronouns in this construction, as given below.


%     saapa kuthumugle baae
%         whoever they see, they are good

%     New york ka saapayang kuthumungle, derampada kaaya

\xbox{14}{
\ea
\gll \textbf{Aapa} bìlli=\textbf{le}, baae. \\
      what buy=\textsc{addit} good \\
    `Anything you buy is good.' (K081104eli05) 
\z
}


% \xbox{14}{
% \ea
% \gll Mana thumpath pii=le panthas. \\
%        \\
%     `wherever you go, it's beautiful.' (nosource)
% \z
% } \\

\xbox{14}{
\ea
\gll \textbf{Mana}=ka=\textbf{le} pii, panthas. \\
     where=\textsc{loc}=\textsc{addit} go beautiful  \\
    `Anywhere you go, it's beautiful.' (K081104eli05)
\z
}


While the construction with the additive clitic \em =le \em has a maximizing function, it is also possible to use the `undetermined' clitic \em, =so\em, in which case the meaning is more that of an arbitrary referent than of universal quantification. While \em =le \em is used with a bare verb, a finite verb must be used with the \em =so\em-construction.


\xbox{14}{
\ea
\gll \textbf{Saapa} arà-nyaanyi=\textbf{so},  bannyak upaama.\footnotemark{}  \\
     who \textsc{non.past}-sing=\textsc{undet} much honour   \\
    `Whoever sings, it will be a big honour.' (K081104eli05)
\z
}

\footnotetext{The word for `honour' was given as both \em upaama \em and \em uthaama\em.}


The interrogative pronoun can be reduplicated in the \em =so\em-construction, which is not possible in the \em =le\em-construction. This yields an exhaustive meaning, analogous to the normal reduplicated question words \formref{sec:wc:Interrogativepronouns}.


\xbox{14}{
\ea
\gll \textbf{Saapa\~{}saapa} arà-nyaanyi=\textbf{so}, bannyak upaama. \\
     who\~{}red \textsc{non.past}-sing=\textsc{undet} much honour  \\
    `Whoever all sing, it will be a big honour.' (K081104eli05)
\z
}


%     saapa nyaanyile, incayang nang bannyak upaama
%     *saapa saape ara nyaanyi le

Different TAM-prefixes can be used in this construction, as shown in \xref{ex:constr:NP:interr:so:redup:tam:su} and \xref{ex:constr:NP:interr:so:redup:tam:anthi}.


\xbox{14}{
\ea\label{ex:constr:NP:interr:so:redup:tam:su}
\gll Aapa\~{}aapa \textbf{su}-bìlli=so, itthu baae. \\
     what\~{}red \textsc{past}-buy=\textsc{undet} \textsc{dist} good  \\
    `Whatever you have bought, it is good.' (K081104eli05)
\z
}


\xbox{14}{
\ea\label{ex:constr:NP:interr:so:redup:tam:anthi}
\gll Aapa\~{}aapa \textbf{anthi}-bìlli=so, itthu baae. \\
     what\~{}red \textsc{irr}-buy=\textsc{undet} \textsc{dist} good  \\
    `Whatever you will buy, it will be good.' (K081104eli05)
\z
}

The formalization of this is given in \xref{cb:constr:NP:WHWHso}.

\cb[\label{cb:constr:NP:WHWHso}]{$\left[WH(\sim\textsc{red})~N_i(=POSTP)~\NP*~TAM-V_i=\textit{so}\right]~PRED$}



%     incayang aapa aapa anthi bìlli so (thàrà-thaau)
%         I don't know, what and  what he will buy


The only naturalistic example of this construction is   \xref{ex:constr:NP:unknownargumentclause}.


\xbox{16}{
\ea\label{ex:constr:NP:unknownargumentclause}
\gll Inni     \textbf{saapa}\Tilde\textbf{saapa}=ka inni  mlaayu pakeyan pada aada\textbf{=so}, lorang  pada ini       mlaayu  pakeyan=samma ini       kaving=nang mà-dhaathan    bannyak uthaama.\footnotemark{}  \\
prox who\Tilde who=\textsc{loc} \textsc{prox} Malay dress \textsc{pl} exist=\textsc{disj} \textsc{2pl} \textsc{pl} \textsc{prox} Malay dress=with \textsc{prox} wedding=\textsc{dat} \textsc{inf}-come much honour\\
`Whoever owns such  Malay dresses, your coming together with this Malay dress to the wedding will be greatly appreciated.' (K060116nar04)
\z
}

\footnotetext{The word for `honour' was given as both \em upaama \em and \em uthaama\em.}



The examples above have shown the use of \em =le \em on an item other than the question word. It is also possible to attach \em =le \em directly to the question word if the interrogative pronoun provides some quantifiable semantic content, as is the case for  \trs{saapa}{who} in the following two example.

\cb{WH(=POSTP)=le}




\xbox{16}{
\ea\label{ex:constr:UQ:saapanang}
\gll Cinggala  bahasa   \textbf{saapa=nang=le}        bole=bicaara      siini. \\
     Sinhala language who=\textsc{dat}=\textsc{addit} can-speak here  \\
    `Sinhala could be spoken by anybody around here.' (K051206nar14,K081105eli02) 
\z
}




% \xbox{16}{
% \ea\label{ex:constr:UQ:aapanang}
%    \gll {\em excitement} aapa=nang=le  lai hathu  kavulan thaaro. \\
%   excitement what=\textsc{dat}=\textsc{addit} other \textsc{indef} vow put \\
% `For every excitement you can make another vow' (K051220nar01)(test)
% \z
% }









\subsection{Relative order of postnominal modifiers}\label{sec:nppp:Relativeorderofpostnominalmodifiers}
We have seen above that  the indefinite article, adjectives, numerals, and the plural marker can follow the noun. Relatively little can be said about the order of these items. The plural marker is always at the rightmost positions, and can cooccur with any of the other postnominal modifiers but the indefinite article.  It is not possible to have more than one of the other modifiers in postnominal position. We can schematize the postnominal field  as in \xref{cb:np:postn}. Postnominal adjectives and nouns are listed in this schema for illustrative reasons as well, but are actually analyzed as being compounds in this grammar  \formref{sec:wofo:Compounding}. The plural marker cannot cooccur with the indefiniteness marker. This is not represented in the schema.

\cb[\label{cb:np:postn}]{Noun $\left\{\begin{array}{c} (N) \\(ADJ) \\INDEF\\QUANT \\ NUM) \end{array}\right\}$ PL}



\subsection{Relative order in the prenominal field}\label{sec:nppp:Relativeorderintheprenominalfield}

% B060115nar04.txt:  asabaayar   samma anabìssar    thumpath

There are many more prenominal modifiers than postnominal modifiers. The possible order between them are indicated in Table \ref{tab:OrderOfPrenominalModifiers}. 
The following sections give the logical possibilities and also give an example, if available. Sections without an example indicate that this combination was not found.



\begin{table}
	\centering
 \begin{tabular}{rccccccc}
~
 &
REL &
DEIC &
POSS &
QUANT &
NUM &
ADJ &
N\\ 
REL &
? &
+ &
+ &
+ &
+ &
+ &
+\\
DEIC &
+ &
/ &
+ &
+ &
+ &
+ &
+\\
POSS &
(+) &
+ &
/ &
+ &
+ &
+ &
+\\
QUANT &
-- &
-- &
+ &
/ &
/ &
+ &
+\\
NUM &
-- &
-- &
-- &
/ &
(--) &
+ &
+\\
ADJ &
-- &
-- &
-- &
-- &
(+) &
+ &
+\\
N &
-- &
-- &
-- &
-- &
-- &
-- &
+\\
\end{tabular} 
	\caption[Order of prenominal modifiers]{Order of prenominal modifiers. The cells indicate whether the item at the left end of the row can precede the item at the top of the column. A + indicates that this is possible, a -- indicates that it is impossible, a / indicates that the combination is ruled out semantically, e.g. combination of a numeral and a quantifier.}
	\label{tab:OrderOfPrenominalModifiers}
\end{table} 


\paragraph{RELC RELC N}
It is very difficult to find contexts where a head noun could be modified by two clauses of the Relative Clause type, which comprises both true relative clauses and fact clauses in SLM (see \formref{sec:cls:Relativeclause} for a discussion). Even if this occurs, normally, the two clauses are coordinated and put into \em one \em relative clause, itself consisting of two lower clauses. Example \xref{ex:constr:NP:prenom:relcrelc} shows such a case. The two verbs \trs{mintha}{beg} and \trs{naangis}{cry} both depend on the head noun \trs{svaara}{sound}. But they are found in \em one \em clause.

\xbox{16}{
\ea\label{ex:constr:NP:prenom:relcrelc}
\gll [[Banthu-an asà-mintha]$_{CLS}$  [arà-naangis]$_{CLS}$]$_{RELC}$ [svaara]$_{N}$ hatthu derang=nang su-dìnngar. \\ % bf
      help-\textsc{nmlzr} \textsc{cp}-beg \textsc{simult}-cry sound \textsc{indef} \textsc{3pl}=\textsc{dat} \textsc{past}-hear\\
    `They heard a sound of crying and begging for help.'  (K070000wrt04)
\z
}

To sum up, there is no case of two clauses preceding a head noun as far as syntax is concerned. Semantically speaking, it is possible to have two propositions depending on a head noun, as in \xref{ex:constr:NP:prenom:relcrelc}, but these are packaged into one clause in syntax before they attach to the head noun. The structure is then
[
% 	[
		[
			[...]$_{infinite clause}$ ...
		]$_{finite relative clause}$
% 	]$_{relative clause}$...
]$_{main clause}$


\paragraph{RELC DEIC N}
This order is found in two examples in the corpus.

\xbox{16}{
\ea\label{ex:constr:NP:prenom:relcdeic1}
\gll [Seelong=ka   su-mnii\u n\u ggal]$_{RELC}$ [ithu]$_{DEIC}$ thiiga [oorang]$_N$=pe naama kithang thàrà-thaau. \\ % bf
      Ceylon=\textsc{loc} \textsc{past}-die \textsc{dist} three man=\textsc{poss} name \textsc{1pl} \textsc{neg}-know \\
    `We do not know the names of the three people who died in Ceylon.' (K060108nar02)
\z
}

\xbox{14}{
\ea\label{ex:constr:NP:prenom:relcdeic2}
\gll Laiskali,  inni Aajuth  [incayang=pe jiiva anà-salba-king]$_{RELC}$ [itthu]$_{DEIC}$ [aanak]$_N$ pompang duuva=]nang thriima thàrà-kaasi. \\
      again \textsc{dem.prox} dwarf \textsc{3s.polite}=\textsc{poss} life \textsc{past}-escape-\textsc{caus} \textsc{dem.dist} child female two=\textsc{dat} thanks \textsc{neg.past}-give \\
    `Once again, the dwarf did not thank the two girls who had saved his life.' (K070000wrt04)
\z
}


\paragraph{RELC POSS N}
There is one instance of a possessive pronoun intervening between the relative clause and the head noun, \xref{ex:constr:NP:prenom:relcposs}.

\xbox{16}{
\ea\label{ex:constr:NP:prenom:relcposs}
\gll [Ini      {\em British} government=samma pii=apa     mà-oomong]$_{RELC}$ [kithang=pe]$_{POSS}$     [{\em statesmen}]$_{N}$  pada \el{}\\ % bf
      \textsc{prox} British government=\textsc{comit} go=after \textsc{inf}-talk \textsc{1pl}=\textsc{poss} statesmen \textsc{pl} \el{} \\
    `Our statesmen who had gone to negotiate with the British Government ...' (N061031nar01)
\z
}

\paragraph{RELC QUANT N}
There is one instance of this order.

\xbox{16}{
\ea
\gll [Anà-libbi]$_{RELC}$  [konnyong]$_{QUANT}$  mlaayu$_{N}$=jo   Seelong=ka  (asà-)thii\u n\u ggal aada. \\
      \textsc{past}-remain few Malay=\textsc{emph} Ceylon=\textsc{loc} \textsc{cp}-settle exist \\
    `The remaining few settled in Sri Lanka.' (K051222nar06,K081105eli01)
\z
}

\paragraph{RELC NUM N}
The following two examples give this order with a cardinal \xref{ex:constr:NP:prenom:RELCNUM:card} and an ordinal numeral \xref{ex:constr:NP:prenom:RELCNUM:ord}.

\xbox{16}{
\ea\label{ex:constr:NP:prenom:RELCNUM:card}
\gll [Seelong=ka   su-mnii\u n\u ggal]$_{RELC}$ ithu [thiiga]$_{NUM}$ [oorang]$_N$=pe naama kithang thàrà-thaau. \\ % bf
      Ceylon=\textsc{loc} \textsc{past}-die \textsc{dist} three man=\textsc{poss} name \textsc{1pl} \textsc{neg}-know \\
    `We do not know the names of the three people who died in Ceylon.' (K060108nar02)
\z
}

\xbox{16}{
\ea\label{ex:constr:NP:prenom:RELCNUM:ord}
\gll [{\em Terrorist} hatthu=dering  anà-maathi]$_{RELC}$ [kàthaama]$_{NUM}$ oorang$_{N}$=jo    incayang. \\ % bf
      terrorist \textsc{indef}=\textsc{abl} \textsc{past}-die first man=\textsc{emph} \textsc{3s} \\
    `The first man killed by a terrorist was him.' (K051206nar02)
\z
}

\paragraph{RELC ADJ N}
This order is common. There are several instances of it in the corpus. A simple example is \xref{ex:constr:NP:prenom:RELCADJ1}, where \trs{bìssar}{big} is found between the relative clause and the head noun.

\xbox{16}{
\ea\label{ex:constr:NP:prenom:RELCADJ1}
\gll [Anà-kijja]$_{RELC}$ [bìssar]$_{ADJ}$  [thumpath]$_{N}$ pada. \\ % bf
 \textsc{past}-make big place \textsc{pl}\\
`The big lands that were made.' (N060113nar02)
\z
}

\em Bìssar \em is also found as the intervening adjective in \xref{ex:constr:NP:prenom:RELCADJ2}.

\xbox{16}{
\ea\label{ex:constr:NP:prenom:RELCADJ2}
\gll [Sithu=ka     aada]$_{RELC}$  [bìssar]$_{ADJ}$ [oorang]$_{N}$ pada=yang   asà-{\em attack}-kang ... \\ % bf
     there=\textsc{loc} exist big man \textsc{pl}=\textsc{acc} \textsc{cp}-attack-\textsc{caus}    \\
    `He attacks the leaders who are there and  ... ' (K051206nar02)
\z
}

The head of a relative clause does not have to be a noun actually, an interrogative pronoun is possible as well, as shown in \xref{ex:constr:NP:prenom:RELCADJ4}, where additionally the adjective \trs{lai}{other} intervenes between the relative clause and the interrogative pronoun.

\xbox{16}{
\ea\label{ex:constr:NP:prenom:RELCADJ4}
\gll Incalla   [lai     thaau sudaara sudaari pada]=ka    bole=caanya    ambel [[nya-gijja]$_{RELC}$    [lai]$_{ADJ}$     [saapa=kee]$_{HEAD}$  aada=si    katha]. \\ % bf
      Hopefully other know brother sister \textsc{pl}=\textsc{loc} can-ask take \textsc{past}-make other who=\textsc{simil} exist=\textsc{interr} \textsc{quot} \\
    `Hopefully, you can enquire from another person you know whether there is someone else who did something.' (N061031nar01)
\z
}

A somewhat different case is given in \xref{ex:constr:NP:prenom:RELCADJ:purp}, where we are dealing with a purposive clause, which precedes the adjective.

\xbox{16}{
\ea\label{ex:constr:NP:prenom:RELCADJ:purp}
\gll Kithang lorang=nang baaye mliiga athi-kaasi, [mà-kaaving]$_{RELC}$ [panthas]$_{ADJ}$ [pompang]$_{N}$ pada athi-kaasi,  duvith athi-kaasi. \\ % bf
      \textsc{1pl} \textsc{2pl}=\textsc{dat} good palace \textsc{irr}-give \textsc{inf}-marry beautiful female \textsc{pl} \textsc{irr}-give money \textsc{irr}-give \\
    `We will give you nice palaces, we will give you beautiful girls to marry, we will give you money.' (K051213nar06)
\z
}


% \xbox{16}{
% \ea\label{ex:constr:NP:prenom:RELCADJ}
% \gll Laayeng [kithang=nang   aada]$_{RELC}$  [laayeng]$_{ADJ}$  [makanan]$_{N}$        pada saathe. \\
%       other \textsc{1pl}=\textsc{dat} exist other food \textsc{pl} sate \\
%     `Another dish we have is sate.' (K061026rcp03)
% \z
% } \\


% \xbox{16}{
% \ea\label{ex:constr:NP:prenom:RELCADJ}
% \gll Ini [kuurang arà-duuduk]$_{RELC}$     [laayeng]$_{ADJ}$  [kumpulan]$_{N}$   pada=yang   mà-{\em represent}-kang=nang. \\
%       \textsc{prox} few \textsc{non.past}-exist.\textsc{anim} other group \textsc{pl}=\textsc{acc} \textsc{inf}-represent-\textsc{caus}=\textsc{dat} \\
%     `To represent the other groups with few people staying in them (i.e. minorities).' (N061031nar01)
% \z
% } \\



\paragraph{RELC N N}
A relative clause can precede a sequence of nouns.

\xbox{16}{
\ea\label{ex:constr:NP:prenom:RELCN}
\gll [Ruuma duuva subala=ka   su-aada]$_{RELC}$ [rooja]$_{N}$ [pohong  komplok]$_{N}$ duuva \\
      house two side=\textsc{loc} \textsc{past}-exist rose tree bush two \\
    `The rose bushes that stood at both sides of the house' (K070000wrt04)
\z
}

\paragraph{DEIC RELC      N}
Both the proximal \xref{ex:constr:NP:prenom:deicrelc:ini} and the distal deictic \xref{ex:constr:NP:prenom:deicrelc:ithu} have been found to precede a relative clause.

 \xbox{16}{
\ea\label{ex:constr:NP:prenom:deicrelc:ini}
   \gll Incayang  [ini]$_{DEIC}$    [Seelong=ka  anà-aada]$_{RELC}$    [lakuan]$_{N}$   [baathu]$_{N}$ asà-caari. \\ % bf
    \textsc{3s.polite} \textsc{prox} Seelon=\textsc{loc} \textsc{past}-exist wealth stone \textsc{cp}-find \\
`He was looking for the gems which were present in Ceylon.' (K060103nar01)
\z
}

 \xbox{16}{
\ea\label{ex:constr:NP:prenom:deicrelc:ithu}
   \gll [Ithu]$_{DEIC}$     [spaaman anà-nii\u n\u ggal]$_{RELC}$   [thumpath]$_{N}$=nang=le        [Passara   katha  arà-biilang    nigiri]=nang=le   dìkkath. \\ % bf
    \textsc{dist} \textsc{3s.polite} \textsc{past}-die place=\textsc{dat}=\textsc{addit} Passara \textsc{quot} \textsc{non.past}-say village=\textsc{dat}=\textsc{addit} vicinity\\
`The place where he died and the village called Passara are close to each other.' (B060115nar05)
\z
}

\paragraph{DEIC DEIC N}
This order was not found, which is probably due to the opposite meanings of `proximal' and `distal', which do not combine well.

\paragraph{DEIC POSS N}
The deictics can be found preceding a possessor and a noun.

\xbox{16}{
\ea\label{ex:constr:NP:prenom:DEICPOSSN1}
\gll [Ini]$_{DEIC}$ [indonesia=pe]$_{POSS}$ [oorang]$_{N}$=si giithu kalthraa {\em Malaysian} oorang=si. \\ % bf
     \textsc{prox} Indonesia=\textsc{poss} man=\textsc{disj} that.way if.not Malaysian man\textsc{disj}  \\
    `These Indonesians or otherwise Malaysians.'  (K060108nar02)
\z
}


\xbox{16}{
\ea\label{ex:constr:NP:prenom:DEICPOSSN2}
\gll {\em Eleventh}=ka su-aada [itthu]$_{DEIC}$ [kithang=pe igaama=pe]$_{POSS}$       [mosthor]$_{N}$=nang. \\ % bf
     eleventh=\textsc{loc} \textsc{past}-exist \textsc{dist} \textsc{1pl}=\textsc{poss} religion=\textsc{poss} manner=\textsc{dat}  \\
    `It was on the 11$^{th}$, according to the customs of our religion.' (B060115cvs01)
\z
}

\paragraph{DEIC QUANT N}
Deictics and quantifiers precede the noun, in that order.

\xbox{16}{
\ea\label{ex:constr:NP:prenom:DEICQUANTN}
\gll {\em School}=nang arà-pii=subbath=jo [inni]$_{DEIC}$ [samma]$_{QUANT}$ [seksa]$_{N}$. \\ % bf
    school=\textsc{dat} \textsc{non.past}-go=because=\textsc{emph} \textsc{prox} all problem   \\
    `It is because we go to school that there are all these problems.'  (B060115cvs01)
\z
}

\paragraph{DEIC NUM N}
Deictics and numerals precede the noun, in that order.

\xbox{16}{
\ea\label{ex:constr:NP:prenom:DEICNUMN}
\gll [Ini]$_{DEIC}$ [duuva]$_{NUM}$ [{\em army} {\em captain}]$_{N}$ thàrà-maathi. \\ % bf
       \textsc{prox} two army captain \textsc{neg.past}-dead\\
    `These two army captains did not die.' (K051213nar06)
\z
}

\paragraph{DEIC ADJ N}
Deictics precede the adjective in the noun phrase.

\xbox{16}{
\ea\label{ex:constr:NP:prenom:DEICADJN}
\gll [Ini]$_{DEIC}$ [laama]$_{adj}$ [{\em car}]$_{N}$ pada=jo                   kithang       arà-baapi. \\ % bf
      \textsc{prox} old               car             \textsc{pl}=\textsc{emph} \textsc{1pl} \textsc{non.past}-bring \\
    `It is these old cars we take [to Iraq].'  (K051206nar19)
\z
} 

% \xbox{16}{
% \ea\label{ex:constr:NP:unreferenced}
% \gll Derang anà-cuuri  baae      [nni]$_{DEIC}$       [kaaya]$_{ADJ}$ oorang$_{N}$ pada=dering. \\
%     \textsc{3pl} \textsc{past}-steal    good     \textsc{prox} rich man \textsc{pl}=\textsc{abl}  \\
%    `they stole for good from these rich people.'  (K051206nar02)
% \z
% }\\


% \xbox{16}{
% \ea\label{ex:constr:NP:unreferenced}
% \gll Itthu    baapa=le      ithukang       ithu    kiccil aanak=le     asà-baa. \\
%      \textsc{dist} father=\textsc{addit} then \textsc{dist} small child=\textsc{addit} \textsc{cp}-bring  \\
%     `That father then also brought that small child.' (B060115nar02)
% \z
% } \\


\paragraph{DEIC N N}
The following two examples show deictics preceding two nouns. If these strings of two nouns are analyzed as syntactic combinations, we get the sequence DEIC N N, i.e. a deictic preceding a noun modifying a noun.


\xbox{16}{
\ea\label{ex:constr:NP:prenom:DEICN:righthead:1}
\gll Kitham=pe=le   [inni]$_{DEIC}$   [prompang]$_{N}$ [kumpulan]$_{N}$    bannyak samma=ka    arà-banthu. \\ % bf
      \textsc{1pl}=\textsc{poss}=\textsc{addit} \textsc{prox} woman association much every=\textsc{loc} \textsc{non.past}-help \\
    `This women's association of ours has also helped a lot with everything.'  (B060115cvs01)
\z
}


\xbox{16}{
\ea\label{ex:constr:NP:prenom:DEICN:righthead:2}
\gll Aapcara=ke       incayang  [ini]$_{DEIC}$ [ciina]$_{N}$ [oorang]$_{N}$ Islam=nang   asà-dhaathang ... \\ % bf
     how=\textsc{simil} \textsc{3s.polite} \textsc{prox} China man Islam=\textsc{dat} \textsc{cp}-come ...\\
    `Somehow he, this Chinaman converted to Islam and ...' (K051220nar01)
\z
}

\paragraph{POSS RELC    N}
Very short relative clauses can be preceded by a possessor, as in \xref{ex:constr:NP:prenom:POSSRELC}, where the possessor \trs{se=ppe}{my} immediately precedes the relative clause \trs{nya-laaher}{being born}. Note that this example has a lot of code-mixing with English in it, so that its value could be contested.

\xbox{16}{
\ea\label{ex:constr:NP:prenom:POSSRELC}
\gll [Se=ppe]$_{POSS}$ [nya-laaher]$_{RELC}$ [{\em date}]$_{N}$ duuva duuva {\em 1960}. \\ % bf
     \textsc{1s}=\textsc{poss} \textsc{past}-be.born date two two 1960  \\
    `My birthday is 2-2-1960.' (K061019prs01)
\z
}

% \xbox{16}{
% \ea\label{ex:constr:NP:unreferenced}
% \gll [Itthu=pe]$_{POSS}$        [nya-puunya]$_{RELC}$ \zero$_{HEAD}$. \\
%      \textsc{dist}=\textsc{poss} \textsc{past}-own  \\
%     `The owner of this.' (K051213nar04)(test)
% \z
% } \\
%

\paragraph{POSS DEIC N}
Normally, the location of possessed items is retrievable for the hearer, and the deictics \em ini \em and \em itthu \em do not combine with possessed items then. In the following two examples, the deictics are used for emphasis.

\xbox{16}{
\ea\label{ex:constr:NP:prenom:POSSDEICN1}
\gll Kandi=pe     raaja=nang   [kitham=pe]$_{POSS}$ [inni]$_{DEIC}$     [banthu-an]$_{N}$  asà-kamauvan      se-aada. \\ % bf
     Kandy=\textsc{poss} king=\textsc{dat} \textsc{1pl}=\textsc{poss} \textsc{prox} help-\textsc{nmlzr} \textsc{cp}-want \textsc{past}-exist  \\
    `The Kandyan king had needed this help of ours.' (K060108nar02)
\z
}

\xbox{16}{
\ea\label{ex:constr:NP:prenom:POSSDEICN2}
   \gll [Kithang=pe]$_{POSS}$  [ini]$_{DEIC}$      [pompang kumpulan]$_{N}$      bannyak samma=nang   arà-banthu. \\ % bf
    \textsc{1pl}=\textsc{poss} \textsc{prox} female association much all=\textsc{dat} \textsc{non.past}-help \\
`This women's association of ours helps a lot with everything' (B060115cvs01,K081105eli02)
\z
}

In examples \xref{ex:constr:NP:prenom:POSSDEICN3}, the possessor is used to recall that the Malays being talked about are actually forefathers of the speaker.

\xbox{16}{
\ea\label{ex:constr:NP:prenom:POSSDEICN3}
\gll Indonesia=dering Sri Lanka=nang [kithang=pe]$_{POSS}$ [ini]$_{DEIC}$ [mlaayu]$_{N}$ pada asà-dhaathang ini {\em Malay} {\em regiment} atthu. \\ % bf
      Indonesia=\textsc{abl} Sri Lanka=\textsc{dat} \textsc{1pl}=\textsc{poss} \textsc{prox} Malay \textsc{pl} \textsc{cp}-come \textsc{prox} Malay regiment \textsc{indef} \\
    `Our Malays came from Indonesia to Sri Lanka in this Malay regiment.' (G051222nar03,K081105eli02)
\z
}

\paragraph{POSS POSS N}
There is strictly speaking no modification of a noun by two possessors, since an item normally has only one possessor. What is possible is recursive possession, but this is not a modification of one head noun by several possessors, but a modification of the head noun on level $n$-1 by the possessor on level $n$ \xref{ex:constr:NP:pepe1}\xref{ex:constr:NP:pepe2}.

\xbox{16}{
\ea\label{ex:constr:NP:pepe1}
\gll [Se=ppe$_{POSS}$      dhaatha]$_{N}$=pe]$_{POSS}$          thiiga [aanak]$_{N}$=le      Dubai=ka     arà-duuduk. \\ % bf
 \textsc{1s}=\textsc{poss} elder.sister=\textsc{poss} three child=\textsc{addit} Dubai=\textsc{loc} \textsc{non.past}=stay\\
`My elder sister's three children also live in Dubai.' (B060115prs21)
\z
}

\xbox{16}{
\ea\label{ex:constr:NP:pepe2}
\gll [[Se=ppe]$_{POSS}$ bìssar aade=pe]$_{POSS}$  [manthu]$_{N}$. \\ % bf
      \textsc{1s}=\textsc{poss} big younger.sibling=\textsc{poss} child.in.law \\
    `My elder younger brother's son-in-law.'  (K060116nar02 )
\z
}



\paragraph{POSS QUANT N}

This order of prenominal modifiers is possible and is given in the following three examples.

\xbox{16}{
\ea\label{ex:constr:NP:prenom:POSSQUANTN1}
\gll Itthu blaakang [kithang=pe]$_{POSS}$     [bannyak]$_{QUANT}$ [sudaari-sudaara]$_{N}$ pada,  kithang anà-pii    ruma  saakith=nang. \\ % bf
     \textsc{dist} after \textsc{1pl}=\textsc{poss} much sister-brother \textsc{pl} \textsc{1pl} \textsc{past}-go house sick=\textsc{dat}  \\
    `After that, many of our brothers and sisters, we all went to the hospital.' (B060115nar02)
\z
}

\xbox{16}{
\ea\label{ex:constr:NP:prenom:POSSQUANTN2}
\gll [Ini kaaving=nang aada] haadath-saadath pada, [kitham=pe]$_{POSS}$ [bannya]$_{QUANT}$ [ooram]$_{N}$ pada arà-kijja. \\ % bf
     \textsc{prox} wedding=\textsc{dat} exist   traditions \textsc{pl} 1p=\textsc{poss} many people \textsc{pl} \textsc{non.past}-make\\
    `The traditions that were at this wedding, many of our people follow them.'  (K061122nar01,K081105eli02)
\z
}



\xbox{16}{
\ea\label{ex:constr:NP:prenom:POSSQUANTN3}
   \gll Derang=nang   [Kluu\u mbu=pe]$_{POSS}$    [samma]$_{QUANT}$ [{\em association}]$_{N}$=le      {\em support}. \\ % bf
    \textsc{3pl}=\textsc{dat} Colombo=\textsc{poss} all association=\textsc{addit} support \\
`All Colombo associations supported them.' (K060116nar06,K081105eli02)
\z
}

\paragraph{POSS NUM N}
This order is given in the following two examples.

\xbox{16}{
\ea\label{ex:constr:NP:prenom:POSSNUMN1}
\gll [Mliige=pe]$_{POSS}$     [duuva]$_{NUM}$ [subla]$_{N}$=ka su-thaanàm. \\ % bf
       palace=\textsc{poss} two side=\textsc{loc} \textsc{past}-plant \\
    `They planted them on both sides of the palace.' (K070000wrt04)
\z
}

\xbox{16}{
\ea\label{ex:constr:NP:prenom:POSSNUMN2}
\gll Se=ppe   [dhaatha=pe]$_{POSS}$  [thiiga]$_{NUM}$ [aanak]$_{N}$=le  Dubai=ka     arà-duuduk. \\ % bf
 \textsc{1s}=\textsc{poss} elder.sister=\textsc{poss} three child=\textsc{addit} Dubai=\textsc{loc} \textsc{non.past}=stay\\
`My elder sister's three children also live in Dubai.' (B060115prs21)
\z
}

\paragraph{POSS ADJ N}
The possessor can precede the adjective preceding a noun.

\xbox{16}{
\ea\label{ex:constr:NP:prenom:POSSADJN1}
\gll [Se=ppe]$_{POSS}$ [bìssar]$_{ADJ}$ [aade]$_{N}$=pe  manthu. \\ % bf
      \textsc{1s}=\textsc{poss} big younger.sibling=\textsc{poss} child.in.law \\
    `My elder younger brother's son-in-law.'  (K060116nar02 )
\z
}

\xbox{16}{
\ea\label{ex:constr:NP:prenom:POSSADJN2}
\gll Itthu=le [oorang mlaayu=pe]$_{POSS}$ [baaye]$_{ADJ}$ hathu [{\em traditional} {\em food}]$_{N}$ hatthu. \\ % bf
      \textsc{dist}=\textsc{addit} man Malay=\textsc{poss} good \textsc{indef} tradtional food \textsc{indef} \\
    `That is also one of the Malays' good traditional food.' (K061026rcp04)
\z
}

\paragraph{POSS N N}
A possessor can precede a sequence of two nouns.

\xbox{16}{
\ea\label{ex:constr:NP:prenom:POSSNN}
\gll [Se=ppe]$_{POSS}$ [aanak]$_{N}$ [pompang]$_{N}$=nang duuva aanak klaaki. \\ % bf
     \textsc{1s}=\textsc{poss} child female=\textsc{dat} two child male  \\
    `My daughter has two sons.'  (K051201nar01)
\z
}

\paragraph{QUANT RELC N}
This order was not found.

\paragraph{QUANT DEIC N}
This order was not found.

\paragraph{QUANT POSS N}
The noun can be preceded by a quantifier and a possessor in that order.

\xbox{16}{
\ea\label{ex:constr:NP:prenom:quantposs}
   \gll [Samma]$_{QUANT}$ [kithang=pe]$_{POSS}$     [mlaayu]$_{N}$, hathu  muusing su-aada,      samma cinggala=dering=jo      athi-oomong. \\
    all   \textsc{1pl}=\textsc{poss} Malay  \textsc{indef}  time    \textsc{past}-exist all   Sinhalese=\textsc{abl}=\textsc{emph} \textsc{irr}-talk \\
`There once was a time when all Malays would speak in Sinhala ([n our homes, but we have taken measures against it].' (B060115cvs01)
\z
}

\paragraph{QUANT QUANT N}
This order was not found.

\paragraph{QUANT NUM N}
This order was not found.

\paragraph{QUANT ADJ N}

The quantifier precedes the adjective, as shown in the following example.


\xbox{16}{
\ea\label{ex:constr:NP:prenom:QUANTADJN}
\gll [Spaaru]$_{QUANT}$ [bìssar]$_{ADJ}$ [ruuma]$_{N}$ pada aada. \\
      some big house \textsc{pl} exist \\
    `There are some big houses.'  (K081105eli02)  
\z
}




\paragraph{QUANT N N}
A quantifier can precede a string of two nouns, as shown in the following example.


\xbox{16}{
\ea\label{ex:constr:NP:prenom:quantn}
\gll Derang derang=pe  umma=nang butthul saayang=kee=jo [samma]$_{QUANT}$ [ruuma]$_{N}$ [pukurjan]$_{N}$=nang=le anà-banthu. \\ % bf
 \textsc{3pl} \textsc{3pl}=\textsc{poss} mother=\textsc{dat} correct love=\textsc{simil}=\textsc{emph} all house work=\textsc{dat}=\textsc{addit} \textsc{past}-help  \\
    `They also helped their mother with all the housework.'  (K070000wrt04)
\z
}


%
% \xbox{16}{
% \ea\label{ex:constr:NP:prenom:quantn}
% \gll Bannyak$_{QUANT}$ Muslim$_{N}$ oorang$_{N}$ pada araduuduk. \\
%      many Muslim man \textsc{pl} \textsc{non.past}-exist.\textsc{anim}  \\
%     `There are many Muslims (in the Middle East.' (K061026prs01)
% \z
% } \\

\paragraph{NUM   RELC  N}
This order was not found.
\paragraph{NUM   DEIC N}
This order was not found.
\paragraph{NUM   POSS N}
This order was not found.
\paragraph{NUM   QUANT N}
This order was not found.
\paragraph{NUM   NUM N}

A head noun can be modified by two adjacent numerals, which conveys uncertainty about the exact amount.
In the following example, the speaker is not sure whether his stay in the Navy lasted twelve or thirteen years. The two numeral \trs{doblas}{twelve} and \trs{thigablas}{thirteen} jointly modify the noun.

\xbox{16}{
\ea\label{ex:constr:NP:prenom:NUMNUMN}
\gll Oman      {\em Navy}=ka    se-duuduk hatthu [doblas]$_{NUM}$  [thiga-blas]$_{NUM}$ [thaaun]$_{year}$=ke. \\ % bf
     Oman Navy=\textsc{loc}      \textsc{past}-stay   one   twelve   three-teen year=\textsc{simil}\\
    `I stayed in the Oman Navy for about twelve or thirteen years, something like that.'  (K051206nar17)
\z
}

However, it could be argued that in this case, we are not dealing with two modifications, but rather with one complex modification. It is not the case that the years had a cardinality of twelve and that the years at the same time had a cardinality of thirteen. Rather, they had only one cardinality, which is vague and therefore expressed by the complex numeral modifier [doblas thigablas]$_{NUM}$.

\paragraph{NUM ADJ N}
The numeral precedes the adjective and the noun.

\xbox{16}{
\ea\label{ex:constr:NP:prenom:NUMADJN}
\gll thiiga kiccil aanak pada aada. \\
     three small child \textsc{pl} exist  \\
    `There are three small children.'  (K081105eli02)
\z
}

\paragraph{NUM N N}
A string of two nouns can be preceded by a numeral.

\xbox{16}{
\ea\label{ex:constr:NP:prenom:NUMNN}
\gll Se=dang duuva$_{NUM}$ pompang$_{N}$ aade$_{N}$=le hathu$_{NUM}$ klaaki$_{N}$ aade$_{N}$=le anà-duuduk. \\ % bf
     \textsc{1s=dat} two woman younger.sibling=\textsc{addit} one man younger.sibling=\textsc{addit} \textsc{past}-exist  \\
    `I had two younger sisters and one younger brother.'  (B060115prs03)
\z
}
%
%  \xbox{16}{
%  \ea\label{ex:constr:NP:unreferenced}
%    \gll Se=dang duuva   aade  [duuva]$_{NUM}$ klaaki$_{N}$   aade$_{N}$ pada=le hatthu  pompang aade=le   arà-duuduk. \\
% \textsc{1s=dat} two younger.sibling two male younger.sibling \textsc{pl}=\textsc{addit} one female  younger.sibling=\textsc{addit} \textsc{non.past}-exist.\textsc{anim}\\
%  `I have two younger brothers and one younger sister.' (K060108nar01)
% \z

%army captain

\paragraph{ADJ   RELC  N}
This order was not found.
\paragraph{ADJ   DEIC N}
This order was not found.
\paragraph{ADJ   POSS N}
This order was not found.
\paragraph{ADJ   QUANT N}
This order was not found.

\paragraph{ADJ   NUM N}
Very rarely, a numeral can be found between an adjective and its head noun.

\xbox{16}{
\ea\label{ex:constr:NP:prenom:ADJNUMN1}
\gll [Mlaayu]$_{ADJ}$ [thiga-pulu tuuju]$_{NUM}$ [baasa]$_{N}$    aada. \\ % bf
 Malay three-ty seven language exist\\
`There are 37 Malay languages [in the world].' (K060116nar02)
\z
}


\xbox{16}{
\ea\label{ex:constr:NP:prenom:ADJNUMN2}
\gll Malay thiiga aanak pada arà-duuduk. \\ % bf
     Malay three child \textsc{pl} \textsc{non.past}-stay  \\
    `There are three Malay children [in this house].'  (G051222nar01) 
\z
} 


While above, we are dealing with a genuine adjective, the two examples below show this pattern with \trs{hathyang}{next} and \trs{lai}{additional}, where it can be argued that the scope relations are slightly different, similar to English \em Another three men\em, as compared to \em three other men\em.

\xbox{16}{
\ea\label{ex:constr:NP:prenom:ADJNUMN3}
\gll [Hathyang]$_{ADJ}$ [thiiga]$_{NUM}$ [oorang]$_{N}$=yang {\em Malaysia}=nang su-ambel baapi. \\ % bf
     next three man=\textsc{acc} Malaysia=\textsc{dat} \textsc{past}-take bring \\
    `They took three more men to Malaysia.'  (K060108nar02)
\z
}

\xbox{16}{
\ea\label{ex:constr:NP:prenom:ADJNUMN4}
\gll Kitham=nang duppang [lai]$_{ADJ}$ [duuva]$_{NUM}$ [bàrgaada]$_{N}$ asà-dhaathang aada. \\ % bf
 \textsc{1pl}=\textsc{dat} before additional two group \textsc{non.past}-come exist \\
`Before us, there are two more families.' (K060108nar02)
\z
}

\paragraph{ADJ ADJ N}
Two adjectives can stack before a noun. Two instances of this (\trs{puuthi paanjang}{white long} and \trs{kiccil jillek}{small ugly}) can be found in the following example.


\xbox{14}{
\ea
\gll Aanak pompang duuva=nang [[[hathu duuri pohong]=nang [[\textbf{puuthi}]$_{ADJ}$ [\textbf{paanjang}]$_{ADJ}$ [\textbf{jee\u n\u ggoth}]$_{N}$]=yang anà-kànà-daapath kìnna] hathu [\textbf{kiccil}]$_{ADJ}$ [\textbf{jillek}]$_{ADJ}$ [\textbf{Aajuth}]$_{N}$ hatthu]=yang su-kuthumung.   \\
     child female two=\textsc{dat} \textsc{indef} thorn tree=\textsc{dat} white long beard=\textsc{acc} \textsc{past}-patfoc-get strike \textsc{indef} small ugly dwarf \textsc{indef}=\textsc{acc} \textsc{past}-see  \\
    `The two girls  saw a small ugly dwarf whose long white beard had got stuck  in a thorn tree.'
\z
}
%K060116nar06  inni     katha sin        baae  kiccil atthu    pukkjang


\paragraph{ADJ   N N}
An adjective can precede a string of two \xref{ex:constr:NP:prenom:adjn:two} or more \xref{ex:constr:NP:prenom:adjn:more}  nouns.


\xbox{16}{
 \ea\label{ex:constr:NP:prenom:adjn:two}
   \gll Suda se=ppe    [thuuva]$_{ADJ}$ [anak]$_{N}$  [klaaki]$_{N}$ asàdhaathang dhlapan-blas    thaaun. \\  % bf
     so \textsc{1s}=\textsc{poss} old child male \textsc{copula} eight-teen year \\
`So my eldest son is eighteen' (K060108nar02)
\z
}

\xbox{16}{
 \ea\label{ex:constr:NP:prenom:adjn:more}
\gll [Panthas]$_{ADJ}$ [rooja]$_{N}$  [kumbang]$_{N}$   [pohong]$_{N}$  [komplok]$_{N}$ duuva  asà-jaadi su-aada. \\ % bf
      beautiful rose flower tree bush two \textsc{cp}-grow \textsc{past}-exist \\
    `Two beautiful rose bushes had grown.'  (K070000wrt04)
\z
}




\paragraph{N RELC  N}
This order was not found.
\paragraph{N DEIC N}
This order was not found.
\paragraph{N POSS N}
This order was not found.
\paragraph{N QUANT N}
This order was not found.
\paragraph{N NUM N}
This order was not found.
\paragraph{N ADJ N}
This order was not found.


\paragraph{N N N}
Strings of three or more nouns are also possible:

\xbox{16}{
\ea\label{ex:constr:NP:prenom:NNN}
\gll Panthas  [rooja]$_{N}$  [kumbang]$_{N}$ [pohong]$_{N}$  [komplok]$_{N}$ duuva asà-jaadi su-aada. \\ % bf
      beautiful rose flower tree bush two \textsc{cp}-grow \textsc{past}-exist \\
    `Two beautiful rose bushes had grown.'  (K070000wrt04)
\z
}

\subsection{Preliminary summary of prenominal modifications}\label{sec:nppp:Preliminarysummaryofprenominalmodifications}

We can schematize the findings about prenominal modifications as in \xref{cb:np:prenom}:

\cb[\label{cb:np:prenom}]{
$\left\{\begin{array}{l} \rm DEIC\\\rm POSS\end{array}\right\}$
RELC
$\left\{\begin{array}{l} \rm POSS\\\rm DEIC\end{array}\right\}$
QUANT
$\left\{\begin{array}{l} \rm NUM\\\rm ADJ\end{array}\right\}$*
N
Noun
}

Up to now, we have omitted discussion of the position of the indefiniteness marker \em (h)a(t)thu\em. The position of this marker within the NP will be discussed in the next section.

\subsection{The position of the indefinite modifier}\label{sec:nppp:Thepositionoftheindefinitemodifier}
Above, we have treated the order of some prenominal modifiers. There is one modifier which we have not treated, which is the indefiniteness marker \em (h)a(t)thu\em. This marker can intervene between any two markers in \xref{cb:np:prenom}, and additionally can also occur more than once \citep[cf.][]{Slomanson2006cll}. Example \xref{ex:constr:NP:atthu:triple} gives a sentence with multiple occurrence of \em atthu \em in one NP. The possibility to occur more than once points to an appositional nature of the NP, which will be discussed in more detail in \formref{sec:nppp:TheSLMNPasappositional}.

\xbox{16}{
\ea\label{ex:constr:NP:atthu:triple}
\gll Itthu=le \textbf{hathu}  mlaayu oorang pada=pe \textbf{hathu}  baae \textbf{hathu} makanan=jo. \\ % bf
     \textsc{dist}=\textsc{addit} \textsc{indef} Malay man \textsc{pl}=\textsc{poss} \textsc{indef} good \textsc{indef} food=\textsc{emph}  \\
    `That is also a good dish of the Malays.' (K061026rcp02)
\z
}



\paragraph{\em hatthu \em preceding a relative clause}
This order was not found.


\paragraph{\em hatthu \em following a relative clause}

The following four examples show the use of \em hatthu \em after a subordinate clause modifying the noun. This is a relative clause in \xref{ex:constr:NP:prenom:atthu:relc:follow1} to \xref{ex:constr:NP:prenom:atthu:relc:follow3} and a purposive clause in \xref{ex:constr:NP:prenom:atthu:relc:follow4}.

\xbox{16}{
\ea\label{ex:constr:NP:prenom:atthu:relc:follow1}
\gll [anà-birthi-king]$_{RELC}$ [hatthu]$_{INDEF}$ [paapang]$_{N}$=ka \\ % bf
      \textsc{past}-stand-\textsc{caus} \textsc{indef} pole=\textsc{loc} \\
    `on a plank put upright' (K081105eli02)
\z
}  

\xbox{16}{
\ea\label{ex:constr:NP:prenom:atthu:relc:follow2}
\gll [Kìrras pinthu=nang arà-thatti]$_{RELC}$ [hathu]$_{INDEF}$ [svaara]$_{N}$ su-dìnngar. \\ % bf
     strong door=\textsc{dat} \textsc{simult}-hammer] \textsc{indef} noise \textsc{past}-hear \\
    `They heard a noise of hard hammering at the door.'  (K070000wrt04)
\z
}

\xbox{16}{
\ea\label{ex:constr:NP:prenom:atthu:relc:follow3}
\gll  [Seelon=le kithang=pe mlaayu=nang=le hatthu bagiyan anà-aada]$_{RELC}$ [hatthu]$_{INDEF}$ [nigiri]$_{N}$ su-jaadi\\
 Ceylon=\textsc{addit}  \textsc{1pl}=\textsc{poss} Malay=\textsc{dat}=\textsc{addit} \textsc{indef} part exist \textsc{indef} country  \textsc{past}-become\\
`Ceylon became a country where our Malays also have a part in.' (K051222nar04)
\z
}

\xbox{16}{
\ea\label{ex:constr:NP:prenom:atthu:relc:follow4}
\gll  [Thaangang  mà-saapu]$_{RELC}$    [hatthu]$_{INDEF}$ [{\em paper}]$_{N}$ kapang-mintha,  baapa=yang       su-kuthumung. \\ % bf
    hand \textsc{inf}-sweep one paper when-ask father=\textsc{acc} \textsc{past}-see  \\
    `When he asked a paper to clean his hands, he saw father.' (K051205nar05)
\z
}

%  \xbox{16}{
% \ea\label{ex:constr:NP:prenom:atthu:relc:follow4}
%    \gll Go  [bannyak pasiyeth   nya-baapi]$_{RELC}$   [hathu]$_{INDEF}$  oorang$_{N}$ \\ %     \textsc{1s}  {\em much}   {\em trouble} \textsc{past}-{\em bring} \textsc{indef} man. \\
% `I am a man who has had lots of troubles' (B060115nar04)
% \z
% }
%
% \xbox{16}{
% \ea\label{ex:constr:NP:prenom:atthu:relc:follow5}
% \ea
% \gll Inni     pukujan=nang  kam-pii. \\
%      \textsc{prox} work=\textsc{dat} when-go \\
%    `When they go to this work.'
% \ex
% \gll Deram pada [itthu makumpul]]$_{RELC}$ [hatthu=]$_{INDEF}$ [mosthor]$_{N}$ thraa. \\
% \textsc{3pl} \textsc{dist} \textsc{inf}-add \textsc{indef}=manner neg\\
% `They lack a habit of gathering' (G051222nar01)
% \z
% \z
% } \\

\paragraph{\em hatthu \em following or preceding a deictic}
Given the semantics of \em hatthu \em marking indefiniteness, and the deictics marking definite referents, these morphemes cannot cooccur.

\paragraph{\em hatthu \em preceding a possessor}

\em Hatthu \em can precede the possessor \xref{ex:constr:NP:prenom:atthu:poss:prec}, but is more likely to follow it as will be discussed in the next section

\xbox{16}{
\ea\label{ex:constr:NP:prenom:atthu:poss:prec}
\gll Itthu=le [hathu]$_{INDEF}$  [mlaayu oorang pada=pe]$_{POSS}$ hathu baae hathu [makanan]$_{N}$=jo. \\ % bf
     \textsc{dist}=\textsc{addit} \textsc{indef} Malay man \textsc{pl}=\textsc{poss} \textsc{indef} good \textsc{indef} food=\textsc{emph}  \\
    `That is also a good dish of the Malays.' (K061026rcp02)
\z
}

\paragraph{\em hatthu \em following a possessor}

To indicate that within a set of possessed items, an indefinite one is talked about, \em hatthu \em and the possessor can cooccur. In those cases, \em atthu \em normally follows the possessor. In example \xref{ex:constr:NP:atthu:poss:follow1}, the speaker has many sons, but only one of them, who is at the time of speaking not known to the hearer, is on the estates.

\xbox{16}{
\ea\label{ex:constr:NP:atthu:poss:follow1}
\gll Karang [se=ppe]$_{POSS}$    [hathu]$_{INDEF}$  [aanak]$_{N}$ {\em estate}=ka. \\ % bf
     now \textsc{1s=poss} \textsc{indef} child estate=\textsc{loc}  \\
    `Now one of my sons is on the estates.' (K051201nar01)
\z
}

Similar things can be said about \xref{ex:constr:NP:atthu:poss:follow2}, where the identity of the Malay woman's child is unknown to the hearer.

\xbox{16}{
\ea\label{ex:constr:NP:atthu:poss:follow2}
\gll Lai=le  hatthu  [sudaari=pe]$_{POSS}$ [atthu]$_{INDEF}$ [aanak]$_{N}$, kiccil aanak atthu  puruth=ka    se-mnii\u n\u ggal. \\ % bf
 more=\textsc{addit} \textsc{indef} sister=\textsc{poss} \textsc{indef} child small child \textsc{indef} womb=\textsc{loc} \textsc{past}-die  \\
    `A child of yet another sister, a small child, died in the womb.'  (B060115nar02)
\z
}

The third example for this patterns has again to do with indicating that the speaker is not supposed to be able to identify which one of the children the speaker is talking about.


\xbox{16}{
\ea\label{ex:constr:NP:atthu:poss:follow3}
\gll [Itthu    kaake=pe]$_{POSS}$          [hatthu]$_{INDEF}$ [aanak]$_{N}$=jo    se=ppe    umma. \\ % bf
      \textsc{dist} grandfather=\textsc{poss} once child=\textsc{emph} \textsc{1s=poss} mother \\
    `One of that grandfather's children is my mother.' (K051205nar05,K081105eli02)
\z
}
%
% \xbox{16}{
% \ea\label{ex:constr:NP:atthu:poss:follow1}
% \gll Maana=ke hathu [government=pe]$_{POSS}$   [hathu]$_{INDEF}$   [thumpath]$_{N}$ =ka    asà-pii   pukurjan bole=girja. \\
%     where=\textsc{simil} \textsc{indef} government=\textsc{poss} \textsc{indef} place=\textsc{loc} \textsc{cp}-go work can-do   \\
%     `They can go to any government place and work there.' (K051222nar05)
% \z
% } \\
%


\paragraph{\em hatthu \em following or preceding a quantifier}
These orders were not found.
\paragraph{\em hatthu \em preceding a numeral}
\em Hatthu \em can precede a numeral, in which case it marks uncertainty about the exact amount. Still, this is an instance of \em hatthu \em modifying the numeral expression rather than the noun.

\xbox{16}{
\ea\label{ex:constr:NP:prenom:INDEFNUMN}
\gll Oman {\em Navy}=ka se-duuduk  [hatthu$_{INDEF}$ doblas$_{NUM}$  thiga-blas$_{NUM}$]$_{NUM}$ [thaaun]$_{N}$=ke. \\ % bf
     Oman         Navy=\textsc{loc}       \textsc{past}-stay   \textsc{indef}   twelve   three-teen year=\textsc{simil}\\
    `I stayed in the Oman Navy for about 12 or thirteen years, something like that.'  (K051206nar17)
\z
}

\paragraph{\em hatthu \em following a numeral}
This order was not found.

\paragraph{\em hatthu \em preceding an adjective}
\em Hatthu \em can precede an adjective modifying a noun.


\xbox{16}{
\ea\label{ex:constr:NP:prenom:INDEFADJN}
\gll Hathu muusing=ka ...  [hathu]$_{INDEF}$  [kiccil]$_{ADJ}$ [ruuma]$_{N}$ su-aada \\
     \textsc{indef} time=\textsc{loc} ... \textsc{indef} small house \textsc{past}-exist  \\
    `Once upon a time, there was a small house.'  (K07000wrt04)
\z
}


% \xbox{16}{
% \ea\label{ex:constr:NP:prenom:INDEFADJN}
% \gll Samma hatthu$_{INDEF}$ baae$_{ADJ}$  kumpulan$_{N}$    an-duuduk. \\ % bf
%       every \textsc{indef} good association \textsc{past}-stay \\
%     `They all stayed together as one association'  (K060116nar02)
% \z
% }\\
%
% \xbox{16}{
% \ea\label{ex:constr:NP:prenom:}
% \gll {\em majority} katha arà-biilang hathu$_{INDEF}$ liivath$_{ADJ}$ blaangan$_{N}$ {\em votes}=dering su-bunnang. \\ % bf
%      majority \textsc{quot} \textsc{non.past}-say \textsc{indef} excessive amount votes=\textsc{abl} \textsc{past}-win \\
%     ` ``Majority'' means that he had obtained an excessive amount of all the votes.'  (K051222nar06)
% \z
% }\\



\paragraph{\em hatthu \em following an adjective}
Just like \em hatthu \em can precede the adjective, it can also follow, as shown in \xref{ex:constr:NP:prenom:ADJINDEFN}.

% \xbox{16}{
% \ea
% \gll Blaakang se=dang=jo        [kiccil] [hatthu] [seksa]  hatthu se-jaadi. \\ % bf
%      after \textsc{1s=dat} small \textsc{indef} problem \textsc{indef} \textsc{past}-become  \\
%     `After that I happened to have a problem.'  (B060115prs01)
% \z
% }\\

\xbox{14}{
\ea \label{ex:constr:NP:prenom:ADJINDEFN}
\gll Itthu [bannyak [laama]$_{ADJ}$ [hathu]$_{INDEF}$ [ruuma]$_{N}$]. \\ % bf
      \textsc{dist} very old \textsc{indef} house \\
    `That one was a very old house.'  (K070000wrt04)
\z
}

\paragraph{\em hatthu \em preceding a modifying noun}

It is the normal case for \em atthu \em to precede a modifying noun.

\xbox{16}{
\ea\label{ex:constr:NP:prenom:INDEFNN}
\gll Sithu=ka, [hathu]$_{INDEF}$ bìssar [beecek caaya]$_{N}$ [Buruan]$_{N}$  su-duuduk.\\ % bf
     there=\textsc{loc} \textsc{indef} big mud colour bear \textsc{past}-exist \\
    `There was a big brown bear.'  (K070000wrt04)
\z
}


\paragraph{\em hatthu \em following a modifying noun}

In some instances, \em hatthu \em can intervene between the two nouns of a two-noun sequence. The following two examples show this for a monkey group and a street bend.

\xbox{16}{
\ea\label{ex:constr:NP:prenom:NINDEFN1}
\gll Ini pohong atthas=ka [moonyeth]$_{N}$ [hathu]$_{INDEF}$=[kavanan]$_{N}$ su-aada. \\ % bf
     \textsc{prox} tree \textsc{top=loc}  monkey \textsc{indef}=group \textsc{past}-exist\\
    `On top of this tree was a group of monkeys.'   (K070000wrt01)
\z
}

\xbox{16}{
\ea\label{ex:constr:NP:prenom:NINDEFN2}
\gll [[Jaalang$_{N}$ hathu$_{INDEF}$=pii\u n\u ggir$_{N}$]=ka anà-aada hathu=pohong] baava=ka su-see\u nder. \\ % bf
      road \textsc{indef}=border=\textsc{loc} \textsc{past}-exist.inanim \textsc{indef}=tree down=\textsc{loc} \textsc{past}-rest   \\
    `Because he was tired, he sat down under a tree which stood at a side of the street.'  (K070000wrt01)
\z
}

% \xbox{16}{
% \ea\label{ex:constr:NP:unreferenced}
% \gll Itthu    vakthu kithang=nang nya-aada     asàdhaathang ini     [JVP katha]$_{UTT}$ [hathu]$_{INDEF}$  [{\em problem}]$_{N}$ hatthu=jo. \\
%       \textsc{dist} time \textsc{1pl}=\textsc{dat} \textsc{past}-exist \textsc{copula} \textsc{prox} JVP \textsc{quot} \textsc{indef} problem \textsc{indef} \\
%     `What we had at that time was the so-called JVP problem.' (K051206nar10)
% \z
% } \\


\subsection{The final structure of the noun phrase}\label{sec:nppp:Thefinalstructureofthenounphrase}

By combining the order of postnominal modifiers, presented in   \xref{cb:np:postn}, with the order of the prenominal modifiers presented in \xref{cb:np:prenom} and the possible occurrences of the indefiniteness marker \em atthu\em, we get the full templatic structure of the NP, represented in  \xref{cb:np:prepostnom}. The arrows represent positions where the indefinite marker \em atthu \em can occur. The semantic (in)compatibility of certain items is not reflected in \xref{cb:np:prepostnom}.

\cbx[\label{cb:np:prepostnom}]{ 
$\downarrow$
$\left\{\begin{array}{l} \rm DEIC\\\rm POSS\end{array}\right\}$
RELC
$\left\{\begin{array}{l} \rm POSS\\\rm DEIC\end{array}\right\}$
$\downarrow$
QUANT
$\left\{\begin{array}{l} \rm NUM\\\rm ADJ\\\end{array}\right\}$*
$\downarrow$
N
$\downarrow$
\textbf{Noun}
$\begin{array}{c}\downarrow\\\rm N (PL) \\\rm ADJ (PL)\\\rm QUANT (PL) \\\rm NUM (PL) \end{array}$}{NP}

The only order which cannot be captured by this schema is QUANT POSS N \formref{ex:constr:NP:prenom:quantposs}, but this order could be explained by quantifier floating \formref{ex:constr:NP:prenom:quantposs}.

Analyzing this schema, we can observe a certain number of points: The relative clause belongs to the leftmost elements. This can be explained by the desire to have little material separating heads from modifiers \citep{Hawkins1994}. Given that relative clauses are generally  heavier than the other modifiers, an obvious solution to minimize separation is to put them at the margins of the NP \citep[298f]{Rijkhoff2002}, the left edge in the case of SLM. In schema \xref{cb:np:prepostnom}, the relative clause is listed together with the deictics and the possessors because of isolated examples which permit the order DEIC RELC and POSS RELC. It might also be possible to discount these examples and have the relative clause as the leftmost element, which is the position it assumes in the near totality of cases. The position of the deictics and the possessors with regard to the other modifiers is unremarkable. What is more remarkable is the possibility to arrange ADJ and NUM in both possible orders. It seems that the order ADJ NUM violates  semantic scope; the numeral should have scope over the adjective, but its syntactic position suggests that it has not. Another interesting aspect to note is that the postnominal field only allows one element, while the prenominal field allows an arbitrary number of elements. One could argue that postmodifications are archaic constructions which are somehow lexicalized, like \trs{orang ikkang}{man'+`fish'=fisherman} or \trs{aer meera}{water'+`red'=`tea}. These lexicalized postmodifications persist, but generally do not involve more than one modifier. More elaborate modifications are ad hoc, and prenominal.\footnote{See \citet[53]{Saldin2001} for a diverging view.} The limited possibility of postmodification is then a consequence of the limited lexicalization of collocations with more than two elements.

A final puzzle is the position of the indefiniteness marker \em hatthu\em, the theoretical implications of this will be discussed in more detail in \formref{sec:nppp:TheSLMNPasappositional}.



%
%\xbox{16}{
%\ea\label{ex:constr:NP:unreferenced}
%\gll Baae  hatthu  kittham=pe     mlaayu kumpulan. \\
%      good \textsc{indef} \textsc{1pl}=\textsc{poss} Malay association \\
%    `It was a good one, our Malay association.'  (K060116nar06)
%\z
%}\\
%
%
%\xbox{16}{
%\ea\label{ex:constr:NP:unreferenced}
%\gll Sepe       \textbf{oorang} \textbf{tuva} $^\curvearrowleft$pada    anà-biilang kitham pada {\em Malaysia}=dering    anà-dhaathang    katha. \\
% \textsc{1s}=\textsc{poss} man old \textsc{pl} \textsc{past}-say \textsc{1pl} \textsc{pl} Malaysia=\textsc{abl} \textsc{past}-come \textsc{quot}\\
%`My elders told me that we had come from Malaysia.' (K060108nar02)
%\z
%}
%
%
%\xbox{16}{
%\ea\label{ex:constr:NP:unreferenced}
%\gll \textbf{hatthu}$\curvearrowright$  \textbf{samping}  \textbf{hatthu}$\curvearrowright$  \textbf{oorang} balle=kaasi. \\
%     one goat one man can-give  \\
%    `One man can sacrifice one goat.'  (K060112nar01.txt)
%\z
%}\\
%
%
%
%
%\xbox{16}{
%\ea\label{ex:constr:NP:unreferenced}
%\gll Itthuka       bernaama   anà-pii      {\em Saints}  pada=jo     sudaara $\curvearrowleft$thuuju. \\
%     then before \textsc{past}-go  s \textsc{pl}=\textsc{emph} sibling seven  \\
%    `There were seven sibling saints then.'  (K060108nar02)
%\z
%}\\
%
%The indefinite article can  occur preposed, postposed, or on both sides.
%
%\xbox{16}{
%\ea\label{ex:constr:NP:unreferenced}
%\gll Laile \textbf{hatthu}$^\curvearrowright$ \textbf{sudaari}=pe  \textbf{atthu}$^\curvearrowright$   \textbf{aanak}, kiccil \textbf{aanak} $^\curvearrowleft$\textbf{atthu}  puruth=ka    se-mnii\u n\u ggal. \\
%     another \textsc{indef} sister=\textsc{poss} \textsc{indef} child small child \textsc{indef} womb=\textsc{loc} \textsc{past}-die  \\
%    `A child of another sister, a small child, died in the womb.'  (B060115nar02)
%\z
%}\\
%
%
%
%Deictics,  possessors and relative clauses can only be preposed.  Having a preposed and a postposed modification is also possible \xref{ex:constr:NP:seppeaanakklaakipada}.
%
%\xbox{16}{
%\ea\label{ex:constr:NP:seppeaanakklaakipada}
%\gll Se=ppe$^\curvearrowright$ aanak $^\curvearrowleft$klaaki pada. \\
% \textsc{1s}=\textsc{poss} child boy \textsc{pl}\\
%`My sons.' (K060108nar02)
%\z
%}
%
%
%\xbox{16}{
%\ea\label{ex:constr:NP:unreferenced}
%\gll Se=dang aade pada mpath arà-duuduk. \\
%1=dat younger.sibling four \textsc{non.past}-stay \\
%`.' (nosource)
%\z
%}
%karang arà-biilang    kalu bunnar=nang                             kitham=pe      inni     British
%
%\tx Rule=ka           annaduuduk   mlaayu pada baae  thumpath eaada N060113nar01



\section{Noun phrases based on a deictic}\label{sec:nppp:Nounphrasesbasedonadeictic}
Next to nouns, NPs can also be formed based on a deictic, which replaces the lexical content. The following examples show the use of an NP based on the proximal deictic \em ini \em \xref{ex:constr:NP:np:deictic:ini} and the distal deictic \em itthu \em \xref{ex:constr:NP:np:deictic:itthu}. The postposition \em =yang \em following the deictics shows that there is no other content in the NP.

\xbox{16}{
\ea\label{ex:constr:NP:np:deictic:ini}
\gll Thraa thraa \textbf{inni=yang} masà-picca-kang katha biilang. \\
      no no \textsc{prox}=\textsc{acc} must=broken-\textsc{caus} \textsc{quot} say \\
    ` ``No, no'', he said, ``you must break this one.'' '  (K051220nar01)
\z
}

\xbox{16}{
\ea\label{ex:constr:NP:np:deictic:itthu}
\gll Asà-cuuci, \textbf{itthu=yang} baaye=nang asà-rubbus, ... \\
     \textsc{cp}-wash \textsc{dist}=\textsc{acc} good=\textsc{dat} \textsc{cp}-boll, ...  \\
    `You wash (it), then you boil it well and ... .'  (B060115rcp01)
\z
}


%
% \xbox{16}{
% \ea\label{ex:constr:NP:unreferenced}
% \gll Se=dang kalu suda bannyak thàrà-thaau  inni=pe         atthas mà-biilang=nang. \\
%      \textsc{1s=dat} if thus much \textsc{neg}-know \textsc{prox=poss} about \textsc{inf}-say=\textsc{dat}  \\
%     `So, as for me, I cannot tell you much about this.' (K051205nar04)
% \z
% }

NPs based on a deictic can only be modified by the plural marker \em pada, \em as in \xref{ex:constr:NP:np:deictic:ini:pada} \xref{ex:constr:NP:np:deictic:itthu:pada}. Other modifications are not possible.

\xbox{16}{
\ea\label{ex:constr:NP:np:deictic:ini:pada}
\gll Mlaayu pada=jo \textbf{inni} \textbf{pada}=ka punnu pukurjan anà-girja. \\
     Malay \textsc{pl}=\textsc{emph} \textsc{prox} \textsc{pl}=\textsc{loc} much work \textsc{past}-do  \\
    `It was the Malays who did a lot of work in these (jobs) [i.e army, navy, police].'  (K051222nar05)
\z
}

\xbox{16}{
\ea\label{ex:constr:NP:np:deictic:itthu:pada}
\gll Iiya, \textbf{itthu} \textbf{pada}=jo su-dhaathang. \\
      yes \textsc{dist} \textsc{pl}=\textsc{emph} \textsc{past}-come \\
    `Yeah, those [people] came.'  (K051201nar02)
\z
}

A special case is \xref{ex:constr:NP:np:deictic:itthu:pe}, where the NP consisting of a deictic hosts the possessive marker \em =pe\em, which in turn is made into a new NP by conversion.

\xbox{16}{
\ea\label{ex:constr:NP:np:deictic:itthu:pe}
\gll [[[Itthu$_{DEIC}$]$_{NP}$=pe]$_{PP}$=\zero]$_{NP}$        pada=jo    bannyak mlaayu pada karang siini aada. \\ % bf
     \textsc{dist=poss} \textsc{pl=foc} much Malay \textsc{pl} now here exist  \\
    `It's their folks we get a lot of today here.' (K051205nar04)
\z
}

\section{Noun phrases based on a personal pronoun}\label{sec:nppp:Nounphrasesbasedonapersonalpronoun}
NPs can also be formed based on a personal pronoun. Normally, personal pronouns occur unmodified, as in \xref{ex:constr:NP:np:pron:unmodified}.

\cbx{PERSPRON}{NP}

\xbox{16}{
\ea\label{ex:constr:NP:np:pron:unmodified}
\gll \textbf{Kithang}$_{NP}$=nang baaye=nang mulbar bole=baaca. \\
      \textsc{1pl}=\textsc{dat} good=\textsc{dat} Tamil can=read \\
    `We can read Tamil well.'  (K051222nar06)
\z
}

It is possible to use appositions of number for plural pronouns. These can be the plural marker \em pada \em as in \xref{ex:constr:NP:np:pron:app:pada}, a numeral as in \xref{ex:constr:NP:np:pron:app:num1} or a whole expression as in \xref{ex:constr:NP:np:pron:app:num2}\xref{ex:constr:NP:np:pron:app:expr}.


\cbx{PERSPRON  $\left\{\begin{array}{l} \rm \rm PL\\\rm NUM\\\rm  EXPR\end{array}\right\}$}{NP}


\xbox{16}{
\ea\label{ex:constr:NP:np:pron:app:pada}
\gll Itthu=nam blaakang=jo, \textbf{kitham} \textbf{pada} anà-bìssar. \\
 \textsc{dist} after=\textsc{emph} \textsc{1pl} \textsc{pl} \textsc{past}-big\\
`It was after that that we grew up.' (K060108nar02)
\z
}


\xbox{16}{
\ea\label{ex:constr:NP:np:pron:app:num1}
\gll Mr    Sebastian            aada, se aada, \textbf{kitham}  \textbf{duuva} arà-oomong. \\
 Mr Sebastian exist \textsc{1s} exist \textsc{1pl} two \textsc{non.past}-speak\\
`You are here, I am here, the two of us are talking.' (K060116nar05)
\z
}

\xbox{16}{
\ea\label{ex:constr:NP:np:pron:app:num2}
\gll \textbf{Derang} \textbf{duuva} \textbf{oorang}=pe naama pada kalu {\em Snow-white} hattheyang {\em Rose-red}. \\
      \textsc{3pl} two man=\textsc{poss}  name \textsc{pl} if Snow.White other Rose.Red\\
    `The names of the two of them were, if you ask,  Snow White, the other one Rose Red.'  (K070000wrt04)
\z
}

\xbox{16}{
\ea\label{ex:constr:NP:np:pron:app:expr}
\gll Itthu=kaapang se=ppe   baapa  se=ppe kaake      se=ppe      kaake=pe      baapa, \textbf{kithang} \textbf{samma} \textbf{oorang} Seelon=pe oorang pada. \\
      \textsc{dist}=when \textsc{1s}=\textsc{poss} father \textsc{1s}=\textsc{poss} grandfather \textsc{1s}=\textsc{poss} grandfather=\textsc{poss} father, \textsc{1pl} all man Ceylon=\textsc{poss} man \textsc{pl}\\
    `Then my father and my grandfather and my grandfather's father, all of us people became Ceylon people.'  (K060108nar02)
\z
}

The use of these appositions is optional, as the following example shows, where \em pada \em is now present, now absent.


\xbox{16}{
\ea\label{ex:constr:NP:np:pron:app:pada:double}
\gll \textbf{Lorang}=le, \textbf{lorang} \textbf{pada}=pe umma=le see=yang baaye=nang anà-kuaather. \\
      \textsc{2pl}=\textsc{addit} \textsc{2pl} \textsc{pl}=\textsc{poss} mother=\textsc{addit} \textsc{1s}=\textsc{acc} good=\textsc{dat} \textsc{past}-look.after \\
    `You and your mother took good care of me.' (K070000wrt04,K081105eli02)
\z
}

\section{Noun phrases based on interrogative pronouns}\label{sec:nppp:Nounphrasesbasedoninterrogativepronouns}
There are three ways to form NPs based on interrogative pronouns, bare, reduplicated and combined with clitics.

The simplest one is to use only the pronoun. This means that the referent is unknown and implies a question, as in \xref{ex:constr:NP:np:interr:bare}.

\cbx{WH}{NP}

\xbox{16}{
\ea\label{ex:constr:NP:np:interr:bare}
\gll Itthu=nang blaakang \textbf{aapa} nya-gijja? \\
     \textsc{dist}=\textsc{dat} after what \textsc{past}-make  \\
    `What did (they) do then?'  (K051206nar07)
\z
}


An NP can be formed by a reduplicated interrogative pronoun as well. In this case, an exhaustive answer is required.

 \xbox{16}{
\ea\label{ex:constr:NP:np:interr:redupl}
\gll [Aapa\~{}aapa]$_{NP}$ kitham Kandi=pe {\em cultural} {\em show} atthu=le thaaro? \\ % bf
 what\~{}red \textsc{1pl} Kandy=\textsc{poss} cultural show one=\textsc{addit} put\\
`What did we also put on the Kandy Cultural show?' (K060116nar11)
\z
}

An interrogative pronoun can be combined with the clitics \em =so, =ke \em or \em =pon \em to yield the reading of an indefinite pronoun.

\cbx{WH=$\left\{\begin{array}{c}=so\\=ke\\=pon\end{array}\right\}$}{NP}

\xbox{16}{
\ea\label{ex:constr:NP:interr:so}
\gll \textbf{Saapa=so} {\em Malay} {\em exam} arà-girja. \\
     who=\textsc{undet} Malay exam \textsc{non.past}-make \\
   `Someone is taking a Malay exam.' (K060103cvs01)
\z
}


\xbox{16}{
\ea\label{ex:constr:NP:interr:ke}
\gll Incalla   [lai     thaau sudaara sudaari pada]=ka    bole=caanya    ambel [nya-gijja    lai     \textbf{saapa=kee}  aada=si    katha]. \\
      Hopefully other know brother sister \textsc{pl}=\textsc{loc} can-ask take \textsc{past}-make other who=\textsc{simil} exist=\textsc{interr} \textsc{quot} \\
    `Hopefully, you can enquire from another person you know whether there is someone else who did something.' (N061031nar01)
\z
}


\xbox{16}{
\ea\label{ex:constr:NP:interr:pon}
\gll Kithang \textbf{craapa=pon} kithang=pe kappal asà-ambel, kithang su-baalek kithang=pe {\em harbour}=nang. \\
     \textsc{1pl} how=any  \textsc{1pl}=\textsc{poss} ship \textsc{cp}-take \textsc{1pl} \textsc{past}-turn \textsc{1pl}=\textsc{poss} harbour=\textsc{dat}\\
    `Anyhow, we managed to take our ship and return to the harbour.'  (K051206nar20)
\z
}

% \xbox{16}{
% \ea\label{ex:constr:NP:unreferenced}
% \gll Saapa=nang=le ini hadarath masà-thaau. \\
%       who=\textsc{dat}=\textsc{addit} \textsc{prox} procedures must-know \\
%     `Everybody must know these procedures.' (K061127nar03)
% \z
% } \\

% K051205nar05.txt:\tx blaakang aapaso        asagiiling   patthuke       suthaaro

There are subtle differences between the meanings conveyed by the three clitics. In example \xref{ex:constr:NP:interr:keeso}, the speakers are lost in the woods, but finally arrive at their destination. If the similative clitic \em =kee \em is used, the meaning is one of concession, or canceling of implicatures, similar to English \em anyhow\em. If the `undetermined' clitic \em =so \em is used instead of \em =kee\em, the meaning is that the walkers were not aware of their having reached their destination.


\xbox{16}{
\ea\label{ex:constr:NP:interr:keeso}
\gll Kithang \textbf{caraapacara}=kee/so [mà-pii anà-aada thumpath]=ka su-sampe. \\
     \textsc{1pl} how=\textsc{simil/undet} \textsc{inf}-go \textsc{past}-exist place=\textsc{loc} \textsc{past}-reach  \\
    `We arrived at our destination unknowingly (=kee)/anyhow (=so).' (K081105eli02) %kee anyhow, soo unknowingly,
\z
}

\section{Noun phrases based on a numeral/quantifier}\label{sec:nppp:Nounphrasesbasedonanumeral/quantifier}
Numerals and quantifiers can constitute a noun phrase on their own, even if they still need some quantifiable content in discourse to relate to.

\cbx{ (DEIC) NUM (PL) }{NP}
\cbx{ (DEIC) QUANT (PL) }{NP}

Example \xref{ex:constr:NP:quant:alone} shows the use of a quantifier as the only element of an NP.

 


\xbox{14}{
\ea\label{ex:constr:NP:quant:alone}
\gll  Mr  Yusuf \textbf{samma} asà-ambel=apa,  {\em Commercial} {\em Bank}=ka    su-thaaro \\
      Mr Yusuf all \textsc{cp}-take=after Commercial Bank=\textsc{loc} past-put \\
    `Mr Yusuf took everything and deposited it at the Commercial Bank.' (K060116nar09)
\z
} 


Example \xref{ex:constr:NP:num} shows the use of a numeral as the only element of an NP.

\xbox{16}{
\ea\label{ex:constr:NP:num}
\gll `Siking' katha arà-biilang {\em`that-wise/So'}, inni \textbf{duuva}=le buthul. \\
      siking \textsc{quot} \textsc{non.past}-say that-wise/So \textsc{prox} two=\textsc{addit} correct \\
    ` ``siking'' means ``that-wise/so''; both of them are correct.'  (K071113eml01)
\z
}


NPs based on quantifiers can also be modified by the plural marker, as shown in \xref{ex:constr:NP:QUANT:pl}.


% \xbox{16}{
% \ea\label{ex:constr:NP:NUM:pl}
% \gll Igaama, arà-muuji mosthor, samma hatthu=jo; arà-biilang, \textbf{d(h)ua} \textbf{pada}. \\
%       religion \textsc{non.past}-pray manner all one=foc; \textsc{non.past}-say two \textsc{pl} \\
%     `Embracing of all religions is the same if the pleading is alike.'  (K061026prs01)(test)
% \z
% }\\

\xbox{16}{
\ea\label{ex:constr:NP:QUANT:pl}
\gll \textbf{Spaaru} \textbf{pada} bannyak suuka arà-blaajar. \\
      some \textsc{pl} much like \textsc{non.past}-learn\\
    `(Only) some like to study a lot.'  (B060115cvs01)
\z
}

\section{Reciprocal noun phrases}\label{sec:nppp:Reciprocalnounphrases}
The reciprocal construction is formed by adding \em hatthunang hatthu \em to a noun with plural reference\footnote{See \citet[177]{Beythan1943} and \citet[11]{Malten1989} for an analogous Tamil construction.}

\xbox{14}{
\ea
\gll Oorang pada \textbf{hatthu}=\textbf{nang} \textbf{hatthu} maara. \\
      man \textsc{pl} \textsc{indef}=\textsc{dat} \textsc{indef} anger \\
    `People are angry with each other.' (K081106eli01) 
\z
}

The dative marker \em =nang \em is   present in the construction, even if the verb used would normally subcategorize for another case marker, like \trs{buunung}{kill} in \xref{ex:constr:NP:reciprocal:nangyang}, which normally governs the accusative.

\xbox{14}{
\ea\label{ex:constr:NP:reciprocal:nangyang}
\gll Oorang pada \textbf{hatthu=(nang/*yang)} \textbf{hatthu} arà-buunung ambel. \\
       man \textsc{pl} \textsc{indef}=\textsc{dat}/\textsc{acc} \textsc{indef} \textsc{non.past}-kill take\\
    `The people kill each other.' (K081106eli01)% must be nang
\z
}


\section{Noun phrases based on a postpositional phrase}\label{sec:nppp:Nounphrasesbasedonapostpositionalphrase}
Postpositional phrases can convert into noun phrases, as shown in \xref{ex:constr:NP:PP} and schematized in \xref{ex:constr:NP:PP:schema}.

\xbox{16}{
\ea\label{ex:constr:NP:PP}
\gll \textbf{Itthu=pe}        \textbf{pada}=jo    bannyak mlaayu pada karang siini aada. \\
     \textsc{dist=poss} \textsc{pl=foc} much Malay \textsc{pl} now here exist  \\
    `It's their folks we get a lot of today here.' (K051205nar04)
\z
}

\ea\label{ex:constr:NP:PP:schema}
$
[
	[
		[itthu_{N}]_{NP}
	=pe]_{PP}
~\zero{}~ pada~]_{NP}
$
\z


\section{Noun phrases based on a clause}\label{sec:nppp:Nounphrasesbasedonaclause}
In SLM, clauses can be used as noun phrases as they are. No further morphological flagging of this use is necessary. While in English, NPs consisting of clauses are indicated by special means, such as the complementizer \em that \em in \em I appreciated that you came\em, this is not the case in SLM. Clauses can be used as they are as complements of verbs   \formref{sec:nppp:ArgumentClause}, or as NPs (headless relative clauses) \formref{sec:nppp:Headlessrelativeclauses}.


\subsection{Argument Clause}\label{sec:nppp:ArgumentClause}
SLM clauses can be used as the head of a term without further measures (like nominalizations or complementizers) being taken.

\cbx{ CLAUSE }{NP}

Example \xref{ex:constr:NP:CLAUSE:finite} shows the use of a finite clause as a complement of the verb \trs{suuka}{like}.

\xbox{16}{
\ea\label{ex:constr:NP:CLAUSE:finite}
\gll Kitham=pe baapa su-biilang [[\textbf{lorang} \textbf{suurath=yang} \textbf{mlaayu=dering} \textbf{anà-thuulis}]$_{CLAUSE}$=nang bannyak arà-suuka]. \\ % bf
      \textsc{1pl}=\textsc{poss} father \textsc{past}-say \textsc{2pl} letter=\textsc{acc} Malay=\textsc{abl} past=write=\textsc{dat} much simult-like \\
    `Daddy said that he liked very much that you wrote the letter in Malay.'  (Letter 26.06.2007)
\z
}

Postpositions can then attach to this new NP like to any other NP. This produces the curious situation that a verbal lexeme can have a verbal prefix on the left side and a case marker on the right side, like \trs{anà-}{past}  and \trs{=nang}{\textsc{dat}} both attaching to the verb \trs{thuulis}{write} in  \xref{ex:constr:NP:CLAUSE:finite}.\footnote{It would also be possible to analyze these embedded clauses as subordinates, and the postpositions as conjunctions. But this needlessly increases the number of categories without adding to our understanding. On the contrary, it obscures the fact that lexical and clausal arguments are treated exactly alike when it comes to assigning semantic roles.}

An alternative to using finite clauses is to put the verb into the infinitive, which yields slightly different semantics, often purposive as in \xref{ex:constr:NP:np:clause:inf:purp}.


\xbox{16}{
\ea\label{ex:constr:NP:np:clause:inf:purp}
\gll Hathu haari, hathu oorang [[\textbf{thoppi} \textbf{mà-juval}]$_{CLS}$]$_{NP}$=nang kampong=dering kampong=nang su-jaalang pii. \\
     \textsc{indef} day \textsc{indef} man hat \textsc{inf}-sell=\textsc{dat} village=\textsc{abl} village=\textsc{dat} \textsc{past}-walk go  \\
    `One day, a man walked from village to village to sell hats.'  (K070000wrt01)
\z
}


We have seen in \xref{ex:constr:NP:CLAUSE:finite} that a clause whose verb is inflected with \em anà- \em can be used as an NP. \xref{ex:constr:NP:np:clause:inf:purp} shows the same for a verb in the infinitive. The use of \em arà- \em in its use as simultaneous tense marker is also possible, as the following three examples show.



\xbox{16}{
\ea\label{ex:constr:NP:clause:ara:zero}
\gll Blaakang=jo incayang anà-kuthumung [[moonyeth pada thoppi asà-ambel pohong atthas=ka arà-maayeng]$_{CLS}$]$_{NP}$. \\ % bf
     after=\textsc{emph} \textsc{3s.polite} \textsc{past}-see monkey \textsc{pl} hat \textsc{cp}-take tree top=\textsc{loc} \textsc{simult}-play  \\
    `Then only he saw that the monkeys had taken his hats and were playing on the top of the tree.'  (K070000wrt01)
\z
}


\xbox{16}{
\ea\label{ex:constr:NP:clause:ara:zero2}
\gll Derang su-kuthumung [[ithu buurung=pe kuuku=ka Aajuth asà-sìrrath kìnna arà-duuduk]$_{CLS}$]$_{NP}$. \\ % bf
     \textsc{3pl} \textsc{past}-see \textsc{dist} bird=\textsc{poss} claw=\textsc{loc} dwarf \textsc{cp}-stuck strike \textsc{non.past}-stay  \\
    `They saw that the dwarf sat stuck in the claws of the bird.'  (K070000wrt04)
\z
}


\xbox{16}{
\ea\label{ex:constr:NP:clause:ara:yang}
\ea
\gll Thàrà-kalu [[ini oorang thoppi arà-kumpul]$_{CLS}$]$_{NP}$=\textbf{yang} asà-kuthumung=apa \\
       \textsc{neg}-if \textsc{prox} man hat \textsc{non.past}-collect=\textsc{acc} \textsc{cp}-see=after\\
    `Furthermore, when (they) had seen the man collect the hats'
\ex
\gll moonyeth pada=le asà-dhaathang creeveth  athi-kaasi katha. \\ % bf
       monkey \textsc{pl}=\textsc{addit} \textsc{cp}-come trouble \textsc{irr}=give \textsc{quot}\\
    `the monkeys would certainly go and cause (some other) trouble.'  (K070000wrt01)
\z
\z
}

The use of clauses as NPs where the verb is inflected with \em anà- \em is shown in the following two examples. \em Su- \em is shown in \xref{ex:constr:NP:clause:su}.

\xbox{16}{
\ea\label{ex:constr:NP:clause:ana1}
\gll [[{\em School}=nang   anà-pii]$_{CLS}$]$_{NP}$=nang      blaakang. \\
      school=\textsc{dat} \textsc{past}-go=\textsc{dat} after \\
    `After having gone to school.' (K051222nar04)
\z
}

\xbox{16}{
\ea\label{ex:constr:NP:clause:ana2}
\gll [[Seelong {\em independent} {\em state} anà-jaadi]$_{CLS}$]$_{NP}$=nang=apa  [[kithang=nang  {\em independence} anà-daapath]$_{CLS}$]$_{NP}$=nang=apa. \\
      Ceylon independent state \textsc{past}-become=\textsc{dat}=after \textsc{1pl}=\textsc{dat} independence \textsc{past}-get=\textsc{dat}=after \\
    `After Ceylon had become an independent state, after we had obtained independence.' (K051222nar06)
\z
}


\xbox{16}{
\ea\label{ex:constr:NP:clause:su}
\gll [[Se=ppe    {\em uncle}=ka su-dhaathang]$_{CLS}$]$_{NP}$=nang       blaakang \\ % bf
     \textsc{1s=poss} uncle=\textsc{loc} \textsc{past}-come=\textsc{dat} after  \\
    `After I had come to my uncle's' (K051201nar02)
\z
}

The use of \em asà- \em as inflection on the clause serving as NP is exemplified by \xref{ex:constr:NP:sudabutthulsuuka}. But the use of an uninflected verb as in \xref{ex:constr:NP:clause:uninfl1} \xref{ex:constr:NP:clause:uninfl2} is also possible.


\xbox{16}{
\ea\label{ex:constr:NP:sudabutthulsuuka}
\gll Suda butthul suuka [[asà-dhaathang]$_{CLS}$]$_{NP}$=nang. \\ % bf
 thus very like cp-come=\textsc{dat}\\
`So, I am very pleased that you have come.' (G051222nar01)
\z
}

\xbox{16}{
\ea\label{ex:constr:NP:clause:uninfl1}
\gll [[Manis-an maakang]$_{CLS}$]$_{NP}$=nang go suuka bannyak. \\ % bf
 sweet-\textsc{nmlzr} eat=\textsc{dat} \textsc{1s.familiar} like much\\
`I like very much to eat sweets.' (B060115prs20)
\z
}


\xbox{16}{
\ea\label{ex:constr:NP:clause:uninfl2}
\gll Kithang=nang maau, [[kitham=pe mlaayu looang \zero-blaajar, lorang=pe mlaayu kitham \zero-blaajar]$_{CLS}$]$_{NP}$. \\ % bf
 \textsc{1pl}=\textsc{dat} want \textsc{1pl}=\textsc{poss} Malay \textsc{2pl} learn \textsc{2pl}=\textsc{poss} Malay \textsc{1pl} learn\\
`We want that you learn our Malay and that we learn your Malay.' (K060116nar02)
\z
}



% \xbox{16}{
%  \ea\label{ex:constr:NP:unreferenced}
%    \gll Itthu    kithang=pe     igaama=pe       mosthor=nang   ithu     Hajj  katha arà-biilang    Mecca  arà-pii. \\
%  \textsc{dist} \textsc{1pl}=\textsc{poss} religion=\textsc{poss} manner=\textsc{dat} \textsc{dist} Hajj \textsc{quot} \textsc{non.past}-say Mecca \textsc{non.past}-go\\
% `According to our religion, "Hajj" means to go to Mecca' (B060115cvs01)
% \z
% }

To sum up, we see that clauses which function as an NP can be headed by verbs in various tenses. There seem to be no restrictions on the character of the verb or the TAM-prefix.

\subsection{Headless relative clauses}\label{sec:nppp:Headlessrelativeclauses}
The last possibility to form NPs is the headless relative clause. On the surface, it looks exactly like the argument clause above, but the semantics are different.

\cbx{[[CLAUSE ]\zero]}{NP}

The headless relative clause has the same form as any other relative clause \formref{sec:cls:Relativeclause}. The difference between the two is that the head noun is not expressed. Headless  relative clauses  then do  not have a \em restrictive \em function (among all head nouns, select those which comply with the relative clause), but a \em maximizing \em function (among \em all \em nouns, select those which comply with the relative clause). Since there is no head noun, it is impossible to claim that the headless relative clause is a modifier in the NP; it must be the head. Example \xref{ex:constr:NP:hrelc:relc} gives a normal relative clause with a head. Example \xref{ex:constr:NP:hrelc:hrelc} gives a headless relative clause, which fulfills the function of the NP.

\xbox{16}{
\ea\label{ex:constr:NP:hrelc:relc}
\gll [Lorang anà-maasak ikkang] eenak. \\
     \textsc{2pl} \textsc{past}-cook fish tasty  \\
    `The fish you cooked is tasty.'  (K081105eli02)
\z
}

\xbox{16}{
\ea\label{ex:constr:NP:hrelc:hrelc}
\gll [Lorang anà-maasak \zero] eenak. \\
     \textsc{2pl} \textsc{past}-cook { } tasty  \\
    `What you cooked is tasty.'  (K081105eli02)
\z
}

These NPs formed by headless relative clauses can take case markers, as in \xref{ex:constr:NP:np:hrelc:yang}.

\xbox{16}{
\ea\label{ex:constr:NP:np:hrelc:yang}
\gll [Lorang=ka aada=\zero]=yang kaasi. \\
      \textsc{2pl}=\textsc{loc} exist=\zero=\textsc{acc} give \\
    `Give me whatever you have.'  (K081105eli02)% OK
\z
}

NPs based on headless relative clauses can only be modified by the plural marker \em pada \em \xref{ex:constr:NP:np:hrelc:pada}, but by nothing else.


\xbox{16}{
\ea\label{ex:constr:NP:np:hrelc:pada}
\gll [Lorang=ka aada \zero] pada(=yang) kaasi. \\
      \textsc{2pl}=\textsc{loc} exist { } \textsc{pl}(=\textsc{acc}) give \\
    `Give me all you have.'  (K081105eli02)
\z
}

Examples from the corpus with headless relative clauses as NPs are given in \xref{ex:constr:NP:np:hrelc:nonv} to \xref{ex:constr:NP:np:hrelc:case}.
\xref{ex:constr:NP:np:hrelc:nonv} shows an equational sentence, where the headless relative clause is without any doubt the first term, which is furthermore modified by \em pada\em.

\xbox{16}{
\ea\label{ex:constr:NP:np:hrelc:nonv}
\gll [Seelon=nang anà-dhaathang \zero] pada mlaayu pada. \\
 Ceylon=\textsc{dat} \textsc{past}-come { } \textsc{pl} Malay \textsc{pl}\\
`Those who had come to Ceylon were the Malays.' (N060113nar01)
\z
}

Example \xref{ex:constr:NP:np:hrelc:verbal} shows a parallel construction with two headless relative clauses for verbal predications.


\xbox{16}{
\ea\label{ex:constr:NP:np:hrelc:verbal}
\ea
   \gll [Se=dang nya-boole \zero]    pada see nya-ambel. \\
    \textsc{1s=dat} \textsc{past}-can { }  \textsc{pl} \textsc{1s} \textsc{past}-take \\
`I took what I could.'
\ex
   \gll [Se=dang thàràboole \zero]   pada see thàrà-ambel. \\
    \textsc{1s=dat} cannot { }  \textsc{pl} \textsc{1s} \textsc{neg.past}-take \\
`What I couldn't take, I didn't take.' (K051213nar01)
\z
\z
}

%
% \xbox{16}{
% \ea\label{ex:constr:NP:np:hrelc:nonv2}
% \gll [Andare kanabisan=nang anà-mintha] [hathu raaja=ke asà-paake=apa kampong=nang mà-pii maau katha]. \\
%     Andare last=\textsc{dat} \textsc{past}-ask \textsc{indef} king=\textsc{simil} \textsc{cp}-dress=after village=\textsc{dat} \textsc{inf}-go want \textsc{quot}   \\
%     `What Andare wanted as a last wish, was to go to the village dressed up as a king.'  (K070000wrt03)
% \z
% }\\

Example \xref{ex:constr:NP:np:hrelc:nonv:copula} shows again a non-verbal predication, this time supported by the copula \em asà\-dhaa\-thang\em.

\xbox{16}{
\ea\label{ex:constr:NP:np:hrelc:nonv:copula}
\gll Itthu    vakthu [kithang=nang nya-aada \zero{}]     asàdhaathang ini      JVP katha hathu  {\em problem} hatthu=jo. \\ % bf
      \textsc{dist} time \textsc{1pl}=\textsc{dat} \textsc{past}-exist { } \textsc{copula} \textsc{prox} JVP \textsc{quot} \textsc{indef} problem \textsc{indef}=\textsc{emph} \\
    `What we had at that time was the so-called JVP problem.' (K051206nar10)
\z
}
%
% \xbox{16}{
%  \ea\label{ex:constr:NP:unreferenced}
%    \gll Godang    arà-daapath    go=ppe     aanak pada uuvang godang arà-kiiring. \\
%     \textsc{1s=dat} \textsc{non.past}-get \textsc{1s}=\textsc{poss} child \textsc{pl} wealth \textsc{1s=dat} \textsc{non.past}-send\\
% `I get what my children send me' (B060115nar04)
% \z
% }

Example \xref{ex:constr:NP:np:hrelc:case} shows the use of a headless relative clause marked for case.

\xbox{16}{
\ea\label{ex:constr:NP:np:hrelc:case}
\gll [Derang anà-kuthumung \zero] pada=\textbf{nang} asà-thaakuth  ruuma=nang mà-laari kapang-pii derang=nang byaasa svaara hatthu su-dìnngar. \\
      \textsc{3pl} \textsc{past}-see { } \textsc{pl}=\textsc{dat} \textsc{cp}-fear house=\textsc{dat} \textsc{inf}-run when-go \textsc{3pl}=\textsc{dat} habit sound \textsc{indef} \textsc{past}-hear \\
    `They feared what they saw and when they went running back to their home, they heard a familiar voice.'  (K070000wrt04)
\z
}

The most audacious use of a headless relative clause is probably \xref{ex:constr:NP:np:hrelc:audacious}, where a clause containing the verb \trs{biilang}{say} is used to refer to  the person of the name given as an argument for \em biilang\em.

\xbox{16}{
\ea\label{ex:constr:NP:np:hrelc:audacious}
\gll [Andare katha arà-biilang  \zero{}]  raaja mliiga=ka    \textbf{hathu}  \textbf{oorang} \textbf{koocak}. \\
     Andare \textsc{quot} \textsc{non.past}-say { }] king palace=\textsc{loc} \textsc{indef} man joke \\
    `(The man) called Andare was   jester at the royal palace.'  (K070000wrt05)
\z
}


%\xbox{16}{
%\ea\label{ex:constr:NP:unreferenced}
%\gll Ka-duuva anà-dhaathang {\em slaves} pada. \\
% \textsc{ord}-two \textsc{past}-come slaves \textsc{pl}\\
%`Those arrived second were slaves.' (K060108nar02)
%\z
%}

% \xbox{16}{
% \ea\label{ex:constr:NP:unreferenced}
% \gll [Incayang nya-biilang]=jo buthul katha anà-biilang. \\
%       \textsc{3s.polite} \textsc{past}-say=\textsc{emph} correct \textsc{quot} \textsc{past}-say \\
%     `He said; ``what he said is the correct thing''.' (K061127nar03)
% \z
% } \\


% \section{Apposition of noun phrases}\label{sec:nppp:Appositionofnounphrases}
% Two noun phrases referring to the same entity can be put into apposition.  The following two examples show this. In \xref{ex:constr:NP:app2}, a person is first referred to by his function and then by his name. In \xref{ex:constr:NP:app1}, two people are first referred to by the third person pronoun \em derang \em and then by the noun phrase \trs{duuva oorang}{two people}.
% 
% \xbox{16}{
% \ea\label{ex:constr:NP:app2}
% \gll See su-diya [kithang=pe {\em president}]$_1$ [Mr. Nizam Samath]$_2$ {\em subscription}=yang thàrà-kiiring\\ % bf
%       \textsc{1s} \textsc{past}-see \textsc{1s}=\textsc{poss} president Mr. Nizam Samath subscription=\textsc{acc} \textsc{neg.past} send \\
%     `I saw that our president, Mr. Nizam Samath, had not sent the subscription.' (K060116nar10)
% \z
% } \\
% 
% \xbox{16}{
% \ea\label{ex:constr:NP:app1}
% \gll Derang$_1$  [duuva oorang]$_2$=le asà-kaaving  derang=nang=le aanak pada aada. \\ % bf
%       \textsc{3pl} two man=\textsc{addit} \textsc{cp}-married \textsc{3pl}=\textsc{dat}=\textsc{addit} child \textsc{pl} exist \\
%     `They, the two of them are married and have children.' (B060115prs01)
% \z
% } \\
% 
% % K060116nar07.trs:abisan nya jaadi apa
% % K060116nar07.trs:mr.yusuf
% % K060116nar07.trs:bannyak thuuva oorang nya blaajar oorang
% % K060116nar07.trs:paanjang nang derang duuva oorag yang le asà salba
% 
% 
% These appositions are very similar in form to unmarked coordination \formref{sec:constr:Unmarkedcoordination}, but not in semantics. In \xref{ex:constr:NP:app2}, there is only one human referent in the proposition, whereas if we had an instance of unmarked coordination, there would be more than one referent.

\section{The SLM NP as fundamentally appositional}\label{sec:nppp:TheSLMNPasappositional}
As discussed in Section \ref{sec:nppp:Relativeorderintheprenominalfield}, the order of elements in the SLM NP is quite free. Furthermore, just about any item can constitute a NP on its own, without the need for dummy elements (like English \em a big \textbf{one}\em, \formref{sec:nppp:NPscontaininganadjective}-\formref{sec:nppp:NPscontainingrelativeclauses}. Representing these facts in a hierarchical structure is difficult. Which element should be the head, if any element could be the head? How to account for the many possible permutations? Furthermore, how can one explain the multiple occurrences of \em hatthu \em in one NP \formref{sec:nppp:Thepositionoftheindefinitemodifier}? All this suggests, that a hierarchical structure might not be the best analysis of the SLM NP. If we assume an appositional structure on the other hand, the three problems mentioned above can be resolved \citep[cf.][275]{Rijkhoff2002}.

I will exemplify this with the following example.


\xbox{16}{
\ea\label{ex:nppp:apposition:intro}
\gll [Itthu=le] [hathu  mlaayu oorang pada=pe hathu baae hathu makanan]=jo. \\ % bf
     \textsc{dist}=\textsc{addit} \textsc{indef} Malay man \textsc{pl}=\textsc{poss} \textsc{indef} good \textsc{indef} food=\textsc{emph}  \\
    `That is also a good dish of the Malays.' (K061026rcp02)
\z
}

This example is an ascriptive sentence which attributes to the anaphoric referent \em itthu \em membership in the class of tasty Malay foods. This class of foods is expressed by the string \trs{hathu  mlaayu oorang pada=pe hathu baae hathu makanan}{Malay people's good food}, an NP. We will take this (rather complex) NP as a point of departure for our analysis. In a first step, we will disregard all occurrences of the indefinite article. We will return to it afterwards.


The following two trees show the difference between a hierarchical representation \xref{ex:np-pp:tree:hierarchy} and an appositional representation \xref{ex:np-pp:tree:apposition}.\footnote{The trees follow Givón's (2001) \nocite{Givon2001a,Givon2001b} argument for the usefulness of trees based in the style of \citet{Chomsky1957,Chomsky1965} to represent constituency, hierarchy, category labels, and linear order, in an elegant way. Generative grammar has used, and modified, trees and tree structure in a number of ways since then, but these more recent types of trees would not serve the illustrative purpose they are intended for here any better than their more traditional counterparts, regardless of whether one finds the theoretical arguments for the more modern trees compelling or not.}


\ea \label{ex:np-pp:tree:hierarchy} hierarchical structure

\Tree   [.NP  \qroof{Mlaayu oorang padape}.POSS  [.N' [.ADJ baae ] [.N makanan ]]]\z

\ea  \label{ex:np-pp:tree:apposition} appositional structure

\Tree  [.NP \qroof{Mlaayu oorang padape}.POSS     [.ADJ baae ] [.N makanan ]]\z

The hierarchical structure is more nested, and clear dependency and government relations hold between the nodes of the tree. In the appositional structure, this is not the case. All elements are of equal importance, and none governs another one. Since in SLM, just about anything can head an NP (see above), a difference in prominence of the elements in the NP does not seem to exist.  A theoretical representation which does not imply such difference in prominence, like the appositional structure in \xref{ex:np-pp:tree:apposition}, is superior to a representation which makes additional assumptions, which are unwarranted.

If one node is left out of the appositional structure, the structure remains intact, whereas in \xref{ex:np-pp:tree:hierarchy}, leaving out certain nodes leaves the structure ill-formed.  If the order of elements is changed in \xref{ex:np-pp:tree:hierarchy}, the hierarchy will have to be modified to accommodate the new linear order. In \xref{ex:np-pp:tree:apposition} on the other hand, the structure for the new order will resemble the structure for the former order very much.

Turning to the position of the indefiniteness marker, the multiple occurrences of  \em hatthu \em are difficult to justify within one hierarchical NP. Under the appositional account, multiple occurrences can be explained if we tweak the structure a bit, as show in \xref{ex:np-pp:tree:apposition:tweak}.

\ea \label{ex:np-pp:tree:apposition:tweak} tweaked appositional structure

\Tree  [.NP
	[.NP [.INDEF hatthu ] \qroof{Mlaayu oorang padape}.POSS ]
	[.NP [.INDEF hatthu ] [.ADJ baae ] ]
	[.NP [.INDEF hatthu ] [.N makanan ] ]]\z

Instead of an apposition of POSS, ADJ and N, as in \xref{ex:np-pp:tree:apposition}, we now have an apposition of three NPs, which are in turn headed by said elements. Since we are dealing with three complete NPs (four if we include the overarching NP), the three occurrences of \em hatthu \em are no problem since they modify three different entities.

The above discussion has shown that an appositional representation of the SLM NP can account for the permutation of elements, arbitrary heads, and multiple occurrence of \em hathu\em. These three aspects could not be explained satisfactorily by a hierarchical analysis. There is one thing, however, which posits a problem for the appositional analysis: the obligatory position of the noun in final or prefinal position (if a noun occurs in the NP). It is impossible to have more than one postmodifying element in an SLM NP (excluding \em pada \em for the moment), illustrated in \xref{ex:np-pp:tree:final}. If the structure was fully appositional, this is not what we would expect. The noun should then be able to occur in any position within the NP. I have no good explanation at the moment for this semi-free structure we find in the SLM NP.

\ea \label{ex:np-pp:tree:final}
\Tree   [.NP
	 [.pre  mod mod {...}	 ]
	   [.N ]
	   [.post mod ] 
	]
\z


% 
% 
%  Rather, it is the case that the premodifiers of N are appositional, while N itself is not, nor are its postmodifiers.
% 
%  N' would be appositional while the NP would have some minimal hierarchical structure.\footnote{I would like to thank Peter Austin for this suggestion}
% This is represented schematically in \xref{ex:np-pp:tree:final}. This semi-hierarchical structure is actually not limited to the NP, in \formref{sec:cls:Declarativeclause} we will see that the constituent structure of the clause is also semi-hierarchical, very similar to the structure of the NP.
% 
% \ea \label{ex:np-pp:tree:final}
% \Tree   [.NP
% 	 [.~  PREMOD PREMOD PREMOD	 ]
% 	 [.NP   N POSTMOD ]
% 	]
% \z
%  



\chapter{The postpositional phrase}\label{sec:form:ConstructionsPP}

A postpositional phrase consists of an NP plus a postposition.

\cbx{ NP=POSTP}{PP}


There are no restrictions on the character of the NP (Noun, pronoun, numeral, ...) or the character of the postposition as examples \xref{ex:constr:PP:NPclause} to \xref{ex:constr:PP:DEIC} show, as long as semantic interpretability is possible.

\xbox{16}{
\ea\label{ex:constr:PP:NPclause}
\gll [Seelong$_{NP}$\textbf{=nang} dhaathang]$_{CLAUSE}$\textbf{=nang} \textbf{blaakang}=jo incayang [cinggala asà-blaajar]$_{CLAUSE}$\textbf{=apa} sini=pe raaja$_{NP}$\textbf{=nang} mà-banthu anà-mulain. \\ % bf
     Ceylon=\textsc{dat} come=\textsc{dat} after=\textsc{emph} \textsc{3s.polite} Sinhala \textsc{cp}-learn=after here=\textsc{poss} king=\textsc{dat} \textsc{inf}-help \textsc{past}-start \\
    `It was after that he had come to Ceylon that he learned Sinhala and began to help the local king.'  (K060108nar02)
\z
}


\xbox{16}{
\ea\label{ex:constr:PP:pron}
\gll Incayan$_{PRON}$\textbf{=nang} baae$_{ADJ}$\textbf{=nam} mlaayu mà-oomong butthul suuka. \\ % bf
 \textsc{3s.polite}=\textsc{dat} good=\textsc{dat} Malay \textsc{inf}-speak very like\\
`He likes very much to speak Malay well.' (K051222nar01)
\z
}


\xbox{16}{
\ea\label{ex:constr:PP:NUM}
\gll Samma \textbf{hatthu}$_{NUM}$\textbf{=na} mas-aada. \\ % bf
 all one=\textsc{dat} must-exist\\
`We must all go together.'
\z
}

\xbox{16}{
\ea\label{ex:constr:PP:DEIC}
\gll Se=dang kalu suda bannyak thàrà-thaau  \textbf{inni}$_{DEIC}$=pe         atthas mà-biilang=nang. \\ % bf
     \textsc{1s=dat} if thus much \textsc{neg}-know \textsc{prox=poss} about \textsc{inf}-say=\textsc{dat}  \\
    `So, as for me, I cannot tell you much about this.' (K051205nar04)
\z
}

A relator noun \formref{sec:wc:Relatornouns} can be used instead of a pure postposition \citep[cf.][]{Adelaar2005struct}. The use of the possessive marker \em =pe \em between the host and the relator noun is optional.

\cbx{ NP(=pe) RELN=POSTP
}{PP}




\xbox{16}{
\ea\label{ex:constr:NP:pp:reln:pe}
\gll Andare [[hathu pohong]$_{NP}$=\textbf{pe} \textbf{baava}=ka]$_{PP}$ kapang-duuduk. \\ % bf
     Andare \textsc{indef} tree=\textsc{poss} bottom=\textsc{loc} when-sit  \\
    `When Andare sat down below a tree.' (K070000wrt03)
\z
}

\xbox{16}{
\ea\label{ex:constr:NP:pp:reln:nope}
\gll [[Ini pohong]$_{NP}$=\zero{} \textbf{atthas}=ka]$_{NP}$ moonyeth hathu kavanan su-aada. \\ % bf
     \textsc{prox} tree  top=\textsc{loc}  monkey \textsc{indef} group \textsc{past}-exist\\
    `On top of this tree was a group of monkeys.'   (K070000wrt01)
\z
}

The relator noun can also attach to clausal NPs \xref{ex:constr:NP:reln:clause}.


\xbox{16}{
\ea\label{ex:constr:NP:reln:clause}
\gll [[Mlaayu pada anà-dhaathang]$_{CLS}$]$_{NP}$=\textbf{pe} \textbf{atthas} se=dang hatthu=le mà-biilang thàràboole. \\
       Malay \textsc{pl} \textsc{past}-come=\textsc{poss} about \textsc{1s=dat}  \textsc{indef}=\textsc{addit} \textsc{inf}-say cannot\\
    `I cannot tell you anything about the coming of the Malays.'  (K081105eli02) %OK
\z
}

%
% PPs can be used as NPs, which can in turn host other postpositions, as the following example shows.
%
%
% \xbox{16}{
% \ea\label{ex:constr:NP:PP:recursive}
% \gll [[Inni=yang       mà-peegang]=nang] subla  nigiri=dering   suda anà-dhaathang    inni     {\em forces} pada. \\ % bf
%       \textsc{prox=acc} \textsc{inf}-catch=\textsc{dat}=because country=\textsc{abl} thus \textsc{past}-come \textsc{prox} forces \textsc{pl} \\
%     `So these forces came from all over the country in order to catch him.' (K051206nar02)
% \z
% }
%
% \ea\label{ex:constr:NP:PP:recursive:schema}
% $
% [
%  [
%   [
%    [
%     [
%      [inni=yang~ mà-peegang]_{CLS}
%     ]_{NP}=nang
%    ]_{PP}
%   ]_{NP}
%  ]=subla
% ]_{PP}
% $
% \z

% Postnominal adpositions  and relator nouns are a major typological feature of Sri Lanka Malay \citep[cf.][]{Jayasuriya2002}.



