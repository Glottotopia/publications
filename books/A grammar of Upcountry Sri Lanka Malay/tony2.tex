\section*{Tony Salim's story}
This text is about the earlier history of the Malays, from the very beginning to the present day, including some demographic history and the family history of the speaker, Tony Salim. There is some information about the different generations of Malays, but it is unclear, how the word `generation' is used, and whether there are five, eight, or yet a different number of generations. Finally, there is some material on the Malay saints, in Sri Lanka. This text is shows some features of spoken language, like repetitions, sentence breaks, low coherence, lots of discourse markers, etc., but few of the reductions found in other texts (dropping of prefixes and clitics for instance), which distinguishes it from the next text, \em Saradiyel\em.

\glossSTDmode
\xbox{16}{
\ea
\gll  Seelon=dika karang ka-umpath {\em generation}.\\
      Ceylon=\textsc{loc} now \textsc{ord}-four generation \\
    `Now it's the fourth generation in  Ceylon.'
\z
}


\xbox{16}{
\ea
\gll Ka-umpath {\em generation} katha arà-biilang se=dang duppang lai thiiga {\em generation} pada ini nigiri=ka anà-duuduk.\\
    \textsc{ord}-four generation \textsc{quot} \textsc{past}-say \textsc{1s=dat} front more three generation \textsc{pl} \textsc{dem.prox} country=\textsc{loc} \textsc{past}-exist.\textsc{anim}\\
    `The fourth generation is to say that before me there were three generations that existed in this country.'
\z
}


\xbox{16}{
\ea
\gll  Se=ppe oorang thuuva pada anà-biilang kithang pada  {\em Malaysia}=dering anà-dhaathang katha.\\
      \textsc{1s=poss} man old \textsc{pl} \textsc{past}-say \textsc{1pl} \textsc{pl} Malaysia=\textsc{abl} \textsc{past}-come \textsc{quot} \\
    `My elders said that we had come from  Malaysia.'
\z
}


\xbox{16}{
\ea
\gll   Itthu anà-dhaathang, kithang=pe oorang thuuva pada, samma Seelong=ka see\footnotemark{} arà-duuduk raaja pada=nang mà-banthu=nang.\\
       \textsc{dem.dist} \textsc{past}-come \textsc{1pl}=\textsc{poss} man old \textsc{pl} all Ceylon=\textsc{loc} 1s??? \textsc{non.past}-exist.\textsc{anim} king \textsc{pl}=\textsc{dat} \textsc{inf}-help=\textsc{dat} \\
    `Those who came, our forefathers were all in Ceylon to help the kings.'
\z
}
\footnotetext{The use of \trs{see}{I} here seems to be a performance error. The speaker was born after the last king had already been dead for over a century.}
 

\xbox{16}{
\ea
\gll Derang punnu=le anà-dhaathang mà-pìrrang=nang.\\
     \textsc{3pl} full=\textsc{addit} \textsc{past}-come \textsc{inf}-wage.war=\textsc{dat}\\
    `Most of them came to wage war.'
\z
}


\xbox{16}{
\ea
\gll  Spaaru oorang pada arà-biilang kithang=pe oorang thuuva pada=yang {\em Dutch}  Seelong=nang anà-aaji.baa\footnotemark{} katha.\\
     some man \textsc{pl} \textsc{non.past}-say \textsc{1pl}=\textsc{poss} man old \textsc{pl}=\textsc{acc} Dutch Ceylon=\textsc{dat} \textsc{past}-bring.\textsc{anim} quot\\
    `Some people say our forefathers were brought to Ceylon by the  Dutch.'
\z
}
\footnotetext{\em Aaji baa \em can only be used with animates. \em Aaji \em does not seem to be used outside of this collocation, but \em baa \em means `to bring' when used alone.}



\xbox{16}{
\ea
\gll Spaaru oorang pada arà-biilang Seelong=nang  {\em English} anà-aaji.baa katha.\\
     some man \textsc{pl} \textsc{non.past}-say Ceylon=\textsc{dat} English \textsc{past}-bring.\textsc{anim} \textsc{quot}\\
    `Some people say it is the  English who brought them to Sri Lanka.'
\z
}


\xbox{16}{
\ea
\gll Itthu muusing  Seelong=ka  {\em Dutch}=pe=le  {\em English}=pe=le raame su-aada.\\
     \textsc{dem.dist} time Ceylon=\textsc{loc} Dutch=\textsc{poss}=\textsc{addit} English=\textsc{poss}=\textsc{addit} trouble \textsc{past}-exist \\
    `During that period there was trouble between the  Dutch and the English.'
\z
}


\xbox{16}{
\ea
\gll Itthu raame anà-aada vakthu,  {\em Kandy}=pe raaja=nang kithang=pe ini banthu-an asà-kamauvan su-aada.\\
      \textsc{dem.dist} trouble \textsc{past}-exist time Kandy=\textsc{poss} king=\textsc{dat} \textsc{1pl}=\textsc{poss} \textsc{dem.prox} help-\textsc{nmlzr} \textsc{cp}-need \textsc{past}-exist \\
    `While that trouble took place the king of Kandy needed our help.'\footnotemark
\z
}

\footnotetext{The speakers construes the community of helping Malays as including him as well, although several centuries separate the soldiers and the speaker.}

\xbox{16}{
\ea
\gll  Ini raaja=yang mà-jaaga,\footnotemark\addtocounter{footnote}{-1} itthu mà-jaaga=nang anà-baa mlaayu=dering,\footnotemark\addtocounter{footnote}{-1} anà-aaji.baa mlaayu=dering\footnotemark{} satthu oorang=jo see.\\
      \textsc{dem.prox} king=\textsc{acc} \textsc{inf}-protect \textsc{dem.dist} \textsc{inf}-protect=\textsc{dat} \textsc{past}-bring Malay=\textsc{abl} \textsc{past}-bring.\textsc{anim} Malay=\textsc{abl} one man=\textsc{emph} \textsc{1s}\\
    `Out of the  Malays that were brought to protect this king, I am one.'
\z
}
\footnotetext{Three starts to get the correct phrasing of the complex NP right.}


\xbox{16}{
\ea
\gll Suda, inni, see karang, blaakang=nang laiskali kapang-pii, see ka-umpath {\em generation} katha kapang-biilang.\\
     thus \textsc{dem.prox} \textsc{1s} now after=\textsc{dat} again when-come \textsc{1s} \textsc{ord}-four generation \textsc{quot} when-say\\
    `So, when  I repeat again that I belong to the fourth generation,'
\z
}


\xbox{16}{
\ea
\gll Itthukapang se=ppe baapa, se=ppe kaake, kaake=pe baapa, kithang samma oorang  Seelong=pe oorang pada.\\
     \textsc{dem.dist}=when \textsc{1s=poss} father \textsc{1s=poss} grandfather, grandfather=\textsc{poss} father \textsc{1pl} all man Ceylon=\textsc{poss} man \textsc{pl}\\
    `then my father, my grandfather, great-grandfather, we all are Ceylonese.'
\z
}


\xbox{16}{
\ea
\gll  Ikang, se=ppe kaake=pe baapa=nang sithari=jo se=dang mà-biilang thàrà-thaau.\\
     then \textsc{1s=poss} grandfather=\textsc{poss} father=\textsc{dat} that.side \textsc{1s=dat} \textsc{inf}-say \textsc{neg}-know\\
    `Then,  I cannot tell you anything beyond my great grandfather.'
\z
}


\xbox{16}{
\ea
\gll  Incayang=jo sathu vakthu=nang {\em Malaysia}=dering dhaathang aada.\\
      3\textsc{s.polite}=\textsc{emph} one time=\textsc{dat} Malaysia=\textsc{abl} come exist \\
    `He is the one who would have come from  Malaysia at some point in time.'
\z
}


\xbox{16}{
\ea
\gll Ithu=nang=aapa incayang\footnotemark\addtocounter{footnote}{-1} kapang-dhaathang cinggala incayang=nang\footnotemark{} thàrà-thaau.\\
      \textsc{dem.dist}=\textsc{dat}=after 3\textsc{s.polite} when-come Sinhala 3\textsc{s.polite}=\textsc{dat} \textsc{neg}-know \\
    `But, when he came he didn't know  Sinhala.'
\z
}

\footnotetext{The first \em incayang \em established the topic, while the second one is used to host the semantic role, experiencer, indicated by \em =nang\em, which had been omitted at the first occurrence.}


\xbox{16}{
\ea
\gll Siini Seelong=nang dhaathang=nang blaakang=jo, incayang cinggala asà-blaajar=apa siini=pe raaja=nang mà-banthu anà-mulain.\\
      here Ceylon=\textsc{dat} come=\textsc{dat} after=\textsc{emph} 3\textsc{s.polite} Sinhala \textsc{cp}-learn=after here=\textsc{poss} king=\textsc{dat} \textsc{inf}-help \textsc{past}-start\\
    `Only after having come to Ceylon, he learned  Sinhala and began to help the local king.'\footnotemark
\z
}

\footnotetext{\citet{Hussainmiya1987,Hussainmiya1990} states that the Malays first helped the Dutch. Then, some of them deserted and fought for the Kandyan king. There is no indication of Malays going directly to the Kandyan king.}

\xbox{16}{
\ea
\gll Itthu vakthu ini {\em Malaysia}=ka anà-duuduk milt... {\em military} {\em regiment} {\em Malay} {\em Military} {\em Regiment} katha.\\
      \textsc{dem.dist} time \textsc{dem.prox} Malaysia=\textsc{loc} \textsc{past}-exist.\textsc{anim} milt... military regiment Malay Military Regiment \textsc{quot} \\
    `At that point in time in  Malaysia there was the military regiment called the
 Malay Military Regiment.'\footnotemark
\z
}
\footnotetext{It is probable that this is a reference to the Ceylon Rifle Regiment established in the 19$^{th}$ century by the British in Ceylon.}


\xbox{16}{
\ea
\gll  Itthu {\em regiment}=pe oorang pada=jo..., =nang=jo see=le araa...kumpul-an.\\
      \textsc{dem.dist} regiment=\textsc{poss} man \textsc{pl}=\textsc{emph} =\textsc{dat}=\textsc{emph} \textsc{1s}=\textsc{addit} \textsc{non.past}-gather-\textsc{nmlzr} \\
    `I also belong to the so called military regiment people.'\footnotemark
\z
}
\footnotetext{More precisely, the descendents of the soldiers.}


\xbox{16}{
\ea
\gll Se=ppe=le kumpul-an itthu {\em regiment}=nang=jo.\\
     \textsc{1s=poss}=\textsc{addit} gather-\textsc{nmlzr} \textsc{dem.dist} regiment=\textsc{dat}=\textsc{emph}\\
    `My association is also with that regiment.'
\z
}


\xbox{16}{
\ea
\gll Ikang\footnotemark\addtocounter{footnote}{-1} Seelong=nang dhaathang aada mlaayu oorang ikkang.\footnotemark\\
      then Ceylon=\textsc{dat} come exist Malay man fish \\
    `Then those who came to  Ceylon were the Malay fishermen.'
\z
}
\footnotetext{\trs{ikang (=itthukang)}{then} and \trs{ikkang}{fish} only differ in the gemination of /k/.}



\xbox{16}{
\ea
\gll Derang pada=jo  Hambanthota=ka arà-duuduk, iiya.\\
     \textsc{3pl} \textsc{pl}=\textsc{emph} Hambantota=\textsc{loc} \textsc{non.past}=exist.\textsc{anim}, yes\\
    `They live in  Hambantota.'
\z
}



\xbox{16}{
\ea
\gll Ikang  Seelong=nang lai hathu kavanan anà-dhaathang.\\
     then Ceylon=\textsc{dat} other one group \textsc{past}-come\\
    `Then there came another group to Ceylon.'
\z
}


\xbox{16}{
\ea
\gll  {\em Slave}  {\em Island} subla=ka arà-duuduk.\\
      Slave Island side=\textsc{loc} \textsc{non.past}-stay \\
    `They stay in  Slave Island.'\footnotemark
\z
}
\footnotetext{The name `Slave Island' probably has no relation to the Malay slaves.}

\xbox{16}{
\ea
\gll  Itthu mlaayu, ithu {\em regiment} mlaayu pada=nang mà-banthu anà-dhaathang oorang pada, itthu hathu kavanan laayeng.\\
      \textsc{dem.dist} Malay \textsc{dem.dist} regiment Malay \textsc{pl}=\textsc{dat} \textsc{inf}-help \textsc{past}-come man \textsc{pl} \textsc{dem.dist} \textsc{indef} group different \\
    `Those  Malays came to serve the regiment Malays, that's another group.'
\z
}


\xbox{16}{
\ea
\gll  Karang Kandi=ka arà-duuduk {\em military} {\em regiment} mlaayu=le; hathyang kubbong, kubbong mlaayu.\footnotemark\\
      now Kandy=\textsc{loc} \textsc{non.past}-exist.\textsc{anim} military regiment Malay=\textsc{addit} other garden garden Malay \\
    `Now those who live in  Kandy are the Malays of the Military Regiment and the others, estate, estate Malays.'
\z
}
\footnotetext{\trs{kubbong mlaayu}{estate Malays} is a nonce compound of two nouns, which is right-headed. Established compound, like \trs{oorang ikkang}{fisherman} are left-headed.}

\xbox{16}{
\ea
\gll  Punnu oorang pada anà-dhaathang thee thee daavong maa... thaanàm=nang thee pohong=yang mà-thaanàm=nang.\\
      full man \textsc{pl} \textsc{past}-come tea tea leaf \textsc{inf}-... plant=\textsc{dat} tea tree=\textsc{acc} \textsc{inf}-plant=\textsc{dat} \\
    `Many people came to plant tea leaves, to plant the tea bush.'
\z
}


\xbox{16}{
\ea
\gll Suda see arà-biilang kaarang  Kandi=ka hathu thiga-pulu riibu=kee mlaayu pada arà-duuduk.\\
     thus \textsc{1s} \textsc{non.past}-say now Kandy=\textsc{loc} \textsc{indef} three-ty thousand=\textsc{simil} Malay \textsc{pl} \textsc{non.past}-exist.\textsc{anim}\\
    `So I will tell you now that there could be about 30,000\footnotemark{} Malays living in Kandy.'
\z
}
\footnotetext{The actual number of Malays in Kandy is probably only one tenth of this figure.}


\xbox{16}{
\ea
\gll Libbi mlaayu samma Kluu\u mbu=ka arà-duuduk punnu=le.\\
     remain Malay all Colombo=\textsc{loc} \textsc{non.past}-exist.\textsc{anim} full=\textsc{addit}\\
    `The balance Malays are all living in Colombo.'
\z
}

\xbox{16}{
\ea
\gll  Seelong=ka arà-duuduk hathu ìnnam-pulu liima riibu=ke mlaayu pada.\\
      Ceylon=\textsc{loc} \textsc{non.past}-exist.\textsc{anim} \textsc{indef} six-ty five thousand=\textsc{simil} Malay \textsc{pl} \\
    `The maximum that you will find in Sri Lanka is about sixty-five thousand Malays.'
\z
}


\xbox{16}{
\ea
\gll  Suda kaarang pada kithang sajja=jo asà... libbi arà-duuduk.\\
     thus nom \textsc{pl} \textsc{1pl} only=\textsc{emph} \textsc{cp}-... remain \textsc{non.past}-exist.\textsc{anim}\\
    `So now, its only we who remain in balance together.'
\z
}


\xbox{16}{
\ea
\gll Itthu=sàsaama kithang lai=sangke=le mlaayu arà-oomong.\\
      \textsc{dem.dist}=\textsc{comit} \textsc{1pl} more=until=\textsc{addit} Malay \textsc{non.past}-speak \\
    `We still happen to speak Malay.'
\z
}


\xbox{16}{
\ea
\gll Inni sumbaring oomong kithang laile kithang=pe ithu laama {\em traditions} pada  {\em Malaysia}=ka anà-duuduk mosthor, anà-boole mosthor kithang itthu=yang    itthu mosthor=nang arà-baapi, sumbaring baapi.\\
     \textsc{dem.prox} while speak \textsc{1pl} still \textsc{1pl}=\textsc{poss} \textsc{dem.dist} old traditions \textsc{pl} Malaysia=\textsc{loc} \textsc{past}-exist.\textsc{anim} manner \textsc{past}-can manner \textsc{1pl} \textsc{dem=dist}=\textsc{acc} \textsc{dem.dist} manner=\textsc{dat} \textsc{non.past}-bring while bring.\\
    `While we talk we still stick to those old traditions of ours just as we were in Malaysia, whatever we can preserve we still continue with it in the process.'
\z
}


\xbox{16}{
\ea
\gll  Pake-yan=ka=le punnu bedahan thraa.\\
      dress=\textsc{nmlzr}=\textsc{loc}=\textsc{addit} full difference \textsc{neg} \\
    `Even at our dress there is not much difference.'
\z
}


\xbox{16}{
\ea
\gll Ithee mosthor arà-paake, anà-kamauvan vakthu=nang, kithang=nang\footnotemark\addtocounter{footnote}{-1} itthu mosthor=nang, {\em Malaysian} hathu mosthor=nang kithang=nang\footnotemark{} bole=duuduk.\\
     \textsc{dem.dist} manner \textsc{non.past}-dress \textsc{past}-want time=\textsc{dat} \textsc{1pl}=\textsc{dat} \textsc{dem.dist} manner=\textsc{dat} Malaysian \textsc{indef} manner=\textsc{dat} \textsc{1pl}=\textsc{dat} can-exist.\textsc{anim}\\
    `We still dress like that, when it is needed some times we still can be in the
 Malaysian way.'  
\z
}
\footnotetext{\em kithangnang \em is  present twice in this clause.}


\xbox{16}{
\ea
\gll  Itthu pada samma ithee mosthor aada.\\
     \textsc{dem.dist} \textsc{pl} all \textsc{dem.dist} manner exist\\
    `They all prevail according to the same custom.'
\z
}


\xbox{16}{
\ea
\gll  Suda see bìsaran=nang nyaari arà-biilang: itthu mlaayu mosthor lai=sangke kithang=ka aada.\\
      thus \textsc{1s} pride=\textsc{dat} today \textsc{non.past}-say \textsc{dem.dist} Malay manner other=until \textsc{1pl}=\textsc{loc} exist \\
    `So, today I tell you proudly: we still possess the same old Malay traditions.'
\z
}

\xbox{16}{
\ea
\gll Hathu vakthu=nang kithang cinggala pada samma=le astha-kumpul aada,\\
      one time=\textsc{dat} \textsc{1pl} Sinhala \textsc{pl} all=\textsc{addit} \textsc{cp}-gather exist\\
    `Some times we may have got mixed together with the Sinhalese.'
\z
}


\xbox{16}{
\ea
\gll spaaru {\em Christian} pada astha-kumpul aada, a... mulbar astha-kumpul aada.\\
      some Christian \textsc{pl} \textsc{cp}-gather exist a... Tamil \textsc{cp}-gather exist. \\
    `Some mix together with Christians, or with Tamils.'\footnotemark
\z
}
\footnotetext{The speaker does not mention the Moors, even if in his own family, there are several marriages with Moors.}

\xbox{16}{
\ea
\gll  Itthukapang, kumpul-an=nang=le, punnu mlaayu pada lai  Seelong=ka aada.\\
     then mix-\textsc{nmzlr}=\textsc{dat}=\textsc{addit} full Malay \textsc{pl} more Ceylon=\textsc{addit} exist\\
    `Then, and in spite of all this mixing, there are still a good lot of  Malays in Sri Lanka.'
\z
} 


\xbox{16}{
\ea
\gll  Itthee mlaayu mosthor=nang arà-duuduk.\\
      \textsc{dem.dist} Malay manner=\textsc{dat} \textsc{non.past}-stay \\
    `They still continue to live according to their old customs.'
\z
}


\xbox{16}{
\ea \em A... what more do you want?\z
}


\xbox{16}{
\ea
\gll Se=ppe baapa {\em Royal} {\em Ceylon} {\em Air} {\em Force}=ka anà-bagijja.\footnotemark\\
     \textsc{1s=poss} father Royal Ceylon Air Force=\textsc{loc} \textsc{past}-work\\
    `My father worked in the  Royal Ceylon Air Force.'
\z
}
\footnotetext{\trs{bagijja}{work(v)} seems to be an incorporation of \trs{pukurjan}{work} into the verb \trs{kijja}{make}. The more common order would be \em pukuran anà-gijja\em, with the object preceding the tense marker. But due to its high frequency and strong conceptual unity, the collocation might have become fused \citep[cf.][]{Mithun1984}.}

\xbox{16}{
\ea
\gll  {\em Royal} {\em Ceylon} {\em Air} {\em Force}=ka asà-bagijja itthu=nang blaakang asà-{\em retire} incayang thuju-pul liima thaaun=sangke incayang anà-iidop.\\
      Royal Ceylon Ari Force=\textsc{loc} \textsc{cp}-work \textsc{dem.dist}=\textsc{dat} after \textsc{cp}-retire 3\textsc{s.polite} seven-ty five year=until 3\textsc{s.polite} \textsc{past}-live \\
    `He worked in the R.C.A.F. and then after that, he retired and lived up to 75 years.'
\z
}


\xbox{16}{
\ea
\gll ini\footnotemark=dìkkath=jo anà-mnii\u n\u ggal.\\
       \textsc{dem.prox}=vicinity=\textsc{emph} \textsc{past}-die \\
    `He passed away just recently.'  
\z
} \\
\footnotetext{Note the temporal use of the proximal deictic \em ini \em in this sentence.}

\xbox{16}{
\ea
\gll Suda ithu=nang blaakang see arà-duuduk, se=ppe blaakang arà-dhaathang se=ppe {\em fifth} {\em generation}, se=ppe anak klaaki pada.\\
     Thus \textsc{dem.dist}=\textsc{dat} after \textsc{1s} \textsc{non.past}-stay \textsc{1s=poss} after \textsc{non.past}-come \textsc{1s=poss} fifth generation \textsc{1s=poss} child male \textsc{pl}\\
    `So after that, now, I am staying; following behind me is the fifth generation, my sons.'\footnotemark
\z
}
\footnotetext{It is not clear why the sons are treated as fifth generation here.}

\xbox{16}{
\ea
\gll Suda se=ppe thuuva anak klaaki asàdhaathang dhlapan-blas thaaun.\\
     thus \textsc{1s=poss} old child male \textsc{copula} eight-teen year\\
    `So my eldest son is now eighteen years old.'
\z
}


\xbox{16}{
\ea
\gll Anà-muuda=jo anak klaaki lai=le duuva thaaun.\\
      \textsc{superl}-young=\textsc{emph} child male more=\textsc{addit} two year \\
    `The youngest son is still (only) two years.'
\z
}


\xbox{16}{
\ea
\gll  Se=dang liima anak klaaki pada aada.\\
      \textsc{1s=dat} five child male \textsc{pl} exist \\
    `I have five sons.'
\z
}


\xbox{16}{
\ea
\gll Incayang=yang anà-braanak Udapussallava=dika.\footnotemark\addtocounter{footnote}{-1}\\
     3\textsc{s.polite}=\textsc{acc} \textsc{past}-be.born Udapussallawa=\textsc{loc}\\
    `He was born at  Udapussalawa.'
\z
}


\xbox{16}{
\ea
\gll  Incayang Udupussallava=ka\footnotemark{} anà-braanak.\\
     3\textsc{s.polite} Udapussallawa=\textsc{loc} \textsc{past}-be.born\\
    `He was born in  Udapussalawa.'
\z
}
\footnotetext{The locative enclitic in its uncommon full form \em dika \em and the more common reduced form \em ka \em in the second sentence.}


\xbox{16}{
\ea
\gll Se=ppe kaake asàdhaathang {\em estate} {\em tea} {\em factory} {\em officer}, {\em estate} {\em tea} {\em factory} {\em officer}.\\
      \textsc{1s=poss} grandfather \textsc{copula} estate tea factory officer estate tea factory officer \\
    `My grandfather was an estate tea factory officer.'
\z
} \\

\xbox{16}{
\ea
\gll Itthu=subbath=jo incayang=yang siithu anà-braanak.\\
     \textsc{dem.dist}=because=\textsc{emph} 3\textsc{s.polite}=\textsc{acc} there \textsc{past}-be.born\\
    `Because of that only he was born there.'  
\z
}


\xbox{16}{
\ea
\gll Siithu asà-braanak incayang=pe plajaran\footnotemark=nang incayang Kandi=nang anà-dhaathang.\\
     there \textsc{cp}-be.born 3\textsc{s.polite}=\textsc{poss} education=\textsc{dat} 3\textsc{s.polite} Kandy=\textsc{dat} \textsc{past}-come\\
    `After he was born there, for his studies he came to  Kandy.'
\z
}
\footnotetext{The voiceless stop in \em plajaran \em stems from an old nominalizing circumfix \em p\E r-...-an\em, which is no longer prodcutive today. The unnominalized verb is \trs{blaajar}{learn}, with a voiced stop.}


\xbox{16}{
\ea
\gll  Kandi=nang asà-dhaathang,\footnotemark\addtocounter{footnote}{-1} Kandi=ka asà-kaaving=apa,\footnotemark\addtocounter{footnote}{-1} itthu=nang blaakang\footnotemark=jo kithang pada anà-bìssar.\\
      Kandy=\textsc{dat} \textsc{cp}-come Kandy=\textsc{loc} \textsc{cp}-marry=after \textsc{dem.dist}=\textsc{dat} after=\textsc{emph} \textsc{1pl} \textsc{pl} \textsc{past}-big \\
    `He came to Kandy and after marrying in Kandy, after that, we grew up.'
\z
}
\footnotetext{Note the different strategies for marking of subsequence, \em asà-, asà-...=apa, blaakang\em.}


\xbox{16}{
\ea
\gll Baapa=pe kaake=pe naama asàdha..., baapa=pe baapa=pe naama asàdhaathang T N Salim, Thuan Nahim Salim.\\
     father=\textsc{poss} grandfather=\textsc{poss} name ... father=\textsc{poss} father=\textsc{poss} name \textsc{copula}  T N Salim Thuan Nahim Salim\\
    `Father's, grandfather's name, father's father's name was  T N Salim, Thuan Nahim Salim.'
\z
}


\xbox{16}{
\ea
\gll  Incayang=nang sithari se=dang mà-biilang thàrà-thaau.\\
      3\textsc{s.polite}=\textsc{dat} that.side \textsc{1s=dat} \textsc{inf}-say \textsc{neg}-know \\
    `I don't know anything to say after him any further.'
\z
}


\xbox{16}{
\ea
\gll Incayang=pe baapa=pe naama see thàrà-thaau, iiya a...\\
      3\textsc{s.polite}=\textsc{poss} father=\textsc{poss} name \textsc{1s} \textsc{neg}-know yes a...\\
    `I do not know his father's name, yes.'
\z
}



\xbox{16}{
\ea
\gll Derang pada, kithang=pe oorang thuuva pada bannyak dhaathang aada  {\em Malaysia}=dering.\\
     \textsc{3pl} \textsc{pl} \textsc{1pl}=\textsc{poss} man old \textsc{pl} much come exist Malaysia=\textsc{abl} \\
    `Many of them, of our forefathers came from Malaysia.'
\z
}


\xbox{16}{
\ea
\gll Spaaru Indonesia=dering dhaathang aada.\\
     some Indonesia=\textsc{abl} come exist\\
    `Some came  from  Indonesia.'
\z
} \\


\xbox{16}{
\ea
\gll  Siini dhaathang=nang blaakang, derang kaaving=apa, itthu, ithu maana\~{}maana\footnotemark\addtocounter{footnote}{-1} thumpath katha kithang=nang buthul=nang mà-biilang thàrboole.\\
     here come=\textsc{dat} after \textsc{3pl} marry=after \textsc{dem.dist} \textsc{dem.dist} which\~{}\textsc{red} place \textsc{quot} \textsc{1pl}=\textsc{dat} correct=\textsc{dat} \textsc{inf}-say cannot\\
    `And after they came here, they got married; where all\footnotemark{} they lived, we cannot tell exactly.'
\z
}
\footnotetext{The reduplicated interrogative pronoun indicates the need for an exhaustive answer. This is difficult to render in English}


\xbox{16}{
\ea
\gll  Hatthu se=dang bàrà-biilang: {\em Malaysia} samma oorang=pe naama pada `Maas'.\\
      one \textsc{1s=dat} can-say Malaysia all man=\textsc{poss} name \textsc{pl} Maas \\
    `One thing  I can say: the names of all  Malaysians are ``Maas''.'
\z
}

\xbox{16}{
\ea
\gll Maas=dering=jo arà-{\em start}, derang=pe naama pada samma arà-gijja buthul Maas=dering.\\
     Maas=\textsc{abl}=\textsc{emph} \textsc{non.past}-start \textsc{3pl}=\textsc{poss} name \textsc{pl} all \textsc{non.past}-make correct Maas=\textsc{abl}\\
    `They make their names with the beginning ``Maas''.'
\z
}


\xbox{16}{
\ea
\gll   Indonesia=dering arà-dhaathang naama pada samma `Thuan'.\\
      Indonesia=\textsc{abl} \textsc{non.past}-come name \textsc{pl} all Thuan \\
    `The names which come from Indonesia are all ``Thuan''.'
\z
}


\xbox{16}{
\ea
\gll   Suda itthu=dering=jo kithang=nang ini Indonesia=pe oorang=si giithu kalthraa {\em Malaysian} oorang=si katha bara-thaau ambe, iiya.\\
      thus \textsc{dem.dist}=\textsc{abl}=\textsc{emph} \textsc{1pl}=\textsc{dat} \textsc{dem.prox} Indonesia=\textsc{poss} man=\textsc{interr} that.way otherwise Malaysian man=\textsc{interr} \textsc{quot} can-know take, yes \\
    `Thus, we will come to understand whether this is an Indonesian or otherwise a Malaysian, yeah.'\footnotemark
\z
}
\footnotetext{At least as far as the paternal ancestor is concerned. There is no indication that the different immigrant groups stayed separate. It is likely that most Sri Lankan Malays have both Indonesian and Malaysian blood in them.}

\xbox{16}{
\ea
\gll  Lai aapa lai aapa mà-biilang aada, {\em right}?\\
      more what more what \textsc{inf}-say exist, right \\
    `What else do  I have to say?'
\z
}


\xbox{16}{
\ea
\gll  Seelong=ka arà-duuduk a..., se=dang thaau mosthor=nang, kaarang=nang kà-dhlaapan {\em generation} arà-pii, kà-dhlaapan.\\
       Ceylon=\textsc{loc} \textsc{non.past}-stay a... \textsc{1s=dat} know manner=\textsc{dat} now=\textsc{dat} \textsc{ord}-eight generation \textsc{non.past}-go \textsc{ord}-eight\\
    `staying in Ceylon, er,  as far as I know up to now we are in the eighth generation, eighth.'
\z
}


\xbox{16}{
\ea
\gll  Ka-dhlaapan {\em generation} katha kapang-biilang.\\
      \textsc{ord}-eight generation \textsc{quot} when-say \\
    `When  I say it is the eighth generation, ...'
\z
}


\xbox{16}{
\ea
\gll  Seelong=nang duppang duppang\footnotemark{} anà-dhaathang mlaayu asàdhaathang oorang ikkang, oorang ikkang, {\em drifted} Indonesians pada.\\
      Ceylon=\textsc{dat} front front \textsc{past}-come Malay \textsc{copula} man fish man fish drifted Indonesians \textsc{pl} \\
    `The Malays who came as the very first to Ceylon were fishermen, fishermen, drifted
 Indonesians.'  
\z
}
\footnotetext{Note the reduplication of the temporal adverb to get an intensive reading.}


\xbox{16}{
\ea
\gll Derang pada=jo mlaayu pada, muula anà-dhaathang mlaayu pada.\\
     \textsc{3pl} \textsc{pl}=\textsc{emph} Malay \textsc{pl} before \textsc{past}-come Malay \textsc{pl}.\\
    `They were Malays, the Malays who had come in the beginning.'
\z
}

\xbox{16}{
\ea
\gll Itthu=nang blaakang dhaathang aada {\em as} {\em slaves}.\\
      \textsc{dem.dist}=\textsc{dat} after come exist as slaves \\
    `The ones who came after that were slaves.'
\z
}


\xbox{16}{
\ea
\gll Kà-duuva anà-dhaathang {\em slaves} pada, soojor pada anà-baa oorang pada.\\
     \textsc{ord}-two \textsc{past}-come slaves \textsc{pl} European \textsc{pl} \textsc{past}-bring man \textsc{pl}\\
    `The second ones who came were slaves, people brought by the Europeans.'
\z
}


\xbox{16}{
\ea
\gll Kà-thiiga=joo {\em regiment}.\\
     \textsc{ord}-three=\textsc{emph} regiment.\\
    `The third, soldiers.'
\z
}


\xbox{16}{
\ea
\gll Kà-thiiga anà-dhaathang {\em regiment}.\\
     \textsc{ord}-three \textsc{past}-come regiment.\\
    `The third ones to come were soldiers.'
\z
}


\xbox{16}{
\ea
\gll Ithu=kapang, see arà-{\em belong} kà-ùmpath {\em generation}=nang.\\
      \textsc{dem.dist}=when \textsc{1s} \textsc{non.past}-belong \textsc{ord}-four generation=\textsc{dat} \\
    `Then  I belong to the fourth generation.'
\z
}


\xbox{16}{
\ea
\gll Ikang asàdhaathang kithang {\em delay} aada.\\
     then \textsc{copula} \textsc{1pl} late exist\\
    `Then we were late.'\footnotemark\addtocounter{footnote}{-1}
\z
} 

\xbox{16}{
\ea
\gll Kithang=pe {\em regiment}=yang Seelong=nang mà-dhaathang anà-{\em delay}.\\
     \textsc{1pl}=\textsc{poss} regiment=\textsc{acc} Ceylon-\textsc{dat} \textsc{inf}-come \textsc{past}-late\\
    `Our regiment was late to come to  Ceylon.'\footnotemark
\z
}
\footnotetext{The speaker does/did not belong to the Regiment, but construes it as comprising him as well.}

\xbox{16}{
\ea
\gll Ithu=na=apa kithang=nang duppang lai duuva bergaada dhaathang aada.\\
      \textsc{dem.dist}=\textsc{dat}=after \textsc{1pl}=\textsc{dat} front more two generation come exist \\
    `So before us, there are two other generations.'
\z
}


\xbox{16}{
\ea
\gll Itthu duuva bergaada=jo kà-thaama ikkang, oorang ikkang.\\
      \textsc{dem.dist} two generation=\textsc{emph} first fish man fish \\
    `those two generations, the first were fishermen.'
\z
}


\xbox{16}{
\ea
\gll Kà-duuva=jo {\em slaves} pada.\\
      \textsc{ord}-two=\textsc{emph} slave \textsc{pl}\\
    `The second slaves.'
\z
}


\xbox{16}{
\ea
\gll Itthu=le kithang=pe muusing=ka=jo anà-baa ...\\
      \textsc{dem.dist}=\textsc{addit} \textsc{1pl}=\textsc{poss} time=\textsc{loc}=\textsc{emph} \textsc{past}-bring ...\\
    `But in our time they brought...'
\z
}


\xbox{16}{
\ea
\gll Itthu, kithang, hathu vakthu=nang, kithang=nang duppang dhaathang athi-aada {\em exiles} pada=jo anà-baa.\\
     \textsc{dem.dist} \textsc{1pl} one time=\textsc{dat} \textsc{1pl}=\textsc{dat} before come \textsc{irr}-exist exiles \textsc{pl}=\textsc{emph}  \textsc{past}-bring \\
    `Then, at that time, the exiles, who would have come before us, were brought.'
\z
}


\xbox{16}{
\ea
\gll  Itthu=nang=jo see anà-biilang oorang soojor anà-baa katha, iiya.\\
     \textsc{dem.dist}=\textsc{dat}=\textsc{emph} \textsc{past}-say man European \textsc{past}-bring \textsc{quot} yes\\
    `That's why I say that we were brought by Europeans.'
\z
}

\xbox{16}{
\ea
\gll Suda itthu oorang pada=le  Seelong=ka arà-duuduk.\\
      thus \textsc{dem.dist} man \textsc{pl}=\textsc{addit} Ceylon=\textsc{loc} \textsc{non.past}-exist.\textsc{anim}\\
    `So these people are staying in Ceylon as well.'
\z
}


\xbox{16}{
\ea
\gll Derang pada bannyak bisaran oorang pada.\\
      \textsc{3pl} \textsc{pl} much pride man \textsc{pl} \\
    `They are very proud people.'
\z
} 


\xbox{16}{
\ea
\gll Konnyong=ke thaakuth thraa.\\
     few=\textsc{simil} fear \textsc{neg}\\
    `They are not afraid of the slightest thing.'
\z
}


\xbox{16}{
\ea
\gll  Konnyong katha arà-biilang sdiikith.\\
      konnyong \textsc{quot} \textsc{non.past}-say few \\
    ` {\em konnyong} is to say ``few''.'
\z
}


\xbox{16}{
\ea
\gll  Thaakuth thraa.\\
       fear \textsc{neg}\\
    `They had no fear.'  
\z
}


\xbox{16}{
\ea
\gll  Punnu buthul braani, kalbu braani.\\
      full correct brave, mind brave \\
    `They are very brave.'
\z
}
 

\xbox{16}{
\ea
\gll Itthu anà-dhaathang, derang pada=ka=jo Seelong=ka avuliya pada.\\
      \textsc{dem.dist} \textsc{past}-come \textsc{3pl} \textsc{pl}=\textsc{loc}=\textsc{emph} Ceylon=\textsc{loc} saint \textsc{pl}\\
    `Among the one who came, among them,  there are a lot of saints.'
\z
}



\xbox{16}{
\ea    You call them saints  iiya...  \z
}  

\xbox{16}{
\ea
\gll  Mlaayu=ka=jo  bannyak avuliya  Seelong=ka aada.\\
      Malay=\textsc{loc}=\textsc{emph} much saint \\
    `There are many saints among the Malays in Sri Lanka.'  
\z
} \\


\xbox{16}{
\ea
\gll  See laskali arà-biilang itthu {\em saints} pada, itthu=ka bàrnaama anà-pii {\em saints} pada=jo, sudaara thuuju.\\
       \textsc{1s} again \textsc{non.past}-say \textsc{dem.dist} saints \textsc{pl} \textsc{dem.dist}=\textsc{loc} famous \textsc{non.past}-go saint \textsc{pl}=\textsc{emph} brother seven\\
    `I say again that they are saints.  There are saints who have become famous, seven siblings.'
\z
}

\xbox{16}{
\ea
\gll Thuan Thungku, Thuan Idriis, Thuan Skiilan.\\
     Thuan Thungku Thuan Idriis Thuan Skiilan\\
    `Thuan Thungku, Thuan Idriis, Thuan Skiilan'  
\z
}


\xbox{16}{
\ea
\gll Hathyang Thuan Kuddhuus su-biilang, thraa Thuan Kuddhuus Thuan Thungku Thuan Skiilan Thuan Idriis.\\
     other Thuan Kuddhuus su-biilang, \textsc{neg} Thuan Kuddhuus, Thuan Thungku Thuan Skiilan Thuan Idriis\\
    `another one is called Thuan Kuddhuus, no, Thuan Kuddhuus, Thuan Thungku, Thuan Idriis, Thuan Skiilan.'\footnotemark
\z
}
\footnotetext{The speaker and the people around are not too sure about the names and deeds of the saints, and some confusion arises.}

\xbox{16}{
\ea
\gll  Thuan ... {\em uncle} hathyang naama pada saapa {\em uncle}.\\
      Thuan ... uncle other name \textsc{pl} who uncle \\
    `uncle, who(=what) are the other's names?'
\z
}


\xbox{16}{
\ea
\gll  Binthan {\em auntie}=ka caanya, Binthan {\em auntie}=yang konnyong panggel, iiya, Faathima Naaciaar, Faathima Naaciaar, iiya.\\
     Binthan auntie=\textsc{loc} ask Binthan auntie=\textsc{acc} few call yes, Faathima Naaciaar Faathima Naaciaar, yes \\
    `Ask auntie Binthan, call  Binthan auntie yeah,  Fatima.'
\z
}


\xbox{16}{
\ea
\gll  Faathimaa Naaciaar=le hathyang thiiga oorang thiiga oorang laskali  {\em Malaysia}=nang su-baapi.\\
      Faathimaa Naaciaar=\textsc{addit} other three man three man again Malaysia=\textsc{dat} \textsc{past}-take.away \\
    `Naciar, and another three men.  Three other men were brought to Malaysia.'
\z
}


\xbox{16}{
\ea
\gll  Seelong=ka su-mnii\u n\u ggal, ithu thiiga oorang=pe naama kithang thàrà-thaau.\\
      Ceylon=\textsc{loc} \textsc{past}-die \textsc{dem.dist} three man=\textsc{poss} name \textsc{1pl} \textsc{neg}-know \\
    `They died in  Ceylon, but we do not know those three men's names.'
\z
}


\xbox{16}{
\ea
\gll Derang samma oorang pada {\em saints} pada anà-jaadi.\\
     \textsc{3pl} all man \textsc{pl} saint \textsc{pl} \textsc{past}-become\\
    `Those people all became saints.'
\z
}

\xbox{16}{
\ea
\gll Aapa derang pada sgiithu braani, derang pada samma raaja aanak pada, samma avuliya anà-jaadi.\\
      what \textsc{3pl} \textsc{pl} that.much brave \textsc{3pl} \textsc{pl} all king child \textsc{pl} all saint \textsc{past}-become \\
    `They were so brave, they were all princes, they all became saints.'
\z
}

\xbox{16}{
\ea
\gll  {\em right}, hatthu avuliya aada kithang=pe ruuma dìkkath.\\
     right  one saint exist \textsc{1pl}=\textsc{poss} house vicinity\\
    `There is one saint close to our house.'
\z
}



\xbox{16}{
\ea
\gll Itthu Thuan Skiilan, lai hatthu Avuliya su-aada Kandi {\em town}=ka, karang {\em Malay} {\em Mosque} katha arà-biilang.\\
      \textsc{dem.dist} Thuan Skiilan more \textsc{indef} saint \textsc{past}-exist Kandy downtown=\textsc{loc} now Malay Mosque \textsc{quot} \textsc{non.past}-say \\
    `That's Thuan Skiilan.  There is another saint in Kandy town, they call it now Malay mosque.'
\z
}


\xbox{16}{
\ea
\gll Kandi=ka {\em Malay} {\em Mosque}=pe blaakang=ka incayang=pe zihaarath aada.\\
      Kandy=\textsc{loc} Malay Mosque=\textsc{poss} behind=\textsc{loc} 3\textsc{s.polite}=\textsc{poss} shrine exist \\
    `In Kandy, behind the Malay Mosque, there is his shrine.'
\z
}


\xbox{16}{
\ea
\gll Ikang derang=pe sudaara pompang=jo aada Hanthane=ka.\\
     then \textsc{3pl}=\textsc{poss} sibling female=\textsc{emph} exist Hanthane=\textsc{loc}\\
    `Then their sister is in Hanthane.'
\z
}


\xbox{16}{
\ea
\gll  Hanthane, iiya, guunung=ka aada.\\
      Hanthane yes mountain=\textsc{loc} exist\\
    `Yeah, in Hanthane, on the mountain.'
\z
}


\xbox{16}{
\ea
\gll Lai hatthu  avuliya=pe aada Kluu\u mbu Dematamaram=ka, iiya.\\
       more \textsc{indef}   saint=\textsc{poss} exist Colombo Dematamaram=\textsc{loc} yes\\
    `Ehm, there is another saint in Colombo, in Dematamaram yes.'
\z
} \\


\xbox{16}{
\ea
\gll Derang ùmpath oorang=pe=jo karang Seelong=ka aada.\\
      \textsc{3pl} four man=\textsc{poss}=\textsc{emph} now Ceylon=\textsc{loc} exist\\
    `These  four saints' are now in  Ceylon.'
\z
}


\xbox{16}{
\ea
\gll Hathyang thiiga oorang=yang a... {\em Malaysia}=nang su-ambel baapi iiya.\\
      other three man=\textsc{acc} a... Malaysia=\textsc{dat} \textsc{past}-take take.away yes\\
    `Three others were brought to  Malaysia.'
\z
} \\

