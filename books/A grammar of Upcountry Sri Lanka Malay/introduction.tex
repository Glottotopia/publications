 \chapter{Introduction}
\section{Overview}
This book is a description of Sri Lanka Malay. As its name indicates, the language is spoken in Sri Lanka. Contrary to what the name indicates, it is not a dialect of Standard Malay but a language in its own right \citep{Adelaar1991}. It differs greatly from Standard Malay in many aspects, for example in its phonological inventory, basic constituent order, tense system, and case system, to the extent that Standard Malay and Sri Lanka Malay are  mutually unintelligible.

While the lexical material is overwhelmingly of Malay origin, the grammatical structure is heavily influenced by the contact languages Sinhala (Indo-Aryan) and Tamil (Dravidian). Sri Lanka Malay can thus be said to have three `parents' contrary to many other languages, which are clearly traceable to one parent language only.

The speakers of Sri Lanka Malay are the descendants of immigrants from what are today Indonesia and Malaysia, mainly brought by the colonial powers of the Dutch and the British between 1650 and 1850 as soldiers, exiles, convicts, and slaves. The soldiers were the most important contingent and have shaped the Sri Lankan Malay society, whose members mainly took up profession in the military, the police and related professions until very recently. During the 19th century, the Ceylon Rifle Regiment consisted exclusively of Malays, and most of the island's Malays were associated in one way or another with this regiment.

During the colonial period, the Sri Lankan Malays had close ties with the colonial powers. After the disbandment of the regiment in 1873 and the independence of Ceylon in 1948, the Malays lost the close ties with the ruling elite and saw their social position deteriorate. Nationalist policies in the aftermath of the independence led to the decline of minority languages. According to the 2001 census, there are 48.000 ethnic Malays in Sri Lanka \citep{SmithEtAl2006cll},  not all of which speak the language. Sinhala and English are replacing Sri Lanka Malay as the language of child care. The main concentrations of Malays are found in Colombo, the Upcountry towns, and the Southern settlements of Hambantota and Kirinda. The variety described in this grammar is the Upcountry variety. There seem to be some phonological differences between this variety and the South. As for the Malay spoken in Colombo, it seems very similar to the variety described in this grammar, but I do not have enough information to establish that they can indeed be regarded as the same.\footnote{\citet[34]{Saldin2001} sees all three as different dialects.}

\begin{figure}[p]
% \includegraphics[width=\textwidth]{pics/sloverview.eps}
% malaycomp.eps: 1179666x1179666 pixel, 300dpi, 9987.84x9987.84 cm, bb=14 14 846 1365
\end{figure}

Sri Lanka Malay is characterized by the left-branching word order typical of South Asia, with the arguments preceding the verb, all modifiers normally preceding the noun, and postpositions.
Compared to Malay varieties elsewhere, Sri Lanka Malay has a more elaborate morphology: there are more than a dozen verbal prefixes. Closely related to this is the presence of a large number of enclitics, including more than ten case marking postpositions.
Sri Lanka Malay differs from European languages in that a number of notions common in the latter are absent in Sri Lanka Malay. These absent categories include `stress', `subject', and `conjunction'.

\section{Organization of this book}
This grammatical description is divided in three parts. The first part covers sociohistorical background and methodological assumptions. The second part takes a form-to-function approach and describes  grammatical morphemes and  constructions found in Sri Lanka Malay, and what functions they serve. The third part takes a function-to-form approach and describes how Sri Lanka Malay can communicate content, e.g. referring to time, space, entities etc., and what morphemes and constructions are used for that. The two latter parts heavily cross-reference each other.\footnote{A theoretical justification for this approach can be found in \citet{Nordhoff2008jldc}.} The appendix contains five texts by different speakers/writers, next to some other additional material.


 
 

% 
% \begin{quote}
% The island covers a total area of 65.610 km^{[2]}. It stretches
% 435 km firm south to north at its longest and 225km a t its widest
% from east to west. Located at the extreme southern tip of the
% Indian Subcontinent [\dots] the island is within latitudes
% 6\textdegree to 10\textdegree N. \citet[15]{Hussainmiya1990}
% \end{quote}
 
\section{History of research}
Little research had been undertaken on the Malay population in Sri Lanka until the publication of \citet{Bichsel} and \citet{Hussainmiya1987,Hussainmiya1990}. \citet[7,22]{Hussainmiya1990} informs us that
Goonetileke's  bibliography on Ceylon \nocite{Goonetileke19701983}
 only cites nine articles whatsoever that treat Malay issues. These are
mostly short (4-5 pages) and not scientific. 
References to Malays in general Sri Lankan history books are limited to a few sentences \citep{Bichsel}. 

\citet{Bichsel} and \citet{Tapovanaye1986,Tapovanaye1995} treated phonology, while diachronic issues became more important in the 1990s \citep{Adelaar1991,Bakker1995nl,Bakker1996stuf,Bakker2000convergence,Bakker2000rapid,Bakker2006}. Since the turn of the millennium, Sri Lanka Malay has received considerable scholarly attention by Umberto Ansaldo, Peter Bakker,  BA Hussainmiya,  Scott Paauw,  Shihan De Silva Jayasuriya, Peter Slomanson, and Ian Smith. This research has  mainly focused on morphosyntax and diachrony. See the `References'  section in the appendix.

\section{Terminology}
There are a number of terminological pitfalls to deal with in the description of Sri Lanka Malay. The smallest one is whether to use Sri Lanka or Sri Lankan Malay with or without `n'. In this grammar, I use `Sri Lanka Malay' without `n', but I have no strong preference. The `Malay' part is more problematic, since the immigrants   belonged to a large variety of the ethnic groups of what today is Indonesia, and only a minority were Malays. The Dutch actually referred to the immigrants as \trs{Oosterlingen}{Easterners} or \trs{Javanen}{Javanese}, the latter because the port of departure was Batavia in Java. This reference to Java is still found today in the other Sri Lankan languages, which use \em J\=a Minissu \em (Sinhala), \em C\=avakar \em (Tamil) or   \em J\=av\=a Manusar \em  (Moorish Tamil) \citep{Hussainmiya1990}. The British however were not aware of the ethnic origin of the troops they took over from the Dutch, and named them after the language they spoke, which was based on Malay. This usage is still the norm today, and has actually made its way back into the Malay community at least in the Upcountry, where \trs{mlaayu}{Malay} is the normal way to designate the Sri Lanka Malays. \citet{Ansaldo2005ms} reports that in the South, Sri Lanka Malays use the term \em Orang Java. \em He uses \em Kirinda Java \em to refer to the language. The term \em Java \em  is not found in the Upcountry, so  the adoption of this practice is not possible. The academic tradition has been Sri Lanka(n) Malay (SLM), and I will stick to that tradition.

The island where the language is spoken has had several names in the course history, of which `Ceylon' and `Sri Lanka' are the most widely known. The Sri Lankan Malays refer to the island as \em Seelon\em. `Ceylon' and `Sri Lanka' are in complementary distribution as far as historical periods are concerned. Until 1972, the island was known to the Europeans as Ceylon, or related names in other European languages. With the creation of the republic in 1972, the term `Sri Lanka' was coined. While `Lanka' and related forms are local designations of the island, the `Sri' is an addition of the late 20th century.\footnote{Furthermore, the addition of `Sri' is occasionally seen as an expression of Sinhalese nationalism. This is due to a law from 1958 requiring to include the Sinhalese character \trs{{\SHb\char53iir} }{\'sr\=\i} on number plates of cars. Tamils opposed this law and painted over the {\SHb\char53iir}. The issue escalated and led to riots with 400 casualties and 12,000 homeless people, mostly among Tamils \citep[35f]{NissanEtAl1990}.} The use of `Ceylon' for the republic is obviously anachronistic, but the use of Sri Lanka for the historical period is anachronistic as well. I will therefore use `Ceylon' for the historical period,\footnote{One could argue that `Ceylon' is colonialist, but since the Sri Lankan Malays use a related term, \em Seelon\em, this argument does not hold in their context.} and `Sri Lanka' for the modern period, and for discussions which do not refer to any period in particular.

It is often necessary to refer to the places of origin of the Malay speakers. As of today, two main countries occupy the relevant area, Malaysia and Indonesia, but these countries did not exist during the relevant period. It is often not possible or desirable to make a difference between what today is administrated by Malaysia and what is administrated by Indonesia. I will use the term ``South East Asian Archipelago'' (SEAA) to cover to the whole geographical region, without implying any national borders which may or may not have been relevant at a certain point of history. I occasionally also use the term ``Nusantara'' to refer to that region.