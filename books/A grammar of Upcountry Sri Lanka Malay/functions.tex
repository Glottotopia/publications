


% \begin{motto}
% Es gibt kein sprachliches Mittel f\"ur jede Situation, aber f\"ur  jede Situation ein Mittel\citep[202]{Keller1990}
% \end{motto}


In the morphosyntactic part of the book, we discussed how SLM can construct well-formed sentences. For every morpheme and every construction, it was shortly described what functions it could fulfill. That is, we adopted a form-to-function (`semasiological') approach In this chapter we will switch viewpoint to a function-to-form (`onomasiological') approach, and investigate what means SLM uses to encode universal functional domains \citep{Givon2001a,Givon2001b}.\footnote{The ordering
	of sections is loosely inspired by Functional Discourse Grammar \citep{HengeveldEtAl2008fdg}.}
Since form and function are closely interrelated, there will also be a lot of references to the formal chapter. These references are indicated by a boxed arrow $\boxdotright$. References to other functional domains are indicated by a circled arrow $\circledotright$.

We will first address how propositional content relating to, among other things, participants, states-of-affairs, space, and time can be encoded \funcref{sec:func:FunctionalDomains}. Then we will turn to the encoding of information structure \formref{sec:Informationflow}, and to speech acts \formref{sec:Pragmatics}.

\chapter{Encoding propositional content}\label{sec:func:FunctionalDomains}
In this chapter, we will deal with the factual semantic content of the message  that speakers want to transmit to the hearer, i.e. by what means different aspects of the proposition (participants, space, time) are encoded in SLM. The next chapters will deal with the packaging of said information \funcref{sec:Informationflow}.

To return to the semantic content, which is the topic of this chapter: We will first take a look at how participants in propositions are encoded in SLM
 \funcref{sec:func:Particpants}.
Then we will turn to predication
\funcref{sec:func:Predication},
and modification \funcref{sec:func:Modification}.
After these nuclear elements, we will discuss the encoding of space \funcref{sec:func:Space} and time \funcref{sec:func:Time}. A number of functional domains which have received some typological interests are discussed in the remainder of this chapter. These are
\begin{itemize}
 \item quantity \funcref{sec:func:Quantity},
 \item modality \funcref{sec:func:Modality},
 \item conditionals \funcref{sec:func:Conditionals},
 \item gradation \funcref{sec:func:Gradation},
 \item comparison \funcref{sec:func:Comparison},
 \item possession \funcref{sec:func:Possession},
 \item negation \funcref{sec:func:Negation}, and
 \item kin \funcref{sec:func:Kin}.
\end{itemize}



\section{Particpants}\label{sec:func:Particpants}
The participants of a proposition are the entities which participate in the state or event. In the sentence \em Peter gives Mary the book\em, \em Peter, Mary \em and \em the book \em are the three participating entities. Entities can be of different orders, which will be discussed in \funcref{sec:func:Participantsofdifferententityorders}. These participants can fulfill different roles in the proposition, like \textsc{agent}, \textsc{patient} or \textsc{instrument}. These will be discussed in \funcref{sec:func:Participantroles}. Normally, the number of participants and semantic roles are equal and every participant has one role. Sometimes this is not the case, for instance in \em Peter cut himself\em, there is only one participant, \em Peter\em, but there are two roles, Agent and Patient. These cases are discussed in \funcref{sec:func:Mismatchesbetweennumberofsemanticrolesandnumberofsyntacticarguments}. In yet other cases, one wants to include participants in the proposition whose reference is not established, like \em someone \em or \em whoever\em. These cases are discussed in \funcref{sec:func:Unknownparticipants}. Finally, participants can be modified in a number of ways. These modifications are discussed in \funcref{sec:func:Modifyingparticipants}.

\subsection{Participants of different entity orders}\label{sec:func:Participantsofdifferententityorders}
We can distinguish between the following entity orders \citep{Lyons1977,Hengeveld1992nvpttd,Hengeveld2004soac,Keizer1992,Dik1997,HengeveldEtAl2008fdg}.

\begin{itemize}
 \item properties f  
 \item individuals x  
 \item states-of-affairs e  
 \item propositional content p  
 \item utterances U  
\end{itemize}

In SLM, all of these can be participants. We will treat them in turn.


\subsubsection{Properties}\label{sec:func:Properties}
Properties are distinguished from individuals by not having an existence of their own. They are furthermore distinguished from states-of-affairs by not being located in space and time.

Properties can be used as participants. An example is the property `amount', which participates in the state of ignoring, \em thàràthaau\em, in \xref{ex:func:ptcpt:prop}.

\xbox{16}{
\ea\label{ex:func:ptcpt:prop}
\gll Se=dang kalu [\textbf{blaangang}] thàrà-thaau. \\
    \textsc{1s=dat} if number \textsc{neg}-know   \\
    `As for me, I do not know the number (of the descendants).' (K051205nar05)
\z
}

Often, properties are derived with the nominalizer \em -an \em \formref{sec:morph:-an} when they are to be used as participants, as in \xref{ex:func:ptcpt:prop:an:tsunami}.

\xbox{16}{
\ea\label{ex:func:ptcpt:prop:an:tsunami}
\gll Tsunaami anà-jaadi     \textbf{cupath-an}=nang kithang=nang   thàrà-thaau. \\
      tsunami \textsc{past}-become quick-\textsc{nmlzr} \textsc{1s=dat} \textsc{neg}-know \\
    `We did not know about the speed with which the tsunami came.' (B060115nar02)
\z
}

Note that in \xref{ex:func:ptcpt:prop:an:tsunami}, the property is modified by \trs{tsunaami anà-jaadi}{with which the tsunami came}. While the coming of the tsunami is of course localizable in space and time, the property \trs{cupathan}{speed} is not. It is impossible to say \em the speed was at the coast at 7 a.m.\em.



%
% K060103nar01.txt:\tx ini      baarang pada=yang      asà-baapi   laayeng nigiri=ka     anà-juuval
% K060103nar01.txt:\tx blaangan arga=nang        anà-juuval



\subsubsection{Individuals}\label{sec:func:Individuals}
The prototypical participant is an individual. Participants of the individual type are encoded as a noun phrase, which is headed by a noun or a pronoun, as the case may be. In example \xref{ex:func:ptcpt:ent:indiv}, there are two individual participants, a thief, who is encoded by a noun phrase headed by the third person pronoun \em dee \em and the victims of the thief, who are also head of a noun phrase.


\xbox{16}{
\ea \label{ex:func:ptcpt:ent:indiv}
\gll [\textbf{Dee}] [\textbf{oorang} \textbf{pada}]=nang arà-cuuri. \\
      3 man \textsc{pl}=\textsc{dat} \textsc{non.past}-steal \\
    `He steals from the people.'
% \ex
% \gll Oorang=pe baarang pada samma arà-cuuri. \\
%     man=\textsc{poss} goods \textsc{pl} all \textsc{non.past}-steal\\
%  `He steals all the peoples' goods. (K051205nar02)
% \z
\z
}



\subsubsection{States-of-affairs}\label{sec:func:States-of-affairs}
States of affairs are second-order entities, which have temporal extension. The morphosyntactic expression of a state-of-affairs as a referent depends on its realization status. If the state-of-affairs is realized, a finite subordinate clause is used, as in \xref{ex:ptcpt:ent:soa:cl:realis}, where the writing of the letter has already happened. The past tense prefix \em anà- \em indicates that the subordinate clause is finite.

\xbox{16}{
\ea \label{ex:ptcpt:ent:soa:cl:realis}
\gll [[Lorang suurath=yang  mlaayu=dering \textbf{anà}-thuulis]=nang bannyak arà-suuka]. \\
       \textsc{2pl} letter=\textsc{acc} Malay=\textsc{abl} past=write=\textsc{dat} much simult-like \\
    `He liked very much that you wrote the letter in Malay.'  (Letter 26.06.2007)
\z
}

If the state-of-affairs is not realized yet, the subordinate clause will not be finite but be marked with the infinitive \em mà-\em.\footnote{Or the negative infinitive \em jamà-\em.} In example \xref{ex:ptcpt:ent:soa:cl:irrealis}, the predicate \trs{suuka}{like} has two participants, a liker and a likee. The liker is an individual, which is encoded by the first person pronoun \em se(dang)\em, while the likee is a state-of-affairs, meeting the wife. This state-of-affairs is not realized yet. As a consequence, the verbal predicate carries the infinitive marker \em mà- \em \citep[cf.][95ff]{Noonan1985}.

\xbox{16}{
\ea \label{ex:ptcpt:ent:soa:cl:irrealis}
\gll Se=dang Andare=pe biini=yang \textbf{mà-caa\u nda} suuka. \\
      \textsc{1s=dat} Andare=\textsc{poss} wife=\textsc{acc} \textsc{inf}-meet like \\
    `I would like to meet your wife, Andare.' (K070000wrt05)
\z
}


If the state-of-affairs is the topic, it can receive the postposition \em =yang \em \formref{sec:morph:=yang} \xref{ex:ptcpt:ent:soa:cl:yang}.

\xbox{16}{
\ea \label{ex:ptcpt:ent:soa:cl:yang}
\gll [Se   arà-maakang]$_{cls}$=\textbf{yang}  lorang=nang atthu creeveth=si? \\
     \textsc{1s} \textsc{non.past}-eat=\textsc{acc} \textsc{2s}=\textsc{dat}  \textsc{indef} problem=\textsc{interr}  \\
    `Do you have a problem with my eating?'   (K081103eli04)
\z      
} 

For some conventionalized states-of-affairs a lexical solution (i.e. a noun like \trs{pìrrang}{war})  is available.




% A state-of-affairs like `war' can also be expressed by a noun, in this case \em pìrrang\em.
%
% \xbox{16}{
% \ea \label{ex:ptcpt:ent:soa:n}
% \gll Derang bannyak \textbf{pìrrang}=ka=jo arà-buunung. \\
%       \textsc{3pl} many war=\textsc{loc}=\textsc{emph} \textsc{non.past}-kill\\
%     `They killed many in the war.' (K051213nar06)
% \z
% } \\






% \xbox{16}{
% \ea \label{ex:ptcpt:ent:soa:subord}
% \gll Itthusubbath=jo incayang=nang, ini sri Lankan {\em Malay} mà-blaajar maau. \\
%   therefore=\textsc{emph} 3\textsc{s.polite}=\textsc{dat} \textsc{prox} Sri Lankan Malay \textsc{inf}-learn want \textsc{quot} \\
% `This is why he wants to learn this Sri Lanka Malay.' (B060115prs15)
% \z
% }



\subsubsection{Propositional content}\label{sec:func:Propositionalcontent}
The difference between states-of-affairs and propositions is that the  former are located in space in time, but have no truth value, while the latter are not located in space and time, but do have a truth value. Propositional content can be true or false, and asserted and denied; this is not possible for states-of-affairs. Propositional content is normally indicated by the quotative \em katha\em, as in the following example, where people discover that they like politics. However, their liking of politics could be an illusion, and the positive truth value asserted here could turn out to be negative. We are thus dealing with propositional content here, and not with a state-of-affairs.

 \xbox{16}{
\ea \label{ex:ptcpt:ent:prop:nclause:politic}
   \gll Itthukapang=jo       derang  nya-thaau   ambel [derang pada {\em politic}=nang   suuka] katha. \\
    then=\textsc{emph} \textsc{3pl}  \textsc{past}-know take \textsc{3pl} \textsc{pl} politic=\textsc{dat} like  \textsc{quot}\\
`Only then will they come to know that they like politics' (K051206nar12)
\z
}

The quotative marker is not necessary, as is shown in the following example, where the speaker asserts that the president had not sent the subscription money, but again, he could be mistaken and the negative truth value he assigns to the proposition could turn out to be false.

\xbox{16}{
\ea \label{ex:ptcpt:ent:prop:nclause:president}
\gll See su-diya [kithang=pe {\em president}  {\em subscription}=yang thàrà-kiiring] \zero{}. \\ % bf
      \textsc{1s} \textsc{past}-see \textsc{1s=poss} president   subscription=\textsc{acc} \textsc{neg.past}-send { } \\
    `I saw that our president had not sent the subscription.' (K060116nar10)
\z
}

% Another example of a property being used as a participant is \trs{suuka}{wishing}  in \xref{ex:func:ptcpt:prop:an:sukahan}. This property is used referentially, as is clear from its possessive premodification \trs{Andarepe}{Andare's}. Again, the nominalizer \em -an \em is used here to make the word \em suuka \em denoting a property fit in the argument structure required by the predicate \trs{abbisking}{fulfill}.
%
% \xbox{16}{
% \ea\label{ex:func:ptcpt:prop:an:sukahan}
% \gll Suda raaja=le Andare=pe \textbf{suka-han}=yang mà-abbis-king=nang baaye katha su-biilang.  \\
%       thus king=\textsc{addit} Andare=\textsc{poss} suka-\textsc{nmlzr}=\textsc{acc} \textsc{inf}-finish-\textsc{caus} good \textsc{quot} \textsc{past}-say \\
%     `So the king agreed to fulfill Andare's wish.' (K070000wrt03a)
% \z
% } \\



\subsubsection{Utterances}\label{sec:func:Utterances}
Utterances can also play the role of participants in SLM.   While propositional content can be true or false, utterances do not have a truth value. On the other hand, utterances have illocutionary force and can be judged as to their pragmatic adequacy and felicity, which is not possible for propositional content. In SLM, utterances are marked by \em katha \em in most of the cases. In example \xref{ex:ptcpt:ent:utt}, the content of the dwarf's complaint is reported. This content can be judged as to whether it was appropriate or not, but it cannot be evaluated for its truth value.

\xbox{16}{
\ea \label{ex:ptcpt:ent:utt}
\gll [Incayang=pe jee\u n\u ggoth=yang asà-thaarek=apa  incayang=nang su-sakith-kang katha]$_{utt}$ anà-maaki. \\ % bf
      \textsc{3s}=\textsc{poss} beard=\textsc{acc} \textsc{cp}-pull=after  \textsc{3s}=\textsc{dat} \textsc{past}-pain-\textsc{caus} \textsc{quot} \textsc{past}-scold\\
    `The dwarf complained that they had pulled his beard and hurt him.'      (K070000wrt04)
\z

}

Also non-declarative utterances can be used as participants, an example is a reported question in \xref{ex:ptcpt:ent:utt:interr}.
In this example, the interrogative illocution is overtly marked by \em =si\em.

\xbox{16}{
\ea \label{ex:ptcpt:ent:utt:interr}
\gll [Aashik=nang hathu {\em soldier} mà-jaadi suuka=si katha]$_{utt}$ arà-caanya. \\
     Aashik=\textsc{dat} \textsc{indef} soldier \textsc{inf}-become like=\textsc{interr} \textsc{quot} \textsc{non.past}-ask  \\
    `He asks if you want to become a soldier, Ashik.' (B060115prs10)
\z
}

We see that \em katha \em is used both for propositional content and for utterances used as entities. The distinction between these two levels does  therefore not seem to be very relevant in SLM grammar.

% The fact that we are indeed dealing with an utterance and not with a clause is evident from \xref, where the quotative marker \em katha \em has scope over two clauses
%
%
% \xbox{16}{
% \ea\label{ex:func:unreferenced}
% \ea
% \gll [[Sama oorang masà-thaksir kithang=pe nigiri=ka kanaapa kithang ini bedahan arà-simpang]$_{CLS}$. \\
%      all man must-think \textsc{1pl}=\textsc{poss} country=\textsc{loc} why \textsc{1pl} \textsc{prox} difference \textsc{non.past}-keep  \\
%     `All men must think why we maintain these differences in our country.'
% \ex
% \gll [Cinggala laayeng mulbar laayeng {\em Moor} laayeng mlaayu laayeng sraani laayeng]$_{CLS}$]$_{UTT}$ \textbf{katha}. \\
%       Sinhala different Tamil different Moor different Malay different Burgher different \textsc{quot} \\
%     `Sinhalese, Tamils, Moors, Malays and Burghers are all different.'
% \ex
% \gll [Itthu caara igaama pada=ka {\em catholic} laayeng {\em protestant} laayeng {\em hindu} igaama laayeng buddha igaama laayeng islaam pada laayeng]$_{UTT}$ \textbf{katha}. \\
%       \textsc{dist} way religion \textsc{pl}=\textsc{loc} catholic different protestant different Hindu religion different Buddhist different Islam \textsc{pl} different \textsc{quot} \\
%     `That way, Catholicism, Protestantism, Hinduism, Buddhism and Islam are different.' (K061127nar03)
% \z
% \z
% } \\
%

\subsection{Participant roles}\label{sec:func:Participantroles}
The participants of a proposition can have different semantic roles. The sentence \em Mary cut the bread with the knife on the balcony \em has four participants, \em Mary, the bread, the knife \em and \em the balcony\em. In this example, Mary has the semantic role of agent, the bread is patient, the knife instrument, and the balcony, location. In the following, we will discuss a number of common semantic roles and how they are expressed in Sri Lanka Malay.

With the exceptions of propositional content and utterances, all entities discussed in the previous section can be marked for different semantic roles. The two exceptions are always of the semantic role \textsc{theme}, which is never marked morphosyntactically in their case.

Participant marking is done by means of postpositions.  If the role of a participant is inferable from context or general knowledge of the world, the marking can be left out (\citet[26]{Ansaldo2005ms}, \citet[31]{Ansaldo2008genesis}).
Leaving out the participant role is very often done for topical participants in initial position, like  in \xref{ex:ptcpt:role:unmarkedtopic}. In this case, the speaker trusts the hearer that the latter will be able to retrieve the participant role of \trs{thumpath}{place(s)} from context, \textsc{location} for this lexeme. This is not too difficult here, since \trs{thumpath}{place(s)} is unlikely to have any other role.

\xbox{16}{
\ea \label{ex:ptcpt:role:unmarkedtopic}
\gll Seelong=\zero{} samma thumpath=\zero{} mlaayu aada. \\
 Ceylon all place Malay exist\\
    `In Sri Lanka, there are Malays everywhere.' (K051222nar04)
\z
}

Sometimes, marking of participant roles is less straightforward. In the following example, the undergoers of \trs{thiikam}{stab} are not marked for semantic role while the undergoers of \trs{thee\u mbak}{shoot} are.

\xbox{16}{
\ea \label{ex:ptcpt:role:double:thiikam}
\ea
\gll Itthukapang      oorang pada=\zero{} thiikam=apa, \\
      then man \textsc{pl} stab=after \\
    `Then people were stabbed'
\ex
\gll oorang pada=\textbf{nang}   thee\u mbak=apa, \\
      man \textsc{pl}=\textsc{dat} shoot=after \\
    `and people were shot'
\ex
\gll se=dang bannyak creeveth pada su-aada. \\ % bf
      \textsc{1s=dat} much trouble \textsc{pl} \textsc{past}-exist \\
    `and I got a lot of trouble.' (K051213nar01)
\z
\z
}


\subsubsection{Agent}\label{sec:func:Agent}
The agent is normally \zero-marked \citep[19]{Ansaldo2005ms}. An example of this is \trs{kithang}{1pl} in \xref{ex:ptcpt:role:agent}.


\xbox{16}{
\ea \label{ex:ptcpt:role:agent}
\gll Ithu=kapang lorang=pe leher=yang kithang=\zero{}$_{ag}$ athi-poothong. \\ % bf
     \textsc{dist}=when \textsc{2pl}=\textsc{poss} neck=\textsc{acc} \textsc{1pl} \textsc{irr}=cut  \\
    `Then we will cut your neck.' (K051213nar06)
\z
}

If the agent is not a natural person, but an institution, the instrumental marker \em =dering \em is found, as in the following two examples, where the Government and the police are said to be agents of capturing and interrogating.\footnote{Instrumental marking of institutional actors is also found in Sinhala (\citet[31]{GairEtAl1997}, \citet[791]{Gair2003}).}

\xbox{16}{
\ea\label{ex:ptcpt:role:agen:institution1}
\gll {\em British}  {\em Government}=\textbf{dering}   {\em Malaysia} Indonesia,  inni nigiri pada    samma peegang. \\
    British Government=\textsc{abl}  Malaysia Indonesia \textsc{prox} country \textsc{pl}  all catch\\
   `The British Government captured Malaysia and Indonesia,  those countries.' (K051213nar06)
\z
}


\xbox{16}{
\ea\label{ex:ptcpt:role:agen:institution2}
\gll See=yang {\em police}=\textbf{dering} nya-preksa. \\
     \textsc{1s}=\textsc{acc} police=\textsc{abl} \textsc{past}-enquire  \\
    `I was questioned by the police.' (K051213nar01)
\z
}


\subsubsection{Patient}\label{sec:func:Patient}
The patient of transitive clauses can be marked either with \em =yang \em or with \em =nang\em, but zero-marking is also found \citep{Ansaldo2008genesis}. The  semantics of the verb seems to influence the choice of the postposition, but lexical subcategorization seems to play a part as well. The verb \trs{puukul}{hit}  subcategorizes for \em =nang\em,\footnote{Its
 Sinhala counterpart, \em gahanawaa \em also subcategorizes for the dative \citep[18f]{Gair1991infl}. See \citet[36,92]{Garusinghe1962} for a list of Sinhala verbs with dative marking for patients, which partly overlaps with the SLM verbs.}
 as do  \trs{thee\u mbak}{shoot} \xref{ex:func:ptcpt:semrole:pat:theembak} and \trs{thiikam}{stab} \xref{ex:func:ptcpt:semrole:pat:thiikam}.

\xbox{16}{
\ea\label{ex:func:ptcpt:semrole:pat:puukul}
\gll {\em Dutch}=\textbf{nang} mà-\textbf{puukul}=jo cinggala raaja pada pii aada. \\
      Dutch=\textsc{dat} \textsc{inf}-hit=\textsc{emph} Sinhala king \textsc{pl} go exist\\
    ` The Sinhalese kings went to fight the Dutch.'  (K051206nar04)
\z
}



\xbox{16}{
\ea\label{ex:func:ptcpt:semrole:pat:theembak}
\gll Oorang pada\textbf{=nang}   \textbf{thee\u mbak}=apa se=dang bannyak creeveth pada su-aada. \\
      man \textsc{pl}=\textsc{dat} shoot=after \textsc{1s=dat} much trouble \textsc{pl} \textsc{past}-exist \\
    `People were shot and and I got a lot of trouble.' (K051213nar01) '
\z
}


\xbox{16}{
\ea\label{ex:func:ptcpt:semrole:pat:thiikam}
\gll Incayang=\textbf{nang}    su-\textbf{thiikam}. \\
     3=\textsc{dat}  \textsc{past}-stab\\
    `They stabbed him.' (K051220nar01)
\z
}

The verb \trs{poothong}{cut}, on the other hand, governs the accusative marker \em =yang\em.

\xbox{16}{
\ea\label{ex:func:ptcpt:semrole:pat:poothong:yang}
\gll Ithu=kapang lorang=pe leher=\textbf{yang} kithang athi-\textbf{poothong}. \\
     \textsc{dist}=when \textsc{2pl}=\textsc{poss} neck=\textsc{acc} \textsc{1pl} \textsc{irr}=cut  \\
    `Then we will cut your neck.' (K051213nar06)
\z
}

\xbox{16}{
\ea\label{ex:func:ptcpt:semrole:pat:poothong:zero}
\gll Derang hathu  papaaya=\textbf{yang}   asà-\textbf{poothong}. \\
      3 \textsc{indef} papaya=\textsc{acc} \textsc{cp}-cut \\
    `They cut open a papaya.' (K051220nar01)
\z
}


It appears that dative marking is more likely for activities affecting humans or other animate entities, like hitting, stabbing and shooting, whereas more general-purpose verbs like \trs{ambel}{take} and \trs{thaaro}{put} govern \em =yang\em. One could also speculate that there is a shade of recipient meaning assigned by these verbs (\em receive a blow, shot, stab (wound)\em), which might condition the dative marking. Table \ref{tab:func:semrole:pat:yangnang} gives an overview over different verbs.

\begin{table}
\begin{center}
% use packages: array
\begin{tabular}{lll|l}
 & =yang &  & =nang \\
\hline
\trs{kuthumung}{see} & \trs{biilang}{say} &\trs{preksa}{interrogate} &\trs{puukul}{hit} \\
\trs{encokang}{fool (trs.)}&\trs{baa}{bring}  & \trs{salbakang}{save} &\trs{thee\u mbak}{shoot} \\
\trs{diyath}{see} & \trs{thaaro}{put} &\trs{caari}{search/find} & \trs{thiikam}{stab} \\
\trs{poothong}{cut} & \trs{muuji}{venerate} & \trs{rubbus}{boil} &   \\
\trs{admit-kang}{admit} &\trs{angkath}{lift} &  \trs{abbisking}{finish(trs)} &  \\
\trs{ambel}{take} & \trs{caa\u nda}{meet} &\trs{iingath}{think} & \\
\trs{panggel}{call} & \trs{buunung}{kill} &\trs{sakithkang}{hurt} &\\
\trs{uubar}{chase} &\trs{kiiring}{send} & \trs{luppas}{let go} &  \\
\trs{thaanàm}{plant} & \\
\end{tabular}
\caption[Transitive verbs governing the accusative,  or the dative]{Transitive verbs which govern the accusative \em =yang \em or the dative \em =nang \em for the undergoer argument.}
\label{tab:func:semrole:pat:yangnang}
\end{center}
\end{table}

% K070000wrt01a.txt~: Oorang su  baavung, thoppi pada yang anà caari.
%
% K070000wrt02a.txt: Andare yang ma enco king nang raaja su biilang itthu paasir katha.
%
% K070000wrt04.txt~: <91>See lorang nang thama sakith kang
%
% K070000wrt04.txt~: Derang incayang yang su salbaking.
%
% incayang pe jee\u n\u ggoth yang asà thaarek=apa incayang nang su sakith kang
%
% K070000wrt04.txt~: Aajuth yang buurung ma angkath baapi su diyath.


% \ea\label{ex:func:unreferenced} Alaudin Farook=yang arà-maati\z
% \ea\label{ex:func:unreferenced} Alaudin Farooko=nang arà-puukul\z

If there is no risk of confounding agent and patient, the patient is commonly not marked by a postposition. This is most notably the case
\begin{itemize}
 \item if the patient is much lower on the animacy hierarchy than the agent,
 \item if the patient has  plural reference,
 \item if the action performed on the patient does not affect him much (see Theme below),
 \item if we are dealing with a sentential complement,
 \item if the patient is not individuated,
 \item if the patient is not topical,
 \item if the other argument is experiencer rather than agent,
\end{itemize}

Very often, the factors given above conspire, and the occurrence of \em =yang \em in a sentence can be argued for or against based on more than one of these factors. Table \ref{tab:func:semrole:pat:yang:noyang} gives an overview of which examples can be used to illustrate the factors given above.



\begin{table}
% use packages: array
\begin{tabular}{lp{1cm}p{1cm}p{1cm}p{1cm}p{1cm}p{1cm}p{1cm}}
 	& +anim & +pl 	& +aff 	& +indiv & +top & +exp 	& +sent \\
-yang 	&
%anim
   \xref{ex:semrole:pat:unmarked:affected}	&
%pl
 \xref{ex:semrole:pat:unmarked:plurality}	&
%aff
   \xref{ex:semrole:pat:unmarked:animacy} 	&
%indiv
   \xref{ex:semrole:pat:unmarked:animacy}
 \xref{ex:semrole:pat:unmarked:affected}
\xref{ex:semrole:pat:unmarked:exp}	&
%top
\xref{ex:semrole:pat:unmarked:topical}  	&
%exp
   \xref{ex:semrole:pat:unmarked:affected}
\xref{ex:semrole:pat:unmarked:exp}
\xref{ex:semrole:pat:unmarked:sententialcomplement}	&
%sent
\xref{ex:semrole:pat:unmarked:sententialcomplement}
       \\
\end{tabular}


\begin{tabular}{lp{1cm}p{1cm}p{1cm}p{1cm}p{1cm}p{1cm}p{1cm}}
 	& -\textsc{anim} & -\textsc{pl} 	& -aff 	& -indiv & -top & -exp 	& -sent \\
-yang 	&
%anim
   \xref{ex:semrole:pat:unmarked:animacy}
 \xref{ex:semrole:pat:unmarked:plurality}
 \xref{ex:semrole:pat:unmarked:individuated}
\xref{ex:semrole:pat:unmarked:topical}
\xref{ex:semrole:pat:unmarked:exp}	&
%pl
  \xref{ex:semrole:pat:unmarked:animacy}
 \xref{ex:semrole:pat:unmarked:affected}
\xref{ex:semrole:pat:unmarked:individuated}
\xref{ex:semrole:pat:unmarked:topical}
\xref{ex:semrole:pat:unmarked:exp}&
%aff
  \xref{ex:semrole:pat:unmarked:plurality}
\xref{ex:semrole:pat:unmarked:affected}
\xref{ex:semrole:pat:unmarked:individuated}
\xref{ex:semrole:pat:unmarked:exp}	&
%indiv
  \xref{ex:semrole:pat:unmarked:plurality}
\xref{ex:semrole:pat:unmarked:individuated}
\xref{ex:semrole:pat:unmarked:topical}	&
%top
  \xref{ex:semrole:pat:unmarked:animacy}
\xref{ex:semrole:pat:unmarked:plurality}
  \xref{ex:semrole:pat:unmarked:affected}
\xref{ex:semrole:pat:unmarked:individuated}
\xref{ex:semrole:pat:unmarked:exp}	&
%exp
  \xref{ex:semrole:pat:unmarked:animacy}
\xref{ex:semrole:pat:unmarked:plurality}	&
%sent
   \xref{ex:semrole:pat:unmarked:animacy}
 \xref{ex:semrole:pat:unmarked:plurality}
 \xref{ex:semrole:pat:unmarked:affected}
\xref{ex:semrole:pat:unmarked:exp}   \\

\end{tabular}
\caption[Zero-marking of the patient]{Zero-marking of the patient as a function of a number of factors. Numbers refer to the examples below.}
\label{tab:func:semrole:pat:yang:noyang}
\end{table}

\xbox{16}{
\ea \label{ex:semrole:pat:unmarked:animacy}
\gll Itthu=kapang baapa  derang=pe     kubbong=ka  hatthu pohong]$_{PAT}$=\zero{} nya-poothong. \\ % bf
       \textsc{dist}=when father \textsc{3pl}=\textsc{poss} estate=\textsc{loc} \textsc{indef} tree \textsc{past}-cut\\
    `The father felled a tree in their estate.' (K051205nar05)
\z
}

\xbox{16}{
\ea \label{ex:semrole:pat:unmarked:plurality}
\gll [Oorang anà-baava samma thoppi=pada]$_{PAT}$=\zero{} asà-ambel. \\ % bf
      man \textsc{past}-bring all hat=\textsc{pl} \textsc{cp}-take\\
    `(The monkeys) took all the hats the man had brought.'  (K070000wrt01)
\z
}

\xbox{16}{
\ea \label{ex:semrole:pat:unmarked:affected}
\gll Ithukapang       oorang pada    [hathu  oorang]$_{PAT}$=\zero{} asà-kuthumung. \\ % bf
then man \textsc{pl} \textsc{indef} man \textsc{cp}-see \\
    `Then the men saw a man.' (K051220nar01)
\z
}


\xbox{16}{
\ea \label{ex:semrole:pat:unmarked:individuated}
\gll Baapa=le       aanak=le      [guula]$_{PAT}$=\zero{} su-maakang. \\ % bf
      father=\textsc{addit} son=\textsc{addit} sugar \textsc{past}-eat \\
    `Father and son ate sugar.' (K070000wrt02)
\z
}

\xbox{16}{
\ea \label{ex:semrole:pat:unmarked:topical}
\ea
\gll Itthu=nang [aayer]$_{PAT}$=\zero{} asà-thaaro=apa \\ % bf
       \textsc{dist}=\textsc{dat} water \textsc{cp}-put=after\\
    `Having added water to it'
\ex
\gll [itthu aayer]$_{PAT}$=yang baaye=nang arà-{\em boil}-kang. \\
       \textsc{dist} water=\textsc{acc} good=\textsc{dat} \textsc{non.past}-boil-caus\\
    `boil the water well.'
\z
\z
}

\xbox{16}{
\ea \label{ex:semrole:pat:unmarked:exp}
\gll [Kìrras pinthu=nang arà-thatti hathu svaara]$_{PAT}$=\zero{} su-dìnngar. \\ % bf
     strong door=\textsc{dat} \textsc{simult}-hammer \textsc{indef} noise] \textsc{past}-hear \\
    `They heard a noise of hard hammering at the door.'  (K070000wrt04)
\z
}


\xbox{16}{
\ea \label{ex:semrole:pat:unmarked:sententialcomplement}
\gll Blaakang=jo incayang anà-kuthumung [moonyeth pada thoppi asà-ambel pohong atthas=ka arà-maayeng]$_{PAT}$=\zero{}. \\ % bf
     after=\textsc{emph} \textsc{3s.polite} \textsc{past}-see monkey \textsc{pl} hat \textsc{cp}-take tree top=\textsc{loc} \textsc{simult}-play  \\
    `Then only he saw that the monkeys had taken his hats and were playing on the top of the trees.'  (K070000wrt01)
\z
}

To illustrate several factors conspiring,  \em thoppi \em in example \xref{ex:semrole:pat:unmarked:plurality} is at the same time unindividuated, plural in reference, low on the animacy hierarchy and relatively unaffected (cf. Table \ref{tab:func:semrole:pat:yang:noyang}).\\

If on the other hand, there is a risk of confounding agent and patient, \em =yang \em is used more often. This is notably the
case under the following conditions.

\begin{itemize}
 \item   the agent is less animate than the patient \xref{ex:semrole:pat:marked:revanim}
 \item   the patient is  individuated \xref{ex:semrole:pat:marked:indiv}.
\end{itemize}


\xbox{16}{
\ea \label{ex:semrole:pat:marked:revanim}
\gll Aajuth=\textbf{yang} buurung mà-angkath baapi su-diyath. \\
      dwarf=\textsc{acc} bird \textsc{inf}-lift take.away \textsc{past}-try \\
    `The bird tried to carry the dwarf away.' (K070000wrt04)
\z
}

In this example, the dwarf is comparatively little affected, but he is individuated, and, more importantly, the normal order of more animate entities acting upon less animate ones is inverted here.
Again these factors conspire very often, and in example \xref{ex:semrole:pat:marked:indiv} the speaker and his persecutors are a) close on the animacy hierarchy, and b) the speaker is individuated. Both these conditions favour the use of \em =yang\em.

\xbox{16}{
\ea \label{ex:semrole:pat:marked:indiv}
\gll Se=ppe oorang pada [see saapa] katha thàrà-thaau subbath \textbf{see=yang} su-uubar. \\
     \textsc{1s=poss} man \textsc{pl} \textsc{1s} who \textsc{quot} \textsc{neg}-know because \textsc{1s}=\textsc{acc} \textsc{past}-chase\\
    `Because my folks did not know who I was, they chased me.'  (K070000wrt04)
\z
}

It should be noted that no hard and fast rule for the occurrence of \em =yang \em can be given; its occurrence depends on multiple factors, and even then there are unexplainable sentences. An examples is \xref{ex:semrole:pat:double} where two things are bought, but one is marked with \em =yang \em and the other one is not.

\xbox{16}{
\ea \label{ex:semrole:pat:double}
\gll Lai se {\em computer}=nang baaru {\em optical} {\em mouse} atthu=\zero=le {\em Encarta2006} {\em software}=\textbf{yang}=le su-bìlli. \\
     other \textsc{1s} computer=\textsc{dat} new optical mouse \textsc{indef}=\textsc{addit} Encarta2006 software=\textsc{acc}=\textsc{addit} \textsc{past}-buy  \\
    `Then I also bought a new optical mouse and the Encarta 2006 software for the computer.'  (Letter 26.06.2007)
\z
}

The semantic factors sketched above have an influence, but also idiolectal and possibly style factors. For every simple determinant factor for the occurrence of \em =yang\em, counterexamples can be found, but a combination of several of the criteria mentioned above could possibly yield good predictions. Such a multivariate analysis is beyond the scope of this grammatical description.

%
%
% K070000wrt04.txt~: Kaaki ka gaa\u ndas kang ambel anà duuduk Aajuth yang sangke luppas hathu pollu dering Rose-red buurung nang su puukul.
%
% K070000wrt04.txt~:Aajuth thaakuth ka su naangis,  See yang luppas, Thuan Buruan.
%
%
% K070000wrt04.txt~: Itthule see yang ma kiiring nang duppang incayang see yang hathu Buruan ma jaadi su baleking.
%

% \xbox{16}{
% \ea\label{ex:func:unreferenced}
% \gll Oorang arà-buunung    samma. \\
%      man \textsc{non.past}-kill all  \\
%     `He killed everybody.' (K051205nar02)
% \z
% } \\


The affectedness of a patient can additionally be highlighted by the vector verb \em thaaro \em \formref{sec:wc:thaaro}, as in the following examples.

\xbox{16}{
\ea\label{ex:semrole:pat:thaaro1}
\gll Incayang=yang    siaanu  asà-buunung   \textbf{thaaro}=apa. \\
    \textsc{3s.polite}=\textsc{acc} 3s.prox \textsc{cp}-kill put=after    \\
    `This one has killed him.' (K051220nar01)
\z
}

\xbox{16}{
\ea\label{ex:semrole:pat:thaaro2}
\gll See=yang dhaathang {\em remand}=ka mà-thaarek \textbf{thaaro}=nang thàràboole su-jaadi. \\
     \textsc{1s}=\textsc{acc} come remand=\textsc{loc} \textsc{inf}-pull put=\textsc{dat} cannot \textsc{past}-become  \\
    `It became impossible to remand me.' (K061122nar03)
\z
}

\citet{SmithEtAl2004} claim that accusative (patient) and dative (recipient) are conflated in SLM, but the base for their claim is unclear \citep{Ansaldo2005ms}. The forms \em =yang \em and \em =nang \em are clearly distinct (with the exception of a very infrequent \em =nya\em, which might be an allomorph of either, and for whose occurrence no explanation can be given as of now). The verbs also clearly subcategorize either for the one or the other, so that there are neither morphophonological nor distributional reasons to assume a conflation of patient marking and recipient marking.

\subsubsection{Theme}\label{sec:func:Theme}
Themes are either marked by \zero{} or by \em =yang\em.

\xbox{16}{
\ea\label{ex:semrole:pat:theme:zero}
\gll Se=ppe    baapa  incayang=nang    ummas=\zero{} su-kaasi. \\ % bf
      \textsc{1s=poss} father \textsc{3s.polite}=\textsc{dat} gold \textsc{past}-give\\
    `My father gave him gold.'  (K070000wrt04)
\z
}

\xbox{16}{
\ea\label{ex:semrole:pat:theme:yang}
\gll Derang=pe pàrhaal  pada=\textbf{yang}   mà-thuulis    ambel=nang. \\
      \textsc{3pl}=\textsc{poss} problem \textsc{pl}=\textsc{acc} \textsc{inf}-write take=\textsc{dat} \\
    `To write down their problems.' (K051213nar01)
\z
}

Since themes are by definition less affected than patients, marking with \em =yang \em is less common. Themes are never marked by \em =nang\em.
Themes of  utterance verbs can also be marked by \trs{atthas}{about}, as in \xref{ex:semrole:pat:theme:atthas}.

\xbox{16}{
\ea\label{ex:semrole:pat:theme:atthas}
\gll Se=ppe \textbf{atthas} laskalli masà-biilang=si? \\
      \textsc{1s=poss} about again must-say=\textsc{interr} \\
    `Do I have to tell about myself again?' (B060115prs05)
\z
}

An interesting case is \xref{ex:semrole:pat:theme:venerate}, where the theme is marked by \em =yang\em,  possibly to underscore the individuated nature and anthropomorphic image of the cow venerated by Hindus. I have tried to render this by capitalization in English.

\xbox{16}{
\ea\label{ex:semrole:pat:theme:venerate}
\gll Hindu pada sampi=\textbf{yang}   arà-muuji. \\
      Hindu \textsc{pl} cow=\textsc{acc} \textsc{non.past}-venerate \\
    `Hindus venerate the Cow' (K060112nar01)
\z
}

\subsubsection{Recipient}\label{sec:func:Recipient}
Recipients are always coded with the postposition \em =nang\em. \xref{ex:semrole:rec} gives an example.

\xbox{16}{
\ea\label{ex:semrole:rec}
\gll Se=ppe    baapa  incayang=\textbf{nang}    ummas su-kaasi. \\
      \textsc{1s=poss} father \textsc{3s.polite}=\textsc{dat} gold \textsc{past}-give\\
    `My father gave him gold.'  (K070000wrt04)
\z
}



\subsubsection{Experiencer}\label{sec:func:Experiencer}
The semantic role of experiencer is indicated by the dative marker \em =nang \em  \citep{Ansaldo2005ms,Ansaldo2008genesis} \formref{sec:morph:=nang}, as common in South Asia \citep{Masica1976,Sridhar1976cls,Sridhar1976sils,Sridhar1979,VermaEtAlEd1990,
%Abbi1990exp,
Abbi1994,BhaskararaoEtAlEd2004I,BhaskararaoEtAlEd2004II}.
%Jayaseelan2004
The following sentences give examples for experiencers of mental predicates \xref{ex:ptcpt:semrole:exp:mental1}-\xref{ex:ptcpt:semrole:exp:mental4}, sensory  predicates \xref{ex:ptcpt:semrole:exp:sensory1}\xref{ex:ptcpt:semrole:exp:sensory2}, and bodily predicates \xref{ex:ptcpt:semrole:exp:bodily1}.



\xbox{16}{
\ea \label{ex:ptcpt:semrole:exp:mental1}
\gll {\em Tailoring} go\textbf{dang}    baaye=nang   \textbf{thaau}$_{mental}$. \\
       tailoring \textsc{1s=dat} good=\textsc{dat} know\\
    `I know tailoring very well.' (B060115nar04)
\z
}


\xbox{16}{
\ea \label{ex:ptcpt:semrole:exp:mental2}
\gll Inni     oorang=\textbf{nang}   itthu    \textbf{thàrà-thaau}$_{mental}$. \\
      \textsc{prox} man=\textsc{dat} \textsc{dist} \textsc{neg}-know \\
    `This man did not know that.' (K070000wrt01)
\z
}




\xbox{16}{
\ea \label{ex:ptcpt:semrole:exp:mental3}
\gll Derang pada=\textbf{nang}   karang {\em Malay} arà-\textbf{luupa}$_{mental}$. \\
     \textsc{3pl} \textsc{pl}=\textsc{dat} now Malay \textsc{non.past}-forget  \\
    `They are forgetting (their) Malay now.' (G051222nar02)
\z
}



\xbox{16}{
\ea \label{ex:ptcpt:semrole:exp:mental4}
\gll Suda Andare=\textbf{nang}=le        buthul \textbf{suuka}$_{mental}$. \\
      thus Andare=\textsc{dat}=\textsc{addit} very like \\
    `So Andare also liked it a lot.' (K070000wrt05)
\z
}




\xbox{16}{
\ea \label{ex:ptcpt:semrole:exp:sensory1}
\gll Derang=\textbf{nang} byaasa svaara hatthu su-\textbf{dìnngar}$_{sensory}$. \\
      \textsc{3pl}=\textsc{dat} habit noise \textsc{indef} \textsc{past}-hear\\
    `They heard a familiar voice.' (K070000wrt04)
\z
}


\xbox{16}{
\ea \label{ex:ptcpt:semrole:exp:sensory2}
\gll Itthu haari=ka=jo aanak pompang duuva=\textbf{nang} ... kiccil jillek Aajuth hatthu=yang su-\textbf{kuthumung}$_{sensory}$ \\
     \textsc{dist} day=\textsc{loc}=\textsc{emph} child female two=\textsc{dat} ... small ugly dwarf \textsc{indef}=\textsc{acc} \textsc{past}-see  \\
    `On that very day, the two girls perceived a small ugly dwarf.' (K070000wrt04)
\z
}




\xbox{16}{
\ea \label{ex:ptcpt:semrole:exp:bodily1}
\gll Go=\textbf{dang}    karang bannyak \textbf{thàràsìggar}$_{bodily}$. \\
      \textsc{1s.familiar}=\textsc{dat} now lot sick \\
    `I am very sick now.' (B060115nar04)
\z
}



%
%
% \xbox{16}{
% \ea \label{ex:ptcpt:semrole:exp:social1}
% \gll Se=ppe bungkus su iilang. \\
%      \textsc{1s=poss} purse \textsc{past}-disappear  \\
%     `I lost my purse.' (test)
% \z
% } \\
%
%
% \xbox{16}{
% \ea
% \gll *se=dang suiilang. \\
%        \\
%     `.' (nosource)
% \z
% } \\
%
%




%
% \xbox{16}{
% \ea\label{ex:func:unreferenced}
% \gll Aanak pompang duuva su-thaakuth. \\
%      child woman two \textsc{past}-fear  \\
%     `The two girls were scared.'  (K070000wrt04)
% \z
% }\\

The fact that the modals \trs{maau}{want}, \trs{thussa}{\textsc{neg}.want}, \trs{boole}{can} and \trs{thàrboole}{cannot} mark the argument can also be explained by the fact that the person having the desire or ability (or lack thereof) is normally not actively responsible for that in the moment of speaking (i.e. lacks control, cf. \citet{Ansaldo2005ms}.).  True, a person who can drive a car has actively acquired that knowledge at some point in time, but at the time of speaking, this person is passive.

%
%
% \xbox{16}{
% \ea\label{ex:func:unreferenced}
% \gll Cinggala mulbar thàrà thaau. \\
%  Sinhala Tamil \textsc{neg} know\\
% `Sinhalese do not know Tamil.' (K051220nar02 )
% \z
% }
%
%


\subsubsection{Source}\label{sec:func:Source}
Source can be either marked by \em =dering \em \xref{ex:ptcpt:semrole:src:dring} \formref{sec:morph:=dering} or the conjuctive participle of the animate existential \em (a)s(à)duuduk \em \formref{sec:wc:Existentialverbs:duuduk}  \xref{ex:ptcpt:semrole:src:sduuduk1}.


\xbox{16}{
\ea \label{ex:ptcpt:semrole:src:dring}
\ea
\gll Kitham=pe      oorang thuuva pada  bannyak dhaathang aada {\em Malaysia}=\textbf{dering}; \\
     \textsc{1pl}=\textsc{poss} man old \textsc{pl} many come exist Malaysia=\textsc{abl}  \\
      `Many of our ancestors came from Malaysia;'
\ex
\gll Spaaru Indonesia=\textbf{dering}      dhaathang aada. \\
     some Indonesia=\textsc{abl} come exist\\
  `some came from Indonesia.' (K060108nar02)
\z
\z
}



\xbox{16}{
\ea \label{ex:ptcpt:semrole:src:sduuduk1}
\gll Suda see {\em Trinco}=ka  \textbf{asàduuduk} Kluu\u mbu=nang   su-dhaathang. \\
      So \textsc{1s} Trincomalee=\textsc{loc} from Colombo=\textsc{dat} \textsc{past}-come \\
    `So I went from Trincomalee to Colombo.' (K051206nar20)
\z
}

Since \em asàduuduk \em is the conjunctive participle of the animate existential verb \em duuduk\em, it can only be used to indicate the source of  animate entities. The stones in \xref{ex:ptcpt:semrole:src:dring} are not animate, and \em asàduuduk \em cannot be used there to indicate the origin of the stones.


The following examples show some more uses of \em =dering \em to indicate source

\xbox{16}{
\ea \label{ex:ptcpt:semrole:src:dering:extra1}
\gll Hathu haari, hathu oorang thoppi mà-juval=nang kampong=\textbf{dering} kampong=nang su-jaalang pii. \\
     \textsc{indef} day \textsc{indef} man hat \textsc{inf}-sell=\textsc{dat} village=\textsc{abl} village=\textsc{dat} \textsc{past}-walk go  \\
    `One day, a man walked from village to village to sell hats.'  (K070000wrt01)
\z
}


%
% \xbox{16}{
% \ea\label{ex:func:unreferenced}
% \gll {\em 1967}     asàduuduk {\em Kandy}=ka    su-duuduk. \\
%       1967 from Kandy=\textsc{loc} \textsc{past}-stay \\
%     `From 1967 onwards, I have been living in Kandy.' (K051201nar01)
% \z
% } \\

\xbox{16}{
\ea \label{ex:ptcpt:semrole:src:extra2} 
\gll See asà-{\em retire} aada {\em police}=\textbf{dering}. \\
      \textsc{1s} \textsc{cp}-retire after police=\textsc{abl} \\
    `Having retired from the police' 
\z
}

\xbox{16}{
\ea \label{ex:ptcpt:semrole:src:extra3}
\gll {\em Kandy} {\em Malay} {\em Association}=\textbf{dering} hatthu hatthu oorang pada arà-lompath {\em Hill} {\em Country}=nang. \\
     Kandy Malay association=\textsc{abl} \textsc{indef} \textsc{indef} man \textsc{pl} \textsc{non.past}-jump hill country=\textsc{dat}  \\
    `More and more people stepped over from the KMA to the Hill Country Malay Association.' (K060116nar07)
\z
}

The use of \em (a)s(à)duuduk \em is exemplified by the sentences below.

\xbox{16}{
\ea \label{ex:ptcpt:semrole:src:sduuduk:extra1}
\gll Kithan       nya-pii Anuradhapura=dang, Katunaayaka     \textbf{sduuduk}. \\
 \textsc{1pl} \textsc{past}-go Anuradhapura=\textsc{dat} Katunayaka from\\
`We went to Anuradhapura, from Katunayaka [airport].' (K051206nar16)
\z
}


\xbox{16}{
\ea \label{ex:ptcpt:semrole:src:sduuduk:extra2}
\gll Katugastota \textbf{asàduuduk} {\em St.Anthony's}=nang      arà-jaalang \\
     Katugastota from St.Anthony's=\textsc{dat} \textsc{non.past}-walk  \\
    `I walk from Katugastota (train station) to St. Anthony's (school).' (K051201nar02)
\z
}

% The participant marked by \em sduuduk\em, does not have to undergo a change of location, even if the etymology of \em asà-duuduk \em (having sat) suggests otherwise. In the following example, the angels do not change their position, they remain immobile on both sides of the grave, but the provenance of the questions is from those sides. One can either argue that in \xref, \em sduuduk \em refers to the questions, which do indeed `travel' from the mouth to the ear as it were, and thereby change location. This is a bit strange, though, since inanimate participants, such as questions, can normally not combine by \em sduuduk\em. It would also be possible to see
%
% \xbox{16}{
% \ea\label{ex:func:unreferenced}
% \ea
% \gll Mayyeth arà-kubuur-kang vakthu. \\
%       corpse \textsc{non.past}-bury-\textsc{caus} time \\
%     `When the corpse is buried'
% \ex
% \gll Mayyeth=pe kubur-an paapang=nya arà-thuuthup vakthu=le. \\
%       corpse=\textsc{poss} bury-nmlzr pole=\textsc{acc} \textsc{non.past}-close time=\textsc{addit} \\
%     `and when the corpse's grave pole is closed'
% \ex
% \gll Mà-liyath thaakuth. \\
%      \textsc{inf}-watch fear   \\
%     `To watch the angel.'
% \ex
% \gll Mleekath karakiye  duva subla=le asàduuduk percayahan arà-caanya. \\
%       angel ??? two side=\textsc{addit} from question \textsc{non.past}-ask\\
%     `The angel ??? will ask questions from both sides.' (K060116sng02)
% \z
% \z
% } \\


\subsubsection{Goal}\label{sec:func:Goal}
Goal is normally coded by \em =nang \em \formref{sec:morph:=nang} (\citet[19,22]{Ansaldo2005ms},\citet[24]{Ansaldo2008genesis}), which attaches to the place name \xref{ex:ptcpt:semrole:goal:placename}, the spatial entity \xref{ex:ptcpt:semrole:goal:spatialnoun} or a relator noun \xref{ex:ptcpt:semrole:goal:relatornoun},  but deictics do not take this marker when used as goal \xref{ex:ptcpt:semrole:goal:deictic}.


\xbox{16}{
\ea \label{ex:ptcpt:semrole:goal:placename}
\gll Soojer pada  incayang=sàsaama Seelon=\textbf{nang} asà-dhaathang. \\
European \textsc{pl}  \textsc{3s.polite}=\textsc{comit} Ceylon=\textsc{dat} \textsc{cp}-come\\
`The Europeans came to Sri Lanka together with him.' (K060103nar01)
\z
}


\xbox{16}{
\ea \label{ex:ptcpt:semrole:goal:spatialnoun}
\gll Suda lorang=yang ruuma=\textbf{nang} anthi-aaji.baapi \\
    thus \textsc{2pl}=\textsc{acc} house=\textsc{dat} \textsc{irr}=bring.\textsc{anim}   \\
    `So I will take you to my house.' (K070000wrt04)
\z
}



%
% \xbox{16}{
% \ea\label{ex:func:unreferenced}
% \gll Hathu haari, hathu oorang thoppi mà-juval=nang kampong=dering kampong=\textbf{nang} su-jaalang pii. \\
%      \textsc{indef} day \textsc{indef} man hat \textsc{inf}-sell=\textsc{dat} village=\textsc{abl} village=\textsc{dat} \textsc{past}-walk go  \\
%     `One day, a man walked from village to village to sell hats.'  (K070000wrt01)
% \z
% }\\



%
%  \xbox{16}{
%  \ea\label{ex:func:unreferenced}
%    \gll Blaakang see angkath   thaangang=nang   su-biilang     Badulla  thama-dhaathang=si. \\
%     \textsc{neg.nonpast}- come      -Q \\
% (B060115cvs01)
% \z
% }

%   00033
% f K051206nar20
%  suda see Trincoka         asàduuduk Kluu\u mbunang   sudhaathang
%  So I went from Trincomalee to Colombo


\xbox{16}{
\ea \label{ex:ptcpt:semrole:goal:relatornoun}
\gll Ithukang ithu bambu giithu=jo luvar=\textbf{nang} arà-dhaathang. \\
      then this bamboo like.that=\textsc{emph} outside=\textsc{dat} \textsc{non.past}-come\\
    `Then the bamboo comes out like that.' (K061026rcp04)
\z
}


\xbox{16}{
\ea \label{ex:ptcpt:semrole:goal:deictic}
\gll Se=ppe    {\em profession}=subbath se=dang  siini\textbf{=\zero} mà-pii    su-jaadi. \\
      \textsc{1s=poss} profession=because \textsc{1s=dat} here \textsc{inf}-go \textsc{past}-become \\
    `I had to come here because of my profession.' (G051222nar01)
\z
}



If the goal of motion is human, the relator noun \em dìkkath \em is obligatory, as shown in \xref{ex:ptcpt:semrole:goal:human} \citep[cf.][]{SmithEtAl2004}.

\xbox{16}{
\ea \label{ex:ptcpt:semrole:goal:human}
   \gll Kithang=nang   hathu  {\em job} hatthu mà-ambel=nang      kithang=nang   hathu  {\em application} mà-sign  kamauvan vakthu=nang=jo      kithang arà-pii    inni     {\em politicians} pada \textbf{dìkkath=nang}. \\ % bf
    \textsc{1pl}=\textsc{dat} \textsc{indef} job \textsc{indef} \textsc{inf}-take=\textsc{dat} \textsc{1pl}=\textsc{dat} \textsc{indef} application \textsc{inf}-sign want time=\textsc{dat}=\textsc{emph} \textsc{1pl} \textsc{non.past}-go \textsc{prox} politicians \textsc{pl} vicinity=\textsc{dat}\\
`When we want to take a job, when we want to sign an application, we approach these politicians' (K051206nar12)
\z
}



\em =nang \em can attach to proper nouns referring to non-human entities, even if they are not spatial in the strict sense of the term, such as in \xref{ex:ptcpt:semrole:goal:association}, where the goal of motion is an association.

\xbox{16}{
\ea \label{ex:ptcpt:semrole:goal:association}
\gll {\em Kandy} {\em Malay} {\em Association}=dering hatthu hatthu oorang pada arà-lompath \textbf{{\em Hill}} \textbf{{\em Country}=nang}. \\
     Kandy Malay association=\textsc{abl} \textsc{indef} \textsc{indef} man \textsc{pl} \textsc{non.past}-jump hill country=\textsc{dat}  \\
    `More and more people stepped over from the KMA to the Hill Country Malay Association.' (K060116nar07)
\z
}

Goal of motion can also be marked by the locative marker \em =ka \em \xref{ex:ptcpt:semrole:goal:ka:government}\xref{ex:ptcpt:semrole:goal:ka:hathuhathu} \citep{SmithEtAl2004}.

\xbox{14}{
\ea \label{ex:ptcpt:semrole:goal:ka:government}
\gll {\em Government}=pe     hathu  \textbf{thumpath}=\textbf{ka}   asà-pii   pukurjan bole=girja \\
     government=\textsc{poss} \textsc{indef} place=\textsc{loc} \textsc{cp}-go work can-make  \\
    `They can go to a government place and work there.' (K051222nar05)
\z
}


\xbox{14}{
\ea \label{ex:ptcpt:semrole:goal:ka:hathuhathu}
\gll Derang samma oorang hatthu hatthu \textbf{thumpath} \textbf{pada}=\textbf{ka} asà-pii pukurjan su-gijja \\
      \textsc{3pl} all man \textsc{indef} \textsc{indef} place \textsc{pl=loc} \textsc{cp}-go work \textsc{past}-make \\
    `All those people go to one place or another and work.' (B060115cvs06)
\z
}



\subsubsection{Path}\label{sec:func:Path}
Path is coded by \em =dering \em \formref{sec:morph:=dering}. In the following example, the bird's trajectory has an unknown source and an unknown goal, but leads over the girls' heads. This is indicated by the ablative marker \em =dering\em.

\xbox{16}{
\ea \label{ex:ptcpt:semrole:path}
\gll Bìssar hathu buurung derang=pe atthas=\textbf{dering} su-thìrbang. \\
      big \textsc{indef} bird \textsc{3pl}=\textsc{poss} top=\textsc{abl} \textsc{past}-fly \\
    `A big bird flew over them.'  (K070000wrt04)
\z
}

\subsubsection{Instrument}\label{sec:func:Instrument}
Instrument is indicated by \em =dering \em \formref{sec:morph:=dering}. The instrument can be used by a sentient being, as in \xref{ex:ptcpt:semrole:instr:sent}, but \em =dering \em can also be used in a wider sense for participants not manipulated by a sentient being as in \xref{ex:ptcpt:semrole:instr:nonsent}.

\xbox{16}{
\ea\label{ex:ptcpt:semrole:instr:sent}
\gll Thaangang=\textbf{dering} bukang kaaki=\textbf{dering} masà-maayeng. \\
      hand=\textsc{abl} \textsc{neg.nonv} leg=\textsc{abl} must-play \\
    `You must play not with the hands, but with the feet.' (N060113nar05)
\z
}

\xbox{16}{
\ea \label{ex:ptcpt:semrole:instr:nonsent}
\gll \textbf{Daavon=dering}     thuuthup ada  gaaja    hatthu asdhaathang. \\
     leaf=\textbf{instr} close exist elephant \textsc{indef} \textsc{cp}-come  \\
    `An elephant, which had been hidden by leaves, appeared.' (B060115nar05)
\z
}

% Languages can also be conceived as an instrument for communication and are marked with \em =dering \em as well.
%
% \xbox{16}{
% \ea\label{ex:func:unreferenced}
% \gll Arà-biilang mlaayu=dering. \\
%  \textsc{non.past}-speak Malay=abl\\
% `Speak in Malay.' (K060103nar01)
% \z
% }

The semantic role of instrument is normally always marked overtly. An exception to this rule is found in \xref{ex:ptcpt:semrole:instr:drop}, where the ears could be said to be the instruments of hearing, yet they are not marked by \em =dering\em. Since ears are so prototypically associated with hearing, marking of their semantic role does not seem necessary.

\xbox{16}{
\ea \label{ex:ptcpt:semrole:instr:drop}
\gll Andare ruuma=nang asà-pii biini=nang su-biilang puthri=nang \textbf{kuuping} arà-dìnngar kuurang katha. \\
    Andare house=\textsc{dat} \textsc{cp}-go wife=\textsc{dat} \textsc{past}-say queen=\textsc{dat} ear \textsc{non.past}-hear little \textsc{quot}  \\
    `Andare went home and said to his wife that the queen did not hear well (lit.: that the queen heard little with her ears).' (K070000wrt05)
\z
}



% \xbox{16}{
% \ea
% \gll Muralipe boolayang wicketka sukìnna. \\
%        \\
%     `.' (nosource)
% \z
% } \\


\subsubsection{Beneficiary}\label{sec:func:Beneficiary}
Beneficiary is normally coded by \em =nang \em \formref{sec:morph:=nang} \xref{ex:ptcpt:semrole:ben:overt1}\xref{ex:ptcpt:semrole:ben:overt2} \citep{Ansaldo2005ms,Ansaldo2008genesis}, but can also be marked with a relator noun \trs{bagiyan}{behalf}. As with the other roles, the overt marking can be dropped if the role is clear from context \xref{ex:ptcpt:semrole:ben:dropped}. A further possibility is the use of the vector verbs \trs{kaasi}{give} or \trs{ambel}{take}.


\xbox{16}{
\ea \label{ex:ptcpt:semrole:ben:overt1}
\gll Cinggala   raaja\textbf{=nang} deram  pada banthu aada. \\
 Sinhala king=\textsc{dat} \textsc{3pl} \textsc{pl} help exist\\
`They have helped the Sinhalese king.' (K051206nar04)
\z
}

\xbox{16}{
\ea \label{ex:ptcpt:semrole:ben:overt2}
\gll Derang=pe umma derang=\textbf{nang} jaith-an=le, jaarong pukurjan=le su-aajar. \\
       \textsc{3pl}=\textsc{poss} mother \textsc{3pl}=\textsc{dat} sew-\textsc{nmlzr}=\textsc{addit} needle work=\textsc{addit} \textsc{past}-teach\\
    `Their mother taught them sewing and needle work.'  (K070000wrt04)
\z
}


% B060115cvs01.txt- itthu    kumpulan=dang      derang=jo     bannyak arà-banthu

\xbox{16}{
\ea \label{ex:ptcpt:semrole:ben:dropped}
\gll Kettha  drampada=\zero{}      bannyak banthu. \\ % bf
      \textsc{1pl} \textsc{3pl} much help \\
    `We helped them a lot.' (B060115nar02)
\z
}

% \xbox{16}{
% \ea\label{ex:func:unreferenced}
% \gll Derang derang=pe umma=nang butthul saayang=kee=jo samma ruuma pukurjan=nang=le anà-banthu. \\
%       \textsc{3pl} \textsc{3pl}=\textsc{poss} mother=\textsc{dat} correct love=\textsc{simil}=\textsc{emph} all house work=\textsc{dat}=\textsc{addit} \textsc{past}-help  \\
%     `They also helped their mother with all the housework.'  (K070000wrt04)
% \z
% }\\

% Very often, the role of beneficiary is found with three-place  predicates, but it is also possible to find it in two place predicates as in \xref{ex:ptcpt:semrole:ben:twoplace}.
%
% \xbox{16}{
% \ea\label{ex:ptcpt:semrole:ben:twoplace}
% \gll Itthu=nang blaakang [kithang=\textbf{nang}   anà-daapath    {\em government}]=ka incayang=yang    uthaama mlaayu=nang   anà-duuduk. \\
%       \textsc{dist}=\textsc{dat} after \textsc{1pl}=\textsc{dat} \textsc{past}-get government=\textsc{loc} \textsc{3s.polite}=\textsc{acc} honour Malay=\textsc{dat} \textsc{past}-exist.\textsc{anim}  \\
%     `After that, he stayed in a respectable way for the Malays in the government we got.' (N061031nar01)
% \z
% } \\

Beneficiary can also be added to another predication which does not require a beneficiary strictly speaking, if the action turned out to be beneficial. This is shown in \xref{ex:ptcpt:semrole:ben:constr}.

\xbox{16}{
\ea\label{ex:ptcpt:semrole:ben:constr}
\gll Giini   duuduk     bannyak  [\textbf{kithang=pe} \textbf{oorang} \textbf{pada}]=nang  anà-caape. \\
      like.this stay much \textsc{1pl}=\textsc{poss} man \textsc{pl}=\textsc{dat} \textsc{past}-tired \\
    `Being there, he toiled a lot for our people.' (N061031nar01)
\z
}

The beneficial nature of an action can additionally be highlighted by the vector verb \trs{kaasi}{give} as in \xref{ex:ptcpt:semrole:ben:kaasi:drop}. Note that the benefiting persons are not overtly realized because they have been established as a topic before.


\xbox{16}{
\ea\label{ex:ptcpt:semrole:ben:kaasi:drop}
\gll Itthu muusing  Islam igaama  nya-\textbf{aajar} \textbf{kaasi} Jaapna  Hindu {\em teacher}. \\
      \textsc{dist} time Islam religion \textsc{past}-teach give Jaffna Hindu teacher \\
    `At that time, those who taught Islamic religion were Hindu teachers from Jaffna.' (K051213nar03)
\z
}

It is also possible to overtly realize the benefitting participant when \em kaasi \em is used, as in \xref{ex:ptcpt:semrole:ben:kaasi:overt}.

\xbox{16}{
\ea\label{ex:ptcpt:semrole:ben:kaasi:overt}
   \gll Kithang=pe     ini      {\em younger} {\em generation}=nang=jo     konnyong masà-\textbf{biilang} \textbf{kaasi}, masà-aajar. \\
    \textsc{1pl}=\textsc{poss} \textsc{prox} younger generation=\textsc{dat}=\textsc{emph} few must-say give must-teach\\
 `It is to the younger generation that we must explain it, must teach it.' (B060115cvs01)
\z
}

If the action is beneficial to the agent itself, instead of \trs{kaasi}{give}{}, \trs{ambel}{take}{} is used.

\xbox{16}{
\ea\label{ex:ptcpt:semrole:ben:ambel}
\gll [Tony Hassan {\em uncle}=nang asà-kaasi duvith] athi-\textbf{mintha} \textbf{ambel}=si? \\
     Tony Hassan uncle=\textsc{dat} \textsc{cp}-give money \textsc{irr}-ask take=\textsc{interr}  \\
    `Shall I ask for the money you gave to uncle Tony Hassan?' (K071011eml01)
\z
}

In this case, the person performing the action of asking would also profit from it since he has a chance of getting the money.

In rare cases, \em ambel \em can also be used if the action profits other entities than the agent, as in \xref{ex:ptcpt:semrole:ben:ambel:cross}.

\xbox{16}{
\ea\label{ex:ptcpt:semrole:ben:ambel:cross}
\gll See=yang lorang=susamma diinging muusing sangke-habbis anà-\textbf{simpang} \textbf{ambel}. \\
    \textsc{1s}=\textsc{acc} \textsc{2pl}=\textsc{comit} cold season until-finish \textsc{past}-keep take \\
    `You have kept me together with you until the cold season was over.' (K070000wrt04)
\z
}

In this case of a family providing shelter for a bear, it is the bear who profits, not the family, but still \em ambel \em is used in this sentence.

The following four examples show the use of \em ambel \em and \em kaasi \em for indicating self-benefactive \xref{ex:func:semrole:ben:ambel:proto} or alter-benefactive \xref{ex:func:semrole:ben:kaasi:proto}. Note that the use of the vector verbs implies control, which renders the use of the dative impossible in sentences with these vector verbs.


\xbox{16}{
\ea
\gll Se=dang su-mirthi. \\
     \textsc{1s=dat} \textsc{past}-understand  \\
    `I understood.' (K081106eli01)
\z
} 


\xbox{16}{
\ea\label{ex:func:semrole:ben:ambel:proto}
\gll Se su-mirthi ambel. \\
     \textsc{1s} \textsc{past}-understand take  \\
    `I understood/I could understand.' (K081106eli01)
\z
} 


\xbox{16}{
\ea
\gll Se su-mirthi-king. \\
     \textsc{1s} \textsc{past}-understand-\textsc{caus}  \\
    `I made him understand.' (K081106eli01)
\z
} 

\xbox{16}{
\ea\label{ex:func:semrole:ben:kaasi:proto}
\gll Se su-mirthi-king kaasi. \\
     \textsc{1s} \textsc{past}-understand-\textsc{caus} give  \\
    `I made him understand.' (K081106eli01)
\z
} 



\subsubsection{Comitative}\label{sec:func:Comitative}
The comitative is coded by the postposition \em (=sà)saama \em \formref{sec:morph:=sesaama} \xref{ex:ptcpt:semrole:comit:intro}.

\xbox{16}{
\ea\label{ex:ptcpt:semrole:comit:intro}
\gll Se=ppe mma-baapa=le=\textbf{sàsaama}=jo arà-duuduk. \\
 \textsc{1s=poss} mother-father=\textsc{addit}=\textsc{comit}=\textsc{emph} \textsc{non.past}-stay\\
`It is together with my parents that I live.' (B060115prs06)
\z
}


% \em Sà \em in \em sàsaama \em does not necessarily refer to the speaker, even if the shape of \trs{see}{1s} would suggest that. The next example shows this, where \em sàsaama \em indicates the comitative role of Sindbad the Sailor (referred to by the pronoun \em incayang\em).
% 
% 
% \xbox{16}{
% \ea\label{ex:ptcpt:semrole:comit:3rdperson}
% \ea
% \gll Soojer pada  \textbf{incayang=sàsaama} Seelon=nang asà-dhaathang, \\
%      Europeans \textsc{pl} \textsc{3s}=\textsc{comit} Ceylon=\textsc{dat} \textsc{cp}-come \\
%  `The Europeans came together with him to Sri Lanka and'
% \ex
% \gll inni daganan=yang derang=le        anà-blaajar. \\ % bf
% 	\textsc{prox} trade=\textsc{acc} \textsc{3pl}=\textsc{addit} \textsc{past}-learn\\
%     `they also learned this trade.' (K060103nar01)
% \z
% \z
% }

% The use of the \em sà\em-part is optional, as the following example shows.
%
%
% \xbox{16}{
% \ea\label{ex:ptcpt:semrole:comit:dropse}
% \gll Soore=ka, Snow-white=le Rose-red=le derang=pe umma=\zero=\textbf{samma} appi dìkkath=ka arà-duuduk ambel.  \\
%       Evening=\textsc{loc} Snow.white=\textsc{addit} Rose.Red=\textsc{addit} \textsc{3pl}=\textsc{poss} mother=\textsc{comit} fire vicinity=\textsc{loc} \textsc{simult}-sit take \\
%     `In the evening, Snow White and Rose Red used to sit down next to the fire with their mother.'  (K070000wrt04)
% \z
% }\\

% In a wider sense, the comitative can be used for parties engaged on different sides of a fight. While they do not fight together, they engage jointly in the act of fighting.
% 
% \xbox{16}{
% \ea\label{ex:ptcpt:semrole:comit:fight}
% \ea
% \gll Cinggala  raaja=nang=le          anà-banthu. \\ % bf
%      Sinhala king=\textsc{dat}=\textsc{addit} \textsc{past}-help  \\
%     `(They) helped  the Sinhalese king'
% \ex
% \gll Mà-{\em fight}=nang       {\em British}=\textbf{saama}. \\
%      \textsc{inf}-fight=\textsc{dat} British=\textsc{comit} \\
%     `to fight with the British.'
% \ex
% \gll {\em British} oorang pada=nang=le na-banthu. \\ % bf
%      British man \textsc{pl}=\textsc{dat}=\textsc{addit}  \\
%     `and they helped the British men'
% \ex
% \gll Mà-{\em fight}=nang cinggala  raaja=\textbf{saama}. \\
%       \textsc{inf}-fight=\textsc{dat} Sinhala king=\textsc{comit} \\
%     `to fight with the Sinhala king.' (K051206nar04)
% \z
% \z
% } \\

Very often, simple coordination with \em =le \em \formref{sec:constr:CoordinationwithBooleanclitics} is used instead of the comitative.

\subsubsection{Purpose}\label{sec:func:Purpose}
Purpose is expressed by either an infinitive clause \formref{sec:cls:Purposiveclauses} \xref{ex:ptcpt:semrole:purp:ma1}\xref{ex:ptcpt:semrole:purp:ma2}, or the postposition \em =nang \em \formref{sec:morph:=nang} \xref{ex:ptcpt:semrole:purp:nang1}\xref{ex:ptcpt:semrole:purp:nang2}, or a combination thereof \xref{ex:ptcpt:semrole:purp:manang1}-\xref{ex:ptcpt:semrole:purp:manang3}.


\xbox{16}{
\ea \label{ex:ptcpt:semrole:purp:ma1}
\gll Blaakang Andare [Kandi=ka asduuduk Dikwella  arà-pii jaalang]=ka aayer \textbf{mà}-miinong Udamalala kampong=ka su-birthi. \\
      after Andare Kandy=\textsc{loc} from Dikwella \textsc{non.past}-go road=\textsc{loc} water \textsc{inf}-drink Udamalala village=\textsc{loc} \textsc{past}-stop\\
    `Then, Andare stopped on the street which leads from Kandy to Dikwella at the hamlet Udamalala to drink water.'  (K070000wrt03)
\z
}


\xbox{16}{
\ea \label{ex:ptcpt:semrole:purp:ma2}
\gll Arà-blaajar mlaayu ini buk \textbf{mà}-thuulis kiyang. \\
      \textsc{non.past}-learn Malay \textsc{prox} book \textsc{inf}-write \textsc{evid} \\
    `(Sebastian) is learning Malay to write this book of his, it seems.'  (K051222nar07)
\z
}


\xbox{16}{
\ea \label{ex:ptcpt:semrole:purp:nang1}
\gll De laaye hathu nigiri=nang anà-baapi, buunung-king=\textbf{nang}. \\
      3\textsc{s.impolite} other \textsc{indef} country=\textsc{dat} \textsc{past}-bring kill-\textsc{caus}=\textsc{dat}\\
    `They brought him to another country to have him executed.'  (K051206nar02)
\z
}


\xbox{16}{
\ea \label{ex:ptcpt:semrole:purp:nang2}
\gll Karang masiigith=nang  arà-pii  liima vakthu sbaayang=\textbf{nang}. \\
  now mosque=\textsc{dat} \textsc{non.past}-go five time pray=\textsc{dat}     \\
    `now he goes to the mosque five times (a day) to pray.' (K051220nar01)
\z
}

\xbox{16}{
\ea \label{ex:ptcpt:semrole:purp:manang1}
\gll {\em Freedom}=yang   \textbf{mà}-daapath=\textbf{nang}  kithang=nang   bannyak caape aada. \\
      freedom=\textsc{acc} \textsc{inf}-get=\textsc{dat} \textsc{1pl}=\textsc{dat} much tired exist \\
    `We have worked hard to get our freedom.' (N061031nar01)
\z
}


\xbox{16}{
\ea \label{ex:ptcpt:semrole:purp:manang2}
\gll Hathu haari, hathu oorang thoppi \textbf{mà}-juval=\textbf{nang} kampong=dering kampong=nang su-jaalang pii. \\
     \textsc{indef} day \textsc{indef} man hat \textsc{inf}-sell=\textsc{dat} village=\textsc{abl} village=\textsc{dat} \textsc{past}-walk go  \\
    `One day, a man walked from village to village to sell hats.'  (K070000wrt01)
\z
}


\xbox{16}{
\ea \label{ex:ptcpt:semrole:purp:manang3}
\gll Derang [dìkkathka aada laapang]=nang \textbf{mà}-maayeng=\textbf{nang} su-pii. \\
     \textsc{3pl} vicinity=\textsc{loc} exist ground=\textsc{dat} \textsc{inf}-play=\textsc{dat} \textsc{past}-go  \\
    `They went to play on the ground nearby.'  (K070000wrt04)
\z
}

These strategies can be found in the same idiolects. The following example shows the use of \em mà- \em with and without \em =nang \em in purposive clauses in one stretch of discourse by one speaker.

\xbox{16}{
\ea \label{ex:ptcpt:semrole:purp:double}
\ea
\gll Kandi=pe     raaja=nang   kitham=pe  inni     banthu-an  asà-kamauvan      se-aada. \\ % bf
     Kandy=\textsc{poss} king=\textsc{dat} \textsc{1pl}=\textsc{poss} \textsc{prox} help-\textsc{nmlzr} \textsc{cp}-want \textsc{past}-exist  \\
    `The Kandyan king wanted this help of ours'
\ex
\gll Inni raaja=yang   \textbf{mà}-jaaga=\zero{}. \\
    \textsc{prox} king=\textsc{acc} \textsc{inf}-protect\\
`to protect this king.'
\ex
\gll [Itthu   \textbf{mà}-jaaga\textbf{=nang} anà-baa mlaayu]=dering   satthu oorang=jo    se. \\
 \textsc{dist} \textsc{inf}-protect=\textsc{dat} \textsc{past}-bring Malay]=\textsc{abl} one man=\textsc{emph} 1s\\
`One of the Malay men brought to protect that (man) is me.' (K060108nar02)
\z
\z
}

While purpose is most often marked on clauses, it is also possible to mark it on NPs, as is the case in the following example, where the noun \trs{thakuthan}{fear} hosts the postposition \em =nang\em.


\xbox{16}{
\ea \label{ex:ptcpt:semrole:purp:NP}
\gll {\em Second} {\em world} {\em war}  {\em time}=ka  Kluu\u mbu=nang   {\em Japanese} arà-{\em bomb}-king [thakuth-an]=\textbf{nang}. \\
      second world war time=\textsc{loc} Colombo=\textsc{dat} Japanese \textsc{non.past}-bomb-\textsc{caus} fear-\textsc{nmlzr}=\textsc{dat} \\
    `During the second world war, the Japanese bombed Colombo to cause fear.' (N060113nar03)
\z
}



\subsubsection{Cause and reason}\label{sec:func:Causeandreason}
Cause is normally coded by the postposition \em subbath \em \formref{sec:morph:=subbath}.

\xbox{16}{
\ea \label{ex:ptcpt:semrole:caus:n}
\gll [Non-Muslims pada]=\textbf{subbath} kithang muuka konnyong arà-cunjikang siini. \\
      non-Muslims \textsc{pl}=because \textsc{1pl} face little \textsc{non.past}-show here \\
    `Because of the non-Muslims we do not wear the veil.' (K061026prs01)
\z
}

\xbox{16}{
\ea \label{ex:ptcpt:semrole:caus:pron}
\gll Lorang [see]=\textbf{subbath} ithu Aajuth=yang su-salba-king. \\
      \textsc{2pl} \textsc{1s}=because \textsc{dist} dwarf=\textsc{acc} \textsc{past}-safe-\textsc{caus} \\
    `You saved that dwarf because of me.'  (K070000wrt04)
\z
}

\xbox{16}{
\ea \label{ex:ptcpt:semrole:caus:cl}
\gll [Ini oorang giini kapang-jaalang pii caape]=\textbf{subbath} jaalang hathu pii\u n\u ggir=ka anà-aada hathu pohong baava=ka su-see\u nder. \\
    \textsc{prox} man this.way then-walk go tired=because road \textsc{indef} border=\textsc{loc} \textsc{past}-exist.inanim \textsc{indef} tree down=\textsc{loc} \textsc{past}-rest   \\
    `Because he was tired from walking then, this man sat down under a tree which stood at the side of the street.'  (K070000wrt01)
\z
}

\xbox{16}{
\ea \label{ex:ptcpt:semrole:caus:deic}
\gll Suda \textbf{itthusubbath}=jo,   se laile        {\em Marine} {\em Engineering}                asà-kijja ambel arà-pii. \\
   so therefore=\textsc{emph} \textsc{1s} again Marine Engineering \textsc{cp}-make take \textsc{non.past}-go    \\
    `So, it was because of that that I took up again the Marine Engineering work and went (away).'  (K051206nar20)
\z
}



Another possibility is the use of \trs{lanthran}{because} \formref{sec:morph:=lanthran} \xref{ex:ptcpt:semrole:reason:lanthran:clause}\xref{ex:ptcpt:semrole:reason:lanthran:NP}.

\xbox{16}{
\ea \label{ex:ptcpt:semrole:reason:lanthran:clause}
\gll Derang hathu suurath nya-kiiring [see ini Kandi Mlaayu {\em Association}=dering nya-kiisar]=\textbf{lanthran}. \\
      \textsc{3pl} \textsc{indef} letter \textsc{past}-send \textsc{1s} \textsc{prox} Kandy Malay association=\textsc{abl} \textsc{past}-go.aside=\textbf{because} \\
    `They had written a letter because I had left the Kandy Malay Association.' (K061122nar03)
\z
}

\xbox{16}{
\ea\label{ex:ptcpt:semrole:reason:lanthran:NP}
\ea
\gll Beeso luusa lubaarang arà-dhaathang. \\ % bf
     tomorrow later.in.the.future festival \textsc{non.past}-come  \\
    `The day after tomorrow is the festival.'
\ex
\gll Itthu=\textbf{lanthran} kithang=pe ruuma see arà-cuuci. \\
       \textsc{dist}=because \textsc{1pl}=\textsc{poss} house \textsc{1s} \textsc{non.past}-clean\\
    `That's why I am cleaning the house.' (K061019prs01)
\z
\z
}

Finally, a third possibility is \em =sikin \em \formref{sec:morph:=sikin}.


\xbox{16}{
\ea\label{ex:ptcpt:semrole:reason:sikin}
\gll Inni aari pada=ka kithang=nang {\em test}. Itthu=\textbf{siking}=jo see thàrà-kiiring  \\
      \textsc{prox} day \textsc{pl}=\textsc{loc} \textsc{1pl}=\textsc{dat} test \textsc{dist}=because=\textsc{emph} \textsc{1s} \textsc{neg.past}-send \\
    `We are having tests these days. That is why I have not sent (anything).'  (K071203eml01)
\z      
} 


\subsubsection{Value}\label{sec:func:Value}
The value of an item is indicated by \em =nang \em \formref{sec:morph:=nang}. In \xref{ex:ptcpt:semrole:value}, the value of the stones indicated earlier in discourse is said to be \trs{baaye lakuvan}{a good price}. This NP carries \em =nang\em.

\xbox{16}{
\ea\label{ex:ptcpt:semrole:value}
\gll [Baaye  lakuvan]=\textbf{nang}    anà-juuval. \\
     good wealth=\textsc{dat} \textsc{past}-sell  \\
    `(He) sold (them) for a good price.' (060103nar01)
\z
}



% \xbox{16}{
% \ea\label{ex:func:unreferenced}
% \gll See lorang=nang arà-simpa kaapang=ke see lorang=nang ithu uuthang arà-baayar katha. \\
%     \textsc{1s}  \textsc{2pl}=\textsc{dat} \textsc{non.past}-promise when=\textsc{simil} \textsc{1s} \textsc{2pl}=\textsc{dat} \textsc{dist} debt \textsc{non.past}-pay \textsc{quot} \\
%     `I promise you that I will pay back that debt to you some time (lit. somewhen).' (K070000wrt04)
% \z
% } \\


\subsubsection{Portion}\label{sec:func:Portion}
When something is divided into portions, the fraction is also indicated by \em =nang \em \formref{sec:morph:=nang}, as is the case in dividing meat into three portions in the example below

\xbox{16}{
\ea\label{ex:ptcpt:semrole:portion}
\gll Thapi ithu \textbf{thiiga=nang} arà-baagi. \\
     but \textsc{dist} three=\textsc{dat} \textsc{non.past}-divide  \\
 `But you divide it into three (parts). (K060112nar01)
\z
}


\subsubsection{Set domain}\label{sec:func:Setdomain}

The superset among which a member is chosen is indicated in English by \em among\em. In SLM, this is either indicated by the locative \em =ka \em \formref{sec:morph:=ka} \xref{ex:ptcpt:semrole:setdomain:ka1}  or the ablative \em =dering \em \formref{sec:morph:=dering} \xref{ex:ptcpt:semrole:setdomain:dering1}-\xref{ex:ptcpt:semrole:setdomain:dering3}.


\xbox{16}{
\ea \label{ex:ptcpt:semrole:setdomain:ka1}
\gll \textbf{Mlaayu=ka}=jo  bannyak avuliya Seelong=ka   aada. \\
      Malay=\textsc{loc}=\textsc{emph} many saint Ceylon=\textsc{loc} exist \\
    `It is among the Malays that there are many saints in Sri Lanka.' (K060108nar02)
\z
}

\xbox{16}{
\ea \label{ex:ptcpt:semrole:setdomain:dering1}
\ea
\gll Karang inni     {\em Kandy} nigiri=ka aada  mlaayu avuliya pada. \\ % bf
     now \textsc{prox} Kandy town=\textsc{loc} exist Malay saint \textsc{pl}  \\
    `Now there are Malay saints here in Kandy.'
\ex
\gll Derang=\textbf{dering}  hatthu avuliya  dhaathang aada sini=ka dìkkath. \\
     \textsc{3pl}=\textsc{abl} \textsc{indef} saint come exist here=\textsc{loc} vicinity  \\
    `Among them there is one saint close to here.' (K051220nar01)
\z
\z
}

\xbox{16}{
\ea \label{ex:ptcpt:semrole:setdomain:dering2}
   \gll [Itthu    mà-jaaga=nang        anà-baa      mlaayu]=\textbf{dering}    satthu      oorang=jo    see. \\
 \textsc{dist} \textsc{inf}-protect=\textsc{dat} \textsc{past}-bring Malay=\textsc{abl}  one man=\textsc{emph} 1s\\
`One of the Malays brought to protect him is me' (K060108nar02)
\z
}

\xbox{16}{
\ea \label{ex:ptcpt:semrole:setdomain:dering3}
\gll Seelong=pe  makanan=\textbf{dering}         aapa makanan        suuka? \\
     Ceylon food=\textsc{abl} what food like  \\
    `What food do you like within Sri Lankan cuisine?' (B060115cvs02)
\z
}

\subsubsection{Temporal domain}\label{sec:func:Temporaldomain}
The temporal domain in which an event occurs is indicated by \em =nang \em \formref{sec:morph:=nang}. In the following examples, this is the week and the day. There is usually some quantifying element elsewhere in the clause indicating the relation, like \trs{duuva skalli}{two times}{} or \trs{ùmpathblas kaayu}{fourteen miles}.

\xbox{16}{
\ea \label{ex:ptcpt:semrole:tempdomain1}
\gll Suda \textbf{hathu}  \textbf{{\em week}=nang}   duuva skali  arà-dhaathang    {\em daughter}. \\
     so one week=\textsc{dat} two time \textsc{non.past}-come daughter  \\
    `Thus my daughter comes twice a week.' (K051201nar01)
\z
}

\xbox{16}{
\ea \label{ex:ptcpt:semrole:tempdomain2}
\gll Kithang \textbf{hathu}  \textbf{{\em week}=nang}  hathu skali duva skaali=ke arà-maakang. \\
     \textsc{1pl} one week=\textsc{dat} one time two time=\textsc{simil} \textsc{non.past}-eat  \\
    `We might eat it once or twice a week.' (K061026rcp04)
\z
}

\xbox{16}{
\ea \label{ex:ptcpt:semrole:tempdomain3}
\gll Derang su-jaalang  ùmpath-blas kaayu \textbf{hathu} \textbf{aari=nang}. \\
      \textsc{3pl} \textsc{past}-walk four-teen mile one day=\textsc{dat} \\
    `They walked fourteen miles a day.' (K051213nar03)
\z
}


\subsubsection{Duration}\label{sec:func:Duration}
Duration is normally not overtly marked. In the following two examples, the time spans \trs{thuuju thaaun}{seven years}{} and \trs{spuulu thaaun}{ten years}{} do not receive any special marking.

\xbox{16}{
\ea \label{ex:ptcpt:semrole:duration1}
\gll See  [\textbf{thuuju} \textbf{thaaun}]=\zero{} luvar nigiri=ka asà-duuduk  karang abbis dhaathang aada. \\
      \textsc{1s} seven year outside country=\textsc{loc} \textsc{cp}-exist.\textsc{anim} now finish coming exist \\
    `After having stayed abroad seven years, I have now arrived.' (B060115prs13)
\z
}


\xbox{16}{
\ea \label{ex:ptcpt:semrole:duration2}
\gll See [\textbf{spuulu} \textbf{thaaun}]=\zero{} siini sri Lanka=ka pukurjan nya-kirja. \\
     \textsc{1s} ten year here Sri Lanka=\textsc{loc} work \textsc{past}-do  \\
    `I  worked here in Sri Lanka for ten years.' (K061026prs01)
\z
}

If the duration shall be emphasized, \em =dering \em \formref{sec:morph:=dering} can be used.


\xbox{16}{
\ea \label{ex:ptcpt:semrole:duration:emph}
\gll [Bannyak aari]=\textbf{dering} saapa=yang=ke thàrà-enco-kang. \\
      many day=\textsc{abl} who=\textsc{acc}=\textsc{simil} \textsc{neg.past}-fool-\textsc{caus} \\
    `For how many days have I not fooled anybody!' (K070000wrt05a)
\z
}

\subsubsection{Role}\label{sec:func:Role}
The semantic role of `role', e.g. for profession or functions, is indicated by the noun \trs{caara}{way}.

\xbox{16}{
\ea\label{ex:func:unreferenced}
\gll Seelong=nang  {\em exiles} \textbf{caara} nya-kiiring. \\
      Ceylon=\textsc{dat} exiles way \textsc{past}-send \\
    `The soldiers were sent to Sri Lanka as exiles.' (K051213nar06)
\z
}

\subsubsection{Summary}\label{sec:func:Summary}
Table \ref{tab:semroles} gives an overview of the distribution of semantic roles on different morphemes


\begin{table}[h]
\begin{center}
% use packages: array
\begin{tabular}{cccccc}
sduuduk		&	=dering 	&	=\zero{} 	&	=yang	& 	=nang 	& 	=ka 		\\
		&\framebox[2cm]{INSTR}	&  \multicolumn{3}{c}{\framebox[6.8cm]{PAT}} 		&\framebox[2cm]{LOC}	\\
		&\multicolumn{2}{c}{\framebox[4.4cm]{AGENT}}	&  		&\multicolumn{2}{c}{\framebox[4.4cm]{GOAL}}\\
\multicolumn{2}{c}{\framebox[4.2cm]{SRC}}&\multicolumn{2}{c}{\framebox[4.4cm]{THEME}}& \framebox[2cm]{REC}  		\\
		&\multicolumn{2}{c}{\framebox[4.4cm]{TEMP DOMAIN}}&		& \framebox[2cm]{BEN}	 		\\
		&\framebox[2cm]{PATH} 	&\framebox[1.6cm]{ROLE}	& 		& \framebox[2cm]{PURP}  		\\
		&\framebox[2cm]{~SET DOM.\begin{picture}(0,0)(0,0)
					\put(1,6){\line(1,0){225}}
					\put(1,4){\line(1,0){225}}
					\end{picture}
					}&			&		& 		&\framebox[2cm]{SET DOM.}\\
		&\multicolumn{2}{c}{\framebox[4.4cm]{DURATION}}	& 		& \framebox[2cm]{EXP}    		\\
		&			&			& 		& \framebox[2cm]{PORTION} 		\\
		&			&			& 		& \framebox[2cm]{VALUE} 		\\
\end{tabular}
\end{center}
\caption[Repartition of semantic roles on morphemes]{Repartition of semantic roles on morphemes. \em =nang \em is used to express eight different semantic roles, \em =dering \em is used for seven, while the other morphemes have lower numbers of semantic roles they can express. Eight semantic roles can be expressed by more than one morpheme. The two morphemes which can be used for the role \textsc{set domain} are linked by a line because for typographical reasons they could not be made contiguous. The semantic roles of \textsc{comitative}, \textsc{role}, and \textsc{reason} are left out of the graphic.}
\label{tab:semroles}
\end{table}



\subsection{Mismatches between number of semantic roles and number of syntactic arguments}\label{sec:func:Mismatchesbetweennumberofsemanticrolesandnumberofsyntacticarguments}
It is possible that the predicate demands more semantic roles than there are participants in the discourse. This is most notably the case for reflexives \funcref{sec:func:Reflexive}, self-benefactives \funcref{sec:func:Self-benefactive} and reciprocals \funcref{sec:func:Reciprocals}.  The opposite is that a semantic role applies to more than one participant. In that case, the participants are coordinated \funcref{sec:func:Morethanoneentityinaterm}.


\subsubsection{Reflexive}\label{sec:func:Reflexive}
If one term has more than one role (as is the case of reflexives), there exist three possibilities:  Use the vector verb \em ambel \em \formref{sec:wc:ambel}, use the emphatic clitic \em =jo\em \formref{sec:morph:=jo}, or separate the term into two non-identical subterms. These possibilities will be discussed in turn.

The first possibility is to use the vector verb \em ambel \em to express reflexive action, as in \xref{ex:func:refl:ambel:buunung} and \xref{ex:func:refl:ambel:banthu}. The case assigned by the main verb remains unaffected by this (accusative by \trs{buunung}{kill}, dative by \trs{banthu}{help}). The emphatic marker \em =jo \em can be used optionally.


\xbox{16}{
\ea\label{ex:func:refl:ambel:buunung}
\gll Incayang incayang=\textbf{yang}(=jo) su-buunung \textbf{ambel}. \\
      \textsc{3s.polite} \textsc{3s.polite}=\textsc{acc}=\textsc{emph} \textsc{past}-kill take \\
    `He killed himself.' (K081106eli01) %ambel must be there
\z
}


\xbox{16}{
\ea\label{ex:func:refl:ambel:banthu}
\gll Incayang incayang=\textbf{nang}=(jo) su-banthu \textbf{ambel}. \\
      \textsc{3s.polite} \textsc{3s.polite}=\textsc{acc}=\textsc{emph} \textsc{past}-help take \\
    `He helped himself.' (K081106eli01)
\z
}

The second possibility, already alluded to above, is the use of the emphatic clitic \em =jo\em. This is obligatory in contexts where \em ambel \em is not possible, for instance in some possessive contexts \xref{ex:func:refl:jo1}\xref{ex:func:refl:jo2}. The emphatic clitic can attach either to the possessor or to the possessee, entailing slight changes in meaning, rendered by `he himself' and `his own' in English.



\xbox{16}{
\ea\label{ex:func:refl:jo1}
\gll \textbf{Incayang=pe=jo} ruuma=yang incayang su-ronthok-king. \\
      \textsc{3s.polite}=\textsc{poss}=\textsc{emph} house=\textsc{acc} \textsc{3s.polite} \textsc{past}-demolished-\textsc{caus} \\
    `He demolished his own house.' (K081106eli01)
\z
}


\xbox{16}{
\ea\label{ex:func:refl:jo2}
\gll Incayang=pe ruuma=yang \textbf{incayang=jo} su-ronthok-king. \\
     \textsc{3s.polite} house=\textsc{acc} \textsc{3s.polite}=\textsc{emph} \textsc{past}-demolished-\textsc{caus}  \\
    `He himself demolished his house.' (K081106eli01)
\z
}



The third possibility is to use an additional term to express the undergoer in a more precise manner, thus making it different from a real reflexive construction. The additional term is normally in a meronymic relationship to the main term, like \trs{diiri}{body} to the speaker in \xref{ex:func:refl:diiri:yang} and \xref{ex:func:refl:diiri:nang}. The normal case markers are used in this construction as assigned by the verb.
Even in this case, \em ambel \em and \em =jo \em are normally present.


\xbox{16}{
\ea\label{ex:func:refl:diiri:yang}
\gll Se se=ppe diiri=yang(=jo) su-poothong ambel. \\
     \textsc{1s} \textsc{1s=poss} body=acc(=foc) past-cut take  \\
    `I cut myself.' (K081106eli01)
\z
}


\xbox{16}{
\ea\label{ex:func:refl:diiri:nang}
\gll Se se=ppe diiri=nang(=jo) su-puukul ambel. \\
     \textsc{1s} \textsc{1s=poss} body=\textsc{dat}=\textsc{emph} \textsc{past}-hit take  \\
    `I hit myself.' (K081106eli01)
\z
}



% \xbox{16}{
% \ea
% \gll Incayang pe haal nang incayang su dhaathang. \\
%        \\
%     `He came by his own wish.' (K081106eli01)
% \z
% } \\



\subsubsection{Self-benefactive}\label{sec:func:Self-benefactive}
It is possible that an agent performs an action of which he himself benefits. In English, a sentence like \em John bought himself a car \em comes close to this, since \em John bought a car \em normally already pragmatically implies, in absence of other context, that John will be the owner of the car. This fact can be highlighted by \em himself \em in English. In SLM, a different construction is used for actions which profit the agent, namely the vector verb \em ambel \em \formref{sec:wc:ambel}.


\xbox{16}{
\ea\label{ex:ptcpt:mismatch:selfben:ambel}
\gll [Tony Hassan {\em uncle}=nang asà-kaasi duvith] athi-mintha \textbf{ambel}=si? \\ % bf
     Tony Hassan uncle=\textsc{dat} \textsc{cp}-give money \textsc{irr}-ask take=\textsc{interr}  \\
    `Shall I ask for the money you gave to uncle Tony Hassan (for my own benefit)?' (K071011eml01)
\z
}

In example \xref{ex:ptcpt:mismatch:selfben:ambel}, the use without \em ambel \em would be perfectly fine, but the use with \em ambel \em highlights the fact that the speaker would perform an action which would profit himself. It is difficult to argue for this example that it changes the valency of the verb \trs{mintha}{ask}. The verb is still trivalent (asker, askee, theme), but the emphasis is more on the asker than on the askee or the theme.

Note that overt coding of the self-benefactive is optional, as shown by the following example without \em ambel\em.


\xbox{16}{
\ea\label{ex:ptcpt:mismatch:selfben:noambel}
\gll Derang pada arà-mintha \zero{}   nigiri. \\ % bf
      \textsc{3pl} \textsc{pl} \textsc{non.past}-ask { } country \\
    `They are asking for a country (of their own).' (K051206nar10)
\z
}


%
% \xbox{16}{
% \ea
% \gll Se se=dang(jo) hatthu kaar su bìlli (ambel). \\
%        \\
%     `.' (K081106eli01)
% \z
% } \\
%
%
%
% \xbox{16}{
% \ea
% \gll Seeyang {\em university} ka su ambel bìrrath. \\
%        \\
%     `.' (K081106eli01)
% \z
% } \\

\em Ambel \em can only be used if the beneficiary is actively involved. For \trs{daapath}{find} in \xref{ex:ptcpt:mismatch:selfben:daapath}, this is not the case. From the context, we do not learn that the speaker was actively involved in getting a new job, hence \em ambel \em cannot be used. Also note the dative marking on \trs{se}{1s}, which indicates lack of control.

\xbox{16}{
\ea\label{ex:ptcpt:mismatch:selfben:daapath}
\gll Se=dang baaru hatthu idopan su-daapath (*ambel). \\
    \textsc{1s=dat} new \textsc{indef} job \textsc{past}-get take   \\
    `I got a new job.' (K081106eli01)
\z
}

If the speaker has found a new job after being actively involved in finding it, \em ambel \em is possible \xref{ex:ptcpt:mismatch:selfben:sucaariambel}. In this case, nominative (\zero) is used instead of the dative, taking away the semantics of lack of control.\footnote{The difference between searching and finding is not marked lexically in SLM, but only aspectually. If searching is completed, this implies finding.}


\xbox{16}{
\ea\label{ex:ptcpt:mismatch:selfben:sucaariambel}
\gll Se=\zero{} baaru hatthu idopan su-caari  ambel. \\
     \textsc{1s} new \textsc{indef} job \textsc{past}-look.for take  \\
    `I found a new job.' (K081106eli01)
\z
}

On the other hand, if the process of finding a job is not finished yet, \em ambel \em cannot be used. In \xref{ex:ptcpt:mismatch:selfben:aracaariambel}, we find the non-past marker \em arà-\em, indicating that at the time of speaking the process of finding/searching is still ongoing. The job has not been found yet, hence no benefit is materialized, and \em ambel \em cannot be used.

\xbox{16}{
\ea\label{ex:ptcpt:mismatch:selfben:aracaariambel}
\gll Se baaru hatthu idopan arà-caari (*ambel). \\
     \textsc{1s} new \textsc{indef} job non.\textsc{past}-look.for take  \\
    `I am looking for a new job.' (K081106eli01)
\z
}



\subsubsection{Reciprocals}\label{sec:func:Reciprocals}
In a reciprocal situation, the participants taking part in the reciprocity are both actor and undergoer (or recipient) and their double involvement in the event is indicated by two occurrences of the indefiniteness marker \em hatthu \em \formref{sec:morph:Indefinitenessclitic}. The following three examples show this for nominal predications.

\xbox{16}{
\ea\label{ex:func:reciproc:muhabbath}
\gll Oorang pada \textbf{hatthu}=\textbf{nang} \textbf{hatthu} muhabbath. \\
      man \textsc{pl} \textsc{indef}=\textsc{dat} \textsc{indef} love \\
    `People like each other.' (K081106eli01)
\z
}

\xbox{16}{
\ea\label{ex:func:reciproc:maara}
\gll Oorang pada \textbf{hatthu}=\textbf{nang} \textbf{hatthu} maara. \\
      man \textsc{pl} \textsc{indef}=\textsc{dat} \textsc{indef} angry \\
    `People are angry with each other.' (K081106eli01)
\z
}


\xbox{16}{
\ea\label{ex:func:reciproc:precaaya}
\gll Oorang pada \textbf{hatthu}=\textbf{nang} \textbf{hatthu} buthul percaaya. \\
     man \textsc{pl} \textsc{indef}=\textsc{dat} \textsc{indef} correct trust  \\
    `People (have) trust in each other.' (K081106eli01)
\z
}

With verbal predications, the vector verb \em ambel \em can also be used to convey reciprocal semantics \xref{ex:func:reciproc:ambel1}\xref{ex:func:reciproc:ambel2}.


\xbox{16}{
\ea\label{ex:func:reciproc:ambel1}
\gll Ini  nigiri=pe oorang pada bannyak arà-buunung \textbf{ambel}. \\
      \textsc{prox} country=\textsc{poss} man \textsc{pl} much \textsc{non.past}-kill take \\
    `The people in this country kill each other.' (K081106eli01)
\z
}

\xbox{16}{
\ea\label{ex:func:reciproc:ambel2}
\gll Faarok=le Oomar=le su-buunung \textbf{ambel}. \\
      Faarok=\textsc{addit} Oomar=\textsc{addit}  \textsc{past}-kill take \\
    `Farook and Oomar killed each other.' (K081106eli01)
\z
}

Both possibilities (\em hatthunang hatthu \em and \em ambel\em) can be combined \xref{ex:func:reciproc:ambel1nang1}.





\xbox{16}{
\ea\label{ex:func:reciproc:ambel1nang1}
\gll Oorang pada \textbf{hatthu=nang} \textbf{hatthu} arà-ciong \textbf{ambel}. \\
      man \textsc{pl} \textsc{indef}=\textsc{dat} \textsc{indef} \textsc{non.past}-kiss take \\
    `People kiss each other.' (K081106eli01)
\z
}


The verb \trs{ciong}{kiss} in \xref{ex:func:reciproc:ambel1nang1} normally governs the dative, which aligns nicely with the dative case marker \em =nang \em found on \em hatthu \em in \xref{ex:func:reciproc:ambel1nang1}.  There are other verbs which govern the accusative (\em =yang\em), like \trs{buunung}{kill} in \xref{ex:func:reciproc:ambel1nang:acc:intro}. If these verbs are used in the reciprocal construction \xref{ex:func:reciproc:ambel1nang:acc}, the \em =nang \em assigned by reciprocity takes precedence over the \em =yang \em assigned by the verb. We find \em hatthunang hatthu \em and not \em *hatthuyang hatthu\em.


\xbox{14}{
\ea\label{ex:func:reciproc:ambel1nang:acc:intro}
\gll Oorang pompang=\textbf{yang} arà-buunung \\
     man female=\textsc{acc} \textsc{non.past}-kill  \\
    `The man kills the woman.' (K081106eli01)
\z
}

\xbox{16}{
\ea\label{ex:func:reciproc:ambel1nang:acc}
\gll Oorang pada hatthu=\textbf{nang} hatthu arà-buunung ambel. \\
      man \textsc{pl} \textsc{indef}=\textsc{dat} \textsc{indef} \textsc{non.past}-kill take \\
    `People kill each other.' (K081106eli01) %must be nang
\z
}


If through the use of  a modal \em =nang \em is assigned, as in \xref{ex:func:reciproc:ambel1nang:acc:boole}, the verb returns to assigning its normal role, accusative in \xref{ex:func:reciproc:ambel1nang:acc:boole}.

\xbox{16}{
\ea\label{ex:func:reciproc:ambel1nang:acc:boole}
\gll Oorang pada=nang hatthu (*nang) hatthu=yang bole-buunung. \\
      man \textsc{pl}=\textsc{dat} \textsc{indef} \textsc{dat} \textsc{indef}=\textsc{acc} can-kill \\
    `People can kill each other.' (K081106eli01)
\z
}



If the reciprocal involves other roles than actor and undergoer, such as recipient in \xref{ex:func:reciproc:ambel1nang:rec}, the position of the case marker is on the second item, not on the first as above in \xref{ex:func:reciproc:ambel1nang1} and \xref{ex:func:reciproc:ambel1nang:acc}.

\xbox{16}{
\ea\label{ex:func:reciproc:ambel1nang:rec}
\gll Oorang pada hatthu=\zero{} hatthu oorang=\textbf{nang} duvith arà-\textbf{kaasi} \textbf{ambel}. \\
     man \textsc{pl} \textsc{indef} \textsc{indef} man=\textsc{dat} money \textsc{non.past}-give take   \\
    `People give money to each other.' (K081106eli01)
\z
}

Note the occurrence of the vector verb \trs{ambel}{take} together with the full verb \trs{kaasi}{give}  in \xref{ex:func:reciproc:ambel1nang:rec}. These normally have opposite semantics, but the semantic content of \em ambel \em is bleached in this construction, and it fulfills a grammatical function instead of contributing semantic content to the propositions.

If the case marker which would be used in a non-reciprocal clause is neither zero nor \em =yang \em nor \em =nang\em, it is omitted in the reciprocal construction. The verb \trs{thaanya}{ask} normally assigns the locative \em =ka \em to the askee. This is not possible in the reciprocal construction, and hence there is no case marker found in \xref{ex:func:reciproc:ambel1nang:loc}.

\xbox{16}{
\ea\label{ex:func:reciproc:ambel1nang:loc}
\gll Oorang pada hatthu hatthu arà-thaanya ambel. \\
     man \textsc{pl} \textsc{indef} \textsc{indef} \textsc{non.past}-ask take  \\
    `People ask each other (questions).' (K081106eli01)
\z
}

An exception to this rule might be the ablative case marker \em =dering\em, which seems to be able to be used prenominally in a reciprocal construction.\footnote{This is very surprising, given that it is normally a \em post\em position. This example should be taken with a grain of salt, but might be a worthwhile starting point for future investigations of reciprocity.}

\xbox{16}{
\ea\label{ex:func:reciproc:ambel1nang:dering}
\gll (dering) oorang pada hatthu hatthu duvith arà-mintha ambel. \\
     \textsc{abl} man \textsc{pl} \textsc{indef} \textsc{indef} money \textsc{non.past}-ask take  \\
    `people ask/take money from  each other.' (K081106eli01)
\z
}

The reciprocal must be distinguished from joint reflexive action as in \xref{ex:func:reciproc:contr:refl}.

\xbox{16}{
\ea\label{ex:func:reciproc:contr:refl}
\gll Farook=le Oomar=le derang derang=yang su-buunung ambel. \\
      Farook=\textsc{addit} Oomar=\textsc{addit} 3 3=\textsc{acc} \textsc{past}-kill take \\
    `Farook and Oomar killed themselves individually.' (K081106eli01)
\z
}



% \xbox{16}{
% \ea
% \gll Ini nigiripe oorang pada butthul saayang sukahaannang arà baapi haari. \\
%        \\
%     `.' (K081106eli01)
% \z
% } \\



%
% \xbox{16}{
% \ea
% \gll Farookle Oomarle derampe saala deringjo hatthu {\em car} {\em accident} asà jaadi sumnii\u n\u ggal. \\
%        \\
%     `Farook and Oomar died in a car accident due to their own fault.' (K081106eli01)
% \z
% } \\





% \xbox{16}{
% \ea
% \gll Oorang pada hatthu hatthuyang arà buunung. \\
%        \\
%     `.' (nosource)
% \z
% } \\







\subsubsection{More than one entity in a term}\label{sec:func:Morethanoneentityinaterm}
If a term consists of more than one entity, these entities are conjoined by a semantically appropriate coordinating construction \formref{sec:constr:Coordinatingconstructions}. The zero coordination \xref{ex:ptcpt:mismatch:coord:zero} and the coordination with a clitic \xref{ex:ptcpt:mismatch:coord:le} are shown below.

%
% \xbox{16}{
% \ea\label{ex:func:unreferenced}
% \gll Se=ppe umma\textbf{=le}  aade\textbf{=le} aade=pe duuva aanak\textbf{=le} kitham=pe baapa=pe aade\textbf{=le} baapape aade\textbf{le} aanak\textbf{=le} thiiga aanak\textbf{=le}, bannyak aanak pada anà-nii\u n\u ggal. \\
% \textsc{1s=poss} mother=\textsc{addit} younger.sibling=\textsc{addit} younger.sibling=\textsc{poss} two child=\textsc{addit} \textsc{1pl}=\textsc{poss} father=\textsc{poss} younger.sibling=\textsc{addit} father=\textsc{poss} younger.sibling=\textsc{addit} child=\textsc{addit} three child=\textsc{addit} many child \textsc{pl} \textsc{past}-die\\
% `.' (nosource)
% \z
% }



\xbox{16}{
\ea\label{ex:ptcpt:mismatch:coord:zero}
\ea
\gll Mlaayu pada  duuduk=apa \\ % bf
 Malay \textsc{pl} stay=after\\
 `After the Malays had settled down'
\ex
 \gll spaaru  mlaayu pada   \textbf{Singgapur=\zero}       \textbf{Indonesia=\zero}  \textbf{{\em Malaysia}=\zero} anà-pii. \\
 some Malay \textsc{pl} Singapore Indonesia Malaysia \textsc{past}-go\\
`(only) some Malays went (back) to Singapore, Indonesia or Malaysia.' (K051213nar07)
\z
\z
}


\xbox{16}{
\ea\label{ex:ptcpt:mismatch:coord:le}
\gll \textbf{Snow-White}=\textbf{nang}=\textbf{le} \textbf{Rose-red}=\textbf{nang}=\textbf{le} ini hatthu=ke thàrà-mirthi. \\
     Snow.white=\textsc{dat}=\textsc{addit} Rose.Red=\textsc{dat}=\textsc{addit}  \textsc{prox} \textsc{indef}=\textsc{simil} \textsc{neg.past}-understand\\
    `Snow White and Rose Red did not understand a thing.'  (K070000wrt04)
\z
 }\\

%
% \xbox{16}{
% \ea\label{ex:func:unreferenced}
% \gll Derang=pe umma derang=nang jaith-an=le, jaarong pukurjan=le su-aajar. \\
%        \textsc{3pl}=\textsc{poss} mother \textsc{3pl}=\textsc{dat} sew-\textsc{nmlzr}=\textsc{addit} needle work=\textsc{addit} \textsc{past}-teach\\
%     `Their mother taught them sewing and needle work.'  (K070000wrt04)
% \z
% }\\
%
%
% \xbox{16}{
% \ea\label{ex:func:unreferenced}
% \gll Lorang se=dang mà-iidop thumpath kala-kaasi see lorang=nang  lorang=pe samma duvith=le baarang pada=le anthi-bale-king. \\
%        \textsc{2pl} \textsc{1s=dat} \textsc{inf}-stay place if-give \textsc{1s} \textsc{2pl}=\textsc{dat} \textsc{2pl}=\textsc{poss} all money=\textsc{addit} goods   \textsc{pl}=\textsc{addit} \textsc{irr}=return-caus\\
%     `If you give me a place to stay, I will return all your money and goods to you.' (K070000wrt04)
% \z
% } \\
%
% \xbox{16}{
% \ea\label{ex:func:unreferenced}
% \gll Ketham pada     makanan pada\textbf{=si }    pakeyan   pada\textbf{=si } su-baavang. \\
%  \textsc{1pl} \textsc{pl} food \textsc{pl}=disj clothing \textsc{pl}=disj \textsc{past}-bring\\
% `We brought food or clothing .' (B060115nar02)
% \z
% }
%
%
% \xbox{16}{
% \ea\label{ex:func:unreferenced}
% \gll {\em Doctors} pada=so {\em police} a.s.p=so  {\em judge}=so, samma oorang thaau see=yang \\
%      doctors \textsc{pl}=\textsc{undet} police a.s.p=\textsc{undet} judge=\textsc{undet} all man know \textsc{1s}=\textsc{acc}  \\
%     `Whether they be doctors, police a.s.p.s or judges, all men know me.'  (B060115nar04)
% \z
% }\\
%
% \xbox{16}{
% \ea\label{ex:func:unreferenced}
% \gll Saudi=so, mana=ka=so; athu nigiri=ka. \\
%      Saudi.Arabia=\textsc{undet} where=\textsc{loc}=\textsc{undet} \textsc{indef}=country=\textsc{loc}  \\
%     `In Saudi Arabia or somewhere, in some country.'  (B060115nar02)
% \z
% }\\
%
%
% \xbox{16}{
% \ea\label{ex:func:unreferenced}
% \gll Suda go buthul baaye=nang bole=thaau, mà-jaaith=so mà-poothong=so. \\
%      thus \textsc{1s} correct good=\textsc{dat} can=know \textsc{inf}-sew=\textsc{undet} \textsc{inf}-cut=\textsc{undet}\\
%     `So I know (those things) very well, be it sewing, be it cutting.'  (B060115nar04)
% \z
% }\\

\subsection{Unknown participants}\label{sec:func:Unknownparticipants}
If a predicate semantically requires an argument, but the precise referential nature of this argument is not known, there are several possibilities: the referent is not important \funcref{sec:func:Unimportantreferent}, the referent is important and individuated \funcref{sec:func:Individuatedreferent}, the referent is important and categorial \funcref{sec:func:Unknowncategorialreferents}, the referent is important and generic \funcref{sec:func:Unknownreferents,generic}. A somewhat different possibility is that the speaker wants the hearer to provide the referent, by forming a question \funcref{sec:func:Queriedreferents}.

\subsubsection{Unimportant referent}\label{sec:func:Unimportantreferent}


Unimportant referents which might be required semantically do not have to be expressed in morphosyntax.
In example \xref{ex:ptcpt:unknown:drop1}, the author of the history book is not known and irrelevant. Therefore, no reference to the author is made.\footnote{It would be possible to read this sentence as `A Sinhala history book has written something', but since books rarely engage in activities such as writing, no hearer will seriously be tempted by this interpretation.}

\xbox{16}{
\ea \label{ex:ptcpt:unknown:drop1}
\gll \zero{} cinggala {\em history} {\em book} atthu thuulis aada. \\ % bf
 {} Sinhala history book atthu written exist\\
`Someone has written a Sinhala history book (about that)/A Sinhala history book has been written (about that).' (K051213nar06)
\z
}



\subsubsection{Individuated referent}\label{sec:func:Individuatedreferent}
If the referent is important and  individuated, but further information is not available (like English \em somebody, something\em), the WH=\em so\em-construction \formref{sec:nppp:Nounphrasesbasedoninterrogativepronouns} is used to yield an indefinite pronoun.\footnote{See \citet[164]{Haspelmath1997indefinite} for a typological analysis of the combination of an interrogative pronoun with `or' to yield an indefinite pronoun, which, interestingly, is again a construction shared between SLM and Japanese.}

\xbox{16}{
\ea\label{ex:func:ptcpt:unknown:indiv:saapa}
\gll \textbf{Saapa=so} {\em Malay} {\em exam} arà-girja. \\
      who=\textsc{addit} Malay exam \textsc{non.past}-make \\
    `Someone was taking a Malay exam.' (K060103nar01)
\z
}

\xbox{16}{
\ea\label{ex:func:ptcpt:unknown:indiv:aapacara}
\gll Thapi \textbf{aapacara=so} itthu samma asà-iilang su-aada. \\
     But how=\textsc{undet} \textsc{dist} all \textsc{cp}-disappeared \textsc{past}-exist  \\
    `But somehow it had all disappeared.'  (K20070920eml01)
\z
}

% K061019sng01.trs:dhaathang thurus police oorang
% K061019sng01.trs:inni aapa katha biilang aanak pe saala peegang saapa so biilang aada nang
%
%
%
% \xbox{16}{
% \ea\label{ex:func:unreferenced}
% \gll Uuthang=ka asà-pii=apa see picakang, ithu aapa=so, ithu daavong pada. \\
%       jungle=\textsc{loc} \textsc{cp}-go after \textsc{1s} break \textsc{dist} what=\textsc{undet} \textsc{dist} leaf \textsc{pl} \\
%     `I went to the woods and I was breaking these what's-their-name, these leaves.' (K061125nar01)
% \z
% } \\


\subsubsection{Unknown categorial referents}\label{sec:func:Unknowncategorialreferents}
In distinction to individuated referents, which are instantiated by a specific entity in discourse, categorial referents do not refer to a specific entity, but to any member of the indicated category.

In the following example, Andare wants to be dressed like a king. In this case, this does not refer to a specific king, but to any member of the class of kings. This is indicated by the indefinite article \em hathu \em \formref{sec:morph:Indefinitenessclitic}.


\xbox{16}{
\ea\label{ex:func:ptcpt:unknown:categor:hatthu}
\gll [Andare kanabisan=nang anà-mintha] [\textbf{hathu} raaja=ke asà-paake=apa kampong=nang mà-pii maau katha]. \\
    Andare last=\textsc{dat} \textsc{past}-ask \textsc{indef} king=\textsc{simil} \textsc{cp}-dress=after village=\textsc{dat} \textsc{inf}-go want \textsc{quot}   \\
    `What Andare wanted as a last wish, was to go to the village dressed up as a king.'  (K070000wrt03)
\z
}


An example where the categorial nature is emphasized would be  English   \em Do whatever you like\em, where \em whatever \em does not refer to an individuated referent, but to any member of the category of liked things that the hearer wants to instantiate the referent with. These unknown categorial referents are formed with the WH\~{}WH ... \em=so \em  construction in SLM \formref{sec:nppp:NPscontaininginterrogativepronounsusedforuniversalquantification}. An example is
\xref{ex:func:ptcpt:unknown:categorial}, where a predication is made about the category of people who have a Malay dress. The precise referents are unknown to the speakers. They are not individuated nor specific. Any member of the set of possessors of Malay dresses is invited to wear them at the wedding.

\xbox{16}{
\ea\label{ex:func:ptcpt:unknown:categorial}
\gll \textbf{Saapa}\Tilde\textbf{saapa}=ka inni  mlaayu pakeyan pada aada\textbf{=so}, lorang  pada ini       mlaayu  pakeyan=samma ini       kaving=nang mà-dhaathang    bannyak uthaama. \\
 who\Tilde who=\textsc{loc} \textsc{prox} Malay dress \textsc{pl} exist=disj \textsc{2pl} \textsc{pl} \textsc{prox} Malay dress=with \textsc{prox} wedding=\textsc{dat} \textsc{inf}-come much honour\\
`Whoever owns such  Malay costumes, your coming to the wedding in these costumes will be highly respected.' (K060116nar04)
\z
}


\subsubsection{Unknown referents, generic}\label{sec:func:Unknownreferents,generic}
The last possibility for unknown referents is to be generic, i.e. no particular referent is intended, but the predicate is thought to be true of any referent. An example in English would be \em one \em as in \em one has to be kind to strangers \em or generic \em you \em as in \em you must obey the law\em. In SLM, generic reference is not expressed overtly. An example is the \xref{ex:ptcpt:unknown:generic} about the general rules of Sepaktakraw. While in the SLM example, no referent is expressed, in English the use of \em you \em is mandatory, alternatively the use of \em one \em or a passive construction.


\xbox{16}{
\ea \label{ex:ptcpt:unknown:generic}
\gll \zero{} kaaki=dering masthi-maayeng. \\
     { } leg=\textsc{abl}  must-play\\
    `You have to play with your feet.' (N060113nar05)
\z
}

\subsubsection{Queried referents}\label{sec:func:Queriedreferents}
Referents are queried for by replacing them with the interrogative pronoun   \formref{sec:wc:Interrogativepronouns} which corresponds to the semantic category of the referent (\trs{aapa}{what}, \trs{mana}{where}, \trs{kaapang}{when}, etc). A  postpositions can be added if needed for a more precise indication of semantic role (\trs{aapa=nang}{for what}, \trs{mana=dering}{from where}, etc). The following sections provide more detailed information for specific semantic domains.

\paragraph{Person}\label{sec:func:q:Person}

\trs{Saapa}{who} is the interrogative pronoun used for persons. It can take the whole array of postpositions to indicate its syntactic and semantic role \formref{sec:wc:saapa}.

\xbox{16}{
\ea \label{ex:func:quest:content:person}
\gll \textbf{Saapa}  anà-maathi? \\
 who \textsc{past}-die\\
`Who died?' (K051213nar07)
\z
}

\paragraph{Animal}\label{sec:func:q:Animal}
There is no special interrogative pronoun to query for animals. One can use \trs{mana binaathang}{which animal}. This differs from the adstrate Sinhala, where a specialized interrogative pronouns for animals exists, \em mokaa \em (\citet[29]{Garusinghe1962},  \citet[259]{Karunatillake2004}).


\xbox{16}{
\ea
\gll Mana binaathang lorang=yang anà-giigith? \\
     which animal \textsc{2pl}=\textsc{acc} \textsc{past}-bite  \\
    `Which animal bit you?' (K051213nar06)
\z
}


\paragraph{Things}\label{sec:func:q:Things}
 \trs{Aapa}{what} \formref{sec:wc:aapa} and \trs{mana}{which} \formref{sec:wc:mana} are used to query for things in the widest sense.   Like \em saapa\em, they can take any postposition to indicate the syntactic and semantic role.

\xbox{16}{
\ea \label{ex:func:quest:content:things:aapa}
\gll \textbf{Aapa}   anà-jaadi       mlaayu pada? \\
 what \textsc{past}-become Malay \textsc{pl}\\
`What became of the Malays?' (K051213nar06)
\z
}

\xbox{16}{
\ea \label{ex:func:quest:content:things:mana}
\gll \textbf{Mana} nigiri=ka arà-duuduk? \\
 which country=\textsc{loc} \textsc{non.past}-stay\\
`Which country do you live in?' (B060115cvs16)
\z
}

\paragraph{Quantity}\label{sec:func:q:Quantity}
Quantity is asked for with \trs{dhraapa}{how much}?


\xbox{16}{
\ea \label{ex:func:quest:content:quant}
\gll \textbf{Dhraapa} thaaun \textbf{dhraapa} buulang lu arà-baapi suusa? \\
      how.many year how.many month \textsc{2s.familiar} \textsc{non.past}-bring sad \\
    `How many years, how man months are you bringing sadness (into my life).'  (K061123sng01)
\z
}



\xbox{16}{
\ea
\gll Bìrras hatthu kilo dhraapa? \\
     raw.rice one kilo how.much  \\
    `How much is one kilo of rice?' (K081106eli01)
\z
}


\paragraph{Location}\label{sec:func:q:Location}
Questions for locations are formed with the interrogative pronoun \trs{mana}{where}, which can be used for stative location \xref{ex:func:quest:content:loc:ka} or goal of motion \xref{ex:func:quest:content:loc:nang}. For source, \em mana=dering \em or \trs{manari}{whence} has to be used \xref{ex:func:quest:content:loc:dering}. An alternative is \em mana asduuduk \em \xref{ex:func:quest:content:loc:asduuduk}.


\xbox{16}{
\ea \label{ex:func:quest:content:loc:ka}
\gll Maana(=ka) se=ppe thoppi? \\
     where=\textsc{loc} \textsc{1s=poss} hat \\
    `Where is my hat?' (K081106eli01)
\z
}

\xbox{16}{
\ea \label{ex:func:quest:content:loc:nang}
\gll Maana(=nang) arà-pii? \\
      where=\textsc{dat} \textsc{non.past}-go \\
    `Where are you going (to)?' (K081106eli01)
\z
}

\xbox{16}{
\ea \label{ex:func:quest:content:loc:dering}
\gll Maana*(=dering) inni arà-dhaathang? \\
     where=\textsc{abl} \textsc{prox} \textsc{non.past}-come  \\
    `Where does this come from?' (K081106eli01)
\z
}

\xbox{16}{
\ea \label{ex:func:quest:content:loc:asduuduk}
\gll Maana asduuduk anà-dhaathang? \\
     where from \textsc{past}-come  \\
    `where do you come from?' (K081106eli01)
\z
}


If the query is about location in one of a given array of items, \em aapa=ka \em or   can also be used.

\xbox{16}{
\ea
\gll Itthu thoppi=yang aapa=ka aada? \\
      \textsc{dist} hat=\textsc{acc} what=\textsc{loc} exist \\
    `On what is that hat?' (K081106eli01)
\z
}


% \xbox{16}{
% \ea
% \gll Itthu thoppi=yang siini aada. \\
%     \textsc{dist} hat=\textsc{acc} here exist    \\
%     `That hat is here.' (K081106eli01) better with yang
% \z
% } \\
%


\paragraph{Time}\label{sec:func:q:Time}
General temporal reference can be queried with \trs{kaapang}{when}.


\xbox{16}{
\ea
\gll Kaapang loram pada siini arà-dhaathang? \\
     when \textsc{2pl} \textsc{pl} here \textsc{non.past}-come  \\
    `When are you coming here?' (K081106eli01)
\z
}


An amount of time can be queried with \trs{dhraapa laama}{how long} (Literally \em how much while\em).

\xbox{16}{
\ea
\gll Ini pukuran=nang dhraapa laama athi-ambel? \\
     \textsc{prox} work=\textsc{dat} how.much while \textsc{irr}-take  \\
    `How long will it take for this work?' (K081106eli01)
\z
}


The time can be queried with \trs{pul-dhraapa}{At what time} (Literally \em at how much o'clock \em).


\xbox{16}{
\ea
\gll Skaarang vakthu pukul dhraapa=ke boole=aada? \\
     now time o'clock  how.many=\textsc{undet} can=exist  \\
    `What time will it be now?' (K081106eli01)
\z
}


\xbox{16}{
\ea
\gll Se pukul dhraapa=nang masà-dhaathang? \\
      \textsc{1s} o'clock how.much-\textsc{dat} must-come \\
    `At what time should I come.' (K081106eli01)
\z
}


\paragraph{Manner}\label{sec:func:q:Manner}
\em Càraapa \em is used to query for manner. The inverted form \em ap(a)caara \em also exists.

\xbox{16}{
\ea
\gll See ini koolang=yang arà-langka apacaara/càraaapa? \\
      \textsc{1s} \textsc{prox} river=\textsc{acc} \textsc{non.past}-cross how \\
    `How can I cross this river?' (K081114eli01)
\z
}

% \xbox{16}{
% \ea \label{ex:func:quest:content:manner}
% \gll See thaau ambel \textbf{aapacara} katha. \\
%      \textsc{1s} know take how \textsc{quot}  \\
%     `I learnt how.' (K051206nar07)
% \z
% } \\




\paragraph{Reason, cause and purpose}\label{sec:func:q:Reasoncauseandpurpose}
These questions are formed by \trs{aapa}{what} followed by  the benefactive postposition \em =nang \em \xref{ex:func:quest:content:purp:aapa}. If the cause is a person, \trs{saapa}{who} can be used instead of \em aapa \em   \xref{ex:func:quest:content:purp:saapa}.

\xbox{16}{
\ea \label{ex:func:quest:content:purp:aapa}
\gll Lorang=nang inni aapa=nang? \\
     \textsc{2pl}=\textsc{dat} \textsc{prox} what=\textsc{dat}  \\
    `Why do you want it?' (K081106eli01)
\z
}



\xbox{16}{
\ea \label{ex:func:quest:content:purp:saapa}
\gll Ini saapa=nang arà-baapi? \\
      \textsc{prox} who=\textsc{dat} \textsc{non.past}-take.away \\
    `For whom are you taking this?' (K081106eli01)
\z
}


A specialized interrogative pronoun to query for reason only is \em kànaapa\em.


\xbox{16}{
\ea
\gll Kànaapa itthu anà-bìlli? \\
     why \textsc{dist} \textsc{past}-buy  \\
    `Why did you buy that?' (K081106eli01)
\z
}





\paragraph{Other}\label{sec:func:q:Other}
Other questions are formed by combining the interrogative pronouns with the relevant postpositions  \formref{sec:wc:Interrogativepronouns}.

\xbox{16}{
\ea
\gll Ini peena saapa=pe? \\
     \textsc{prox} pen who=\textsc{poss}  \\
    `Whose pen is this?' (K081106eli01)
\z
}

\xbox{16}{
\ea
\gll Farook saapa=ke? \\
      Farook who=\textsc{simil} \\
    `Who does Farook look like?' (K081106eli01)
\z
}




\subsection{Modifying participants}\label{sec:func:Modifyingparticipants}
In the preceding sections we have seen how participants are encoded in SLM. In this section, we will see how participants can be modified. This can be done in various number of ways:
modifications pertaining to quality, such as size or colour,
quantitative modifications,
indicating a possessor,
a location, or time.
Temporal modification of a participant is also possible, an English example would be \em ex-wife\em.

We will discuss these different types of modifications in turn in  before we close with a discussion of  restrictive and characterizing modifications..



% In general, modifiers of participants are found on the left of their head word, which contrasts sharply with general Austronesian typology as described by \citet[141]{Himmelmann2005typochar}: ``Adpositions are generally prepositions in western Austronesian languages \el Auxiliaries generally precede main verbs. Negators also generally precede the negated constituent \el possessors generally follow the possessum \el numbers generally follow the head noun \el otherwise, adnominal modifiers generally follow the head, with demonstratives being placed at the very end of an NP.

\subsubsection{Quality}\label{sec:func:Quality}

Indication of quality can be done by either a noun or an adjective, which can be either pre- or postposed, or a relative clause, which can only be preposed \formref{sec:nppp:Thefinalstructureofthenounphrase}. Examples \xref{ex:ptcpt:mod:qual:nom:pre} and \xref{ex:ptcpt:mod:qual:nom:post} show modification by pre- and postposed nouns, while examples \xref{ex:ptcpt:mod:qual:adj:pre} and \xref{ex:ptcpt:mod:qual:adj:post} show the same for pre- and postposed adjectives.


\xbox{16}{
\ea \label{ex:ptcpt:mod:qual:nom:pre}
\gll The$\curvearrowright$ pohong. \\ % bf
 tea tree\\
`tea tree.'
\z
}

\xbox{16}{
\ea \label{ex:ptcpt:mod:qual:nom:post}
\gll Orang  $\curvearrowleft$ikkang. \\ % bf
 man fish\\
`fisherman'
\z
}

\xbox{16}{
\ea \label{ex:ptcpt:mod:qual:adj:pre}
\gll \textbf{Baaru}$^\curvearrowright$ \textbf{oorang} pada masà-thaaro. \\
 new man \textsc{pl} must-put\\
`(We) must put new people.' (K060116nar11)
\z
}

\xbox{16}{
\ea \label{ex:ptcpt:mod:qual:adj:post}
\gll Se=ppe       \textbf{oorang} $^\curvearrowleft$\textbf{thuuva} pada    anà-biilang [kitham pada {\em Malaysia}=dering    anà-dhaathang    katha]. \\
 \textsc{1s=poss} man old \textsc{pl} \textsc{past}-say \textsc{1pl} \textsc{pl} Malaysia=\textsc{abl} \textsc{past}-come  \textsc{quot}\\
`My elders told me that we had come from Malaysia.' (K060108nar02)
\z
}

The adjectival predications may be internally complex, as the negated word for `good' in \xref{ex:ptcpt:mod:qual:adj:compl}.


\xbox{16}{
\ea \label{ex:ptcpt:mod:qual:adj:compl}
\gll Go  \textbf{thàrà-baaye}   pukujan  thama-gijja. \\
     \textsc{1s} \textsc{neg}-good work \textsc{neg.irr}-do  \\
    `I do not do bad work.' (B060115nar04)
\z
}


Dimension is expressed by the adjectives \trs{bìssar}{big} and \trs{kiccil}{small}. There are no diminutives or augmentatives.

\xbox{16}{
\ea \label{ex:ptcpt:mod:qual:dimension}
\ea
\gll \textbf{Bìssar} atthu  kumpulan    thraa. \\
good one association \textsc{neg}\\
`There is no big association.'
\ex
\gll \textbf{Kiccil} kumpulan    pada=jo. \\
 small association \textsc{pl}=\textsc{emph}\\
`The associations are small indeed.' (N060113nar01)
\z
\z
}

% \xbox{16}{
% \ea\label{ex:func:unreferenced}
% \gll Deram  pada arà-duuduk    konnyom konnyom \textbf{kiccil} \textbf{kiccil} kavanang=ka. \\
%  \textsc{3pl} \textsc{pl} \textsc{non.past}-stay little little small small group=loc\\
% `They live in few small groups.' (N060113nar01)
% \z
% }


Colour is expressed by a colour adjective, which is often found combined with the noun \trs{caaya}{colour}.\footnote{This is an influence from the adstrates, where colour terms need to be supported by the noun for `colour', cf. the words for `blue' in Sinhala (\em nil paa\tz a\em) and Tamil (\em niil ni\textsubbar{r}am\em), where \em ni(i)l \em means `blue' and the other word means `colour'.} Table \ref{tab:func:colourterms} gives the terms for common colours.


\xbox{16}{
\ea \label{ex:ptcpt:mod:qual:colour}
\gll Hatthu komplok bannyak=jo \textbf{puuthi} \textbf{caaya}, hathyeng=yang \textbf{meera}=jo \textbf{meera} \textbf{caaya}. \\
     \textsc{indef} bush much=\textsc{emph} white colour, other=\textsc{acc} red=\textsc{emph} red colour  \\
    `One bush was very white, the other one was of the reddest red.'  (K070000wrt04)
\z
}

Some colour terms are not adjectives, but nouns, as is the case for the word for `brown' which is derived from the word \trs{beecek}{mud}.

\xbox{16}{
\ea \label{ex:ptcpt:mod:qual:colour:noun}
\gll Sithu=ka, hathu bìssar \textbf{beecek} \textbf{caaya} Buruan su-duuduk.\\
      there=\textsc{loc} \textsc{indef} big mud colour bear \textsc{past}-stay \\
    `There was a big brown bear.' (K070000wrt04)
\z
}


\begin{table}
\begin{center}
% use packages: array
\begin{tabular}{llllll}
SLM & gloss  & original meaning & SLM  & gloss  & original meaning \\
\hline
iitham & black & black & iijong & green & green \\
puuthi & white & white & kuunyith & yellow & turmeric\footnotemark \\
meera & red & red & beecek & brown & mud \\
niila & blue & blue &  &  &  \\
\end{tabular}
\end{center}
\caption[Colour terms]{Colour terms in SLM, with their original meaning if different.}
\label{tab:func:colourterms}
\end{table}

\footnotetext{The original word \em kuuning \em can also be heard, but because of the phonetic resemblance of \em kuuning \em and \trs{kuunyith}{turmeric}, the latter makes inroads into the domain of colours, also because Sinhala (\em kaha\em) and Tamil (\em ma\ny ca\lz\em)  use the word for `turmeric' to refer to `yellow' as well \citep[cf.][]{Paauw2004}.}


%\xbox{16}{
%\ea\label{ex:func:unreferenced}
%\gll Itthu=nam       mlaayu pada=pe     {\em Tradition}=ke      aada. \\
% therefore Malay \textsc{pl}=\textsc{poss} Tradition=simil\\
%`That's why there are the Malays' traditions.' (N060113nar01)
%\z
%}


\subsubsection{Quantity}\label{sec:func:mod:Quantity}
Next to qualitative modifications, referents can also be modified for quantity. This is discussed in detail in   \funcref{sec:func:Quantity}.

\subsubsection{Possession}\label{sec:func:mod:Possession}
Terms based on nouns can get assigned a possessor, which is marked with \em =pe \em \formref{sec:morph:=pe}. The possessor is always preposed when it is used as a modifier. Possession can also be expressed predicatively. This is discussed in section \funcref{sec:func:Possession}.


\xbox{16}{
\ea \label{ex:ptcpt:mod:poss}
\gll Kithang sama oorang \textbf{lorang=pe} \textbf{suurath}=yang daapath vakthu=ka kithang sama oorang bannyak su-suuka. \\
     \textsc{1pl} all man you=\textsc{poss} letter=\textsc{acc} get time=\textsc{loc} \textsc{1pl} all man much \textsc{past}-like \\
    `All of us were very happy when we received your letter.'  (Letter 26.06.2007)
\z
}

In example \xref{ex:ptcpt:mod:poss}, it is not because of any letter that the speakers are happy, but the set of letters they are happy about is restricted to the one written by the addressee.

\subsubsection{Location}\label{sec:func:mod:Location}
A term can be modified to entities at a certain location. This is done with a relative clause containing a locational predicate with the existential \em aada\em. The existential is obligatory.


\xbox{16}{
\ea \label{ex:ptcpt:mod:loc}
\gll Meeja=ka *(aada) maa\u n\u gga=yang kaasi. \\
     table=\textsc{loc} exist mango=\textsc{acc} give  \\
    `Give me the mango (which is) on the table' (K081118eli01)
\z
}

% \xbox{16}{
% \ea\label{ex:func:unreferenced}
% \ea
% \gll Se [meeja=ka]_{MOD} maa\u n\u gga arà-maakang. \\
%       \textsc{1s} table=\textsc{loc} mango \textsc{non.past}-eat\\
%     `I am eating the mango on the table.' (tes)
% \ex se [meejaka]_{ADJCT} maa\u n\u gga arà-maakang\\
% `I am eating the mango, on the table.' (tes)
% \ex se [meejaka]_{RELP} maa\u n\u gga arà-maakang\\
% `I am eating the mango which is on the table.' (test)
% \z
% \z
% } \\
%
% \xbox{16}{
% \ea \label{ex:ptcpt:mod:loc}
% \gll Sithu=ka     aada  bìssar oorang pada=yang   asà-{\em attack}-kang     {\em mail}=nya    asà-cuuri \textbf{{\em mail}=ka}    duvith arà-baapi. \\
%      there=\textsc{loc} exist big man \textsc{pl}=\textsc{acc} \textsc{cp}-attack-\textsc{caus} mail=\textsc{acc} \textsc{cp}-steal mail=\textsc{loc} money \textsc{non.past}-bring  \\
%     `He attacks the leaders who are there and steals the mail and they bring money in the mail' (K051206nar02) (K051206nar02)
% \z
% } \\


\subsubsection{Time}\label{sec:func:mod:Time}
Terms may be marked as referring to a past referent by means of the adjective \trs{laama}{former}{}.



\xbox{16}{
\ea \label{ex:ptcpt:mod:time}
\gll Se=ppe laama ruuma. \\
      \textsc{1s=poss} old house \\
    `My old house.' (K081106eli01)
\z
}


Furthermore, the relevant period may be used as a possessive modifier with \em =pe\em. Both adverbs like \trs{karang}{now}{} and temporal nouns like \trs{muusing}{time}{} are possible here.


\xbox{16}{
\ea \label{ex:ptcpt:mod:time:pe:karang}
\gll \textbf{Karam=pe} mosthor=nang, mpapulu aari=ka=jo sunnath=le arà-kijja. \\
now=\textsc{poss} manner=\textsc{dat} forty   day=\textsc{loc}=\textsc{emph} circumcision=\textsc{addit} \textsc{non.past}-make  \\
    `For today's way of doing (it), it is on the fortieth day that they also do the circumcision.'  (K061122nar01)
\z
}

\xbox{16}{
\ea \label{ex:ptcpt:mod:time:pe:dovulu}
\gll \textbf{Dovulu=pe}     oorang pada. \\
      before=\textsc{poss} man \textsc{pl} \\
    `People in former times.' (K061026rcp04)
\z
}


\xbox{16}{
\ea \label{ex:ptcpt:mod:time:pe:muusing}
\gll Giithu=jo      derang pada, \textbf{itthu}  \textbf{muusing=pe}     mlaayu pada. \\
     like.that=\textsc{emph} \textsc{3pl} \textsc{pl} \textsc{dist} time=\textsc{poss} Malay \textsc{pl}\\
    `They were like that, the Malays of former times.' (K051206nar04)
\z
}



% \xbox{16}{
% \ea \label{ex:ptcpt:mod:time:pe:muusing}
% \gll Itthu    muusing=pe     {\em British}  government. \\
%     \textsc{dist} time=\textsc{poss} British government   \\
%     `The British government of that time.' (N061031nar01)
% \z
% }


%
% \xbox{16}{
% \ea\label{ex:func:unreferenced}
% \gll Siini baapa ka jona aada samma record pada laama muusing ka anà aada. \\
%      here father=\textsc{loc}=\textsc{phat} exist all record \textsc{pl} old time=\textsc{loc} \textsc{past}-exist  \\
%     `.' (K061030mix01)
% \z
% } \\

% \xbox{16}{
% \ea\label{ex:func:unreferenced}
% \gll Incayang=pe      muusing=ka kithang=pe     {\em Malays} pada atthu oorang=nang=le        [{\em parliament}=nang  mà-dhaathang=nang]      thumpath thàrà-daapath. \\
%      \textsc{3s.polite}=\textsc{poss} time=\textsc{loc} \textsc{1pl}=\textsc{poss} Malays \textsc{pl} \textsc{indef} man=\textsc{dat}=\textsc{addit} parliament=\textsc{dat} \textsc{inf}-come=\textsc{dat} place \textsc{neg.past}.get  \\
%     `During his time, no man of our Malays got a place to go to parliament (i.e. a seat).' (N061031nar01)
% \z
% } \\

% \subsubsection{Similarity}\label{sec:func:Similarity}
% Similarity can be expressed with the clitic \em =ke(e)\em.
%
% \xbox{16}{
% \ea\label{ex:func:similarity}
% \gll Se=dang baapa=\textbf{ke} {\em soldier} mà-jaadi suuka. \\
%      \textsc{1s=dat} father=\textsc{simil} soldier \textsc{inf}-become like  \\
%     `I want to become a soldier like daddy.' (B060115prs10)
% \z
% }







\subsubsection{Restriction and characterization}\label{sec:func:Restrictionandcharacterization}
Modifications can have two different readings, restrictive and characterizing. Restrictive modifications reduce the number of entities the term refers to. \em Rich people are unhappy \em refers to less entities than \em People are unhappy\em. The referents are restricted to those people who also have the characteristic of being rich.

Characterization, on the other hand, does not reduce the number of entities the term refers to, but gives additional information. An example would be  \em loving \em in \em my loving father\em, which characterizes my father, it does not restrict reference.
In SLM, all modifications can be used to restrict or characterize terms based on NPs. Terms based on pronouns can only be characterized. The use of restrictive modification has been exemplified amply in the preceding sections, so that only the less common characterization will be discussed in this section.

In \xref{ex:func:restmod:n:dwarf}, the dwarf has been talked about before in the story, and this time he reappears in the claws of a bird. There is only one dwarf in the story, and it is absolutely clear that it is this dwarf that is being talked about. In \xref{ex:func:restmod:n:dwarf}, the dwarf is nevertheless modified by  a preposed relative clause, which gives more information about his location. This information does not restrict the set of all possible dwarfs to those being held by birds, it rather gives additional information about the dwarf already identified before: it characterizes him.


\xbox{16}{
\ea\label{ex:func:restmod:n:dwarf}
\gll [Kaaki=ka gaa\u ndas-kang ambel anà-duuduk]$_{characterizing}$ Aajuth=yang sangke=luppas hathu pollu=dering Rose-red buurung=nang su-puukul. \\
     leg=\textsc{loc} tie-\textsc{caus} take \textsc{past}-stay dwarf=\textsc{acc} until=leave \textsc{indef} stick=\textsc{abl} Rose-red bird=\textsc{dat} \textsc{past}-hit  \\
    `Rose red hit the bird with a stick until it let go of the dwarf, whom he had taken in his claws.' (K070000wrt04)
\z
}

Another example is the characterization of the Almighty God as the creator of the addressee in \xref{ex:func:restmod:n:allah}. This is clearly not restrictive but rather characterizing.



\xbox{16}{
\ea\label{ex:func:restmod:n:allah}
\gll [Luu nya-laher-kang]$_{characterizing}$ Allah-thaala=nang liima vakthu mà-sbaayang=nang. \\
     \textsc{2s.familiar}=\textsc{acc} be.born-\textsc{caus} Allah-almighty=\textsc{dat} five time \textsc{inf}-pray=\textsc{dat}  \\
    `To pray five times to the Almighty God, who created you.' (K060116sng01)
\z
}


%
% \xbox{16}{
% \ea\label{ex:func:unreferenced}
% \ea
% \gll Luu=nya jadi-kang rabbu saapa lu=ppe nabi pada saapa katha biilang. \\
%       \textsc{2s.familiar}=\textsc{acc} become-\textsc{caus} ??? who \textsc{2s}=\textsc{poss} prophet \textsc{pl} who \textsc{quot} say \\
%     `Say who the RABBU is who made you, who are your prophets.'
% \ex
% \gll Lu=ppe rabbu saapa katha buthul balas-an asà-biilang. \\
%       \textsc{2s}=\textsc{poss} ??? who \textsc{quot} correct answer-\textsc{nmlzr} \textsc{cp}-say \\
%     `Having given the correct answer as to who your RABBU is,'
% \ex
% \gll Lu=ppe nabi saapa katha buthul balas-an asà-biilang. \\
%       \textsc{2s}=\textsc{poss} prophet who \textsc{quot} correct answer-\textsc{nmlzr} \textsc{cp}-say \\
%     `Having given the correct as to who your prophets are,'
% \ex
% \gll Thumpath=yang baaye sorga=nang bunnang-la. \\
%       place=\textsc{acc} good heaven=\textsc{dat} ???-\textsc{imp} \\
%     `???.' (K060116sng02)
% \z
% \z
% } \\

\section{Predication}\label{sec:func:Predication}
Different things  can be said about participants. One can say about them that they are in a certain state, that they take part  in a certain event, that they are a member of a certain semantic class or that they possess things.

These different types of semantic predicates are encoded by different constructions in SLM. Following \citet[106]{Hengeveld1992nvpttd}, I distinguish the following semantic types of predicates:\footnote{The list is reordered and somewhat simplified here for expository reasons. Namely the predicate types of ``classification'', ``instantiation'', ``factuality'', and ``interpretation'' are left out.}

\begin{itemize}
\item states
\begin{itemize}
 \item property assignment \funcref{sec:func:Propertyassignment}
 \item status assignment \funcref{sec:func:Statusassignment}
 \item identity \funcref{sec:func:Identity}
 \item localization \funcref{sec:func:Location}
 \item possession \funcref{sec:func:pred:Possession}
\end{itemize}
 \item events \funcref{sec:func:Events}
\end{itemize}


A related, yet different functional domain is causation \funcref{sec:func:Causation}.

\subsection{States}\label{sec:func:States}
\subsubsection{Property assignment}\label{sec:func:Propertyassignment}
Stative properties can be assigned in SLM by adjectival predicates \formref{sec:pred:Adjectivalpredicate} as in \xref{ex:soa:states:adj} or by verbal predicates \formref{sec:pred:Verbalpredicates} as in \xref{ex:soa:states:v}. Some states like colours are expressed by nominal predicates  \formref{sec:pred:Nominalpredicate}  \xref{ex:soa:states:n}.

\xbox{16}{
\ea \label{ex:soa:states:adj}
\gll Samma oorang \textbf{baae}$_{ADJ}$. \\
all man good \\
`All men are good.' (B060115cvs13)
\z
}

\xbox{16}{
\ea \label{ex:soa:states:v}
\gll Itthu pada [sraathus binthan pada arà-kiilap]=ke su-\textbf{kiilap}$_{V}$. \\
     \textsc{dist} \textsc{pl} 1000 star \textsc{pl} \textsc{simult}-shin=\textsc{simil} \textsc{past}-shine  \\
    `They shone like a thousand stars shine.'  (K070000wrt04)
\z
}


\xbox{16}{
\ea \label{ex:soa:states:n}
\gll Hatthu komplok bannyak=jo puuthi caaya, hathyeng=yang meera=jo [\textbf{meera} \textbf{caaya}]$_{N}$. \\
     \textsc{indef} bush much=\textsc{emph} white colour, other=\textsc{acc}  red=\textsc{emph} red colour  \\
    `One bush was very white, the other one was of the reddest red.'  (K070000wrt04)
\z
}

The defective verbs  \formref{sec:wc:Defectiveverbs} \trs{suuka}{like} and \trs{thaau}{know} also encode states.

\subsubsection{Status assignment}\label{sec:func:Statusassignment}
Status assignment means membership of an entity in a class of ``established functional, professional or ideological group''  \citep[76]{Dik1980}. This is expressed by nominal predicates  \formref{sec:pred:Nominalpredicate} in SLM.


\xbox{16}{
\ea \label{ex:func:pred:class:hatthu}
\gll Sindbad  {\em the}  {\em Sailor}     \textbf{hatthu} \textbf{Muslim}, mlaayu bukang. \\
 Sindbad the Sailor \textsc{indef} Muslim, Malay \textsc{neg.nonv}\\
`Sindbad the sailor was a Moor, he was not a Malay.' (K060103nar01)
\z
}


Optionally, the copula \em (asà)dhaathang(apa) \em \formref{sec:wc:Copula} can be used.



\xbox{16}{
\ea \label{ex:func:pred:class:cop}
\gll Se \textbf{asdhaathang} hatthu butthul {\em moderate} Muslim atthu. \\
 \textsc{1s} \textsc{copula} one very moderate Muslim one\\
`As for me, I am a very moderate Muslim.' (K051206nar18)
\z
}



\subsubsection{Identity}\label{sec:func:Identity}
The predication of identity of two referents is done by the equational construction \formref{sec:pred:Equationalpredicate}, which juxtaposes two NPs. This construction often uses the copula \em asàdhaathang \em \xref{ex:func:pred:ident:cop} or the emphatic clitic \em =jo \em \xref{ex:func:pred:ident:jo}.



\xbox{16}{
\ea \label{ex:func:pred:ident:cop}
\gll Baapa=pe      umma   \textbf{asàdhaathang} kaake=pe           aade. \\
    father=\textsc{poss} mother \textsc{copula} grandfather=\textsc{poss} younger.sibling  \\
    `My paternal grandmother was my grandfather's younger sister.' (K051205nar05)
\z
}

\xbox{16}{
\ea \label{ex:func:pred:ident:jo}
\gll Suda [itthu    kaake=pe aade=pe                aanak]=\textbf{jo}    baapa. \\ % bf
      thus \textsc{dist} grandfather=\textsc{poss} younger.sibling=\textsc{poss} child=\textsc{emph} father \\
    `So that grandfather's younger sister's child is my father.' (K051205nar05)
\z
}
%
% \paragraph{Characterization and specification}
% Stative predicates can be divided in to characterizing predicates and specificational predicates \citep{}. Characterizing predicates can combine with \em among other things\em, as in \em Paris is, among other things, the capital of France\em. Specificational predicates cannot combine with \em among other things\em, as in \em *The capital of France is, among other things, Paris\em.
%
% Characterization receives no special marking in SLM, but specification is frequently found with the emphatic marker \em =jo\em.
%
% \xbox{16}{
% \ea \label{ex:func:pred:spec}
% \gll [Itthu    kaake=pe           hatthu aanak]$_{pred}$=jo    [se=ppe    umma]$_{arg}$. \\ % bf
%       \textsc{dist} grandfather=\textsc{poss} once child=\textsc{emph} \textsc{1s=poss} mother \\
%     `One of that grandfather's children is my mother.' (K051205nar05)
% \z
% }
%
% A more complicated example is \xref{ex:disc:spec:num}, but the argumentation is similar, with the exception that we are not dealing with the indefinite article, but with the numeral \trs{satthu}{one} instead.
%
% \xbox{16}{
%  \ea\label{ex:disc:spec:num}
%    \gll [[Itthu    mà-jaaga=nang        anà-baa      mlaayu]=dering  satthu      oorang]$_{pred}$=jo    [see]$_{arg}$. \\ % bf
%  \textsc{dist} \textsc{inf}-protect=\textsc{dat} \textsc{past}-bring Malay=\textsc{abl}   one man=\textsc{emph} 1s\\
% `One of the Malays brought to protect him is me' (K060108nar02)
% \z
% }
%
%
% % \xbox{16}{
% % \ea\label{ex:form:unreferenced}
% % \gll suda [itthu    kaake=pe aade=pe                aanak]$_{pred}$\textbf{=jo}    baapa$_{arg}$. \\
% %       thus \textsc{dist} grandfather=\textsc{poss} younger.sibling=\textsc{poss} child=\textsc{emph} father \\
% %     `So that grandfather's younger sister's child is my father.' (K051205nar05)
% % \z
% % }
%
% The final example of this section shows a plural predication, where the indefinite article cannot occur, but the specificational structure is clear nevertheless, since pronouns can never be used as predicates. This means that \trs{kithang}{we} is the referent in \xref{ex:disc:spec:pl}, which instantiates the predicate grandchild(X).
%
% \xbox{16}{
%  \ea\label{ex:disc:spec:pl}
% \gll [Aanak cuucu]$_{pred}$=jo [kithang]$_{arg}$. \\ % bf
%       child great.grand.child=\textsc{emph} \textsc{1pl} \\
%     `The great.grandchildren are we.'  (K051205nar04)
% \z
% }\\

\subsubsection{Localization}\label{sec:func:Location}
Localization is expressed by a locational predicate  \formref{sec:pred:Locationalpredicate} with \em =ka\em.

\xbox{16}{
\ea \label{ex:func:loc}
   \gll Kithang=pe     oorang thuuva pada samma Seelong\textbf{=ka}. \\
   \textsc{1pl}=\textsc{poss} man old \textsc{pl} all Ceylon=\textsc{loc} \\
`Our forefathers were all in Ceylon' (K060108nar02)
\z
}

Additionally, localization can be expressed by an existential construction with specification of the semantic role of \textsc{location} on one participant, in the following example \em Dubai\em. The existential predicate differs from the pure locational predicate by the existence of an existential verb, in this case \em duuduk\em.

\xbox{16}{
\ea \label{ex:func:loc:exist}
\gll Se=ppe      dhaatha=pe           thiiga aanak=le      Dubai=\textbf{ka}     arà-\textbf{duuduk}. \\
 \textsc{1s=poss} elder.sister=\textsc{poss} three child=\textsc{addit} Dubai=\textsc{loc} \textsc{non.past}=stay\\
`My sister's three children also live in Dubai.' (B060115prs21)
\z
}


\subsubsection{Possession}\label{sec:func:pred:Possession}
Possessive predicates are expressed by a possessive predicate \formref{sec:pred:Possessivepredicate} with either the dative, used for permanent possession, or the locative, used for temporary possession. For more details on possession, see \funcref{sec:func:Possession}.


\xbox{16}{
\ea \label{ex:func:poss:nang}
\gll Se=\textbf{dang} liima anak  klaaki pada \textbf{aada}. \\
      \textsc{1s=dat} five child male \textsc{pl} exist \\
    `I have five sons.' (K060108nar02)
\z
}


\xbox{16}{
\ea \label{ex:func:poss:ka}
\gll Incayang\textbf{=ka} ... bìssar beecek caaya hathu {\em bag} su-\textbf{aada}. \\
     \textsc{3s.polite}=\textsc{loc} ... big mud colour \textsc{indef} bag \textsc{past}-exist  \\
    `He had a big brown bag with him.' (K070000wrt04)
\z
}

\subsection{Events}\label{sec:func:Events}
Events are characterized by being [+dynamic]. They  are always encoded as verbs \formref{sec:pred:Verbalpredicates}. An example is given in \xref{ex:soa:events:v}.

\xbox{16}{
\ea \label{ex:soa:events:v}
\gll Baapa=le       aanak=le      guula su-maakang. \\
      father=\textsc{addit} child=\textsc{addit} sugar \textsc{past}-eat \\
    `Father and son ate sugar.' (K070000wrt02)
\z
}

It is possible to use lexemes from the adjective class to denote events and not states. In this case, they undergo conversion to verbs, take verbal morphology and refer to  the process of the state denoted by the adjective coming into being. The adjective \trs{bìssar}{big} in \xref{ex:soa:events:adj} is used with verbal morphology and then denotes not the state of being big, but the event of becoming big, i.e. growing up.

\xbox{16}{
\ea \label{ex:soa:events:adj}
\gll Itthu=nam blaakang=jo, kitham pada \textbf{anà-bìssar}. \\
 \textsc{dist} after=\textsc{emph} \textsc{1pl} \textsc{pl} \textsc{past}-big\\
`After that, we grew up.' (K060108nar02)
\z
}

The use of an adjective in a verbal frame changes its aktionsart from static to dynamic. It is important to distinguish aktionsart from grammatical aspect in this regard. Aktionsart  does not interfere with the marking of aspect. The following examples show that dynamic aktionsart can combine with the past tense (with in this case perfective semantics) \xref{ex:soa:events:adj:su}, the conjunctive participle \xref{ex:soa:events:adj:s} also giving a perfective reading, but also with \em arà- \em in \xref{ex:soa:events:adj:ara}, giving an imperfective reading.



\xbox{16}{
\ea \label{ex:soa:events:adj:su}
\gll hatthu spuulu liimablas     thaaun=nang jaalang blaakang,     inni kumpulan    \textbf{sa}-mampus\\
\textbf{indef} ten fifteen year=\textsc{dat} go after \textsc{prox} association \textsc{past}-dead \\
`About ten, fifteen years after that, the association became defunct.
\z
 }

\xbox{16}{
\ea \label{ex:soa:events:adj:s}
\gll Aanak pada \textbf{asà}-bìssar, skuul=nang anà-pii. \\
 child \textsc{pl} \textsc{cp}-big school=\textsc{dat} \textsc{past}-go\\
`The children grew up and went to school.' (K051222nar04)
\z
}


\xbox{16}{
\ea \label{ex:soa:events:adj:ara}
\gll Ruuma \textbf{arà}-kiccil. \\
      house \textsc{non.past}-small\\
    `The houses are getting small.'  (K051222nar04)
\z
}

The above examples show that the use of adjectives in a verbal frame is  not a case of grammatical aspect (inchoative or ingressive), but a change in lexical aspect from [-dynamic] to [+dynamic]. This can further be demonstrated by the fact that adjectives in a verbal frame can be marked for completive aspect. If the verbal frame did indeed indicate inchoative/ingressive aspect, this should be incompatible with the marking of completive. Example \xref{ex:soa:events:adj:abis} shows the use of the adjective \trs{thuu\u nduk}{bent} in a verbal frame, yielding `becoming bent'. This is marked with the vector verb \em abbis\em, which indicates completive aspect. The final meaning is then `having finished becoming bent', which is indeed what one must do to enter a small room like a cave. This example demonstrates that the verbal reading of adjectives has to do with lexical aspect, but not with grammatical aspect, given that grammatical aspect is already expressed elsewhere (\em abbis\em), without leading to ungrammaticality.

\xbox{16}{
\ea\label{ex:soa:events:adj:abis}
\gll Giithu=jo      thuu\u nduk \textbf{abbis}=jo       masà-pii. \\
      like.that=\textsc{emph} bent finish=\textsc{emph} must-go \\
    `You must enter there completely crouched.' (K051206nar02)
\z
}

%
%\xbox{16}{
%\ea\label{ex:func:unreferenced}
%\gll {\em Patients} pada  nyaakith  oorang pada s-pii      thaangan arà-cuuci    nni      saarong samma bassa. \\
% \\
%`.' (nosource)
%\z
%}

The change of aktionsart from static to dynamic through the use in a verbal frame is also possible with loan words, as shown in \xref{ex:soa:events:adj:loan}

\xbox{16}{
\ea \label{ex:soa:events:adj:loan}
\gll Se=dang \textbf{arà-{\em late}} bukang, see arà-dhaathang. \\
      \textsc{1s=dat} \textsc{non.past}-late \textsc{tag} \textsc{1s} \textsc{non.past}-come \\
    `I am getting late, goodbye (goodbye=I will [go and] come).' (B060115cvs08)
\z
}

Example \xref{ex:soa:events:adj:doubleconversion} is more involved. In this example, the adjective \trs{lummas}{soft} is first converted to a verb `become soft', which in turn is the only constituent of the NP to which the postpositions \em =nang \em attaches. Example \xref{ex:soa:events:adj:doubleconversion} thus shows two instances of conversion, one from adjective to verb on the morphological level, and one from clause to NP on the syntactic level.


\xbox{16}{
\ea \label{ex:soa:events:adj:doubleconversion}
\gll [[[[\zero{} Lummas$_{ADJ}$] $-\zero$]$_{V}$]$_{CLS}$ =\zero]$_{NP}$ =nang blaakang minnyak klaapa=ka inni=yang gooreng. \\ % bf
      { } soft -\textsc{vrblzr} =\textsc{nmlzr} =\textsc{dat} after coconut.oil coconut=\textsc{loc} \textsc{prox}=\textsc{acc} fry \\
    `After it has become tender, fry this in coconut oil.' (K060103rec02)
\z
}


While adjectives can undergo this change in lexical aspect by zero-derivation, periphrases have to be used for states denoted by other word classes. States denoted by nouns have to take \trs{jaadi}{become}{} to get a dynamic reading, which allows for the application of perfective aspect. This is the case  in \xref{ex:soa:events:n:dyn:jaadi:su}, where the past tense marker \em su- \em conveys a perfective reading. The speaker in \xref{ex:soa:events:n:dyn:jaadi:su} informs us that in former times, the country was not theirs, but a change of state took place, and now the country is theirs. This change of state implies dynamic aktionsart, which is not a possibility for nouns. Therefore, the periphrasis with the verb \trs{jaadi}{become}{} is used, which allows for the expression of dynamic aktionsart.

\xbox{16}{
\ea \label{ex:soa:events:n:dyn:jaadi:su}
\gll Ini kitham=pe \textbf{nigiri} su-\textbf{jaadi}. \\
prox 1pl:poss country \textsc{past}-become \\
`This became our country.' (K051222nar04)
\z
}

Things are similar with the following example, where at a time $t_0$, the person was not a  cancer patient, but at a time $t_1$ had become a cancer patient. This again has to be expressed by \em jaadi\em.

\xbox{16}{
\ea \label{ex:soa:events:n:dyn:jaadi:se}
\gll Incayang  \textbf{{\em cancer}}  \textbf{{\em patient}} se-\textbf{jaadi}. \\
      \textsc{3s.polite} cancer patient \textsc{past}-become \\
    `He became a cancer patient.' (K060116nar15)
\z
}


For the word \trs{enco}{fooled}{}, the same holds true.

\xbox{16}{
\ea \label{ex:soa:events:n:dyn:jaadi:thara}
\gll Thapi=le Andare thàrà-\textbf{jaadi} \textbf{enco}. \\
     But=\textsc{addit} Andare \textsc{neg.past}-become fooled  \\
    `But Andare did not get fooled.' (K070000wrt02)
\z
}


% \xbox{16}{
% \ea\label{ex:func:unreferenced}
% \gll Itthu=nang      blaakang, [kithang=pe hatthu oorang=le      {\em minister} jaadi  thraa] kithang=nang   nya-aada. \\
%       \textsc{dist}=\textsc{dat} after \textsc{1pl}=\textsc{poss} \textsc{indef} man=\textsc{addit} minister become \textsc{neg} \textsc{1pl}=\textsc{dat} \textsc{past}-exist \\
%     `After that, we had to face the fact that not one of our men became a minister (again).' (N061031nar01)
% \z
% } \\

This process is also possible for mass nouns, as shown in \xref{ex:soa:events:n:dyn:jaadi:mass} for the mass nouns \trs{daara}{blood} and \trs{suusu}{milk}.

\xbox{16}{
\ea \label{ex:soa:events:n:dyn:jaadi:mass}
\gll Oorang pada kapang-laari   dhaathang, ini daara sgiithu=le  \textbf{suusu} \textbf{su-jaadi}. \\
    man \textsc{pl} when-run come \textsc{prox} blood that.much=\textsc{addit} milk \textsc{past}-become   \\
    `When people came running, the blood had turned into milk.' (K051220nar01)
\z
}


The following sentences give some more examples of this use.


\xbox{16}{
\ea \label{ex:soa:events:mod:dyn:jaadi:extra1}
\gll Aashik=nang \textbf{hathu} \textbf{{\em soldier}} \textbf{mà-jaadi} suuka=si katha arà-caanya. \\
     Aashik=\textsc{dat} \textsc{indef} soldier \textsc{inf}-become like=\textsc{interr} \textsc{quot} \textsc{non.past}-ask  \\
    `He asks if you want to become a soldier, Ashik.' (B060115prs10)
\z
}


\xbox{16}{
\ea \label{ex:soa:events:mod:dyn:jaadi:extra2}
\gll Baaye \textbf{meera} \textbf{caaya} kapang-\textbf{jaadi}, thurung-king. \\
     good red colour when-become descend-\textsc{caus} \\
    `When it turns into a nice red colour, remove it (from the fire).' (K060103rec02)
\z
}



The same construction can be used for modal predications. Modal predications are of static aktionsart, but if for some reason a change in deontic  state (from possible to impossible or the other way round) takes place, a verbal periphrasis with \em jaadi \em must be employed to give the dynamic reading.\footnote{See \citet[224]{Karunatillake2004} for a related construction in Sinhala.} In \xref{ex:soa:events:mod:dyn:jaadi:su:modal1}, the high commissioner was first able to come, but finally became unavailable. This change of state is once more expressed by \trs{jaadi}{become}.


\xbox{16}{
\ea \label{ex:soa:events:mod:dyn:jaadi:su:modal1}
\gll Itthu    blaakang=jo,    kitham=pe {\em AGM}=nang  duppang,  {\em high} {\em commissioner} {\em cultural} {\em show}=nang mà-dhaathang=nang        \textbf{thàràboole} \textbf{s-jaadi}. \\
dist after=\textsc{emph} \textsc{1pl}=\textsc{poss} AGM=\textsc{dat} before high commissioner cultural show=\textsc{dat} \textsc{inf}-come-\textsc{dat} cannot \textsc{past}-become\\
    After that, before our Annual General Meeting, it became impossible for the High Commissioner to attend the cultural show. (K060116nar23)
\z
}

A similar constellation obtains in \xref{ex:soa:events:mod:dyn:jaadi:su:modal2}.

\xbox{16}{
\ea \label{ex:soa:events:mod:dyn:jaadi:su:modal2}
\gll See=yang dhaathang {\em remand}=ka mà-thaarek thaaro=nang \textbf{thàràboole} \textbf{su-jaadi}. \\
     \textsc{1s}=\textsc{acc} come remand=\textsc{loc} \textsc{inf}-pull put=\textsc{dat} cannot \textsc{past}-become  \\
    `It became impossible to remand me.' (K061122nar03)
\z
}



% \xbox{16}{
% \ea\label{ex:func:unreferenced}
% \gll Ithu=ka sakith aathi asà-jaadi aada. \\
%       \textsc{dist}=\textsc{loc} sick heart \textsc{cp}-become exist \\
%     `On that, discontent had arisen.' (K060116nar04)
% \z
% } \\




Nominal predicates construed with experiencers have the possibility to mark change of state with the verb \trs{pii}{go}, as in \xref{ex:soa:events:mod:dyn:pii}. In this example, the experiencer \em incayang \em is marked with the dative marker \em =nang \em which has `allative' as an additional meaning. This makes it possible to use it with the verb \trs{pii}{go}, entailing that the semantic role changes from experiencer to goal.

\xbox{16}{
\ea\label{ex:soa:events:mod:dyn:pii}
\gll Incayang=\textbf{nang} baaye=nang maara su-\textbf{pii}. \\
     \textsc{3s.polite}=\textsc{dat} good=\textsc{dat} angry \textsc{past}-go  \\
    `He became really angry (Anger went upon him).'  (K070000wrt01)
\z
}


Finallly, verbs construed with experiencers, like \trs{thàthaava}{laugh} can also use this periphrasis with \em pii \em to emphasize the change of state.\footnote{Note however that no physical object changes places towards the experiencers/goals in  \xref{ex:soa:events:mod:dyn:pii}\xref{ex:soa:events:mod:dyn:pii:exp}; if  that was the case, the relator noun \em dìkkath \em would have to be used on the human referent \funcref{sec:func:Goal}.}


\xbox{16}{
\ea\label{ex:soa:events:mod:dyn:pii:exp}
\gll Itthu vakthu=ka Andare asà-maathi anà-duuduk mosthor kuthumung=apa raaja=nang \textbf{thàthaava} su-\textbf{pii}. \\
   \textsc{dist} time=\textsc{loc} Andare \textsc{cp}-dead \textsc{past}-stay manner see=after king=\textsc{dat} laugh \textsc{past}-come    \\
    `When he then saw the way that Andare had died and lay there, the king started to laugh.' (K070000wrt03)
\z
}



There is one example where a word likely to be a noun is used without \em jaadi\em, but rather in the verbal frame discussed above, marked by a TAM-prefix. This is the loan word \em pension \em in \xref{ex:soa:events:mod:dyn:jaadi:contr}, but it could actually be argued that \em pension \em in SLM does in fact denote the property `retired', and therefore is an adjective, which can undergo conversion to become a verb for dynamic reading, as usual.

\xbox{16}{
\ea\label{ex:soa:events:mod:dyn:jaadi:contr}
\gll Derang pada samma konnyong aari pukurjan asà-gijja, \textbf{su-{\em pension}}. \\
      \textsc{3pl} \textsc{pl} all few day work \textsc{cp}-make \textsc{past}-pension \\
    `They worked some time and then got pensioned.' (K051222nar06)
\z
}



\subsection{Causation}\label{sec:func:Causation}

There are two ways to indicate causation: the causative morpheme \em -king \em \formref{sec:morph:-king}  and a construction using a verb of saying like \trs{biilang}{say}.


% \xbox{16}{
% \ea \label{ex:caus:king:adj}
% \gll Itthuka asà-thaaro, itthu=yang arà-\textbf{panas-king}. \\
%       \textsc{dist}=\textsc{loc} \textsc{cp}-put \textsc{dist}=\textsc{acc} \textsc{non.past}-hot-\textsc{caus} \\
%     `Having put (it) there, you heat it.'  (B060115rcp02)
% \z
% }\\


\xbox{16}{
\ea \label{ex:func:caus:king}
\gll Baaye meera caaya kapang-jaadi, \textbf{thurung-king}. \\
     good red colour when-become, descend-\textsc{caus}  \\
    `When  [the food] has  turned to a nice rose colour, remove (it) [from the fire].'  (K060103rec02)
\z
}

% \xbox{16}{
% \ea \label{ex:caus:king:trans}
% \gll Kitham  arà-\textbf{mirthi-kang}       Kluu\u mbu {\em confederation}=nang     kithang=nang daapath {\em latest} twelve=ka. \\
%      \textsc{1pl} \textsc{non.past}-understand-\textsc{caus} Colombo c.=\textsc{dat} \textsc{1pl}=\textsc{dat} l. twelfth=\textsc{loc} \\
%     `We make the Colombo confederation understand that we have received (the letter) (only) on the last twelfth (of the month).' (K060116nar23)
% \z
% }\\

Verbally causing someone to do something can be expressed with an utterance verb in the main clause and an infinitive clause with the action ordered.

\xbox{16}{
\ea \label{ex:func:caus:biilang}
\gll Oorang pada=nang     \textbf{mà-dhaathang}     \textbf{katha} asà-\textbf{biilang}. \\
     man \textsc{pl}=\textsc{dat} \textsc{inf}-come \textsc{quot} \textsc{cp}-tell  \\
    `He told the people to come.' (B060115cvs01)
\z
}


%\section{Giving extra information beyond the nuclear predication}\label{sec:func:Givingextrainformationbeyondthenuclearpredication}
%Next to the basic information about predicate, arguments, space and time, there are some other, more peripheral, pieces of information people want to communicate about. These have received various names in the literature, like circumstantials\citep{}, satellites\citep{}, or XXX\citep{}. These have in common that they are not central or necessary for the predication. The predication would still be complete without them. Rather, they provide additional information, like reasons, conditions, purpose,

% \subsubsection{Quantity?}\label{sec:func:Quantity?}
%
% \xbox{16}{
% \ea\label{ex:func:unreferenced}
% \gll Ini      svaara liivath. \\
%       \textsc{prox} noise much \\
%     `There is a lot of noise there.' (K051222nar04)
% \z
% } \\


\section{Modification}\label{sec:func:Modification}
Different modification strategies exist in SLM. These depend fully on the lexical category of the head noun, regardless of its semantic class. The different possibilities are   discussed in \formref{sec:nppp:Thefinalstructureofthenounphrase} for referential phrases (NPs) and \formref{sec:pred:Verbalpredicates} for predicate phrases.

% \subsection{Modifying events}\label{sec:func:Modifyingevents}
% Modification of events can be done in several ways. The first one is to use an NP with a dative marker, as in \xref{ex:func:soa:event:mod:nang}. The second one is to use a suitable vector verb, which for the time being would be only \trs{thaaro}{hit} to add a meaning of `violent' to the event.
%
% \xbox{16}{
% \ea\label{ex:func:soa:event:mod:nang}
% \gll {\em Bras}-iyang \textbf{baae=nang} \textbf{cuuci}. \\
%  raw.rice-yang good=\textsc{dat} wash\\
% `Wash the rice well!' (K060103rec01)\z
% }
%
% \xbox{16}{
% \ea\label{ex:func:soa:event:mod:thaaro1}
% \gll [Incayang=pe kàpaala=ka anà-aada] thoppi=dering moonyeth pada=nang su-buvang \textbf{puukul}. \\
%       \textsc{3s.polite}=\textsc{poss} head=\textsc{loc} \textsc{past}-exist hat=\textsc{abl} monkey \textsc{pl}=\textsc{dat} \textsc{past}-throw hit \\
%     `He took the hat from his head and violently threw it  at the monkeys.'  (K070000wrt01)
% \z
% }\\
%
% The third way to modify an event is to use the simultaneous participle, which is formed by reduplication. An example of the event of running being modified by such a participle, which indicates the jumping manner in which the running happens, is given in \xref{ex:func:soa:event:mod:redup}.
%
% \xbox{16}{
% \ea\label{ex:func:soa:event:mod:redup}
% \gll Kancil \textbf{lompath}\~{}\textbf{lompath} arà-laari. \\
%      rabbit jump\~{ }jump      \textsc{non.past}-run \\
%     `The rabbit runs away jumping.'  (K081106eli01)
% \z
% }\\
%
% A fourth way, which might be more marginal, is the modification of an event by indicating what did \em not \em happen at the same time. This arguable only constitutes one event, not two, so that the main event is modified by the (non-existent) second one.
%
%
% \xbox{16}{
% \ea\label{ex:func:soa:event:mod:jama}
% \gll Kithang=nang bole=duuduk hatthu=le \textbf{jamà-maakang=nang}  two duva two o' {\em clock}=ke  sangke bole=duuduk. \\
%        \textsc{1pl}=\textsc{dat} can-stay \textsc{indef}=\textsc{addit} \textsc{neg.nonfin}-eat=\textsc{dat}  two two two o'clock=\textsc{simil} until can-stay  \\
%     `We can stay up until two o' clock without eating anything.' (K061026rcp04)
% \z
% } \\
%
% The indication what indeed \em did \em happen at the same time would probably not be a modification of the first event, but rather information about a second event.
%
% % \xbox{16}{
% % \ea\label{ex:func:soa:event:mod:thaaro2}
% % \gll Ithu=kapang ithu moonyeth pada=le [anà-maayeng duuduk thoppi] pada=dering inni oorang=nang su-\textbf{bale-king} \textbf{puukul}. \\
% %       \textsc{dist}=when \textsc{dist} monkey \textsc{pl}=\textsc{addit} \textsc{past}-play sit hat \textsc{pl}=\textsc{abl} \textsc{prox} man=\textsc{dat} \textsc{past}-return-\textsc{caus} hit \\
% %     `Then the monkeys threw back the hats with which they had been playing.' (K070000wrt01)
% % \z
% % } \\
%
% %
% % \xbox{16}{
% % \ea\label{ex:func:unreferenced}
% % \gll Ini [kuurang arà-duuduk]        laayeng kumpulan  pada=yang   mà-{\em represent}-kang=nang. \\
% %       \textsc{prox} few \textsc{non.past}-exist.\textsc{anim} other group \textsc{pl}=\textsc{acc} \textsc{inf}-represent-\textsc{caus}=\textsc{dat} \\
% %     `To represent the other groups with few people staying in them (i.e. minorities).' (N061031nar01)
% % \z
% % } \\
%
%
% % \xbox{16}{
% % \ea\label{ex:func:jama:negcp1}
% % \ea
% % \gll Liivath aayer \textbf{jamà}-jaadi=\textbf{nang}. \\
% %      much water \textsc{neg.nonfin} become=\textsc{dat}  \\
% %     `Without putting too much water' (K060103rec01)
% % \ex
% % \gll Itthu aayer=yang hathu blaangan=nang luppas. \\
% %      \textsc{dist} water=\textsc{acc} \textsc{indef} amount=\textsc{dat} leave  \\
% %     `leave that water for a while.' (K060103rec01)
% % \z
% % \z
% % } \\
%

\section{Space}\label{sec:func:Space}
Participants, states and events are always located in space. This section discusses the different possibilities to give spatial information in SLM. The next section will deal with the counterpart of space: time. Some concepts, like figure and ground are important for both domains.

We can distinguish three different kinds of spatial reference: absolute (non-deictic) reference (\em in India\em) \funcref{sec:func:Givingthenon-deicticreferencespace}, deictic reference with reference to the speaker (\em here\em) \funcref{sec:func:Givingthedeicticreferencespacewithregardtospeechactparticipants} and figure-ground relations, which involve several entities (\em between a rock and a hard place\em) \funcref{sec:func:Figure-groundrelations}. Finally, events can also have an inherent directionality, like \em come and bring \em or \em go and take \em in English \funcref{sec:func:Indicatingthespatialorientationofanevent}.

\subsection[Non-deictic space]{Giving the non-deictic reference space}\label{sec:func:Givingthenon-deicticreferencespace}
The absolute reference space is normally indicated  by the locative marker \em =ka \em \formref{sec:morph:=ka} attached to an NP with local reference.  This can be a common noun as in \xref{ex:space:nond:cn} or a proper noun as in \xref{ex:space:nond:pn}. In rare cases, the locative marker is not present on proper nouns, as in \xref{ex:space:nond:noka}.


\xbox{16}{
\ea \label{ex:space:nond:cn}
\gll Se   m-blaajar      \textbf{{\em estate}=ka}. \\
 \textsc{1s} \textsc{past}-learn estate=loc\\
`I learned on the estate.' (K051213nar05)
\z
}


\xbox{16}{
\ea \label{ex:space:nond:pn}
\gll [\textbf{Kandi=ka}    arà-duuduk]      {\em military} rejimen mlaayu. \\
     Kandy=\textsc{loc}  \textsc{non.past}-stay military regiment Malay \\
    `The Military regiment Malays who stayed in Kandy.' (K060108nar02)
\z
}


\xbox{16}{
\ea\label{ex:space:nond:noka}
\gll Seelon  samma thumpath=\zero{} mlaayu aada. \\ % bf
      Ceylon all place Malay exist\\
    `There are Malays all over Sri Lanka.' (K051222nar04)
\z
}



\subsection[Deictic space]{Giving the deictic reference space with regard to speech act participants}\label{sec:func:Givingthedeicticreferencespacewithregardtospeechactparticipants}
The spatial location of referents with regard to the speaker can be given by the deictics \trs{ini}{proximal}{} and \trs{itthu}{distal}.



\xbox{16}{
\ea \label{ex:space:inni}
\gll \textbf{Inni}     sudaari=pe   femili=ka    bannyak oorang tsunami=da     s-puukul. \\
     \textsc{prox} sister=\textsc{poss} family=\textsc{loc} many man tsunami=\textsc{dat}  \textsc{cp}-hit\\
    `Of this sister's family, many members were hit by the Tsunami.' (B060115nar02)
\z
}


\xbox{16}{
\ea \label{ex:space:itthu}
\gll \textbf{Itthu}    {\em ports}=ka    laama kar asà-baapi. \\
     \textsc{dist} ports=\textsc{loc} old car \textsc{cp}-bring  \\
    `We brought the old cars to those ports.' (K051206nar19)
\z
}

The spatial location of events with regard to the speaker can be indicated by the spatial adverbs \em siini \em and \em siithu\em. These can optionally bear a locative clitic as in \xref{ex:space:siini:ka}\xref{ex:space:siithu:ka}, or be used without one, as in \xref{ex:space:siini:noka}\xref{ex:space:siithu:noka}.

\xbox{16}{
\ea \label{ex:space:siini:ka}
\gll \textbf{Siini=ka}    {\em settle} daapath, itthu=nam       blaakang bannyak oorang pada \textbf{siini} se-duuduk. \\
      here=\textsc{loc} s. get, \textsc{dist}=\textsc{dat} after many man \textsc{pl} here \textsc{past}-stay\\
    `They settled down here, after that many people came to stay.' (G051222nar03)
\z
}


\xbox{16}{
\ea \label{ex:space:siithu:ka}
\gll \textbf{Siithu=ka}    se   cinggala   oorang=pe cinggala   em-blaajar. \\
 there=\textsc{loc} \textsc{1s} Sinhala man=\textsc{poss} Sinhala \textsc{past}-learn \\
`It was there that I learned the Sinhalas' Sinhala.' (K051213nar02)
\z
}



\xbox{16}{
\ea \label{ex:space:siini:noka}
\gll Spaaru \textbf{siini}=\zero{} su-duuduk. \\ % bf
       few here \textsc{past}-stay\\
    `Few stayed here.' (K051205nar04)
\z
}


\xbox{16}{
\ea \label{ex:space:siithu:noka}
\gll Incayang  \textbf{siithu}=\zero{} asà-kaaving. \\ % bf
       \textsc{3s.polite} there \textsc{cp}-marry\\
    `He married there.' (K051206nar18)
\z
}

While marking of the locative is optional, marking of the allative is impossible for deictics \xref{ex:space:siini:all} and the marking of the ablative is obligatory \xref{ex:space:siini:abl}.


\xbox{16}{
\ea \label{ex:space:siini:all}
\ea   \em *siini=nang \em
\ex \em *siithu=nang \em
\z
\z
}

\xbox{16}{
\ea\label{ex:space:siini:abl}
\gll See arà-sumpa  paanas muusing dhaathang=thingka see siini=\textbf{dering} arà-pii. \\
     \textsc{1s} \textsc{non.past}-promise hot time come=middle \textsc{1s} here=\textsc{abl} \textsc{non.past}-go  \\
    `I promise, I will leave when summer will have come.' (K070000wrt04)
\z
}



The spatial character can be highlighted by the used of \trs{subla}{side}, as is the case in the following example.


\xbox{16}{
\ea\label{ex:space:subla}
\ea
\gll Biini itthu    \textbf{subla}=dering  arà-bitharak; \\
      wife \textsc{dist} side=\textsc{abl} \textsc{non.past}-scream \\
    `(Andare's) wife screamed from that side;'
\ex
\gll puthri inni     \textbf{subla}=dering  arà-bitharak. \\
      queen \textsc{dist} side=\textsc{abl} \textsc{non.past}-scream \\
    `the queen screamed from this side.' (K070000wrt05)
\z
\z
}


Spatial location close to the hearer can be indicated by \em sanaka\em, but this is rarely done. Normally \em  siini \em is used, if the hearer is thought to be close to the speaker, or \em siithu \em if the hearer happens to be farther away.

Among third person pronouns, \em siaanu \em can only be used for persons close to the speaker.

Deictic and non-deictic spatial reference can be combined, as in \xref{ex:space:deicnondeic}, where we find non-deictic \em Sri Lankaka \em and deictic \em siini\em.

\xbox{16}{
\ea\label{ex:space:deicnondeic}
\gll See spuulu thaaun \textbf{siini} \textbf{Sri} \textbf{Lanka=ka}  pukurjan nya-kirja. \\
      \textsc{1s} ten years here Sri Lanka=\textsc{loc} work \textsc{past}-make \\
    `I worked here in Sri Lanka for ten years.' (K061026prs01)
\z
}


More precise indications of spatial location with regard to speech act participants can be given by using one or more of them as ground in a figure-ground relation. This will be explained in the next section.


% \xbox{16}{
% \ea\label{ex:func:unreferenced}
% \gll See ini      Sri  Lanka=ka    nya-blaajar. \\
%       \textsc{1s} \textsc{prox} Sri Lanka=\textsc{loc} \textsc{past}-learn \\
%     `I studied here in Sri Lanka.' (K061026prs01)
% \z
% } \\




\subsection{Figure-ground relations}\label{sec:func:Figure-groundrelations}
Besides the non-deictic reference space and the reference space with regard to speech act participants, it is possible to give the spatial relation between two or more arbitrary entities. The locative marker \em =ka \em can be used to convey a generic figure-ground relation, where nothing about the precise disposition of figure and ground in implied. More precise constellations can be obtained by using relator nouns \formref{sec:wc:Relatornouns} \citep[cf.][]{Adelaar1991}. These relator nouns indicate stative information (top, bottom, front, back). Lative information (from the inside, to the middle) can be obtained by combining them with the relevant postpositions \trs{=ka}{essive}{}, \trs{=nang}{allative}{} \trs{=dering}{ablative}{}.

The most common case is to combine two nouns in such a figure ground relation. The following examples show a number of relator nouns

\xbox{16}{
\ea\label{ex:space:figground:atthas}
\gll Ini pohong$_{ground}$ \textbf{atthas}=ka [moonyeth hathu kavanan]$_{figure}$ su-aada. \\ % bf
     \textsc{prox} tree top=\textsc{loc}  monkey \textsc{indef} group \textsc{past}-exist\\
    `On top of this tree was a group of monkeys.'   (K070000wrt01)
\z
}


% \xbox{16}{
% \ea\label{ex:space:figground:daalng}
%    \gll Kiccil wavvaal pada daalang=ka  arà-duuduk. \\
%      small  bat     \textsc{pl}   inside=\textsc{loc} \textsc{non.past}-exist.\textsc{anim} \\
% `There are small bats inside' (K051206nar02)
% \z
% }



\xbox{16}{
\ea\label{ex:space:figground:baava}
\gll Andare$_{figure}$ [hathu pohong]$_{ground}$=pe \textbf{baava}=ka kapang-duuduk. \\ % bf
     Andare \textsc{indef} tree=\textsc{poss} bottom=\textsc{loc} when-sit  \\
    `When Andare sat down under a tree.' ((K070000wrt03))
\z
} 


\xbox{16}{
\ea\label{ex:space:figground:dikkath}
\gll Soore=ka, [Snow-white=le Rose-red=le]$_{figure}$ derang=pe umma=samma appi$_{ground}$ \textbf{dìkkath}=ka arà-duuduk ambel.  \\ % bf
      Evening=\textsc{loc} Snow.white=\textsc{addit} Rose.Red=\textsc{addit} \textsc{3pl}=\textsc{poss} mother=\textsc{comit} fire vicinity=\textsc{loc} \textsc{simult}-sit take \\
    `In the evening, Snow White and Rose Red used to sit down next to the fire with their mother.'  (K070000wrt04)
\z
}

% \xbox{16}{
% \ea\label{ex:space:figground:blaakang}
% \gll Ruma  saakith Suvasevana {\em hospital}=pe     \textbf{blaakang}=ka. \\
%       house sick Suvasena hospital=\textsc{poss} back=\textsc{loc} \\
%     `Behind the Suvasevana hospital.' (K051220nar01)
% \z
% } \\


Pronouns can be used in this construction as well \xref{space:figgr:pron:3}.

\xbox{16}{
\ea \label{space:figgr:pron:3}
\gll [Incayang]$_{ground}$=pe      baa=ka      [spaaru]$_{figure}$ aada. \\
3\textsc{s.polite}=\textsc{poss} down=\textsc{loc} some exist\\
`There are some [Malays] down the hill from  where he lives.' (K051213nar05)
\z
}


% \xbox{16}{
% \ea \label{space:figgr:pron:1}
% \gll Kithang loram=pe atthas anà-oomong. \\
%       \textsc{1pl} \textsc{2pl}=\textsc{poss} about \textsc{past}-talk \\
%     `We talked about you.' (K081106eli01)
% \z
% }

%
% \xbox{16}{
% \ea \label{space:figgr:pron:1}
% \gll Se=ppe diiri=ka maanjur aada. \\
%        \\
%     `.' (K081106eli01)
% \z
% } \\

The interpretation of the relator noun need not be literal. In \xref{space:figgr:leeway}, the neighbour from above does not live directly on top of the speakers, but rather on the next floor. Still, the use of \trs{ruuma}{house} or \trs{thattu}{roof} is not possible, and the simple pronoun \em kitham \em is used instead.


\xbox{16}{
\ea \label{space:figgr:leeway}
\gll Marian kitham=pe (*ruuma) atthas (*thattu)=ka arà-duuduk. \\
     Marian \textsc{1pl}=\textsc{poss} house top roof=\textsc{loc} \textsc{non.past}-live  \\
    `Marian lives above us.' (K081106eli01)
\z
}


% \xbox{16}{
% \ea
% \gll Kuuthu se=ppe paala=ka aada. \\
%      louse \textsc{1s=poss} head=\textsc{loc} exist  \\
%     `There are lice on my head.' (K081106eli01)
% \z
% } \\

Relative indications of location can be expressed by \trs{thangang naasi subla}{hand' + `rice' + `subla'=`right hand side} and \trs{thangang kiiri subla}{left hand side} (The latter has no transparent meaning). The following four examples show this, as well as using \em thìnnga \em for `in the middle' and \em duppang \em for `opposite'.

\xbox{16}{
\ea
\gll Izi se=ppe \textbf{thangang} \textbf{naasi} subla=ka arà-duuduk. \\
     Izi \textsc{1s=poss} hand rice side=\textsc{loc} \textsc{non.past}-sit  \\
    `Izi is sitting at my right hand side.' (K081106eli01)
\z
}


\xbox{16}{
\ea
\gll Sebastian se=ppe \textbf{thangang} \textbf{kiiri} subla=ka arà-duuduk. \\
       Sebastian \textsc{1s=poss} hand left side=\textsc{loc} \textsc{non.past}-sit  \\
    `Sebastian is sitting at my left hand side.' (K081106eli01)
\z
}


\xbox{16}{
\ea
\gll See \textbf{thìnnga}=ka arà-duuduk. \\
     \textsc{1s} middle=\textsc{loc} \textsc{non.past}-sit  \\
    `I am sitting in the middle.' (K081106eli01)
\z
}


\xbox{16}{
\ea
\gll Sebastian se=ppe \textbf{duppang}=ka arà-duuduk. \\
       Sebastian \textsc{1s=poss} front=\textsc{loc} \textsc{non.past}-sit  \\
    `Sebastian is sitting in front of me.' (K081106eli01)
\z
}



The essive has been exemplified by the sentences above.   Allative \xref{ex:space:figgr:allative} and ablative \xref{ex:space:figgr:ablative} are exemplified by the following examples.


\xbox{16}{
\ea \label{ex:space:figgr:allative}
\gll Kithang arà-pii    inni     {\em politicians} pada dìkkath=\textbf{nang}. \\
      \textsc{1pl} \textsc{non.past}-go \textsc{prox} politicians \textsc{pl} vicinity=\textsc{dat} \\
    `We approach these politicians.' (K051206nar12)
\z
}

\xbox{16}{
\ea \label{ex:space:figgr:ablative}
\ea
\gll Biini itthu    subla=\textbf{dering}  arà-bitharak. \\
      wife \textsc{dist} side=\textsc{abl} \textsc{non.past}-scream \\
    `(Andare's) wife screamed from that side;'
\ex
\gll Puthri inni     subla=\textbf{dering}  arà-bitharak. \\
      queen \textsc{dist} side=\textsc{abl} \textsc{non.past}-scream \\
    `the queen screamed from this side.' (K070000wrt05)
\z
\z
}


The perlative is also formed with \em =dering\em, as shown in the following example.

\xbox{16}{
\ea \label{ex:space:figgr:perlative}
\gll Bìssar hathu buurung derang=pe atthas=\textbf{dering} su-thìrbang. \\
      big \textsc{indef} bird \textsc{3pl}=\textsc{poss} top=\textsc{abl} \textsc{past}-fly \\
    `A big bird flew over them.'  (K070000wrt04)
\z
}


Reciprocal grounding is also possible as in \xref{ex:space:figgr:reciprocal}, where the two places are close to each other. This is highlighted by the use of the dative marker \em =nang \em on both, as well as the use of two additive markers \em =le\em.


\xbox{16}{
 \ea \label{ex:space:figgr:reciprocal}
   \gll [Spaaman anà-nii\u n\u ggal \textbf{thumpath=nang=le}]        [Passara   katha arà-biilang    \textbf{nigiri=nang=le}] \textbf{dìkkath}. \\
         \textsc{3s} \textsc{past}-die place=\textsc{dat}=\textsc{addit} Passara \textsc{quot} \textsc{non.past}-say country=\textsc{dat}=\textsc{addit} vicinity\\
 	`The place where he died and the village called Passara are close to each other.' (B060115nar05)
\z
}


\begin{table}
	\centering
		\begin{tabular}{ll}
			on & atthas\\
			under & baava\\
			in front & duppang\\
			behind &blaakang\\
			next to & dìkkath\\
			in & daalang\\
  			out & luvar\\
% 			around\\
			in the middle& thìnnga\\
			thangang naasi & right\\
 			thangang kiiri & left\\
		\end{tabular}
	\caption[Spatial orientation]{Spatial orientation}
	\label{tab:SpatialOrientation}
\end{table}




\subsection{Indicating the spatial orientation of an event}\label{sec:func:Indicatingthespatialorientationofanevent}
The spatial orientation of an event is indicated by attaching the clitics for the semantic roles of source, goal and location \funcref{sec:func:Source} \funcref{sec:func:Goal} to the  NP expressing the place.

\xbox{16}{
\ea \label{ex:space:orientation}
\gll Itthu    baathu=yang    incayang Seelong=\textbf{dering} laayeng nigiri=\textbf{nang} asà-baapi. \\
 \textsc{dist} stone=\textsc{acc} \textsc{3s.polite} Ceylon-\textsc{abl} other country=\textsc{dat} \textsc{cp}-bring\\
`These stones, he brought them from Ceylon to other countries.' (K060103nar01)
\z
}


The deictic adverbs \em siini \em and \em siithu \em  are never combined with \em =nang \em to indicate  goal.
They can be combined with the ablative marker \em =dering\em, as shown in \xref{ex:space:orientation:deic:dering}.

\xbox{16}{
\ea \label{ex:space:orientation:nodeic}
\gll {\em Second} {\em world} {\em war} vatthu siini-\textbf{\zero} dhaathang aada atthu  kappal. \\
      second world war time here come exist \textsc{indef} ship \\
    `During the second world war a ship came here.' (K051206nar07)
\z
}


\xbox{16}{
\ea \label{ex:space:orientation:nodeic2}
\gll [Boole oorang pada]=nang   siithu=\textbf{\zero} boole=pii. \\
     can man \textsc{pl}=\textsc{dat} there can=go  \\
    `All men who are able to go may go.'  (B060115cvs01)
\z
}


\xbox{16}{
\ea \label{ex:space:orientation:deic:dering}
\gll Oorang mlaayu siithu=\textbf{dering}  dhaathang=apa cinggala  raaja=nang=le anà-banthu. \\
      man Malay there=\textsc{abl} come=after Sinhala king=\textsc{dat}=\textsc{addit} \textsc{past}-help \\
    `The Malays came from there and helped the Sinhala king.' (K051206nar04)
\z
}

For the spatial orientation of motion with regard to a figure, the lexical solutions \trs{pii}{go}, \trs{dhaathang}{come} and \trs{baalek}{return} are available. If a container is involved, \trs{maasok}{enter} and \trs{kuluvar}{exit} are used. For vertical motion, the verbs \trs{naaek}{ascend} and \trs{thuurung}{descend} are available. The verb \trs{bavung}{get up} also includes a spatial component.  \xref{ex:space:orientation:thuurung} gives an example of the verb \trs{thuurung}{descend}.


\xbox{16}{
\ea \label{ex:space:orientation:thuurung}
\ea
\gll Ithu oorang=nang baaye=nang nanthok pii=nang blaakang. \\ % bf
     \textsc{dist}=\textsc{dat} man=\textsc{dat} good=\textsc{dat} sleep go=\textsc{dat} after \\
    `After the man had well fallen asleep'
\ex
\gll Pohong=dering baava=nang asà-\textbf{thuurung}. \\
       tree=\textsc{abl} down=\textsc{dat} \textsc{cp}-descend\\
    `(the monkeys) climbed down from the tree and'
\ex
\gll [Oorang anà-baava] samma thoppi=pada asà-ambel. \\ % bf
      man \textsc{past}-bring all hat=\textsc{pl} \textsc{cp}-take\\
    `took all the hats the man had brought and'
\ex
\gll Mà-maayeng=nang su-mulain. \\ % bf
      \textsc{inf}-play=\textsc{dat} \textsc{past}-start \\
    `started to play.' (K070000wrt01)
\z
\z
}


\section{Time}\label{sec:func:Time}

Just like space, time can be anchored in a non-deictic way. The temporal expression \em in 1984 \em corresponds to the spatial expression \em in India \em \funcref{sec:func:Givingnon-deicticreferencetime}.\footnote{It is true that \em in 1984 \em is actually anchored in a figure-ground relation with the conventional  birth date of Jesus Christ, but speakers normally do not conceptualize this birth when they use such a time reference.}
As a second possibility, time can be indicated relative to the speech act \funcref{sec:func:Givingthereferencetimewithregardtothespeechact}. We can distinguish the periods before, during and after the speech act.
The temporal relation between a figure and a ground can be expressed in very much the same fashion as for space, as in \em before the war\em, which corresponds to \em before the church \em \funcref{sec:func:Figureandgroundinthetemporaldomain}.  Taking into account the start time, end time, duration and overlap of the two events, we can distinguish a number of situations, such as precedence, subsequence, point coincidence, etc.

The location of time of an event contrasts with its internal temporal structure. We can distinguish phasal information (start, progression, end) \funcref{sec:func:Phasalinformation} from aspectual information (perfective, imperfective) \funcref{sec:func:Aspectualstructure}. A discussion of the encoding of temporal frequency concludes this section \funcref{sec:func:Temporalfrequency}.



\subsection[Non-deictic reference time]{Giving non-deictic reference time}\label{sec:func:Givingnon-deicticreferencetime}
The means SLM uses to indicate non-deictic time resemble the means used for absolute space. Temporal reference is marked with the postposition \em=ka \em \xref{ex:time:abs:ka}. This postposition is frequently dropped in rapid speech \xref{ex:time:abs:noka}.


\xbox{16}{
\ea \label{ex:time:abs:ka}
\gll Soore\textbf{=ka}, Snow-white=le Rose-red=le derang=pe umma=samma appi dìkkath=ka arà-duuduk ambel.  \\
      Evening=\textsc{loc} Snow.white=\textsc{addit} Rose.Red=\textsc{addit} \textsc{3pl}=\textsc{poss} mother=\textsc{comit} fire vicinity=\textsc{loc} \textsc{simult}-sit take \\
    `In the evening, Snow White and Rose Red used to sit down next to the fire with their mother.'  (K070000wrt04)
\z
}

\xbox{16}{
\ea \label{ex:time:abs:noka}
\gll Kithang anà-pii    pugi \rm {\em week}=\zero{}. \\ % bf
     \textsc{1pl} \textsc{past}-go last week  \\
    `We went last week.' (K051206nar07)
\z
}
 

Years are indicated by their number in Common Era. For this, English numbers are normally used \xref{ex:time:abs:year}, but Malay numbers can also be found \xref{ex:time:abs:month}. Also months are indicated by their English names \xref{ex:time:abs:month}.

\xbox{16}{
\ea \label{ex:time:abs:year}
\gll Suda \textbf{{\em nineteen}-{\em ninety}-{\em four}}=ka        se=ppe    {\em husband} su-nnii\u n\u ggal. \\
      thus nine-teen-ninety-four=\textsc{loc} \textsc{1s=poss} husband \textsc{past}-die \\
    `So, my husband died in 1994.' (K051201nar01)
\z
}

\xbox{16}{
\ea \label{ex:time:abs:month}
\gll Limapulunnam=ka, \textbf{\rm April}  buulang=dìkka. \\
56=\textsc{loc} april month=vicinity \\
`In '56, around the month of April.' (K060108nar01)
\z
}

% \xbox{16}{
% \ea\label{ex:func:unreferenced}
% \gll Mampus    blaakang sriibus  sbiilan-raathus     nnampuulu     thuuju thaaun vatthu laskalli inni     Melayu Kumpulan    di Kandi=yang         kumpulkang. \\
% croak  after thousand nine-hundred  sixty seven year time other.time \textsc{dist} Melayu Kumpulan Di Kandi-yang add-caus\\
% `.' (K060116nar02)
% \z
% }



For the days of the week as given in Table \ref{tab:DaysOfTheWeek}, both native and English words are used.

\begin{table}
\centering
 \begin{tabular}{ll}
Monday & (h)ari sinnen\\
Tuesday&(h)ari slaasa\\
Wednesday& (h)ari rubbo \\
Thursday& (h)ari k(h)uumis \\
Friday&(h)ari jumahath \\
Saturday&(h)ari satthu \\
Sunday&(h)ari ahath/ahadh \\
 \end{tabular}
 \caption[Days of the week]{Days of the week. The letters in parentheses might be preferred by speakers in the orthography, but are not present phonetically.}
 \label{tab:DaysOfTheWeek}
\end{table}


The hours of the clock are indicated by preposing \em pu(ku)l \em to the number.

\xbox{16}{
\ea\label{ex:func:time:abs:hour}
\gll Pukul ìnnam. \\ % bf
 hit six\\
`Six o'clock.' (B060115cvs09)
\z
}

Special words exist for `noon' (\em thìngaari\em) and `midnight' (\em thìngamaalang\em).
Fractions of the hour are expressed by \trs{spaaru}{half} and \trs{kaarthu}{quarter}.

\subsection[Deictic reference time]{Giving the reference time with regard to the speech act}\label{sec:func:Givingthereferencetimewithregardtothespeechact}
Besides non-deictic time reference, events can be situated with regard to the speech act, to wit, before, during and after the speech act. Another possibility is `general truth', which has no fixed time-reference.

We can distinguish lexical solutions  from grammaticalized solutions.  (Cf. Table \ref{tab:TemporalAdverbs}.)

 \begin{table}
	\centering
		\begin{tabular}{ll}
			itthu muusingka & in former times\\
			kumaareng dovulu & the day before yesterday\\
			kumaareng & yesterday\\
			nyaari & today\\
			beeso(na) & tomorrow\\
			luuso(na) & (the/some) day after tomorrow\\
			pugi\footnotemark X & last X\\
			hathiya X & next X\\
		\end{tabular}
	\caption{Temporal adverbs}
	\label{tab:TemporalAdverbs}
\end{table}\footnotetext{This is a development from historical \trs{*piggi}{go}, \citep[214]{Adelaar2005struct}.}


As for grammaticalized solutions, SLM features a developed TAM system, which is used to convey temporal meaning.

There are no morphosyntactic limitations for lexical expressions of time, but due to the impossibility of having more than one verbal prefix \citep{Slomanson2006cll}, the presence of a non-tense prefix (i.e. aspectual, phasal, modal, infinitive) blocks the expression of tense on the verb. The resulting verb can then have any time reference. In the following example, the verbal prefix \trs{kapang-}{when} blocks expression of tense on the word, and the resulting clause is ambiguous with regard to time-reference.
Furthermore, and unrelated to the use of \em kapang-\em, \em thàràthaau \em also is ambiguous with regard to time-reference, so that this sentence consists of two clauses which do not indicate the reference time.

\xbox{16}{
\ea \label{ex:time:abs:kapang}
\gll Incayang  \textbf{kapang}-dhaathang,  cinggala  incayang=nang    thàrà-thaau. \\
      \textsc{3s.polite} when-come Sinhala \textsc{3s.polite}=\textsc{dat} \textsc{neg}-know \\
    `When he came/comes/will come, he does/did/will not know Sinhala.' (K060108nar02)
\z
}

Both lexical and grammatical solutions will be discussed in more detail for the different logical possibilities.

\subsubsection{Before speech act}\label{sec:func:Beforespeechact}
The first logical possible constellation is that the reference time precedes the time of the speech act. This constellation can be expressed by both lexical and grammatical means. Among the lexical means, we number temporal adverbs like \trs{kumaareng}{yesterday}{} or less specific \trs{(kà)thaama}{earlier}{}, as well as constructions involving the adverb \trs{pugi}{last}{} or the distal deictic \em itthu\em.

\xbox{16}{
\ea \label{ex:func:time:deic:before:kumaareng}
\gll \textbf{Kumaareng}=le thuuju=so dhlaapan=so oorang asà-buunung, ... \\
      yesterday=\textsc{addit} seven=\textsc{undet} eight=\textsc{undet} man \textsc{cp}-kill  ... \\
    `Again yesterday, seven or eight men were killed .'  (K051206nar11)
\z
}

\xbox{16}{
\ea \label{ex:func:time:deic:before:kàthaama}
\gll \textbf{Kàthaama} kithang kiccil muusing=ka   inni     Peeradheniya jaalang=ka samma an-aada      mlaayu. \\
       before \textsc{1pl} small time=\textsc{loc} \textsc{prox} Peradeniya road=\textsc{loc} all \textsc{past}-exist Malay\\
    `Before, when we were children, it was all Malays in the Peradeniya Road.' (K051222nar04)
\z
}

\xbox{16}{
\ea \label{ex:func:time:deic:before:pugi}
\gll Kithang anapii    \textbf{pugi} \rm {\em week}. \\
     \textsc{1pl} \textsc{past}-go last week  \\
    `We went last week.' (K051206nar07)
\z
}

\xbox{16}{
\ea \label{ex:func:time:deic:before:itthumuusing}
\gll \textbf{Itthu}    \textbf{muusing=ka}    cinggala  thraa. \\
     \textsc{dist} time=\textsc{loc} Sinhala \textsc{neg}  \\
    `At that time, there was no Sinhala.' (K051222nar06)
\z
}



\em (Kà)thaama \em can also be used to indicate the amount of time that lies between the event and the speech act.

% \xbox{16}{
% \ea\label{ex:func:unreferenced}
% \gll Itthu avuliya karang sraathus liimapuulu thaaun=nang thaama. \\
%  \textsc{dist} saint now hundred fifty year=\textsc{dat} earlier\\
% `That saint (lived) 150 years ago.' (nosource)
% \z
% }
%
% \xbox{16}{
% \ea\label{ex:func:unreferenced}
% \gll Karang atthu  spuulu thaaun=nang=ke thaama. \\
%  now one ten years=\textsc{dat}=\textsc{simil} earlier\\
% `About ten years ago.' (nosource)
% \z
% }

\xbox{16}{
\ea \label{ex:func:time:deic:before:thaama}
   \gll \textbf{S-raathus} \textbf{limapulu}    \textbf{thaaun=nang}   \textbf{thaama}   incayang  bannyak igaama=pe atthas sbaayang naaji. \\
one-hundred five-ty year=\textsc{dat} early \textsc{3s.polite} much religion=\textsc{poss} about pray recite\\
`150 years  ago,  he recited a lot about religion,' (K051220nar01)
\z
}

The absolute use and the relative use of \em (kà)thaama \em are exemplified at the same time in the following passage.

\xbox{16}{
\ea \label{ex:func:time:deic:before:thaama:double}
\ea
\gll Se=ppe    baapa  dhaathangapa {\em government} {\em servant}. \\ % bf
      \textsc{1s=poss} father cop government servant \\
    `My father is/was a government servant.'
\ex
\gll Inni     {\em railway} {\em department}=ka    {\em head} {\em guard} hatthu \textbf{kàthaama}. \\
      \textsc{prox} railway department=\textsc{loc} head guard \textsc{indef} before \\
    `He was a head guard in the railway department before.'
\ex
\gll Itthu=nang      \textbf{kàthaama} incayang  {\em second} {\em world} {\em war}=ka    {\em CLI} Seelon {\em lightning} {\em infantry}     katha athu   {\em soldier}. \\
    \textsc{dist}=\textsc{dat} before \textsc{3s.polite} second world war=\textsc{loc} CLI Ceylon lightning infantry \textsc{quot} \textsc{indef} soldier   \\
    `Before that he was a soldier in the so-called CLI, the Ceylon lightning infantry, in the second world war.' (G051222nar01)
\z
\z
}

If time reference is inferable from discourse, non-verbal predicates need not carry overt indication of time reference. In example \xref{ex:time:before:nocoding}, the  preceding sentence  had established past time reference, and the nominal predication in \xref{ex:time:before:nocoding} need not be marked for time reference.

\xbox{16}{
\ea \label{ex:time:before:nocoding}
\gll Itthu bannyak laama hathu ruuma. \\ % bf
      \textsc{dist} very old \textsc{indef} house \\
    `That one was a very old house.'  (K070000wrt04)
\z
}


Lexical solutions are available for both verbal and non-verbal sentence types. Grammatical solutions, on the other hand, can only be employed with verbs (or converted adjectives).


Verbal predications with past reference are normally indicated by the prefixes \em anà- \em \formref{sec:morph:ana-} and \em su- \em \formref{sec:morph:su-}.

\xbox{16}{
\ea \label{ex:time:before:verb:ana}
\gll {\em 58}=ka=jo \textbf{anà-}mulain. \\
      58=\textsc{loc}=\textsc{emph} \textsc{past}-start \\
    `It started in `58.'  (B060115prs17)
\z
}

\xbox{16}{
\ea \label{ex:time:before:verb:su}
\gll Aanak pompang duuva=nang slaamath katha \textbf{su-}biilang. \\
     child girl two=\textsc{dat} goodbye \textsc{quot} \textsc{past}-say  \\
    `He said ``Goodbye'' to the two girls.'  (K070000wrt04)
\z
}


Location in time before the speech act can also be expressed by  the  perfect construction \formref{sec:wc:Theperfecttenses}. In distinction to the English present perfect, this does not imply relevance for the time of speaking. Example \xref{ex:time:before:perfect:perfect} shows the use of the perfect tense in a context relevant to the time of speaking, while  \xref{ex:time:before:perfect:past} gives an example of the perfect construction being used in a context without relevance to the speech situation.


\xbox{16}{
\ea \label{ex:time:before:perfect:perfect}
\gll See=le     \textbf{pii} \textbf{aada}  dhraapa=so duuva thiiga skalli. \\
     \textsc{1s}=\textsc{addit} go exist how.many=\textsc{undet} two three time  \\
    `I myself have been there, how many, maybe two or three times.' (B060115nar05)
\z
}

\xbox{16}{
\ea \label{ex:time:before:perfect:past}
\gll {\em Dutch} {\em period}=ka derang pada \textbf{dhaathang} \textbf{aada}. \\
Dutch period=\textsc{loc} \textsc{3pl} \textsc{pl} come exist\\
`They came during the Dutch period' (K051206nar05)
\z
}


Events  which are asserted to not have happened in the past take the negation marker \em thàrà- \em \xref{ex:time:before:past:neg} or postverbal \em thraa \em in the perfect \xref{ex:time:before:perfect:neg}.
As with the affirmative, this does not entail any relevance for the time of speaking, unlike English present perfect.
% \xref{ex:time:before:past.neg} refers to a time long ago and has no relevance for the speech act, while \xref{ex:time:before:past.neg:relevantpresent} refers to a point in time just before the speech act, and has a consequence for the speech act, namely that the story will be retold in Malay.

\xbox{16}{
\ea \label{ex:time:before:past:neg}
\gll S. {\em subscription} \textbf{thàrà}-kiiring \\
     S. subscription \textsc{neg.past}-send  \\
    `S. did not send the subscription.' (K060116nar10)
\z
}


\xbox{16}{
\ea \label{ex:time:before:perfect:neg}
\gll Kithang baaye mlaayu arà-oomong katha incayang \textbf{biilang} \textbf{thraa}. \\
      1lp good Malay \textsc{non.past}-speak \textsc{quot} \textsc{3s.polite} say \textsc{neg} \\
    `He has not said that we speak good Malay.'  (B060115prs15)
\z
}

%  \xbox{16}{
% \ea \label{ex:time:before:past.neg:relevantpresent}
% \gll Inni=yang see mlaayu=dering \textbf{thàrà}-biilang. \\
%       \textsc{prox}=\textsc{acc} \textsc{1s} Malay=\textsc{abl} \textsc{neg.past}=say \\
%     `I didn't tell you this in Malay.' (K061127nar03)
% \z
% }

Non-verbal predications do not show any special negative marking for the past. This is exemplified for an existential predication in \xref{ex:time:before:nonv:loc:neg}.


\xbox{16}{
\ea \label{ex:time:before:nonv:loc:neg}
\gll Oorang pada=nang   hathu  oorang=nang   creeveth \textbf{thraa}. \\
     man \textsc{pl}=\textsc{dat} \textsc{indef} man=\textsc{dat} trouble \textsc{neg}  \\
    `People had no trouble with each other (\em also possible: \em People have no trouble with each other).' (K051220nar01)
\z
}

Subordinate clauses which carry an indication of relative tense (simultaneous (\em arà-\em) or anterior \em asà-\em) cannot express absolute tense. They then inherit the tense meaning of the matrix clause. In example \xref{ex:time:before:simult1}, the verb \trs{kuthumung}{see}{} in the matrix clause is in the past tense, indicated by \em anà-\em. The verb \trs{ambel}{take}{} in the subordinate clause has the anterior prefix \em asà-\em, while \trs{maayeng}{play}{} in the subordinate clause has the prefix \em arà-\em, indicating the simultaneity of seeing and playing and the anteriority of taking with respect to seeing. Neither of the verbs in the subordinate clause is marked for past tense, but the past tense meaning is inherited from the matrix clause.

\xbox{16}{
\ea \label{ex:time:before:simult1}
\gll Blaakang=jo incayang \textbf{anà}-kuthumung [moonyeth pada thoppi \textbf{asà}-ambel pohong atthas=ka \textbf{arà-}maayeng]. \\
     after=\textsc{emph} \textsc{3s.polite} \textbf{past}-see monkey \textsc{pl} hat \textbf{anterior}-take tree top=\textsc{loc} \textbf{simult}-play  \\
    `Then only he saw that the monkeys had taken his hats and were playing on the top of the trees.'  (K070000wrt01)
\z
}


Another example is \xref{ex:time:before:simult2} where the act of hearing is simultaneous to the act of crying. Note that the conjunctive participle does not imply anteriority in this case, but rather coordination. Further note that we are dealing with past reference.

\xbox{16}{
\ea \label{ex:time:before:simult2}
\gll [Banthu-an asà-mintha \textbf{arà}-naangis] svaara hatthu derang=nang \textbf{su}-dìnngar. \\
      help-\textsc{nmlzr} \textsc{cp}-beg \textsc{simult}-cry sound \textsc{indef} \textsc{3pl}=\textsc{dat} \textsc{past}-hear\\
    `They heard a sound of crying and begging for help.'  (K070000wrt04)
\z
}


Durative or habitual actions in the past can also be coded by the progressive marker \em arà-\em. This overrides the expression of past tense. The following two examples show durative marking \xref{ex:time:before:override:durative} and habitual marking \xref{ex:time:before:override:habitual}.

\xbox{16}{
\ea \label{ex:time:before:override:durative}
\gll Itthu    kumpulan=dang      derang=jo     bannyak \textbf{arà}-banthu. \\
      \textsc{dist} association=\textsc{dat} \textsc{3pl}=\textsc{emph} much \textsc{non.past}-help\\
    `It is them who help that association a lot.'  (B060115cvs01)
\z
}

\xbox{16}{
\ea \label{ex:time:before:override:habitual}
\gll Itthu    kalu, [...] Dubai=ka    asà-duuduk     laama kar pada kitham  \textbf{arà}-baapi       Iraq  {\em ports}=nang  \\
 \textsc{dist} if [...] Dubai-\textsc{loc} \textsc{cp}-stay old car \textsc{pl} \textsc{1pl} \textsc{non.past}-take Iraq ports=\textsc{dat}\\
`In that case, we bring old cars from Dubai to the ports in Iraq.' (K051206nar19)
\z
}

The use of \em arà- \em in past contexts is clearly seen in \xref{ex:time:before:override:habitual:double}, where we have a non-deictic indication of time (1958), but the verb is not marked with a past tense prefix, but with \em arà-\em, which marks habitual in this case.

\xbox{16}{
\ea \label{ex:time:before:override:habitual:double}
\gll Muula     pàrthaama, Badulla  ruuma saakith=ka    \textbf{s-riibu}   \textbf{sbiilan} \textbf{raathus} \textbf{lima-pulu}    \textbf{dhlaapan=ka} pukurjan \textbf{arà}-gijja    vakthu. \\
     before first Badulla house sick=\textsc{loc} one-thousand nine hundred five-ty eight=\textsc{loc} work \textsc{simult}-make time \\
    `Before, when I was working in Badulla in 1958.' (K051213nar01)
\z
}


%
% \xbox{16}{
% \ea\label{ex:func:unreferenced}
% \ea\label{ex:func:unreferenced}
% \gll Kitham=pe      oorang thuva pada  bannyak dhaathang aada {\em Malaysia}=dring. \\
%      \textsc{1pl}=\textsc{poss} man old \textsc{pl} many come exist Malaysia=\textsc{abl}  \\
%       `Many of our ancestors came from Malaysia,'
% \ex
% \gll Spaaru indonesia=dring      dhaathang aada. \\
%      some Indonesia=\textsc{abl} come exist\\
%        `some came from Indonesia.' (K060108nar02)
% \z
% \z
% } \\

% Past time reference for non-verbal predications is normally not expressed. When necessary, the adverb \trs{thaama}{earlier} can be used. \em thaama \em must not be confused with \trs{thama-}{\textsc{neg.nonpast}} \formref{sec:form:thamau}.
%
%
% \xbox{16}{
% \ea\label{ex:func:unreferenced}
% \gll Seelon=ka {\em English} thaama aada. \\
%  Ceylon=\textsc{loc} English earlier BE\\
% `There is/will be no English in Sri Lanka.' (nosource)
% \z
% }



Lexical and grammatical indication of time can be combined, as in \xref{ex:time:before:lexgram:double}, where we find the lexical marker \em kàthaama \em and the grammatical marker \em nya-\em.

\xbox{16}{
\ea \label{ex:time:before:lexgram:double}
\gll See \textbf{kàthaama} pukurjan \textbf{nya}-kirja. \\
     \textsc{1s} earlier work \textsc{past}-do  \\
    `I used to work in former times.' (B060115prs01)
\z
}


\subsubsection{Simultaneous to speech act}\label{sec:func:Simultaneoustospeechact}
Temporal situation conceived of as simultaneous to the speech act can be expressed by a limited number of lexical means, or by the verb prefix \em arà- \em \formref{sec:morph:ara-} \citep[164]{SmithEtAl2006cll}.
Lexical solutions include the use of \trs{(s)kaarang}{now}{}, \trs{nyaari}{today} or \trs{ini X}{this X}, where X is a temporal noun like \trs{thaaun}{year} or \trs{muusing}{time, period}. This temporal noun should refer to a period which includes the time of speaking. The following sentences are examples of simultaneity to the speech act expressed lexically.


\xbox{16}{
\ea \label{ex:func:time:simult:skaarang}
\gll Suda \textbf{skaarang}    kitham=pe      aanak pada    laaeng   pukurjan pada    arà-girja. \\
      thus now \textsc{1pl}=\textsc{poss} child \textsc{pl} other work \textsc{pl} \textsc{non.past}-make \\
    `So our children are doing other jobs now.' (K051222nar05)
\z
}


\xbox{16}{
\ea \label{ex:func:time:simult:nyaari}
\gll Itthusubbath=jo    \textbf{nyaari}            go  laile  arà-duuduk. \\
     therefore=\textsc{emph} today \textsc{1s.familiar} again \textsc{non.past}-stay  \\
    `That's why I still stay (here).' (B060115nar04)
\z
}


\xbox{16}{
\ea \label{ex:func:time:simult:inimuusing}
\gll \textbf{Ini} \textbf{muusing}=dika oorang ikkang Hambanthota=ka arà-duuduk. \\
     \textsc{prox} time=\textsc{loc} man fish Hambantota=\textsc{loc} \textsc{non.past}-stay   \\
    `Presently, the fishermen are at Hambantota.' (K081106eli01)
\z
}



Events taking place in the time frame of the speech act are coded by \em arà- \em for verbal predicates \xref{ex:func:time:simult:ara} \formref{sec:morph:ara-}, and \zero-coded at all for other predicates \xref{ex:func:time:simult:zero}.


\xbox{16}{
\ea \label{ex:func:time:simult:ara}
\gll Suda itthu    oorang pada=le      Seelong=ka   \textbf{arà}-duuduk. \\
      thus \textsc{dist} man \textsc{pl}=\textsc{addit} Ceylon=\textsc{loc} \textsc{non.past}-stay \\
    `So these people also stay in Ceylon.' (K060108nar02)
\z
}

\xbox{16}{
\ea\label{ex:func:time:simult:zero}
\gll Se asdhaathang hatthu butthul {\em moderate} Muslim atthu. \\
 \textsc{1s} \textsc{copula} one very moderate Muslim one\\
`As for me, I am/was/will be a very moderate Muslim.' (K051206nar18)
\z
}

Simultaneity can be conceived in a wide sense \citep[164]{SmithEtAl2006cll}. In example \xref{ex:func:time:simult:wide} about a Chinese converted to Islam, it is not clear whether he is going to the mosque right now. Simultaneity to the time of speaking is conceived in a wider sense here, meaning that at the time of speaking the convert has the habit of going to the mosque, without vouching for his going at the present moment.



\xbox{16}{
\ea \label{ex:func:time:simult:wide}
\ea
\gll Ini   ciina oorang Islam=nang   asà-dhaathang \\ % bf
   \textsc{prox} China man Islam=\textsc{dat}  \textsc{cp}-come  \\
    `That Chinaman came to Islam and'
\ex
\gll ini  asà-kaaving=apa \\ % bf
 \textsc{prox} \textsc{cp}-marry=after      \\
    `married and'
\ex
\gll \textbf{karang} masiigith=nang  \textbf{arà}-pii  liima vakthu sbaayang=nang. \\
  now mosque=\textsc{dat} \textsc{non.past}-go five time pray=\textsc{dat}     \\
    `now he goes to the mosque five times (a day) to pray.' (K051220nar01)
\z
\z
}



% The time frame for simultaneity can be conceived wider or narrower. In example \xref{ex:dehrampadayang}, it is not sure whether at the time of speaking, someone actually send relatives abroad. But if the speech situation is conceived as encompassing a greater period, like a month or so, the event of speaking and sending abroad coincide.
%
%
% \xbox{16}{
% \ea\label{ex:func:unreferenced}
% \gll Deram pada=yang laayeng nigiri pada=nang arà-kiiring. \\
%  \textsc{3pl} \textsc{pl}=\textsc{acc} other country \textsc{pl}=\textsc{dat} \textsc{non.past}-send\\
% `They are sent abroad.' (nosource)
% \z
% }


Negated predicates with time reference simultaneous to the speech act are marked with \em thama- \em \xref{ex:func:time:simult:neg:thamau} \formref{sec:morph:thamau-} for verbs. All other predication types take their standard negation, which is the same regardless of time reference.



\xbox{16}{
\ea \label{ex:func:time:simult:neg:thamau}
\gll Kitham=pe      aanak pada \textbf{thama}-oomong. \\ % bf
 \textsc{1pl}=\textsc{poss} child \textsc{pl} \textsc{neg.nonpast}-speak \\
`Our children do not speak.' (G051222nar01)
\z
}

\subsubsection{After speech act}\label{sec:func:Afterspeechact}
Temporal reference to a point in time after the speech act can be made with some lexical means, and grammatical means for verbal predications. Lexical solutions include \trs{beeso}{tomorrow}{} or \trs{(beeso) luusa}{after tomorrow}.

\xbox{16}{
\ea \label{ex:time:after:beeso}
\gll Ruuma birsi=nang arà-simpang. \textbf{Beeso}=nang kithang arà-mnaaji \\
      house clean=\textsc{dat} \textsc{non.past}-keep. Tomorrow=\textsc{dat} \textsc{1pl} \textsc{non.past}-pray \\
    `We keep the house clean. Tomorrow, we will pray.' (K061019prs01)
\z
}

\xbox{16}{
\ea \label{ex:time:after:beesoluusa}
\gll \textbf{Beeso} \textbf{luusa} lubaarang arà-dhaathang. \\
     tomorrow later.in.the.future festival \textsc{non.past}-come  \\
    `The day after tomorrow is the festival.' (K061019prs01)
\z
}


Additionally, \trs{duppang}{future} can be used to indicate future temporal reference to after the speech act. Note that the relator noun \trs{duppang}{before}, constructed with \em =nang\em, is used for past reference, while the full noun \trs{duppang}{future}  is used for future reference. This is similar to German, where the preposition \trs{vor}{ago} is used to indicate   temporal distance in the past (\trs{vor neun Jahren}{nine years ago}), while the future is said to lie before us \em vor uns \em as well, whereas the past lies behind (\em hinter\em).


\xbox{16}{
\ea \label{ex:time:after:duppang:contrast}
\gll Kithang=\textbf{nang} \textbf{duppang}$_{\textsc{reln}}$ lai duuva bàrgaada asà-dhaathang aada. \\
 \textsc{1pl}=\textsc{dat} before other two family \textsc{non.past}-come exist \\
`Before us, there were two other families.' (K060108nar02)
\z
}


\xbox{16}{
\ea
\gll \textbf{Duppang} \textbf{muusing}$_{fullnoun}$=ka=le Dodangwela aapacara=le thama-bìssar. \\
     before time=\textsc{loc}=\textsc{addit} Dodangwela how=\textsc{addit} \textsc{neg.irr}-big  \\
    `Even in the future, Dodangwela [a village close to Kandy]  will not be big.' (K081106eli01)
\z
}

\xbox{16}{
\ea
\gll \textbf{Duppang} \textbf{muusing}$_{fullnoun}$=ka oorang ikkang {\em Negombo}=nang anthi-pii. \\
      future time=\textsc{loc} man fish Negombo=\textsc{dat} \textsc{irr}-go \\
    `In the future, the fishermen will go to Negombo.' (K081106eli01)
\z
}

As for grammatical solutions, verbal predications which are thought to take place after the speech act are either marked with \em arà- \em  \formref{sec:morph:ara-} \xref{time:after:ara} or \em anthi- \em \formref{sec:morph:anthi-} \xref{time:after:anthi}.


% \xbox{16}{
% \ea\label{ex:func:unreferenced}
% \gll Laskalli arà-maakang. \\
%  other.time \textsc{non.past}-eat\\
% `You will come and eat another time.' (B060115cvs16)
% \z
% }

\xbox{16}{
\ea \label{time:after:ara}
\gll Paanas muusing dhaathang=thingka see siini=dering \textbf{arà}-pii. \\
      hot season come=middle \textsc{1s} here=\textsc{abl} \textsc{non.past}-go\\
    `When the hot season  comes, I will leave from here.'  (K070000wrt04)
\z
}

\xbox{16}{
\ea \label{time:after:anthi}
\gll Ithu=kapang lorang=pe leher=yang kithang \textbf{athi}-poothong. \\
     \textsc{dist}=when \textsc{2pl}=\textsc{poss} neck=\textsc{acc} \textsc{1pl} \textsc{irr}-cut  \\
    `Then we will cut your neck.' (K051213nar06)
\z
}

Non-verbal predicates technically do not have to be marked for future reference. However, asserting the future truth of a proposition often implies pragmatically that it is not true at the time of speaking, so that a construction conveying this change of state from false to true is preferred. This can be done by using adjectives in a verbal predication \xref{ex:func:time:after:adj}, or by using a construction involving \trs{jaadi}{become} \xref{ex:func:time:after:jaadi} \formref{sec:wc:Specialconstructionsinvolvingverbalpredicates}. \em Jaadi \em will bear the irrealis marker \em anthi- \em then.

\xbox{16}{
\ea \label{ex:func:time:after:adj}
\gll Ithukapang gaathal \textbf{anthi}-kuurang. \\
	 then itching \textsc{irr}-less\\
    `Then the itching will become less.'  (K060103cvs02)
\z
}


\xbox{16}{
\ea \label{ex:func:time:after:jaadi}
\gll Incayang hatthu guru athi/*arà-jaadi. \\
      \textsc{3s.polite} \textsc{indef} teacher \textsc{irr}/\textsc{non.past}-become \\
    `He will become a teacher.' (K081106eli01)
\z
}

Considering negation for propositions referring to a point in time after the speech act, \em thama- \em is employed for verbs \xref{ex:func:time:after:thamau:v} and adjectives \xref{ex:func:time:after:thamau:adj}, while other predicates will use a periphrasis to indicate that the future state will not come into being \xref{ex:func:time:after:thamajaadi}. In the rare case that the speaker wants to assert that a state is not true at the time of speaking and will not be true in the future either, this can be done by a normal non-verbal predication, but lexical material is required to make the time reference clear \xref{ex:time:after:nonpresnonfut}.

\xbox{16}{
\ea \label{ex:func:time:after:thamau:v}
\gll See lorang=nang   \textbf{thama-}sakith-kang. \\
      \textsc{1s} \textsc{2pl}=\textsc{dat} \textsc{neg.nonpast}-sick-\textsc{caus} \\
    `I will not hurt you.' (K070000wrt04)
\z
}

\xbox{14}{
\ea \label{ex:func:time:after:thamau:adj}
\gll Inni pukuran=yang mà-gijja \textbf{thamau}-gampang \\
     \textsc{prox} work=\textsc{acc} \textsc{inf}-make \textsc{neg.irr}-easy  \\
    `To do that kind of work will never be(come) easy.' (K081106eli01)
\z
}


\xbox{16}{
\ea \label{ex:func:time:after:thamajaadi}
\gll Se hatthu guru \textbf{thama}-jaadi. \\
      \textsc{1s} \textsc{indef} teacher \textsc{neg.irr}-become \\
    `I will not become a teacher.' (K081106eli01)
\z
}


\xbox{16}{
\ea \label{ex:time:after:nonpresnonfut}
\gll See \textbf{innam} \textbf{blaakang} hatthu aanak \textbf{bukang}. \\
     \textsc{1s} \textsc{prox.dat} after \textsc{indef} child \textsc{neg.nonv} \\
    `I will never be a child again.' (K081106eli01)
\z
}


% \xbox{16}{
% \ea
% \gll Se hatthu guru ma jaadi thàrà suuka. \\
%        \\
%     `.' (nosource)
% \z
% } \\


\subsubsection{General truth}\label{sec:func:Generaltruth}
General truth independent of time frame is expressed with the progressive marker \em arà- \em  \formref{sec:morph:ara-} (\em thama- \em in the negative) for verbal predicates and is not coded overtly for non-verbal predicates.

% \xbox{16}{
% \ea \label{ex:CeylonAirportyang}
% \gll Seelong {\em Airport}=yang duva-pulu-ùmpath vakthu=le asà-bukka arà-simpang kiyang. \\
%      Ceylon Airport=\textsc{acc} two-ty-four hour=\textsc{addit}  \textsc{cp}-open \textsc{non.past}-stay \textsc{evid}\\
%     `The Ceylon Airport will stay open 24h, it seems. (lit: They will open it and keep it [like that])'  (Letter 26.2007)
% \z
% }\\
%
In example \xref{ex:func:time:general:tony}, the speaker's name Tony is true irrespective of the time of speaking. In example \xref{ex:func:time:general:sunnath}, the circumcision at the fortieth day is a general truth as well which is true about circumcisions in the past, present and future alike. Both are coded by \em arà-\em.

\xbox{16}{
\ea \label{ex:func:time:general:tony}
\gll See=yang    Tony katha \textbf{arà}-panggel. \\
1s=\textsc{acc}  Tony \textsc{quot} \textsc{non.past}-call\\
`I am called ``Tony''.' (K060108nar01)
\z
}

\xbox{16}{
\ea \label{ex:func:time:general:sunnath}
\gll Karam=pe mosthor=nang, mpapulu aari=ka=jo sunnath=le \textbf{arà}-kijja. \\
 now=\textsc{poss} manner=\textsc{dat} forty   day=\textsc{loc}=\textsc{emph} circumcision=\textsc{addit} \textsc{non.past}-make  \\
    `For today's way of doing (it), it is on the fortieth day that they also do the circumcision.'  (K061122nar01)
\z
}

General truth can sometimes be difficult to distinguish from habitual. In this grammar, I distinguish general truth, coded by \em arà-\em, from habitual, which can be coded by \em arà- \em or \em anthi-\em. See Section \ref{sec:func:asp:Habitual} for a discussion of habitual events.

%
% \xbox{16}{
% \ea\label{ex:func:unreferenced}
% \gll Hatthu {\em chess}  {\em {\em team}}=nangìnnam arà-duuduk. \\
%  one Chess team=\textsc{DAT} six \textsc{non.past}-stay\\
% `There are six (players) in one chess team.' (B060115prs20)
% \z
% }
%
% Also chess teams consist of six players, regardless of time reference.
%
% pompang pada  derang muuka arà-thuuthup
%
% Definitions are also coded by \em arà- \em as a general truth. This is the case in the following example.
%
%
% \xbox{16}{
% \ea\label{ex:func:unreferenced}
% \gll {\em Majority} katha arà-biilang    hathu  liivath blaangan {\em votes}=dering   su-bunnang. \\
%      majority \textsc{quot} \textsc{non.past}-say \textsc{indef} more amount votes=\textsc{abl} \textsc{past}-win \\
%     `Majority means that (he) won a bigger amount of the votes.' (K051222nar06)
% \z
% } \\

The normal negation of \em arà- \em is \em thama-\em. This prefix is also used to negate propositions with general time reference. Example \xref{ex:func:time:general:neg} shows general truth of a negative predication: Hindus generally do no eat beef, which again is independent of time reference to present past or future.  This is coded by \em thuma-\em, an allomorph of \em thama-\em.

\xbox{16}{
\ea \label{ex:func:time:general:neg}
	\ea
	 \gll Hindu \textbf{arà}-maakang kambing. \\
	Hindu \textsc{non.past}-eat goat\\
	`Hindus eat goat.'
	\ex
	\gll samping \textbf{thuma}-maakang \\
		beef \textsc{neg.nonpast}-eat\\
	`(they) don't eat beef.' (K060112nar01)
	\z
\z
}


General truth of non-verbal predications does not receive special marking. In example \xref{ex:func:time:general:nonv} the general truth of the speaker's profession being engineering is conveyed by the copula, which can also be used with other time references.


\xbox{16}{
\ea \label{ex:func:time:general:nonv}
\gll Se=ppe percariyan asdhaathang {\em engineering}. \\
      \textsc{1s=poss} earning \textsc{copula} engineering \\
    `My profession is engineer.' (K061026prs01)
\z
}

\subsection{Figure and ground in the temporal domain}\label{sec:func:Figureandgroundinthetemporaldomain}
Just like two entities can be related to each other in space, this is also possible in time. Depending on start point, end point, duration and overlap of the two events, a certain number of semantic constellations are possible.\footnote{This list is inspired by \citet[330]{Givon2001b}.}  In SLM, all of them can be expressed lexically, and some can also be expressed grammatically. Very often, relator nouns are employed to indicate the precise relationship.

Any point in time can be referred to anaphorically by the distal deictic \em itthu\em. The proximal deictic \em ini, \em  on the other hand, specifies the time of the speech act as temporal grounding. The precise kind of relation is then indicated by a relator noun like \trs{blaakang}{after}, with  dative marker \em =nang \em added to \em ini\em. In the process \em ini=nang \em is contracted to \em innam \em (not to be confounded with \trs{ìnnam}{six}).

\xbox{16}{
\ea
\gll See innam blaakang hatthu aanak bukang. \\
     \textsc{1s} \textsc{prox.dat} after \textsc{indef} child \textsc{neg.nonv} \\
    `I will never (=from now onwards) be a child again.' (K081106eli01)
\z
}



\subsubsection{Precedence}\label{sec:func:Precedence}
Precedence obtains when the time frame of the figure precedes the time frame of the ground.

\ea {\Large $\stackrel{fig}{\multimap}~~\stackrel{gr}{\multimapdotinv}$}\z

In SLM, it is coded by \trs{duppang}{before} or \trs{(kà)thaama}{before}.


\em Duppang \em is used as an adverb in this function. The ground is coded with the dative marker \em =nang\em, while the figure does not receive special marking. The ground can be expressed by a noun, as in \xref{time:precedence:duppang:adv:n}, or by a verb in the infinitive, as in \xref{time:precedence:duppang:adv:v}


\xbox{16}{
\ea \label{time:precedence:duppang:adv:n}
\gll Itthu    blaakang=jo,    [\textbf{kitham=pe} \textbf{{\em AGM}}]$_{ground}$=\textbf{nang}  duppang,  [{\em high} {\em commissioner} {\em cultural} {\em show}=na mà-dhaathang=nang       thàràboole s-jaadi]$_{figure}$. \\
dist after=\textsc{emph} \textsc{1pl}=\textsc{poss} AGM=\textsc{dat} before high commissioner cultural show=\textsc{dat} \textsc{inf}-come-\textsc{dat} cannot \textsc{past}-become\\
    After that, before our Annual General Meeting, it became impossible for the High Commissioner to attend the cultural show.    (K060116nar23)
\z
}


\xbox{16}{
\ea \label{time:precedence:duppang:adv:v}
\gll Itthule [\textbf{see}=\textbf{yang} \textbf{mà}-\textbf{kiiring}=\textbf{nang}]$_{ground}$ duppang [incayang see=yang hathu Buruan mà-jaadi su-bale-king]$_{figure}$. \\ % bf
     but \textsc{1s}=\textsc{acc} \textsc{inf}-send=\textsc{dat} before \textsc{3s.polite} \textsc{1s}=\textsc{acc} \textsc{indef} bear \textsc{inf}-become \textsc{past}-turn-\textsc{caus}  \\
    `But before he sent me back, he turned me into a bear.' (K070000wrt04)
\z
}

If the ground is not given, the time of the speech situation is taken as a default. In that case, the amount of time separating the event from the speech act can be indicated by \em =nang\em, as in \xref{time:precedence:duppang:adv:speechsituation}.


\xbox{16}{
\ea \label{time:precedence:duppang:adv:speechsituation}
\gll \zero$_{ground}$ \textbf{Sdiikith} \textbf{thaaun}=\textbf{nang} duppang, [see ini Aajuth=nang su-kìnna daapath]$_{figure}$. \\ % bf
     { }  few year=\textsc{dat} before \textsc{1s} \textsc{prox} dwarf=\textsc{dat} \textsc{past}-patfoc get \\
    `Some years ago, I fell prey to this dwarf.' (K070000wrt04)
\z
}


The situation is similar with \em (kà)thaama\em. Here as well, the ground is marked by \em =nang. \em The following example also shows that  a deictic can serve as ground.


\xbox{16}{
\ea \label{time:precedence:kàthaama}
\gll [Inni]$_{ground}$=nang       kàthaama [see Navalapitiya=ka    duuduk aada]$_{figure}$. \\ % bf
      \textsc{prox}=\textsc{dat} before \textsc{1s} Nawalapitiya=\textsc{loc} stay exist \\
    `Before that, I was in Nawalapitiya.' (G051222nar01)
\z
}


\subsubsection{Subsequence}\label{sec:func:Subsequence}
Subsequence obtains when the time frame of the ground precedes the time frame of the figure.


\ea {\Large $\stackrel{gr}{\multimapdot}~~\stackrel{fig}{\multimapinv}$}\z

Subsequence of events need not be coded but can be inferred by the succession of utterances, which is iconic.
Subsequence can be coded lexically by \trs{itthuka(apa)ng}{and then} \formref{sec:wofo:itthuka(apa)ng} or \trs{blaakang}{after} \formref{sec:wc:blaakang}. As for grammatical means, the conjunctive participle \em asà- \em \formref{sec:morph:asa-} very often indicates subsequence.


The non-coding of succeeding events is exemplified by \xref{ex:time:subseq:non}, where the getting up is followed by the search, but no grammatical or lexical element indicates the temporal relation between the two events. This is then inferred to be subsequence as the default case.

\xbox{16}{
\ea \label{ex:time:subseq:non}
\ea
\gll Oorang su-baavung \\ % bf
      man \textsc{past}-get.up \\
    `The man got up'
\ex
\gll \zero{} thoppi pada yang anà-caari. \\ % bf
      { } hat \textsc{pl} \textsc{pat} \textsc{past}-search \\
    `(and) looked for the hats.'  (K070000wrt01)
\z
\z
}

A lexical solution for subsequence is \trs{blaakang}{after}. As with precedence, the ground is indicated by \em =nang\em.

\xbox{16}{
\ea \label{ex:time:subseq:blaakang}
\gll   [Itthu]$_{ground}$=nam \textbf{blaakang} [bannyak oorang pada siini se-duuduk]$_{figure}$. \\ % bf
       \textsc{dist}=\textsc{dat} after many man \textsc{pl} here \textsc{past}-stay  \\
     `After that, many people settled down here.' (G051222nar03)
\z
}


Just like \em duppang\em, \em blaakang \em can also be used without overt grounding, in which case it just emphasizes that the event is subsequent to whatever was there before.

\xbox{16}{
\ea \label{ex:time:subseq:blaakang:noground}
\gll [\zero]$_{ground}$  Blaakang, [incayang=nang    baaye=nang   nanthok su-pii]$_{figure}$. \\  % bf
     { }  after \textsc{3s.polite}=\textsc{dat} good=\textsc{dat} sleepy \textsc{past}-go  \\
    `Then the man became very sleepy.' (K070000wrt01)
\z
}


%
% \xbox{16}{
% \ea\label{ex:func:unreferenced}
% \gll Dr Draaman \textbf{duuva} \textbf{thaaun} \textbf{blaakang} incayang se-nnii\u n\u ggal. \\
%  Dr Draaman two year after \textsc{3s.polite} \textsc{past}-die\\
% `Dr Draman died two years later.' (nosource)
% \z
% }


The combination of a deictic and a postposition \trs{ithukapang}{then}{}  and the particle \trs{suda}{then} are rather discourse markers \formref{sec:wofo:Deictic+X}, but also convey a meaning of subsequence.


\xbox{16}{
\ea \label{ex:time:subseq:ithukapang}
\gll \textbf{Ithu=kapang}       umma-baapa  su-biilang     lorang=nang   kaaving thàrboole. \\
      \textsc{dist}=then mother-father \textsc{past}-say 2\textsc{pl}=\textsc{dat} marry cannot \\
    `Then the parents said, you cannot marry.' (K051220nar01)
\z
}

 \xbox{16}{
\ea \label{ex:time:subseq:suda}
   \gll \textbf{Suda} derang=nang   hathyang muusing=sangke mà-duuduk  su-jaadi. \\
    then \textsc{3pl}=\textsc{dat} other time=until \textsc{inf}-stay \textsc{past}-become \\
`So they had to wait until another time' (K051220nar01)
\z
}


There exist three grammaticalized means to express subsequence, the conjunctive participle \em asà- \em \formref{sec:morph:asa-} \citep[cf.][]{Slomanson2008lingua}, the postposition \em =apa \em \formref{sec:morph:=apa} and the (plu)perfect construction \formref{sec:wc:Theperfecttenses}. \em Asà- \em and \em =apa \em are frequently combined, but this is not necessary. The following examples show the exclusive use of \em asà- \em \xref{ex:time:subseq:s} and  \em =apa \em \xref{ex:time:subseq:apa}, and the combination thereof \xref{ex:time:subseq:asa+apa}.


\xbox{16}{
\ea \label{ex:time:subseq:s}
\ea
\gll Pìrrang=nang anà-dhaathang oorang pada \textbf{asà}-pìrrang. \\
 war=\textsc{dat} \textsc{past}-come man \textsc{pl} \textsc{cp}-wage.war\\
`The people who had come to the war battled (and)'

\ex
\gll Derang=nang \textbf{asà}-banthu. \\
\textsc{3pl}=\textsc{dat} \textsc{cp}-help\\
`helped him [the king] (and)'

\ex
\gll Siini=jo se=cii\u n\u ggal. \\ % bf
here=\textsc{emph} \textsc{past}-settled\\
`settled down (here).' (K051222nar03)
\z
\z
}


\xbox{16}{
\ea \label{ex:time:subseq:apa}
\ea
\gll Mlaayu pada  duuduk=\textbf{apa}. \\
 Malay \textsc{pl} stay=after\\
 `After the Malays had settled down'
\ex
 \gll Spaaru  mlaayu pada   Singapur       Indonesia  {\em Malaysia} anà-pii. \\ % bf
 some Malay \textsc{pl} Singapore Indonesia Malaysia \textsc{past}-go\\
`(only) some Malays went (back) to Singapore, Indonesia or Malaysia.' (K051213nar07)
\z
\z
}

\xbox{16}{
\ea \label{ex:time:subseq:asa+apa}
\ea
\gll Thàrà-kalu [ini oorang thoppi arà-kumpul]=yang \textbf{asà}-kuthumung=\textbf{apa}, \\
       \textsc{neg}-if \textsc{prox} man hat \textsc{simult}-collect=\textsc{acc} \textsc{cp}-see=after\\
    `Furthermore, when (they) had seen the man collect the hats'
\ex
\gll moonyeth pada=le asà-dhaathang creeveth  athi-kaasi katha. \\ % bf
       monkey \textsc{pl}=\textsc{addit} \textsc{cp}-come trouble \textsc{irr}=give  \textsc{quot}\\
    `the monkeys would certainly go and cause (some other) trouble.'  (K070000wrt01)
\z
\z
}

The pluperfect with \em asà- \em is given in \xref{ex:time:subseq:asa+aada:pluperf1}-\xref{ex:time:subseq:asa+aada:pluperf2}. The theoretical possibility of forming the pluperfect with \em =apa \em was not found in the corpus.


\xbox{14}{
\ea\label{ex:time:subseq:asa+aada:pluperf1}
\gll Incayang mliiga=nang kapang-pii, Raaja hathu thiikar=ka guula \textbf{asà}-siibar mà-kìrring simpang \textbf{su}-aada.  \\
     \textsc{3s.polite} palace=\textsc{dat} when-go king \textsc{indef} man=\textsc{loc} sugar \textsc{cp}-spread \textsc{inf}-dry keep \textsc{past}-exist  \\
    `When he was going to the palace, the King had sprinkled sugar in a mat and had left it to dry.'
\z
}

\xbox{14}{
\ea \label{ex:time:subseq:asa+aada:pluperf2}
\gll Itthu bannyak laama hathu ruuma, itthule ruuma duuva subla=ka panthas rooja kumbang pohong komplok duuva \textbf{asà}-jaadi \textbf{su}-aada.  \\
      \textsc{dist} very old \textsc{indef} house, but house two side=\textsc{loc} beautiful rose flower tree bush two \textsc{cp}-become \textsc{past}-exist \\
    `That was a very old house, but still on the two sides of the house, there had grown two beautiful rose bushes.'
\z
}


Next to the pluperfect, the normal perfect can also be used to refer to the past of the past. The marking of past tense on the existential \em aada \em is optional, as the following two examples show.

\xbox{16}{
\ea \label{ex:time:subseq:asa+aada:perf1}
\gll Kanabisan=ka=jo duva oorang=le anà-thaau ambel [Andare duva oorang=yang=le \textbf{asà}-enco-kang \zero{}-\textbf{aada} katha]. \\
       last=\textsc{loc}=\textsc{emph} two man=\textsc{addit} \textsc{past}-know take Andare two man=\textsc{acc}=\textsc{addit} \textsc{cp}-fool-\textsc{caus} exist  \textsc{quot}\\
    `Finally the two women understood that Andare had fooled both of them.' (K070000wrt05a)
\z
}



\xbox{16}{
\ea \label{ex:time:subseq:asa+aada:perf2}
\ea
\gll Derang pada kathahan thama-thuukar. \\
      \textsc{3pl} \textsc{pl} word \textsc{neg.nonpast}-change \\
    `They would not change their word.'
\ex
\gll Karang  {\em British} {\em government}=nang  derang kathahan \textbf{asà}-kaasi \zero-\textbf{aada}. \\
      now British Government=\textsc{dat} \textsc{3pl} word \textsc{cp}-give exist \\
    `Now, they had given their word to the British government.' (K051213nar06)
\z
\z
}


%
% \xbox{16}{
% \ea\label{ex:func:unreferenced}
% \ea\label{ex:func:unreferenced}figure
% \gll Nyaakith  oorang pada \textbf{s-}pii. \\
%  sick man \textsc{pl} \textsc{cp}-go\\
% `The patients go (and)'
% \ex
% \gll Thaangan arà-cuuci. \\
% hand \textsc{non.past}-wash\\
% `wash their hands.' (nosource)
% \z
% \z
% }

% \xbox{16}{
% \ea\label{ex:func:unreferenced}
% \ea\label{ex:func:unreferenced}
% \gll Asà-girijja incayang, su-\textbf{{\em pass}}$_i$. \\ % bf
%  \textsc{cp}-do 3\textsc{s.polite} \textsc{past}-pass\\
% `Having done that, he passed (the exam).'
% \ex
% \gll Asà-\textbf{{\em pass}}$_i$, kaarang,  {\em Advanced}  {\em Level} arà-\textbf{\em girijja}$_j$ \textbf{skaarang}. \\ % bf
% cp-pass now {} {}  \textsc{non.past}-make now\\
% `Having passed the exam, then, he will do  the Advance Level then.'
% \ex
% \gll \textsc{Karang}, {\em Advanced} {\em Level} asà-\textbf{girijja}$_j$, {\em 1978}=ka. \\ % bf
% now Advanced Level \textsc{cp}-make, 1978=loc\\
% `Then, having done the Advanced Level exams, for 1978ers,'
% \ex
% \gll {\em Late} {\em Exam} anthi-girijja. \\ % bf
% late exam \textsc{irr}-make\\
% `he will do the  late exam.' (K051222nar08)
% \z
% \z
% }

The following passage of a narrative shows the use of four different strategies to encode subsequence: zero coding, \em blaakang, ithukapang \em and \em asà-\em.

\xbox{16}{
\ea \label{ex:time:subseq:quadruple}
\ea
\gll Incayang=nang baaye=nang maara su-pii. \\ % bf
      \textsc{3s.polite}=\textsc{dat} good=\textsc{dat} anger \textsc{past}-go \\
    `He got very angry.'
\ex
\gll \textbf{\zero} Incayang=pe kàpaala=ka anà-aada thoppi=dering moonyeth pada=nang su-buvang puukul. \\
     { } \textsc{3s.polite}=\textsc{poss} head=\textsc{loc} \textsc{past}-exist hat=\textsc{abl} monkey \textsc{pl}=\textsc{dat} \textsc{past}-throw hit \\
    `He took the hat from his head and violently threw it  at the monkeys.'  (K070000wrt01)
\ex
\gll \textbf{Ithu=kapang} ithu moonyeth pada=le anà-maayeng duuduk thoppi pada=dering inni oorang=nang su-bale-king puukul. \\
\textsc{dist}=then \textsc{dist} monkey \textsc{pl}=\textsc{addit} \textsc{past}-play exist.\textsc{anim} hat \textsc{pl}=\textsc{abl} \textsc{prox} man=\textsc{dat} \textsc{past}-return-\textsc{caus} hit\\
`Then, the monkeys also threw back to the man the (other) hats with which they had been playing.'
\ex
\gll Itthu=nang \textbf{blaakang} inni oorang lìkkas\~{}lìkkas thoppi pada=yang \textbf{asà-}kumpul ambel sithu=ka=dering su-pii. \\
dist after \textsc{prox} man fast\~{}\textsc{red} hat \textsc{pl}=\textsc{acc} \textsc{cp}-collect take there=\textsc{loc}=\textsc{abl} \textsc{past}-go\\
`After that the man quickly picked up his hats and left that place.'
\z
\z
}


% \xbox{16}{
% \ea\label{ex:func:unreferenced}
% \gll See   {\em leave} asà-abbis ambel=apa     see nya-pii    {\em Middle} {\em East}. \\
%       \textsc{1s} leave \textsc{cp}-finish take=after \textsc{1s} \textsc{past}-go Middle East \\
%     `After having quit, I went to the Middle East.' (K061026prs01)
% \z
% } \\


% \xbox{16}{
% \ea\label{ex:func:unreferenced}
% \gll Incayang  mnii\u n\u ggal blaakang Zahir  Lahiyang anà-thaaro. \\
%       \textsc{3s.polite} die after Zahir Lahiyang \textsc{past}-put \\
%     `After he had died, they put Zahir Lahiyang).' (K051213nar08)
% \z
% } \\


\subsubsection{Simultaneity}\label{sec:func:Simultaneity}
Simultaneity obtains when two states-of-affairs are true at the same point in time.

\ea {\Large $\begin{array}{c}\stackrel{fig}{----}\\\stackrel{gr}{----}\end{array}$}\z

Simultaneity is lexically coded by \trs{vatthu}{time} or \trs{=kapang}{when} \formref{sec:wc:kaapang}. Both\footnote{\trs{Vatthu}{time} is
  seen as a `real' noun in this grammar, although it shows some behaviour typical of relator nouns. It appears that \em vatthu \em is acquiring the function of relator noun through grammaticalization. See \citet{DeLancey1997relator} for a discussion of this path.}
are used as a postposition on an NP, which can be attached on a noun \xref{ex:time:simult:n}, a deictic \xref{ex:time:simult:deic} or a clause
%\xref{ex:time:simult:cl1}
\xref{ex:time:simult:cl2}.

\xbox{16}{
\ea \label{ex:time:simult:n}
\gll [{\em World} {\em war}]=\textbf{vatthu}. \\
     world war=time \\
    `During the world war.' (K051206nar07)
\z
}

\xbox{16}{
\ea \label{ex:time:simult:deic}
\gll [Itthu]=\textbf{vatthu} {\em Malaysia}=ka anà-duuduk    {\em Military}  {\em Regiment}. \\
      \textsc{dist}=time Malaysia=\textsc{loc} \textsc{past}-stay Military Regiment \\
    `During that time there was a Military Regiment from Malaysia.' (K060108nar02)
\z
}

% \xbox{16}{
% \ea \label{ex:time:simult:cl1}
% \gll Suda [kitham pada  arà-{\em escort}    dhaathang]=\textbf{vatthu}. \\
%       so, \textsc{1pl} \textsc{pl} \textsc{simult}-escort come=time\\
%     `While we were escorting.' (K051206nar16)
% \z
% } \\

\xbox{16}{
\ea \label{ex:time:simult:cl2}
\gll [Kiccil]=\textbf{kapang} kithang sudaara pada samma {\em cricket} arà-maayeng. \\
     small=when \textsc{1pl} siblings \textsc{pl} all cricket \textsc{non.past}-play  \\
    `When we were small, us children used to play cricket.'  (K051201nar02)
\z
}


In subordinate clauses, the prefix \em arà- \em codes simultaneity. In \xref{ex:time:simultaneity:ara}, the acts of seeing and playing are simultaneous, which is encoded by \em arà- \em on the verb \trs{maayeng}{play}. Taking (\em ambel\em) precedes the playing, but can be construed as simultaneous to seeing, or as preceding seeing.

\xbox{16}{
\ea \label{ex:time:simultaneity:ara}
\gll Blaakang=jo incayang \textbf{anà}-kuthumung [moonyeth pada thoppi \textbf{asà}-ambel pohong atthas=ka \textbf{arà}-maayeng]. \\
     after=\textsc{emph} \textsc{3s.polite} \textsc{past}-see monkey \textsc{pl} hat \textsc{anterior}-take tree top=\textsc{loc} \textsc{simult}-play  \\
    `Then only he saw that the monkeys had taken his hats and were playing on the top of the trees.'  (K070000wrt01)
\z
}

Furthermore, simultaneity of two events can be expressed by putting one of the events in a reduplication construction \xref{ex:time:simult:redup:intro} \formref{sec:wofo:Verbalreduplication}.



\xbox{16}{
\ea \label{ex:time:simult:redup:intro}
\gll Kithang nyaanyi\~{}nyaanyi su-thaa\u ndak. \\
     \textsc{1pl} sing\~{}\textsc{red} \textsc{past}-dance  \\
    `We danced while singing/we sang and danced.' (K081114eli01)
\z
}

A naturalistic example of this construction being used for simultaneity is \xref{ex:time:simult:redup:naturalistic}.


\xbox{16}{
\ea \label{ex:time:simult:redup:naturalistic}
\gll [Lu=ppe muuluth=ka=le paasir, se=ppe muuluth=ka=le paasir  katha \textbf{biilang\~{}biilang}] baaye=nang baapa=le aanak=le guula su-maakang. \\
      \textsc{2s.familiar}=\textsc{poss} mouth=\textsc{loc}=\textsc{addit} sand \textsc{1s=poss} mouth=\textsc{loc}=\textsc{addit} sand \textsc{quot} say\~{}red good=\textsc{dat} father=\textsc{addit} child=\textsc{addit} sugar \textsc{past}-eat  \\
    `Saying ``There is sand in your mouth and there is sand in my mouth'' both father and son ate the sugar.'  (K070000wrt02)
\z
}

\subsubsection{Point coincidence}\label{sec:func:Pointcoincidence}
Point coincidence obtains when the end point of the first event coincides with the beginning of the second event.

\ea {\Large $\stackrel{gr~~}{\multimapdot}\hspace{-.18cm}\stackrel{fig}{\multimapinv}$} \z

It is expressed  by the verbal prefix \em kam-/ka-/kapang- \em \formref{sec:morph:kapang-}. In example \xref{ex:time:pc:kapang1}, the  lifting of the bamboo commences exactly at that point in time when the steam comes out.

\xbox{16}{
\ea \label{ex:time:pc:kapang1}
\gll Aavi luvar=nang \textbf{kapang}-dhaathang,  itthu bambu=yang giini angkath=apa  pullang arà-thoolak. \\
      steam outside=\textsc{dat} when-come \textsc{dist} bamboo=\textsc{acc} like.this lift=after slow \textsc{non.past}-push \\
    `When the steam comes out, lift the bamboo like this and push it slowly.' (K061026rcp04)
\z
}


In example \xref{ex:time:pc:kapang2}, the habit of gathering is lost as soon as the people take up work.


\xbox{16}{
\ea \label{ex:time:pc:kapang2}
\ea
\gll Inni     pukurjan=nang  \textbf{kam}-pii, \\
      \textsc{prox} work=\textsc{dat} when-go \\
    `When they go to this work,'
\ex
\gll deram pada               itthu mà-kumpul     hatthu mosthor thraa. \\
3\textsc{pl} pl  \textsc{dist} \textsc{inf}-add \textsc{indef} habit \textsc{neg}\\
`they lack the habit of gathering' (G051222nar01)
\z
\z
}


% \xbox{16}{
% \ea \label{ex:time:pc:kam1}
% \gll Kam-pii   siithu=ka    hatthu maccang hatthu duuduk aada jaalang=ka. \\
%      when-go there=\textsc{loc} \textsc{indef} tiger \textsc{indef} sit exist street=\textsc{loc}  \\
%     `When he got there, there was a tiger sitting on the road.' (B060115nar05)
% \z
% } \\
%
%
% \xbox{16}{
% \ea\label{ex:func:unreferenced}
% \gll Laskalli ka-dhaathang,    mari. \\
%      other.time when-come come.imp \\
%     `Visit us when you come again.' (B060115cvs17)
% \z
% } \\
%



% \xbox{16}{
% \ea\label{ex:func:unreferenced}
% \gll Itthu thaaun=jo Mahathma Gandhi arà-buunu thaaun. \\
% \textsc{dist} year=\textsc{emph} Mahathma Gandhi \textsc{non.past}-kill year \\
% `That was the year Mahathma Gandhi was killed.' (nosource)
% \z
% }


% \xbox{16}{
% \ea
% \gll Se ruuma=dering kapang-kuluvar se su-jaatho. \\
%      \textsc{1s} house=\textsc{abl} when-exit \textsc{1s} \textsc{past}-fall  \\
%     `On going out of the house, I fell.' (K081114eli01)
% \z
% } \\
%
%
% \xbox{16}{
% \ea
% \gll Se {\em Point} Pedro dìkkath=dering kapang-pii, lavut=nang su-jaatho. \\
%      \textsc{1s} Point Pedro vicinity=\textsc{abl} when-go sea=\textsc{dat} \textsc{past}-fall  \\
%     `When I went beyond Point Pedro [northernmost point of Sri Lanka] I fell into the sea.' (K081114eli01)
% \z
% } \\

Even if the second state-of-affairs becomes true immediately on the completion of the first one, \trs{blaakang}{after} is an alternative to \em kapang-\em. In \xref{ex:time:pc:blaakang}, the speaker became an orphan as soon as his parents died, not after that. Still \em blaakang \em is used.

\xbox{16}{
\ea \label{ex:time:pc:blaakang}
\gll Se=ppe umma=le baapa=le mnii\u n\u ggal=nang blaakang, se hatthu yathiim su-jaadi. \\
     \textsc{1s=poss} mother=\textsc{addit} father=\textsc{addit} die=\textsc{dat} after \textsc{1s} \textsc{indef} orphan \textsc{past}-become  \\
    `After my father and mother had died, I became an orphan.' (K081114eli01)
\z
}


\subsubsection{Terminal boundary}\label{sec:func:Terminalboundary}

The configuration of a ground point in time indicating the terminal boundary of the figure can be schematized as follows.

\parbox{8cm}{
\ea {\Large $\stackrel{fig}{\multimap }$}\z
 {\Large\vspace{-0.5cm}\hspace{1.5cm} $\uparrow_{gr} $}
}

Terminal boundary is expressed by the adposition \trs{sangke}{until} \formref{sec:morph:=sangke}, which can be a postposition \xref{ex:time:tb:postnom1}-\xref{ex:time:tb:postnom3} or a preposition \xref{ex:time:tb:prenom}.



\xbox{16}{
 \ea \label{ex:time:tb:postnom1}
   \gll Suda derang=nang   [hathyang muusing]=\textbf{sangke} mà-duuduk  su-jaadi. \\
    thus \textsc{3pl}=\textsc{dat} other time=until \textsc{inf}-stay \textsc{past}-become\\
`So they had to wait until another time' (K051220nar01)
\z
}


\xbox{16}{
\ea \label{ex:time:tb:postnom2}
\gll Incayang  [thujupul     liima thaaun]=\textbf{sangke} incayang  anà-iidop. \\
     \textsc{3s.polite} seventy five year=until \textsc{3s.polite} \textsc{past}-live  \\
    `He lived until seventy-five.' (K060108nar02)
\z
}


\xbox{16}{
\ea \label{ex:time:tb:postnom3}
\gll Suda thiga-pulu    satthu=ka duudukapa         thiga-pulu    ìnnam=\textbf{sangke}. \\
      so three-ty one=\textsc{loc} from three-ty six=until \\
    `So, from '31 to '36.' (N061031nar01)
\z
}



\xbox{16}{
\ea \label{ex:time:tb:prenom}
\gll [Kaaki=ka gaa\u ndas-kang ambel anà-duuduk Aajuth=yang \textbf{sangke}=luppas] hathu pollu=dering Rose-red buurung=nang su-puukul. \\
     leg=\textsc{loc} tie-\textsc{caus} take \textsc{past}-stay dwarf=\textsc{acc} until=leave \textsc{indef} stick=\textsc{abl} Rose-red bird=\textsc{dat} \textsc{past}-hit  \\
    `Rose red hit the bird with a stick until it let go of the dwarf he had taken in his claws.' (K070000wrt04)
\z
}

Like the other postpositions, \em sangke \em can also attach to a clause \xref{ex:time:tb:clause}. This is not possible for the prepositional use.


\xbox{16}{
\ea \label{ex:time:tb:clause}
\gll [Buruan diinging abbis]=\textbf{sangke} siithu su-sii\u n\u ggal. \\
      bead cold finish=until there \textsc{past}-stay \\
    `The bear stayed there until the cold season was over.' (K070000wrt04)
\z
}

The terminal boundary can coincide with the moment of speaking, but does not imply that the state-of-affairs will not continue afterwards, as in the following example.

\xbox{16}{
\ea \label{ex:time:tb:speechsituation}
\gll Nyaari=\textbf{sangke} se   inni ruuma=ka=jo       arà-duuduk. \\
 today=until \textsc{1s} \textsc{prox} house=\textsc{loc}=\textsc{emph} \textsc{non.past}-stay\\
`I have been living here until today.' (K060108nar01)
\z
}

In the example above, the terminal boundary is the time of speaking, but this does not imply that the speaker moves out of his house. It just indicates that for the moment the terminal boundary of the event `living', which is still progressing, is the time of speaking.

% \xbox{16}{
% \ea\label{ex:func:unreferenced}
% \gll Itthu muusing asà-duuduk sampe nyaari se pukuran arà-gijja. \\
%  \textsc{dist} time \textsc{cp}-stay reach today \textsc{1s} work \textsc{non.past}-make\\
% `From that time on until today I have been working.' (K060108nar01)
% \z
% }


\subsubsection{Initial boundary}\label{sec:func:Initialboundary}
The initial boundary of a ground point in time, after which the period covered by the figure begins, can be schematized as follows.

\parbox{8cm}{
\ea {\Large $\stackrel{fig}{\multimapinv }$}\z
 {\Large \vspace{-0.5cm}\hspace{1.15cm}  $\uparrow_{gr} $}\\
}

Initial boundary is expressed by means of \em (a)s(à)duuduk \em \formref{sec:wc:Existentialverbs:duuduk} \xref{ex:time:ib:asduuduk}. Whereas source in a spatial sense can be expressed by both \em (a)s(à)duuduk \em and \em =dering\em, the latter is not possible for temporal relations. An alternative realization of \em asduuduk \em is \em duuduk=apa \em \xref{ex:time:ib:duudukapa}.


\xbox{16}{
\ea\label{ex:time:ib:asduuduk}
\gll Itthu muusing \textbf{asduuduk} sangke=nyaari se pukuran arà-gijja. \\
 \textsc{dist} time from until=today \textsc{1s} work \textsc{non.past}-make\\
`From that time on until today I have been working.' (K060108nar01)
\z
}
%
% \xbox{16}{
% \ea\label{ex:func:unreferenced}
% \gll Thirteen     \textbf{asduuduk}     go          blaajar aada {\em tailoring}. \\
%  thirteen \textsc{cp}-stay \textsc{1s.familiar} learn exist tailoring\\
% `From thirteen (years) onwards, I had learned tailoring.' (B060115nar04)
% \z
% }



\xbox{16}{
\ea\label{ex:time:ib:duudukapa}
\gll Suda thiga-pulu    satthu=ka \textbf{duudukapa}         thiga-pulu    ìnnam=sangke. \\
      so three-ty one=\textsc{loc} from three-ty six=until \\
    `So, from '31 to '36.' (N061031nar01)
\z
}

%
% \xbox{16}{
% \ea
% \gll Ka-thiiga January asduudukapa, {\em current} blaangan arà-liivath. \\
%     \textsc{ord}-three January from electricity amount \textsc{non.past}-more  \\
%     `Starting January 3rd, the electricity prices will go up.' (K081114eli01)
% \z
% } \\

Initial boundary can also be expressed by \trs{kapang-}{when}, as in \xref{ex:time:ib:kapang}, or by \trs{vatthu=ka}{at the time of} as in \xref{ex:time:ib:vatthuka}. In these examples, we are not dealing with point coincidence, since the blackness of the speaker did not come about on completion of the process of birth. This contrasts with becoming an orphan \xref{ex:time:pc:blaakang}, which is only true as soon as both parents have died.

\xbox{16}{
\ea\label{ex:time:ib:kapang}
\gll See \textbf{kapang}-laaher, bannyak iitham. \\
      \textsc{1s} when-be.born much dark \\
    `When I was born, I was very dark.' (K081114eli01)
\z
}



\xbox{16}{
\ea\label{ex:time:ib:vatthuka}
\gll See=yang anà-braanak \textbf{vatthu=ka} asàduuduk, see  iitham. \\
     \textsc{1s}=\textsc{acc} \textsc{past}-bear time=\textsc{loc} from \textsc{1s} dark  \\
    `Ever since I was born, I have been very black.' (K081114eli01)
\z
}



\subsubsection{Intermediacy}\label{sec:func:Intermediacy}

\ea {\Large $\begin{array}{c}\stackrel{fig}{\multimapinv\hspace{-.14cm}\multimap}\\\stackrel{gr}{\multimapdotinv----\multimapdot}\end{array}$}
\z

Intermediacy is also expressed by \trs{kapang-}{when} \formref{sec:morph:kapang-}, as in \xref{ex:time:im:quake}, where the time of the  earth quake entirely falls within the stay of the speaker in Pakistan.



%
% kithang=nang   duppang lai     duuva bergaada   dhaathang
% K060108nar02.txt: aada

%
% \xbox{16}{
% \ea
% \gll Lorang see=yang  diinging=dering  kala-aapith. \\
%       \textsc{2pl} \textsc{1s}=\textsc{acc} cold=\textsc{abl} if-??\\
%     `.' (K070000wrt04b)(test)
% \z
% } \\


\xbox{16}{
\ea\label{ex:time:im:quake}
\gll Se Pakistan=ka kapang-duuduk hatthu buumi ginthar-an anà-jaadi. \\
      \textsc{1s} Pakistan=\textsc{loc} when-stay \textsc{indef} earth shiver-\textsc{nmlzr} \textsc{past}-become \\
    `When I was in Pakistan, there was an earth quake.' (K081106eli01)
\z
}


\subsection{Phasal information}\label{sec:func:Phasalinformation}
Besides being embedded in the time frame of the outside world, events also have an internal temporal structure. For expository reasons, we distinguish phasal information, which is concerned with the beginning, progress and end of an event from aspectual information, which is concerned with boundedness of the event. Depending on theoretical orientation, linguists may or may not agree with this choice, but it makes the presentation more straightforward. At this point, I do not want to endorse any particular theoretical relation between phasal and aspectual information, the separation is purely   practical.

Phasal information is concerned with indicating the progress that the event has made in its completion. We distinguish the beginning, the progress and the end.

\subsubsection{Beginning}
To indicate that the action is in its inception, SLM uses the verb \trs{mulain}{start}. The main verb is marked with the infinitive prefix \em mà- \em and normally precedes \em mulain\em.

\xbox{16}{
\ea\label{ex:func:time:phase:begin:mulain1}
\gll
[sini=pe      raaja=nang     mà-banthu] \textbf{anà-mulain}\\
here=\textsc{poss} king=\textsc{dat} \textsc{inf}-help \textsc{past}-start\\
    `(They) started to help the local king.' (K060108nar02)
\z
}

\xbox{16}{
\ea\label{ex:func:time:phase:begin:mulain2}
\gll Siithu=jo see [mà-blaajar] anà-\textbf{mulain}. \\
      there=\textsc{emph} \textsc{1s} \textsc{inf}-learn \textsc{past}-start \\
    `It was there that I started to study.' (K051213nar02)
\z
}

\subsubsection{Progression}\label{sec:func:phase:progression}
Progression is normally not coded when referring to the present. In the past, progressive in subordinates can be coded by \em arà- \em \formref{sec:morph:ara-}, used instead of the normal marking for past tense (\em su-/anà-\em). However, this could be analyzed as simultaneous relative tense as well.

\xbox{16}{
\ea\label{ex:func:time:asp:prog}
\gll Blaakang=jo incayang anà-kuthumung [moonyeth pada thoppi asà-ambel pohong atthas=ka \textbf{arà}-maayeng]. \\
     after=\textsc{emph} \textsc{3s.polite} \textsc{past}-see monkey \textsc{pl} hat \textsc{cp}-take tree top=\textsc{loc} \textsc{simult}-play  \\
    `Then only he saw that the monkeys had taken his hats and were playing on the top of the trees.'  (K070000wrt01)
\z
}

Progression can be emphasized by the use of the vector verb \trs{duuduk}{sit}, as in the following three examples.

\xbox{16}{
\ea\label{ex:func:time:asp:prog:duuduk1}
\gll [Dee arà-sbuuni   \textbf{duuduk}     {\em cave}] asaraathang  sini=ka    asàduuduk hathu  {\em three}  {\em miles} cara  jaau=ka. \\
      3\textsc{s.impolite} \textsc{simult}-hide sit cave] \textsc{copula} here=\textsc{loc} from \textsc{indef} three miles way far=\textsc{loc} \\
    `The cave where the evildoer remained hidden is about three miles from here.' (K051206nar02)
\z
}


\xbox{16}{
\ea\label{ex:func:time:asp:prog:duuduk2}
\gll Incayang suda [aapa=ke hathu pukurjan] mà-girja arà-diyath \textbf{duuduk}. \\
      \textsc{3s.polite} thus what=\textsc{simil} \textsc{indef} word \textsc{inf}-make \textsc{non.past}-try sit \\
    `Now he is looking forward to do some kind or other of work.' (K051222nar08,K081104eli06)
\z
}

\xbox{16}{
\ea\label{ex:func:time:asp:prog:duuduk3}
\gll Loram pada anà-dhaathang vakthu=dika se spaathu anà-gijja \textbf{duuduk}. \\
       \textsc{2pl} \textsc{pl} \textsc{non.past}-come time=\textsc{loc} \textsc{1s} shoe \textsc{past/irr}-make stay  \\
    `When you came, I was making shoes.' (K081104eli06)
\z
}


\subsubsection{Continuation}
To indicate that an event is still going on, \em laile \em is used.


\xbox{16}{
\ea\label{ex:func:time:phase:continue:laile}
\ea
\gll Seelong {\em independent} {\em state} anà-jaadi=nang=apa, \\ % bf
     Ceylon independent state \textsc{past}-become=\textsc{dat}=after  \\
    `After Sri Lanka had become independent,'
\ex
\gll kithang=nang independence anà-daapath=nang=apa. \\ % bf
      \textsc{1pl}=\textsc{dat} independence \textsc{past}-get=\textsc{dat}=after \\
    `after we had obtained the independence,'
\ex
\gll \textbf{laile} derang anà-duuduk {\em under} {\em the} {\em Commonwealth}. \\
      still \textsc{3pl} \textsc{past}-stay under the Commonwealth \\
    `they still were under the Commonwealth.' (K051222nar06)
\z
\z
}

\em Laile \em is used in both positive and negative contexts, as the following two examples show.

\xbox{14}{
\ea
\gll Se \textbf{laile} lorang=nang saaya. \\
      \textsc{1s} still \textsc{2pl}=\textsc{dat} love \\
    `I still love you.' (K081104eli01)
\z
}


\xbox{14}{
\ea
\gll Se=dang \textbf{laile} piisang \textbf{thàrà}-daapath. \\
      \textsc{1s=dat} still banana \textsc{neg.past}-get \\
    `I still have not got any bananas.' (K081104eli01)
\z
}

Given that their   phonological and semantic similarity leads to confusion, it is important to distinguish \trs{laile}{still}, \trs{lai}{more}, \trs{laskali}{again} and \trs{laayeng}{different}, as illustrated in the following examples.

\xbox{16}{
\ea
\gll Naasi \textbf{laile} asà-maatham thraa. \\
      rice still \textsc{cp}-cooked \textsc{neg} \\
    `The rice is not cooked yet.' (K081104eli01)
\z
}


\xbox{16}{
\ea
\gll \textbf{Lai} masà-rubbus. \\
     more must-boil   \\
    `Cook it some more.' (K081104eli01)
\z
}

\xbox{16}{
\ea
\gll \textbf{Laskali} masà-rubbus. \\
     again must-boil  \\
    `Cook it again.' (K081104eli01)
\z
}


\xbox{14}{
\ea
\gll Se=dang \textbf{laayeng} naasi maau. \\
     \textsc{1s=dat} different rice want  \\
    `  I want different rice.' (K081104eli01) 
\z
}



% more
%     se=dang lai naasi maau
%         Sinh tawa
%         I want more rice
%     se=dang laayeng naasi maau
%         I want different rice
%     se=dang lai piisang maau
%         I want more bananas
%     se=dang laayeng piisang maau
%         I want different bananas
%     laile
%         still
%         se=dang laile piisang thàràdaapath
%             I still have not got bananas
%         se laile lorang=nang saaya
%             I still love you

% \xbox{16}{
% \ea
% \gll Naasi laile asà maatham thraa. \\
%       still \\
%     `The rice is not cooked yet.' (eli08111401)
% \z
% } \\
%
%
% \xbox{16}{
% \ea
% \gll Lai masa rubbus. \\
%      more   \\
%     `Cook it some more.' (eli08111401)
% \z
% } \\
%
% \xbox{16}{
% \ea
% \gll Laskali masa rubbus. \\
%      again  \\
%     `Cook it again.' (eli08111401)
% \z
% } \\
%
% \xbox{16}{
% \ea
% \gll Incayang laile asà dhaathang thraa. \\
%       still \\
%     `He has not yet come.' (eli08111401)
% \z
% } \\
%



\subsubsection{End}\label{sec:func:phase:end}
To express that the event has reached its completion \em ab(b)is \em \formref{sec:wc:vv:abbis} is used. Completive is not always easy to distinguish from subsequent events\funcref{sec:func:Subsequence}, since the inception of the second event often implies the completion of the first.

\xbox{16}{
\ea\label{ex:func:time:phase:end}
\gll Kaaving \textbf{abbis}    derang pada=nang=le        aanak aada. \\
     marry finish \textsc{3pl} \textsc{pl}=\textsc{dat}=\textsc{addit} child exist  \\
    `The wedding finished, they also got children.' (K051206nar07)
\z
}

\subsection{Aspectual structure}\label{sec:func:Aspectualstructure}
Phasal information discussed in the last section is used to highlight a portion of an event, be it the beginning, the progress or the end.  Aspectual information, which will be discussed now, is concerned with the internal structure of the event, i.e. is it conceived of as including its temporal boundaries or not. Iterative, distributive and habitual are not concerned with the internal structure of an event, but rather with several events; these areas are discussed under Event Quantification \funcref{sec:func:Eventquantification}.

\paragraph{Perfective and imperfective}\label{sec:func:asp:Imperfective}
Perfectivity and Imperfectivity are normally not expressed in SLM, but subsequent events marked by a chain of \em asà-\em marked verbs, or events marked for completive with \em abis \em are normally perfective \citep[171]{SmithEtAl2006cll}). All other TAM (quasi-)prefixes are underspecified for (im)perfectivity. It seems to be a reasonable hypothesis that the two past markers \em anà- \em and \em su- \em mark some kind of aspectual distinction, but evidence for this hypothesis could not be found. See \formref{sec:morph:Differencebetweensuandana} for a discussion of the similarities and differences between \em su- \em and \em anà-\em.

%
% \paragraph{Ingressive}
% Ingressive aspects focusses on the coming into being of a state-of-affairs. For different predicate types, different strategies are employed. For nominal predicates, \trs{jaadi}{become} is used.
%
%
% \xbox{16}{
% \ea\label{ex:func:time:asp:ingr:n}
% \gll Oorang pada kapang-laari   dhaathang ini      daara sgiithu=le            \textbf{suusu} su-\textbf{jaadi}. \\
%     man \textsc{pl} when-run come \textsc{prox} blood that.much=\textsc{addit} milk \textsc{past}-become   \\
%     `When people came running, the blood had turned into milk.' (K051220nar01)
% \z
% } \\
%
%
% For adjectival predicates, TAM-marking  implies ingressive, as in the following example.
%
%
% \xbox{16}{
% \ea\label{ex:func:time:asp:ingr:adj}
% \gll Itthu=nam blaakang=jo, kitham pada \textbf{anà-bìssar}. \\
%  \textsc{dist} after=\textsc{emph} \textsc{1pl} \textsc{pl} \textsc{past}-big\\
% `After that, we grew up.' (K060108nar02)
% \z
% }
%
% For verbs, ingressive aspect need not be marked, but can be highlighted by the vector verb \em ambel\em.
%
% \xbox{16}{
% \ea\label{ex:func:time:asp:ingr:v:ambel}
% \gll Kanabisan=ka=jo duva oorang=le anà-\textbf{thaau} \textbf{ambel} [Andare duva oorang=yang=le asà-enco-kang aada] katha. \\
%       last=\textsc{loc}=\textsc{emph} two man=\textsc{addit} \textsc{past}-know take Andare two man=\textsc{acc}=\textsc{addit} \textsc{cp}-fool-\textsc{caus} exist \textsc{emph} \\
%     `At the very end, both women understood that Andare had fooled both of them.' (K070000wrt05)
% \z
% } \\
%
% For modal predications, \em jaadi \em is used, similar to nominal predications.
%
%
% \xbox{16}{
% \ea\label{ex:func:time:asp:ingr:mod:jaadi}
% \gll Itthu    blaakang=jo,    kitham=pe {\em AGM}=nang  duppang,  {\em high} {\em commissioner} {\em cultural} {\em show}=nang mà-dhaathang=nang        \textbf{thàràboole} \textbf{s-jaadi}. \\
% dist after=\textsc{emph} \textsc{1pl}=\textsc{poss} AGM=\textsc{dat} before high commissioner cultural show=\textsc{dat} \textsc{inf}-come-\textsc{dat} cannot \textsc{past}-become\\
%     After that, before our Annual General Meeting, it became impossible for the High Commissioner to attend the cultural show. (K060116nar23)
% \z
% } \\
%
% For ingressive aspect of locational predications, a verb of motion is used.
% First, the entity is not at a certain place, and then it is. This is ingressive aspect, but for locational predicates, this is normally coded lexically by a verb of motion, like \trs{pii}{go}{} in \xref{ex:func:time:asp:ingr:loc}.
%
% \xbox{16}{
% \ea\label{ex:func:time:asp:ingr:loc}
% \gll Incayang itthusubbath {\em Malaysia}=nang asà-\textbf{pii} aada. \\
%      \textsc{3s.polite} therefore Malaysia=\textsc{dat} \textsc{cp}-go exist  \\
%     `Therefore, he has gone to Malaysia.' (B060115cvs06)
% \z
% } \\



\section{Quantification}\label{sec:func:Quantity}
Quantification can be used for referents (\em two book\em) or events (\em he returned twice\em.)  Referent quantification will be discussed in  \funcref{sec:func:Referentquantification};  quantification of events follows in  \funcref{sec:func:Eventquantification}.

\subsection{Quantification of referents}\label{sec:func:Referentquantification}
NPs based on nouns and under restricted circumstances NPs  based on plural pronouns (and very rarely even clauses) can be modified for quantity. Different types of quantitative semantics are:
plurality \funcref{sec:func:mod:Plurality},
definite quantity \funcref{sec:func:mod:DefiniteQuantity},
indefinite quantity \funcref{sec:func:mod:IndefiniteQuantity},
definite order \funcref{sec:func:mod:DefiniteOrder}, and
indefinite order \funcref{sec:func:mod:IndefiniteOrder}.


\subsubsection{Plurality} \label{sec:func:mod:Plurality}
Plurality can be expressed on NPs based on nouns by the plural marker \em pada \em \formref{sec:morph:Pluralclitic}. This is not obligatory if the plurality can be inferred from context.
The following examples show the presence and absence of the plural marker on the word \trs{mlaayu}{Malay}{} in very similar contexts.

\xbox{16}{
\ea\label{ex:func:ptcpt:mod:quant:pl:pada}
\gll Bannyak mlaayu \textbf{pada} Hambanthota=ka    arà-duuduk. \\
	much Malay \textsc{pl} Hambantota=\textsc{loc} \textsc{non.past}-exist.\textsc{anim}\\
`There are many Malays in Hambantota.' (B060115nar02)
\z
}

\xbox{16}{
\ea\label{ex:func:ptcpt:mod:quant:pl:zero}
\gll Cinggala=\zero{}   aada mlaayu=\zero{} aada {\em Moor}=\zero{} aada  mulbar=\zero{} aada. \\ % bf
     Sinhala exists Malay exist Moor exist Tamil exist  \\
    `There are Sinhalese, Malays, Moors, and Hindus.' (G051222nar04)
\z
}

Plurality need not be expressed if a numeral is mentioned in the clause, as in \xref{ex:ptcpt:mod:quant:num:nopada}, but nothing precludes using \em pada \em nevertheless, as in \xref{ex:ptcpt:mod:quant:num:pada}.



\xbox{16}{
\ea \label{ex:ptcpt:mod:quant:num:nopada}
\gll Kithang hathu  {\em week}=nang hathu skali \textbf{duva} skaali=\zero=ke arà-maakang. \\
     \textsc{1pl} one week=\textsc{dat} one time two time={ }=\textsc{simil} \textsc{non.past}-eat  \\
    `We might eat it one or twice a week.' (K061026rcp04)
\z
} 

\xbox{16}{
\ea \label{ex:ptcpt:mod:quant:num:pada}
\gll  Se=dang duppang lai  \textbf{thiiga} generation \textbf{pada} ini nigiri=ka     anà-duuduk\\
 \textsc{1s=dat} before other three generation \textsc{pl} \textsc{prox} country=\textsc{loc} \textsc{past}-exist\\
`Before me another three generation existed in this country' (K060108nar02)
\z
}


Plurality can be emphasized on plural pronouns by adding \em pada \em as well. In the following example, the plurality of the word \trs{kitham}{1pl}{} is emphasized by adding \em pada\em.

\xbox{16}{
\ea\label{ex:func:ptcpt:mod:quant:pl:kithampada}
\gll \textbf{Kitham}  \textbf{pada}=pe     baasa=ka       nni {\em grammar} thraa. \\
      \textsc{1pl} \textsc{pl}=\textsc{poss} language=\textsc{loc} \textsc{prox} grammar \textsc{neg}\\
    `There is no grammar in our language.' (G051222nar02)
\z
}

Unlike the example above, the following example has \em kitham \em in it without \em pada\em.


\xbox{16}{
\ea\label{ex:func:ptcpt:mod:quant:pl:kithamnopada}
\gll Ini \textbf{kitham}=\zero=pe nigiri su-jaadi. \\
 \textsc{prox} \textsc{1pl}=\zero=\textsc{poss} country \textsc{past}-become\\
`This (country) became our country.' (K051222nar04)
\z
}

% The distributedness of an item is also expressed by \em pada\em. For instance, the mass noun \trs{duvith}{money}{} can be combined with \em pada \em to yield the distributive meaning `funds'.
%
%
% \xbox{16}{
% \ea \label{ex:ptcpt:mod:pl:distr}
% \gll Duvith \textbf{pada} aada. \\
%  money \textsc{pl} exist\\
% `Funds were available/Coins and banknotes were available.' (K060116nar09)
% \z
% }

Plurality can also be expressed with \em pada \em on finite headless relative clauses, as in \xref{ex:ptcpt:mod:pl:relc}.

\xbox{16}{
\ea \label{ex:ptcpt:mod:pl:relc}
\gll [[Seelon=nang anà-dhaathang] \textbf{pada}] mlaayu pada. \\
 Ceylon=\textsc{dat} \textsc{past}-come \textsc{pl} Malay \textsc{pl}\\
`Those who had come to Ceylon were the Malays.' (N060113nar01)
\z
}

What is treated as `plurality' here might actually rather be an instance of `collective nominal aspect' \citep{Rijkhoff2002}. See \formref{sec:morph:Thesemanticsofpadaandhatthu} for a discussion.



\subsubsection{Definite quantity}\label{sec:func:mod:DefiniteQuantity}
Absolute number can be indicated by cardinal numbers \formref{sec:wc:Numerals} on nominal or pronominal NPs. Numerals precede or follow the noun \xref{ex:ptcpt:num:pre}\xref{ex:ptcpt:num:post}, but always follow the pronoun \xref{ex:ptcpt:num:postpronominal}.

\xbox{16}{
\ea \label{ex:ptcpt:num:pre}
\gll \textbf{Thiiga} oorang, \textbf{thiiga} oorang=le, \textbf{thiiga} oorang pada=jo itthu ini {\em volleyball} arà-{\em play}-king=kee. \\
 three man, three man=\textsc{addit}, three man \textsc{pl}=\textsc{emph} \textsc{dist} \textsc{prox} volleyball \textsc{non.past}-play-\textsc{caus}=\textsc{simil}       \\
    `Three men, three persons, three people play it, like volleyball.'  (N060113nar05)
\z
}

\xbox{16}{
\ea \label{ex:ptcpt:num:post}
\gll [panthas rooja kumbang pohong komplok  \textbf{duuva}] asà-jaadi su-aada. \\
      beautiful rose flower tree bush two \textsc{cp}-grow \textsc{past}-exist \\
    `Two beautiful rose bushes had grown.'  (K070000wrt04)
\z
}



\xbox{16}{
\ea \label{ex:ptcpt:num:postpronominal}
\gll Mr    Sebastian            aada, se aada, \textbf{kitham}  \textbf{duuva} arà-oomong. \\
 Mr Sebastian exist \textsc{1s} exist \textsc{1pl} two \textsc{non.past}-speak\\
`You are here, I am here, the two of us are talking.' (K060116nar05)
\z
}

The plural particle \em pada \em is often present when numerals modify nouns, but not obligatory \xref{ex:ptcpt:mod:defquant:nopada}. It is not present when the numeral modifies a pronoun \xref{ex:ptcpt:num:postpronominal}.


\xbox{16}{
\ea \label{ex:ptcpt:mod:defquant:pada}
\gll Thuuju  {\em generation}  \textbf{pada}  asà-biilang. \\
 seven generation \textsc{pl} \textsc{cp}-say\\
`Seven generations, they say.' (K060108nar02)
\z
}


\xbox{16}{
\ea \label{ex:ptcpt:mod:defquant:nopada}
\gll Mlaayu thigapuluthuuju  baasa \zero{}   aada. \\ % bf
 Malay 37 language { } exist\\
`There are 37 Malay languages.' (K060116nar02)
\z
}

For nominal NPs, the numeral may also follow, but is more often put in front of the noun. \em pada \em is normally present.

Measures normally do not take \em pada\em. In example \xref{ex:ptcpt:mod:defquant:measure}, the word \trs{kaayu}{mile} is used without \em pada\em.

\xbox{16}{
\ea \label{ex:ptcpt:mod:defquant:measure}
\gll Duuva kaayu \zero{}  kithang masà-pii. \\
      two mile { } \textsc{1pl} must-go \\
    `We have to walk two miles.' (K051213nar03)
\z
}



\subsubsection{Indefinite quantity}\label{sec:func:mod:IndefiniteQuantity}
% Indefinite quantity is expressed by quantifiers \formref{sec:wc:Quantifiers}. It is mainly used  to modify nouns \xref{ex:ptcpt:mod:quant:indef:canonical}, but can also be found on pronouns \xref{ex:ptcpt:mod:quant:indef:pron}.
% 
% 
% 
% \xbox{16}{
% \ea \label{ex:ptcpt:mod:quant:indef:canonical}
% \gll \textbf{Bannyak} \textbf{mlaayu} pada Hambanthota=ka    arà-duuduk. \\
%           many Malay \textsc{pl} Hambantota=\textsc{loc} \textsc{non.past}-stay\\
%     `There are many Malays in Hambantota.' (B060115nar02)
% \z
% } \\
% 
% 
% 
% \xbox{16}{
% \ea \label{ex:ptcpt:mod:quant:indef:pron}
% \gll Suda kithang=nang   boole mosthor \textbf{kithang} \textbf{samma} oorang asà-kumpul. \\
%      thus \textsc{1pl}=\textsc{dat} can manner \textsc{1pl} all man \textsc{cp}-gather  \\
%     `So all of us gathered as we could.' (B060115nar02)
% \z
% } \\






%
% \xbox{16}{
% \ea \label{ex:ptcpt:mod:quant:indef:floated2}
% \gll Oorang=pe baarang pada \textbf{samma} arà-cuuri. \\
%        man=\textsc{poss} goods \textsc{pl} all \textsc{non.past}-stay\\
%     `He steals all the people's goods.' (K051205nar02)
% \z
% } \\


% \xbox{16}{
% \ea\label{ex:func:unreferenced}
% \gll Cinggala su-aada sdiikith. \\
%      Sinhala \textsc{past}-exist few   \\
%     `There were few Sinhalese.' (K051222nar06)
% \z
% } \\
%
%
% \xbox{16}{
% \ea\label{ex:func:unreferenced}
% \gll Thiiga umpath aada pompang pada. \\
%       three four exist girl \textsc{pl} \\
%     `There are three or four girls.' (K061019nar02)
% \z
% } \\
%

% Table \ref{tab:QuantityAdverbs} gives a list of common quantity adverbs.
%
% \begin{table}
% 	\begin{center}
% 	% use packages: array
% 	\begin{tabular}{ll|ll}
% 	SLM & gloss & SLM &  gloss \\
% 	konnyom & few  & spaaru & some\\
%  	sdiikith & few & bannyak & many\\
% 	\end{tabular}
% 	\end{center}
% 	\caption{Common quantity adverbs}
% 	\label{tab:QuantityAdverbs}
% \end{table}




% \paragraph{Relative quantity}\label{sec:func:mod:RelativeQuantity}
% Relative quantity is normally expressed by a relative clause involving the adjectives \trs{liivath}{more} or  \trs{kuuram}{few,less}.
%
% % \xbox{16}{
% % \ea \label{ex:ptcpt:mod:quant:relative:libbi1}
% % \ea
% % \gll \textbf{Spaaru} oorang pada polis  armi. \\
% % some man \textsc{pl} police army\\
% % `Some man in the police, the army.'
% % \ex
% % \gll Derang \textbf{libbi} oorang pada samma siini  arà-duuduk. \\
% %  \textsc{3pl} more man \textsc{pl} all here \textsc{non.past}-stay\\
% % `More than those (living there) are living here.' (K051213nar07)
% % \z
% % \z
% % }
% %
% % \xbox{16}{
% % \ea \label{ex:ptcpt:mod:quant:relative:libbi2}
% % \gll \textbf{Libbi} mlaayu samma Klumbu=ka arà-duuduk. \\
% %  more Malay all Colombo=\textsc{loc} \textsc{non.past}-stay\\
% % `More Malays (than elsewhere) live all in Colombo.' (K060108nar02)
% % \z
% % }
%
% \xbox{16}{
% \ea \label{ex:ptcpt:mod:quant:relative:kuurang}
% \gll Ini \textbf{kuurang} arà-duuduk    laayeng  kumpulan    pada=yang   mà-{\em represent}-kang=nang. \\
%       \textsc{prox} few \textsc{non.past}-exist.\textsc{anim} other group \textsc{pl}=\textsc{acc} \textsc{inf}-represent-\textsc{caus}=\textsc{dat} \\
%    `To represent the other groups with fewer people staying in them (i.e. minorities).' (N061031nar01)
% \z
% } \\


In the domain of indefinite quantity, we can distinguish the extreme points of totality and zero quantity, with a number of intermediate levels.

In SLM, quantity of entities can be expressed by quantifiers like \trs{bannyak}{many} \formref{sec:wc:Quantifiers} or by WH-words combined with boolean clitics like \trs{saapa=so}{someone}, \trs{kaapang=ke}{some day} \trs{kaapang=le}{always} or \trs{kaapang=pon}{never} \formref{sec:nppp:Nounphrasesbasedoninterrogativepronouns}.

Normally, there is a set of referents over which the quantity ranges, like \em aanak \em in \xref{ex:func:quant:intro:aanak:set}, but it is also possible to express general quantities which range over all imaginable referents \xref{ex:func:quant:intro:aanak:noset}.


\xbox{16}{
\ea\label{ex:func:quant:intro:aanak:set}
\gll \textbf{Bannyak} \textbf{aanak} pada karang mlaayu  thama-oomong. \\
       much child \textsc{pl} now Malay \textsc{neg.nonpast}-speak\\
    `Many children do not speak Malay now.' (G051222nar02)
\z
} 

\xbox{16}{
\ea\label{ex:func:quant:intro:aanak:noset}
\gll \textbf{Bannyak} mà-biilang thàrboole. \\
 much \textsc{inf}-say cannot\\
`(I) can't tell you much.' (K051206nar12)
\z
}



\paragraph{Zero proportion}\label{sec:func:Zeroproportion}
A proportion of zero is indicated by a negated predicate, mostly of the verbal, existential or adjectival type. Zero proportion is mostly not marked by any other marker than the negation of the predicate. In \xref{ex:func:quant:zero:thama}, the verbal negator \em thama- \em suffices to imply that none of the fighters changed their words.


\xbox{16}{
\ea\label{ex:func:quant:zero:thama}
\gll Derang pada \zero{} kathahan thama-thuukar.  \\ % bf
      \textsc{3pl} \textsc{pl} { } word \textsc{neg.irr}-change \\
    `They would not change their words.' (K051213nar06)
\z
}

Similar things can be said about \xref{ex:func:quant:zero:thraa}, where the absence of any Malay is only marked by the existential negator \em thraa\em.

\xbox{16}{
\ea\label{ex:func:quant:zero:thraa}
\gll Itthuka \zero{} mlaayu \textbf{thraa}, bannyak=nang {\em English}=jo aada. \\
	dist=\textsc{loc} { }  Malay \textsc{neg} much=\textsc{dat} English=\textsc{emph} exist \\
	`There is no Malay over there, it is all English which is there.'  (B060115prs15)
\z
}

If the total absence of applicable reference shall be emphasized, the enclitic \em =pon \em can be used.
This is the case in \xref{ex:func:quant:zero:pon} where the set to which the negative predicate applies is stated by \trs{oorang}{men}{} and the totality of the absence is indicated by a following \em =pon\em. Note that \em oorang \em is in this case accompanied by the indefinite article \em hatthu\em. In this construction, the indefinite article precedes the noun, whereas in other constructions, it can also follow the noun.

\xbox{16}{
\ea \label{ex:func:quant:zero:pon}
\gll Kithang \textbf{hatthu}=oorang=\textbf{pon} thàrà-iingath. \\
       \textsc{1pl} \textsc{indef}=man=any \textsc{neg.past}-think\\
    `We cannot think of any person.'  (B060115nar02)
\z
}

Total absence of any referent of any set is applied by adding \em =pon \em to the indefinite article \em atthu \em as in \xref{ex:func:quant:zero:atthupon}. The missing mention of domain (cf. \em oorang \em just above) indicates that the negation ranges over any and all possible referents, there is not a single thing which you can do on the bus.

\xbox{16}{
\ea \label{ex:func:quant:zero:atthupon}
\gll {\em Bus}=ka \zero{} \textbf{hatthu=pon} mà-kirja thàràboole. \\
      bus=\textsc{loc} { } \textsc{indef}=\textsc{neg} \textsc{inf}-make cannot \\
    `You can't do anything on the bus.' (K061125nar01)
\z
}

\paragraph{More than one entity}\label{sec:func:Morethanoneentity}
Cardinality greater than one need not be specified, but can be expressed by a numeral \xref{ex:func:quant:1+:num}, the plural word \em pada \em \xref{ex:func:quant:1+:pada}, or both, if necessary \xref{ex:func:quant:1+:numpada}.

\xbox{16}{
\ea \label{ex:func:quant:1+:num}
\gll \textbf{Duva-pulu}    \textbf{ìnnam} \textbf{riibu}    \textbf{ùmpath}  \textbf{raathus} \textbf{lima-pulu}    \textbf{duuva} {\em votes} \zero{}  incayang=nang    anà-daapath. \\
 two-ty six thousand four hundred five-ty two votes { } \textsc{3s.polite}=\textsc{dat} \textsc{past}-get\\
    `He got 26,452 votes.' (N061031nar01)
\z
}



\xbox{16}{
\ea \label{ex:func:quant:1+:pada}
\gll Itthu    vatthu=ka    itthu    nigiri  \textbf{pada}=ka    arà-duuduk. \\
     \textsc{dist} time=\textsc{loc} \textsc{dist} land \textsc{pl}=\textsc{loc} \textsc{non.past}-stay \\
    `At that time, (they) lived in those countries.'  (N060113nar01)
\z
}

\xbox{16}{
\ea \label{ex:func:quant:1+:numpada}
\gll Se=dang \textbf{liima} anak  klaaki \textbf{pada} aada. \\
      \textsc{1s=dat} five child male \textsc{pl} exist \\
    `I have five sons.' (K060108nar02)
\z
}

Furthermore, the following sections also treat different possibilities to indicate sets of cardinality greater than one.

\paragraph{Low proportion}\label{sec:func:Lowproportion}
A low proportion of referents, like English \em few, \em is indicated by \em sdiikith \em or \em konnyom \em and the affirmative or negative predicate, as the case may be.

\xbox{16}{
\ea \label{ex:func:quant:low:sdiikith}
\gll Siithu Dubai=ka Sri Lanka=pe oorang mlaayu pada \textbf{sdiikith} arà-duuduk. \\
      there Dubai=\textsc{loc} Sri Lanka=\textsc{poss} man Malay \textsc{pl} few \textsc{non.past}-saty \\
    `There in Dubai, there are few Sri Lankan Malays.' (K061026prs01)
\z
} 

\xbox{16}{
\ea \label{ex:func:quant:low:konnyong}
\gll \textbf{Konnyong} mlaayu=jo Seelong=ka thiinngal aada. \\
     few Malay=\textsc{emph} Ceylon=\textsc{loc} settle exist  \\
    `Few Malays have settled down in Sri Lanka.' (K051222nar06)
\z
}

% \xbox{16}{
% \ea\label{ex:func:unreferenced}
% \gll Indian oorang pada arà-duuduk  Pakistan oorang pada arà-duuduk  Sri Lankan oorang pada \textbf{sdiikith}=jo arà-duuduk. \\
%      Indian man \textsc{pl} \textsc{non.past}-exist.\textsc{anim} Pakistan man \textsc{pl} \textsc{non.past}-exist.\textsc{anim} Sri Lankan man \textsc{pl} few=\textsc{emph} \textsc{non.past}-exist.\textsc{anim}   \\
%     `There are Indians and Pakistanis, Sri Lankans are very few.  (K061026prs01)
% \z
% } \\



\paragraph{Unclear proportion}\label{sec:func:Unclearproportion}
To indicate that the exact proportion is unclear, but not negligible, \em spaaru \em is used.
\em Spaaru \em indicates a slightly higher proportion than \em sdiikith \em or \em konnyom\em. It can refer to about half of the set, which is not possible for \em sdiikith \em or \em konnyom\em. In this respect, \em spaaru \em resembles English \em some\em, which covers more important proportions than \em few\em.

\xbox{16}{
\ea \label{ex:func:quant:some1}
\gll \textbf{Spaaru} oorang pada su-pii, \textbf{spaaru} oorang pada su-birthi. \\
    some man \textsc{pl} \textsc{past}-go some man \textsc{pl} \textsc{past}-stop   \\
    `Some people left and some people stayed.' (B060115nar01)
\z
}

\xbox{16}{
\ea \label{ex:func:quant:some2}
\gll \textbf{Spaaru} oorang pada thàrà-pii sindari. \\
     some man \textsc{pl} \textsc{neg.past}-go from.here  \\
    `Some men did not go from here.' (K051206nar07)
\z
}

\paragraph{High proportion}\label{sec:func:Highproportion}
A high proportion is indicated by \em bannyak\em.


\xbox{16}{
\ea \label{ex:func:quant:high1}
\gll Itthu=pe        pada=jo    \textbf{bannyak} mlaayu pada karang siini aada. \\
     \textsc{dist=poss} \textsc{pl=foc} much Malay \textsc{pl} now here exist  \\
    `It's their folks we get a lot of today here.' (K051205nar04)
\z
}

\xbox{16}{
\ea \label{ex:func:quant:high2}
\gll Oorang pada  thiikam apa,  oorang pada=nang thee\u mbak apa,  se=dang \textbf{bannyak}  creeveth pada su-aada. \\
      man \textsc{pl} stab after man \textsc{pl}=\textsc{dat} shoot after \textsc{1s=dat} much trouble \textsc{pl} \textsc{past}-exist \\
    `People were stabbed, people were shot, I had a lot of problems.' (K051213nar01)
\z
}


Additionally, a lexical solution like  \trs{punnu}{full}  can be used \xref{ex:ptcpt:mod:quant:indef:lexical:punnu1}\xref{ex:ptcpt:mod:quant:indef:lexical:punnu2}. Using \trs{guunung}{mountain} for this was also overheard, but could not be verified in the corpus and needs additional verification.

\xbox{16}{
\ea \label{ex:ptcpt:mod:quant:indef:lexical:punnu1}
\gll  Pake-yan=ka=le \textbf{punnu} bedahan thraa \\
      dress=\textsc{nmlzr}=\textsc{loc}=\textsc{addit} full difference \textsc{neg} \\
    `Even at our dress there is not much difference.'
\z
}

\xbox{16}{
\ea \label{ex:ptcpt:mod:quant:indef:lexical:punnu2}
\gll \textbf{Punnu}   mlaayu oorang=nang=le        cinggala  mà-blaajar    thàrà-suuka=nang;  derang laayeng nigiri  pada=nang   su-pii. \\
     full Malay man \textsc{dat}=\textsc{addit} Sinhala \textsc{inf}-learn \textsc{neg}-like=\textsc{dat} \textsc{3pl} other country \textsc{pl}=\textsc{dat} \textsc{past}-go  \\
    `Many Malays did not want to learn Sinhala and went to other countries.' (K051222nar06)
\z
}


\paragraph{Totality}\label{sec:func:Totality}
To indicate that all entities of the set are included in the predication, either \trs{samma}{each/every/all}{} \formref{sec:wc:Quantifiers} or the WH=\em le\em-construction \formref{sec:nppp:NPscontaininginterrogativepronounsusedforuniversalquantification} is used.

Example \xref{ex:func:quant:all:samma} shows the use of bare \em samma \em to indicate the totality of the people knowing the speaker.

\xbox{16}{
\ea \label{ex:func:quant:all:samma}
\gll {\em Doctors} pada=so {\em police} ASP=so  {\em judge}=so, \textbf{samma} oorang thaau see=yang \\
     doctors \textsc{pl}=\textsc{undet} police ASP=\textsc{undet} judge=\textsc{undet} all man know \textsc{1s}=\textsc{acc}  \\
    `Whether they be doctors, policemen, assistant superintendents of police or judges, all men know me.'  (B060115nar04)
\z
}

More emphasis on the totality can be given by adding \em =le \em to either the entity itself \xref{ex:func:quant:all:sammanle} or \em samma \em   \xref{ex:func:quant:all:sammale}.

\xbox{16}{
\ea \label{ex:func:quant:all:sammanle}
\gll \textbf{Samma} oorang=\textbf{le}      saanak. \\
      all man=\textsc{addit} relative \\
    `All people are relatives.' (K051206nar07)
\z
}

\xbox{16}{
\ea \label{ex:func:quant:all:sammale}
\gll Suda incayang=pe aanak pada \textbf{samma=le} {\em musicians} pada=jo. \\
     thus \textsc{3s.polite}=\textsc{poss} child \textsc{pl} every=\textsc{addit} musicians \textsc{pl}=\textsc{emph}  \\
    `So all his children are musicians.'  (G051222nar01)
\z
}

Finally, the combination of an appropriate WH-word with again \em =le \em yields a universal quantifier as well. In this case, \em =le \em can be attached to the entity itself as in \xref{ex:func:quant:all:whle:n} or to the predicate as in \xref{ex:func:quant:all:whle:predicate}


\xbox{16}{
\ea \label{ex:func:quant:all:whle:n}
\gll Skarang \textbf{maana} aari\textbf{=le} atthu atthu oorang=yang arà-buunung. \\
     now which day=\textsc{addit} one one man=\textsc{acc} \textsc{past}-kill   \\
    `Now people kill each other every day.'  (K051206nar11)
\z
}

\xbox{16}{
\ea\label{ex:func:quant:all:whle:predicate}
\gll Lai     \textbf{saapa} mlaayu kuthumung=\textbf{le} aapa=ke      {\em connection} hatthu aada. \\
     other who Malay see=\textsc{addit} what=\textsc{simil} connection \textsc{indef} exist \\
    `If you see any other Malay, there will always be some kind of connection.' (K051206nar07)
\z
}



% \xbox{16}{
% \ea\label{ex:func:unreferenced}
% \gll Mlaayu=pe samma criitha pada=le. \\
%      Malay=\textsc{poss} all story \textsc{pl}=\textsc{addit}  \\
%     `All the Malays' stories.' (N061031nar01)
% \z
% } \\



\subsubsection{Definite order}\label{sec:func:mod:DefiniteOrder}
Definite order is expressed by an ordinal derived from the cardinal by \em ka- \em \formref{sec:morph:ka-}.
Ordinal numbers  can only precede the noun. They do not seem to be used with pronouns or clauses.

\xbox{16}{
\ea \label{ex:ptcpt:mod:deford:measure}
\gll Se   asdhaathangpa kitham=pe      femili=ka    \textbf{ka-duuva}
aanak \\
 \textsc{1s} \textsc{copula} \textsc{1pl}=\textsc{poss} family=\textsc{loc} \textsc{ord}-two child\\
`I am the second child in our family.' (K060108nar01)
\z
}

\subsubsection{Indefinite order}\label{sec:func:mod:IndefiniteOrder}
To indicate indefinite order of an entity, the ablative \em =dering \em is used together with the relator noun \trs{daalang}{inside}. A precise quantity like \trs{liima}{five} can be used \xref{ex:ptcpt:mod:order:relative:liima}, or a collective noun like \trs{kumpulan}{group} in \xref{ex:ptcpt:mod:order:relative:liima}.


\xbox{16}{
\ea\label{ex:ptcpt:mod:order:relative:liima}
\gll Se pàrthaama *(liima) oorang=dering daalang {\em race}=yang su-abbis-king. \\
      \textsc{1s} first five man=\textsc{abl} inside race=\textsc{acc} \textsc{past}-finish-\textsc{caus} \\
    `I finished the race among top five.' (K081106eli01)
\z
}


\xbox{16}{
\ea\label{ex:ptcpt:mod:order:relative:kumpulan}
\gll Se pàrthaama kumpulan=dering daalang {\em race}=yang su-abbis-king. \\
      \textsc{1s} first group=\textsc{abl} inside race=\textsc{acc} \textsc{past}-finish-\textsc{caus} \\
    `I finished the race among the first.' (K081106eli01)
\z
}





\subsection{Event quantification}\label{sec:func:Eventquantification}
Next to referents, events can also be quantified \citep[179f]{HengeveldEtAl2008fdg}. I will limit myself to events taking place several times in this discussion. We can distinguish habitual event quantification, which indicates that the event takes place regularly, from iterative event quantification, which indicates that the event takes place more than once, but without any commitment to the regular status. Another type of event quantification is the distributive, which indicates that the event was performed by several participants at different points in time, as in \em the guests arrived two by two\em. There number of events in this case is clearly higher than one, there are several instances of arrivals. Distributive event quantification will be the last type to by discussed.

\paragraph{Habitual}\label{sec:func:asp:Habitual}
Habitual can be marked by either  the present tense quasi-prefix \em arà- \em \citep{Ansaldo2009book}  \formref{sec:morph:ara-} \xref{ex:time:aspect:habit:ara} or the irrealis quasi-prefix \em anthi- \em \formref{sec:morph:anthi-} \xref{ex:time:aspect:habit:irr}. 


\xbox{16}{
\ea \label{ex:time:aspect:habit:ara}
  \ea
 \gll Hindu \textbf{arà}-maakang kambing. \\
	Hindu \textsc{non.past}-eat goat\\
	`Hindus eat goat.'

	\ex
	\gll samping \textbf{thuma}-maakang \\
		beef \textsc{neg.nonpast} eat\\
	`(they) don't eat beef.' (K060112nar01)
	\z
\z
}

\xbox{16}{
\ea \label{ex:time:aspect:habit:irr}
\gll Saudi=ka ontha \textbf{anthi}-kaasi. \\
 Saudi.Arabia=\textsc{loc} camel \textsc{irr}-give\\
`In Saudi Arabia, they give camels.' (K060112nar01)
\z
}

Irrealis marking for habitual context can also be used with past reference \citep[cf.][]{Givon1994sil}. The following example has the adverb \trs{kàthaama}{before}, which indicates past reference, but still uses \em anthi- \em to convey the habitual reading.

\xbox{16}{
\ea \label{ex:time:aspect:habit:irr:past}
\gll Punnu   mlaayu pada    \textbf{kàthaama} {\em English}=jona        \textbf{anthi}-oomong. \\
       many Malay \textsc{pl} before English=\textsc{phat} \textsc{irr}-speak\\
    `Many Malays used to speak English in former times, isn't it?' (K051222nar06)
\z
}


Things are the same with the following example, where an explicit reference to the past (`There once was a time') is combined with the irrealis marker to convey habituality.


\xbox{16}{
\ea \label{ex:time:aspect:habit:past:neg2} 
   \gll Samma kithang=pe     mlaayu, hathu  muusing su-aada,      samma cinggala=dering=jo      \textbf{athi}-oomong. \\
    all   \textsc{1pl}=\textsc{poss} Malay  \textsc{indef}  time    \textsc{past}-exist all   Sinhalese=\textsc{abl}=\textsc{emph} \textsc{irr}-talk \\
`There once was a time where all Malays (in our home) would speak in Sinhala.' (B060115cvs01)
\z 
}

\paragraph{Iterative}
There are no means to indicate iterative or repetitive aspect in the strict sense.
The repetition of an event can be indicated by \trs{laskal(l)i}{again}.

\xbox{16}{
\ea \label{ex:time:aspect:iter:laskalli1}
\gll Asà-kumpul,     blaakang \textbf{laskalli} asà-{\em beat}-king, inni daalang=ka    kithang aayer  masà    mà-libbi-king. \\
      \textsc{cp}-add after again \textsc{cp}-beat-\textsc{caus} \textsc{prox} inside=\textsc{loc} \textsc{1pl} water must \textsc{inf}-remain-\textsc{caus} \\
    `Having added (the ingredients), having then beaten them again, we must reserve the water in the middle.' (B060115rcp02)
\z
}

\em Laskalli \em can be reduplicated as in \xref{ex:time:aspect:iter:laskalli:red}.

\xbox{16}{
\ea \label{ex:time:aspect:iter:laskalli:red}
\gll Se=ppe pake-yan pada bannyak koothor. Itthu=subbath laskalli laskalli mà-cuuci su-jaadi \\
     \textsc{1s=poss} dress-\textsc{nmlzr} \textsc{pl} much dirt \textsc{dist}=because again again \textsc{inf}-wash \textsc{past}-become \\
    `My clothes were very dirty. That's why I had to wash them over and over again.' (K081106eli01)
\z
}

Another construction which comes close in meaning  and involves reduplication of an infinitive is given in \xref{ex:time:aspect:iter}.

\xbox{16}{
\ea \label{ex:time:aspect:iter}
\gll Itthu=nang      blaakang karang aari \textbf{mà-pii}    \textbf{m-pii}      karang kitham=pe      mlaayu arà-mulain. \\
      \textsc{dist}=\textsc{dat} after now day \textsc{inf}-go \textsc{past}-go now \textsc{1pl}=\textsc{poss} Malay \textsc{non.past}-start \\
    `Now, after that, as days went by, our Malay is beginning.' (K051222nar03)(K081106eli01)
\z
}


% \xbox{16}{
% \ea \label{ex:time:aspect:iter:laskalli2}
% \gll Deram pada asà-pii=apa  kettha=nam  \textbf{laskalli} asà-dhaathang bannyak thriima  kaasi. \\
%     \textsc{3pl} \textsc{pl} \textsc{cp}-go after \textsc{1pl}=\textsc{dat} again \textsc{cp}-come lot thanks give   \\
%     `They went and came again to us and thanked us.' (B060115nar02)
% \z
% } \\


\paragraph{Distributive}

Distributive is indicated by a reduplicated numeral, a head noun, and the plural marker.

\cb{NUM\~{}NUM N PL \NP* PRED}

This is possible for numbers greater than one \xref{ex:time:aspect:distr:2}, but also for \trs{hatthu}{one} \xref{ex:time:aspect:distr:1a} \xref{ex:time:aspect:distr:1b}.


\xbox{16}{
\ea \label{ex:time:aspect:distr:2}
\gll Duuva duuva oorang pada su-pii. \\
    two two man \textsc{pl} \textsc{past}-go   \\
    `People left two by two.' (K081106eli01) %continues 23456
\z
}

\xbox{16}{
\ea \label{ex:time:aspect:distr:1a}
\gll Hatthu hatthu oorang pada su-pii. \\
     one one man \textsc{pl} \textsc{past}-go  \\
    `People left one by one.' (K081106eli01)
\z
}




\xbox{16}{
\ea \label{ex:time:aspect:distr:1b}
\gll {\em Kandy} {\em Malay} {\em Association}=dering \textbf{hatthu} \textbf{hatthu} oorang \textbf{pada} arà-lompath {\em Hill} {\em Country}=nang. \\
     Kandy Malay association=\textsc{abl} \textsc{indef} \textsc{indef} man \textsc{pl} \textsc{non.past}-jump hill country=\textsc{dat}  \\
    `More and more people stepped over from the KMA to the Hill Country Malay Association.' (K060116nar07,K081106eli01)
\z
}



% \xbox{16}{
% \ea
% \gll Oorang pada=ka hatthu hatthu arà-thaanya ambel. \\
%        \\
%     `I ask people questions one by one.' (K081106eli01)
% \z
% } \\

% If the numbers are more important, a periphrastic construction is preferred \xref{ex:time:aspect:distr:37}.
% 
% 
% \xbox{16}{
% \ea \label{ex:time:aspect:distr:37}
% \gll Hatthu skaali=ka thiga-pul-thuuju oorang pada su-pii. \\
%      one time=\textsc{loc} three-ty-seven man \textsc{pl} \textsc{past}-go  \\
%     `Every time, 37 people left.' (K081106eli01)
% \z
% } \\


% \xbox{16}{
% \ea\label{ex:func:ptcpt:mismatch:reciprocal}
% \ea
% \gll \textbf{Hatthu} \textbf{hatthu} oorang=yang anà-buunung thraa. \\
% 	one one man=\textsc{acc} \textsc{past}-kill \textsc{neg}\\
% 	`before, they were not killing people one by one.'
% \ex
% \gll Skarang maana aari=le \textbf{atthu} \textbf{atthu} oorang=yang arà-buunung. \\
%      now which day=\textsc{addit} one one man=\textsc{acc} \textsc{non.past}-kill   \\
%     `But now, every day, one by one, people are killed.'  (K051206nar11)
% \z
% \z
% }\\
%
% \xbox{16}{
% \ea
% \gll {\em Police} oorang hatthu hatthuyang su buunung. \\
%        \\
%     `the police killed the people one by one.' (K081106eli01)
% \z
% } \\
%
% \xbox{16}{
% \ea
% \gll Oorang pada hatthu hatthuyang su buunung. \\
%        \\
%     `the people killed each other one by one.' (K081106eli01)
% \z
% } \\


\subsection{Temporal frequency}\label{sec:func:Temporalfrequency}
Generally, temporal frequency can be expressed lexically by a temporal noun like \trs{aari}{day}{} or \trs{vatthu}{time}{} which is modified by a quantifier \formref{sec:wc:Quantifiers}, such as \trs{spaaru vatthu}{some time}. The reader is referred to \funcref{sec:func:mod:IndefiniteQuantity} for more discussion of this pattern involving quantifiers. The temporal domain \funcref{sec:func:Temporaldomain} (like `per week', `per year') is indicated by \em =nang \em \formref{sec:morph:=nang}.

Zero frequency is expressed by a negated predicate. There is no special word for \em never\em, but the negation can be reinforced by \trs{kaapang}{when}{} with the enclitics \trs{=pon}{any} or \trs{=le}{additive}, as in example \xref{ex:func:time:tempfreq:neg:kaapangpon} and \xref{ex:func:time:tempfreq:neg:kaapangle}.

\xbox{16}{
\ea\label{ex:func:time:tempfreq:neg:kaapangpon}
\gll Suda itthu kithang=nang \textbf{kaapang=pon} thama=luupa. \\
     thus \textsc{dist} \textsc{1pl}=\textsc{dat} when=any \textsc{neg.nonpast}=forget  \\
    `So, we will neverever forget this.'  (B060115nar02)
\z
}

\xbox{16}{
\ea\label{ex:func:time:tempfreq:neg:kaapangle}
\gll Go  \textbf{kaapang=le}      saala thama-gijja. \\
     \textsc{1s.familiar} when=\textsc{addit} wrong \textsc{neg.nonpast}-make  \\
    `I never do any wrong.' (B060115nar04)
\z
}

The opposite of this is total frequency (\em always\em). This can be expressed either lexically by \em subbang vatthu \em or by again \trs{kaapangle}{when=\textsc{addit}} \xref{ex:func:time:tempfreq:aff:kaapangle}.


\xbox{16}{
\ea\label{ex:func:time:tempfreq:aff:kaapangle}
\gll Girls' {\em High} {\em School} Kandi=ka se=dang \textbf{kaapang=le}  udahan hatthu arà-kiiring. \\
     girls high school Kandy=\textsc{loc} \textsc{1s=dat} when=\textsc{addit} invitation \textsc{indef} \textsc{non.past}-send  \\
    `I am always invited to the Girls' High School in Kandy.' (K061127nar03)
\z
}

Finally, the WH=\em le\em-construction with a temporal noun can be used as well.

\xbox{16}{
\ea\label{ex:func:time:tempfreq:aff:WHle}
\gll Dee \textbf{maana} aari=\textbf{le}      asà-dhaathang, thìngaari vakthu=nang   kalthraa maalang vakthu=nang ... \\
     3 which day=\textsc{addit} \textsc{cp}-come noon time=\textsc{dat} otherwise night time=\textsc{dat}  \\
    `He came every day, at noon or otherwise during the night (and attacked).' (K051206nar02)
\z
}


\section{Modality}\label{sec:func:Modality}
Speakers do not only exchange absolute truths. They also convey their estimation of the likelihood, necessity and desirability of a situation. This is the domain of modality. Following \citet{Hengeveld2004imm}, modality can be divided along two axes, the target of evaluation \funcref{sec:func:Modality:targets}  and the domain of evaluation \funcref{sec:func:Modality:domains}.

\subsection{Targets of modality}\label{sec:func:Modality:targets}
As for the target of modality, we can distinguish whether it affects an event as in \em One must go to school\em, where the obligation conveyed by \em must \em covers the general need to attend school. This event-oriented modality can be contrasted with participant-oriented modality, as in \em John must go to school at 7h45\em, where the target of the obligation is  a participant, John. In SLM, the difference between participant-oriented modality and event-oriented modality is marked by the presence vs. absence of the participant alone.

In SLM, any event-oriented target can be made participant-oriented by adding an NP including the participant. This participant is normally marked with the dative \em =nang \em \formref{sec:morph:=nang}. Conversely, any participant-oriented modality statement can be changed into an event-oriented one by dropping the participant (Although this is potentially ambiguous due to the high number of dropped topical participants in general, see \funcref{sec:disc:Activereferents}).
In example \xref{ex:func:modality:deont:event}, we are dealing with event modality: no one can go far into that cave. Hence no participant is mentioned. In \xref{ex:func:modality:deont:participant}, on the other hand, we are dealing with participant modality, because it is only the speaker who is unable to go. This is indicated by mentioning \trs{se}{1s} and adding the dative, in this case \em =dang\em.


\xbox{16}{
\ea\label{ex:func:modality:deont:event}
 \gll Bannyak jaau mà-pii \textbf{thàrboole},  itthu=ka   \\
 much far \textsc{inf}-go cannot \textsc{dist}=\textsc{loc}      \\
    `You/one cannot go far there/It is impossible to go far into that cave.' (K051206nar02)
\z
}

\xbox{16}{
\ea\label{ex:func:modality:deont:participant}
\gll \textbf{Se=dang} karang jaau mà-pii \textbf{thàràboole}. \\
     \textsc{1s=dat} now far \textsc{inf}-go cannot  \\
    `I cannot go far now.'  (K061120nar01)
\z
}


In \xref{ex:func:modality:targets:event}, a recipe, we are dealing with event-oriented modality. Eggs are required for the recipe no matter the identity of the participant.

\xbox{16}{
\ea\label{ex:func:modality:targets:event}
\gll Thullor maau. \\  % bf
 egg want  \\
`You need eggs [to make a dessert].' (B060115rcp02)
\z
}

This contrasts with \xref{ex:func:modality:targets:participant}, where the participant for whom the necessity holds is specified, the speaker, expressed by \em sedang\em. People  other than the speaker (who might have different requirements for their jobs) are not under this obligation.

\xbox{16}{
\ea\label{ex:func:modality:targets:participant}
\gll \textbf{Se=dang} se=ppe pukujan pada=nang baaru hatthu kaar \textbf{maau} \\
      \textsc{1s=dat} \textsc{1s=poss} work \textsc{pl}=\textsc{dat} new \textsc{indef} car need \\
    `I need a new car for my work.' (K081114eli01)
\z
}

%  
% This type is concerned with the possibility of doing something, not because someone allowed it or one has the capacity to do so, but because relevant dispositions are available, such as the bus in \xref{ex:func:modality:dyn}. In \xref{ex:func:modality:dyn}, it is not the case that the person in need of transport has the permission or the exceptional capacity to take the bus. It is rather the case that a  bus service operates, which the person can use to get from A to B. Facility is coded by \em boole\em.
% %
% % \xbox{16}{
% % \ea\label{ex:func:modality:dyn}
% % \gll Bas=ka=lle        \textbf{bolle}=pii. \\
% %       bus=\textsc{loc}=\textsc{addit} can=go \\
% %     `You can also go by bus.'  (B060115cvs08 )
% % \z
% % }\\
% 
% % K051222nar05.txt:  thapi English kalablaajar    Ceylong samma thumpath bolekluuling
% 
% 
% \xbox{16}{
% \ea\label{ex:func:modality:dyn}
% \gll Lorang=nang bas=ka Jaapna(=nang) \textbf{bole}=pii. \\
%       \textsc{2pl}=\textsc{dat} bus=\textsc{loc} Jaffna=\textsc{dat} can=go \\
%     `You can go to Jaffna by bus.' (K081114eli01)
% \z
% } \\
% 
% On the other hand, the railway line to Jaffna is interrupted. This  is coded by \em thàrboole\em. Again, this is not a question of permission or capacity, but of the existence of the relevant facilities.
% 
% 
% \xbox{16}{
% \ea
% \gll Jaalang asà-thuuthup=siking, lorang=nang koocci=ka Jaapna mà-pii \textbf{thàràboole}. \\
%       path \textsc{cp}-close=because \textsc{2pl}=\textsc{dat} train=\textsc{loc} Jaffna \textsc{inf}-go cannot \\
%     `Because the line is closed, you cannot go to Jaffna by train.' (K081114eli01)
% \z
% } \\

%
% \xbox{16}{
% \ea
% \gll Lorang nang {\em Maldives} nang koocci ka ma pii thàràboole. \\
%        \\
%     `.' (nosource)
% \z
% } \\



A third type is proposition-oriented modality, covering the speaker's views and beliefs, and commitment to the truth of the utterance. This type can be found in the  epistemic and evidential domains and will be covered below.\footnote{There are some languages where proposition-oriented modality can also be combined in the volitive domain, but this is not the case in SLM.}

\subsection{Domains of modality}\label{sec:func:Modality:domains}
\citet{Hengeveld2004imm} distinguishes 5 domains of modality:

\begin{itemize}
 \item facultative modality, concerned with intrinsic or acquired capacities \funcref{sec:func:Facultativemodality},
 \item deontic modality, concerned with what is permissible \funcref{sec:func:Deonticmodality},
 \item volitive modality, concerned with what is desirable \funcref{sec:func:Volitivemodality},
 \item epistemic modality, concerned with what is known about the actual world \funcref{sec:func:Epistemicmodality},
 \item evidential modality, concerned with the source of information \funcref{sec:func:Evidentialmodality}
\end{itemize}
 
These subtypes will be discussed in turn.
 
\subsubsection[Facultative]{Facultative modality}\label{sec:func:Facultativemodality}
Facultative modality deals with the (in)capacities of participants and general facilities which permit performance of certain actions.

\paragraph{Capacity}\label{sec:func:Capacity}
Capacity is coded by \em boole \em \formref{sec:wc:boole}. There is no difference between physical \xref{ex:func:modality:ability:physical}  and mental ability  \xref{ex:func:modality:ability:mental}.


\xbox{16}{
\ea\label{ex:func:modality:ability:physical}
\gll [\textbf{Boole} oorang pada]=na     siithu boole pii. \\
 can man \textsc{pl}=\textsc{dat} there can go\\
`The men who can go may go.' (B060115cvs01)
\z
}


\xbox{16}{
\ea\label{ex:func:modality:ability:mental}
\gll Cinggala  bahasa   saapa=nang=le        \textbf{bole}=bicaara      siini. \\
     Sinhala language who=\textsc{dat}=\textsc{addit} can-talk here  \\
    `Anybody can talk Sinhala here.' (K051206nar14)
\z
}



Example \xref{ex:func:modality:ability:physical} shows the use of \em boole \em to indicate capacity (first occurrence) and permission (second occurrence). This diverging interpretation of \em boole \em in the two clauses makes the sentence non-tautological. In return, speakers normally do not utter tautologies, so that the fact that \xref{ex:func:modality:ability:physical} was uttered proves that the two tokens of \em boole \em must indeed have different interpretations.

\paragraph{Incapacity}\label{sec:func:Incapacity}
Incapacity is expressed  by \em thàrboole \em \formref{sec:wc:therboole}.


\xbox{16}{
\ea\label{ex:func:modality:imposs1}
\gll Bannyak mà-biilang \textbf{thàrboole}. \\
 much \textsc{inf}-say cannot\\
`(I) can't tell you much.' (K051206nar12)
\z
}


\xbox{16}{
\ea\label{ex:func:modality:imposs2}
\gll Derang pada=nang atthu=le mà-kijja=nang  \textbf{thàràboole}=subbath ....\\
 	\textsc{3pl}         \textsc{pl}=\textsc{dat} one=\textsc{addit}  \textsc{inf}-do=\textsc{dat} cannot=because\\
`Because they couldn't do anything.' (N060113nar01)
\z
}


\xbox{16}{
\ea
\gll Lorang=nang aayer=ka appi mà-mnyala-king thàràboole. \\
      \textsc{2pl}=\textsc{dat} water=\textsc{loc} fire \textsc{inf}-burn-\textsc{caus} cannot \\
    `You cannot light fire in water.' (K081114eli01)
\z
}




\subsubsection[Deontic]{Deontic modality}\label{sec:func:Deonticmodality}
Deontic modality covers the external or internal need to (not) perform a certain action.
  

\paragraph{Permission}\label{sec:func:Permission}
Permission is coded by the modal particle \em boole \em \formref{sec:wc:boole} \formref{sec:wc:boole}.

\xbox{16}{
\ea\label{ex:func:modality:permission}
\gll Lai   sdiikith       aari=jo            go=dang \textbf{bolle}=duuduk. \\
 other little day=\textsc{emph} \textsc{1s.familiar} can=stay\\
`I may stay some more days [on Earth because Allah allows me to].' (B060115nar04)
\z
}



% \xbox{16}{
% \ea
% \gll Municipal kithang nang {\em permission} anà kaasi siking, kithang nang siinika ruuma hatthu bole ikkath. \\
%       Municipal \textsc{1pl}=\textsc{dat} permission  \\
%     `.' (K081114eli01)
% \z
% } \\


\paragraph{Interdiction}\label{sec:func:Interdiction}
Deontic interdiction is normally expressed by a periphrasis involving the adjective \trs{thàràbaae}{not.good}. Another possibility is the use of \trs{thàrboole}{cannot} \formref{sec:wc:therboole}. The interdiction is inferred from the primary meaning of incapacity.

\xbox{16}{
\ea\label{ex:func:modality:interd}
\gll {\em Cigarette} mà-miinong \textbf{thàràboole}. \\
      cigarette \textsc{inf}-drink cannot \\
    `It is forbidden to smoke.' (K060116nar04)
\z
}


\xbox{16}{
\ea
\gll Maalang=nang mà-svara-kang \textbf{thàràboole}. \\
     night=\textsc{dat} \textsc{inf}-noise-\textsc{caus} cannot  \\
    `It is forbidden to make noise at night.' (K081114eli01)
\z
}


The speech act of interdiction is not deontic and is  covered under prohibition \funcref{sec:pragm:Requestingaction}.

\paragraph{Obligation}\label{sec:func:Obligation}
Obligation can be expressed by the quasi-prefix  \em masthi- \em \formref{sec:morph:masthi-}. This is an exception to the rule in that it  does not take the dative, but the nominative. Still, it can be used for participant modality as in \xref{ex:func:modality:obl:part} or for event modality as in \xref{ex:func:modality:obl:event}.

\xbox{16}{
\ea\label{ex:func:modality:obl:part}
\ea
\gll Luu=\zero{} baaye=nang \textbf{masà}-blaajar baaye=nang \textbf{masà}-mnaaji. \\
      \textsc{2s.familiar} good=\textsc{dat} must-learn good=\textsc{dat} must-recite \\
    `You have to learn well and you have to recite well.'
\ex
\gll Lu=ppe umma-baapa=nang baaye=nang \textbf{masà}-kaasi thaangang. \\
      \textsc{2s}=\textsc{poss} mother-father=\textsc{dat} good=\textsc{dat} must-give hand \\
    `You must lend a hand to your parents.' (K060116sng01)
\z
\z
}

\xbox{16}{
\ea\label{ex:func:modality:obl:event}
\gll Hathu oorang kala-pasiith, hathu {\em chance} \textbf{masà}-kaasi ithu  oorang=nang=le. \\
     \textsc{indef} man if-trouble \textsc{indef} chance must-give \textsc{dist} man=\textsc{dat}=\textsc{addit}  \\
    `When a person has trouble, we must give him a chance.' (K060116nar07)
\z
}

Periphrastic constructions involving the existential \em aada \em and \trs{jaadi}{become} are also possible  \formref{sec:wc:Specialconstructionsinvolvingverbalpredicates}.

\xbox{16}{
\ea\label{ex:func:modality:obl:jaadi}
\gll Se=ppe    {\em profession}=subbath \textbf{se=dang}  siini \textbf{mà}-pii    su-\textbf{jaadi}. \\
      \textsc{1s=poss} profession=because \textsc{1s=dat} here \textsc{inf}-go \textsc{past}-become \\
    `I had to come here because of my profession.' (G051222nar01)
\z
}

\xbox{16}{
\ea\label{ex:func:modality:obl:aada}
\gll \textbf{Se=dang} \textbf{aada} ini {\em army} pada=yang \textbf{mà}-salba-kang=\textbf{nang}. \\
     \textsc{1s=dat} exist \textsc{prox} solder \textsc{pl}=\textsc{acc} \textsc{inf}-escape-\textsc{caus}=\textsc{dat}  \\
    `I had to save these soldiers.' (K051213nar01)
\z
}


The construction with \em aada \em indicates a general obligation, whereas the construction with \em jaadi \em indicates an obligation which was caused by a change of circumstances. In example \xref{ex:func:modality:obl:jaadi}, the need to move to another town was caused by a change in professional status, whereas in \xref{ex:func:modality:obl:aada}, no such change happened, it was just the general human obligation to help the soldiers which obtained. The obligation to help people in need is not brought about by a change in environment, but is a moral value (even if in this case it is applied to a concrete situation). This general obligation irrespective of circumstances is indicated by \em aada\em.

% \xbox{16}{
% \ea\label{ex:func:unreferenced}
% \gll Se=dang karang ruuma=nang masà-pii. \\
%       \textsc{1s=dat} now house=\textsc{dat} must-go \\
%     `I have to go home now.' (B060115cvs08)
% \z
% } \\

The following  examples show the difference between the constructions with \em jaadi \em and with \em aada\em. The need to swim in case of a flood is indeed caused by the circumstances, which is why \em jaadi \em is used. That need is not caused by authority, which is why it is impossible to use \em aada \em there.

\xbox{16}{
\ea\label{ex:func:modality:obl:disc:flood}
\gll Hatthu baanjir anà-dhaathang=siking, se=dang \textbf{mà}-bìrnang su-\textbf{jaadi}/*su-\textbf{aada}. \\
    \textsc{indef} flood \textsc{past}-come=because \textsc{1s=dat} \textsc{inf}-swim \textsc{past}-become/\textsc{past}-exist   \\
    `Because a flood was coming, I had to swim.' (K081114eli01)
\z
}

The inverse situation holds in \xref{ex:func:modality:obl:disc:prayers}, where the need to pray five times a day is imposed by religious authority and not by the circumstances. This is why \em aada \em is possible, and \em jaadi \em is not.

% \xbox{16}{
% \ea
% \gll Lorang hatthu aari nang liima skali masa sbaayang. \\
%        \\
%     `.' (K081114eli01) normal
% \z
% } \\


\xbox{16}{
\ea\label{ex:func:modality:obl:disc:prayers}
\gll Lorang=nang hatthu aari=nang liima skali \textbf{mà}-sbaayang \textbf{aada/*arà-jaadi}. \\
      \textsc{2pl}=\textsc{dat} one day=\textsc{dat} five time \textsc{inf}-pray exist/\textsc{non.past}-become \\
    `You have to pray five times a day.' (K081114eli01) 
\z
}



% \xbox{16}{
% \ea
% \gll *se=dang liima skali masbaayang nang sujaadi. \\
%        \\
%     `.' (K081114eli01)
% \z
% } \\

%
% \xbox{16}{
% \ea
% \gll Se naapas arà-ambel. \\
%        \\
%     `I am breathing.' (K081114eli01)
% \z
% } \\
%
%
% \xbox{16}{
% \ea
% \gll Se naapas masa ambel. \\
%        \\
%     `I have to take a breath.' (nosource) when at dump and longing for fresh air
% \z
% } \\
%
%
% \xbox{16}{
% \ea
% \gll Se=dang naapas ma ambel maau. \\
%        \\
%     `I need/want to breathe.' (K081114eli01)
% \z
% } \\
%
%
% \xbox{16}{
% \ea
% \gll *se=dang naapas ma ambel su aada. \\
%        \\
%     `.' (nosource)
% \z
% } \\

Finally, there are some cases where the need to do something is permanent, and not caused by authority or exceptional circumstances. One of these cases is breathing. Then, only the quasi-prefix \em masà- \em is possible.

\xbox{16}{
\ea\label{ex:func:modality:obl:disc:breath:masa}
\gll Mà-iidop=nang, naapas \textbf{masà}-ambel. \\
      \textsc{inf}-live=\textsc{dat} breath must-take \\
    `You must breathe in order to survive.' (K081114eli01)
\z
}


\xbox{16}{
\ea\label{ex:func:modality:obl:disc:breath:aada}
\gll *mà-iidop=nang naapas \textbf{mà}-ambel \textbf{aada}. \\
      \textsc{inf}-live=\textsc{dat} breath \textsc{inf}-take exist \\
    `(You are compelled to breathe in order to survive).' (K081114eli01)
\z
}


\paragraph{Necessity}\label{sec:func:Necessity}
Necessity is expressed by \em (ka)mau(van) \em \formref{sec:wc:(ka)mau(van)}. This particle is also used for desire.

\xbox{16}{
\ea\label{ex:func:modality:obl:nec:mau}
\gll Thullor maau. \\  % bf
 egg want  \\
`You need eggs [to make a dessert].' (B060115rcp02)
\z
}

\xbox{16}{
\ea\label{ex:func:modality:obl:nec:kamauvan}
   \gll Kithang=nang   hathu  {\em application} mà-sign  \textbf{kamauvan} vakthu=nang=jo,      kithang arà-pii    inni     {\em politicians} pada dìkkath=nang. \\
   \textsc{1pl}=\textsc{dat} \textsc{indef} application \textsc{inf}-sign want time=\textsc{dat}=\textsc{emph} \textsc{1pl} \textsc{non.past}-go \textsc{prox} politicians \textsc{pl} vicinity=\textsc{dat} \\
`When we want/have to sign an application, we approach these politicians' (K051206nar12)
\z
}

The borderline between obligation, necessity and desire is generally not clear cut in the Lankan languages. Sri Lanka Malay has a more clear cut distinction between obligation conveyed by \em masthi- \em and desire conveyed by \em (ka)maau(van)\em, but this seems to be eroding, and indiscriminate use of \em maau \em for both desire and obligation is becoming more common, paralleling the Sinhala and Tamil semantics. Example \xref{ex:func:modality:obl:nec:blur} shows the use of \em maau \em to convey mild obligation. This particle is normally used to express desire (see above).

\xbox{16}{
\ea\label{ex:func:modality:obl:nec:blur}
\gll Baapa=nang {\em mosque}=nang mà-pii \textbf{maau}. \\ % bf
 father=\textsc{dat} mosque=\textsc{dat} \textsc{inf}-go want\\
`Father, you should go to the mosque. (\em literally\em: Father, you want to go to the mosque)' (B060115nar04)
\z
}

\paragraph{Lack of necessity}\label{sec:func:Lackofnecessity}
Lack of necessity has to be distinguished from lack of desire \funcref{sec:func:Lackofdesire}. Lack of necessity is coded by \em thàr(ka)mauvan \em \formref{sec:wc:(ka)mau(van)}, while lack of desire is coded by \em thussa \em \formref{sec:wc:thussa}.\footnote{My informants furthermore added that \em thàrà\textbf{j}amauvan \em was a `common mistake', so that that form seems to have a certain frequency as well, even if I have not come across it personally.}

\xbox{16}{
\ea\label{ex:func:modality:lackofnecessity:maau}
\gll Se=dang se=ppe pukujan pada=nang baaru hatthu kaar \textbf{maau} \\
      \textsc{1s=dat} \textsc{1s=poss} work \textsc{pl}=\textsc{dat} new \textsc{indef} car need \\
    `I need a new car for my work.' (K081114eli01)
\z
}

\xbox{16}{
\ea\label{ex:func:modality:lackofnecessity:tharmauvan}
\gll Loram pukujan=nang baaru hatthu kaar \textbf{thàràmauvan}. \\
     \textsc{2pl} work=\textsc{dat} new \textsc{indef} car \textsc{neg}.need  \\
    `You do not need a new car for your work.' (K081114eli01)
\z
}

\subsubsection[Volitive]{Volitive modality}\label{sec:func:Volitivemodality}
Volitive modality is concerned with what is desirable. The desire can be participant-oriented as in \em John wants to give money to the poor\em, event-oriented \em Giving money to the poor is desirable\em, or proposition-oriented \em Mary wants John to give money to the poor. \em The difference is that the person feeling the desire participates in the desired event in the first case and is not mentioned in the second case. In the third case, the person feeling the desire is mentioned but does not participate in the event.

We can distinguish presence of desire from absence of desire. These two possibilities will be discussed below.

\paragraph{Desire}\label{sec:func:Desire}
Desire is coded by the modal particle \em (ka)mau(van) \em \formref{sec:wc:(ka)mau(van)}, which derives from the lexical word \trs{kamauvan}{desire}. The meaning of \em kamauvan \em is ambiguous between `need' and `desire' \xref{ex:func:modality:desire:intro}.\footnote{This ambiguity only exists in affirmative contexts, in negative contexts, lack of need is coded by \em thàrkamauvan \em \xref{ex:func:modality:lackofnecessity:tharmauvan}, and lack of desire is coded by \em thussa \em \xref{ex:func:modality:lackofdesire}.}

\xbox{16}{
\ea\label{ex:func:modality:desire:intro}
\gll Se=dang baaru hatthu kaar \textbf{(ka)maau(van)}. \\
      \textsc{1s=dat} new \textsc{indef} car want \\
    `I want/need a new car.' (K081114eli01)
\z
}

The following three examples show the use of \em maau \em for a desired object \xref{ex:func:modality:desire:ent} and a desired state \xref{ex:func:modality:desire:soa}. Both examples are participant-oriented in that the person(s) feeling the desire are given.

\xbox{16}{
\ea\label{ex:func:modality:desire:ent}
\gll Deran=nang    thumpath \textbf{maau}. \\
     \textsc{3pl}=dat place want  \\
    `They wanted land.'  (N060113nar01)
\z
}

\xbox{16}{
\ea\label{ex:func:modality:desire:soa}
\gll Itthusubbath=jo incayang=nang  ini Sri Lankan {\em Malay} mà-blaajar \textbf{maau}. \\
  therefore=\textsc{emph} 3\textsc{s.polite}=\textsc{dat} \textsc{prox} Sri Lankan Malay \textsc{inf}-learn want  \\
`This is why he wants to learn this Sri Lanka Malay.' (B060115prs15)
\z
}

As for event-oriented volitive modality, it appears that SLM, like the other Sri Lankan languages, does not draw a sharp distinction between what is desirable (volitive) and what is required (by law or morale, deontic). Event-oriented volitive modality seems to be expressed by \em masthi-\em, otherwise used for obligation, as shown below.


\xbox{16}{
\ea 
\gll Minnyak klaapa ini raa\u mbuth=dering mas-goosok. \\ % bf
      coconut.oil coconut \textsc{prox} hair=\textsc{abl} must-rub \\
`You must rub coconut oil (over the itching) with (human) hair.' (K060103cvs02)
\z
}


There is only one example of proposition-oriented volitive modality in the corpus, which is by a speaker who frequently drops grammatical morphemes, so that this example must be taken with a grain of salt. It is conjectured that the infinitive prefix \em mà- \em should surface in the subordinate clause, but this could not be tested.

\xbox{16}{
\ea  
\gll Kithang=nang maau kitham=pe mlaayu lorang blaajar, lorang=pe mlaayu kitham blaajar. \\
 \textsc{1pl}=\textsc{dat} want \textsc{1pl}=\textsc{poss} Malay \textsc{2pl} learn \textsc{2pl}=\textsc{poss} Malay \textsc{1pl} learn\\
`We want that you learn our [Sri Lankan] Malay, and we learn your [Malaysian] Malay.' (K060116nar02)
\z
}

% \xbox{16}{
% \ea\label{ex:func:modality:desire:soas}
% \gll Kitha=nang \textbf{maau} kitham=pe mlaayu loorang blaajar lorang=pe mlaayu kitham blaajar. \\
%  \textsc{1pl}=\textsc{dat} want \textsc{1pl}=\textsc{poss} Malay \textsc{2pl} learn \textsc{2pl}=\textsc{poss} Malay \textsc{1pl} learn\\
% `We want that you learn our [Sri Lankan] Malay, and we learn your [Malaysian] Malay.' (K060116nar02)
% \z
% }



% \xbox{16}{
% \ea\label{ex:func:modality:desire:lexical}
% \gll Lorang [se=dang \textbf{kamauvan}] pada=yang gijja kaasi. \\
%      \textsc{2pl} \textsc{1s=dat} desire \textsc{pl}=\textsc{acc} make give  \\
%     `Please fulfill my wishes.'  (K051220nar01)
% \z
% }\\




\paragraph{Lack of desire}\label{sec:func:Lackofdesire}
Lack of desire is expressed by \em thussa \em \formref{sec:wc:thussa}. This is thus different from the coding of lack of necessity, which is \em thàrkamauvan \em \formref{sec:wc:(ka)mau(van)} \xref{ex:func:modality:lackofnecessity:tharmauvan}.

\xbox{16}{
\ea\label{ex:func:modality:lackofdesire}
\gll Se=dang baaru hatthu kaar thussa. \\
      \textsc{1s=dat} new \textsc{indef} car \textsc{neg}.want \\
    `I do not want a new car.' (K081114eli01)
\z
}



\subsubsection[Epistemic]{Epistemic modality}\label{sec:func:Epistemicmodality}
Epistemic modality is concerned with the likelihood, probability and faith that speakers have in the truth of the propositions they state.
\citet[213]{FoleyEtAl1984} give the following continuum for epistemic modality

\ea real $\leftarrow$ necessary -- probable -- possible $\rightarrow$ unreal \z

In SLM, five levels of certainty can be distinguished. This was tested with a small made up setting. Some relatives have left for Badulla, and the question is whether at the time of speaking they have reached Badulla or not. The informants have spontaneously proposed percentages of probability for the different constructions, and this is repeated here.\\

$
\begin{array}{l}
$\parbox{6.2cm}{
\ea \gll Derang Badulla=na sampe\\
they Badulla=\textsc{dat} reach\\
\z
}$
\end{array} 
\left\{
\begin{array}{lr}
$
\parbox{7.5cm}{\vspace{-.3cm}
\gll aada\\
exist\\\vspace{-.3cm}
`They   have arrived at Badulla.'
}
$& 100\%
\\
$
\parbox{7.5cm}{
\gll anthi=jo aada\\
\textsc{irr=foc} exist\\\vspace{-.3cm}
`They will  have arrived at Badulla.'
}
$& 75\%
\\
$
\parbox{7.5cm}{
\gll anthi-aada\\
\textsc{irr}-exist\\\vspace{-.3cm}
`They might have arrived at Badulla.'
}
$& 50\%
\\
$
\parbox{7.5cm}{
\gll thama=jo aada\\
\textsc{neg.irr}=\textsc{emph} exist\\\vspace{-.3cm}
`They might  not have arrived at Badulla yet.'
}
$& 25\%
\\
$
\parbox{7.5cm}{
\gll thama-aada\\
\textsc{neg.irr}-exist\\\vspace{-.3cm}
`They have not arrived at Badulla yet.'
}\vspace{.3cm}
$& 0\%
\\
\end{array}\right.
$\\
 

We see that the perfect construction with \em aada \em conveys certainty, while its negation \em thama-aada \em conveys certainty of the contrary. The irrealis marker \em anthi- \em conveys the most uncertain estimation (50\%). By attaching the emphatic clitic \em =jo \em \formref{sec:morph:=jo} to either \em anthi- \em or \em thama-, \em a somewhat greater likelihood of the event is expressed.

Other means to express epistemic modality, like adverbs or particles do not seem to exist.  \xref{ex:func:modality:epist} is the only occurence of the construction given above in the corpus.


\xbox{16}{
\ea\label{ex:func:modality:epist}
\gll Bìssar aanak asà-dhaathang \textbf{anthi-aada} ruuma=nang. \\
      big child \textsc{cp}-come \textsc{irr}=exist house=\textsc{dat} \\
    `My big child will have come home.' (B060115cvs08)
\z
}

The following sentence is from an unrelated elicitation session, but illustrates the epistemic use of \em thama-aada\em.

\xbox{16}{
\ea
\gll Incayang hatthu thookal, itthu=le spaaru vakthu=nang, incayang {\em Maldives}=nang \textbf{asà-birnang} \textbf{thama-aada}. \\
      \textsc{3s.polite} \textsc{indef} fool, but some time=\textsc{dat} \textsc{3s.polite} Maldives=\textsc{dat} \textsc{cp}-swim \textsc{neg.irr}-exist \\
    `He may be a fool, but never would he have tried to swim to the Maldives.' (K081114eli01)
\z
}

\subsubsection[Evidential]{Evidential modality}\label{sec:func:Evidentialmodality}

Speakers can indicate  the source of  information: is it first-hand knowledge or hearsay?
In SLM, first-hand knowledge does not receive special marking, while non-first-hand information can be indicated by the evidential marker \em kiyang \em \formref{sec:morph:kiyang} for hearsay. The use of \em kiyang \em is optional, though, and not very frequent.

\xbox{16}{
\ea \label{ex:func:modality:evid:kiyang}
\gll Seelong {\em Airport}=yang duva-pulu-ùmpath vakthu=le asà-bukka arà-simpang \textbf{kiyang}. \\
     Ceylon Airport=\textsc{acc} two-ty-four hour=\textsc{addit}  \textsc{cp}-open \textsc{non.past}-stay \textsc{evid}\\
    `The Ceylon Airport will stay open 24h, it seems.'  (Letter 26.06.2007)
\z
}

In example \xref{ex:func:modality:evid:kiyang}, the speaker cannot vouch for the truth of the information he provides about the opening hours of the Colombo airport. He has only second hand knowledge of this, presumably taken from the media. This lack of first-hand knowledge is indicated by \em kiyang\em.

A lexical solution to convey evidential modality is to use \trs{biilang}{say}


\xbox{16}{
\ea \label{ex:func:modality:evid:biilang}
\gll Itthu    nya-aada     katha=le      \textbf{arà-biilang}. \\
      \textsc{dist} \textsc{past}-exist \textsc{quot}=\textsc{addit} \textsc{non.past}-say \\
    `[About a talisman] It is also said that there was a talisman.' (K051206nar02)
\z
}


Both possibilities are exemplified below.


\xbox{16}{
\ea
\gll {\em Maldives}=ka baaru hatthu {\em president} aada katha \textbf{arà-biilang}. \\
      Maldives=\textsc{loc} new \textsc{indef} president exist \textsc{quot} \textsc{non.past}-say \\
    `It is said that there is a new president in the Maldives.' (K081114eli01)
\z
}


\xbox{16}{
\ea
\gll {\em Maldives}=ka baaru hatthu {\em president} aada \textbf{kiyang}. \\
     Maldives=\textsc{loc} new \textsc{indef} president exist \textsc{evid}  \\
    `It is said that there is a new president in the Maldives.' (K081114eli01)
\z
}

Finally, a third possibility involves the `undetermined' clitic \em =so \em and indicates that the speaker is not completely sure about the propositional content he is conveying.

\xbox{16}{
\ea
\gll {\em President}=yang asà/su/anà-buunung=\textbf{so} \textbf{mana}=\textbf{so}. \\
     president=\textsc{acc} \textsc{cp}-/\textsc{past}-/\textsc{past}-kill=\textsc{undet} which=\textsc{undet}  \\
    `The president was assassinated or something.' (K081114eli01)
\z
}


% \xbox{16}{
% \ea
% \gll {\em President} yang asà/su/anà  buunung kiyang. \\
%        \\
%     `.' (K081114eli01)
% \z
% } \\



\section{Conditionals}\label{sec:func:Conditionals}
Condition is expressed by the particle \em kalu \em \formref{sec:wc:kalu} (affirmative) or \em kalthra \em (negative) \formref{sec:wc:kalthra} in the conditional clause. The consequence is normally marked with the irrealis marker \em anthi- \em in the affirmative or \em thama- \em in the negative.\footnote{Talking about conditions, especially counterfactuals, is very rare and the aim of elicitation sessions on counterfactuals was not always clear to the informants.}


\xbox{16}{
\ea\label{ex:func:conditionals:intro}
\ea
\gll See lorang=nang \textbf{thama}=sakith-kang. \\
      \textsc{1s} \textsc{2pl}=\textsc{dat} \textsc{neg.irr}=pain-\textsc{caus} \\
    `I will do you no harm'
\ex
\gll Lorang see=yang diinging=dering \textbf{kala}-aapith. \\
     \textsc{2pl} \textsc{1s}=\textsc{acc} cold=\textsc{abl} if-look.after \\
    `if you protect me from the cold.'  (K070000wrt04)
\z
\z
}


\xbox{16}{
\ea\label{ex:func:conditionals:wick}
\ea
\gll Derang hathu  papaaya=yang   asà-poothong=apa, \\ % bf
      \textsc{3pl} \textsc{indef} papaw=\textsc{acc} \textsc{cp}-cut=after\\
    `When they cut open a papaya.'
\ex
\gll Papaaya=ka    suu\u mbu hatthu \textbf{kal}-thaaro, \\
     papaw=\textsc{loc} wick \textsc{indef} when-put \\
    `and put a wick in the papaya.'
\ex
\gll Aayer=ka    \textbf{kal}-thaaro    \textbf{thama}=myaalak. \\
     water=\textsc{loc} when-put \textsc{neg.nonpast}=burn  \\
    `and put it into water, it would not burn.'
\ex
\gll Minnyak     \textbf{kal}-thaaro    \textbf{anthi}-myaalak. \\
     coconut.oil when-put \textsc{irr}-burn  \\
    `When they put coconut oil, it would burn.' (K051220nar01) 
\z
\z
}

If a modal prefix or particle is used, this supersedes the use of \em anthi-\em. This is the case in the following examples, where \em masthi- \em and \em bole= \em take the preverbal position and block \em anthi- \em from occurring there.

\xbox{16}{
\ea\label{ex:func:conditionals:masthi}
\gll Hathu oorang \textbf{kala}-pasiyeth, hathu {\em chance} \textbf{masà}-kaasi ithu  oorang=nang=le. \\
     \textsc{indef} man if-suffer \textsc{indef} chance must-give \textsc{dist} man=\textsc{dat}=\textsc{addit}  \\
    `When one man suffers, we must give him a chance.' (K060116nar07)
\z
}

% \xbox{16}{
% \ea\label{ex:func:conditionals:boole1}
% \gll Go=dang    asà-poothong        \textbf{kala}-kaasi   lorang=nang   \textbf{bole}=jaaith. \\
%       \textsc{1s.fam}=\textsc{dat} \textsc{cp}-cut if-give \textsc{2pl}=\textsc{dat} can-sew \\
%     `If you cut it and give it to me, you can sew it.' (B060115nar04)
% \z
% } \\


\xbox{16}{
\ea\label{ex:func:conditionals:boole2}
\gll Siini=jo incayang=yang \textbf{kala}-baava, \textbf{bole}=thaau ambel. \\
     here=\textsc{emph} \textsc{3s.polite}=\textsc{acc} if-bring can=know take  \\
    `If you bring him here, he can come to know.' (K061030mix01)
\z
}

Conditionals of unavailability can be formed by adding  \em kalthraa \em to the NP expressing the lacking substance \xref{ex:func:conditionals:noun:neg}.

\xbox{16}{
\ea\label{ex:func:conditionals:noun:neg}
\gll Lorang=ka duvith (*aada)  \textbf{kal-thra}, kithang anthi-banthu. \\
     \textsc{2pl}=\textsc{loc} money exist if-\textsc{neg} \textsc{1pl} \textsc{irr}-help  \\
    `If you have no money available, we will help you.'  (K081103eli04)
\z
}


In the affirmative, the use of the existential \em aada\em, which is lacking in \xref{ex:func:conditionals:noun:neg}, is obligatory \xref{ex:func:conditionals:noun:aff}.

\xbox{16}{
\ea\label{ex:func:conditionals:noun:aff}
\gll Se=dang saayap \textbf{kala-aada}, bole=thìrbang. \\
     \textsc{1s=dat} wing if-exist can=fly  \\
    `If I had wings I could fly.' (K081114eli01)
\z
}

The conditional marker attaching to a present tense predicate indicates possibility \xref{ex:func:conditionals:realis}, while its being used on a perfect tense predication indicates counterfactuality \xref{ex:func:conditionals:counterfactual}.

\xbox{16}{
\ea\label{ex:func:conditionals:realis}
\gll Lorang asà-caape \textbf{kala-blaajar}, lorang=nang A/L  bole={\em pass}. \\
      \textsc{2pl} \textsc{cp}-tired if-learn \textsc{2pl}=\textsc{dat} A/L can=pass \\
    `If you study hard, you will be able to pass the A/L exams.' (K081114eli01)
\z
}


\xbox{16}{
\ea\label{ex:func:conditionals:counterfactual}
\gll Sampi! Luu maalas! A/L thàrà-pass. Lorang baae=nang asà-caape \textbf{asà}-blaajar \textbf{kala-aada}, mà-{\em pass}=nang su-aada \\
      cow \textsc{2s.familiar} lazy A/L \textsc{neg.past}-pass \textsc{2pl} good=\textsc{dat} \textsc{cp}-tired \textsc{cp}-learn if-exist \textsc{inf}-pass=\textsc{dat} \textsc{past}-exist \\
    `Lazybones! If you had studied harder, you would have passed your A/L exam.' (K081114eli01)
\z
}


However, if the context is clear, counterfactuals can also be formed with the present tense. In  \xref{ex:func:conditionals:counterfactual:contr} it is clear that humans will not grow wings, and that we are dealing with a counterfactual condition.

\xbox{16}{
\ea\label{ex:func:conditionals:counterfactual:contr}
\gll Se=dang saayap kala-aada, bole=thìrbang. \\
     \textsc{1s=dat} wing if-exist can=fly  \\
    `If I had wings I could fly.' (K081114eli01)
\z
}

% Negative conditionals with \textsc{past}-tense verbal predicates have to be formed with a stacking of \em kal- \em and the relevant past tense form, as in \xref{ex:func:caus:biilang}
%
%
% \xbox{16}{
% \ea\label{ex:func:conditionals:past}
% \gll Lorang=nang duvith kal-thàrà-daapath, kithang anthi-banthu. \\
%      \textsc{2pl}=\textsc{dat} money if-\textsc{neg.past}-get \textsc{1pl} \textsc{irr}-help  \\
%     `Should you get no money, we would help.' (K081103eli04)
% \z
% \z
% } \\

Another possibility to express conditionals, which needs further research, is the use of the `undetermined' clitic on a past tense verb in the main clause, as given in \xref{ex:func:conditionals:so}.

\xbox{16}{
\ea\label{ex:func:conditionals:so}
\gll {\em Wicket}=ka su-kìnna=so, {\em out}. \\
     wicket=\textsc{loc} \textsc{past}-trike=\textsc{undet} out  \\
    `When the ball strikes the wicket, you are out.' (K081105eli02)
\z
}


% \xbox{16}{
% \ea
% \gll {\em wicket}=ka kalsukìnna, {\em out}. \\
%        \\
%     `.' (K081105eli02)
% \z
% } \\
%



%
% \subsubsection{Potentialis and Irrealis}\label{sec:func:PotentialisandIrrealis}
% Potentialis and irrealis are not distinguished,
%
% \xbox{16}{
% \ea\label{ex:func:unreferenced}
% \gll [Sirikaya mà-maasakh=dang]      kalu inni thullor  mau. \\
%  Wattalapam \textsc{inf}-cooked-\textsc{dat} if \textsc{prox} egg want\\
% `If you want to prepare Wattalapam, you have to use these eggs.' (B060115rcp02)
% \z
% }
%
%
% \xbox{16}{
% \ea\label{ex:func:unreferenced}
% \gll [ìnnam  thùllor arà-ambe]      kalu. \\
%       six egg \textsc{non.past}-take if \\
%     `If you take six eggs, ... .' (B060115rcp02)
% \z
% } \\
%
%
%
% \xbox{16}{
% \ea\label{ex:func:unreferenced}
% \gll [Kumpulang] kalu tuuju oorang. \\
%       party if seven man \\
%     `In case of a party [sacrifice], (you) (need) seven men.' (K060112nar01)
% \z
% } \\
%
%
%
%
%
% \xbox{16}{
% \ea\label{ex:func:unreferenced}
% \ea\label{ex:func:unreferenced}
% \gll {\em voting} m-ambel thàràbole. \\
% v. \textsc{inf}-take cannot\\
% `You cannot hold a vote.'
% \ex
% \gll [Lorang pada voting an-ambel]     kalu two {\em weeks} {\em notice} kaasi apa. \\
% \textsc{2pl} \textsc{pl} v. \textsc{past}-take if t. w. n. give after \\
%     `If you want to hold a vote, (that will be ) after (you) have given notice two weeks (before).' (K060116nar13)
% \z
% \z
% } \\
%
% Negated conditions can be formed by adding \em kalu \em to a negated clause, or by the contracted form \em kalthra\em.
%
% \xbox{16}{
% \ea\label{ex:func:unreferenced}
% \gll Minnyak      klaapa, gitthu    thraada  kalu, [...] matheega=ka    goreng \\
%      coconut.oil coconut, like.that  nexist if [...] ghee=\textsc{loc} fry \\
%     `(Fry them in) coconut oil. If that is not available, fry them in ghee.' (K060103rec01)
% \z
% } \\
%
%
%
% \xbox{16}{
% \ea\label{ex:func:unreferenced}
% \gll Inni=yang samma minnyak      klaapa  giithu   kalthra inni, aapa=yang, matheega=ka    goreng. \\
%      \textsc{prox}=\textsc{acc} with coconut.oil coconut like.that if.not this what=\textsc{acc} ghee=\textsc{loc} fry   \\
%     `Together with coconut oil, otherwise fry it in what, with ghee.' (K060103rec01)
% \z
% } \\

\section{Gradation}\label{sec:func:Gradation}
In SLM, gradation is done lexically, via   \trs{butthul}{correct} or \trs{bannyak}{a lot} \citep[67]{Saldin2001}, which are all given with their primary meaning here, although in the context of secondary modification, they would probably all rather be glossed as `very'. The following sentences give examples of these words first in their primary meaning and then used as modifiers of modifiers.

The word \em but(t)hul \em has `correct' as its primary meaning, this is shown in \xref{ex:func:ptcpt:mod:modmod:butthul:lit}. The intensive reading is shown in \xref{ex:func:ptcpt:mod:modmod:butthul:intens}. Note that the intensive reading in \xref{ex:func:ptcpt:mod:modmod:butthul:intens} has no geminated stop, which might be indicative of its being not a lexical, but a functional morpheme.

\xbox{16}{
\ea\label{ex:func:ptcpt:mod:modmod:butthul:lit}
\gll Lu=ppe nabi saapa katha \textbf{butthul} balas-an asà-biilang, ... \\
      \textsc{2s}=\textsc{poss} prophet who \textsc{quot} correct answer-\textsc{nmlzr} \textsc{cp}-say \\
    `Having given the correct answer as to who your prophet is, ...'
\z
}


\xbox{16}{
\ea\label{ex:func:ptcpt:mod:modmod:butthul:intens}
\gll Dee \textbf{buthul} jahhath. \\
      3\textsc{s.impolite} very wicked \\
    `He was very wicked.' (K051205nar02)
\z
}


% \xbox{16}{
% \ea\label{ex:func:ptcpt:mod:modmod}
% \gll Buthul konnyong=jo mulbar hatthu {\em period}. \\
%      very few=\textsc{emph} Tamil \textsc{indef} period  \\
%     `There was little Tamil at a time.' (K051222nar06)
% \z
% } \\


% \xbox{16}{
% \ea\label{ex:func:unreferenced}
% \gll Buthul eenak. \\
%       very tasty \\
%     `It is very tasty.' (K061026rcp02)
% \z
% } \\


The second intensifier is \trs{bannyak}{much}, whose literal meaning is given in \xref{ex:func:ptcpt:mod:modmod:bannyak:lit}, while the intensifier meaning is shown in \xref{ex:func:ptcpt:mod:modmod:bannyak:intens}.

\xbox{16}{
\ea\label{ex:func:ptcpt:mod:modmod:bannyak:lit}
\gll \textbf{Bannyak} Muslim oorang pada arà-duuduk. \\
     many Muslim man \textsc{pl} \textsc{non.past}-exist.\textsc{anim}  \\
    `There are many Muslims (in the Middle East).' (K061026prs01)
\z
}

\xbox{16}{
\ea\label{ex:func:ptcpt:mod:modmod:bannyak:intens}
\gll \textbf{Bannyak} thuuva oorang, nya-blaajar oorang. \\
     much old man \textsc{past}-learn man  \\
    `A very old man, an educated man.' (K060116nar07)
\z
}

For positive predications, \trs{baaye}{good} can also be used as an intensifier.


\xbox{16}{
\ea\label{ex:func:ptcpt:mod:modmod:baae:intens1}
\gll \textbf{Baaye} meera caaya kapang-jaadi, thurung-king. \\
     good red colour when-become descend-\textsc{caus} \\
    `When it turns into a nice red colour, remove it (from the fire).' (K060103rec02)
\z
} 

\xbox{16}{
\ea\label{ex:func:ptcpt:mod:modmod:baae:intens2}
\gll Dee arà-cuuri   \textbf{baaye} kaaya      oorang pada=dering. \\
      3\textsc{s.impolite} \textsc{non.past}-steal good rich man \textsc{pl}=\textsc{abl}   \\
`He steals from very rich people.'
\z
}

% The third intensifier is \trs{punnu}{full}, with the literal reading exemplified in \xref and the intensifier reading, in \xref.
%
% punnu   oorang pada anadhaathang
% K060108nar02.txt:\tx thee
% K060108nar02.txt:\tx thee daavong maa....       thaanàmnang

As for decreasing intensity, it appears that \em sdiikith \em or \em konnyong \em are used, both having `few' as their literal meaning. Unfortunately, there are no instances of \em konnyong \em and \em sdiikith \em used for modification of modification in the corpus, only for modification of an adjectival predicate. These are given here for reference.


\xbox{16}{
\ea\label{ex:func:ptcpt:mod:modmod:konnyong}
\gll \textbf{Konnyong} thàràsìggar    go=dang. \\
     few sick \textsc{1s=dat}  \\
    `I am a little sick.' (B060115nar04)
\z
}


\xbox{16}{
\ea\label{ex:func:ptcpt:mod:modmod:sdiikith}
\gll Se=ppe biini \textbf{sdiikith} thuuli. \\
      \textsc{1s=poss} wife few deaf \\
    `My wife is a little bit deaf.' (K070000wrt05)
\z
}


%
% \xbox{16}{
% \ea\label{ex:func:unreferenced}
% \gll [[[Ini      {\em British} government=samma pii=apa     mà-oomong] kithang=pe     {\em statesmen}  pada]=ka  Dr  Jaayah=le      bannyak kàthaama hathu  bergaada]   katha  bole=biilang. \\
%        \textsc{prox} British government=\textsc{comit} go=after \textsc{inf}-talk \textsc{1pl}=\textsc{poss} statesmen \textsc{pl}=\textsc{loc} Dr Jaayah much first \textsc{indef} ??? \textsc{quot} can-say \\
%     `You can say that Dr Jaayah was also a very prominent BERGAADA among our statesmen who had gon to negotiate with the British Government.' (N061031nar01)
% \z
% } \\
%


\section{Comparison}\label{sec:func:Comparison}
Within the domain of comparison, we can distinguish types with an overt indication of the standard of comparison (equation, similarity, superiority, inferiority, superlative) and types where the standard of comparison is not overtly indicated  (elative, abundantive, caritive). The aforementioned types typically refer to properties. An additional type, referring to processes, is correlative comparison of the type \em the more you X, the more Y\em. These types will now be discussed in turn.


\subsection{Equation}\label{sec:func:Equation}

Equation is formed by adding \em keejo \em to the second element of the equation. The first element is normally marked by \em =le\em. Neither of the elements of the equation receives special case marking.

% \xbox{16}{
% \ea
% \gll Se=ppe ruuma loram=pe\textbf{=kee=jo} bìssar. \\
%      \textsc{1s=poss} house \textsc{2pl}=\textsc{poss}=\textsc{simil}=\textsc{emph} big  \\
%     `My house is as big as yours.' (K081106eli01)
% \z
% } \\

\xbox{16}{
\ea
\gll Incayang*(=le) see=kee=jo kaaya. \\
     \textsc{3s.polite}=\textsc{addit} \textsc{1s}=\textsc{simil}=\textsc{emph} rich  \\
    `He is as wealthy as me.' (K081114eli01)
\z
}


\subsection{Similarity}\label{sec:func:mod:Similarity}
Similarity is expressed by the clitic \em =ke \em  \formref{sec:morph:=ke} on the item the term is similar to.

\xbox{16}{
\ea\label{ex:func:simil}
\gll Se=dang \textbf{baapa=ke} {\em soldier} mà-jaadi suuka. \\
     \textsc{1s=dat} father=\textsc{simil} soldier \textsc{inf}-become like  \\
    `I want to become a soldier like daddy.' (B060115prs10)
\z
}

\subsection{Superiority}\label{sec:func:Superiority}
Superiority is marked either by \trs{liivath}{more} or \trs{libbi}{remain}. The standard is marked with the dative.

\xbox{16}{
\ea
\gll Lorang  se=\textbf{dang} \textbf{libbi} kaaya. \\
     \textsc{2pl} \textsc{1s=dat} remain rich  \\
    `You are  more wealthy than me.' (K081114eli01)
\z
}

There appear to be some particular semantics with regard to the use of \em libbi \em and \em liivath\em. Depending on the adjective, one or the other might be better \xref{ex:func:comp:sup:libbiliivath:liivath} \xref{ex:func:comp:sup:libbiliivath:libbi}.


\xbox{16}{
\ea\label{ex:func:comp:sup:libbiliivath:liivath}
\gll Se=ppe ruuma loram=pe=nang liivath/*libbi bìssar. \\
     \textsc{1s=poss} house \textsc{2pl}=\textsc{poss}=\textsc{dat} more/remain big  \\
    `My house is bigger than yours.' (K081106eli01)
\z
}

\xbox{16}{
\ea\label{ex:func:comp:sup:libbiliivath:libbi}
\gll Se=ppe ruuma loram=pe=nang *liivath/libbi kiccil. \\
    \textsc{1s=poss} house \textsc{2pl}=\textsc{poss}=\textsc{dat} more/remain small  \\
    `My house is smaller than yours.' (K081106eli01)
\z
}

The above examples seem to point to a distinction between positive qualities, graded by \em liivath \em and negative qualities graded by \em libbi\em. However, things are more difficult, as the following examples show, where both \trs{mlaarath}{difficult} and \trs{gampang}{easy} are graded by \em liivath. \em This aspect is in need of further research.


\xbox{14}{
\ea
\gll {\em Japanese} {\em English}=nang liivath mlaarath. \\
     Japanese English=\textsc{dat} more difficult  \\
    `Japanese is more difficult than English.' (K081106eli01) %14
\z
}


\xbox{14}{
\ea
\gll {\em Japanese} {\em English}=nang liivath gampang. \\
     Japanese English=\textsc{dat} more easy  \\
    `Japanese is easier than English.' (K081106eli01) %14
\z
}



\subsection{Inferiority}\label{sec:func:Inferiority}
Inferiority is formed like superiority, but \trs{kuurang}{few, less} is added after the property word.

\xbox{16}{
\ea
\gll Lorang  se=dang libbi kaaya \textbf{kuurang}. \\
     \textsc{2pl} \textsc{1s=dat} remain rich few  \\
    `You are less wealthy than me.' (K081114eli01)
\z
}

\subsection{Superlative}\label{sec:func:Superlative}
The superlative is expressed by \em anà- \em \formref{sec:morph:anà-(Superlative)} \xref{ex:func:comp:superl:ana}, the emphatic clitic \em =jo \em \formref{sec:morph:=jo} \xref{ex:func:comp:superl:jo}, or a combination thereof \xref{ex:func:comp:superl:anajo}.

\xbox{16}{
 \ea\label{ex:func:comp:superl:ana}
\gll Bill Gates duniya=ka anà-kaaya oorang. \\
    Bill Gates world=\textsc{loc} \textsc{superl}-rich man   \\
    `Bill Gates is the richest man in the world.' (K081103eli04)
\z
}

\xbox{16}{
\ea\label{ex:func:comp:superl:jo}
\gll Anjing oorang=pe baae=jo thumman. \\
      dog man=\textsc{poss} good=\textsc{emph} friend \\
    `The dog is man's best friend.' (K081104eli06)
\z
}


% \xbox{16}{
%  \ea\label{ex:func:comp:superl:jo}
% \gll Seelon=ka bìssar=jo  pohong. \\ % bf
%      Ceylon=\textsc{loc} big=\textsc{emph} tree  \\
%     `The biggest tree in Sri Lanka.'   (K081103eli04)
% \z
% }\\


\xbox{16}{
 \ea\label{ex:func:comp:superl:anajo}
   \gll Anà-muuda=jo       anak  klaaki. \\
    \textsc{superl}-young=\textsc{emph} child male \\
`The youngest one is a son' (K060108nar02,K081103eli04)
\z
}



%
%
% \xbox{16}{
% \ea
% \gll Seelonka anà bìssar pohong. \\
%        \\
%     `.' (K081104eli06)
% \z
% } \\





% \xbox{16}{
% \ea
% \gll Duniyaka anà-kaaya=jo oorang Bill Gates. \\
%        \\
%     `.' (K081103eli04)
% \z
% } \\
%
% \xbox{16}{
% \ea
% \gll Bill Gatesjo duniyaka anà-kaayajo oorang. \\
%        \\
%     `.'  (K081103eli04)
% \z
% } \\
%
%
% \xbox{16}{
% \ea
% \gll Se=ppe ruuma subla oorangjo Kandika anà-miskin=jo. \\
%        \\
%     `.' (K081103eli04)
% \z
% } \\

%
% \xbox{14}{
% \ea
% \gll Bannyak kìrras anà/ara laari oorang pompangjo Susanthika. \\
%        \\
%     `.' (nosource)
% \z
% } \\

\subsection{Elative}\label{sec:func:Elative}
The elative is expressed by \em =jo \em \formref{sec:morph:=jo}.

\xbox{16}{
\ea \label{ex:jo:elative}
\gll Hatthu komplok \textbf{bannyak=jo} puuthi caaya. \\
     one bush very=\textsc{emph} white colour \\
    `One bush was very, very white.'  (K070000wrt04)
\z
}

\subsection{Abundantive}\label{sec:func:Abundantive}
There is no special form for abundantive. The normal intensifiers \em bannyak \em or \em buthul \em can be used.

\xbox{16}{
\ea
\gll Incayang buthul gummuk. \\
     \textsc{3s.polite} correct fat  \\
    `He is too/very fat.' (K081114eli01)
\z
}

The excess can be additionally marked by \trs{liivath}{more}, but this is optional.

\xbox{16}{
\ea
\gll Itthu {\em bag}=yang lorang=nang bannyak bìrrath (\textbf{liivath}). \\
     \textsc{dist} bag=\textsc{acc} \textsc{2pl}=\textsc{dat} much heavy more  \\
    `That bag is too heavy for you.' (K081114eli01)
\z
}


\subsection{Caritive}\label{sec:func:Insufficientive}
To indicate that the degree is not sufficient as compared to a non-expressed standard, \trs{thàràsampe}{does not reach} is used. \em Thàràsampe \em can be used with nouns or with adjectives \xref{ex:func:comp:insuf:kaake}.

%
% \xbox{16}{
% \ea
% \gll Itthu gaaji thàràsampe. \\
%       \textsc{dist} salary insufficient \\
%     `That salary is not enough.' (K081114eli01)
% \z
% } \\
%
%
%
% \xbox{16}{
% \ea
% \gll Ini pliitha=pe thìrrang thàràsampe. \\
%     \textsc{prox} lamp=\textsc{poss} light insufficient   \\
%     `This lamp is not bright enough.' (nosource)
% \z
% } \\
%
%
% \xbox{16}{
% \ea
% \gll Ini kameeja=pe/=ka puuthi thàràsampe. \\
%      \textsc{prox} shirt=\textsc{poss}/=\textsc{loc} white insufficient  \\
%     `This shirt is not white enough.' (nosource)
% \z
% } \\


\xbox{16}{
\ea\label{ex:func:comp:insuf:kaake}
\gll Se=ppe kaake=nang sìggar/sìgaran thàràsampe. \\
     \textsc{1s=poss} grandfather=\textsc{dat} healthy/health insufficient  \\
    `My grandfather is not well enough.' (K081114eli01)
\z
}

The purpose for which the degree is insufficient can be indicated by an infinitive clause.

\xbox{16}{
\ea\label{ex:func:comp:sufficient}
\gll \textbf{Pohong} \textbf{mà-seereth=nang} oorang thàrà-sampe. \\
     tree \textsc{inf}-drag=\textsc{dat} man \textsc{neg.past}-reach  \\
    `There were not enough men to drag the tree.' (K051205nar05)
\z
}

For adjectives, the normal negation can be used, complemented by \trs{punnu}{full} and an intensifier.

\xbox{16}{
\ea
\gll Ini makanan \textbf{bannyak/giithu} \textbf{punnu} pùddas \textbf{thraa}. \\
      \textsc{prox} food much/like.that full spicy \textsc{neg} \\
    `This food is not spicy enough.' (K081114eli01)
\z
}

\subsection{Correlative}\label{sec:func:Correlative}
Correlative comparison is formed by a reduplicated infinitive.


\xbox{14}{
\ea
\gll Caabe \textbf{mà-thaaro} \textbf{mà-thaaro}, pùddas liivath. \\
      chili \textsc{inf}-put \textsc{inf}-put spicy more \\
    `The more chili you put, the spicier (the food becomes).' (K081114eli01)
\z
}


\xbox{14}{
\ea
\gll Haari \textbf{mà-pii} \textbf{mà-pii}, laa\u mbath liivath. \\
     day \textsc{inf}-go \textsc{inf}-go delay more  \\
    `Days pass by and the delay is caused/The more days pass, the more delay is caused.' (K081114eli01)
\z
}

% \xbox{16}{
% \ea\label{ex:constr:pred:unreferenced}
% \gll Caabe mathaaro mathaarojo, pùddas arà liivath. \\
%        \\
%     `.'  (K081104eli06)
% \z
% }\\








%
% \xbox{16}{
% \ea
% \gll ?lorang se=dang liivath kaaya. \\
%        \\
%     `.' (K081114eli01) sociolinguistics
% \z
% } \\
%
%
% \xbox{16}{
% \ea
% \gll ?lorang se=dang liivath kaaya kurang. \\
%        \\
%     `.' (K081114eli01)
% \z
% } \\






\section{Possession}\label{sec:func:Possession}
Within the realm of possession, we can distinguish three different constellations (assertions in boldface): either the possessee is asserted (\em I have \textbf{a car}\em), or the possessor is asserted (\em the car is \textbf{mine}\em), or they are both part of the presupposition (\em My car \textbf{broke down}\em). Furthermore, permanent and temporary possession can be distinguished, which is relevant for SLM, as are differences in animacy, while differences in alienability are not relevant.

There is no verb meaning `to have', in line with general South Asian \citep[166]{Masica1976}  and Austronesian typology \citep[139]{Himmelmann2005typochar}. There is a verb \trs{puunya}{possess}, but this is hardly ever used.\footnote{The possessive marker \em =pe \em has grammaticalized from \em pu(u)nya \em and is frequent.}

\subsection{Assertion of the possessee}\label{sec:func:Assertionofthepossessee}
The assertion of the possessee's being in a possessive relationship with the possessor is done with a combination of the postpositions \em =nang  \em \formref{sec:morph:=nang},  \em =ka \em  \formref{sec:morph:=ka} or \em =samma \em \formref{sec:morph:=sesaama} with the existentials \em aada \em \formref{sec:wc:Existentialverbs:aada} or \em duuduk \em \formref{sec:wc:Existentialverbs:duuduk}. \em =nang \em is used for permanent possession, while \em =ka \em is used for temporary possession. \em Duuduk \em can only be used for animate possessees (which are almost always kin), while \em aada \em can be used for any possessee. \em =samma \em is only used for temporary possession of animates. The following sections show various combinations of these parameters.

\begin{center}
% use packages: array
\begin{tabular}{lccc}
 	  & animate & inanimate & abstract \\
\hline
permanent & \parbox{5cm}{\vspace{.2cm} =nang +duuduk \xref{ex:poss:perm:anim:duuduk1} \xref{ex:poss:perm:anim:duuduk2}\\=nang + aada \xref{ex:poss:perm:anim:aada}}
			& =nang + aada \xref{ex:poss:perm:inanim:aada}\xref{ex:poss:perm:inanim:aada2}
					& \multirow{3}{*}{=nang + aada \xref{ex:poss:abstr:aada}}\\
\cline{1-3}
temporary & =samma+duuduk  \xref{ex:poss:temp:anim}
			& =ka + aada \xref{ex:poss:temp:inanim:aada:ka}\xref{ex:poss:temp:inanim:aada:ka2}
					&  
\end{tabular}
\end{center}

\subsection[Permanent \& animate]{Permanent possession of animates}\label{sec:func:Permanentpossessionofanimates}
Possessed animates like kin are always construed with the dative marker \em =nang \em on the possessor. The existential can be either \em duuduk \em or \em aada\em.

% \xbox{16}{
% \ea
% \gll Thiiga klaaki aade=le hatthu pompang aade=le \textbf{se=dang} arà-\textbf{duuduk}. \\
%      three male younger.sibling=\textsc{addit} one female younger.sibling \textsc{1s=dat} \textsc{non.past}-exist.\textsc{anim}  \\
%     `I have three younger sisters and one younger brother.' (K060108nar01)
% \z
% } \\


\xbox{16}{
\ea\label{ex:poss:perm:anim:duuduk1}
\gll Se=dang hathu maven arà-\textbf{duuduk}. \\
      \textsc{1s=dat} \textsc{indef} son \textsc{non.past}-exist.\textsc{anim} \\
    `I have a son.' (B060115prs05)
\z
}

\xbox{16}{
\ea\label{ex:poss:perm:anim:duuduk2}
\gll Se=dang duuva pompang aade=le hathu klaaki aade=le anà-\textbf{duuduk}. \\
     \textsc{1s=dat} two female younger.sibling=\textsc{addit} one male younger.sibling=\textsc{addit} \textsc{past}-exist.\textsc{anim}  \\
    `I had two younger sisters and a younger brother.'
\z
}

\xbox{16}{
\ea \label{ex:poss:perm:anim:aada}
\gll \textbf{Se=dang} liima anak  klaaki pada \textbf{aada}. \\
      \textsc{1s=dat} five child male \textsc{pl} exist \\
    `I have five sons.' (K060108nar02)
\z
}


\subsection[Temporary \& animate]{Temporary possession of animates}\label{sec:func:Temporarypossessionofanimates}
The temporary possession of animates uses \em duuduk \em as expected, but instead of the locative \em =ka\em, the comitative \em =samma \em is used. 

\xbox{16}{
\ea \label{ex:poss:temp:anim}
\gll Maaling=samma moonyeth hatthu arà-duuduk. \\
    thief=\textsc{comit} monkey \textsc{indef}  \textsc{non.past}-exist.\textsc{anim}   \\
    `The thief has a monkey (with him).' (K090327eml01)
\z
}

\subsection[Permanent \& inanimate]{Permanent possession of inanimates}\label{sec:func:Permanentpossessionofinanimates}
Inanimates can never be construed with \em duuduk\em. \em Aada \em always has to be used.

\xbox{16}{
\ea \label{ex:poss:perm:inanim:aada}
\gll Mr. Yusuf, karang, incayang=\textbf{nang} [ini {\em private} {\em bank}=ka aada duvith pada] \textbf{aada} \\
      Mr. Yusuf now \textsc{3s.polite}=\textsc{dat} \textsc{prox} private bank=\textsc{loc} exist money \textsc{pl} exist \\
    `Mr. Yusuf owned the money which was deposited in this private bank.' (K060116nar09)
\z
}


% \xbox{16}{
% \ea\label{ex:func:unreferenced}
% \gll Seelong=le     kithang=pe     mlaayu=nang=le        hathu  bagiyan an-aada
%   \\
%        \\
%     `.' (K051222nar04)
% \z
% } \\


\xbox{16}{
\ea \label{ex:poss:perm:inanim:aada2}
\gll Se=dang hatthu ruuma aada. \\
     \textsc{1s=dat} \textsc{indef} house exist  \\
    `I have a house.' (K081114eli01)
\z
}


\subsection[Temporary \& inanimate]{Temporary possession of inanimates}\label{sec:func:Temporarypossessionofinanimates}
Temporary possession of inanimates is construed with the locative \em =ka \em and the inanimate existential \em aada\em.

\xbox{16}{
\ea \label{ex:poss:temp:inanim:aada:ka}
\gll Incayang=\textbf{ka} ... bìssar beecek caaya hathu {\em bag} su-\textbf{aada}. \\
     \textsc{3s.polite}=\textsc{loc} ... big mud colour \textsc{indef} bag \textsc{past}-exist  \\
    `He had a big brown bag with him.' (K070000wrt04)
\z
}


\xbox{16}{
\ea \label{ex:poss:temp:inanim:aada:ka2}
\gll See=\textbf{ka}=jo bannyak ini panthong \zero{}. \\
     \textsc{1s}=\textsc{loc}=\textsc{emph} much \textsc{prox} song  \\
    `I possess a lot of these songs [on sheets].' (K060116nar04)
\z
}

% B060115nar04.txt: karang itthu    textile       goka     aada

\subsection{Possession of abstract concepts}\label{sec:func:Possessionofabstractconcepts}
Abstract concepts can never be construed with \em =ka \em or \em duuduk\em. \em =nang \em and \em aada \em have to be used.

\xbox{16}{
\ea \label{ex:poss:abstr:aada}
\gll Se=\textbf{dang} bannyak creeveth pada su-\textbf{aada}. \\
     \textsc{1s=dat} lot trouble \textsc{pl} \textsc{past}-exist  \\
    `I had a lot of trouble.' (K051213nar01)
\z
}

\subsection{Assertion of the possessor}\label{sec:func:Assertionofthepossessor}
If the possessee is established, but the possessor not, an equational construction  \formref{sec:pred:Equationalpredicate} is used \citep[cf.][104f]{Hengeveld1992nvpttd}. This construction assigns the possessum to the class of items possessed by the possessor. This class is indicated by marking the possessor with the possessive postposition \em =pe\em.

% \xbox{16}{
% \ea\label{ex:func:unreferenced}
% \gll Ini ankel=\textbf{pe}? \\
%  \textsc{prox} uncle=poss\\
% `Is this the uncle's?' (nosource)
% \z
% }\\


\xbox{16}{
\ea\label{ex:func:poss:asspr2}
\gll Se=ppe    saayang jiiva biilang [\textbf{se=ppe}] katha. \\
     \textsc{1s=poss} love life say \textsc{1s=poss} \textsc{quot}  \\
    `Love of my life, say that you are mine.' (K061123sng03)
\z
}

\xbox{16}{
\ea\label{ex:func:poss:asspr1}
\gll Itthu    muusing bannyak {\em teacher} pada [\textbf{Jaapna=pe}]. \\
      \textsc{dist} time many teacher \textsc{pl} Jaffna=\textsc{poss} \\
    `Back then, many teachers were from Jaffna.' (K051213nar03)
\z
}

\subsection{Presupposition of the possessor and the possessee}\label{sec:func:Presuppositionofthepossessorandthepossessee}
If the possessive relation is presupposed, the possessive postposition \em =pe \em  \formref{sec:morph:=pe} is used \xref{poss:presup:n}. If the possessor is expressed by a monosyllabic pronoun, \em =ppe \em is used \xref{poss:presup:pron12}. Other pronouns take the normal form \xref{poss:presup:pron}.


\xbox{16}{
\ea \label{poss:presup:n}
\gll Kithang=pe \textbf{baapa=pe} naama Mahamud. \\
       \textsc{1pl}=\textsc{poss} father=\textsc{poss} name Mahamud\\
    `Our father's name is Mahamud.' (B060115nar03)
\z
}


\xbox{16}{
\ea \label{poss:presup:pron12}
\gll Lu=\textbf{ppe} muuluth=ka=le paasir, se=\textbf{ppe} muuluth=ka=le paasir. \\
      \textsc{2s.familiar}=\textsc{poss} mouth=\textsc{loc}=\textsc{addit} sand \textsc{1s=poss} mouth=\textsc{loc}=\textsc{addit} sand    \\
    `There is sand in your mouth and there is sand in my mouth.'   (K070000wrt02)
\z
}

\xbox{16}{
\ea \label{poss:presup:pron}
\gll \textbf{Kitham=pe}     ruuma dìkkath=ka. \\
      \textsc{1pl}=\textsc{poss} house vicinity=\textsc{loc}  \\
    `Close to our house.' (K051220nar01)
\z
}
%
% \section{Questions}\label{sec:func:Questions}
% Question are distinguished from declarative sentences by their lack of assertive power. Question can be subdivided into polar or yes-no questions \funcref{sec:func:Yes-noquestions}, alternative questions \funcref{sec:func:Alternativequestions} and content or WH-questions \funcref{sec:func:Contentquestions}.
%
% \subsection{Yes-no questions}\label{sec:func:Yes-noquestions}
% SLM Yes-no-questions are formed by adding the interrogative clitic \em=si \em to the questioned element. This is accompanied by rising intonation. If it is the predicate that is questioned, then the interrogative particle is sometimes left out and the question is solely indicated by intonation. See \formref{sec:phon:Intonation} for some intonation curves. If some constituent is questioned \em =si \em cannot be left out.
%
% It is possible to question the predicate as in \xref{ex:func:quest:si:pred}, or a constituent as in \xref{ex:func:quest:si:const}.
%
% \xbox{16}{
% \ea \label{ex:func:quest:si:pred}
% \gll Se=pe uumur masà-\textbf{biilan=si}? \\
%  1=\textsc{poss} age must-tell=\textsc{interr}\\
% `Do I have to tell my age?' (B060115prs01)
% \z
% }
%
% \xbox{16}{
% \ea \label{ex:func:quest:si:const}
% \gll Saapa? \textbf{se=si}? \\
%  who \textsc{1s}=\textsc{interr}\\
% `Who? Me?' (B06015prs18)
% \z
% }
%
% \subsection{Alternative questions}\label{sec:func:Alternativequestions}
% are formed like yes-no-questions, but with \em =si \em attached to all the alternative elements.
%
%
% \xbox{16}{
% \ea
% \gll Piisang\textbf{=si} maa\u n\u gga\textbf{=si} maau? \\
%      banana=\textsc{interr} mango=\textsc{interr} want  \\
%     `Is it plantain or mango that you want?'  (K081105eli02)
% \z
% }\\
%
%
%
% %
% % \section{Commands}\label{sec:func:Commands}
% % % Commands do not convey propositional content to the hearer, but invite him to (not) perform  a certain action. Commands are formed by imperative clauses. See the discussion there. Additionally, it is common to formulate requests as declaratives with the modal \trs{boole}{can}. `You can also come later' the means `please come later.' There are no examples of this pattern in the corpus, but I found it very frequently in personal interaction.
% % % Commands are formed by using the imperative construction, which consist of the bare verb \formref{sec:form:ImperativeClause}.
% % %
% % % \xbox{16}{
% % % \ea\label{ex:func:commands:pos:zero}
% % % \gll Aajuth thaakuth=ka su-naangis, ``See=yang luppas''. \\
% % %      dwarf fear=\textsc{loc} \textsc{past}-cry \textsc{1s}=\textsc{acc} leave  \\
% % %     `The dwarf screamed in fear: ``Leave me!'' '  (K070000wrt04)
% % % \z
% % % }\\
% % %
% % % An optional enclitic \em =la \em can be added to the verb. This is considered more polite.
% % %
% % % \xbox{16}{
% % % \ea\label{ex:func:commands:pos:la}
% % % \gll Allah, diyath-\textbf{la} inni pompang pada dhaathang aada. \\
% % %       Allah see-\textsc{imp} \textsc{dist} female \textsc{pl} come exist \\
% % %     Oh, see that woman has come.' (K061019nar02)
% % % \z
% % % } \\
% % %
% % %
% % % Another possibility is to use the preverbal particle \trs{mari}{come}.
% % %
% % % \xbox{16}{
% % % \ea\label{ex:func:commands:pos:mari}
% % % \gll \textbf{Mari} maakang. \\
% % %  come eat\\
% % % `Eat!' (B060115rcp02)
% % % \z
% % % }
% % %
% % % If the command is about coming, then \em ma{sec:func:Contentquestions}ri \em can be used alone.
% % %
% % % \xbox{16}{
% % % \ea\label{ex:func:commands:pos:maricome}
% % % \gll Laskalli \textbf{mari}. \\
% % %  other time come\\
% % % `Come again!' (G051222nar01)
% % % \z
% % % }
% % %
% % % \em mari \em and \em -la \em can be combined.
% % %
% % % \xbox{16}{
% % % \ea\label{ex:func:commands:pos:marila}
% % % \ea
% % % \gll Saayang se=ppe thuan \textbf{mari} laari-\textbf{la}. \\
% % %       love \textsc{1s=poss} sir come.imp run-\textsc{imp} \\
% % %     `Come my beloved gentleman, come here.'
% % % \ex
% % % \gll See=samma kumpul \textbf{mari} thaa\u ndak-\textbf{la}. \\
% % %      \textsc{1s}=\textsc{comit} gather come.imp dance-\textsc{imp}  \\
% % %     `Come and dance with me.' (N061124sng01)
% % % \z
% % % \z
% % % } \\
% % %
% % % Suggestions can be formed by using the imperative with \em mari \em followed by the interrogative clitic \em =si\em. \em=la \em is not possible in this context.
% %
% %
% % \xbox{16}{
% % \ea
% % \gll Marà-maakang. \\
% %        \\
% %     `Let's eat.' (K081106eli01)
% % \z
% % } \\
% %
% %
% % \xbox{16}{
% % \ea
% % \gll Kithang mara maakang si. \\
% %        \\
% %     `Shall we eat.' (K081106eli01)
% % \z
% % } \\
% %
% %
% % \xbox{16}{
% % \ea
% % \gll See ma maakang si. \\
% %        \\
% %     `Shall I eat' (K081106eli01)
% % \z
% % } \\
%
% Prohibtions can be formed with \em jamà- \em \formref{ex:func:commands:neg:jama} or with \em thussa \em \formref{ex:func:commands:neg:thussa}.
%
% % \xbox{16}{
% % \ea\label{ex:func:commands:neg:jama}
% % \gll Biilang, maalu, \textbf{jamà-maalu}. \\
% %      say shy \textsc{neg.nonfin} shy  \\
% %     `Speak! You are shy, don't be shy.' (B060115prs07)
% % \z
% % } \\
%
% \xbox{6}{
% \ea\label{ex:func:commands:neg:jama}
% \gll See=yang jamà-liiyath. \\
%      \textsc{1s}=\textsc{acc} \textsc{neg.imp}-look  \\
%     `Don't stare at me!' (K081106eli01)
% \z
% }
% \xbox{6}{
% \ea\label{ex:func:commands:neg:thussa}
% \gll Se=yang \textbf{thussa} mà-liiyath. \\
%      \textsc{ \textsc{1s}=acc} \textsc{neg}.want \textsc{inf}-look\\
%     `Don't stare at me!'  (K081106eli01)
% \z1
% }


\section{Negation}\label{sec:func:Negation}
Negation gives negative truth value to a proposition.
In SLM negation patterns vary with predication type, clause type, tense,  and information structure. Every different type of predicate (verbal, existential, modal, nominal, adjectival, circumstantial) has a corresponding negation pattern. The verbal predicate has an additional negation type used in subordinate clauses. In the domain of tense, we can distinguish past, perfect, non-past and future as relevant tense domains. As for information structure, predicate negation is different from constituent negation. Table \ref{tab:func:negation} gives an overview of the different negation patterns. Please see section \ref{sec:pred} and the relevant subsections for more discussion and examples.

\begin{table}
\begin{center}
% use packages: array
\begin{tabular}{r|c|c|c|c|}
 & past & perfect & present & future \\\hline
\hline
\textit{predicate negation} & \multicolumn{4}{|c|}{~}  \\\hline
finite verbal clause & thàrà-V & V thraa &  \multicolumn{2}{|c|}{thama-V}  \\\hline
infinite verbal clause &  \multicolumn{4}{|c|}{jamà}  \\\hline
existential & \multicolumn{4}{|c|}{thraa}  \\\hline
nominal &  \multicolumn{3}{|c|}{bukang} & thama-jaadi/bukang \\\hline
adjectival1 &  \multicolumn{3}{|c|}{ADJ thraa} & thama-ADJ \\\hline
adjectival2 &  \multicolumn{3}{|c|}{thàrà-ADJ} & thama-ADJ \\\hline
circumstantial &  \multicolumn{3}{|c|}{bukang} &  \\\hline
locational & \multicolumn{4}{|c|}{thraa\footnotemark}  \\\hline
animate locational & thàràduuduk & duuduk thraa & \multicolumn{2}{|c|}{thama-duuduk}\\\hline
~(ka)maau(van) & \multicolumn{4}{|c|}{thàrkamauvan/thussa}  \\\hline
~boole & \multicolumn{4}{|c|}{thàrboole}   \\ \hline
\hline
\textit{constituent negation} &  \multicolumn{4}{|c|}{bukang} \\
\hline
\end{tabular}
\end{center}
\caption[Negation patterns for various predicate types and tenses]{Negation patterns for various predicate types and tenses.}
\label{tab:func:negation}
\end{table}
\footnotetext{This suppletive negation of the existential is an exception to the generalization formulated by \citet[138]{Himmelmann2005typochar}, that the existential is negated by the common verbal negator.}

Investigating Table \ref{tab:func:negation}, we observe a number of generalizations. For instance, negation of circumstantial predication and constituent negation are both done by \em bukang\em. This could point to circumstantial predicates not being predicates in their own rights, but rather constituents with a `real' predicate which is not overtly realized.

Another item which occurs frequently is \em thraa\em, which is used for perfect tense negation of verbs, negation of locational predicates and negation of adjectives. The first two of this can be explained by the presence of \em aada \em in the affirmative counterpart. Since \em thraa \em is the negative form of \em aada\em, the occurrence of \em thraa \em in perfect and locational predicates where \em aada \em is used in the affirmative is not surprising. The negation of adjectives by \em thraa \em cannot be explained in this manner. Note that speakers differ as to the negation patterns of adjectives. Depending on speaker and individual lexeme, the pattern with \em thraa \em or the pattern with \em thàrà-	 \em can be found.

When \em thàrà- \em is used to negate adjectives, it can be used in both past and present tenses, while in its other use for negating verbs, it necessarily has past tense reference.

For negation of non-verbal predications referring to the future, some special periphrases exist. All these involve the non-past negative marker \em thama- \em and an extra verb, like \trs{jaadi}{become}.

Conjunctive participle clauses and purposive clauses are marked by \em jamà- \em when negated. Other subordinate clauses like relative clauses or argument clauses take the same negation as main clauses.

Indicating that none of the possible referents would yield a positive truth value (\em no student came\em) and indicating that for all referents, the truth value is negative (\em The students did not come\em) is done in a very similar fashion in SLM. The only difference between the two is that in the former case the indefinite article \em atthu \em is used before  the noun and a suitable coordinating clitic like \em =le\em, \em =ke \em or \em =pon \em is used after the NP.

\xbox{16}{
 \ea\label{ex:func:neg:le1}
\gll Kithang=pe \textbf{hatthu} oorang=\textbf{le}      {\em minister} jaadi  \textbf{thraa}. \\
        \textsc{1pl}=\textsc{poss} \textsc{indef} man=\textsc{addit} minister become \textsc{neg}\\
    	`Not one of our has (ever) become a minister.' (N061031nar01)
\z
}

\xbox{16}{
 \ea\label{ex:func:neg:le3}
   \gll Derang=nang   Kluu\u mbu=pe    samma {\em association}=le      {\em support}; Kluu\u mbu=pe    \textbf{hatthu} {\em association}=\textbf{le}   kithang=nang   \textbf{thàrà}-{\em support}. \\
    \textsc{3pl}=\textsc{dat} Colombo=\textsc{poss} all association=\textsc{addit} support Colombo=\textsc{poss} \textsc{indef} association=\textsc{addit} \textsc{1pl}=\textsc{dat} \textsc{neg.past}-support\\
`All Colombo associations supported them, not one association from Colombo supported us' (K060116nar06)
\z
}



\xbox{16}{
 \ea\label{ex:func:neg:pon}
\gll Kithang \textbf{hatthu}=oorang=\textbf{pon} \textbf{thàrà}-iingath. \\
       \textsc{1pl} \textsc{indef}=man=any \textsc{neg.past}-think\\
    `We cannot think of any person.'  (B060115nar02)
\z
}

Rarely, the indefinite article and the clitic are both found after then noun.

\xbox{16}{
 \ea\label{ex:func:neg:pon:inv}
\gll See pukaran=\textbf{hatthu=pon} \textbf{thama}=gijja, ruuma=ka arà-duuduk. \\
     \textsc{1s} work=\textsc{indef}=any \textsc{neg.nonpast}-make house=\textsc{loc} \textsc{non.past}-stay \\
    `I don't do any work, I stay at home.'  (B060115prs03)
\z
}

The clitic attaches after the postposition, as can be seen from the following two examples, where the postposition \em =nang \em intervenes between the noun \trs{oorang}{man} and the clitic \em =le\em.

\xbox{16}{
 \ea\label{ex:func:neg:postp1}
\gll Incayang=pe      muusing=ka kithang=pe     {\em Malays} pada \textbf{atthu} \textbf{oorang=nang=le}        [{\em parliament}=nang  mà-dhaathang=nang      thumpath] \textbf{thàrà}-daapath. \\
   \textsc{3s.polite}=\textsc{poss} time=\textsc{loc} \textsc{1pl}=\textsc{poss} Malays \textsc{pl} \textsc{indef} man=\textsc{dat}=\textsc{addit} parliament=\textsc{dat} \textsc{inf}-come=\textsc{dat} place \textsc{neg.past}.get  \\
    `During his time, no man of our Malays got a place to go to parliament (i.e. a seat).' (N061031nar01)
\z
}

\xbox{16}{
 \ea\label{ex:func:neg:postp2}
\gll Incayang=le       kithang=pe     mlaayu pada \textbf{hatthu} oorang\textbf{=nang=le} thumpath \textbf{thàrà}-kaasi. \\
     \textsc{3s.polite}=\textsc{addit} \textsc{1pl}=\textsc{poss} Malay \textsc{pl} \textsc{indef} man=\textsc{dat}=\textsc{addit} place \textsc{neg.past}-give  \\
    `He did not not give any position to any of our Malays either.' (N061031nar01)
\z
}

It is also possible to use the combination of \em atthu \em and a clitic without a noun. In this case, the clitic \em =ke \em was also found in the corpus, next to \em =le \em and \em =pon\em.


\xbox{16}{
 \ea\label{ex:func:neg:zeronoun:le}
\gll Laayeng   \textbf{hatthu=\zero=le}      thraa. \\
      different \textsc{indef}=\textsc{addit} \textsc{neg} \\
    `There is nothing else.' (B060115nar04)
\z
}


\xbox{16}{
 \ea\label{ex:func:neg:zeronoun:pon}
\gll {\em Bus}=ka \textbf{hatthu=\zero=pon} mà-kirja thàràboole. \\
      bus=\textsc{loc} \textsc{indef}=\textsc{neg} \textsc{inf}-make cannot \\
    `You can't do anything on the bus.' (K061125nar01)
\z
}


\xbox{16}{
 \ea\label{ex:func:neg:zeronoun:ke}
\gll Snow-white=nang=le Rose-red=nang=le ini \textbf{hatthu=\zero=ke} \textbf{thàrà}-mirthi. \\
     Snow.white=\textsc{dat}=\textsc{addit} Rose.Red=\textsc{dat}=\textsc{addit}  \textsc{prox} \textsc{indef}=\textsc{simil} \textsc{neg.past}-understand\\
    `Snow White and Rose Red did not understand a thing.'  (K070000wrt04)
\z
}
%  luwar nigiri nang  pii thraanang arà-duuduk=yang tharabaae 03110201

\section{Kin}\label{sec:func:Kin}
The family relations of people show differences in their encoding between the languages of the world.
The SLM kinship system distinguishes between males and females, generations, and relative age. Siblings and cousins are not distinguished, but there are many different types of uncles and aunts, depending on whether they are male or female, maternal or paternal, elder or younger. Parents' elder siblings and their spouses are only distinguished for sex. Parents' younger siblings are also distinguished for sex of the parent, and spouses are different from consanguineous relatives.  Grandparents and grandchildren are only distinguished by sex, while greatgrandparents and greatgrandchildren are not distinguished by sex either, but the word for greatgrandparents includes the morpheme for the grandparents.

The terms \trs{maven}{son, nephew, younger man} and \trs{mavol}{daughter, niece} can also be heard, but it was not clear to which individuals they could refer. When addressing older members of the community, younger speakers use \em uncle, auntie \em (English) or \trs{kaake, neene}{grandfather, grandmother}, depending on the age. When elder speakers address younger ones, they can use \em maven \em or \em mavol\em. Among peers, the sibling terms \trs{kaaka}{elder male}, \trs{dhaatha}{elder female} or \trs{aade}{younger sibling} are used. These follow the proper name, so \trs{Imi kaaka}{Elder brother Imi}.


\begin{sidewaysfigure}
% \begin{figure}
 \centering

\begin{tabular}{c@{\hspace{-0.2cm}}c@{\hspace{-0.2cm}}c@{\hspace{-0.2cm}}c@{\hspace{-0.2cm}}c@{\hspace{-0.2cm}}c@{\hspace{-0.2cm}}c@{\hspace{-0.2cm}}c@{\hspace{-0.2cm}}c@{\hspace{-0.2cm}}}
&
\multicolumn{3}{c}{\pile{0.8}{}{\dreieck}{moyang\\\footnotesize kaake}\hspace{-0.3cm}=\hspace{-0.3cm}\pile{0.8}{}{\kreis}{moyang\\\footnotesize neene}}  &
&
&
\multicolumn{3}{l}{\pile{0.8}{}{\dreieck}{moyang\\\footnotesize kaake}\hspace{-0.3cm}=\hspace{-0.3cm}\pile{0.8}{}{\kreis}{moyang\\\footnotesize neene}}
\\
 &
\pile{0.8}{}{\dreieck}{kaake}\hspace{-0.3cm}=\hspace{-0.3cm}\pile{0.8}{$\mid$\vln{-1}{8}{105}}{\kreis}{neene} &
\pile{0.8}{$\mid$}{\dreieck}{kaake} &
\pile{0.8}{$\mid$}{\dreieck}{kaake}\hspace{-0.3cm}=\hspace{-0.3cm}\pile{0.8}{}{\kreis}{neene} &
 &
\pile{0.8}{}{\dreieck}{kaake}\hspace{-0.3cm}=\hspace{-0.3cm}\pile{0.8}{$\mid$\vln{-1}{8}{109}}{\kreis}{neene} &
\pile{0.8}{$\mid$}{\kreis}{neene} &
\pile{0.8}{$\mid$}{\dreieck}{kaake}\hspace{-0.3cm}=\hspace{-0.3cm}\pile{0.8}{}{\kreis}{neene} &
\\\\
\pile{0.8}{$\mid$\vln{-1}{8}{264}}{\dreieck}{muuda}\hspace{-0.3cm}=\hspace{-0.3cm}\pile{0.8}{}{\kreis}{maami} &
\pile{0.8}{$\mid$}{\kreis}{maami}\hspace{-0.3cm}=\hspace{-0.3cm}\pile{0.8}{}{\dreieck}{muuda} &
\pile{0.8}{$\mid$}{\dreieck}{\textbf{uuva}}\hspace{-0.3cm}=\hspace{-0.3cm}\pile{0.8}{}{\kreis}{uuvama} &
\pile{0.8}{$\mid$\hln{9}{8}{20}}{\kreis}{\textbf{uuvama}}\hspace{-0.3cm}=\hspace{-0.3cm}\pile{0.8}{}{\dreieck}{uuva} &
%
\pile{0.8}{$\mid$}{\dreieck}{baapa}\hspace{-0.3cm}=\hspace{-0.3cm}\pile{0.8}{$\mid$\vln{-1}{8}{238}}{\kreis}{umma} &
%
\pile{0.8}{$\mid$\hln{12}{8}{20}}{\dreieck}{maama}\hspace{-0.3cm}=\hspace{-0.3cm}\pile{0.8}{}{\kreis}{maami} &
\pile{0.8}{$\mid$}{\kreis}{\parbox{0.8cm}{biibi/\\\vspace{-0.3cm}caaci}}\hspace{-0.3cm}=\hspace{-0.3cm}\pile{0.8}{}{\dreieck}{muuda} &
\pile{0.8}{$\mid$}{\dreieck}{\textbf{uuva}}\hspace{-0.3cm}=\hspace{-0.3cm}\pile{0.8}{}{\kreis}{uuvama} &
\pile{0.8}{$\mid$}{\kreis}{\textbf{uuvama}}\hspace{-0.3cm}=\hspace{-0.3cm}\pile{0.8}{}{\dreieck}{uuva}
\\
\\
\multicolumn{2}{l}{\hln{34}{8}{40} \footnotesize uncles' ch same as siblings} &
\pile{0.8}{$\mid$\vln{-1}{8}{269}}{\dreieck}{aade\\\footnotesize klaaki}\hspace{-0.3cm}=\hspace{-0.3cm}\pile{0.8}{}{\kreis}{kiccil\\\footnotesize umbo} &
\pile{0.8}{$\mid$}{\kreis}{aade\\\footnotesize pompang}\hspace{-0.3cm}=\hspace{-0.3cm}\pile{0.8}{}{\dreieck}{kiccil\\\footnotesize aabang} &
\pile{0.8}{$\mid$\hln{15}{8}{20}}{}{EGO\\~} =\parbox{1cm}{\footnotesize laaki\male\\~\\ \footnotesize  biini\female}  &
\pile{0.8}{$\mid$}{\dreieck}{\textbf{kaaka}\\~}\hspace{-0.3cm}=\hspace{-0.3cm}\pile{0.8}{}{\kreis}{umbo\\~} &
\pile{0.8}{$\mid$}{\kreis}{\textbf{dhaatha}\\~}\hspace{-0.3cm}=\hspace{-0.3cm}\pile{0.8}{}{\dreieck}{aabang\\~} &
\multicolumn{2}{r}{\hln{90}{8}{40}\footnotesize aunts' ch same as siblings}
\\\\
\multicolumn{4}{r}{\pile{0.8}{}{\dreieck}{manthu\\}\hspace{-0.3cm}=\hspace{-0.3cm}\pile{0.8}{$\mid$\vln{-1}{8}{120}}{\kreis}{aanak\\\footnotesize klaaki}} &
 &
\multicolumn{4}{l}{\pile{0.8}{$\mid$\hln{-60}{8}{40}}{\dreieck}{aanak\\\footnotesize pompang}\hspace{-0.3cm}=\hspace{-0.3cm}\pile{0.8}{}{\kreis}{manthu\\}}
\\\\
&
&
\pile{0.8}{$\mid$\vln{-1}{8}{68}}{\dreieck}{cuucu}&
\pile{0.8}{$\mid$\hln{-6}{8}{20}}{\kreis}{ciici}&
&
\pile{0.8}{\vln{1}{8}{68}\hln{16}{8}{20}$\mid$}{\dreieck}{cuucu}&
\pile{0.8}{$\mid$}{\kreis}{ciici}&
&
\\
&
 &
\pile{0.8}{\hln{-1}{0}{12}}{\kreis\dreieck}{cangavaari}&
\pile{0.8}{\hln{-1}{0}{12}}{\kreis\dreieck}{cangavaari}&
 &
\pile{0.8}{\hln{-1}{0}{12}}{\kreis\dreieck}{cangavaari}&
\pile{0.8}{\hln{-1}{0}{12}}{\kreis\dreieck}{cangavaari}&
 &
\\\\
\end{tabular}

 % kin.eps: 141954512x146425912 pixel, 300dpi, 120.881.50x1239739.38 cm, bb14 14 1021 422
 \caption[Kinship relations]{Kinship relations in SLM. Bold face denotes elder siblings, where applicable. Normal font denotes younger siblings, if there is a distinction with bold face in the same generation. Relative age of spouses of parents' siblings is not relevant.}
 \label{fig:func:kin}
\end{sidewaysfigure}
% \end{figure}


\citet{Kekulawala1982} first studied the kinship system of SLM (cited in \citet{Bichsel}) and analyzed it as lineo-bifurcate collateral type.

