\section*{Snow-White and Rose-Red}
This text is a translation of the English version of the Brothers Grimm's tale. It is clearly written modality, with long and complex sentences and explicit reference tracking, some of which might be due to influences from the English text, e.g. the frequent use of the demonstrative \em itthu \em where English would require the definite article.


\glossSTDmode
\xbox{14}{
\ea
\gll Snow-White=le  Rose-Red=le. \\
      Snow-White=\textsc{addit} Rose-Red=\textsc{addit} \\
    `Snow-White and Rose-Red.'
\z
}


\xbox{14}{
\ea
\gll Hathu muusing=ka, sendiiri thurung guunung hatthu=ka,  kampong=nang konyong jaau sindari, hathu kiccil ruuma su-aada.  \\
     \textsc{indef} time=\textsc{loc} alone slope mountain \textsc{indef}=\textsc{loc} village few far \textsc{prox.abl} \textsc{indef} small house \textsc{past}-exist  \\
    `Once upon a time,  alone by  a hill slope, a little far away from this side of the village, there was a small house.'
\z
}


\xbox{14}{
\ea
\gll Itthu bannyak laama hathu ruuma, itthule ruuma duuva subla=ka [panthas rooja kumbang pohong komplok] duuva asà-jaadi su-aada.  \\
      \textsc{dem.dist} very old \textsc{indef} house, but house two side=\textsc{loc} beautiful rose flower tree bush two \textsc{cp}-become \textsc{past}-exist \\
    `That was a very old house, but still on the two sides of the house, there had grown two beautiful rose bushes.'
\z
}


\xbox{14}{
\ea
\gll Hatthu komplok bannyak=jo puuthi caaya,  hathyeng=yang meera=jo meera caaya. \\
     one bush very=\textsc{emph} white colour, other red=\textsc{emph} red colour \\
    `One bush was pure white, the other one was pure red.'
\z
}


\xbox{14}{
\ea
\gll Itthu ruuma=ka [laaki anà-mnii\u n\u ggal hathu pompang]=le, incayang=pe buthul panthas aanak pompang duuva=le su-aada.  \\
     \textsc{dem.dist} house=\textsc{loc} husband \textsc{past}-die \textsc{indef} female=\textsc{addit}, \textsc{\textsc{3s.polite}} very beautiful child female two=\textsc{addit} \textsc{past}-exist  \\
    `In that house, there lived a widow with her two very beautiful daughters.'
\z
}


\xbox{14}{
\ea
\gll Derang duuva oorang=pe naama pada kalu, Snow-White, hattheyang Rose-Red. \\
     3lp two man=\textsc{poss} name \textsc{pl} cond Snow-White, other Rose-Red  \\
    `Their names were: Snow-White and Rose-Red.'
\z
}


\xbox{14}{
\ea
\gll Kumbang rooja komplok duuva=pe naama=kee=jo.  \\
     flower red bush two=\textsc{poss} name=\textsc{simil}=\textsc{emph}  \\
    `Just like the two bushes of roses.'
\z
}


\xbox{14}{
\ea
\gll Derang=pe panthas=dering=le,  pehel=dering=le,  duduk-an=dering=le butthul baaye.  \\
     \textsc{3pl}=\textsc{poss} beautiful=\textsc{abl}=\textsc{addit}, behaviour=\textsc{abl}=\textsc{addit} live-\textsc{nmlzr}=\textsc{abl}=\textsc{addit} very good  \\
    `Their beauty,  behaviour and  the way they lived were very good.'
\z
}


\xbox{14}{
\ea
\gll Derang derang=pe umma=nang butthul saayang=kee=jo samma ruuma pukurjan=nang=le anà-banthu.  \\
      \textsc{3pl} \textsc{3pl}=\textsc{poss} mother=\textsc{dat} very love=\textsc{simil}=\textsc{emph} all house work=\textsc{dat}=\textsc{addit} \textsc{past}-help \\
    `They helped her with all the homework,  while they loved their mother very much.'
\z
}


\xbox{14}{
\ea
\gll Ruuma pukurjan abbis=nang blaakang,  derang dìkkath=ka aada laapang=nang mà-maayeng=nang su-pii.  \\
      house work finish=\textsc{dat} after \textsc{3pl} vicinity=\textsc{loc} exist ground=\textsc{dat} \textsc{inf}-play=\textsc{dat} \textsc{past}-go \\
    `After finishing their homework, they went out to play at the ground close to their home.'
\z
}


\xbox{14}{
\ea
\gll Soore=ka,  Snow-White=le Rose-Red=le derang=pe umma=samma appi dìkkath=ka arà-duuduk ambel.  \\
      evening=\textsc{loc} Snow-White=\textsc{addit} Rose-Red=\textsc{addit} \textsc{3pl}=\textsc{poss} mother=\textsc{comit} fire close=\textsc{loc} \textsc{simult}-sit take \\
    `In the evening, Snow-White and Rose-Red would retire to sit near the chimney with their mother.'
\z
}



\xbox{14}{
\ea
\gll Derang=pe umma derang=nang jaith-an=le,  jaarong pukurjan=le su-aajar.   \\
     \textsc{3pl}=\textsc{poss} mother \textsc{3pl}=\textsc{dat} sew-\textsc{nmlzr}=\textsc{addit}  needle work=\textsc{addit} \textsc{past}-teach \\
    `Their mother taught them sewing and needle work.'
\z
}


\xbox{14}{
\ea
\gll Derang anà-duuduk samma vakthu=ka=le derang suuka=ka vakthu anà-empas.  \\
      \textsc{3pl} \textsc{past}-stay all time=\textsc{loc}=\textsc{addit} \textsc{3pl} like=\textsc{loc} time \textsc{past}-spend \\
    `All the time when they were together, they spent the time very happily.'
\z
}


\xbox{14}{
\ea
\gll Hathu haari,\footnotemark{} diinging hathu soore=ka,  kìrras pinthu=nang arà-thatti  hathu svaara su-dìnngar.   \\
     \textsc{indef} day cold \textsc{indef} evening=\textsc{loc} strong door=\textsc{dat} \textsc{simult}-knock \textsc{indef} sound \textsc{past}-hear  \\
    `One cold evening, they heard a noise of somebody knocking at the door.'
\z
}
\footnotetext{The use of an initial \graphem{h} in this word is typical of the written variety.}

\xbox{14}{
\ea
\gll Itthu saapa anthi-aada!  \\
      \textsc{dem.dist} who \textsc{irr}-exist \\
    `Who would that be (there)!?'
\z
}


\xbox{14}{
\ea
\gll Snow-White=le Rose-Red=le pinthu su-bukka.  \\
     Snow-White=\textsc{addit} Rose-Red=\textsc{addit} door \textsc{past}-open  \\
    ` Snow-White and Rose-Red opened the door.'
\z
}


\xbox{14}{
\ea
\gll Sithu=ka,  hathu bìssar beecek caaya Buruan\footnotemark{}  su-duuduk.  \\
     there=\textsc{loc} \textsc{indef} big mud colour bear \textsc{past}-exist.\textsc{anim}  \\
    `There was a big brown bear.'
\z
}
\footnotetext{The bear is always capitalized in this text, which might be taken over from the English original.}

\xbox{14}{
\ea
\gll Aanak pompang duuva su-thaakuth.  \\
     child female two \textsc{past}-fear  \\
    `The two girls got afraid.'
\z
}


\xbox{14}{
\ea
\gll ``Thussa mà-thaakuth'',  Buruan su-biilang.  \\
       \textsc{neg.imp} \textsc{inf}-fear bear \textsc{past}-say\\
    `The bear said: ``Don't get afraid!'' '
\z
}



\xbox{14}{
\ea
\gll  ``See lorang=nang thama-sakith-kang,  lorang see=yang diinging=dering kala-aapith; \\
       \textsc{1s} \textsc{2pl}=\textsc{dat} \textsc{neg.irr}-hurt-\textsc{caus} \textsc{2pl} \textsc{1s}=\textsc{acc} cold=\textsc{abl} cond-protect\\
    ` ``I will not harm you, if you protect me from the cold;'
\z
}


\xbox{14}{
\ea
\gll see arà-sumpa,  paanas muusing dhaathang thingka see siini=dering arà\footnotemark-pii.   \\
     \textsc{1s} \textsc{non.past}-promise hot time come middle \textsc{1s} here=\textsc{abl} \textsc{non.past}-go \\
    `I promise, soon when the spring arrives I will leave from here.'
\z
}
\footnotetext{The use of the non-past marker \em arà- \em instead of the irrealis marker \em anthi- \em confers more commitment to the promise.}



\xbox{14}{
\ea
\gll Aanak pompang duuva=le  derang=pe umma=le Buruan=yang ruuma daalang=nang su-panggel.  \\
     child female two=\textsc{addit} \textsc{3pl}=\textsc{poss} mother=\textsc{addit} bear=\textsc{acc} house inside=\textsc{dat} \textsc{past}-call  \\
    `The two girls and their mother invited the bear into their home.'
\z
}



\xbox{14}{
\ea
\gll Buruan diinging abbis=sangke siithu su-siinga.   \\
      bear cold finish=until there \textsc{past}-stay \\
    `The bear stayed there until the cold was gone.'
\z
}



\xbox{14}{
\ea
\gll Demikian,  thahun\footnotemark{} baaru muusing=pe kàthaama haari=ka beecek caaya Buruan,  incayang=yang baaye=nang anà-kuvaather aanak pompang duuva=nang slaamath katha su-biilang.  \\
      absolutely, year new time=\textsc{poss} first day=\textsc{loc} mud colour bear \textsc{3s.polite}=\textsc{acc} good=\textsc{dat} \textsc{past}-protect child female two=\textsc{dat} greetings \textsc{quot} \textsc{past}-say \\
    `Absolutely, on the first day of the new year, this brown bear bid farewell to the two girls who had taken good care of him.'
\z
}
\footnotetext{The short vowel in \em thahun \em might be due to it being in a close collocation with \em baaru\em, thereby being parsed into one phonological word, rather than two. As such, there is no need to build an additional foot in \em thahun\em, given that one is already present in \em b(a$_\mu$a$_\mu$)$<$ru$>$\em.}

\xbox{14}{
\ea
\gll Itthu haari=ka=jo aanak pompang duuva=nang [[[hathu duuri pohong=nang] [puuthi  paanjang jee\u n\u ggoth=yang] anà-kànà-daapath kìnna] hathu kiccil jillek Aajuth\footnotemark{} hatthu=yang su-kuthumung.   \\
     \textsc{dem.dist} day=\textsc{loc}=\textsc{emph} child female two=\textsc{dat} \textsc{indef} thorn tree=\textsc{dat} white long beard=\textsc{acc} \textsc{past}-\textsc{invol}-get strike \textsc{indef} small ugly dwarf \textsc{indef}]\footnotemark=\textsc{acc} \textsc{past}-see  \\
    `On the same day, the two girls  saw a small ugly dwarf whose long white beard had got stuck  in a thorn tree.'
\z
}
\footnotetext{Like the bear, the dwarf is capitalized throughout the story.}
\addtocounter{footnote}{1}
\footnotetext{This very complex NP is given as a tree on the next page.}

\begin{sidewaysfigure}
\footnotesize
\Tree   [.S
	  \qroof{\parbox{4cm}{\gll Itthu haari=ka=jo\\ \textsc{dem.dist} day=\textsc{loc}=\textsc{emph}\\}}.NP !\qsetw{8cm}
	  \qroof{\parbox{6cm}{\gll aanak pompang duuva=nang\\ child female two=\textsc{dat}\\}}.PP !\qsetw{6cm}
	  [.PP
	     [.NP
	    	[.RELC
			[\qroof{\parbox{4cm}{\gll hathu duuri  pohong=nang\\ \textsc{indef} thorn tree=\textsc{dat}\\}}.PP ]
			[\qroof{\parbox{5cm}{\gll puuthi  paanjang jee\u n\u ggoth=yang\\ white long beard=\textsc{acc} \\}}.PP ]
			[\qroof{\parbox{4cm}{\gll anà-kànà-daapath kìnna\\ \textsc{past}-\textsc{invol}-get strike\\}}.VP ]
		] !\qsetw{5cm}
	    	[.INDEF hatthu ] !\qsetw{1cm}
	    	[\qroof{\parbox{1.8cm}{\gll kiccil jillek\\ small ugly \\}}.ADJ ]!\qsetw{1.8cm}
	    	[.N \parbox{1cm}{\gll Aajuth\\ dwarf\\} ]
	    	[.INDEF hatthu ]
	  	] !\qsetw{8cm}
	    [.P =yang ]
	 ]   !\qsetw{6cm}
% 	  \qroof{pohong=nang puuthi,  paanjang jee\u n\u ggoth=yang anà-kànà-daapath kìnna hathu kiccil jillek Aajuth hatthu=yang}.G
	  [.V sukuthumung ]
	]
\end{sidewaysfigure}

\xbox{14}{
\ea
\gll Derang incayang=yang su-salba-king.  \\
     \textsc{3pl} \textsc{3s.polite}=\textsc{acc} \textsc{past}-escape-\textsc{caus}  \\
    `They saved him.'
\z
}


\xbox{14}{
\ea
\gll Incayang,  derang=nang thriima jamà-kaasi=nang,  [anà-banthu vakthu=ka incayang=pe jee\u n\u ggoth=yang asà-thaarek=apa incayang=nang su-sakith-kang katha] anà-maaki.  \\
     \textsc{3s.polite} \textsc{3pl}=\textsc{dat} thanks \textsc{neg.nonfin}-give=\textsc{dat} \textsc{past}-help time=\textsc{loc} \textsc{3s.polite}=\textsc{poss} beard=\textsc{acc} \textsc{cp}-pull=after \textsc{3s.polite}=\textsc{dat} \textsc{past}-hurt-\textsc{caus} \textsc{quot} \textsc{past}-scold   \\
    `He did not thank them and instead he scolded them saying that they pulled his beard and hurt him while they were helping him.'
\z
}


\xbox{14}{
\ea
\gll Hathyang aari=ka Snow-White=le Rose-Red=le laapang pii\u n\u ggir=ka,  {\em berry} kapang-picca-kang,  bìssar hathu buurung derang=pe atthas=dering su-thìrbang. \\
      other day=\textsc{loc} Snow-White=\textsc{addit} Rose-Red=\textsc{addit} ground border=\textsc{loc} berry when-broken-\textsc{caus} big \textsc{indef} bird \textsc{3pl}=\textsc{poss} top=\textsc{abl} \textsc{past}-fly \\
    `The next day, when Snow-White and Rose-Red were picking berries by the side of the ground,  there was a big bird flying over them.'
\z
}


\xbox{14}{
\ea
\gll Banthu-an asà-mintha arà-naangis svaara hatthu, derang=nang su-dìnngar.   \\
     help-\textsc{nmlzr} \textsc{cp}-ask \textsc{simult}-cry sound \textsc{indef} \textsc{3pl}=\textsc{dat} \textsc{past}-hear  \\
    `They heard a noise crying and asking for help.'
\z
}



\xbox{14}{
\ea
\gll Derang su-kuthumung ithu buurung=pe kuuku=ka Aajuth asà-sìrrath kìnna arà-duuduk.   \\
     \textsc{3pl} \textsc{past}-see \textsc{dem.dist} bird=\textsc{poss} claw=\textsc{loc} dwarf \textsc{cp}-stuck strike \textsc{simult}-exist.\textsc{anim}  \\
    `They saw the dwarf stuck in the claws of the bird.'
\z
}


\xbox{14}{
\ea
\gll Aajuth=yang buurung mà-angkath baapi su-diyath.  \\
      dwarf=\textsc{acc} bird \textsc{inf}-lift take.away \textsc{past}-try \\
    `The bird was trying to take the dwarf with him.'
\z
}


\xbox{14}{
\ea
\gll [[Kaaki=ka gaa\u ndas-kang ambel anà-duuduk Aajuth]=yang sangke-luppas], [hathu pollu=dering] [Rose-Red] [buurung=nang] [su-puukul].   \\
     leg=\textsc{loc} tie-\textsc{caus} take \textsc{past}-exist.\textsc{anim} dwarf=\textsc{acc} until-leave \textsc{indef} stick=\textsc{abl} Rose-Red bird=\textsc{dat} \textsc{past}-hit \\
    `Rose-Red whacked the bird  with a stick until the bird let loose of the dwarf who was held under the legs of the bird.'
\z
}


\xbox{14}{
\ea
\gll Laiskali,  inni Aajuth  [incayang=pe jiiva anà-salba-king itthu aanak pompang duuva=]nang thriima thàrà-kaasi. \\
      again \textsc{dem.prox} dwarf \textsc{3s.polite}=\textsc{poss} life \textsc{past}-escape-\textsc{caus} \textsc{dem.dist} child female two=\textsc{dat} thanks \textsc{neg.past}-give \\
    `Once again, the dwarf did not thank the two girls who had saved his life.'
\z
}


\xbox{14}{
\ea
\gll ``Lorang=pe guuna thàràsampe! hathyang skali bannyak masà-aapith!'' Aajuth su-butharak. \\
       \textsc{2pl}=\textsc{poss} respect not.enough other time much must-care dwarf \textsc{past}-scream\\
    ` ``Your respect is not enough! Next time, you have to be more careful!'' the dwarf shouted.'
\z
}



\xbox{14}{
\ea
\gll Blaakang,  derang Aajuth=yang laiskali su-kuthumung,  incayang=ka [[bannyak panthas ummas baarang pada=le,  bathu inthan\footnotemark{} pada=le anà-punnu-kang] bìssar beecek caaya hathu {\em bag}] su-aada.   \\
      after \textsc{3pl} dwarf=\textsc{acc} again \textsc{past}-see \textsc{3s.polite}=\textsc{loc} much beautiful gold item \textsc{pl}=\textsc{addit} stone shine \textsc{pl}=\textsc{addit} \textsc{past}-full-\textsc{caus} big mud colour \textsc{indef} bag \textsc{past}-exist \\
    `Later, they saw the dwarf again, he had a big brown sack of very beautiful golden items and gems with him.'
\z
}
\footnotetext{Note the short vowel in the first part of the compound \trs{bathu inthan}{gems}.}


\xbox{14}{
\ea
\gll Incayang ithu pada=yang kapang-thumpa-king, itthu pada sraathus binthan pada arà-kiilap=ke su-kiilap.  \\
      \textsc{3s.polite} \textsc{dem.dist} \textsc{pl}=\textsc{acc} when-spill-\textsc{caus} \textsc{dem.dist} \textsc{pl} 100 stars \textsc{pl} \textsc{simult}-shine=\textsc{simil} \textsc{past}-shine \\
    `When he spilled them, they shone like a hundred shining stars.'
\z
}



\xbox{14}{
\ea
\gll Itthu vakthu=ka hathu bìssar beecek caaya Buruan mlaarath=ka uuthang=dering luvar=nang su-dhaathang. \\
      \textsc{dem.dist} time=\textsc{loc} \textsc{indef} big brown colour bear difficulty=\textsc{loc} forest=\textsc{abl} outside=\textsc{dat} \textsc{past}-come \\
    `At the same time, a big brown bear came out of the woods.'
\z
}


\xbox{14}{
\ea
\gll [Anà-kaageth kìnna Aajuth] mà-laari=nang su-baalek,   \\
     \textsc{past}-startle strike dwarf \textsc{inf}-run-\textsc{dat} \textsc{past}-turn  \\
    `The shocked dwarf got back to run.'
\z
}


\xbox{14}{
\ea
\gll Buruan incayang=yang asà-thoolak,  ummas  gooni dìkkath=nang su-pii.  \\
     bear \textsc{2s.polite}=\textsc{acc} \textsc{cp}-push gold bag vicinity=\textsc{dat} \textsc{past}-go  \\
    `The bear pushed the dwarf aside and went up to the sack.'
\z
}


\xbox{14}{
\ea
\gll Aajuth thaakuth=ka su-naangis,  \\
    dwarf fear=\textsc{loc} \textsc{past}-weep   \\
    `The dwarf wept in fear:.'
\z
}

\xbox{14}{
\ea
\gll  ``See=yang luppas,  Thuan Buruan.   \\
      \textsc{1s}=\textsc{acc} leave Sir bear \\
    ` ``Please release me, Mister Bear.'
\z
}



\xbox{14}{
\ea
\gll Lorang se=dang mà-hiidop thumpath kala-kaasi, see lorang=nang  lorang=pe samma duvith=le,  baarang pada=le anthi-bale-king'' \\
      \textsc{2pl} \textsc{1s=dat} \textsc{inf}-stay place if-give \textsc{1s} \textsc{2pl}=\textsc{dat} \textsc{2pl}=\textsc{poss} all money=\textsc{addit} item \textsc{pl}=\textsc{addit} \textsc{irr}-turn-\textsc{caus} \\
    `If you give me a chance to live, I will return back all your money and your things.'' '
\z
}


\xbox{14}{
\ea
\gll Snow-White=nang=le Rose-Red=nang=le ini hatthu=ke thàrà-mirthi.  \\
     Snow-White=\textsc{dat}=\textsc{addit} Rose-Red=\textsc{dat}=\textsc{addit} \textsc{dem.prox} \textsc{indef}=\textsc{simil} \textsc{neg.past}-understand  \\
    `Neither Snow-White nor Rose-Red understood a single thing.'
\z
}


\xbox{14}{
\ea
\gll  Derang anà-kuthumung pada=nang asà-thaakuth,  ruuma=nang mà-laari kapang-pii,  derang=nang byaasa svaara hatthu su-dìnngar.  \\
     \textsc{3pl} \textsc{past}-see \textsc{pl}=\textsc{dat} \textsc{cp}-fear house=\textsc{dat} \textsc{inf}-run when-run \textsc{3pl}=\textsc{dat} habit sound \textsc{indef} \textsc{past}-hear \\
    `They only got afraid at what they saw, and when they were about to run back home, they heard a familiar voice.'
\z
}


\xbox{14}{
\ea
\gll ``Snow-White, Rose-Red,  thussa mà-thaakuth!''  \\
       Snow-White Rose-Red \textsc{neg.imp} \textsc{inf}-fear\\
    `Snow-White, Rose-Red do not get afraid!'
\z
}

\xbox{14}{
\ea
\gll Duuduk! See=le lorang=samma arà-dhaathang''. \\
     stay \textsc{1s}=\textsc{addit} \textsc{2pl}=\textsc{comit} \textsc{non.past}-come  \\
    `Wait!  I am coming with you, too.'' '
\z
}


\xbox{14}{
\ea
\gll Derang anà-baalek sajja=jo,  sithu=ka panthas hathu Aanak raaja su-aada! \\
     \textsc{3pl} \textsc{past}-turn only=\textsc{emph} there=\textsc{loc} beautiful \textsc{indef} child king \textsc{past}-exist  \\
    `When they turned around, there was a beautiful prince!.'
\z
}


\xbox{14}{
\ea
\gll Buruan=pe kuulith incayang=pe kaaki baava=ka jaatho su-aada! \\
      bear=\textsc{poss} skin \textsc{3s.polite}=\textsc{poss} leg bottom=\textsc{loc} fall \textsc{past}-exist \\
    `The skin of the bear had fallen at his feet!.'
\z
}


\xbox{14}{
\ea
\gll Aanak raaja su-biilang, \\
     child king \textsc{past}-say  \\
    `The prince said.'
\z
}


\xbox{14}{
\ea
\gll ``Lorang=nang see=yang ingath-an\footnotemark=si? \\
      \textsc{2pl}=\textsc{dat} \textsc{1s}=\textsc{acc} think-\textsc{nmlzr}=\textsc{interr} \\
    ` ``Do you remember me?'
\z
}
\footnotetext{Note the use of a nominalized form here, instead of the verb \trs{arà-iingath}{remember}.}



\xbox{14}{
\ea
\gll Diinging vakthu=ka ruuma daalang=nang anà-ambel bìssar beecek caaya buruan.  \\
     cold time=\textsc{loc} house inside=\textsc{dat} \textsc{past}-take big mud colour bear  \\
    `The big brown bear whom you gave shelter at your home during the cold season (is me).'
\z
}

\xbox{14}{
\ea
\gll See hathu bìnnar Aanak raaja.   \\
     \textsc{1s} \textsc{indef} real child king  \\
    `I am a real prince.'
\z
}


\xbox{14}{
\ea
\gll Sdiikith thaaun=nang duppang, see ini Aajuth=nang su-kìnna-daapath.\\
     few year=\textsc{dat} front \textsc{1s} \textsc{dem.prox}   dwarf=\textsc{dat} \textsc{past}-\textsc{invol}-get\\
    `A few years before, I was captured by this dwarf.'
\z
}


\xbox{14}{
\ea
\gll Ini Aajuth se=ppe baapa=nang su-simpa,  hathu ummas gooni kala-kaasi, see=yang athi=kasi-bìrrath katha.   \\
      \textsc{dem.prox} dwarf \textsc{1s=poss} father=\textsc{dat} \textsc{past}-promise \textsc{indef} gold bag if-give, \textsc{1s}=\textsc{acc} \textsc{irr}-give-heavy\footnotemark{} \textsc{quot} \\
    `This dwarf promised to my father that he would return me to my father, if my father gave him a sack of gold.'
\z
}
\footnotetext{\em Bìrrath kaasi \em is an idiom meaning `to admit, to accept'.}

\xbox{14}{
\ea
\gll Se=ppe baapa incayang=nang ummas su-kaasi. \\
     \textsc{1s=poss} father \textsc{3s.polite}=\textsc{dat} gold \textsc{past}-give  \\
    `My father gave him gold.'
\z
}



\xbox{14}{
\ea
\gll Itthule see=yang mà-kiiring=nang duppang incayang see=yang hathu Buruan mà-jaadi su-bale-king.   \\
     but \textsc{1s}=\textsc{acc} \textsc{inf}-send=\textsc{dat} front \textsc{3s.polite} \textsc{1s}=\textsc{acc} \textsc{indef} bear \textsc{inf}-become \textsc{past}-turn-\textsc{caus}  \\
    `Yet, before sending me back, he changed me into a bear.'
\z
}


\xbox{14}{
\ea
\gll [Se=ppe oorang pada [see saapa katha] thàrà-thaau]=subbath see=yang su-uubar.'' \\
     \textsc{1s=poss} man \textsc{pl} \textsc{1s} who \textsc{quot} \textsc{neg.past}-know=because \textsc{1s}=\textsc{acc} \textsc{past}-chase  \\
    `Because my people did not know who I was and they chased me.'' '
\z
}


\xbox{14}{
\ea
\gll ``Lorang=le,  lorang pada=pe umma=le see=yang baaye=nang anà-kuaather,  \\
      \textsc{2pl}=\textsc{addit} \textsc{2pl} \textsc{pl}=\textsc{poss} mother=\textsc{addit} \textsc{1s}=\textsc{acc} good=\textsc{dat} \textsc{past}-care \\
    `You two and your mother treated me very well.'
\z
}


\xbox{14}{
\ea
\gll see=yang lorang=susamma diinging muusing sangke-habbis anà-simpang ambel.  \\
      \textsc{1s}=\textsc{acc} \textsc{2pl}=\textsc{comit} cold time until-finish \textsc{past}-keep take \\
    `You all kept me until the cold season ended.'
\z
}

\xbox{14}{
\ea
\gll See lorang=nang arà-simpa kaapang=ke see lorang=nang ithu uuthang arà-baayar katha. \\
     \textsc{1s} \textsc{2pl}=\textsc{dat} \textsc{non.past}-promise when=\textsc{simil} \textsc{1s} \textsc{2pl}=\textsc{dat} \textsc{dem.dist} debt \textsc{non.past}-pay \textsc{quot}  \\
    `I promise you someday I will pay back that debt.'
\z
}


\xbox{14}{
\ea
\gll Lorang see=subbath ithu Aajuth=yang su-salba-king \\
     \textsc{2pl} \textsc{1s}=because \textsc{dem.dist} dwarf=\textsc{acc} \textsc{past}-escape-\textsc{caus}  \\
    `You all saved the dwarf because of me.'
\z
}


\xbox{14}{
\ea
\gll Itthu=nang blaakang=jo se=dang laiskali se=ppe bìnnar mosthor anà-jaadi,   \\
     \textsc{dem.dist}=\textsc{dat} after=\textsc{emph} \textsc{1s=dat} again \textsc{1s=poss} real manner \textsc{past}-become  \\
    `After that only, I returned to my true self.'
\z
}



\xbox{14}{
\ea
\gll suda lorang=yang ruuma=nang anthi-aaji.baapi\footnotemark \\
     thus \textsc{2pl}=\textsc{acc} house=\textsc{dat} \textsc{irr}-take.away.\textsc{anim}  \\
    `So, I will take you back home.'
\z
}
\footnotetext{\em Aaji \em occurs only in collocation with \trs{baapi}{take (away)}, where it indicates that the theme is animate.}

\xbox{14}{
\ea
\gll Thapi,  boole lìkkas=ka\footnotemark{} see lorang=yang mliige=nang anthi=panggel'' \\
     but can quick=\textsc{loc} \textsc{1s} \textsc{2pl}=\textsc{acc} palace=\textsc{dat} \textsc{irr}=call \\
    `But, as soon as possible I will invite you to the palace.'
\z
}
\footnotetext{\em Boole lìkkas=ka \em is an idiom meaning `As soon as possible'.}


\xbox{14}{
\ea
\gll [Aanak raaja=pe perkathahan=yang] [Snow-White=nang=le Rose-Red=nang=le suka-han=dering su-punnu hathu hidopan] su-thunjiking. \\
      child king=\textsc{poss} word=\textsc{acc} Snow-White=\textsc{dat}=\textsc{addit} Rose-Red=\textsc{dat}=\textsc{addit} like-\textsc{nmlzr}=\textsc{abl} \textsc{past}-fill \textsc{indef} prospect \textsc{past}-show \\
    `The words uttered by the prince showed prospects which filled Snow-White and Rose-Red with delight.'
\z
}


\xbox{14}{
\ea
\gll Sdiikith haari=nang blaakang ini sudaari duuva=nang mliige\footnotemark=dering su-panggel. \\
     few day=\textsc{dat} after \textsc{dem.prox} sister two=\textsc{dat} palace=\textsc{abl} \textsc{past}-call  \\
    `After a few days, these two sisters were invited by the palace.'
\z
}
\footnotetext{The word for `palace' is sometimes written \em mliiga\em, sometimes \em mliige\em.}

\xbox{14}{
\ea
\gll  Aanak raaja laiskali anà-dhaathang=subbath, samma oorang bannyak suuka=ka anà-duudu.\footnotemark \\
      child king again \textsc{past}-come=because all man much like=\textsc{loc} \textsc{past}-exist.\textsc{anim}\\
    `Because the prince had returned, all his subjects were very happy.'
\z
}
\footnotetext{The form \em duuduk \em with a \em k \em would be more common in written texts.}

\xbox{14}{
\ea
\gll Sdiikith haari=nang blaakang Snow-White Aanak raaja=yang=le,  Rose-Red incayang=pe sudaara=yang=le su-kaaving.   \\
      few day=\textsc{dat} after Snow-White child king=\textsc{acc}=\textsc{addit} Rose-Red \textsc{3s.polite}=\textsc{poss} brother=\textsc{acc}=\textsc{addit} \textsc{past}-marry \\
    `Some days later, the prince was married to Snow-White and his brother was married to Rose-Red.'
\z
}


\xbox{14}{
\ea
\gll Derang=pe umma=le derang=samma hatthu=nang\footnotemark{} mà-hiidop=nang mliiga=nang su-dhaathang.   \\
     \textsc{3pl}=\textsc{poss} mother=\textsc{addit} \textsc{3pl}=\textsc{comit} one=\textsc{dat} \textsc{inf}-stay=\textsc{dat} palace=\textsc{dat} \textsc{past}-come  \\
    `Their mother also came to live together with  them in the palace.'
\z
}
\footnotetext{\em Hatthunang \em has a lexicalized meaning, `together'.}

\xbox{14}{
\ea
\gll Ruuma duuva subala=ka su-aada rooja pohong komplok duuva=yang asà-baa=apa mliige=pe duuva subla=ka su-thaanàm.   \\
      house two side=\textsc{loc} \textsc{past}-exist  red tree bush two=\textsc{acc} \textsc{cp}-bring=after palace=\textsc{poss} two side=\textsc{loc} \textsc{past}-plant\\
    `The two rose bushes which were at one time standing at both sides of  the home were brought  and planted on both sides of the palace.'
\z
}


\xbox{14}{
\ea
\gll Maana thaahun=ka=le ini rooja pohong komplok duuva kumbang=dering arà-punnu \\
     which year=\textsc{loc}=\textsc{addit} \textsc{dem.prox} rose tree bush two flower=\textsc{abl} \textsc{non.past}-full  \\
    `Every year, these two bushes fill with beautiful roses.'
\z
}


\xbox{14}{
\ea
\gll Anà-ambel deri\footnotemark{} :-THE BROTHERS GRIMM.  \\
     \textsc{past}-take from :-THE BROTHERS GRIMM  \\
    `taken from The Brothers Grimm .'
\z
}
\footnotetext{\em Deri \em is used as a preposition here, which cannot be explained as of now.}

\xbox{14}{
\ea
\gll Mlaayu=nang anà-bale-king :- SALIM.    \\
     Malay=\textsc{dat} \textsc{past}-turn-\textsc{caus} :- SALIM  \\
    `translated into Malay: Salim.'
\z
}