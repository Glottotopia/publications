\section*{How Andare Ate Sugar}
This text is a translation of a Sinhalese tale about the jester Andare. Andare being a jester, he has the habit of fooling people. The king in turn wants to fool Andare this time, but in the end, it is again Andare who prevails. This text is clearly of written modality, with full sentences, clear reference tracking, little dropping of referents, no code-switching etc.


\glossSTDmode
 
% 
% \xbox{14}{
% \ea
% \gll eins\footnotemark{} zwei\footnotemark{} drei\footnotemark{}  \\
%      one\footnotemark{} two\footnotemark{}\addtocounter{footnote}{-1} three\footnotemark{}  \\
%     `un\footnotemark{} deux\footnotemark{} trois\footnotemark{}.' (nosource)
% \z
% } \\ 
% \footnotetext{2}
% \footnotetext{3}
% \footnotetext{4}
% \footnotetext{5}
% \footnotetext{6}
% \footnotetext{7}
% \footnotetext{8}
% \footnotetext{9}

\xbox{14}{
\ea
\gll  ANDARE GUULA  ANÀ-MAAKANG MOSTHOR.\\
      Andare sugar \textsc{past}-eat manner \\
    `The way how Andare ate sugar.'
\z
}

\xbox{14}{
\ea
\gll Hathu  haari Andare=le aanak=le mliiga=nang anà-pii. \\
     \textsc{indef} day Andare=\textsc{addit} child=\textsc{addit} palace=\textsc{dat} \textsc{past}-go  \\
    `One day, Andare and his son went to the castle.'
\z
}
 

\xbox{14}{
\ea
\gll Incayang mliiga=nang kapang-pii, Raaja hathu thiikar=ka guula asà-siibar mà-kìrring simpang su-aada.\footnotemark{}  \\
     3\textsc{s.polite} palace=\textsc{dat} when-go king \textsc{indef} mat=\textsc{loc} sugar \textsc{cp}-spread \textsc{inf}-dry keep \textsc{past}-exist  \\
    `When he was going to the palace, the King had sprinkled sugar in a mat and had left it to dry.'
\z
}
\footnotetext{The main verb in this string of four verbs is \trs{simpang}{keep}. \em Aada \em is used to put \em simpang \em in the perfect tense. \em Simpang \em has a preceding purposive clause \trs{màkiiring}{so that it dries}. The three mentioned verbs form a clause, which is preceded by a conjunctive participle clause which describes the previous action, sprinkling sugar on a mat.}


\xbox{14}{
\ea
\gll Andare raaja=ka su-caanya inni mà-kìrring simpang aada aapa=yang katha.  \\
     Andare king=\textsc{loc} \textsc{past}-ask \textsc{dem.prox} \textsc{inf}-dry keep exist what=\textsc{acc} \textsc{quot}  \\
   `Andare inquired from the King  what was it that was left [on the mat] to dry.'   
\z
}


\xbox{14}{
\ea
\gll Andare=yang mà-enco-king=nang raaja su-biilang itthu paasir katha\\
      Andare=\textsc{acc} \textsc{inf}-fooled-\textsc{caus}=\textsc{dat} king \textsc{past}-say \textsc{dem.dist} sand \textsc{quot} \\
    `To tease Andare the King said it was sand.'
\z
}


\xbox{14}{
\ea
\gll Giithu asà-biilang=apa raaja Andare=yang su-enco-king. \\
     like.that \textsc{cp}-say=after king Andare=\textsc{acc} \textsc{past}-fooled-\textsc{caus}  \\
    `The King teased Andare by speaking like that.'
\z
}


\xbox{14}{
\ea
\gll  Thapi=le\footnotemark{} Andare thàrà-jaadi enco.\\
      But=\textsc{addit} Andare \textsc{neg.past}-become fooled \\
    `But Andare was not fooled.'
\z
}
\footnotetext{Note the use of \em two \em adversative markers here, \em thapi \em and \em =le\em.}


\xbox{14}{
\ea
\gll Andare [[bale-king=apa\footnotemark{}] raaja=yang mà-enco-king=nang] su-iingath  \\
     Andare turn-\textsc{caus}=after king=\textsc{acc} \textsc{inf}-fooled-\textsc{caus}=\textsc{dat} \textsc{past}-think  \\
    `Andare thought that he should fool the king instead.'
\z
}
\footnotetext{The conjunctive participle of \trs{baleking}{return(tr.)} is used in the meaning of `instead' here. One could translate in a more literal way: `Andare thought that he should take his turn and fool the king.'}


 

\xbox{14}{
\ea
\gll Andare aanak\footnotemark=nang su-biilang:  \\
     Andare child=\textsc{dat} \textsc{past}-say  \\
    `Then, Andare told his son .'
\z
}
\footnotetext{The child had not been introduced before, but it is inferred from context that Andara must be the father. Furthermore, one would infer that we are dealing with a boy child, the sex of a girl would probably be indicated, by \em pompang\em.}

\xbox{14}{
\ea
\gll Aanak, [[[lu=ppe umma su-maathi]\footnotemark{} katha bitharak=apa]\footnotemark{} asà-naangis]\addtocounter{footnote}{-1}{}\footnotemark{} mari! \\
     child \textsc{2s.familiar}=\textsc{poss} mother \textsc{past}-die \textsc{quot} scream=after \textsc{cp}-cry come.imp  \\
    `Son, cry loudly and come towards me saying that your mother died!'
\z
}
\addtocounter{footnote}{-1}{}
\footnotetext{The reported string `luppe umma su maathi' still has the second person pronoun \em luu \em instead of the first person pronoun \em see\em, which would be expected if we were dealing with direct speech.}
\addtocounter{footnote}{1}{}
\footnotetext{This clause chain is formed by first \em =apa \em and then \em asà-\em.}

\xbox{14}{
\ea
\gll Suda Andare=pe aanak=le baapa anà-biilang=kee=jo asà-naangis ambel\footnotemark{} su-dhaathang \\
      thus Andare=\textsc{poss} child=\textsc{addit} father \textsc{past}-say=\textsc{simil}=\textsc{emph} cry take \textsc{past}-come \\
    `So, Andare's son also did exactly what he was asked to do and came crying\footnotemark{} (to the king).'
\z
}
\addtocounter{footnote}{-1}{}
\footnotetext{The vector verb \em ambel \em indicates either the beginning of the crying here, or the beneficial nature of the crying, which will become clear later.}
\addtocounter{footnote}{1}{}
\footnotetext{A more faithful translation would be `cried and came', with a perfect participle instead of the present participle `crying'.}

\xbox{14}{
\ea
\gll Itthu=nang blaakang Andare guula thiikar=ka asà-lompath \\
      \textsc{dem.dist}=\textsc{dat} after Andare sugar mat=\textsc{loc} \textsc{cp}-jump \\
    `And then, Andare jumped on to the mat of sugar and'
\z
}


\xbox{14}{
\ea
\gll  lu=ppe muuluth=ka=le paasir, se=ppe muuluth=ka=le paasir\footnotemark{} katha biilang\~{}biilang baaye=nang baapa=le aanak=le guula su-maakang\\
       \textsc{2s.familiar}=\textsc{poss} mouth=\textsc{loc}=\textsc{addit} sand \textsc{1s=poss} mouth=\textsc{loc}=\textsc{addit} sand \textsc{quot} say\~{}\textsc{red} good=\textsc{dat} father=\textsc{addit} child=\textsc{addit} sugar \textsc{past}-eat\\
    `crying out  ``your mouth is filled with sand, my mouth is also filled with sand'', repeating this over and over again they both ate sugar well.'
\z
}
\footnotetext{Dead people will have sand in their mouth. To express his grief, Andare wants to have sand in his mouth to. This is of course a trick to take profit of the king's assertion that the substance on the mat was sand. Andare can now put the substance   in his mouth, and the king can say nothing against it since he himself had asserted that it was sand, rather than sugar.}


\xbox{14}{
\ea
\gll Suda kanabisan=ka, raaja Andare=yang mà-enco-king asà-pii, raaja=jo su-jaadi enco \\
      thus last=\textsc{loc} king Andare=\textsc{acc} \textsc{inf}-fooled-\textsc{caus} \textsc{cp}-go king=\textsc{emph} \textsc{past}-become fooled \\
    `Ultimately what happened, the King set out to fool Andare, but the king  was fooled himself.'
\z
}
 