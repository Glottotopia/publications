\chapter{Major and minor word classes}\label{sec:form:freemorphemes}
\glossSTDmode
This chapter gives an overview over SLM free morphemes.  The chapter thereafter will treat bound morphology. Free morphemes comprise both lexical and functional morphemes. These fall into a number of classes, which will be discussed in turn.

Four major word classes are distinguished in SLM:
verbs like \trs{maakan}{eat} \formref{sec:wc:Verbs},
nouns like \trs{aanak}{child} \formref{sec:wc:Nouns},
adjectives like \trs{kiccil}{small} \formref{sec:wc:Adjectives},
and adverbs like \trs{kumaareng}{yesterday} \formref{sec:wc:Adverbs}. These classes are all fairly large and can accept new members. The members of a class all share certain characteristics, by which class membership can be checked.

There are also a number of minor classes such as personal pronouns, which are rather small and which cannot accept new members. Of these classes, a full extensional list of members will be given, along with some characteristic traits.

\section{Verbs}\label{sec:wc:Verbs}
SLM Verbs can be subdivided into canonical verbs, which are the largest subclass \formref{sec:wc:Canonicalverbs}, two existentials \formref{sec:wc:Existentialverbs:intro}, and a number of defective verbs, which cannot take the whole array of morphology available for canonical verbs \formref{sec:wc:Defectiveverbs}. Verbs can combine with other verbs in multi-verb constructions, of which an overview is given in Section \formref{sec:wc:Multi-verbconstructions}.  Some canonical verbs have also acquired a grammatical meaning when combined with other verbs. These are the vector verbs, which will be discussed in Section \formref{sec:wc:Vectorverbs}. Finally, verbs can participate in a number of special constructions, like the perfect construction, which will be discussed in Section \formref{sec:wc:Specialconstructionsinvolvingverbalpredicates}.

\subsection{Canonical verbs}\label{sec:wc:Canonicalverbs}
Sri Lanka Malay has a large class of verbs, which most often denote events like \trs{laari}{run}, but  states like \trs{thiidor}{sleep} can also be found in this class. All verbs share the following characteristics:

\begin{enumerate}
\item They can take the following affixes:
	non-past tense \em arà-\em,
	past tense \em anà-\em,
	past tense \em su-\em,
	anterior tense \em asà-\em,
	irrealis \em anthi-\em,
	infinitive \em mà-\em, as well as some additional minor TAM-affixes.
\item They are negated by preposed \em thàrà- \em for past tense, postposed \em thraa \em for perfect,  preposed \em thama- \em for present and future and \em jamà- \em for nonfinite contexts.
\item They are never negated by \em bukang\em.
\item They must be reduplicated in order to modify other predications
\end{enumerate}

There is a risk to mistake adjectives for verbs because adjectives can easily convert to verbs and take verbal morphology. Therefore, test for adjectivehood should also be done for candidate verbs to exclude this. See further discussion in the section on adjectives \formref{sec:wc:Adjectives}. An example of a verb would be \trs{laari}{run}, which could be inflected as \trs{aràlaari, anàlaari, anthilaari, {\em etc.}}{run, ran, will run, etc.}. This verb would be negated as \em thàràlaari \em for past tense, as \em thamalaari \em for non-past tense, and as \em laari thraa \em in the perfect. \em Laari  \em can never be negated by \em bukang\em, and would have to be used as \em laari\~{}laari \em in order to modify another predication.



\subsection{Existential verbs}\label{sec:wc:Existentialverbs:intro}
There are two existentials,  \em aada \em \formref{sec:wc:Existentialverbs:aada} and \em duuduk \em \formref{sec:wc:Existentialverbs:duuduk}. The latter can only be used with animate referents, while the former  can be used with both animate and inanimate referents. \em Duuduk \em thus carries a positive specification for animacy, while \em aada \em is underspecified for animacy.

Example \xref{ex:Kluumbukaruuma} shows that the animate existential \em duuduk \em cannot combine with the inanimate referent \trs{ruuma}{house}. Here, \em aada \em has to be used. Example \xref{ex:Kluumbukamlaayu} shows that both \em aada \em and \em duuduk \em can combine with the animate referent \trs{mlaayu pada}{Malays}.

\xbox{16}{
\ea \label{ex:Kluumbukaruuma}
\gll Kluu\u mbu=ka ruuma pada \textbf{aada}/*arà-duuduk. \\
     Colombo=\textsc{loc} house \textsc{pl} exist/\textsc{non.past}-exist.\textsc{anim}. \\
    `There are houses in Colombo.' (K081103eli02)
\z
}


\xbox{16}{
\ea\label{ex:Kluumbukamlaayu}
\gll Kluu\u mbu=ka mlaayu pada \textbf{aada/arà-duuduk}. \\
     Colombo=\textsc{loc} house \textsc{pl} exist/\textsc{non.past}-exist.\textsc{anim}. \\
    `There are Malays in Colombo.' (K081103eli02)
\z
}


% \xbox{16}{
% \ea\label{ex:form:aada:anim}
% \gll Appointed MPs             pada  \textbf{anà-duuduk}. \\
%       appointed MPs \textsc{pl} \textsc{past}-exist.\textsc{anim} \\
%     `There were appointed MPs.' (N061031nar01)
% \z
%. \\
% \xbox{16}{
% \ea
% \gll Bannyak Muslim oorang pada araduuduk. \\
%      many Muslim man \textsc{pl} \textsc{non.past}-exist.\textsc{anim}. \\
%     `There are many Muslims (in the Middle East.' (K061026prs01)
% \z
%. \\

\subsubsection{\em aada\em}\label{sec:wc:Existentialverbs:aada}
\em Aada \em is one of the most frequent morphemes in SLM. It is normally pronounced \phonet{aa\dz a} and rarely \phonet{aa\textsubplus{\:r}a}. Most frequently, \em aada \em is used to refer to present situations. In distinction to other verbs, it does not carry the  prefix \em arà- \em in those cases \citep[cf.][169]{SmithEtAl2006cll}.\footnote{This defective status of the existential verb is a common feature of Austronesian languages \citep[138]{Himmelmann2005typochar}.} In other temporal contexts, \em aada \em can take the relevant verbal morphology, just like any other verb, Example \xref{ex:ara:standard} shows the standard use of \em arà-, \em while example \xref{ex:ara:notaada} shows the impossibility to combine \em arà- \em with  \em aada. \em


\xbox{16}{
\ea \label{ex:ara:standard}
\gll Spaaru mlaayu pada \textbf{arà}-oomong. \\
    some Malay \textsc{pl} \textsc{non.past}-speak. \\
    `(Only) some Malays speak (it).' (G051222nar04)
\z
}

\xbox{16}{
\ea \label{ex:ara:notaada}
\gll Avuliya pada \textbf{(*arà-)}aada. \\
     saint \textsc{pl} \textsc{non.past}-exist. \\
    `There are saints.' (B060115nar05)
\z
}


In past contexts, \em aada \em can take the relevant morphology, i.e. the past tense prefixes \em su- \em \xref{ex:form:aada:su1}\xref{ex:form:aada:su2} and \em anà- \em \xref{ex:form:aada:ana1}\xref{ex:form:aada:ana2}.

\xbox{16}{
\ea\label{ex:form:aada:su1}
\gll Hathu muusing=ka \el{} hathu kiccil ruuma \textbf{su-aada} \\
     \textsc{indef} time=\textsc{loc} \el{} \textsc{indef} small house \textsc{past}-exist. \\
    `Once upon a time, there was a small house.'  (K07000wrt04)
\z
}

 \xbox{16}{
 \ea\label{ex:form:aada:su2}
   \gll Se=dang bannyak creeveth pada \textbf{su-aada}. \\
    1s.\textsc{dat} much    trouble  \textsc{pl} \textsc{past}-exist \\
`I had a lot of troubles' (K051213nar01)
\z
}

\xbox{16}{
\ea\label{ex:form:aada:ana1}
\gll {\em Talisman} hatthu \textbf{anà-aada}     kiyang. \\
     talisman \textsc{indef} \textsc{past}-exist \textsc{evid}. \\
    `Apparently, there was a talisman.' (K051206nar02)
\z
}

\xbox{16}{
\ea\label{ex:form:aada:ana2}
\gll Itthukang    \textbf{anà-aada}     [Mr  Janson  katha hathu  oorang]. \\
     then \textsc{past}-exist Mr Janson \textsc{quot} one man. \\
    `Then there was a certain Mr Janson.' (K051206nar04)
\z
}


Examples \xref{ex:form:aada:su:contrast} and \xref{ex:form:aada:ana:contrast} show that there is no difference in the use of the prefix between \em aada \em and other verbal predications as far as the prefixes \em su- \em and \em anà- \em are concerned.


\xbox{16}{
\ea \label{ex:form:aada:su:contrast}
\gll Baapa=le       aanak=le      guula \textbf{su-maakang}. \\
      father=\textsc{addit} son=\textsc{addit} sugar \textsc{past}-eat \\
    `Father and son ate sugar.' (K070000wrt02)
\z
}



\xbox{16}{
\ea\label{ex:form:aada:ana:contrast}
\gll Incayang=pe      plajaran=nang incayang  Kandi=nang   \textbf{anà-dhaathang}. \\
       \textsc{3s.polite} education=\textsc{dat} \textsc{3s.polite} Kandy=\textsc{dat} \textsc{past}-come\\
    `For his education, he came to Kandy.' (K060108nar02)
\z
}


	The following examples show the inflectional potential of \em aada \em in other domains, namely irrealis mood \xref{ex:form:aada:infl:irr}, negation \xref{ex:form:aada:infl:neg} and deontic modality \xref{ex:form:aada:infl:massa}. Example \xref{ex:form:aada:infl:relptl} finally shows the use of \em aada \em  in a relative clause.


\xbox{15}{
\ea\label{ex:form:aada:infl:irr}
\gll [Sebastian=ka su-{\em meet}-king]=nang derang=dang baae=dang suuka \textbf{thi-aada}. \\
 Sebastian=\textsc{loc} \textsc{past}-meet-\textsc{caus}=\textsc{dat}  \textsc{3pl}=\textsc{dat} good=\textsc{dat} like \textsc{irr}=exist\\
`They would have very much liked to meet you.' (B060115cvs07)
\z
}



\xbox{16}{
\ea\label{ex:form:aada:infl:neg}
\gll {\em Confrontation}  pada \textbf{thàrà-aada}    {\em Tigers}=samma. \\
   Confrontation \textsc{pl} \textsc{neg.past}-exist Tigers=\textsc{comit} \\
`There were no confrontations with the Tigers.' (K051206nar20,K081105eli02)
\z
}


\xbox{16}{
\ea\label{ex:form:aada:infl:massa}
\gll Prompang klaaki samma oorang inni=ka        \textbf{masà-aada}. \\
  	girl boy all man \textsc{prox}=\textsc{loc} must=exist  \\
`Boys and girls, everybody must be in this.' (K060116nar05)
\z
}

\xbox{16}{
\ea\label{ex:form:aada:infl:relptl}
\gll [Incayang=pe kàpaala=ka \textbf{anà-aada} thoppi]=dering moonyeth pada=nang su-buvang puukul. \\
      \textsc{3s.polite}=\textsc{poss} head=\textsc{loc} \textsc{past}-exist hat=\textsc{abl} monkey \textsc{pl}=\textsc{dat} \textsc{past}-throw hit \\
    `He took the hat from his head and violently threw it  at the monkeys.'  (K070000wrt01)
\z
}


The negation of \em aada \em if formed with \em thàrà- \em for past contexts, as seen in example \xref{ex:form:aada:infl:neg}, but for present tense, the negative particle \em thraa \em normally substitutes \em aada \em \xref{ex:form:aada:thraa}.

\xbox{16}{
\ea \label{ex:form:aada:thraa}
\gll Se=ppe umma \textbf{thraa}. \\
 \textsc{1s}=\textsc{poss} mother \textsc{neg} \\
`My mother is dead.' (lit. `is not here') (B060115prs03)
\z
}

For negating events in the future, \em thama-aada \em can be used, but this is rarely ever found in the corpus. An elicited examples is given in \xref{ex:form:aada:thamaaada}.


\xbox{16}{
\ea \label{ex:form:aada:thamaaada}
\gll {\em Iceland}=ka mlaayu thama-aada. \\
     Iceland=\textsc{loc} Malay \textsc{neg.irr}-exist. \\
    `There will never be Malays in Iceland.' (K081103eli02)
\z
}

\em Aada \em  plays a key role in a number of constructions, namely the existential construction \xref{ex:aadaexistential}, the locational construction \xref{ex:aada:locational} and the possessive construction \xref{ex:aada:possessive}. Furthermore, \em aada \em is also used in a periphrastic construction with the bare verb or the conjunctive participle to form the perfect tense \xref{ex:aada:perfect}, and in another construction with a verb in the infinitive to form a periphrastic construction that conveys obligation \xref{ex:aada:obligation}. A more detailed discussion of these constructions can be found in Sections \formref{sec:wc:Theperfecttenses} and \formref{sec:wc:Dative+infinitive+V+aada}.


\xbox{16}{
\ea \label{ex:aadaexistential}
\gll Hatthu kumpulan    \textbf{aada}. \\
      \textsc{indef} association exist \\
    `There is an association.'  (K060116nar23)
\z
}


\xbox{16}{
\ea \label{ex:aada:locational}
\gll Mlaayu=ka=jo bannyak avuliya Seelon=\textbf{ka} \textbf{aada}. \\
     Malay=\textsc{loc}=\textsc{emph} many saint Ceylon=\textsc{loc} exist. \\
    `Among the Malays there are many saints in Sri Lanka.'  (K060108nar02)
\z
}

\xbox{16}{
\ea \label{ex:aada:possessive}
\gll Lorang=\textbf{ka}  duvith \textbf{aada}=si? \\
      \textsc{2pl}=\textsc{loc} money exist=\textsc{interr} \\
    `Is the money with you?'  (K060116nar21)
\z
}

\xbox{16}{
\ea \label{ex:aada:perfect}
\gll Uumur=nang kuurang,  sdiikith oorang pada  \textbf{kaving}=le \textbf{aada}. \\
age=\textsc{dat} few few man \textsc{pl} marry=\textsc{addit} exist\\
`Below that age, there are few people who are married, too.' (K061122nar01)
\z
}

\xbox{16}{
\ea \label{ex:aada:obligation}
\gll Lai aapa, lai aapa \textbf{mà}-biilang \textbf{aada}? \\
     other what other what \textsc{inf}-say exist. \\
    `What else (do I) have to say?'  (K060108nar02)
\z
}


Etymologically \em aada \em  stems from the old TM existential \em*ada \em and is thus related to the present tense prefix \em arà- \em \formref{sec:morph:ara-}. This might also be a reason why they cannot co-occur.\\


\subsubsection{\em duuduk\em}\label{sec:wc:Existentialverbs:duuduk}
The second existential verb,  \em duuduk, \em is always inflected as a full verb. The final consonant is eroding, and pronunciations like \phonet{\dz u:\dz uP} and \phonet{\dz u:\dz u} can also be heard (in the South as well \phonet{\dz u:\dz uN} \citet{Slomanson2008ismil}).  Just like \em aada, \em it can be used in existential \xref{ex:form:wc:duuduk:existential}, locational \xref{ex:form:wc:duuduk:locational} and possessive contexts \xref{ex:form:wc:duuduk:possessive}, provided that the referent existing, located or possessed is animate.


\xbox{16}{
\ea\label{ex:form:wc:duuduk:existential}
\gll Miskin oorang arà-\textbf{duuduk}.  Derang pada=nang      samma bole=kaasi \\
      poor man \textsc{non.past}-exist.\textsc{anim} \textsc{3pl} \textsc{pl}=\textsc{dat} all can-give \\
    `Poor people exist. (I) can give everything to them.' (B060115nar04)
\z
}


\xbox{16}{
\ea\label{ex:form:wc:duuduk:locational}
\gll Sithu=\textbf{ka}, hathu bìssar beecek caaya Buruan   su-\textbf{duuduk}.\\ % bf
     there=\textsc{loc} \textsc{indef} big mud colour bear \textsc{past}-exist \\
    `There was a big brown bear there.'  (K070000wrt04)
\z
}


\xbox{16}{
\ea\label{ex:form:wc:duuduk:possessive}
\gll Thiiga klaaki aade=le hatthu pompang aade=le se=\textbf{dang} arà-\textbf{duuduk}.\\
     three male younger.sibling=\textsc{addit} one female younger.sibling 1s.\textsc{dat} \textsc{non.past}-exist.\textsc{anim}. \\
    `I have three younger brothers and one younger sister.' (K060108nar01)
\z
}

\em Duuduk \em cannot be used in either the perfect construction or the debitive construction. It is possible to combine the conjunctive participle of a verb with \em duuduk\em, but this is semantically different from the perfect construction with \em aada\em. In example \xref{ex:aada:perfect}, there is semantically only one predication with \em aada \em only contributing the aspectual value. This is different from \em duuduk \em in  examples \xref{ex:duudukpseudoperfect}, which is a full lexical verb. Example \xref{ex:duudukpseudoperfect} must be interpreted as two predications which follow each other in time.



\xbox{16}{
\ea \label{ex:duudukpseudoperfect}
\gll Baarang \textbf{as}-ambel arà-\textbf{duuduk}. \\
      goods \textsc{cp}-taken \textsc{non.past}-stay \\
    `Having taken the goods, he stayed/sat/was there.' (not:`He has taken the goods.')  (K081103eli02)collect items and wait
\z
}

 \em Duuduk \em has a second meaning, which is `to sit', already mentioned in the example above.
Additionally, it can  mean `stay'. Sinhala also has two different existential verbs, \em tiyenavaa \em (inanimate) and \em innavaa \em (animate). Interestingly, the Sinhala animate verb historically also means `to sit', so that the grammaticalization path of \em duuduk \em can be explained by this adstrate influence.\footnote{Tamil on the other hand does not code animacy in the choice of the existential verb, and Tamil existential verbs do not stem from a historic verb `to sit'.}

While \em duuduk \em has as its etymological origin `to sit', today it can be used as an existential even for animate beings that are unable to perform the act of sitting, for instance bats in \xref{ex:duudukbats}. \em Duuduk \em has thus grammaticalized from a full verb `to sit' to an existential.

\xbox{14}{ 
\ea \label{ex:duudukbats}
\gll Kiccil vavvaal pada daalang=ka    arà-duuduk. \\
       small bat \textsc{pl} inside=\textsc{loc} \textsc{non.past}-exist.\textsc{anim}. \\
`There are small bats inside.'
\z
}

\em Duuduk \em is also present on a second grammaticalization path, namely the ablative. When used in its conjunctive participle form, it is used to denote source of motion \funcref{sec:func:Source}. The  use of the perfective form of an existential for \textsc{source} closely parallels the Sinhala form \em innavaa, i\und alaa \em and the Tamil \em iru(kkiradu), irundu. \em Example \xref{ex:duudukablative} shows this construction.

\xbox{16}{
\ea \label{ex:duudukablative}
\gll Suda see {\em Trinco}=ka  \textbf{asàduuduk} Kluu\u mbu=nang   su-dhaathang. \\
      So \textsc{1s} Trincomalee=\textsc{loc} from Colombo=\textsc{dat} \textsc{past}-come \\
    `So I went from Trincomalee to Colombo.' (K051206nar20)
\z
}

Furthermore, \em duuduk \em can be used as a vector verb to indicate continuous aspect \formref{sec:wc:vv:duuduk}.


\subsection{Defective verbs}\label{sec:wc:Defectiveverbs}
A number of verbs are defective in the sense that they cannot combine with the  prefix \trs{arà-}{non-past}. In addition to \em aada \em discussed above, these are most notably \trs{suuka}{like}{} and \trs{thaau}{know}. In affirmative sentences in the present, they do not take a prefix. \xref{ex:thaunotense} and \xref{ex:suukanotense} give examples for this. These verbs also rarely mark past tense overtly, which is obligatory for canonical verbs.\footnote{The temporal underspecification of `know' is also found in Sinhala   and Tamil.}


\xbox{16}{
\ea \label{ex:thaunotense}
\gll Go=dang baae=nang  \zero{} \textbf{thaau}. \\
 1s.familiar.\textsc{dat} good=\textsc{dat} {} know\\
`I know very well.' (B060115nar04)
\z
}


% \xbox{16}{
% \ea \label{ex:thaunotense2}
% \gll See=yang  samma oorang \zero{} \textbf{thaau}. \\
%        \textsc{1s}=\textsc{acc} all man know\\
%     `Everybody knows me.' (B060115nar04)
% \z
%. \\


\xbox{16}{
\ea \label{ex:suukanotense}
\gll Kithang=nang  \zero{} \textbf{suuka} mà-diyath=nang. \\
       \textsc{1pl}=\textsc{dat} { } like \textsc{inf}-see=\textsc{dat} \\
    `We would like to meet them.' (B060115cvs03)
\z
}



\xbox{16}{
\ea\label{ex:suukanotense2}
\gll Se=dang baapa=ke {\em soldier} mà-jaadi \zero{} \textbf{suuka}. \\
     1s.\textsc{dat} father=\textsc{simil} soldier \textsc{inf}-become { } like. \\
    `I want to become a soldier like daddy.' (B060115prs10)
\z
}

% \xref{ex:thaau:ana} shows the use of the past tense marker  \em anà-/nya- \em on the defective verb \trs{thaau}{know}.
% 
% \xbox{16}{
% \ea\label{ex:thaau:ana}
% \gll Itthu=jo       kithang \textbf{nya-thaau}  oorang nya-aada. \\
%      \textsc{dist}=\textsc{emph} \textsc{1pl} \textsc{past}-know man \textsc{past}-exist. \\
%     `Like that, there were people we knew/know.' (N061031nar01)
% \z
% } \\


\em Anà- \em can be used on \em thaau \em when it is combined with the vector verb \trs{ambel}{take} to yield the meaning of \em got to know.\em
In example \xref{ex:wc:defectiveverbs:anathaauambel}, the allomorph \em nya- \em is used instead of \em anà-\em.

 \xbox{16}{ 
\ea\label{ex:wc:defectiveverbs:anathaauambel}
   \gll Itthukapang=jo derang \textbf{nya-thaau} \textbf{ambel} derang pada {\em politic}=nang suuka katha. \\
    then=\textsc{emph} \textsc{3pl} \textsc{past}-know take \textsc{3pl} \textsc{pl} politic=\textsc{dat} like \textsc{quot} \\
`Only then will they come to know that they like politics' (K051206nar12,K081105eli02)
\z
}

As for negation, the defective verbs have a particular pattern. They are negated by \em thàrà- \em regardless of reference time, which distinguishes them from other verbs, which take \em thama- \em in non-past tenses \xref{ex:thaunegation}\xref{ex:suukanegation}. In the cited examples, the prefix \em thàrà- \em is used despite of the time reference clearly being to the present. It should be noted, however, that for \em suuka\em, an alternative negation is also available. This negation patterns with adjectival negation. The latter pattern can refer to present \xref{ex:suukanegationadjective:present} or past \xref{ex:suukanegationadjective:past}.

\xbox{16}{
 \ea\label{ex:thaunegation}
   \gll Incayang=pe      baapa=pe      naama see \textbf{thàrà-thaau}. \\
    \textsc{3s.polite}=\textsc{poss} father=\textsc{poss} name  \textsc{1s}  \textsc{neg}-know\\
`I do not know his father's name' (K060108nar02)
\z
}



\xbox{16}{
\ea \label{ex:suukanegation}
\gll Luvar   nigiri  kithang=nang   mà-pii    \textbf{thàrà-suuka}. \\
 outside country \textsc{1pl}=\textsc{dat} \textsc{inf}-go \textsc{neg}-like\\
`We do not want to go abroad.' (K051222nar04)
\z
}


\xbox{16}{
\ea \label{ex:suukanegationadjective:present}
\gll Se \textbf{suuka} \textbf{thraa}. \\
 \textsc{1s} like \textsc{neg} \\
`I don't like (it).' (K060116nar08)
\z
}


\xbox{16}{
\ea \label{ex:suukanegationadjective:past}
\gll Bìssar pukurjan see \textbf{suuka} \textbf{thraa}. \\
      big work \textsc{1s} like \textsc{neg} \\
    `I did not want a high post.' (K060116nar08)
\z
}


% K051206nar10.txt:  derang pada thàràsuuka   maambelnang

% \xbox{16}{
% \ea
% \gll Itthu    kithan=nam      butthul=nang mà-biilang \textbf{thàrà-thaau}. \\
%  \textsc{dist} \textsc{1pl}=\textsc{dat} correct=\textsc{dat} \textsc{inf}-say \textsc{neg}.know\\
% `We can't tell you that exactly.' (B060115nar03)
% \z
% }


% \xbox{16}{
% \ea \label{ex:thaunegation2}
% \gll Se=dang thàràthaau {\em interview} katha. \\
%  1s.\textsc{dat} \textsc{neg}.know interview \textsc{quot} \\
% `I didn't know what an interview is.' (K060116nar10)
% \z
% }

%
% \xbox{16}{
% \ea \label{ex:thaunegation2}
% \gll Inni oorang=nang itthu thàràthaau. \\
%       \textsc{prox} man=\textsc{dat} \textsc{dist} not.know \\
%     `The man didn't know that.'  (K07000wrt01)
% \z
%. \\


% \xbox{16}{
% \ea
% \gll Derang duuva oorang  kaala=nang thàrà suuka. \\
%. \\
%     `.' (nosource)
% \z
%. \\

% For use of \em thaau \em with future reference, a periphrastic construction has to be used. This has to do with the fact that stative predicates such as knowing cannot become true out of the blue, but that an accomplishment process leading from state A (ignorance) to state B (knowing) is required. It is then this process which is expressed linguistically in SLM, with help of \trs{pii}{go}.

% 
% \xbox{16}{
% \ea\label{ex:form:unreferenced}
% \gll Kithang=le {\em English} thaau \textbf{tham=pii}. \\
%  \textsc{1pl}=\textsc{addit} English know \textsc{neg=nonpast}=go\\
% `We will not  learn English.' (K060116nar01)(test)
% \z
% }
% 
% 


%
% \xbox{16}{
% \ea\label{ex:form:unreferenced}
% \gll [Sgiini lakuvan de  sindari arà-baa katha] asà-thaau blaakang,  soojer pada  incayang=sàsaama Seelon=nang asà-dhaathang. \\
%  this.much  wealth  \textsc{3s} from.here \textsc{non.past}-take \textsc{quot} \textsc{cp}-know after European \textsc{pl} \textsc{3s} with Ceylon=\textsc{dat} \textsc{cp}-come\\
% ` After having learnt that he brings the wealth from here, the Europeans came to Ceylon together with him (and ...)' (nosource)
% \z
% }
%
% \xbox{16}{
% \ea\label{ex:form:unreferenced}
% \gll {\em The}  davon  arà-kijja      butthul se arà-thaau-wan. \\
%  tea leaf \textsc{non.past}-make very \textsc{1s} \textsc{non.past}-know-nmlzr???\\
% `???.' (K051213nar05)
% \z
% }



%
% \xbox{16}{
% \ea\label{ex:form:unreferenced}
% \gll Tsunaami anà-jaadi     cupathan=nang kithang=nang   thàràthaau. \\
%       tsunami \textsc{past}-become quick-\textsc{nmlzr} \textsc{1s}=\textsc{dat} \textsc{neg}.know \\
%     `We did not know about the speed with which the tsunami came.' (B060115nar02)
% \z
%. \\
%
%
%
% \xbox{16}{
% \ea\label{ex:form:unreferenced}
% \gll Aashik=nang hathu {\em soldier} mà-jaadi suuka=si katha arà-caanya. \\
%      Aashik=\textsc{dat} \textsc{indef} soldier \textsc{inf}-become like=\textsc{interr} \textsc{quot} \textsc{non.past}-ask. \\
%     `He asks if you want to become a soldier, Ashik.' (B060115prs10)
% \z
%. \\




%
% \xbox{16}{
% \ea\label{ex:form:unreferenced}
% \gll Punnu mlaayu oorang=nang=le cinggala mà-blaajar thàrà-suuka=nang. \\
%       full Malay man=\textsc{dat}=\textsc{addit} Sinhala \textsc{inf}-learn \textsc{neg}-like=\textsc{dat} \\
%     `Many Malays do not want to learn Sinhala.' (K051222nar06)
% \z
%. \\
%
% \xbox{16}{
% \ea\label{ex:form:unreferenced}
% \gll Saapa=nang=le ini haadarath masà-thaau. \\
%       who=\textsc{dat}=\textsc{addit} \textsc{prox} ??? must-know \\
%     `Everybody must know this hadarath.' (K061127nar03)
% \z
%. \\


\subsection{Multi-verb constructions}\label{sec:wc:Multi-verbconstructions}
SLM verbs can combine in a number of ways to yield multi-verb constructions. If two verbs are compounded and parsed into one phonological word, we are dealing with verbal compounds \formref{sec:wc:mvc:Verbalcompounds}. If the verbs are parsed into at least two phonological words, there are several possibilities. If one of the verbs is bleached and only provides aspectual or other grammatical information, we are dealing with a vector verb construction  \formref{sec:wc:mvc:Vectorverbs}. If  both of the verbs contribute with their literal meaning, we are dealing with a serial verb construction \formref{sec:wc:mvc:Fullverbserialization}. As a last possibility, which is a multi-clause construction rather than a multi-verb construction, clause chains with the conjunctive participle \em asà- \em also comprise several verbs \formref{sec:wc:mvc:Clausechains}. These four types of multi-verb constructions can then be differentiated by looking at the number of phonological words, the number of full verbs, the number of morphological words, and the number of events \citep{Aikhenvald2006svctp,Dixon2006svccac}. Table \ref{tab:form:mvc} gives an overview.

\begin{table}
\begin{center}
% use packages: array
\begin{tabular}{ccccc}
 & phonological words & full verbs & morphological words & events \\
verbal compound & 1 & 2 & 1 & 1 \\
full verb + vector verb & 2 & 1 & 1 & 1 \\
full verb + full verb & 2 & 2 & 1 & 1 \\
clause chain & 2+ & 2+ & 2+ & 2+
\end{tabular}
\end{center}
\caption[Different types of multi-verb constructions]{Different types of multi-verb constructions in SLM and their characteristics}
\label{tab:form:mvc}
\end{table}


The following sections will discuss prototypical instances of the four types.

\subsubsection{Verbal compounds}\label{sec:wc:mvc:Verbalcompounds}
The string \em kasithaau \em in \xref{ex:form:v:mvc:verbalcompound} is parsed into one phonological word, which can be seen from the absence of a long vowel in \em kasi\em. It includes two full verbs, \trs{ka(a)si}{give} and \trs{thaau}{know}, which both contribute to the meaning with their literal reading. We are dealing with one morphological word, which is only inflected once for TAM, with the debitive prefix \em masà-\em. Finally, the act of informing is only one event, not two separate events of first giving and then knowing.

\xbox{16}{
\ea\label{ex:form:v:mvc:verbalcompound}
\gll Badulla {\em Kandy} Matale samma {\em association}=nang \textbf{masà-kasi-thaau}. \\
      Badulla Kandy Matale samma association=\textsc{dat} must-give-know \\
    `Badulla, Kandy, Matale, we must inform all other associations.' (K060116nar06)
\z
}

Verbal compounds are discussed in more detail in \formref{sec:wofo:Compoundsinvolvingtwoverbs}.

\subsubsection{Vector verbs}\label{sec:wc:mvc:Vectorverbs}
The string \em naangis ambel \em in \xref{ex:form:v:mvc:vectorverb} is parsed into two phonological words, which can be seen from the long vowel in \em naangis\em. This string includes one full verb, \trs{naangis}{weep}, which contributes it literal meaning, and one vector verb, \em ambel\em, which only contributes inchoative aspectual information, and not the literal meaning of `take'. There is no action of taking taking place. We are dealing with one morphological word   \em naangis ambel\em, which can be seen from the TAM-inflection, which is only present for \em naangis \em and not for \em ambel\em. Finally, the crying of the child is only one event, not one event of crying and a separate event of taking.

\xbox{16}{
\ea\label{ex:form:v:mvc:vectorverb}
\gll Suda Andare=pe     aanak=le      baapa  anà-biilang=kee=jo           \textbf{asà-naangis}   \textbf{ambel} su-dhaathang. \\
      thus Andare=\textsc{poss} child=\textsc{addit} father \textsc{past}-say=\textsc{simil}=\textsc{emph} \textsc{cp}-cry take \textsc{past}-come \\
    `So Andare's son started to cry and came (running) just like the father had told him.' (K070000wrt02)
\z
}

Vector verbs are discussed in more detail below in \formref{sec:wc:Vectorverbs}.
 
\subsubsection{Full verb serialization}\label{sec:wc:mvc:Fullverbserialization}
The string \em salba laari \em in \xref{ex:form:v:mvc:fullverbs} constitutes two phonological words. There are two full verbs in it, \trs{salba}{escape} and \trs{laari}{run}, which both equally contribute with their literal reading. We are dealing with one morphological word, indicated by one TAM marking, perfect \em aada\em. We are also only dealing with one event, escaping in a running way, not with one event  of escaping and a further event of running.

\xbox{16}{
\ea\label{ex:form:v:mvc:fullverbs}
\gll Incayang  thee\u mbak abbis,    \textbf{salba}  \textbf{laari} aada  thumpath=nang. \\
      \textsc{3s.polite} shoot    finished escape run   exist place=\textsc{dat}. \\
    `After the shooting he escaped to the (aforementioned) place.' (K051206nar02)
\z
}

Full verb serialization is discussed in more detail in \formref{sec:pred:Verbalpredicateswithtwofullverbs}.

\subsubsection{Clause chains}\label{sec:wc:mvc:Clausechains}
Example \xref{ex:form:v:mvc:clausechain} contains two subordinate clauses where the verb is inflected with the conjunctive participle prefix \em (a)s(à)-\em, and one main clause in the end. All the verbs are full verbs which contribute their literal meaning. We are dealing with much more than one phonological word, and every verb is inflected individually, indicating that we are dealing with three morphological words as far as the verbs are concerned. Finally, we are not dealing with one event, but with three, first fighting, then helping, then settling down.

\xbox{16}{
\ea\label{ex:form:v:mvc:clausechain}
 \ea 
 \gll Oorang pada \textbf{asà}-pirrang \\
	 man \textsc{pl} \textsc{cp}-wage.war. \\
	`After having waged war'
	\ex
 \gll derang=nang \textbf{asà}-banthu \\
	\textsc{3pl}=\textsc{dat} \textsc{cp}-help\\
	`and after having helped them'
	\ex
	\gll siini=jo se-cii\u n\u ggal. \\ % bf
	there=\textsc{emph} \textsc{past}-settle\\
	`the people settled down right here.' (K051222nar03)
	\z
\z
}

Clause chains are discussed in more detail in \formref{sec:cls:Conjunctiveparticipleclause}. 



\subsection{Vector verbs}\label{sec:wc:Vectorverbs}
% " the light verb is always form-identical to a
% main verb in the language …historically
% stable, very much unlike what has been
% documented for auxiliaries." (Butt 2003:15)
Vector verbs  \citep{Pray1970,Hook1974,Masica1991,Kachru1993} (called `auxiliaries' by \citet{SmithEtAl2006cll} for SLM) are verbs which are used after full verbs to highlight a certain semantic aspect of the verb, like perfectivity, intensity or beneficiary.\footnote{Similar verbs in languages of South Asia and beyond have been
 given a variety of names. `Aspectual verbs', `intensifying verbs', `verbal auxiliaries', `explicator verbs', `light verbs'. For SLM, some of these terms are not appropriate: not all of the verbs convey aspectual or intensifying meaning, so that those terms would be  misnomers. `Auxiliary' implies that the verb is used to carry tense or agreement, neither is the case in SLM, so this is a bad term as well. `Light verb' and `Explicator verb'  seem to be viable alternatives to the term `Vector verb' used in this grammar. `Explicator Compound Verb' is actually also found very often \citep{AbbiEtAl1991evc,Abbi1994}, but, in this grammar, `Compound Verbs'\formref{sec:wofo:Compoundsinvolvingtwoverbs} is used for a different construction, so that this name might lead to confusion. `Light verb' is very often used in combination with a noun \citep{Butt2003jungle}, which is not what we find in SLM, hence this term is not used either.}
\citet{AbbiEtAl1991evc} defines these verbs as ``a sequence of at least two verbs where the first member is the main or predicating verb and the second member, although homophonous with an independent verb in the language, does not appear in its primary  lexical meaning; V2 only occurs  in the sequence to mark the main verb V1 for certain `grammatical' features.''

Thus, the verb \em aajar \em means `teach', but complemented with the vector verb \trs{kaasi}{give}, it still means `teach', but highlights the profitable aspect of the teaching. \em Buvang \em means to throw, and combined with the vector verb \trs{puukul}{hit} it means `throw violently'. A characteristic of vector verb is that they always occur after another verbs and that their meaning can be bleached as compared to the full verb. The full verb can carry a TAM-prefix, while the vector verb never does \citep[171]{SmithEtAl2006cll}.\footnote{This is actually a striking difference to many other South Asian languages, where the main verb is typically in the (non-finite) conjunctive participle form, and the vector verb carries tense and agreement \citep[141]{Masica1976}.}

All vector verbs have a corresponding full verb. The number of vector verbs is limited, the following eight vector verbs have been retrieved:\footnote{This list fits well with the list compiled by \citet[173f]{AbbiEtAl1991evc} for common vector verbs in India, which includes GO, COME, GIVE, TAKE, KEEP, PUT, SIT and FALL. FINISH, HIT and STRIKE are additions found in SLM.}

\begin{itemize}
 \item \trs{ambel}{take}{} for (self)-benefactive and ingressive
 \item \trs{kaasi}{give}{} for benefactive
 \item \trs{abbis}{finish}{} for completive
 \item \trs{thaaro}{put}{} for affectedness
 \item \trs{simpang}{keep}{} for continuative
 \item \trs{puukul}{hit}{} for intensity
 \item \trs{duuduk}{sit}{} for progressive
 \item \trs{kìnna}{strike}{} for adversative 
\end{itemize}


\citet{AbbiEtAl1991evc} divide vector verbs into three groups: \textsc{aspectual}, \textsc{adverbial} and \textsc{attitudinal}. It appears that the SLM vector verbs can carry aspectual information (inchoative, completive etc.), adverbial information (intensity, affectedness) and  attitudinal information (adversative).

Examples of  non-aspectual information are found in the following examples: Intensity of the action is expressed by \trs{puukul}{hit} in \xref{ex:wc:vv:puukul}; motivation of the action is expressed by \trs{kaasi}{give} in \xref{ex:wc:vv::kaasi:biilang}.


% \xbox{16}{
% \ea\label{ex:constr:pred:unreferenced}
% \gll Incayang=yang    siaanu  asà-\textbf{buunung}   \textbf{thaaro}=apa. \\
%     \textsc{3s.polite}=\textsc{acc} 3s.prox \textsc{cp}-kill put=after \textsc{quot} \\
%     `This one has killed him.' (K051220nar01)
% \z
%. \\

\xbox{16}{
\ea\label{ex:wc:vv:puukul}
\gll Incayang=pe kàpaala=ka anà-aada thoppi=dering moonyeth pada=nang su-\textbf{buvang} \textbf{puukul}. \\
      \textsc{3s.polite}=\textsc{poss} head=\textsc{loc} \textsc{past}-exist hat=\textsc{abl} monkey \textsc{pl}=\textsc{dat} \textsc{past}-throw hit \\
    `He took the hat from his head and violently threw it violently at the monkeys.'  (K070000wrt01)
\z
}


\xbox{16}{
 \ea\label{ex:wc:vv::kaasi:biilang}
   \gll Kithang=pe     ini      {\em younger} {\em generation}=nang=jo     konnyong masà-\textbf{biilang} \textbf{kaasi}, masà-aajar. \\
    \textsc{1pl}=\textsc{poss} \textsc{prox} younger generation=\textsc{dat}=\textsc{emph} few must-say give must-teach\\
 `It is to the younger generation that we must explain it, must teach it.' (B060115cvs01)
\z
}

Note that  \trs{biilang}{say} and \trs{kaasi}{give} do not differ in their basic valency as both can be used with both a theme and a goal. \em Kaasi \em thus does not serve to introduce a new participant here, but rather   highlights the beneficial nature of the action. This is rephrased at the end of the sentence as \trs{aajar}{teach}. One could be tempted to see \trs{biilang+kaasi}{say+give} as equivalent to \em aajar\em, but this does not capture the entirety of the facts, as \trs{aajar+kaasi}{teach+give} can also be found \xref{ex:wc:vv::kaasi:aajar}.

\xbox{16}{
 \ea\label{ex:wc:vv::kaasi:aajar}
\gll Itthu muusing  Islam igaama  nya-\textbf{aajar} \textbf{kaasi} Jaapna  Hindu {\em teacher}. \\
      \textsc{dist} time Islam religion \textsc{past}-teach give Jaffna Hindu teacher \\
    `At that time, those who taught Islamic religion were Hindu teachers from Jaffna.' (K051213nar03)
\z
}

As in the preceding example, \em kaasi \em is not used to introduce a new participant here, but rather to highlight the beneficial nature of the action. To drive the point home, if \em aajar \em is equivalent to the trivalent construction \em biilang+kaasi \em in example \xref{ex:wc:vv::kaasi:biilang}, then there is certainly no need to use \em kaasi \em again on \em aajar \em in example \xref{ex:wc:vv::kaasi:aajar} to introduce yet another participant.

TAM-prefixes can only attach to the first verb, while postverbal material can only be used after the last verb. No material can intervene between the two verbs. This distinguishes vector verbs from  the similar clause chaining construction \formref{sec:cls:Conjunctiveparticipleclause}, which permits intervening material, as in the following example, where a second TAM-prefix (\em su-\em) separates the two verbs.
 


\xbox{16}{
\ea\label{ex:wc:vv::contr}
\gll Spaaru oorang pada biini pada=yang      \textbf{asà-ambel}   \textbf{su-pii}. \\
     some man \textsc{pl} wife \textsc{pl}=\textsc{acc} \textsc{cp}-take \textsc{past}-go. \\
    `Some men took wives and left/Some men left with wives.' (K051206nar07)
\z
}

It is possible to use vector verbs in clause chains, as the following example shows.

\xbox{16}{
\ea\label{ex:wc:vv::contr:double}
\ea 
\gll Itthu=nang blaakang [inni oorang lìkkas\~{}lìkkas thoppi pada=yang asà-[\textbf{kumpul} \textbf{ambel}]$_{vector}$]$_{chained clause}$ \\
dist after \textsc{prox} man fast\~{}\textsc{red} hat \textsc{pl}=\textsc{acc} \textsc{cp}-collect take. \\
`After that the man quickly picked up his hats and' (K070000wrt01)
\ex
\gll [sithu=ka=dering su-pii]$_{main clause}$. \\
  there=\textsc{loc}=\textsc{abl} \textsc{past}-go\\
`left from there.' (K070000wrt01)
\z
\z
}

In this example, the vector verb combination \trs{kumpul ambel}{collect for oneself} is used within a clause chain \trs{asàkumpul ambel ... supi}{having collected, he left}. The conjunctive participle \em asà- \em  is used only once, on \em kumpul\em, so that we are dealing with two clauses, comprising a total of three verbs. Two are in the dependent clause and one is in the matrix clause. The two verbs in the dependent clause count as a unity for matters of TAM-marking and so the prefix only attaches to the first one.



\subsubsection{\trs{ambel}{take}}\label{sec:wc:ambel}
\em Ambel \em has `take' as its meaning when used as a full verb. When used as a vector verb,  it can be used to highlight the benefactive aspect of a verb. Most often this benefactive aspect applies to the speaker, but it can occasionally also be used for other people profiting from the action. The second use of \em ambel \em as a vector verb is to indicate ingressive aspect.\footnote{\citet{SmithEtAl2004} note that \em ambel \em denotes progressive aspect  \citep[also see][]{Ansaldo2009book}. \citet{Slomanson2008ismil} shows that this reading of \em ambe(l) \em is not related to the verb `take', but a development from  the historical adposition \trs{*sambil}{while}, which is reanalyzed as progressive aspect marker. This marker  seems to be a Southern morpheme and is not found in the Upcountry.}

\em Ambel \em as a vector verb has to be distinguished from both the use as a full verb and from the homonymous postposition found by \citet{Slomanson2008ismil}.

The first use of \em ambel \em is to indicate the profitable nature of the action denoted by the full verb \funcref{sec:func:Beneficiary}.\footnote{The use of the verb TAKE in this function is typical of Indo-Aryan languages \citep[175]{AbbiEtAl1991evc},   found in Sinhala (\citet[44]{Jayawardena2004}, \citet[81]{Matzel1983}), but also in Tamil (\citet[193]{Arden1934}, \citet[95]{Schiffman1999}, \citet{Fedson1993}).} In example \xref{ex:vector:ambel:ben1}, the speaker inquires whether he should ask money from an elder member of the community. Since this money would profit him, the full verb \trs{mintha}{ask} is complemented by \em ambel \em as a vector verb.


\xbox{16}{
\ea\label{ex:vector:ambel:ben1}
\gll [Tony Hassan {\em uncle}=nang anà-kaasi duvith] athi-mintha \textbf{ambel}=si? \\
     Tony Hassan uncle=\textsc{dat} \textsc{cp}-give money \textsc{irr}-ask take=\textsc{interr} \\
    `Shall I ask for the money you gave to uncle Tony Hassan?' (K071011eml01) 
\z
}

The use of \em mintha \em without \em ambel \em is perfectly possible, as the following example shows. \em Ambel \em is thus optional.


\xbox{16}{
\ea  \label{ex:vector:ambel:ben:contrast}
\gll Derang pada arà-mintha \zero{}   nigiri. \\
      \textsc{3pl} \textsc{pl} \textsc{non.past}-ask { } country \\
    `They are asking for a country of their own.' (K051206nar10)
\z
}

Like in  example \xref{ex:vector:ambel:ben1},   example  \xref{ex:vector:ambel:ben2} shows a use of \em ambel \em for an action which profits the agent, namely picking up hats that a group of monkeys had thrown to the ground. The man is a hat-seller and has thus an interest in gathering his hats.

\xbox{16}{
\ea\label{ex:vector:ambel:ben2}
\gll [Itthu=nang blaakang inni oorang lìkkas\~{}lìkkas thoppi pada=yang asà-\textbf{kumpul} \textbf{ambel}], sithu=ka=dering su-pii. \\
dist after \textsc{prox} man fast\~{}\textsc{red} hat \textsc{pl}=\textsc{acc} \textsc{cp}-collect take there=\textsc{loc}=\textsc{abl} \textsc{past}-go\\
`After that, the man quickly picked up his hats and left that place.' (K070000wrt01,K081104eli06)% optional
\z
}

\citet{Slomanson2008ismil} postulates a rule that aspect morphemes are always realized pre-verbally in non-finite clauses. Example \xref{ex:vector:ambel:ben2} shows that this rule does not hold. The subordinate clause, indicated by brackets, contains the verb \trs{kumpul}{collect}, which is marked with the non-finite prefix \em asà\em. The vector verb \em ambel \em contributes aspectual information, but is found \em post\em-verbally in the subordinate clause, invalidating the rule stated by Slomanson.

% The next example also has to do with collecting something, namely friends, an action which surely profits the agent.
% 
% 
% \xbox{16}{
% \ea\label{ex:vector:ambel:ben3}
% \gll Thàthaava thumman pada mà-\textbf{livath-kang} \textbf{ambel}=nang baae=jo mosthor. \\
%       smile=\textsc{emph} friend \textsc{pl} \textsc{inf}-much-\textsc{caus} take=\textsc{dat} good=\textsc{emph} manner \\
%     `Smiling is the best way to make friends (lit. to increase friends).' (K081103eli04)
% 
% \z
% }

In example \xref{ex:vector:ambel:peegang}, the British captured countries, to their profit. The verb \trs{peegang}{catch} is used for this, but the additional use of \em ambel \em makes clear that the British profited from the action.

\xbox{16}{
\ea\label{ex:vector:ambel:peegang}
\gll {\em British} government {\em Malaysia} Indonesia ini nigiri pada samma anà-\textbf{peegang} \textbf{ambel}. \\
     British government Malaysia Indonesia \textsc{prox} country \textsc{pl} all \textsc{past}-catch take. \\
    `The British government captured Malaysia, Indonesia, all these countries.' (K051213nar06,K081104eli06)% optional
\z
}

% Another instance of \em peegang ambel \em is the use of a lawyer in \xref{}
%
% \xbox{16}{
% \ea
% \gll See=le hathu lawyer asà-peegang ambel pii. \\
%. \\
%     `.' (nosource)
% \z
%. \\

This contrast with the use of \em peegang \em in \xref{ex:vector:ambel:peegang:nonben}, where the action is not beneficial.


\xbox{16}{
\ea\label{ex:vector:ambel:peegang:nonben}
\gll {\em Heart} attack asà-\textbf{peegang}   baapa=le       su-nii\u n\u ggal. \\
       heart attack \textsc{cp}-catch father=\textsc{addit} \textsc{past}-die\\
    `My father got a heart attack and died as well.' (K051205nar05,K081104eli06)% ambel not possible
\z
}

Still the use of \em peegang \em is optional even in beneficial context, as the following example shows.



\xbox{16}{
\ea\label{ex:vector:ambel:peegang:ben:contrast}
\gll {\em Singapore}=jona        anà-\textbf{peegang} \zero{}. \\
     Singapore=\textsc{phat} \textsc{past}-catch  {} \\
    `They captured Singapore.' (K051206nar07,K081104eli06) % ambel possible
\z
}

The benefit need not be realized at the time of speaking, as shown by \xref{ex:vector:ambel:fut}, where the wish is expressed that in the future the addressee will be able to enquire from other people the things he is interested in.

\xbox{16}{
\ea\label{ex:vector:ambel:fut}
\gll Incalla   [lai     thaau sudaara sudaari pada]=ka    bole=\textbf{caanya}    \textbf{ambel} [nya-gijja    lai     saapa=kee  aada]=si    katha. \\
      Hopefully other know brother sister \textsc{pl}=\textsc{loc} can-ask take \textsc{past}-make other who=\textsc{simil} exist=\textsc{interr} \textsc{quot} \\
    `Hopefully, you can enquire from another person you know whether there is someone else who did something.' (N061031nar01,K081104eli06)%optional
\z
}

% Another example of modality combined with \em ambel \em is \xref{ex:vector:ambel:masa}, where the children are obliged to understand something for their profit.
% 
% \xbox{16}{
% \ea \label{ex:vector:ambel:masa}
% \gll Aanak pada {\em class} pada=ka=le masà-\textbf{iinngath} [\textbf{ambel}] bedahan thàràbaae simpang (ambel) katha. \\
%      child \textsc{pl} class \textsc{pl}=\textsc{loc}=\textsc{addit} must-think take difference bad keep take \textsc{quot} \\
%     `The children must understand in the classes that it is bad to keep grudges/ all the children in a class should be aware that  there should not be differences between them.' (K061127nar03,K081104eli06) % modified
% \z
% }

Normally, the beneficiary of the action is the agent, but it is also possible that the action denoted by the full verb followed by \em ambel \em is beneficial for another participant. This is the case  in the following example, where the family keep the bear in their home. This  is profitable for the bear because it protects him from the coldl it is not so profitable  for the family.

\xbox{16}{
\ea
\gll See=yang lorang=susamma diinging muusing sangke-habbis anà-\textbf{simpang} \textbf{ambel}. \\
    \textsc{1s}=\textsc{acc} \textsc{2pl}=\textsc{comit} cold season until-finish \textsc{past}-keep take \\
    `You have kept me together with you until the cold season was over.' (K070000wrt04,K081104eli06)% optional
\z
}

The second meaning that can be conveyed by \em ambel \em as a vector verb is ingressive, especially with the verb \trs{thaau}{know}. In the first example, \xref{ex:vector:ambel:thaau:change1}, two fooled women learn about their being fooled. Before they were not aware of this, but now they are, so that there is a change of state taking place, expressed by \em ambel\em.


\xbox{16}{
\ea \label{ex:vector:ambel:thaau:change1}
\gll Kanabisan=ka=jo duva oorang=le anà-\textbf{thaau} \textbf{ambel} [Andare duva oorang=yang=le asà-enco-kang aada] katha. \\
      last=\textsc{loc}=\textsc{emph} two man=\textsc{addit} \textsc{past}-know take Andare two man=\textsc{acc}=\textsc{addit} \textsc{cp}-fool-\textsc{caus} exist \textsc{emph} \\
    `At the very end, both women understood that Andare had fooled both of them.' (K070000wrt05,K081104eli06)% necessary
\z
}

A similar situation obtains in \xref{ex:vector:ambel:thaau:change2}, where the young people will come to know about their interest in politics only when a certain condition is met. Only then will the state change from ignorance to knowledge, again conveyed by \em ambel\em.

\xbox{16}{
 \ea\label{ex:vector:ambel:thaau:change2}
   \gll Itthukapang=jo         derang \textbf{nya-thaau}   \textbf{ambel}, derang pada {\em politic}=nang   suuka katha. \\
    then=\textsc{emph} \textsc{3pl} \textsc{past}-know take \textsc{3pl} \textsc{pl} politic=\textsc{dat} like \textsc{quot} \\
`Only then will they come to know that they like politics' (K051206nar12,K081105eli02)
\z
}

This ingressive meaning can be combined with expression of modality. In \xref{ex:vector:ambel:thaau:change:modality}, the change of state is put into facultative modality.

\xbox{16}{
 \ea\label{ex:vector:ambel:thaau:change:modality}
\ea
\gll Malaysia samma oorang=pe     naama pada   Maas. \\ % bf
Malaysia all man=\textsc{poss} name \textsc{pl} Maas \\
`People from Malaysia are all called ``Maas''.'
\ex
\gll Suda itthu=dering=jo        kithang=nang   ini      Indonesia=pe     oorang=si giithu   kalthraa    {\em Malaysian} oorang=si    katha \textbf{bàrà=thaau} \textbf{ambel}. \\
    thus \textsc{dist}=\textsc{abl}=\textsc{emph} \textsc{1pl}=\textsc{dat}  \textsc{prox} Indonesia=\textsc{poss} man=disj that.way if-\textsc{neg}   Malaysian    man=disj \textsc{quot} can-know take \\
`So with that we can come to know whether someone is Indonesian or otherwise Malaysian' (K060108nar02)
\z

\z}

In some cases, it is difficult to decide whether \em ambel \em is used in the beneficiary sense or the ingressive sense because both are possible.
In \xref{ex:vector:ambel:thaau:changeorben1}, a student writes down the correct answers in an exam after she had seen the answers in dream. \em Ambel \em in this case can both mean that she started to write down, or that the action of writing down was beneficial to her.


\xbox{16}{
\ea \label{ex:vector:ambel:thaau:changeorben1}
\gll Baaye=nang   ingath-an       tak  tak  katha su-\textbf{thuulis}     \textbf{ambel}. \\
      good=\textsc{dat} think-\textsc{nmlzr} tak tak \textsc{quot} \textsc{past}-write take \\
    `She wrote the thoughts down correctly like tak tak tak.' (K051220nar01) % necessary, completive
\z
}


Another example is taking up again well-paid work, which is both a change of state and beneficial.

\xbox{16}{
\ea\label{ex:vector:ambel:thaau:changeorben2}
\ea
\gll Bannyak {\em experience} se=dang {\em engineering} {\em branch}=ka    anà-daapath. \\ % bf
     much experience 1s.\textsc{dat} engineering branch=\textsc{loc} \textsc{past}-get. \\
    `I got a lot of experience in the engineering branch.'
\ex
\gll Suda itthusubbath=jo see laile {\em modern} {\em engineering} asà-\textbf{gijja}   \textbf{ambel} arà-pii. \\
      thus therefore=\textsc{emph} \textsc{1s} again modern engineering \textsc{cp}-make take \textsc{non.past} go \\
    `So therefore I still take up  modern engineering and continue.' (K051206nar20,K081104eli06)% optional
\z
\z
}. \\ 


Finally, in a story where Andare fools the king with the help of his son who starts to cry, \em ambel \em could be interpreted as inceptive, or as beneficial to the cryer, since it furthers  the cause of fooling the king.


\xbox{16}{
\ea
\gll Suda Andare=pe     aanak=le      baapa  anà-biilang=kee=jo           \textbf{asà-naangis}   \textbf{ambel} su-dhaathang. \\
      thus Andare=\textsc{poss} child=\textsc{addit} father \textsc{past}-say=\textsc{simil}=\textsc{emph} \textsc{cp}-cry take \textsc{past}-come \\
    `So Andare's son started to cry and came (running) just like the father had told him.' (K070000wrt02)
\z
}


% 
%          Soore=ka , Snow-white=le Rose-red=le derang=pe umma=samma appi dìkkath=ka arà-duuduk ambel.
% 
%       derang pada nya puuthar ambel
%     B060115cvs01.txt: pada=le      juuval ambel
% 
% kithang Islampe     atthas araoomong    ambel
%         see pukurjan=nang kalu thama-pii  ruuma pukurjan asà-kirja ambel ruuma=ka arà-duuduk
% 
% 

\citet[171f]{SmithEtAl2006cll} give the meaning of this vector verb as progressive, the relevant example is repeated in \xref{ex:form:vector:ambel:smith:29}

\xbox{16}{
 \ea\label{ex:form:vector:ambel:smith:29}
\gll Dera\ng{} daata\ng{}   dupa\ng{} n-duuduk siini$_($,$_)$ Islam ayn-duuduk ambel jo aa\dz a, tu\V an. \\
     3PL come-NOMZ-DAT before past-be here Islam past-be pres foc be sir \\
      `[Islam] was [here] before they came, Islam was [present] here [all along], sir [Lit: was being here, was staying here].'
     (\citet[171(29)]{SmithEtAl2006cll}, original orthography, comma added)
\z
}

My informants found it very hard to make sense out of this example. They offered the following correction:


\xbox{14}{
\ea
\gll Duppang, derang anà-dhaathang duuduk siini, Islam anà-duuduk ambel=jo aada \\
     earlier \textsc{3pl} \textsc{past}-come stay here, Islam \textsc{past}-stay take=\textsc{emph} exist  \\
    `Earlier, they have come here, at that time also they were Muslims.'  (K081104eli06)
\z
}

For the English meaning given by \citet{SmithEtAl2006cll}, they give the following translation:


\xbox{14}{
\ea
\gll Derang siini mà-dhaathang=nang duppang, siini Islam su-aada \\
     \textsc{3pl} here \textsc{inf}-come=\textsc{dat} before here Islam \textsc{past}-exist  \\
    `Before they came here, Islam was (already) here.' (K081104eli06)%Tony Salim
\z
}





% duudup=mastered
% 
% 
% \xbox{16}{
% \ea
% \gll Se=ppe aanak Iz{\em van} {\em computer} ka asà duudup aada. \\
% . \\
%     `My son Izvan has mastered the computer.' (nosource)
% \z
% } \\
% 
% \citet{SmithEtAl2006cll} gloss \em ambel \em as progressive here, which is actually at odds with the overall semantics of the sentence, evidenced as well in the weirdness of the literal English translation.\footnote{They argue that this mirrors the semantics of Tamil, but even in Tamil, the verb \trs{koo}{take} has an inchoative reading, next to the continuous reading they mention \citep[95f]{Schiffman1999}.} There are two major problems with the analysis. First, the gloss of \em ambel \em as `progressive', second the gloss of \em duuduk \em as `be'. Next to the meaning of `exist', \em duuduk \em also means `sit'; \em duuduk ambel \em is a very frequent combination of the `sit'-meaning of \em duuduk \em and the inchoative aspect conveyed by \em ambel\em, together it means `to sit down, to have a seat' (=`to start sitting'). In the context of \xref{ex:form:vector:ambel:smith:29}, it is used metaphorically to mean `established'. A third problem, less important, is that \em Islam \em in this context probably refers to the ethnic group of the Moors (a consequence of Sri Lankan census practices) and not to the religion. The correct translation is then `Before they$_{i}$ came, they$_{j}$ (=Moors) were [already] here. The Moors had already established themselves (=taken a seat), Sir.'

The second example which \citet{SmithEtAl2006cll} use to show the progressive reading of \em ambel \em   is given below in the original orthography.


\xbox{16}{
\ea
\gll Kaaram su\dz a kubaali le ini reepot atu su baa\V u\ng{} ambel aada. \\
      now  link again emph this trouble  one past arise prog be\\
    `Now, see, this trouble is arising again.' (\citet[171(30)]{SmithEtAl2006cll}, original orthography)
\z
}

My informants reacted to this sentence by stating: ``Anybody who has talked to you would have tried to convince you that he knows a bit of Malaysian also.'' While this statement is to be taken with a grain of salt, it illustrates the  puzzlement which the sentence caused. There are some reasons for this:
\begin{itemize}
 \item Given the past tense marker \em su-\em, the gloss as present progressive is surprising. On the other hand, the gloss `was arising' seems problematic as well.
 \item The presence of both a deictic and an indefiniteness marker around \trs{reepot}{trouble} makes little sense.
 \item Why is there marking for past tense (\em su-\em) and perfect tense (\em aada\em) at the same time (similar to English \em have arose \em instead of past \em arose \em or  perfect \em have arisen\em)?
 \item How is the cooccurrence of \trs{kaaram}{now} and the past inflection on the predicate to be explained?
\end{itemize}

When asked how they would render the sentence, the informants gave the following:

\xbox{14}{
\ea
\gll Karang suda laskalli ini reepoth hatthu arà-bavung (ambel) dhaathang\\
     now thus again \textsc{prox} trouble \textsc{indef} \textsc{non.past}-rise take come\\
   `Now, once again, another problem is arising'  (K081104eli06)
\z
}

The use of \em ambel \em is optional. In this sentence, \em ambel \em could have the meaning of progressive, as suggested by \citet{SmithEtAl2006cll}, or the meaning of inchoative, as argued for in the rest of the preceding section.


\subsubsection{\trs{kaasi}{give}}\label{sec:wc:kaasi}
\em Kaasi \em has as its literal meaning  `give' and is used to highlight the beneficial nature of an action for another person than the agent (termed `alterbenefaction' by \citet[227]{Lehmann1989} for Tamil\footnote{Also see \citet[93]{Chater1815} for the analogous Sinhala construction.}) \funcref{sec:func:Beneficiary}. This is the case in the following two examples, which deal with imparting knowledge. Note that \em aajar \em is used as a synonym for \em biilang kaasi \em in \xref{ex:form:vector:kaasi:ben:biilang}, but as \em aajar kaasi \em in \xref{ex:form:vector:kaasi:ben:aajar}, so that a valency changing function of \em kaasi \em is unlikely.

\xbox{16}{
 \ea\label{ex:form:vector:kaasi:ben:biilang}
   \gll Kithang=pe     ini      {\em younger} {\em generation}=nang=jo     konnyong masà-\textbf{biilang} \textbf{kaasi}, masà-aajar. \\
    \textsc{1pl}=\textsc{poss} \textsc{prox} younger generation=\textsc{dat}=\textsc{emph} few must-say give must-teach\\
 `It is to the younger generation that we must explain it, must teach it.' (B060115cvs01)
\z
}

\xbox{16}{
\ea\label{ex:form:vector:kaasi:ben:aajar}
\gll Itthu muusing  [Islam igaama  nya-\textbf{aajar} \textbf{kaasi} \zero{}] Jaapna  Hindu {\em teacher}. \\
      \textsc{dist} time Islam religion \textsc{past}-teach give { } Jaffna Hindu teacher \\
    `At that time, those who taught Islamic religion were Hindu teachers from Jaffna.' (K051213nar03)
\z
}

Another example of the beneficial nature of an action highlighted by \em kaasi \em is \xref{ex:form:vector:kaasi:ben:kaasi2}, where needy people receive soup. \em Kaasi \em is optional, but used to emphasize the benefit.

\xbox{16}{
\ea\label{ex:form:vector:kaasi:ben:kaasi2}
\gll Bannyak=le      {\em soup}=jo    nya-baapi    \textbf{kaasi}. \\
      much soup=\textsc{emph} \textsc{past}-bring give \\
    `We brought (them) a lot of soup.' (B060115nar02)
\z
}

Looking up things for another person is also marked by \em kaasi \em in \xref{ex:form:vector:kaasi:ben:kaasi3}.

\xbox{16}{
\ea\label{ex:form:vector:kaasi:ben:kaasi3}
\gll Se=dang aathi-ka  asà-kluuling bole=caari \textbf{kaasi} {\em Malays} pada=pe atthas. \\
     1s.\textsc{dat} heart=\textsc{loc} \textsc{cp}-roam can-search give Malays \textsc{pl}=\textsc{poss} about \\
    `I am going to go through my mind, and I will forward it to you what I find about the Malays.' (G051222nar01)
\z
}



\subsubsection{\trs{abbis}{finish}}\label{sec:wc:vv:abbis}
This vector verb is used to indicate completive aspect \funcref{sec:func:phase:end} (\citet[171]{SmithEtAl2006cll}, \citet{Ansaldo2009book}). It can occur with a simple or a geminate stop.
It is related to the full verb \em abbis\em, which means `finished'.

Examples \xref{ex:form:wc:vector:abis:fullverb1}-\xref{ex:form:wc:vector:abis:fullverb3} give the use of \em abbis \em as a full verb, the only verb in the respective clauses.


\xbox{16}{
\ea \label{ex:form:wc:vector:abis:fullverb1}
\gll {\em Water} boothol samma \textbf{abbis}. \\
      water bottle all finish \\
    `All water bottles were finished.'  (K051206nar16)
\z
}

\xbox{16}{
\ea\label{ex:form:wc:vector:abis:fullverb2}
\ea
\gll [Ruuma pukurjan \textbf{abbis}]=nang  blaakang, \\
      house work finish=\textsc{dat} after. \\
    `After the house work was finished,'
\ex
\gll Derang [dìkkath=ka     aada  laapang]=nang   mà-maayeng=nang      su-pii. \\
       \textsc{3pl} vicinity=\textsc{loc} exist ground=\textsc{dat} \textsc{inf}-play=\textsc{dat} \textsc{past}-go. \\
    `they went to play to the nearby ground.' (K070000wrt04)
\z
\z
}

\xbox{16}{
\ea\label{ex:form:wc:vector:abis:fullverb3}
\ea
\gll Ini      mlaayu pada siini pìrrang samma      \textbf{abbis}=nang  blaakang, \\
       \textsc{prox} Malay \textsc{pl} here war all finish=\textsc{dat} after \\
    `After these Malays had finished the war,' (K051206nar07)
\ex
\gll derang pada=pe     {\em duty} samma \textbf{abbis}=nang blaakang, ... \\
       \textsc{3pl} \textsc{pl}=\textsc{poss} duty all finish=\textsc{dat} after \\
    `after they had completed their duty, ...' (K051206nar07)
\z
\z
}

The above examples do not show tense marking on the full verb \em abbis\em. Tense marking is often suppressed for \em abbis\em, but the following example indicates that tense marking is possible.


\xbox{16}{
\ea \label{ex:abis:past}
\gll Karang inni     hamma \textbf{s-abis},        bukang. \\
      now \textsc{prox} all \textsc{past}-finish, \textsc{tag} \\
    `Now all was finished, wasn't it?'  (K060116nar11)
\z
}



%
%
% \xbox{16}{
% \ea\label{ex:form:abis:adj:n}
% \gll Buruan \textbf{diinging} \textbf{abbis}=sangke siithu su-sii\u n\u ggal. \\
%       bead cold finish=until there \textsc{past}-stay \\
%     `The bear stayed there until the cold season was over.' (K070000wrt04)
% \z
%. \\


The use as a full verb exemplified above contrasts with the use as a vector verb given in \xref{ex:form:wc:vector:abis:vector1}, where \em abbis \em indicates the completive aspect of the  full verb \trs{rubbus}{boil}.


\xbox{16}{
\ea\label{ex:form:wc:vector:abis:vector1}
\gll Baaye=nang   rubbus \textbf{abbis}. \\
      good=\textsc{dat} boil finish \\
    `When it has boiled well.' (K061026rcp01)
\z
}


Other examples of \em abbis \em as a vector verb  are \xref{ex:form:wc:vector:abis:vector2}, where it indicates completive aspect of the converted adjective \trs{maasak}{cooked}, and \xref{ex:form:wc:vector:abis:vector3}, where it indicates the extent of bending the back which is  necessary to enter the cave.

\xbox{16}{
\ea\label{ex:form:wc:vector:abis:vector2}
\gll Vattakka maasak \textbf{abbis}, ... \\
 pumpkin cooked compl\\
`When the pumpkin was (completely) cooked ...' (K051206nar16)
\z
}

\xbox{16}{
\ea\label{ex:form:wc:vector:abis:vector3}
\gll Giithu=jo      thuu\u nduk \textbf{abbis}=jo       masà-pii. \\
      like.that=\textsc{emph} bent finish=\textsc{emph} must-go \\
    `You must enter there completely crouched.' (K051206nar02)
\z
}


%
% \xbox{16}{
% \ea\label{ex:abis:compl:past2}
% \gll Picca-kang    \textbf{abbis},    samma juval aada. \\
%       break-\textsc{caus} finish all sell exist \\
%     `It's all demolished, they sold all of it.' (K051206nar03)
% \z
%. \\

When combined with \trs{dhaathang}{come}, the meaning is lexicalized as `arrive' as seen in the following examples:

\xbox{16}{
\ea\label{ex:form:wc:vector:abis:dhaathang1}
\gll \textbf{Dhaathang} \textbf{abbis} \el{}  Seelon=ka arà-duuduk \\
 come finish \el{} Ceylon=\textsc{loc} \textsc{non.past}-stay\\
`When they had finished coming (=arrived), they stayed in Ceylon.' (K051206nar07)
\z
}


\xbox{16}{
\ea\label{ex:form:wc:vector:abis:dhaathang2}
\gll Suda derang pada \textbf{dhaathang} \textbf{abbis}=jo derang pada itthu=jo arà-biilang. \\
      thus \textsc{3pl} \textsc{pl} come finish=\textsc{emph} \textsc{3pl} \textsc{pl} \textsc{dist}=\textsc{emph} \textsc{non.past}-say \\
    `So when they had arrived, they said the following:' (K051206nar05)
\z
}

The use of \em abbis \em as a vector verb has to be distinguished from its use as a full verb, given above. Furthermore, there is a less common use of \em abbis- \em as a completive verbal prefix \citep{Slomanson2008lingua}. An example of this is \xref{ex:form:abis:cp}, where \em abbis \em is found in preverbal position.

\xbox{16}{
\ea\label{ex:form:abis:cp}
\gll Itthu \textbf{abbis} \textbf{maakang} kalu  kithang=nang bole=duuduk hatthu=le jamà-maakang=nang  {\em two} duuva {\em two}  {\em o'clock}=ke=sangke bole=duuduk. \\
     \textsc{dist} finish eat if \textsc{1pl}=\textsc{dat} can-stay \textsc{indef}=\textsc{addit} \textsc{neg.nonfin}-eat=\textsc{dat}  two two two o'clock=\textsc{simil}=until can-stay. \\
    `If we eat it up, we can stay up until two o'clock without eating anything.' (K061026rcp04)
\z
}

A similar example is \xref{ex:form:abis:cp2}. It is probable that preverbal \em abbis- \em is a variant (and possibly the etymological source) of the conjunctive participle, normally pronounced \em asà- \em \formref{sec:morph:asa-}.


\xbox{16}{
\ea\label{ex:form:abis:cp2}
\gll Thuuju thaaun luvar   nigiri=ka  asà-duuduk karang \textbf{abbis}-dhaathang   aada. \\
     seven year outside country=\textsc{loc} \textsc{cp}-stay now finish-come exist. \\
    `Having stayed seven years abroad, I have now returned.' (B060115prs13)
\z
}



% \em Abbis \em can also be used as a base for derivation by the causativizer \em -king\em, as shown in the following example
%
% \xbox{16}{
% \ea\label{ex:form:abis:king}
% \gll Suda raaja=le Andare=pe sukahan=yang mà-\textbf{abbis-king}=nang baaye katha su-biilang. \\
%       thus king=\textsc{addit} Andare=\textsc{poss} desire=\textsc{acc} \textsc{inf}-finish-\textsc{caus} good \textsc{quot} \textsc{past}-say \\
%     `So the king agreed to fulfill Andare's wish.' (K070000wrt03a)
% \z
%. \\

%
% \xbox{16}{
% \ea\label{ex:abis:compl:past3}
% \gll Kaaving \textbf{abbis}    derang pada=nang=le        aanak aada. \\
%      marry finish \textsc{3pl} \textsc{pl}=\textsc{dat}=\textsc{addit} child exist. \\
%     `The wedding finished, they also got children.' (K051206nar07)
% \z
%. \\


%
% The completive aspect is less salient in the following example.
%
% \xbox{16}{
% \ea\label{ex:unreferenced}
% \gll Inni     samma dee ambel \textbf{abbis}. \\
%       \textsc{prox} all 3 take finish \\
%     `He stole all that.' (K051206nar02)
% \z
%. \\



\subsubsection{\trs{thaaro}{put}}\label{sec:wc:thaaro}
\em Thaaro \em  has `put' as its literal meaning and highlights the affectedness of the patient \funcref{sec:func:Patient}. \citet[171]{SmithEtAl2006cll} give its meaning as `wilful action with irreversible change of state or condition', which captures the facts presented below very well.\footnote{Also see the discussion of the semantically related Tamil verb \trs{poo\dz u}{put} in \citet[89f]{Schiffman1999}.}

The following two examples show that \em thaaro \em can be used when the change of state is important, detrimental and permanent, such as tearing \xref{ex:wc:vv:thaaro:change}, whereas \em thaaro \em cannot be used when no such permanent detrimental change of state takes place, as with writing, which leaves the book basically unaffected \xref{ex:wc:vv:thaaro:nochange}.


\xbox{14}{
\ea\label{ex:wc:vv:thaaro:change}
\gll Se ini buk arà-soovek-kang thaaro. \\
      \textsc{1s} \textsc{prox} book \textsc{non.past}-torn-caus put \\
    `I am tearing the book to pieces.' (K081104eli06)
\z
}


\xbox{14}{
\ea\label{ex:wc:vv:thaaro:nochange}
\gll Se ini buk arà-thuulis (*thaaro). \\
     \textsc{1s} \textsc{prox} book \textsc{non.past}-writing put \\
    `I am writing this book.' (K081104eli06)
\z
}

 
A naturalistic example of the use of \em thaaro \em as a vector verb is \xref{ex:wc:vv:thaaro:naturalistic}, where the detrimental nature of killing is highlighted by \em thaaro.\em

\xbox{16}{
\ea\label{ex:wc:vv:thaaro:naturalistic}
\gll Incayang=yang    \textbf{siaanu}  asà-buunung   thaaro. \\
    \textsc{3s.polite}=\textsc{acc} 3\textsc{s.prox} \textsc{cp}-kill put \\
    `This one has killed him.' (K051220nar01)(K081103eli02)
\z
}



% \xbox{16}{
% \ea
% \glll See=yang dhaathang {\em {\em remand}}=ka mà-\textbf{thaarek} \textbf{thaaro}=nang thàràboole su-jaadi. \\
%       seeyang derangnang remandka mathaarek thaaro nang tharboole sujaadi\\  	
%      \textsc{1s}=\textsc{acc} come remand=\textsc{loc} \textsc{inf}-pull put=\textsc{dat} cannot \textsc{past}-become. \\
%     `it became impossible for them to put me into remand.' (K061122nar03,K081103eli02)%doubt
% \z
%. \\
% 



\subsubsection{\trs{simpang}{keep}}\label{sec:wc:simpang}
\em Simpang \em indicates that the state-of-affairs is continuous and will persist beyond the reference time frame. An example for this is \xref{ex:wc:vv:simpang}, where the state of sugar lying on a mat to dry will persist for some more time. The example is a bit involved in that it consists of a vector verb construction in a purposive clause (\trs{mà-kìrring simpang}{to keep to dry}), which in turn is a complement of the verbal predicate \trs{siibar}{spread}  in the pluperfect, marked by the conjunctive participle form (\textsc{cp}) \em asà- \em and the past tense form of the existential, \em su-aada\em.


\xbox{16}{
\ea\label{ex:wc:vv:simpang}
\gll Raaja hathu thiikar=ka guula [[asà-siibar [\textbf{mà-kìrring} \textbf{simpang}]$_{purpclause}$]$_{cp}$ su-aada]$_{verbalpredicate}$. \\
     king \textsc{indef} man=\textsc{loc} sugar \textsc{cp}-spread \textsc{inf}-dry keep \textsc{past}-exist. \\
    `The King had sprinkled sugar on a mat and had left it to dry.'
\z
}


% \xbox{16}{
% \ea
% \gll Se spaathu anthi kijja simpang. \\
%. \\
%     `I will make the shoes and keep them ready.' (K081104eli06)
% \z
%. \\

\subsubsection{\trs{puukul}{hit}}\label{sec:wc:puukul}
\em Puukul \em has as its literal meaning `hit'. It is used to indicate that an action was very violent. The only instances are related to throwing an item to \xref{ex:form:vector:puukul1} and fro \xref{ex:form:vector:puukul2}.

\xbox{16}{
\ea\label{ex:form:vector:puukul1}
\gll [Incayang=pe kàpaala=ka anà-aada] thoppi=dering moonyeth pada=nang su-buvang \textbf{puukul}. \\
      \textsc{3s.polite}=\textsc{poss} head=\textsc{loc} \textsc{past}-exist hat=\textsc{abl} monkey \textsc{pl}=\textsc{dat} \textsc{past}-throw hit \\
    `He took the hat from his head and violently threw it  at the monkeys.'  (K070000wrt01,K081104eli06) % volition
\z
}


\xbox{16}{
\ea\label{ex:form:vector:puukul2}
\gll Ithu=kapang ithu moonyeth pada=le [anà-maayeng duuduk thoppi] pada=dering inni oorang=nang su-\textbf{bale-king} \textbf{puukul}. \\
      \textsc{dist}=when \textsc{dist} monkey \textsc{pl}=\textsc{addit} \textsc{past}-play sit hat \textsc{pl}=\textsc{abl} \textsc{prox} man=\textsc{dat} \textsc{past}-return-\textsc{caus} hit \\
    `Then the monkeys threw back the hats with which they had been playing.' (K070000wrt01)
\z
}

% 
% 
% \xbox{16}{
% \ea
% \gll Incayang yang su thiikam (thaaro). \\
%. \\
%     `.' (K081104eli06)
% \z
%. \\
% 
% 
% \xbox{16}{
% \ea
% \gll Ini suurath yang  su thuulis thaaro. \\
%. \\
%     `.' (K081104eli06)
% \z
%. \\
% 
% 
% \xbox{16}{
% \ea
% \gll Se incayangpe leher yang se cikketh (thaaro). \\
%. \\
%     `.' (K081104eli06)
% \z
%. \\





\subsubsection{\trs{duuduk}{sit}}\label{sec:wc:vv:duuduk}
This verb has been discussed extensively above \formref{sec:wc:Existentialverbs:duuduk}. When used as a vector verb, it indicates progressive aspect \funcref{sec:func:phase:progression}. This is the case in \xref{ex:form:vector:duuduk:prog1}, where the playing of the monkeys is ongoing \citep{SmithEtAl2004, Ansaldo2009book}. \citet[175]{AbbiEtAl1991evc} note that the use of SIT as a vector verb is typical of Indo-Aryan and generally not found in Dravidian.

\xbox{16}{
\ea\label{ex:form:vector:duuduk:prog1}
\gll [Dee arà-\textbf{sbuuni}   \textbf{duuduk}     {\em cave}] asaraathang  sini=ka    asàduuduk hathu  {\em three}  {\em miles} cara  jaau=ka. \\
      3\textsc{s.impolite} \textsc{non.past}-hide sit cave] \textsc{copula} here=\textsc{loc} from \textsc{indef} three miles way far=\textsc{loc} \\
    `The cave where he remained hidden is about three miles from here.' (K051206nar02)
\z
}


Another example is \xref{ex:form:vector:duuduk:prog2}, where the continuous nature of the job search is highlighted by \em duuduk\em.

\xbox{16}{
\ea\label{ex:form:vector:duuduk:prog2}
\gll Incayang suda [aapa=ke hathu  pukurjan] mà-girja arà-\textbf{diyath} \textbf{duuduk}. \\
      \textsc{3s.polite} thus what=\textsc{simil} \textsc{indef} word \textsc{inf}-make \textsc{non.past}-try sit \\
    `Now he is looking forward to do some kind or other of work.' (K051222nar08,K081104eli06)
\z
}

\em Duuduk \em can be used in different temporal references, such as past  in \xref{ex:form:vector:duuduk:prog:past} of future in \xref{ex:form:vector:duuduk:prog:future}.

\xbox{16}{
\ea\label{ex:form:vector:duuduk:prog:past}
\gll Loram pada anà-dhaathang vakthu=dika se spaathu \textbf{anà}-gijja \textbf{duuduk}. \\
       \textsc{2pl} \textsc{pl} \textsc{past}-come time=\textsc{loc} \textsc{1s} shoe \textsc{past}-make stay  \\
    `When you came, I was making shoes.' (K081104eli06)
\z
}

\xbox{16}{
\ea\label{ex:form:vector:duuduk:prog:future}
\gll Loram pada arà-dhaathang vakthu=dika se spaathu \textbf{anthi}-gijja \textbf{duuduk}. \\
     \textsc{2pl} \textsc{pl} \textsc{non.past}-come time=\textsc{loc} \textsc{1s} shoe \textsc{irr}-make stay \\
    `When you will come, I will be making shoes.' (K081104eli06)
\z
}
 



\subsubsection{\trs{kìnna}{strike}}\label{sec:wc:vv:kinna}
\em Kìnna \em is a vector verb used to highlight the surprising and adversative nature of the event for the undergoer. When used as a full verb, its meaning is `to strike'. It differs from the other vector verbs by preceding the main verb, where other vector verbs follow.

First, let us discuss the use of \em kìnna \em as a full verb. In example \xref{ex:form:wc:vv:kinna:fullverb}, \em kìnna \em is the only verb, hence a full verb.   This is also confirmed by the  semantics, which involve a clear instance of striking.

\xbox{16}{
\ea\label{ex:form:wc:vv:kinna:fullverb}
\gll Boola {\em wicket}=ka su-kìnna. \\
    ball wicket=\textsc{loc} \textsc{past}-strike \\
    `The ball struck the wicket.' (K081104eli06)
\z
}

A  naturalistic example of \em kìnna \em being used as a full verb is \xref{ex:form:wc:vv:kinna:fullverb:naturalistic}.

\xbox{16}{
\ea \label{ex:form:wc:vv:kinna:fullverb:naturalistic}
\gll Suda itthu=dering de=dang thama-\textbf{kìnna} kiyang  \\
     thus \textsc{dist}=\textsc{abl} 3\textsc{s.impolite}=\textsc{dat} \textsc{neg.nonpast}=strike \textsc{evid}. \\
    `So, because of it [the talisman], (his persecutors' bullets) would not hit him, it seems.'  (K051206nar02,K081105eli02)
\z      
} 


% \xbox{16}{
% \ea
% \gll Kaarang boolanang hatthu pukulan arakìnna. \\
%. \\
%     `Now the ball is going to get a strike.' (nosource)
% \z
%. \\


% \xbox{16}{
% \ea
% \gll Farook dìkka Stewart  arakìnna pukulan. \\
%. \\
%     `Stewart is getting hit from Farook.' (nosource)
% \z
%. \\
% 
%  

% \xbox{16}{
% \ea
% \gll Farook dering Stewart nang  arakìnna pukulan. \\
%. \\
%     `Stewart is getting hit from Farook.' (nosource)
% \z
%. \\


% \xbox{16}{
% \ea
% \gll Farook Stewart nang arà puukul. \\
%. \\
%     `.' (nosource)
% \z
%. \\
 

% 
% \xbox{16}{
% \ea
% \gll Stewartpe spaathu(yang) wicketka sukìnna. \\
%. \\
%     `.' (nosource)
% \z
%. \\

 
Let us now turn to clauses where \em kìnna \em is accompanied by another verb.

\xbox{16}{
\ea \label{ex:form:wc:vv:kinna:vector:daapath}
\gll Sdiikith thaaun=nang duppang see ini Aajuth=nang \textbf{su-kìnna} \textbf{daapath}. \\
      few year=\textsc{dat} before \textsc{1s} \textsc{prox} dwarf=\textsc{dat} \textsc{past}-kìnna get \\
    `Some years before, I was captured by this dwarf.'  (K070000wrt04)
\z      
} 

In this example, the verb \em daapath\em, which normally means `to get', is modified by \em kìnna \em to highlight the surprising and adversative nature of this act of getting, which is then best translated into English as  `to capture'. If the event of getting is not surprising and adversative, as in \xref{ex:form:wc:vv:kinna:vector:daapath:contrast}, \em kìnna \em is not used.\footnote{This only holds if the selection of the speakers as a bridegroom does not come as a bad surprise for the bride.} Importantly, the argument structure remains the same (dative for the recipient, zero for the undergoer). It is therefore not the case that \em kìnna \em changes the syntactic status of arguments, as a passive construction would do. Rather, it contributes a semantic shade of meaning, very much in the way other vector verbs do.

\xbox{16}{
\ea \label{ex:form:wc:vv:kinna:vector:daapath:contrast}
\gll Se=dang bannyak panthas atthu pon su-daapath. \\
    1s.\textsc{dat} much beautiful \textsc{indef} bride \textsc{past}-get. \\
    `I got  a very nice bride (from the matchmaker).' (K081103eli02))
\z
}
 

 
% \ea\label{ex:form:unreferenced}
% \gll Bannyak duvith ini Aajuth=nang \textbf{su-daapath}. \\
%       much money \textsc{prox} dwarf=\textsc{dat} \textsc{past}-get \\
%     `This dwarf got/received much money.'  (K081103eli02)
% \z      
%. \\
 
This `unfortunate' reading of \em kìnna \em  is also found in the following example. The dwarf is grasped and carried away by a big bird.

\xbox{16}{
\ea\label{ex:form:wc:vv:kinna:vector:sìrrath}
\gll Derang su-kuthumung [ithu buurung=pe kuuku=ka Aajuth \textbf{asà-kìnna} \textbf{sìrrath} arà-duuduk]. \\
     \textsc{3pl} \textsc{past}-see \textsc{dist} bird=\textsc{poss} claw=\textsc{loc} dwarf \textsc{cp}-strike stuck \textsc{simult}-stay. \\
    `They saw that the dwarf sat stuck in the claws of the bird.'  (K070000wrt04)
\z      
} 

This contrasts with \xref{ex:form:wc:vv:kinna:vector:sìrrath:contrast}, where the act of getting stuck is (although being unpleasant) not particularly adversative for the undergoer, the shoe.

\xbox{16}{
\ea\label{ex:form:wc:vv:kinna:vector:sìrrath:contrast}
\gll Se=ppe spaathu beecek=ka asà-sìrrath. \\
     \textsc{1s}=\textsc{poss} shoe mud=\textsc{loc} \textsc{cp}-stuck. \\
    `My shoe got stuck in the mud.' (K081105eli02)
\z
}


A further argument against an analysis as syntactic passive comes from the fact that \em kìnna \em can be used with intransitive verbs like \trs{picca}{break down}\footnote{In this case actually a converted adjective. In stative reading, it means `be.broken', in dynamic reading, `to break down'.} \xref{ex:form:wc:vv:kinna:vector:daapath:quake}. The transitive verb \em picca-king \em involves a causitivizer, which is absent here. Generally, intransitives do not lend themselves to passivization in the languages of the world, and if they do, we get an impersonal reading like German \trs{Hier wird getanzt}{Here is being danced=People dance here} \citep[cf.][325]{FoleyEtAl1985infpack}. This is not the case in \xref{ex:form:wc:vv:kinna:vector:daapath:quake}, where the argument \trs{seppe ruuma}{my house} is still present.


\xbox{16}{
\ea \label{ex:form:wc:vv:kinna:vector:daapath:quake}
\gll Buumi ginthar=subbath se=ppe ruuma su-kìnna picca. \\
    earth shake=because \textsc{1s}=\textsc{poss} house \textsc{past}-strike break.down \\
    `My house broke down because of the earth quake.' (K081105eli02)
\z
}

The second component of \em kìnna \em is surprise. An expected adversative result cannot be marked by \em kìnna\em. The following three statements about US presidential elections show this. In 2000, Al Gore lost to George W. Bush by a small margin, after having already declared his victory. In this context, \em kìnna \em is the perfect choice, since the event was surprising and adversative. Note that \trs{kaala}{lose} is intransitive in \xref{ex:form:wc:vv:kinna:vector:Gore}, proving again that  \em kìnna \em cannot be analyzed as a syntactic passive.

\xbox{16}{
\ea\label{ex:form:wc:vv:kinna:vector:Gore}
\gll Al Gore su-kìnna kaala. \\
    Al Gore \textsc{past}-strike lose. \\
    `Al Gore lost (with a small margin).' (K081105eli02)
\z
}


If the event is not adversative, such as victory, \em kìnna \em cannot be used.

\xbox{16}{
\ea
\gll *George W Bush su-kìnna bìnnang. \\
     George W. Bush \textsc{past}-strike win. \\
    (no meaning, unless the victory turned out to be adversative to GWB)(K081105eli02)
\z
}

If the event is adversative, but not surprising, \em kìnna \em cannot be used either. This is the case for the result for the 2008 presidential elections in the USA, where the defeat of John McCain occurred as expected.

\xbox{16}{
\ea
\gll *John McCain su-kìnna kaala. \\
     John McCain \textsc{past}-strike lose. \\
    (John McCain lost surprisingly)(K081105eli02)
\z
}




% \xbox{16}{
% \ea
% \gll Hathu duuri se=ppe thungoorok ka asà tha sìrrath. \\
%. \\
%     `.' (nosource)
% \z
%. \\
% 
% \xbox{16}{
% \ea
% \gll Hathu duuri se=ppe thungoorok ka asà kìnna sìrrath. \\
%. \\
%     `.' (nosource)
% \z
%. \\
% 
% 
% \xbox{16}{
% \ea
% \gll Hathu duuri se=ppe thu\u n\u ggoorok ka se kenna daapath. \\
%. \\
%     `.' (nosource)
% \z
%. \\

% 
% \xbox{16}{
% \ea
% \gll Hathu duuri se=ppe muuluth ka se kenna daapath. \\
%. \\
%     `.' (nosource)
% \z
%. \\
% 
% 
% \xbox{16}{
% \ea
% \gll Hathu raa\u mbuth se=ppe muuluth ka se kenna daapath. \\
%. \\
%     `.' (nosource)
% \z
%. \\
% 
% 
% \xbox{16}{
% \ea
% \gll Gulaaring se=ppe muuluth ka se kenna daapath. \\
%. \\
%     `.' (nosource)
% \z
%. \\

% 
% \xbox{16}{
% \ea
% \gll Ini oorang koocci nang se kìnna daapath. \\
%. \\
%     `.' (nosource)
% \z
%. \\

This element of surprise can also be found in the following three examples, where a person, probably a masochist, gives money to be beaten. If \em kìnna \em is used, the beater cannot be the receiver, because that would not come as a surprise. Rather, the identity of the beater must come as a surprise to the client, i.e. a third person must hit \xref{ex:form:wc:vv:kinna:vector:maso:third}. If, through the use of the emphatic clitic \em =jo\em, the interpretation of identity of receiver and beater is forced, the sentence becomes ungrammatical \xref{ex:form:wc:vv:kinna:vector:maso:jo}.


\xbox{16}{
\ea\label{ex:form:wc:vv:kinna:vector:maso}
\gll Se ini oorang=nang duvith su-kaasi incayang se=dang mà-puukul=nang. \\
     \textsc{1s} \textsc{prox} man=\textsc{dat} money \textsc{past}-give \textsc{3s.polite} 1s.\textsc{dat} \textsc{inf}-hit=\textsc{dat} \\
    `I gave money to this man so that he would hit me.' (K081105eli02)
\z
}

\xbox{16}{
\ea\label{ex:form:wc:vv:kinna:vector:maso:third}
\gll Se ini oorang=nang duvith su-kaasi incayang se=dang mà-\textbf{kìnna} puukul=nang. \\
        \textsc{1s} \textsc{prox} man=\textsc{dat} money \textsc{past}-give \textsc{3s.polite} 1s.\textsc{dat} \textsc{inf}-strike hit=\textsc{dat} \\
    `I gave money to this man so that he would have me hit by a third person by surprise.' (K081105eli02)
\z
}

\xbox{16}{
\ea\label{ex:form:wc:vv:kinna:vector:maso:jo}
\gll *se ini oorang=nang duvith su-kaasi incayang=\textbf{jo} se=dang mà-\textbf{kìnna} puukul=nang. \\
      \textsc{1s} \textsc{prox} man=\textsc{dat} money \textsc{past}-give \textsc{3s.polite}=\textsc{emph} 1s.\textsc{dat} \textsc{inf}-strike hit=\textsc{dat} \\
    `(I gave money to this man so that he himself would hit me.)' (K081105eli02)
\z
}



\em Kìnna \em does not combine well with the expression of voluntary actors, or actors in control. It is thus difficult to combine \em kìnna \em with wilful actions, such as assassinations. \xref{ex:form:wc:vv:kinna:vector:Kennedy:nokinna} shows the normal way of expressing an assassination. If \em kìnna \em is to be used, as in \xref{ex:form:wc:vv:kinna:vector:Kennedy:kinna}, the actor may not be overtly specified, but must be put in a periphrasis.

\xbox{16}{
\ea\label{ex:form:wc:vv:kinna:vector:Kennedy:nokinna}
\gll Oswald Kennedy=yang su-buunung. \\
     Oswald Kennedy=\textsc{acc} \textsc{past}-kill. \\
    `Oswald assassinated Kennedy.' (K081105eli02)
\z
}


\xbox{16}{
\ea \label{ex:form:wc:vv:kinna:vector:Kennedy:kinna}
\gll Oswald=pe thaangang=dering Kennedy(=yang) se-kìnna buunung. \\
      Oswald=\textsc{poss} hand=\textsc{abl} Kennedy=\textsc{acc} \textsc{past}-strike kill \\
    `Kennedy was killed by the hands of Oswald.' (K081105eli02)
\z
}

% 
% \xbox{16}{
% \ea
% \gll Se se=ppe thummanyang suthee\u mbak. \\
%. \\
%     `.' (nosource) 
% \z
%. \\
% 
% 
% \xbox{16}{
% \ea
% \gll Se=ppe thaangang dering thummanyang sukìnna thee\u mbak. \\
%. \\
%     `.' (nosource) accidentally
% \z
%. \\

If no agent is specified for verbs modified by \em kìnna\em, a non-volitional actor, or a non-human actor is automatically assumed. In example \xref{ex:form:wc:vv:kinna:vector:Saddam} and \xref{ex:form:wc:vv:kinna:vector:Saddam:kinna}, we are dealing with an execution. If \em kìnna \em is used for this state-of-affairs, it automatically implies that the entity performing the execution was not volitional. Since executions cannot be done involuntarily, the lack of volition expressed by \em kìnna \em implies that some entity incapable of volition must intervene between the instigator of the execution and the end, such as a machine, the guillotine.

\xbox{16}{
\ea \label{ex:form:wc:vv:kinna:vector:Saddam} 
\gll Saddam Hussein=yang su-buunung. \\
      Saddam Hussein=\textsc{acc} \textsc{past}-kill \\
    `Sassam Hussein was executed.' (K081105eli02)
\z
}

\xbox{16}{
\ea \label{ex:form:wc:vv:kinna:vector:Saddam:kinna}
\gll Saddam Hussein=yang su-kìnna buunung. \\
       Saddam Hussein=\textsc{acc} \textsc{past}-strike kill\\
    `Saddam Hussein was guillotined (or killed by another device, but not directly by a human).' (K081105eli02)
\z
}


This lack of volition expressed by \em kìnna \em can also be seen in the following two sentences, where an explicit statement of volition \trs{kamauvan}{desire} cannot combine with \em kìnna\em.

\xbox{16}{
\ea
\gll Oswald=nang Kennedy=yang mà-buunung kamauvan su-aada. \\
     Oswald=\textsc{dat} Kennedy=\textsc{acc} \textsc{inf}-kill desire \textsc{past}-exist. \\
    `Oswald wanted to kill Kennedy.' (K081105eli02)
\z
}

\xbox{16}{
\ea
\gll *Oswald=nang Kennedy=yang mà-kìnna buunung kamauvan su-aada. \\
      Oswald=\textsc{dat} Kennedy=\textsc{acc} \textsc{inf}-strike kill desire \textsc{past}-exist \\
    `(Oswald wanted to kill Kennedy involuntarily).' (K081105eli02)
\z
}

\em Kìnna \em  is of Malayic descent; today we still find the cognate \em kena \em in Std. Malay \citep{Chung2005kena} and Malay contact varieties \citep{Ansaldo2009book}. \em Kìnna \em is probably also related to the verbal derivational prefix \em kanà- \em \formref{sec:morph:kana-}. This needs further research. The main difference between the two seems to be that \em kanà- \em does not have the adversative component; it only has the involuntary/surprise component.
It might be possible to present a unified analysis of \em kìnna \em and \em kanà-\em. What speaks against this is that they can cooccur in the same predication \xref{ex:form:wc:kinna:kana}, which shows that they are not in complementary distribution.

  
\xbox{14}{
\ea\label{ex:form:wc:kinna:kana}
\gll Itthu haari=ka=jo aanak pompang duuva=nang hathu duuri pohong=nang   jee\u n\u ggoth=yang anà-\textbf{kànà}-daapath \textbf{kìnna} hathu  Aajuth hatthu=yang su-kuthumung. \\
     \textsc{dist} day=\textsc{loc}=\textsc{emph} child female two=\textsc{dat} \textsc{indef} thorn tree=\textsc{dat}   beard=\textsc{acc} \textsc{past}-\textsc{invol}-get strike \textsc{indef}   dwarf \textsc{indef}=\textsc{acc} \textsc{past}-see. \\
    `On the same day, the two girls  saw a    dwarf whose   beard had got stuck  in a thorn tree.' (K070000wrt04)
\z
}




% \xbox{16}{
% \ea
% \gll Allah, se=dang ini naasiyang se kìnna maakang. \\
%. \\
%     `.' (nosource)
% \z
%. \\


% \xbox{16}{
% \ea
% \gll Se=ppe thaangang=dering mangkokyang se kìnna picca. \\
%. \\
%     `.' (nosource)
% \z
%. \\
%  
% 
% 
% \xbox{16}{
% \ea
% \gll Iasà nang Ibrahim pe thaangang dering ma kìnna buunung kamauvan su aada. \\
%. \\
%     `.' (nosource)
% \z
%. \\

% 
% 
% \xbox{16}{
% \ea
% \gll Seeyang ini kaar nang su kìnna bunthur. \\
%. \\
%     `.' (nosource)
% \z
%. \\
% 
% 
% \xbox{16}{
% \ea
% \gll Seeyang ini kaar dering su kìnna bunthur. \\
%. \\
%     `.' (nosource)car's fault
% \z
%. \\
% 
% \xbox{16}{
% \ea
% \gll Seeyang ini kaar ka su kìnna bunthur. \\
%. \\
%     `.' (nosource)in the car, got hit by sth else
% \z
%. \\


 
% 
% \trs{sìrrath}{stuck} does already have the unergative reading,as \xref{ex:kinna:sìrrath:unergative} shows, so that \em kìnna \em cannot be motivated by the need to create that unergative reading. \em kìnna \em is then not motivated syntactically, but semantically, conveying the meaning that the dwarf's being captured was a mishap for him.


%\xbox{16}{
%\ea\label{ex:form:unreferenced}
%\gll Itthusubbath kithang=nang arà-kìnna konnyong  arà-{\em tally} kìnna. \\
%     therefore \textsc{1pl}=\textsc{dat} \textsc{non.past}-kìnna few \textsc{non.past}-tally kìnna \\
%    `.'  (K061026prs01)
%\z      
%} 
% 
% The final example with \em kìnna \em involves a relative clause
% 
% \xbox{16}{
% \ea\label{ex:form:unreferenced}
% \gll [Anà-\textbf{kaageth} \textbf{kìnna} Aajuth] mà-laari=nang su-baalek. \\
%      \textsc{past}-startle kìnna dwarf \textsc{inf}-run=\textsc{dat} \textsc{past}-turn. \\
%     `The startled dwarf took to his heels and ran away.'  (K070000wrt04)
% \z      
%. \\
% 
% As discussed in more detail in \formref{}, relativization is possible on both agents and patients without prior passivization. So \em anà-giigith oorang \em can mean `biting man' or `man being bitten', unlike English \em the biting man\em, which can only mean `the man who is biting', and not `the man who is being bitten'. To return to our example above, \em kìnna \em in the relative clause is not motivated by a syntactic need, but rather expresses the adverse situation the dwarf is in.
% 
% As for the morphosyntactic properties of \em kìnna\em, note that we have \em su-kìnna daapath\em, with \em kìnna \em being the first verb, and TAM being marked on \em kìnna\em, whereas we have \em asà-sìrrath kìnna \em and \em anà-kaageth kìnna\em, with \em kìnna \em being the second word, not receiving any TAM-marking.
% 
% There are two instances of \em kìnna \em in spoken texts:
% 
% \xbox{16}{
% \ea\label{ex:form:unreferenced}
% \gll Deram  pada se-\textbf{kìnna} \textbf{kuurang}. \\
%  \textsc{3pl} \textsc{pl} \textsc{past}-strike few\\
% `They became few.' (K081105eli02)
% \z
%. \\
% 
% 
% \xbox{16}{
% \ea
% \gll Deram pada su kuurang. \\
%. \\
%     `.' (nosource)
% \z
%. \\

% \xbox{16}{
% \ea\label{ex:form:unreferenced}
% \gll Inni {\em {\em Power}} samma kuurang blaakang, deram pada se-\textbf{kenna} \textbf{picca}. \\
%  \textsc{prox} . all few after \textsc{3pl} \textsc{pl} \textsc{past}-kìnna broken\\
% `After that power had become less, they became broken.' (nosource)
% \z
%. \\


% 
% We see again the adversative component. And again, \em kìnna \em does not have a syntactic effect. \em derram pada se-kuurang \em is a perfectly good sentence, with identical semantics, as is \em deram pada sepicca\em.


% \xbox{16}{
% \ea\label{ex:form:unreferenced}
% \gll Giithu=jo      ini      {\em Malays} pada asà-kànà      {\em spread}. \\
%       like.that=\textsc{emph} \textsc{prox} Malays \textsc{pl} \textsc{cp}-patfoc spread \\
%     `Like that the Malays spread.' (G051222nar03)
% \z
%. \\

% \ref K061026prs01
%  suda spaaru vakthu kithang a...
% \mb suda spaaru vatthu kithang *a...
% \ge thus some   time   1pl     ****...
% 
%  once  a    monthke
% \mb *once *a   *month =ke
% \ge ****  **** ****   -SIMIL
% 
%  arameetkìnna
% \mb arà-  meet -king =nang
% \ge tns- LOAN -CAUS -DAT
% 
% \ft So we meet sometimes, like once a month




\subsection{Special constructions involving verbal predicates}\label{sec:wc:Specialconstructionsinvolvingverbalpredicates}
Verbs are used in a number of periphrastic constructions which yield TAM meanings. The perfect tenses are formed with a verb followed by the existential \em aada \em \formref{sec:wc:Theperfecttenses}. Obligational constructions can be formed by a verb in the infinitive followed by \em aada \em \formref{sec:wc:Dative+infinitive+V+aada} or \trs{jaadi}{become} \formref{sec:wc:Dative+infinitive+V+jaadi}.

\subsubsection{The perfect tenses}\label{sec:wc:Theperfecttenses}
The perfect is formed by a verb followed by the existential \em aada \em in the affirmative and its negation \em thraa \em in the negative (cf. \citet[143]{Slomanson2006cll},
\citet{Slomanson2008lingua},
\citet[169f]{SmithEtAl2006cll},
and \citet{Ansaldo2009book}).
The verb can either be in the stem form \xref{ex:perfect:bare}, or the conjunctive participle prefix \em asà- \em can be prefixed \xref{ex:perfect:asa} (\citet[143]{Slomanson2006cll},\citet{Slomanson2008lingua}). Normally, the existential is present, but this is not always the case. The perfect is used to refer to events located in the past. No particular aspectual reading is implied \funcref{sec:func:Beforespeechact}.

\cb{ 
	\NP	(\textit{asà}-)V
	$\left\{
		\begin{array}{l}
						(aada) \\thraa
		\end{array}
		\right\}$           
	 
}


%\xbox{16}{
%\ea\label{ex:constr:pred:unreferenced}
%\gll Punnung kaaving aada bungaali. \\
%many marry exist Bengali \\
%`Many have married Bengalis.' (K051206nar08)
%\z
%}

\xbox{16}{
\ea\label{ex:perfect:bare}
\gll Uumur=nang kuurang,  sdiikith oorang pada \textbf{\zero-}kaving=le \textbf{aada}. \\
age=\textsc{dat} few few man \textsc{pl} marry=\textsc{addit} exist\\
`Below that age, there are few people who are married, too.' (K061122nar01)
\z      
}

\xbox{16}{
\ea\label{ex:perfect:asa}
\gll Itthu=le kitham=pe mlaayu pada=jo  itthu thumpath samma \textbf{asà-}kaasi \textbf{aada}. \\
but \textsc{1pl}=\textsc{poss} Malay \textsc{pl}=\textsc{emph} \textsc{dist} place all \textsc{cp}-give exist \\
`Therefore it was our Malays who have  all given away those lands.' (B060115cvs04)
\z
}


%B060115cvs07.txt:  kocci  puttu  smaakang      aadasi
%
%G051222nar01.txt: Gamapaha=ka    duuduk aada

This construction is very frequent. Both Sinhala and Tamil have an analogous construction consisting of the conjunctive participle and an existential, but lack the construction with the bare verb instead of the conjunctive participle. The use of cognates of \em aada \em is also widespread in other Malay varieties \citep{AdelaarEtAl1996,Bakker2006}.
 
% It might also be possible to use \em duuduk \em with animates in this construction. This is very infrequent.
% 
% \xbox{16}{
% \ea\label{ex:constr:pred:perf:duuduk1}
% \gll {\em Father}  Rolands  se=ppe      aade kaaving  arà-duuduk. \\ % bf
%      Father Rolands \textsc{1s}=\textsc{poss} younger.sibling marry \textsc{non.past}-exist.\textsc{anim}. \\
%     `Father Roland married my younger sister.'  (B060115nar01)
% \z      
%. \\
%  
% 
% \xbox{16}{
% \ea\label{ex:constr:pred:perf:duuduk2}
% \gll Australian girl hatthu=yang   asà-kaaving   arà-duuduk. \\ % bf
%      Australian girl \textsc{indef}=\textsc{acc} \textsc{cp}-marry \textsc{non.past}-exist.\textsc{anim}. \\
%     `.' (K051206nar18)
% \z
%. \\{use for biclausal cp+duuduk, slomanson2008}

The perfect tense is negated by replacing the existential with \em thraa\em, as in \xref{ex:constr:pred:perf:neg}.

 

\xbox{16}{
\ea\label{ex:constr:pred:perf:neg}
\gll Hatthu dhaatha \textbf{asà-kaaving} \textbf{thraa}. \\
      one elder.sister \textsc{cp}-marry \textsc{neg} \\
    `One elder sister has not married.' (K061019prs01)
\z
}
 
The pluperfect consists of the same element as the perfect construction, but with the existential in the past tense, instead of the present tense. Example  \xref{ex:constr:pred:perf:pluperf}   illustrates this for \em su-\em. No instance of \em anà-aada \em in this construction has been found.

\cb{  
	\NP	(\textit{asà}-)V 
	$\left\{
		\begin{array}{l}
						su- \\an\grave{a}-?
		\end{array}
		\right\}$          		\textit{aada}
}



\xbox{16}{
\ea\label{ex:constr:pred:perf:pluperf}
\gll {\em Letter}=ka thaaro \textbf{s-aada}. \\
 letter=\textsc{loc} put \textsc{past}-exist\\
`(I) had put it on the letter.' (K060116nar10)
\z
}

Note that the use of the past tense marker is not obligatory, the normal perfect tense can also be used where other languages, like English, might require the use of the pluperfect. An example for this is given in \xref{ex:constr:pred:perf:pluperf:nopluperf}, where the action of giving their word is clearly anterior to the point in time the narrative is dealing with, but still the perfect construction is used.

\xbox{16}{
\ea\label{ex:constr:pred:perf:pluperf:nopluperf}
\ea 
\gll Derang pada kathahan thama-thuukar. \\
      \textsc{3pl} \textsc{pl} word \textsc{neg.nonpast}-change \\
    `They would not change their word.'  
\ex
\gll Karang  {\em British} government=nang  derang kathahan \textbf{asà-kaasi} \textbf{aada}. \\
      now British Government=\textsc{dat} \textsc{3pl} word \textsc{cp}-give exist \\
    `Now, they had given their word to the British government.' (K051213nar06)
\z
\z
}
 


The combination of the perfect construction with the irrealis markers \em anthi \em (affirmative) \xref{ex:constr:pred:antirr:pos} \formref{sec:morph:anthi-}and \em thama \em (negative) \xref{ex:constr:pred:antirr:neg}\formref{sec:morph:thamau-} yields the anterior irrealis construction.


\cb{ 
	\NP{}
 (\textit{asà}-)V 
 	$
	\left\{
		\begin{array}{l}
		anthi-\\
		thama-\\
		\end{array}
	\right\}
	$
		\textit{aada}
}


\xbox{16}{
\ea\label{ex:constr:pred:antirr:pos}
\gll Bìssar aanak \textbf{asà-dhaathang} \textbf{anthi-aada} ruuma=nang. \\
      big child \textsc{cp}-come \textsc{irr}=exist house=\textsc{dat} \\
    `My big child will have come home.' (B060115cvs08)
\z
}

\xbox{16}{
\ea\label{ex:constr:pred:antirr:neg}
\gll Inni bedahaan  \textbf{jaadi}=jo \textbf{thuma=aada}. \\
 \textsc{prox} difference become=\textsc{emph} \textsc{neg.nonpast}=exist\\
`These differences would not have arisen.' (K060116nar03)
\z
}


The anterior irrealis is not found very frequently. Tamil has an analogous construction \citep[207]{Lehmann1989}, while in Sinhala, such a construction cannot be found because of the lack of a verbal irrealis marker.

% se pohongka asà naayek arà duuduk I have climbed a tree % 1106 01

\subsubsection[\textsc{dat + inf + V} + \textit{aada}]{Dative+infinitive+V+\em aada\em}\label{sec:wc:Dative+infinitive+V+aada}
Besides the formation of the perfect tense, the existential is also used in an obligational construction \funcref{sec:func:Obligation}, where it combines with a verb in the infinite. The entity for which the obligation holds is marked with the dative. The Kleene star * marks that an element can be present 0, 1, or more times.

\cb{ \NP* NP=\textit{nang}  \textit{mà}-V (TAM)-\textit{aada}}

Example \xref{ex:constr:pred:v:datinfaada:intro} illustrates this construction with the dative, the infinitive and the existential highlighted.

\xbox{16}{
\ea\label{ex:constr:pred:v:datinfaada:intro}
\gll Kithang=pe sudari pada=\textbf{nang} makanan \textbf{mà}-maasak \textbf{aada}. \\
      \textsc{1s}=\textsc{poss} sister \textsc{pl}=\textsc{dat} food \textsc{inf}-cook exist \\
    `Our sisters all have to cook food.' (B060115cvs01)
\z
}

The person under the obligation need not be expressed overtly if he or she in inferable from discourse. This is the case for the speaker in \xref{ex:constr:pred:v:datinfaada:noref}.

\xbox{16}{
\ea\label{ex:constr:pred:v:datinfaada:noref}
\gll Lai aapa lai aapa \zero{} \textbf{mà}-biilang \textbf{aada}. \\ % bf
     other what other what { }  \textsc{inf}-say exist. \\
    `What else (do I) have to say?'  (K060108nar02)
\z      
}

Occasionally, one of the elements can be dislocated to a position to the right of \em aada \em. This is the case for the complement clause in \xref{ex:constr:pred:v:datinfjaadi:disloc}. The underscore indicates the position where this clause would normally be expected.

\xbox{16}{
\ea\label{ex:constr:pred:v:datinfjaadi:disloc}
   \gll Se=dang \_\_\_  aada  [ini      {\em army} pada=yang   mà-salba-kang=nang]$_{CLS}$. \\ % bf
    1s.\textsc{dat} {} exist \textsc{prox} army \textsc{pl}=\textsc{acc} \textsc{inf}-save-\textsc{caus}=\textsc{dat} \\
`I had to save these soldiers' (K051213nar01)
\z
}

This  construction with \em aada \em has a component of duty to it imposed by law or moral values, which distinguished it from general obligation by \em masthi- \em and from the following construction with \em jaadi \em which do not require (\em masthi\em) or allow (\em jaadi\em) the duty component.


\subsubsection[\textsc{dat + inf + V} + \textit{jaadi}]{Dative+infinitive+V+\em jaadi\em}\label{sec:wc:Dative+infinitive+V+jaadi}
This periphrastic construction can also be used to express obligation \funcref{sec:func:Obligation}. It is very similar to the construction with \em aada\em, but \em aada \em is replaced by \em jaadi\em. Its structure is as follows:


\cb{ \NP* NP=\textit{nang}  \textit{mà}-V TAM-\textit{jaadi}}

The three important elements of this construction, the dative, the infinitive and \em jaadi \em are highlighted in the following example.


\xbox{16}{
\ea\label{ex:constr:pred:v:datinfjaadi:intro}
\gll Se=ppe    {\em profession}=subbath se=\textbf{dang}  siini  \textbf{mà}-pii    su-\textbf{jaadi}. \\
      \textsc{1s}=\textsc{poss} profession=because 1s=\textsc{dat} here \textsc{inf}-go \textsc{past}-become \\
    `I had to come here because of my profession.' (G051222nar01)
\z
}

This use of \trs{jaadi}{become} must not be confounded with the literal use, which can also occur with a dative. This is shown in \xref{ex:constr:pred:v:datinfjaadi:contrast}.

\xbox{16}{
\ea\label{ex:constr:pred:v:datinfjaadi:contrast}
\gll Se=dang  buthul seksa pada anà-jaadi. \\ % bf
      1s.\textsc{dat} correct problem \textsc{pl} \textsc{past}-become \\
    `I got a lot of problems.' (K051213nar01)
\z
}

The person experiencing the obligation need not be mentioned if it can be established from discourse. This is the case for \trs{kithang}{we} in \xref{ex:constr:pred:v:datinfjaadi:drop}.


\xbox{16}{
\ea\label{ex:constr:pred:v:datinfjaadi:drop}
\gll Blaakang, \zero{} cinggala \textbf{mà}-blaajar su-\textbf{jaadi}. \\
     after { } Sinhala \textsc{inf}-learn \textsc{past}-become. \\
    `Then (we) had to learn Sinhala.'  (K051222nar06)
\z      
}


Besides the past tenses illustrated above, this construction can also be used with irrealis inflection \em athi- \em on \em jaadi\em.


\xbox{16}{
\ea\label{ex:constr:pred:v:datinfjaadi:irr}
 \gll [Saapa=pe=ke  thaangang di baava]=ka=jo pukurjan \textbf{mà}-gijja athi-\textbf{jaadi}. \\
  who=\textsc{poss}=\textsc{simil}  hand \textsc{link} under=\textsc{loc}=\textsc{emph} work \textsc{inf}-make \textsc{irr}-become \\
`You will always have to work under someone's command' (K051206nar07,K081104eli06)% modified from original
\z      
}

The following two examples also show this construction and are given as additional illustration.

\xbox{16}{
\ea\label{ex:constr:pred:v:datinfjaadi:extra1}
   \gll Suda derang\textbf{=nang}   hathyang muusing=sangke \textbf{mà}-duuduk  su-\textbf{jaadi}. \\
    thus \textsc{3pl}=\textsc{dat} other time=until \textsc{inf}-stay \textsc{past}-become\\
`So they had to wait until another time' (K051220nar01)
\z
}

\xbox{16}{
\ea\label{ex:constr:pred:v:datinfjaadi:extra2}
\gll Ini thaaun=ka derang sama oorang=\textbf{nang} {\em England}=nang \textbf{mà}-pii su-\textbf{jaadi}. \\
     \textsc{prox} year=\textsc{loc} \textsc{3pl} all man=\textsc{dat} England=\textsc{dat} \textsc{inf}-go \textsc{past}-become. \\
    `All men had to go to England this year.' (K051222nar06)
\z
}

The construction with \em jaadi \em conveys some kind of circumstantial obligation. Due to  circumstances, a certain obligation arises. This is different from the construction with \em aada \em discussed above, where some kind of moral authority is responsible for the obligation, and it is also different from general obligation by \em masthi\em, which does not carry any specification as to whether the obligation arose because of circumstances or not.


% K051205nar05.txt: kithang=pe     umma=nang     thàràsìggar    su-jaadi
 
% \xbox{16}{
% \ea\label{ex:constr:pred:unreferenced}
% \gll Spaaru oorang pada kubbong pada anà-jadi-king. \\
%      some man \textsc{pl} garden \textsc{pl} \textsc{past}-become-cause. \\
%     `Some people grow in the gardens.' (K051213nar07)
% \z
%. \\
 
% 
% 
% \subsection{Stacking of verbal modifiers}\label{sec:wc:Stackingofverbalmodifiers}
% stacking modals
%     se=dang kìrras bole laari
%     pàrthaama, se=dang bannyak kìrras bole laari
%     duppang nang le se=dang bannyak kiiras bole laari
%     se=dang ma laari anà ambel boole nang duppang, oorang pada se=dang anthathaawa
%     kaake ma jaalang ambel thàrboole nang duppang, bannyak kìrras anà laari
%         before my grandfather became unable to walk, he ran very fast


\section{Nouns}\label{sec:wc:Nouns}
SLM nouns can be divided into common nouns \formref{sec:wc:Commonnouns}, relator nouns indicating figure-ground relations \formref{sec:wc:Relatornouns}, and proper nouns \formref{sec:wc:Propernouns}.

\subsection{Common nouns}\label{sec:wc:Commonnouns}
Sri Lanka Malay has a large open class of nouns, which most often denote objects like \trs{ruuma}{house} or persons like \trs{aanak}{child}, but also some abstract concepts like \trs{vatthu}{time}. Many abstract concepts are derived from another word class by means of the nominalizer \em -an\em, like \trs{makanan}{food} from \trs{maakang}{eat}.
All nouns share the following characteristics:

\begin{enumerate}
\item They cannot take the following affixes:
	present tense \em arà-\em,
	past tense \em anà-\em,
	past tense \em su-\em,
	anterior tense \em asà-\em,
	irrealis \em anthi-\em,
	infinitive \em mà-\em.
\item  They are always negated with \em bukang\em, never with \em thàrà- \em or \em thama-\em.
\item  They must combine with \em =nang \em in order to modify a predication
\item Many nouns can also combine with \trs{pada}{\textsc{plural}}{} and the deictics \trs{ini}{proximal}{} and \trs{itthu}{distal}.
\end{enumerate}

These criteria will now be discussed in turn.

Example \xref{ex:searaoorang} shows the impossibility of nouns to combine with the present tense marker \em arà-\em. Combination with other TAM-markers is not possible, either.

\xbox{16}{
\ea \label{ex:searaoorang}
\gll *se arà-oorang. \\
      \textsc{1s} \textsc{non.past}-man \\
     (word salad, no possible meaning)
\z
}


% \xbox{16}{
% \ea \label{ex:nouns:seavuliya}
% \gll Se  avuliya hatthu masà-jaadi. \\
%      \textsc{1s} \textsc{indef} saint must-become. \\
%     `I want/must become a saint.'  (nosource)
% \z
%. \\

Nouns in ascriptive position are negated by \em bukang\em.\footnote{Another negator often found in the vicinity of nouns is \em thraa\em, which negates the existence or the availability of the item denoted by the noun in question. But the scope of this negator is the existential predicate, not the nominal argument thereof, whereas \em bukang \em negates the nominal predicate.}

\xbox{16}{
\ea \label{ex:n:bukang}
\gll Deram Islam oorang \textbf{bukang}. \\
 \textsc{3pl} Islam man \textsc{neg.nonv} \\
`They were not Muslims.' (K051213nar03)
\z
}

\xbox{16}{
\ea
\gll See innam blaakang, hatthu aanak \textbf{bukang}. \\
     \textsc{1s} \textsc{prox.dat} after \textsc{indef} child \textsc{neg.nonv} \\
    `I will never be a child again.' (K081106eli01) 
\z
}


The verbal negators \em thàrà- \em or \em thama- \em are not possible.

\xbox{16}{
\ea\label{ex:form:nouns:tharaSindbad}
\gll *Sindbad  thàrà-Islam oorang. \\
      {Sindbad the sailor} \textsc{neg.past}-Moor man \\
     (Sindbad was no Muslim) (K081103eli02)
\z
}

When nouns are used as ascriptive predicates on arguments in the singular, they must obligatorily carry the indefiniteness marker \em hatthu \em \xref{ex:nouns:Farookmaaling}. This differentiates them from adjectives \xref{ex:nouns:Farookkiccil}, which can be used without \em hatthu\em.

\xbox{16}{
\ea \label{ex:nouns:Farookmaaling}
\gll Farook *(atthu) maaling. \\
     Farook \textsc{indef} thief. \\
    `Farook is a thief.'  (K081103eli02)
\z
}

\xbox{16}{
\ea \label{ex:nouns:Farookkiccil}
\gll Farook kiccil. \\
     Farook small. \\
    `Farook is small.'  (K081103eli02)
\z
}

Nouns can be used to modify predications (`adverbial function').  For this, they have to take the dative marker \em =nang \em \xref{ex:nouns:svaara}. This distinguishes them from adjectives, where the use of \em =nang \em is optional \xref{ex:nouns:pullam}.


\xbox{16}{
\ea \label{ex:nouns:svaara}
\gll Incayang svaara$_N$*(=nang) arà-oomong. \\ % bf
     \textsc{3s.polite} noise=\textsc{dat} \textsc{non.past}-speak. \\
    `He speaks loud.' (K081103eli02)
\z
}


\xbox{16}{
\ea \label{ex:nouns:pullam}
\gll Incayang pullang$_{ADJ}$(=nang) arà-oomong. \\ %
     \textsc{3s.polite} slow=\textsc{dat} \textsc{non.past}-speak. \\
    `He speaks softly.' (K081103eli02)
\z
}

Many, but not all nouns can combine with the plural marker \em pada \em and the deictics \em ini \em and \em itthu\em. This distinguishes nouns from verbs, which cannot combine with these markers. It does not distinguish them  from adjectives, some of which have the same possibilities. Example \xref{ex:form:noun:ini} shows the use of the noun \trs{baarang}{goods} combined with the proximal deictic \em ini\em. Example \xref{ex:form:noun:padaitthu} shows the combination of the nouns \trs{vatthu}{time}{} and \trs{nigiri}{country}{} with the plural markers and the distal deictic \em itthu\em.


\xbox{16}{
\ea\label{ex:form:noun:ini}
\gll \textbf{Ini}      \textbf{baarang} pada=yang      asà-baapi   laayeng nigiri=ka     anà-juval. \\
      \textsc{prox} goods \textsc{pl}=\textsc{acc} \textsc{cp}=bring other country=\textsc{loc} \textsc{past}-sell \\
    `He brought the goods and sold them in other countries.' (K060103nar01)
\z
}

\xbox{16}{
\ea\label{ex:form:noun:padaitthu}
\gll \textbf{Itthu}    \textbf{vatthu}=ka    \textbf{itthu}    \textbf{nigiri}  \textbf{pada}=ka    arà-duuduk. \\
     \textsc{dist} time=\textsc{loc} \textsc{dist} land \textsc{pl}=\textsc{loc} \textsc{non.past}-stay \\
    `At that time, (they) lived in those countries.'  (N060113nar01)
\z
}

Nouns are also often found with postpositions marking case as in \xref{ex:form:nouns:case}, but this does not distinguish them from the other word classes because in SLM, case markers can combine with members from any word class \formref{sec:morph:Postpositions}.


\xbox{16}{
\ea  \label{ex:form:nouns:case}
\gll Itthu    baathu$_{N}$=\textbf{yang}    incayang  Seelong$_{N}$=\textbf{dering}         laayeng    nigiri$_{N}$=\textbf{nang} asà-baapi. \\
 \textsc{dist} stone=\textsc{acc} \textsc{3s.polite} Ceylon=\textsc{abl} other country=\textsc{dat} \textsc{cp}-bring\\
`These stones, he brought them from Ceylon to other countries.' (K060103nar01)
\z
}

\subsection{Relator nouns}\label{sec:wc:Relatornouns}
\begin{table}
	\centering
		  \begin{tabular}{lll}
			relator noun & referential meaning & relational meaning\\
\hline
  			blaakang & back & behind, after\\
  			duppang & front & in front of, before, ago\\
			baa(wa)   & bottom & below, under\\
			atthas & top & on, above, about\\
			thìnnga & middle & in\\
			daalang & inside & in\\
			luvar & outside & out\\
			vatthu & time & during\\
			dìkkath & vicinity & at\\
			mosthor & manner & like\\
  		\end{tabular}
 			\caption{Relator nouns}
  		\label{tab:RelatorNouns}
\end{table}

A subset of nouns can be used as relator nouns \citep{Starosta1985, DeLancey1997relator} to indicate (mostly spatial) relationship between two referents \funcref{sec:func:Figure-groundrelations} \citep[25]{Adelaar1991}. This is a typical South Asian feature \citep[23]{Masica1976}.
These relator nouns can be used in postpositional phrases to indicate the relation between figure and ground more precisely than with a simple locative \em =ka\em.

Relator nouns and the NP denoting the ground are either simply juxtaposed as in \xref{ex:reln:juxta1}\xref{ex:reln:juxta2} or they are joined with the possessive \em =pe \em on the ground NP  \xref{ex:reln:poss1}-\xref{ex:reln:poss3}\citep[62]{Saldin2001}.\footnote{It is possible for different  relator nouns to require/permit/ban a  possessive/genitive, as shown  by \citet{DeLancey1997relator} for Tibetan and Tamang.}

\xbox{16}{
\ea \label{ex:reln:juxta1}
\gll Ini \textbf{pohong} \zero{} \textbf{atthas}=ka moonyeth hathu kavanan su-aada. \\
     \textsc{prox} tree { } top=\textsc{loc}  monkey \textsc{indef} group \textsc{past}-exist\\
    `On top of this tree was a group of monkeys.'   (K070000wrt01)
\z
}


\xbox{16}{
\ea \label{ex:reln:juxta2}
\gll Soore=ka, Snow-white=le Rose-red=le derang=pe umma=samma \textbf{appi} \zero{} \textbf{dìkkath}=ka arà-duuduk ambel. \\
      Evening=\textsc{loc} Snow.white=\textsc{addit} Rose.Red=\textsc{addit} \textsc{3pl}=\textsc{poss} mother=\textsc{comit} fire { } vicinity=\textsc{loc} \textsc{non.past}-sit take \\
    `In the evening, Snow White and Rose Red used to sit down next to the fire with their mother.'  (K070000wrt04)
\z
}


\xbox{16}{
\ea \label{ex:reln:poss1}
\gll Andare hathu \textbf{pohong=pe} \textbf{baava}=ka kapang-duuduk. \\
     Andare \textsc{indef} tree=\textsc{poss} bottom=\textsc{loc} when-sit. \\
    `When Andare sat down below a tree.' ((K070000wrt03))
\z
}

\xbox{16}{
\ea \label{ex:reln:poss2}
\gll Bìssar hathu buurung \textbf{derang=pe} \textbf{atthas}=dering su-thìrbang. \\
      big \textsc{indef} bird \textsc{3pl}=\textsc{poss} top=\textsc{abl} \textsc{past}-fly \\
    `A big bird flew over them.'  (K070000wrt04)
\z
}


\xbox{16}{
\ea \label{ex:reln:poss3}
\gll \textbf{Derang} \textbf{pada=pe} \textbf{atthas} kithang=nang bannyak mà-biilang thàràboole. \\
      \textsc{3pl} \textsc{pl}=\textsc{poss} top \textsc{1pl}=\textsc{dat} much \textsc{inf}-say cannot \\
    `We cannot tell you much about them.'  (K051206nar12)
\z
}


Relator nouns are not simply postpositions, because they can occur without a host as in \xref{ex:form:reln:hostless:daalang}, where \trs{daalang}{inside} does not refer to the interior of the immediately preceding constituent \trs{vavvaal pada}{bats},\footnote{This word is a loanword from Tamil, which explains its uncommon word structure with the long vowel in the last syllable.} but rather to the interior of a cave mentioned earlier in discourse.

\xbox{16}{
 \ea\label{ex:form:reln:hostless:daalang}
   \gll Kiccil vavvaal pada \zero{} \textbf{daalang=ka}  arà-duuduk. \\
     small  bat     \textsc{pl}   { } inside=\textsc{loc} prog.exist.\textsc{anim} \\
`There are small bats inside.' (K051206nar02)
\z
}

Another example of a relator noun occurring without a host is \xref{ex:form:reln:hostless:baava}, where the ground to which \trs{baava}{bottom} relates is the house, which is present in discourse, but not mentioned in the sentence.

\xbox{16}{
\ea\label{ex:form:reln:hostless:baava}
\gll Goo \zero{} baava=ka=jo arà-duuduk, \zero{} baava=ka. \\ % bf
      1s.\textsc{familiar} { } bottom=\textsc{loc}=\textsc{emph} \textsc{non.past}-live { } bottom=\textsc{loc} \\
    `I live downstairs.' (B060115cvs13)
\z
}



Relator nouns can also attach to pronouns \xref{ex:reln:pron} and declausal NPs \xref{ex:reln:np}. This is more often found with temporal and causal relator nouns, and less often with local relator nouns. The reason for this is of course that clauses typically denote non-concrete entities, which cannot be located in space.



\xbox{16}{
\ea \label{ex:reln:pron}
\gll \textbf{Se=ppe} \textbf{blaakang} arà-raathang \el{} se=ppe aanak klaaki pada. \\
1s=\textsc{poss} after \textsc{non.past}-come \el{} \textsc{1s}=\textsc{poss} child boy \textsc{pl} \\
`After me are coming my sons.' (K060108nar02)
\z
}

\xbox{16}{
\ea\label{ex:reln:np}
\gll [[Mlaayu pada anà-dhaathang]$_{CLS}$]$_{NP}$=pe \textbf{atthas} se=dang hatthu=le mà-biilang thàràboole. \\
       Malay \textsc{pl} \textsc{past}-come=\textsc{poss} about \textsc{1s=dat}  \textsc{indef}=\textsc{addit} \textsc{inf}-say cannot\\
    `I cannot tell you anything about the coming of the Malays.'  (K081105eli02)
\z      
}

% 
% \xbox{15}{
% \ea \label{ex:reln:np}
	% \gll [Ini oorang giini kapang-jaalang pii caape] \textbf{subbath} jaalang hathupii\u n\u ggirr=ka anà-aada hathu pohong \textbf{baava}=ka su-see\u nder. \\
%     \textsc{prox} man this.way when-walk go tired because road \textsc{indef} border=\textsc{loc} \textsc{past}-exist.\textsc{inanim} \textsc{indef} tree down=\textsc{loc} \textsc{past}-rest. \\
%     `Because he was tired from walking then, this man sat down under a tree which stood at the side of the street.'  (K070000wrt01)
% \z
% }

% 
% Relator nouns also always have a lexical counterpart. That lexical counterpart is fully referential, and can be modified, e.g. by numerals as in \xref{ex:reln:subla}.
% 
% \xbox{16}{
% \ea \label{ex:reln:subla}
% \gll Ruuma duuva subla=ka. \\
%      house two side=loc. \\
%     `On both sides of the house.'  (K070000wrt04)
% \z
% }

In the following, I will discuss some relator nouns in more detail. I will give occurrences with \zero{}-marking, occurrence with \em =pe\em, occurrence with \em =nang\em, relation to space and relation to time if examples for these cases could be found. I give the referential/free form of a relator noun as well, and the use as an adverb, if this exists.

\subsubsection{\trs{blaakang}{back}}\label{sec:wc:blaakang}
This relator noun is used in spatial context with \em =pe \em \xref{ex:form:reln:pe:space} and  in temporal contexts with \em =nang \em \xref{ex:form:reln:blaakang:nang:time}. Additionally, it is also possible to use \em blaakang \em without a postposition. \xref{ex:form:reln:blaakang:zero}.


\xbox{16}{
\ea \label{ex:form:reln:pe:space}
\gll Kandi=ka {\em Malay} {\em mosque}$_{space}$\textbf{=pe} \textbf{blaakang}=ka incayang=pe zihaarath aada. \\
     Kandi=\textsc{loc} Malay mosque=\textsc{poss} back=\textsc{loc} \textsc{3s.polite}=\textsc{poss} shrine exist. \\
    `In Kandy behind the Malay Mosque, there is his shrine.'
\z
}

% \xbox{16}{
% \ea \label{ex:form:blaakang:pe:temp}
% \gll Kithangpe     umma   anà nii\u n\u ggal=\textbf{pe}  \textbf{blaakang} asaraathang   goo samma goo suluppas aanak padanang   samma supii. \\
%     kithangpe umma mamnii\u n\u ggalnang blaakang asà dhaathang seeyang asaluppas aanak pada sàsaama supii. \\
%     `.' (B060115nar04,K081103eli02)(wrong)
%     `'
% \z
% } \\

\xbox{16}{
\ea\label{ex:form:reln:blaakang:nang:time}
\gll Dr Draaman duuva thaaun$_{time}$=\textbf{nang} \textbf{blaakang} incayang su-mnii\u n\u ggal. \\
 Dr Draaman two year=\textsc{dat} after \textsc{3s.polite} \textsc{past}-die\\
`After two years, Dr Draman died.' (K051213nar08)
\z
}



\xbox{16}{
\ea\label{ex:form:reln:blaakang:zero}
\gll Itthu \zero{} \textbf{blaakang} lai [se=dang kalu] bannyak itthu=pe atthas thàràthaau. \\
       \textsc{dist} { } after other 1s.\textsc{dat} if much \textsc{dist}=\textsc{poss} about \textsc{neg}-know\\
    `After that, as for me, I cannot tell you much.' (K051205nar02)
\z
}

When used referentially, \em blaakang \em means `back'; for the human anatomical part, this has to be complemented by \trs{thìnnga}{middle} \xref{ex:form:reln:blaakang:referential}.


\xbox{16}{
\ea\label{ex:form:reln:blaakang:referential}
\gll Se=ppe \textbf{thìnnga} \textbf{blaakang} arà-saakith. \\
 \textsc{1s}=\textsc{poss} middle back \textsc{non.past}-hurt\\
    `My back hurts.' (K081103eli02)
\z
}

Example \xref{ex:form:reln:blaakang:adverb} shows the use of \em blaakang \em as an adverb to indicate the spatial orientation of \trs{diyath}{look}.

\xbox{16}{
\ea\label{ex:form:reln:blaakang:adverb}
\gll \textbf{Blaakang}=yang diyath! \\
back=\textsc{acc} look \\
    `Look behind!' (K081103eli02)
\z
}

Another adverbial use of \em blaakang \em is to indicate sequence of events, like English \em then \em or \em after that\em. This use is shown in \xref{ex:form:reln:blaakang:hostless}.

\xbox{16}{
\ea\label{ex:form:reln:blaakang:hostless}
\ea
\gll [Thàrà-dhaathang oorang pada]=nang   nya-{\em force}-kang   kiyang. \\ % bf
       \textsc{neg.past}-come man \textsc{pl}=\textsc{dat} \textsc{past}-force-\textsc{caus} \textsc{evid} \\
    `They apparently forced the people who did not come.'
\ex
\gll \textbf{Blaakang}$_{adv}$ thàrà-pii   kiyang. \\
       after \textsc{neg.past}-go \textsc{evid} \\
    `(But) after (that) (they) apparently (still would) not go.' (K051206nar07)
\z
\z
}

%seppe thìnnga blaakang arà saakith % K081103eli02)
\subsubsection{\trs{duppang}{front}}\label{sec:wc:duppang}
This relator noun can indicate spatial location in front of the ground. In this case, it is construed with \em =pe \em \xref{ex:form:reln:duppang:pe:spatial}.


\xbox{14}{
\ea\label{ex:form:reln:duppang:pe:spatial}
\gll Ruuma=pe duppang pohong aada  \\
    house=\textsc{poss} front tree exist   \\
    `There are trees in front of the house.' (K081103eli02)
\z
}

As for temporal relations, two constructions involving \em duppang \em have to be distinguished. When used with \em =nang\em, \em duppang \em means `before, ago, earlier' \xref{ex:form:reln:duppang:nang} \funcref{sec:func:Precedence}. When construed with \em =pe\em, it means `future' \xref{ex:form:reln:duppang:pe:fut}. In the former case, \em duppang \em thus refers to a past event, while in the latter case, it refers to the future.


\xbox{16}{
\ea \label{ex:form:reln:duppang:nang}
\gll Kithang=\textbf{nang} \textbf{duppang} lai duuva bàrgaada asà-dhaathang aada. \\
 \textsc{1pl}=\textsc{dat} before other two family \textsc{cp}-come exist \\
`Before us, there were two other families.' (K060108nar02)
\z
}


\xbox{16}{
\ea\label{ex:form:reln:duppang:pe:fut}
\gll Kithang=\textbf{pe}  \textbf{duppang} muusing(=yang)   arà-dhaathang. \\
      \textsc{1pl}=\textsc{poss} future time(=\textsc{acc}) \textsc{non.past}-come. \\
    `Our future is approaching.' (N061031nar01, K081103eli02)
\z
}

% 
% \xbox{16}{
% \ea
% \gll Seelon=pe duppang muusing=yang bannyak panthas anthi aada. \\
%      Sri Lanka=\textsc{poss} future time=\textsc{acc} much beautfiul \textsc{irr}-exist. \\
%     `Sri Lanka's future will be bright.' (K081103eli02)
% \z
% } \\

% duppang
%     incayang pe duppangnang creeveth thraa
%         in his future, there will be now problems
%     incayang nang duppang, lai bergaada aada
%         before him, there are other generations


% 
% \xbox{16}{
% \ea\label{ex:form:reln:duppang:zero}
% \gll Itthu muusing {\em Railway} mà-dhaathang \zero{} \textbf{duppang}. \\
%  \textsc{dist} time railway \textsc{inf}-come {  } before\\
% `That was before the railway came.' (K05120nar02)
% \z
% }
%

When used referentially, \em duppang \em is translated as `front' \xref{ex:form:reln:duppang:ref}.

\xbox{14}{
\ea\label{ex:form:reln:duppang:ref}
\gll Laile          ithu     ruuma pada aada, \textbf{duppang}=ka    sajja sdiikith laayeng-kang   aada. \\
     still \textsc{dist} house \textsc{pl} exist front=\textsc{loc} only few different=\textsc{caus} exist. \\
`Those houses still remain, only the front portion has been renovated.' (K051206nar02)
\z
}
 

\xbox{16}{
\ea\label{ex:form:reln:duppang:hostless}
\gll Sithuka     aada,  duppang$_{adv}$=ka    aada. \\ % bf
      \textsc{dist}=\textsc{loc} exist before=\textsc{loc} exist \\
    `There are some there, there are some in front.' (K051206nar02)
\z
}



\subsubsection{\trs{atthas}{top}}\label{sec:wc:atthas}
This relator noun  has two meanings. The first, literal, is `top', the second one, `about, on' \funcref{sec:func:Theme}.
\em Atthas \em is very often found without \em =pe \em \xref{ex:reln:atthas:top:zero}, but the occurrence of \em =pe \em is also possible \xref{ex:reln:atthas:top:pe}.


\xbox{16}{
\ea \label{ex:reln:atthas:top:zero}
\gll Ini \textbf{pohong}  \zero{} \textbf{atthas}=ka moonyeth hathu kavanan su-aada. \\
     \textsc{prox} tree { } top=\textsc{loc}  monkey \textsc{indef} group \textsc{past}-exist\\
    `On top of this tree was a group of monkeys.'   (K070000wrt01)
\z
}


\xbox{16}{
\ea \label{ex:reln:atthas:top:pe}
\gll Ruuma=\textbf{pe}     \textbf{atthas}=ka=jo       derang asà-duuduk     aada,  laile          itthu    ruuma  pada aada \\
      house=\textsc{poss} top=\textsc{loc}=\textsc{emph}  \textsc{3pl} \textsc{cp}-exist.\textsc{anim} exist still \textsc{dist} house \textsc{pl} exist  \\
`Above the house some others have stayed, the houses still remain.'
\z
}

The relator noun \em atthas \em can in turn host lative information, perlative in \xref{ex:reln:atthas:top:pe:lative}.

\xbox{16}{
\ea  \label{ex:reln:atthas:top:pe:lative}
\gll Bìssar hathu buurung derang=pe \textbf{atthas}=\textbf{dering} su-thìrbang. \\
      big \textsc{indef} bird \textsc{3pl}=\textsc{poss} top=\textsc{abl} \textsc{past}-fly\\
    `A big bird flew over them.'  (K070000wrt04)
\z
}



% \xbox{16}{
% \ea
% \gll Puunu   mlaarath,  guunung$_{space}$  \zero{} atthas=ka. \\ % bf
%      full difficult mountain { } top=loc. \\
%     `(That) is very difficult on the mountain tops.' (K051206nar02)
% \z
% } \\

Used referentially, \em atthas \em means `top' or `upstairs' \xref{ex:form:reln:atthas:top:hostless}.


\xbox{16}{
\ea\label{ex:form:reln:atthas:top:hostless}
\gll \textbf{Atthas}=ka    cumma baarang pada=jo. \\ % bf
      top=\textsc{loc} idle goods \textsc{pl}=\textsc{emph} \\
    `Upstairs, there is only rubbish.' (B060115cvs13)
\z
}

The second, figurative, meaning of \em atthas \em is `about', as given in \xref{ex:form:reln:atthas:about:nominal} for a nominal host and in \xref{ex:form:reln:atthas:about:clausal} for a clausal host.

\xbox{16}{
\ea\label{ex:form:reln:atthas:about:nominal}
\gll Kithang Islam=\textbf{pe} \textbf{atthas} arà-oomong ambel. \\
     \textsc{1pl} Islam=\textsc{poss} about \textsc{non.past}-speak take. \\
    `We talk about Islam.' (K061026prs01)
\z
}

\xbox{16}{
\ea\label{ex:form:reln:atthas:about:clausal}
\gll [[Mlaayu pada anà-dhaathang]=$_{CLS}$]$_{NP}$=pe atthas se=dang hatthu=le mà-biilang thàràboole. \\
       Malay \textsc{pl} \textsc{past}-come=\textsc{poss} about \textsc{1s=dat}  \textsc{indef}=\textsc{addit} \textsc{inf}-say cannot\\
    `I cannot tell you anything about the coming of the Malays.'  (K081105eli02) 
\z      
}

% \xbox{16}{
% \ea\label{ex:form:reln:atthas:about:pe2}
% \gll Se=dang kalu suda bannyak thàràthaau  inni=pe         atthas mà-biilang=nang. \\
%      \textsc{1s=dat} if thus much neg.know \textsc{prox=poss} about \textsc{inf}-say=\textsc{dat}. \\
%     `So, as for me, I cannot tell you much about this.' (K051205nar04)
% \z
% }


% \xbox{16}{
% \ea\label{ex:form:unreferenced}
% \gll Itthu kithang=nang mà-biilang thàrà-boole, itthu {\em difference}=pe atthas. \\
%       \textsc{dist} \textsc{1pl}=\textsc{dat} \textsc{inf}-say cannot \textsc{dist} difference=\textsc{poss} about \\
%     `We cannot tell you anything about that, about that difference.' (K061026prs01)
% \z
%. \\



\subsubsection{\trs{baa(va)}{bottom}}\label{sec:wc:baa(wa)}

This relator noun  indicates position below/under another entity. It must not be confounded with the verb \trs{baa}{bring}. Like the other relator nouns, \em baava \em can be construed with \em =pe \em \xref{ex:form:reln:baava:pe} or without \xref{ex:form:reln:baava:zero}.


\xbox{16}{
\ea\label{ex:form:reln:baava:pe}
\gll Seelong su-aada soojor=\textbf{pe} \textbf{baava}=ka. \\
       Ceylon \textsc{past}-exist European=\textsc{poss} under=\textsc{loc} \\
    `Sri Lanka was under European rule.' (K051222nar06)
\z
}

\xbox{16}{
\ea\label{ex:form:reln:baava:zero}
\gll [Jaalang hathu  pii\u n\u ggir=ka    anà-aada     hathu  pohong]$_{space}$ \zero{} \textbf{baava}=ka  su-see\u nder. \\
      street \textsc{indef} border=\textsc{loc} \textsc{past}-exist \textsc{indef} tree { } bottom=\textsc{loc} \textsc{past}-rest \\
    `(He) rested under a tree which stood at the side of the street.' (K070000wrt01)
\z
}


It can be used without a host to indicate location. This use can be seen either as referential (`on the ground') \xref{ex:form:reln:baava:ref} or adverbial (`below') \xref{ex:form:reln:baava:adv}, but the two uses are difficult to tell apart.


\xbox{16}{
\ea\label{ex:form:reln:baava:ref}
\gll Dee su-thiidor     \textbf{baava}$_{adv}$=ka. \\
     \textsc{3s} \textsc{past}-sleep bottom=loc. \\
    `He slept downstairs.' (K051205nar05)
\z
}


\xbox{14}{
\ea\label{ex:form:reln:baava:adv}
\gll \textbf{Baava}=ka   Kaasim  katha hatthu \textbf{{\em family}}. \\
      below=\textsc{loc} Kaasim \textsc{quot} \textsc{indef} family \\
    `Below, there is a family called ``Kaasim''.' (K051206nar06)
\z
}


% \xbox{16}{
%  \ea\label{ex:form:reln:baava:baa}
%    \gll Lorang pada asàdhaathang  saapa=pe=ke          \textbf{baa}   thaangang=ka=jo       pukurjan mà-gijja    athi-jaadi     katha. \\
% \textsc{2pl} \textsc{pl} \textsc{copula} who=\textsc{simil} under hand=\textsc{loc}=\textsc{emph} work \textsc{inf}-make \textsc{irr}-become \textsc{quot} \\
% `You will always have to work under someone's command' (K051206nar07)
% \z
% }

% \xbox{16}{
% \ea\label{ex:form:reln:baava:hostless}
% \gll Goo \zero{} \textbf{baava}$_{adv}$=ka=jo arà-duuduk, \zero{} \textbf{baava}$_{adv}$=ka. \\
%       1s.\textsc{familiar} { } bottom=\textsc{loc}=\textsc{emph} \textsc{non.past}-live { } bottom=\textsc{loc} \\
%     `It is  downstairs that I live, downstairs.' (B060115cvs13)
% \z
% } \\

\subsubsection{\trs{daalang}{inside}}\label{sec:wc:daalang}

This relator noun  refers to the inside of an entity. A use with \em =pe \em is given in \xref{ex:form:reln:daalang:pe}, an example without \em =pe \em is given in \xref{ex:form:reln:daalang:zero}.


\xbox{16}{
\ea\label{ex:form:reln:daalang:pe}
\gll Ruuma=\textbf{pe} \textbf{daalang}=ka gìllap \\
 house=poss inside=loc dark\\
    `Inside the house, it is dark.' (K081103eli02)
\z
}

\xbox{16}{
\ea\label{ex:form:reln:daalang:zero}
\gll Aanak pompang duuva=le,    derang=pe    umma=le   Buruan=yang  ruuma=\zero{} \textbf{daalang}=nang   su-panggel. \\
     child female two \textsc{3pl}=\textsc{poss} mother=\textsc{addit} bear=\textsc{acc} house  inside=\textsc{dat} \textsc{past}-call. \\
    `The two girls and their mother invited the bear to come inside the house.' (K070000wrt04)
\z
}

The referential use of \em daalang \em is given in \xref{ex:form:reln:daalang:ref}.



\xbox{16}{
 \ea\label{ex:form:reln:daalang:ref}
   \gll Kiccil vavvaal pada \zero{} \textbf{daalang=ka}  arà-duuduk. \\
     small  bat     \textsc{pl}   { } inside=\textsc{loc} prog.exist.\textsc{anim} \\
`There are small bats inside' (K051206nar02)
\z
}



\subsubsection{\trs{luvar}{outside}}\label{sec:wc:luvar}

This relator noun  refers to the outside of an entity. It does not occur in the corpus with the relator function, but elicitation shows that it can be used with or without \em =pe \em \xref{ex:form:reln:luvar:intro}.


\xbox{16}{
\ea\label{ex:form:reln:luvar:intro}
\gll Nigiri(=pe) luvar=ka bannyak uuthan. \\
 town=\textsc{poss} outside=\textsc{loc} much jungle\\
    `Outside of the town is a lot of jungle.' (K081103eli02)
\z
}

The referential use of \em luvar \em is given in \xref{ex:form:reln:luvar:ref}.

\xbox{16}{
\ea\label{ex:form:reln:luvar:ref}
\gll Ruuma=pe luvar panthas \\
 house=poss outside beautiful\\
    `The outside of the house is beautiful.' (K081103eli02)
\z
}

The adverbial use of \em luvar \em is given in \xref{ex:form:reln:luvar:adv1} and \xref{ex:form:reln:luvar:adv2}.
 
\xbox{16}{
\ea\label{ex:form:reln:luvar:adv1}
\gll Itthu    vakthu=ka    hathu  bìssar beecek caaya  Buruan mlaarath=ka     uuthang=dering  \textbf{luvar=nang}     su-dhaathang. \\
      \textsc{dist} time=\textsc{loc} \textsc{indef} big brown colour bear difficulty=\textsc{loc} forest=\textsc{abl} outside=\textsc{dat} \textsc{past}-come \\
    `Then, a big brown bear came out of the woods with difficulties.' (K070000wrt04)
\z
}


\xbox{16}{
\ea\label{ex:form:reln:luvar:adv2}
\gll Aavi \textbf{luvar=nang} kapang-dhaathang  itthu bambu=yang giini angkath=apa  pullang arà-thoolak. \\
      steam outside=\textsc{dat} when-come \textsc{dist} bamboo=\textsc{acc} like.this lift=after slow \textsc{non.past}-push \\
    `When the steam comes out, lift the bamboo like this and push it slowly.' (K061026rcp04)
\z
}

Example \xref{ex:form:reln:luvar:adv:compl} shows a rare instance of the adverbial use of a relator noun with a complement.
 

\xbox{14}{
\ea\label{ex:form:reln:luvar:adv:compl}
\gll Duuva thiiga Kluu\u mbu=deri,   Kluu\u mbu=dering  luvar  duuva=le \\
     ttwo three Colombo=\textsc{abl}, Colombo=\textsc{abl} outside two=\textsc{addit}  \\
    `Two or three (co-presidents) are from Colombo, two more are from outside Colombo.' (K051206nar13)
\z
}

\subsubsection{\trs{dìkkath}{vicinity}}\label{sec:wc:dikkath}
This relator noun  refers to the vicinity of an entity \citep{SmithEtAl2004}.
The use with \em =pe \em is given in \xref{ex:form:reln:dikkath:pe}, the use without \em =pe\em, in \xref{ex:form:reln:dikkath:zero}.

\xbox{16}{
\ea\label{ex:form:reln:dikkath:pe}
\gll {\em Kandy}    {\em town}=ka,    {\em Bank} of  Ceylon=\textbf{pe}     \textbf{dìkkath}=ka     aada  derang=pe   paasar. \\
     Kandy town=\textsc{loc} Bank of Ceylon=\textsc{poss} vicinity=\textsc{loc} exist \textsc{3pl}=\textsc{poss} shop. \\
    `Their shop is in Kandy town, close to the Bank of Ceylon.' (K051220nar01)
\z
}


\xbox{16}{
\ea\label{ex:form:reln:dikkath:zero}
\gll Hatthu avuliya aada kitham=pe ruuma \zero{} \textbf{dìkkath}. \\
      \textsc{indef} saint exist \textsc{1pl}=\textsc{poss} house { } vicinity \\
    `There is a saint close to our house.' (K060108nar02)
\z
}

The referential use of \em dìkkath \em is given in \xref{ex:form:reln:dikkath:hostless1} and \xref{ex:form:reln:dikkath:hostless2}.

\xbox{16}{
\ea\label{ex:form:reln:dikkath:hostless1}
\gll Kithang sama  oorang \textbf{dìkkath}$_{adv}$=ka     arà-duuduk. \\ % bf
      \textsc{1pl} all  man vicinity=\textsc{loc} \textsc{non.past}-exist.\textsc{anim} \\
    `All of us live close together.' (B060115cvs08)
\z
}

In \xref{ex:form:reln:dikkath:hostless2}, the emphatic clitic \em =jo\em, which intervenes between \trs{see}{1s} and \em dìkkath \em shows that \em see \em is not the ground for \em dìkkath \em here. Rather, the ground is the patient, and the figure is the speaker. The speaker is in a constellation to the ground indicated by the relator noun,  i.e. `vicinity' in the case of \em dìkkath\em.

\xbox{16}{
\ea\label{ex:form:reln:dikkath:hostless2}
\gll Mr. Yusuf thàràsìggar vakthu, see=jo \textbf{dìkkath}$_{adv}$=ka asà-duuduk samma incayang=nang {\em nursing} samma girja aada \\ % bf
      Mr Yusuf sick time \textsc{1s}=\textsc{emph} vicinity=\textsc{loc}  \textsc{cp}-stay all \textsc{3s.polite}=\textsc{dat} nursing all make exist\\
    `When Mr. Yusuf was sick, I stayed close to him and I cared for him.' (K060116nar07,K081104eli06)  
\z
}

\subsubsection{\trs{didaalam}{period}}\label{sec:wc:didaalam}

This relator noun  is quite rare and can only combine with temporal expressions, where it has a meaning of `during' or `within'. It is normally used without \em =pe \em as in \xref{ex:form:reln:didaalam:zero}-\xref{ex:form:reln:didaalam:zero3}, but one instance has been found where it is used with \em =pe\em, albeit with a temporal expression metonymically derived from a ruler, Queen Elizabeth \xref{ex:form:reln:didaalam:pe}.

\xbox{16}{
\ea\label{ex:form:reln:didaalam:zero}
\gll Ini vakthu \textbf{didaalam} se=dang Mrs Cunci nya-biilang kithang baae mlaayu baaru mlaayu hatthu marà-mulain. \\
      \textsc{prox} time during 1s.\textsc{dat} Mrs Chunchee \textsc{past}-say \textsc{1pl} good Malay new Malay \textsc{indef} \textsc{adhort}-start\\
    `Within this period, Mrs Chunchee had told me that we must start a new Malay one, a good Malay one (= association).' (K060116nar04,K081105eli02)
\z
}



\xbox{16}{
\ea\label{ex:form:reln:didaalam:zero2}
\gll Duuva  {\em week} \textbf{didaalang} cinggala  mulbar reepoth. \\
     two week within Sinhala Tamil problem. \\
    `Within two weeks, there was the Sinhala-Tamil problem.' (K051213nar01)
\z
}


\xbox{16}{
\ea\label{ex:form:reln:didaalam:zero3}
\gll Hathu  thaaun \textbf{didaalam}      thiiga skali   quran   nya-thamaam-king. \\
      one year within three time Qur'an \textsc{past}-complete-\textsc{caus} \\
    `I completed the Qur'an thrice within one year.' (K051213nar02)(K081103eli02)
\z
}


\xbox{16}{
\ea\label{ex:form:reln:didaalam:pe}
\ea
\gll Independence anà-daapath=nang=apa \\ % bf
Independence \textsc{past}-get=\textsc{dat}=after\\
`After having obtained independence'
\ex
\gll laile    derang anaduuduk        {\em under} {\em the} {\em Commonwealth} \\ % bf
still \textsc{3pl} \textsc{past}-exist.\textsc{anim} under the Commonwealth\\
`they still were under the Commonwealth'
\ex
\gll Puthri {\em Queen} {\em Elizabeth}=\textbf{pe}     \textbf{didaalang}=jo    anà-duudu. \\
      queen Queen Elizabeth=\textsc{poss} during=\textsc{emph} \textsc{past}-exist-\textsc{anim} \\
    `during Queen Elizabth's reign.' (K051222nar06)
\z
\z
}





% \xbox{16}{
% \ea\label{ex:form:unreferenced}
% \gll Kithang arà-biilang kithang=nang suurath 3$^{rd}$ thiiga di buulang=pe suurath doblas di buulang nya-daapath. \\
%      \textsc{1pl} \textsc{non.past}-say \textsc{1pl}=\textsc{dat} letter 3$^{rd}$ three of month=\textsc{poss} letter twelve of month \textsc{past}-get. \\
%     `We will say that we got the letter from the third of the month only on the twelfth.' (K060116nar10)
% \z
%. \\


\subsubsection{Spatial and temporal relator nouns}\label{sec:wc:Spatialandtemporalrelatornouns}


Some of the relator nouns can also be used with \em =nang \em on their complement, predominantly when the item serving as ground is verbal \citep{SmithEtAl2004}.

It appears that \em =nang \em is used with temporal contexts, whereas \em =pe \em is used in spatial contexts. The following examples illustrate this.


\xbox{16}{
\ea\label{ex:form:reln:spacetime:pe:duppang}
\gll Duuva ruuma=pe duppang. \\
      two house=\textsc{poss} front \\
    `In front of the two houses.' (K081103eli02)
\z
}


\xbox{16}{
\ea\label{ex:form:reln:spacetime:nang:duppang}
\gll Duuva aari=nang duppang. \\
      two day=\textsc{dat} front \\
    `Two days before/ago.' (K081103eli02)
\z
}

\xbox{16}{
\ea\label{ex:form:reln:spacetime:pe:blaakang}
\gll Duuva ruuma=pe blaakang. \\
      two house=\textsc{poss} behind \\
    `Behind the two houses.' (K081103eli02)
\z
}


\xbox{16}{
\ea\label{ex:form:reln:spacetime:nang:blaakang}
\gll Duuva aari=nang blaakang. \\
      two day=\textsc{dat} after \\
    `Two days later.' (K081103eli02)
\z
}

The temporal readings indicated by \trs{aari}{day} in the examples above take the dative marking, whereas the spatial readings forced by \trs{ruuma}{house} trigger the possessive \em =pe\em.

In rapid speech however, it is frequent that \em =nang \em as well as \em =pe \em are dropped.


%  itthu nii\u n\u ggalna=apa  \el{}  itthu sdiirina  sakithan aada

% independence anadaapathna apa

% \subsection{vatthu}\label{sec:wc:vatthu}
% is not a relator noun because it occurs without  `pe', but has grammaticalized into a temporal conjunction
%
% \xbox{16}{
% \ea\label{ex:form:reln:vatthu:complementclause}
% \gll Hatthu aari, see ruuma=ka sdiiri nya-duuduk \zero{} \textbf{vakthu}, Mrs. Cunci nya-dhaathang \\
%       one day \textsc{1s} house=\textsc{loc} alone \textsc{past}-saty { } time Mrs. Chunchee \textsc{past}-come \\
%     `One day when I was alone at home, Mrs. Chunchee came.' (K060116nar04)
% \z
% }
%
%
% \xbox{16}{
% \ea
% \gll Pe. \\
%. \\
%     `.' (nosource)
% \z
%. \\
%
%
%


%
%   \xbox{16}{
% \ea
% \gll Kithang sama oorang lorang=pe suurath=yang daapath vakthu=ka kithang sama oorang bannyak su-suuka. \\
%  \textsc{1pl} all man \textsc{2pl}=\textsc{poss} letter=\textsc{acc} get time=\textsc{loc} \textsc{1pl} all man much \textsc{past}-happy\\
%     `When we all received your letter, we were all very happy.' (Letter 26.2007)
%
% \z
% }



%
% \xbox{16}{
%  \ea
%    \gll Kamauvan \zero{} vakthu=nang,   kithang=nang   itthu    mosthor=nang,   {\em Malaysian} hathu  mosthor=nang   kithang=nang   bole=duuduk. \\
%      want {  } time=\textsc{dat} \textsc{1pl}=\textsc{dat} \textsc{dist} manner=\textsc{dat} Malaysian \textsc{indef} manner=\textsc{dat} \textsc{1pl} can-exist.\textsc{anim} \\
% `When we want we can be like this or in a Malaysian way' (K060108nar02)
% \z
% }



% \xbox{16}{
% \ea  \label{ex:form:reln:vatthu:referential}
% \gll \textbf{itthu}    \textbf{vakthu}=ka=jo       Mr  Samath=le   go=dang    nya-{\em introduce}-king      Sebastian=yang. \\
%      \textsc{dist} time=\textsc{loc}=\textsc{emph} Mr Samath=\textsc{addit} 1s.\textsc{dat} \textsc{past}-introduce-\textsc{caus} Sebastian=acc. \\
%     `It was at that moment that Mr Samath introduced Sebastian to me.' (B060115cvs01)
% \z
%. \\


\subsection{Proper nouns}\label{sec:wc:Propernouns}
Proper nouns are a special class of nouns in SLM. They are used to designate individual entities, but do not have semantic content.
Proper nouns share many properties with common nouns, but there are some differences as well. Normally, proper nouns can only refer to exactly one entity, hence the use of the indefiniteness marker, the deictics and the plural marker is normally not found. These markers would give information about the referential status of the referent, but this is not necessary, since proper nouns designate only one entity. In rare cases is it possible to find a combination of a deictic with a proper noun as in \xref{ex:form:propernouns:deic1} or \xref{ex:form:propernouns:deic2}.

\xbox{16}{
\ea\label{ex:form:propernouns:deic1}
\gll See \textbf{ini} \textbf{Sri} \textbf{Lanka}=ka nya-blaajar. \\
      \textsc{1s} \textsc{prox} Sri Lanka=\textsc{loc} \textsc{past}-learn \\
    `I studied here in Sri Lanka.' (K061026prs01)
\z
}


\xbox{16}{
\ea\label{ex:form:propernouns:deic2}
\gll Thuan Shaaban=jo    anà-{\em capture}-king     \textbf{inni}  \textbf{Saradiyel}=yang. \\
      Thuan Shaaban=\textsc{emph} \textsc{past}-capture-\textsc{caus}    \textsc{prox} Saradiyel=\textsc{acc} \\
    `It was Thuan Shaaban who captured this Saradiyel.' (K051205nar02)
\z
}

Proper nouns are normally not modified by adjectives either, but in some metonymical extensions as in \xref{ex:form:propernoun:adj} this is possible.


\xbox{16}{
\ea\label{ex:form:propernoun:adj}
\gll Go=ppe naama \textbf{Badulla} \textbf{buulath} thaau. \\
      1s.\textsc{familiar}=\textsc{poss} name Badulla whole know \\
    `Whole Badulla knows my name.' (B060115nar04)
\z
}


They can take all the postpositions, though. Since proper nouns are very often human, the special constructions to use humans as LOCATION or GOAL must be taken in to account as shown in \xref{ex:propernouns:allative}.


\xbox{16}{
\ea \label{ex:propernouns:allative}
\ea
\gll Tony=pe    dìkkath=nang mari. \\
      Tony=\textsc{poss} vicinity=\textsc{dat} come.\textsc{imp} \\
    `Come close to Tony.'
\ex
\gll *Tony=nang mari. { } \\
      Tony=\textsc{dat} come.\textsc{imp} (K081103eli02) \\
\z
\z
}

As for the frequency of proper names among the Malay population, first names are in the overwhelming majority of the cases of Islamic origin. The letters \graphem{F} and \graphem{Z} are very popular, and alliteration and assonance between siblings' names are very common (e.g. Fazlin, Falin, Fazeer, Faizel, Faiz; Zahira, Zarina, Farina; Imran, Izvan). Very often, the Sri Lanka Malays carry pet names which have nothing to do with their real name, e.g. Tony, Daisy or Joy. The use of this pet name is often so rampant that even very good friends do not know each other's official name. Another common pattern is the use of acronyms of the person's given names, a famous example being BDK Saldin \phonet{bi\postalvd i"ke:}.

There are few genuinely Malay place names (see Table \ref{tab:placenames}). In most cases, the Malays simply use the Sinhala name for Sinhalese place names like \em Singhapitiya\em, which is clearly parseable as `lion place' in Sinhala. There are few Tamil place names in the Upcountry, so the temptation to use Tamil place names is limited. A notable exception is Kandy itself, whose SLM name \em Kandi \em is taken over from Tamil \em Ka\nz\dz i \em (see Table \ref{tab:placenames}). For places abroad, the English name is generally used, even if a Tamil or Sinhala word does exist, so
\phontrs{\dzh@"p\ae:n}{Japan},
not \phonet{\dzh a"pa:niya} (Sinhalese),
 \phonet{\dzh appan} (Tamil)
 or \phonet{\dzh epang} (Std. Malay).
Occasionally, Sinhala placenames also carry the Sinhala locative case in  \em -e, \em   e.g. \em Badull-e \em instead of \em Badulla=ka\em.

\begin{table}[hbt]
\centering
% use packages: array
\begin{tabular}{llll}
English & Sinhala & Tamil & SLM \\
\hline
Kandy & Nuvara & Ka\nz\dz i & Kandi \\
Colombo & Kola\umb a & Ko\lz umbu & Kluu\u mbu \\
Negombo & Migamuva & Ko\lz umbu nir & Guu\u mbu \\
Galle & Galla & Kalla & Gaali \\
% Slave Island & Kompa\ny\ny a vidiya &. \\
Jaffna & Yapanaya & Ya\lz ppa\nz am & Jaapna \\
Batticaloa & Ma\dz akalapuva & Ma\tz\tz akka\lz appu & Batticaloa \\
Trinco(malee) & Tirikuu\nz aama\lz aya & Tirukoo\nz amalai &  Trinco\\
% Chilaw & &  &
\end{tabular}
 \caption{Sri Lankan toponyms in English, Sinhala, Tamil and SLM}
 \label{tab:placenames}
 \end{table} 



\section{Adjectives}\label{sec:wc:Adjectives}
The third major word class in SLM are Adjectives. In distinction to nouns and verbs, it is difficult to give simple tests for adjectivehood because adjectives permit a lot but require little. They accept the whole of verbal morphology and most nominal morphology, but do not require it. In fact it is this permissiveness that singles out adjectives.
In order to ascertain the status of an adjective, one has to test for the possibility to co-occur with a certain other item, and then for the possibility to occur \em without \em said item. If both are possible, then it is probably an adjective, otherwise a noun or a verb, as the following list shows.

\begin{itemize}
	\item They can be used as head of predicate without TAM morphology
	\item They can be used as head of predicate  with TAM morphology
	\item They can take the superlative marker \em anà-\em
	\item They can be used as heads of predicates  together with \em atthu\em
	\item They can be used as a pre- or postnominal modifier of a term without further modification
	\item They can modify predicates with  \em =nang \em
	\item They can modify predicates  without \em =nang \em
% 	\item Bare adjectives in predicate position have a stative reading while adjectives with TAM morphology in predicate position haven a dynamic reading
	\item They are negated by following \em thraa \em in all tenses in most idiolects. Occasionally also prefix \em thàrà- \em or following \em bukang\em.
\end{itemize}

These criteria will now be examined in turn. The adjective can be used as head of predicate without TAM-morphology, as given in \xref{ex:form:adj:bare}. It is also possible to use TAM-morphology with an adjective, but then the predicate gets a dynamic meaning \xref{ex:form:adj:TAM} instead of the static meaning in \xref{ex:form:adj:bare}.\footnote{Interestingly, the same is true in Mauritius Creole \citep[131]{Alleyne2000}.}



\xbox{16}{
\ea\label{ex:form:adj:bare}
\gll Aanak thii\u n\u ggi. \\
      child tall \\
    `The child is tall.' (K081103eli02)
\z
}


\xbox{16}{
\ea \label{ex:form:adj:TAM}
\gll Aanak arà-thii\u n\u ggi. \\
     child \textsc{non.past}-tall. \\
    `The child is growing tall.' (K081103eli02)
\z
}

Adjectives can combine with the superlative marker \em anà- \em \xref{ex:form:adj:anà-}. This marker is not very common. Furthermore, it is homophonous to the verbal past tense marker \em anà-\em, so that it is a less ideal test then what one could imagine.

\xbox{16}{
\ea \label{ex:form:adj:anà-}
\gll Seelon=ka \textbf{anà}-bìssar pohong. \\
      Ceylon=\textsc{loc} \textsc{superl}-big tree \\
    `The biggest tree in Sri Lanka.' (K081104eli06)
\z
}


Adjectives can be used with or without the indefiniteness marker \em hatthu\em.


\xbox{16}{
\ea
incayang   bìssar/iitham \\
      \textsc{3s.polite}   big/black \\
    `He is big/black.' (K081103eli02)
\z
}

\xbox{16}{
\ea\label{ex:form:adj:hatthu}
\gll Incayang hatthu bìssar. \\
      \textsc{3s.polite} \textsc{indef} big \\
    `He is a big one/head.' (K081103eli02)
\z
}


\xbox{14}{
\ea
\gll Incayang hatthu iitham  \\
         \textsc{3s.polite} \textsc{indef} black  \\
    `He is a dark person.' (K081103eli02)
\z
}

% \xbox{16}{
% \ea
% \gll Incayang hatthu iitham/laaeng/*baae. \\
% . \\
%     `.' (K081103eli02)
% \z
% } \\

Adjectival lexemes can head referential phrases.

\xbox{16}{
\ea\label{ex:form:adj:referential}
\gll Iitham/bìssar su-dhaathang. \\
     black/big \textsc{past}-come. \\
    `The black one/big one/boss came.' (K081103eli02)
\z
}

Adjectives can pre- or postmodify nouns, but postmodification is dispreferred for ad hoc formations \xref{ex:form:adj:prepost}. It is possible in lexicalized N+ADJ combination like \trs{baavam puuthi}{onion'+`white'=`garlic}.

\xbox{16}{
\ea\label{ex:form:adj:prepost}
\ea
\gll Bìssar ruuma hatthu aada. \\
     big house \textsc{indef} exist. \\
    `There is a big house.' (K081103eli02)
\ex
\gll ??ruuma bìssar hatthu aada. \\
     big house \textsc{indef} exist. \\
    `There is a big house.' (K081103eli02)

\z
\z
}


Adjectives can modify verbal predications with or without the dative marker \em =nang\em.

\xbox{16}{
\ea\label{ex:form:adj:modpred}
\gll Baae(=nang) arà-nyaanyi. \\
     good=\textsc{dat} \textsc{non.past}-sing. \\
    `He sings well.' (K081103eli02)
\z
}

% Adjectives can combine with verbal morphology and get a [+dynamic] reading then instead of their normal static reading.
% 
% \xbox{16}{
% \ea \label{ex:soa:events:adj}
% \gll Itthu=nam blaakang=jo, kitham pada \textbf{anà-bìssar}. \\
%  \textsc{dist} after=\textsc{emph} \textsc{1pl} \textsc{pl} \textsc{past}-big\\
% `After that, we grew up (=we became big).' (K060108nar02)
% \z
% }



Adjectives are negated with \em thraa \em \xref{ex:form:adj:neg:thraa}. They can also use verbal negation when used in a verbal frame \xref{ex:form:adj:neg:thara}, and nominal negation with \em bukang \em when used in a nominal frame \citep[141]{Slomanson2006cll}.

\xbox{16}{
\ea\label{ex:form:adj:neg:thraa}
\gll Se=ppe aanak \textbf{iitham} \textbf{thraa}. \\
      \textsc{1s}=\textsc{poss} child black \\
    `My child is not black.' (K081103eli02)
\z
}


\xbox{16}{
\ea\label{ex:form:adj:neg:thara}
\gll Aanak thàrà-thii\u n\u ggi. \\
     child \textsc{neg.nonpast}-big. \\
    `The child did not grow tall.' (K081103eli02)
\z
}


% \xbox{16}{
% \ea\label{ex:form:adj:bukang}
% \gll *aanak thii\u n\u ggi bukang. \\
%      child big \textsc{neg.nonv}. \\
%     `The child is not the big one.' (K081103eli02)
% \z
% } \\

A small subclass of adjectives are always negated by \em thàrà-\em, without implying a verbal reading or reference to the past \xref{ex:wc::adj:neg:thara:pres}. Negation with \em thraa \em is not possible for the adjectives in this class.

\xbox{16}{
\ea\label{ex:wc::adj:neg:thara:pres}
\gll Se=dang \textbf{thàrà-sìggar}. \\
     \textsc{1s=dat} \textsc{neg}[adj]-healthy  \\
    `I am/was not healthy.'  (K081104eli06)
\z
}





Adjectives can take both nominal and verbal morphology. A variety of tests are thus necessary to determine whether a lexeme is an adjective, and to rule out the possibility that it could be a noun or a verb.

% A case in point is \trs{picca}{break}, which is very appealing as a verb because of its semantics. A more thorough analysis reveals that it is an adjective though, so that `broken' seems to be  a more appropriate gloss. When used in a verbal frame, it acquires an processual reading. Compare the following two sentences, where \em picca \em is used to denote a state and does not carry TAM-morphology, with the following two sentences, where the use of TAM-morphology conveys the processual reading.
 
% 
% The paradigmatic examples above with \trs{bìssar}{big} are now repeated with sentences from actual discourse. 
% 
% \xbox{16}{
% \ea\label{ex:form:adj:disc:bare1}
% \gll Nni      saarong samma \zero{} \textbf{baasa}. \\
%      \textsc{prox} sarong all {  } wet. \\
%     `These sarongs are all wet.' (K060116nar03)
% \z
% } \\
% 
% 
% 
% \xbox{16}{
% \ea\label{ex:form:adj:disc:bare2}
% \gll \textbf{umma} \textbf{buthul} \zero{}  \textbf{miskin} subbath aanak su-laari kluuling. \\
%      mother correct { } poor because child \textsc{past}-run roam. \\
%     `Because the mother was very poor, the child ran away.' (K061019sng01)
% \z
% } \\
% 
% 
% \xbox{16}{
% \ea\label{ex:form:adj:disc:TAM}
% \gll Aanak pada \textbf{asà}-bìssar, skuul=nang anà-pii. \\
%       child \textsc{pl} \textsc{cp}-big school=\textsc{dat} \textsc{past}-go \\
%     `After having grown up, the children went to school.'  (K051222nar04)
% \z
% }
% 
% 
% \xbox{16}{
% \ea
% \gll Hatthu. \\
% . \\
%     `.' (test)
% \z
% } \\
% 
% 
% \xbox{16}{
% \ea\label{ex:form:adj:disc:referential}
% \gll 3  {\em miles} cara  \textbf{jaau}$_{referential}$=ka    aada  dee anasbuuni   duuduk     {\em cave}=yang. \\
%      3 miles way far=\textsc{loc} exist 3 \textsc{past}-hide stay cave=\textsc{acc} \\
%     `Three miles from here is the cave where he hid.' (K051206nar02)
% \z
% } \\
% 
% prepost
% 
% 
% \xbox{16}{
% \ea\label{ex:form:adj:disc:modpredzero1}
% \gll Aavi luvar=nang kapang-dhaathang  itthu bambu=yang giini angkath=apa  \textbf{pullang}=\zero{}  arà-thoolak. \\
%       steam outside=\textsc{dat} when-come \textsc{dist} bamboo=\textsc{acc} like.this lift=after slow \textsc{non.past}-push \\
%     `When the steam comes out, lift the bamboo like this and push it slowly.' (K061026rcp04)(K081103eli02)
% \z
% } \\
% 
% 
% \xbox{16}{
% \ea\label{ex:form:adj:disc:modpredzero2}
% \gll {\em bus}=ka kapang-pii \textbf{cumma}=\zero{} anà-pii. \\
%      bus=\textsc{loc} when-go idle \textsc{non.past}-go. \\
%     `When I went with the bus, I did not do anything.' (K061125nar01)
% \z
% } \\
% 
% \xbox{16}{
% \ea\label{ex:form:adj:disc:modprednang}
% \gll Derang pada \textbf{baaye=nang}   mlaayu arà-oomong. \\
%      \textsc{3pl} \textsc{pl} good=\textsc{dat} Malay \textsc{non.past}-speak. \\
%     `They speak Malay well.' (K051206nar08)
% \z
% } \\
% 
% 
% \xbox{16}{
% \ea\label{ex:form:adj:disc:thraa}
% \gll Itthu muusing gampang \textbf{thraa}. \\
%      \textsc{dist} time easy neg. \\
%     `It was not easy back then.'  (B060115nar05)
% \z
% }
% 
% 
% \xbox{16}{
% \ea\label{ex:form:adj:disc:thara}
% \gll Go=dang   karang bannyak thàrà-sìggar. \\
%      1s.familiar.\textsc{dat} now very \textsc{neg}-healthy. \\
%     `I am now very sick.' (B060115nar04)
% \z
% } \\
% 
% 
% \xbox{16}{
% \ea
% \gll Se=dang thàrà-sìggar thraa. \\
% 1s.\textsc{dat} \textsc{neg}-healthy \textsc{neg} \\
%     `I am not sick.' (K081103eli02)
% \z
% } \\
% 
% 
% 
% \xbox{16}{
% \ea
% \gll *ADJ bukang. \\
% . \\
%     `.' (K081103eli02)
% \z
% } \\


%
%  \xbox{16}{
%  \ea\label{ex:form:adj}
%    \gll Siini duuduk kalu kaapang=ke  lorang=nang   lorang thàràboole   kaaya. \\
%    here stay if when=\textsc{simil} \textsc{2pl}=\textsc{dat} \textsc{2pl} cannot rich \\
% `If you stay here you can never become rich' (K051206nar07)
% \z
% }
% \xbox{16}{
% \ea\label{ex:form:adj}
% \gll Thapi see nya-bìnnar    criitha=yang   su-biilang. \\
%      But \textsc{1s} \textsc{past}-true story=\textsc{acc} \textsc{past}-say. \\
%     `But I told the truth.' (K051213nar01)
% \z
%. \\


%
% :sathu aari sathu aari kithang
% K061019sng01.trs:maayeng duuduk vakthu
% K061019sng01.trs:dhaathang thurus police oorang
% K061019sng01.trs:samma oorang laari vakthu
% K061019sng01.trs:inni aanak daapath kìnna baapi aada police station nang


\section{Adverbs}\label{sec:wc:Adverbs}

\begin{table}
\begin{tabular}{lll}
\trs{(pàr)thaama, (kà)thaama}{earlier}			& \trs{siini}{here}    		& \trs{thapi}{but}   \\
\trs{kumareng}{yesterday}				& \trs{sana(ka)}{there yonder}  &   \trs{incalla}{hopefully} \\
\trs{nyaari}{today}					& \trs{siithu}{there}    &   \trs{suda}{thus} \\
\trs{(s)karang}{now}					& \trs{giini}{this way}         & \trs{sajja}{only} \\
\trs{beeso}{tomorrow}					& \trs{giithu}{that way}        &  \trs{sinderi}{from here}  \\
\trs{luuso}{later than tomorrow}			& \trs{sgiini}{this much}       & \trs{sanderi}{from there yonder}  \\
\trs{subbang}{often}					& \trs{sgiithu}{that much}    & \trs{sithari}{from there}  \\
\end{tabular}
\caption{SLM Adverbs}
\label{tab:form:adverbs}
\end{table}

Adverbs are words that carry lexical meaning but are neither nouns nor verbs nor adjectives and do not quantify.  This class is quite small in SLM.
Monomorphemic adverbs are given in Table \ref{tab:form:adverbs}.
Adverbs can combine with postpositions as shown in the examples below.

\xbox{16}{
\ea\label{ex:form:adv:pe1}
\gll \textbf{Karam=pe} mosthor=nang, mpapulu aari=ka=jo sunnath=le arà-kijja. \\
		now=\textsc{poss} manner=\textsc{dat} forty   day=\textsc{loc}=\textsc{emph} circumcision=\textsc{addit} \textsc{non.past}-make. \\
    `For today's way of doing (it), it is on the fortieth day that they also do the circumcision.'  (K061122nar01)
\z
}


\xbox{16}{
\ea\label{ex:form:adv:pe2}
\ea
\gll \textbf{Dovulu=pe} oorang pada  itthu anà-kirja hathu  thundu bambu asà-ambel=apa. \\
     before=\textsc{poss} man \textsc{pl} \textsc{dist} \textsc{past}-make \textsc{indef} piece bamboo \textsc{cp}-take=after \\
    `The people from former times to make this, they took a piece of  bamboo and'
\ex
\gll Itthu=yang baaye=nang arà-{\em wrap}-kang  athu kaayeng thundu=dering. \\ % bf
       \textsc{dist}=\textsc{acc} good=\textsc{dat} \textsc{non.past}-wrap-\textsc{caus} \textsc{indef} cloth piece=\textsc{abl} \\
    `wrapped it well in a piece of cloth.' (K061026rcp04)
\z
\z
}

\xbox{16}{
\ea \label{ex:form:adv:pe3}
\gll [Incayang cinggala asà-blaajar]=apa \textbf{sini=pe} raaja=nang mà-banthu anà-mulain. \\
     3s.\textsc{polite} Sinhala \textsc{cp}-learn=after here=\textsc{poss} king=\textsc{dat} \textsc{inf}-help \textsc{past}-start\\
    `He learned Sinhala and began to help the local king.'  (K060108nar02)
\z
}

SLM adverbs can be divided into temporal adverbs, adverbs with a spatial deictic component, and a `remainder' class of other adverbs.

% Adverbs are used in SLM as predicate satellites\footnote{This taxonomy is based on \citet{Hengeveld1997adverbs}.} \funcref{} and predication satellites \funcref{}. Interestingly, no lexical adverbs can be used as satellites of proposition, speech act and utterance (analytic expressionslike \trs{baae vatthu=na}{good time=dat=fortunately} are used instead).

\subsection{Temporal adverbs}\label{sec:wc:Temporaladverbs}
Temporal adverbs normally do not take postpositions, with the exception of \trs{beeso}{tomorrow} and \trs{luusa}{some day after tomorrow}, which often combine with the dative. The following examples illustrate this.

\xbox{16}{
\ea\label{ex:form:adv:temp:perthaama}
\gll Muula     \textbf{pàrthaama} Badulla  ruuma saakith=ka    s-riibu   sbiilan raathus lima-pulu    dhlaapan=ka pukurjan arà-gijja  vakthu. \\
     before first Badulla house sick=\textsc{loc} one-thousand nine hundred five-ty eight=\textsc{loc} work \textsc{non.past}-make time \\
    `Before, when I was working in Badulla in 1958.' (K051213nar01)
\z
}


\xbox{16}{
\ea\label{ex:form:adv:temp:kàthaama}
\gll Inni     {\em railway} {\em department}=ka    {\em head} {\em guard} hatthu \textbf{kàthaama}. \\
      \textsc{prox} railway department=\textsc{loc} head guard \textsc{indef} before \\
    `He was a head guard in the railway department before.'
\z
}


\xbox{16}{
\ea\label{ex:form:adv:temp:kàthaama2}
\gll Punnu mlaayu pada \textbf{kàthaama} {\em English}=jona anthi-oomong. \\
       many Malay \textsc{pl} earlier English=\textsc{phat} \textsc{irr}=speak \\
    `Many Malays would speak English in former times, wouldn't they?'  (K051222nar06)
\z
}

\xbox{16}{
\ea\label{ex:form:adv:temp:kaarang}
\ea
\gll Non-Muslims pada=subbath kithang muuka konnyong arà-cunji-kang siini. \\ % bf
     non-Muslims \textsc{pl}=because \textsc{1pl} face little  \textsc{non.past}-show-\textsc{caus} here \\
    `Because of the non-Muslims, we show our faces here.'
\ex
\gll Thapi \textbf{karang} \textbf{karang}  \textbf{skarang} \textbf{skarang} Sri Lanka=pe=le Islam pada muuka arà-thuuthup karang  dovulu abbis-dhaathang muuka thama-thuuthup. \\
     but now now now now Sri Lanka=\textsc{poss}=\textsc{addit} Islam \textsc{pl} face \textsc{non.past}-close now before \textsc{copula} face \textsc{neg.nonpast}-close \\
    `But now, Muslims in Sri Lanka cover their faces, now; before it was that they would not cover their faces.' (K061026prs01)
\z
\z
}

\xbox{16}{
\ea\label{ex:form:adv:tmp:beesoluusa}
\ea
\gll \textbf{Beeso} \textbf{luusa} lubaarang arà-dhaathang. \\
     tomorrow some.day.after.tomorrow festival \textsc{non.past}-come. \\
    `The day after tomorrow is the festival.'
\ex
\gll Itthu lanthran kithang=pe ruuma see arà-cuuci. \\ % bf
       \textsc{dist} reason \textsc{1pl}=\textsc{poss} house \textsc{1s} \textsc{non.past}-clean\\
    `That's why I am cleaning the house.' (K061019prs01)
\z
\z
}

 


\subsection{Adverbs with a deictic component}\label{sec:wc:Spatialadverbs}
A number of adverbs have clear diachronic relations to the deictics \trs{ini}{proximal} and \trs{itthu}{distal} \formref{sec:wc:Deictics}. Sometimes, a third adverb with \em san- \em exists in the set, whose deictic value is unclear. As for local adverbs, \em siini, siithu \em and \em sana \em exemplify the pattern.


\xbox{16}{
\ea\label{ex:form:spatial:siini}
\gll [Non-Muslims pada=subbath] kithang muuka konnyong arà-cunji-kang \textbf{siini}. \\
     non-Muslims \textsc{pl}=because \textsc{1pl} face little  \textsc{non.past}-show-\textsc{caus} here \\
    `Because of the non-Muslims, we show our faces here.' (K061026prs01)
\z
}

\xbox{16}{
\ea\label{ex:form:spatial:siithu}
\gll \textbf{Siithu} umma-baapa  arà-duuduk=si. \\
      there mother-father \textsc{non.past}-exist.\textsc{anim}=\textsc{interr} \\
    `Do your parents live over there?' (B060115cvs03)
\z
}


\xbox{16}{
\ea\label{ex:form:spatial:sanaka}
\gll Itthu thoppi=yang \textbf{sana}=ka simpang. \\
     \textsc{dist} hat=\textsc{acc} there=\textsc{loc} keep. \\
    `Keep that hat over there.' (K081103eli03)
\z
}

\em Sana- \em seems to be mutually exclusive with \em siini\em; the precise difference in meaning between \em siithu \em and \em sana(ka) \em is unclear. \em Siithu \em is the normal adverb used and occurs frequently in the corpus, \em sana(ka) \em does not occur at all. Note that \em siini \em and \em siithu \em can occur on their own, while \em sana \em must take the locative postposition \em =ka\em. \em Siini \em and \em siithu \em can also combine with \em =ka \em \xref{ex:form:spatial:siinika}\xref{ex:form:spatial:siithuka}, and with the ablative postposition \em =dering \em \xref{ex:form:spatial:siinidering}\xref{ex:form:spatial:siithudering}. Combinations with the allative postposition \em =nang \em are not possible.


\xbox{16}{
\ea\label{ex:form:spatial:siinika}
\gll {\em Malay} {\em regiment} hatthu dhaathang=apa \textbf{sini=ka}           {\em settle}=apa. \\
     Malay regiment \textsc{indef} come=after here=\textsc{loc} settle=after. \\
    `After the Malay regiment had come  and settled here.' (G051222nar03)
\z
}

\xbox{16}{
\ea\label{ex:form:spatial:siithuka}
\gll Derang anà-baalek    sajja=jo,    \textbf{sithu=ka}     panthas   hathu  Aanak  raaja su-aada! \\
       \textsc{3pl} \textsc{past}-return only=\textsc{emph} there=\textsc{loc} beautiful \textsc{indef} child prince \textsc{past}-exist\\
    `And when they returned, there was  beautiful prince!' (K070000wrt04)
\z
}


\xbox{16}{
\ea\label{ex:form:spatial:siinidering}
\gll Incayang \textbf{siini=dering} su-pii. \\
     \textsc{3s.polite} here=\textsc{abl} \textsc{past}-go. \\
    `He went off from here.' (K081103eli03)
\z
}

\xbox{16}{
\ea\label{ex:form:spatial:siithudering}
\gll Incayang \textbf{siithu=dering} su-dhaathang. \\
     \textsc{3s.polite} there=\textsc{abl} \textsc{past}-come. \\
    `He came back from there.' (K081103eli03)
\z
}

The combination of the adverbs with  \em =dering \em is also the origin of a related set of adverbs: \trs{sindari}{nearside}, \trs{sithari}{farside} and \em sandari\em, whose meaning is unclear as of now. These adverbs require the dative postposition \em =nang \em when the ground to which they relate is expressed (a grandfather in \xref{ex:wc:adv:sithari} and \xref{ex:wc:adv:sindari}). In this respect, they resemble temporal relator nouns \formref{sec:wc:Relatornouns}, what distinguishes them is the absence of the lexical counterpart of relator nouns.


\xbox{14}{
\ea\label{ex:wc:adv:sithari}
\gll Incayang=nang    \textbf{sithari}           se=dang mà-biilang    thàràthaau. \\
        \textsc{3s.polite}=\textsc{dat} that.side 1s.\textsc{dat} \textsc{inf}-say cannot. \\
    `From this person onwards, I am unable to explain.' (K060108nar02)
\z
}


\xbox{16}{
\ea\label{ex:wc:adv:sindari}
\gll Incayangnang \textbf{sindari}, se=dang mà-biilang bannyak thaau. \\
     \textsc{3s.polite}=\textsc{dat} this.side 1s.\textsc{dat} \textsc{inf}-say much can. \\
    `Up to this person, I call tell you a lot.' (K081104eli06)
\z
}

Related adverbs are also found in the domains of manner (\em giini \em and \em giithu\em) and amount (\em sgiini \em and \em sgiithu\em), but there, only two degrees of distance are distinguished. The equivalent of \em sana(ka) \em is missing.\footnote{This restricted use of \em sana \em seems to be the normal case in Malay varieties \citep[cf.][276]{Paauw2008phd}.}


% \xbox{16}{
% \ea\label{ex:form:adv:other:giini}
% \gll {\em Colombo}=ka    \textbf{giini}    aari=le, luvar   nigiri=ka     anà-duuduk,        Mekka=ka asàduuduk     karang dhaathang aada. \\
%       Colombo=\textsc{loc} like.this day=\textsc{addit} outside country=\textsc{loc} \textsc{past}-exist.\textsc{anim} Mekka=\textsc{loc} from now come exist \\
%     `My daughther stayed some days in Colombo like this, abroad, now she has come from Mecca.' (K051201nar01)
% \z
% } \\


\xbox{16}{
\ea\label{ex:form:adv:other:giithu}
\gll Ithukang       ithu     bambu  \textbf{giithu}=jo      luvar=nang arà-dhaathang. \\
      then \textsc{dist} bamboo like.that=\textsc{emph}  outside=\textsc{dat} \textsc{non.past}-come\\
    `Then, when the bamboo come out like that.' (K061026rcp04)
\z
}


\xbox{16}{
\ea\label{ex:form:adv:other:sgiini}
\gll [\textbf{Sgiini} lakuvan de  sindari arà-baa katha] asà-thaau blaakang,  soojer pada  incayang=sàsaama Seelon=nang asà-dhaathang ... \\
 this.much  wealth  \textsc{3s} from.here \textsc{non.past}-take \textsc{quot} \textsc{cp}-know after European \textsc{pl} \textsc{3s.polite}=\textsc{comit} Ceylon=\textsc{dat} \textsc{cp}-come\\
`After having learnt that he brings this much wealth from here, the Europeans came to Ceylon together with him (and ...)' (K060103nar01)
\z
}


\xbox{14}{
\ea\label{ex:form:adv:other:sgiithu}
\gll Derang pada    \textbf{sgiithu}       braani \\
     \textsc{3pl} \textsc{pl} that.much strong  \\
    `They were so strong.' (K060108nar02)
\z
}

The proximal and the distal deictic are ubiquitous, while \em sanaka \em is quite rare.
% The 2-way deictic system with the addition of \em sanaka \em is of clear Malayic descent in form and function.


\subsection{Other adverbs}\label{sec:wc:Otheradverbs}
Three other adverbs were found, which do not pattern with the ones discussed above: \trs{suda}{thus} \xref{ex:form:adv:other:suda} \citep[167]{SmithEtAl2006cll}, \trs{sajja}{only} \xref{ex:form:adv:other:sajja} and \trs{incalla}{hopefully} \xref{ex:form:adv:other:incalla}.


\xbox{16}{
\ea\label{ex:form:adv:other:suda}
\gll \textbf{Suda} skaarang    kitham=pe  aanak pada    laayeng pukurjan pada    arà-girja. \\
     thus now \textsc{1pl}=\textsc{poss} child \textsc{pl} other work \textsc{pl} \textsc{non.past}-make. \\
    `So now our children work in other professions.' (K051222nar05)
\z
}

\xbox{16}{
\ea\label{ex:form:adv:other:sajja}
\gll Aathi=yang \textbf{sajja} hatthu oorang=nang bole=ambel hathyang bagiyan bole=baagi sama oorang=nang. \\
     liver=\textsc{acc} only one man=\textsc{dat} can=take other part can=divide all man=\textsc{dat}. \\
    `The liver can only be taken by one person; the other parts can be divided among all men.' (K060112nar01)
\z
}



\xbox{16}{
\ea\label{ex:form:adv:other:incalla}
\gll \textbf{Incalla}   [lai     thaau sudaara sudaari pada]=ka    bole=caanya    ambel [nya-gijja    lai     saapa=kee  aada=si    katha]. \\
      Hopefully other know brother sister \textsc{pl}=\textsc{loc} can-ask take \textsc{past}-make other who=\textsc{simil} exist=\textsc{interr} \textsc{quot} \\
    `Hopefully, you can enquire from another person you know whether there is someone else who did something.' (N061031nar01)
\z
}


\section{Copula}\label{sec:wc:Copula}
SLM has a copula which is derived from the verb \trs{dhaathang}{come} by either the conjunctive participle  prefix \em asà- \em, or the postposition \em =apa\em, or both.
Possible forms then include \em as(à)dhaathang, dhaathang(a)pa, dhaathampa, as(à)dhaathampa \em and \em as(a)dhaathang(a)pa\em. The choice of form depends on the idiolect. The three main patterns with \em asà-, =apa \em and both are illustrated below in sentences introducing names, but there are more uses to be discussed below.

\xbox{16}{
\ea\label{ex:form:copula:asa}
\ea
\gll Ini      {\em head} {\em master} \textbf{asà}dhaathang  hathu  Jayathilaka. \\
      \textsc{prox} head master \textsc{copula} \textsc{indef} Jayathilaka \\
    `This head master was a certain Jayathilaka.'
\ex
\gll Se=ppe    {\em class} {\em master} \textbf{asà}dhaathang  hathu  Mr  Senevirathna. \\
      \textsc{1s}=\textsc{poss} class master \textsc{copula} \textsc{indef} Mr Senevirathna \\
    `My class master was a certain Mr Senevirathna.' (K051201nar02)
\z
\z
}


\xbox{16}{
\ea\label{ex:form:copula:apa}
\gll Karang se=ppe    {\em father-in-law} dhaathang\textbf{apa} Mr  Asali. \\
      now \textsc{1s}=\textsc{poss} father-in-law \textsc{copula} Mr Asali\\
    `Now my father-in-law is Mr Asali.' (G051222nar01)
\z
}

\xbox{16}{
\ea\label{ex:form:copula:asaapa}
\gll {\em Owner} \textbf{as}dhaathang\textbf{apa} Sir Handy Kothalawela. \\
      owner \textsc{copula} Sir Handy Kothalawela \\
    `The owner was Sir Handy Kothalawela.' (K051213nar04)
\z
}

In the above examples, the copula is combined with foreign names (Sinhala and English), but it can also be found with native words, as in \xref{ex:form:copula:native}.


\xbox{16}{
\ea\label{ex:form:copula:native}
\gll Se=ppe    baapa  dhaathangapa \textbf{Jinaan} \textbf{Samath}. \\
     \textsc{1s}=\textsc{poss} father \textsc{copula} Jinaan Samath. \\
    `My father was Jinaan Samath.' (N060113nar03)
\z
}

As the preceding examples show, the copula is underspecified for tense and allows for both a present and a past interpretation.

The use of  \em (asà)dhaathang(apa) \em as a copula has to be distinguished from the use as a conjunctive participle in the perfect construction as in  \xref{ex:form:copula:cp:perf} or in clause chains as in \xref{ex:form:copula:cp:chain}.


\xbox{16}{
\ea\label{ex:form:copula:cp:perf}
\gll See skarang Sri Lanka=nang   \textbf{asà-dhaathang} \textbf{aada}. \\
     \textsc{1s} now Sri Lanka=\textsc{dat} \textsc{cp}-come exist. \\
    `I have now come to Sri Lanka.' (K051206nar17)
\z
}

\xbox{16}{
\ea\label{ex:form:copula:cp:chain}
\gll [Itthu nigiri=deri     \textbf{as-dhaathang}    anà-thii\u n\u ggal  oorang pada]=jo       kithang. \\
     \textsc{dist} country=\textsc{abl} \textsc{cp}-come \textsc{past}-stay man \textsc{pl}=\textsc{emph} 1\textsc{pl}. \\
    `The people who have come from those countries and stayed (here) are we.' (K051222nar03)
\z
}

In example \xref{ex:form:copula:cp:chain}, we are dealing with the literal meaning of \trs{dhaathang}{come}{} as used in a clause chain where it is a non-final element and thus marked with the conjunctive participle prefix \em asà-\em. An interpretation as a copula is not possible here.

When used for naming purposes, the copula can combine with the person herself, as in the examples given above, or with the word \trs{naama}{name}, as in the next example.


\xbox{16}{
\ea\label{ex:form:copula:naming:naama1}
\gll Se=ppe    naama asàdhaathang  Cintha Sinthani. \\
     \textsc{1s}=\textsc{poss} name \textsc{copula} Chintha Sinthani. \\
    `My name is Chintha Sinthani.' (B060115prs04)
\z
}
% \xbox{6}{
% \ea\label{ex:form:copula:naming:naama2}
% \gll Se=ppe aabang=pe naama Nazir Johoran. \\
%       \textsc{1s}=\textsc{poss} husband=\textsc{poss} name Nazir Johoran \\
%     `My husband's name is Nazir Johoran.' (B060115prs05)
% \z
% }
% \xbox{6}{
% \ea\label{ex:form:unreferenced}
% \gll Incayang=pe      naama  Thuan Shaaban. \\
%       \textsc{3s.polite}=\textsc{poss} name Thuan Shaaban \\
%     `His name was Thuan Shaaban.' (K051205nar02)
% \z
%. \\


Besides for introducing names, the copula is mainly used for giving information on kinship status and profession. These two uses are illustrated below for all three forms, yielding six possibilities. Both the use of \em dhaathangapa \em with kin and the use of \em asdhaathangapa \em with professions is not attested. This is thought to be an accidental gap in the corpus.

\xbox{16}{
\ea\label{ex:form:copula:asa:kin}
\gll Baapa=pe      umma   \textbf{asà}dhaathang  kaake=pe           \textbf{aade}$_{kin}$. \\
    father=\textsc{poss} mother \textsc{copula} grandfather=\textsc{poss} younger.sibling. \\
    `My paternal grandmother was my grandfather's younger sister.' (K051205nar05)
\z
}



\xbox{16}{
\ea\label{ex:form:copula:asaapa:kin}
\gll {\em Estate}=pe {\em field} {\em officer} \textbf{asà}dhaathang\textbf{apa}  kithang=pe     \textbf{kaake}$_{kin}$\\
     estate=\textsc{poss} field officer \textsc{copula} \textsc{1pl}=\textsc{poss} grandfather. \\
    `The estate field officer was our grandfather.' (N060113nar03)
\z
}


\xbox{16}{
\ea
\gll [Seelong=nang  duppang duppang anà-dhaathang  mlaayu] \textbf{asà}dhaathang \textbf{oorang} \textbf{ikkang}$_{profession}$. \\
      Ceylon=\textsc{dat} before before \textsc{past}-come Malay] \textsc{copula} man fish \\
    `The Malays who came to Ceylon very early were fishermen.' (K060108nar02)
\z
}



\xbox{16}{
\ea
\gll Ummape       baapa  dhaathang\textbf{apa}  hathu  \textbf{{\em inspector}}          \textbf{{\em of}} \textbf{{\em police}}$_{profession}$. \\
     mother=\textsc{poss} father \textsc{copula} \textsc{indef} inspector of police. \\
    `My mother's father was an inspector of police.' (N060113nar03)
\z
}

Next to these central uses, which are very frequent, the copula can also be used for the following: age \xref{ex:form:copula:age1}\xref{ex:form:copula:age2}, nationality \xref{ex:form:copula:nationality}, sex \xref{ex:form:copula:sex}.


\xbox{16}{
\ea\label{ex:form:copula:age1}
\gll Suda se=ppe    thuuva anak  klaaki \textbf{asàdhaathang} \textbf{dhlapan-blas}    \textbf{thaaun}$_{age}$. \\
      so \textsc{1s}=\textsc{poss} old child male \textsc{copula} eight-teen year\\
    `So my eldest son is eighteen.' (K060108nar02)
\z
}

Just as with names, the predicate (a number in this case) can be attributed to either the person or the quality, in this case the age.

\xbox{16}{
\ea\label{ex:form:copula:age2}
\gll Go=ppe     {\em age} \textbf{asàdhaathang}   \textbf{\em 78}$_{age}$. \\
       \textsc{1s}=\textsc{poss} age \textsc{copula} 78\\
    `My age is 78\footnotemark.' (B060115nar04)
\z
}. \\

\footnotetext{This speaker has little command of English, so that a calque on the English sentence is unlikely.}

\xbox{16}{
\ea\label{ex:form:copula:nationality}
\ea
\gll Se=ppe    {\em daughter-in-law}=pe {\em mother} \textbf{asàdhaathang} \textbf{bìnggaali}$_{ethnic group}$. \\
      \textsc{1s}=\textsc{poss} daughter-in-law=\textsc{poss} mother \textsc{copula} Bengali \\
    `My daughter-in-law's mother is Bengali.'
\ex
\gll Ithukapang       {\em daughter-in-law}=pe     father \textbf{asàdhaathang} \textbf{mlaayu}$_{ethnic group}$. \\
      then daughter-in-law=\textsc{poss} father \textsc{copula} Malay \\
    `Then my daughter-in-law's father is Malay.' (K051206nar08)
\z
\z
}

\xbox{16}{
\ea\label{ex:form:copula:sex}
\gll Kàthama aanak dhaathangapa \textbf{klaaki}$_{sex}$. \\
      first child \textsc{copula} male \\
    `My oldest child is a boy.' (G051222nar01)
\z
}

These are the central uses of the copula. In the following, I will give some more peripheral uses of the copula. Example \xref{ex:form:copula:special:naming} can be seen as a special case of naming. However,   it does not apply to persons, but to a problem caused by an organization.




\xbox{16}{
\ea\label{ex:form:copula:special:naming}
\gll Itthu    vakthu kithang=nang nya-aada     \textbf{asàdhaathang} ini      \textbf{JVP}  \textbf{katha} hathu  {\em problem}. \\
     \textsc{dist} time \textsc{1pl}=\textsc{dat} \textsc{past}-exist \textsc{copula} \textsc{prox} JVP \textsc{quot} \textsc{indef} problem. \\
    `What we had at that time was the so-called JVP-problem.' (K051206nar10)
\z
}

The copula can also be used for more complex relations, like the complement clause in the following sentence.


\xbox{16}{
\ea\label{ex:form:copula:special:complementclause}
\gll Suda karang kithang=nang   aada  {\em problem} \textbf{dhaathangapa} [kithang=pe     aanak pada mlaayu thama-oomong]. \\ % bf
 so now \textsc{1pl}=\textsc{dat} exist problem \textsc{copula} \textsc{1pl}=\textsc{poss} child \textsc{pl} Malay \textsc{neg.irr}-speak \\
    `So our problem is now that our children do not speak Malay.' (G051222nar01)
\z
}


A complement clause is also introduced by the copula in the following sentence. Actually, it could be argued that the whole stretch of following utterances is predicated on \trs{Sepakthakrawpe rules}{the rules of Sepaktakraw}.

\xbox{16}{
\ea\label{ex:form:copula:special:stretch}
\ea
\gll Sepakthakraw=pe {\em rules} \textbf{dhaathangapa}. \\ % bf
      Sepaktakraw=\textsc{poss} rules \textsc{copula} \\
    `The rules of Sepaktakraw are as follows:'
\ex
\gll Ini hathu {\em badminton} {\em court}=ka arà-maayeng. \\ % bf
      \textsc{prox} \textsc{indef} badminton court=\textsc{loc} \textsc{non.past}-play \\
    `You play it on a badminton court.'
\ex
\gll {\em Game} hatthu itthe same {\em measurement}, {\em badminton}=pe {\em height}=le {\em same}=jo. \\ % bf
     game \textsc{indef} other same measurement badminton=\textsc{poss} height=\textsc{addit} same=foc. \\
    `In a game, you use the same measurements and the same height as in badminton.' (N060113nar05)
\z
\z
}


A property is introduced by the copula in the following sentence.


\xbox{16}{
\ea\label{ex:form:copula:special:property}
\gll Gampola=ka    bannyak mulbar. Giithu=le thraa. Siini \textbf{dhaathangapa} \textbf{{\em mixed}} \\
      Gampola=\textsc{loc} many Tamil like.that=\textsc{addit} \textsc{neg} here \textsc{copula} mixed \\
    `There are many Hindus in Gampola. It's not that much here. Here, the population is mixed.' (G051222nar04)
\z
}

An object is specified in the following example.

\xbox{16}{
\ea\label{ex:form:copula:special:object}
\gll Itthu    asàdhaathang [baaye=nang   vaasil-kang     aada  {\em dagger}] hatthu. \\ % bf
      \textsc{dist} \textsc{copula} good=\textsc{dat} blessing-\textsc{caus} exist dagger \textsc{indef} \\
    `That one was a well-blessed dagger.' (K051206nar02)
\z
}

Finally, a date is introduced by the copula in \xref{ex:form:copula:special:date}.


\xbox{16}{
\ea\label{ex:form:copula:special:date}
\gll Ini      laaher  dhaathangapa {\em 1940}=ka. \\ % bf
      \textsc{prox} birth \textsc{copula} 1940=\textsc{loc} \\
    `This birth took place in 1940.' (N060113nar03)
\z
}

Missing arguments from preceding discourse can also be introduced by the copula.


\xbox{16}{
\ea\label{ex:form:copula:special:missingargument1}
\gll Itthu    asàdhaathang inni Raagala subala. \\ % bf
     \textsc{dist} \textsc{copula} \textsc{prox} Raagala side. \\
    `That was in the Raagala region.' (K051205nar05,K081105eli02)
\z
}


\xbox{16}{
\ea\label{ex:form:copula:special:missingargument2}
\gll Itthu abbisdhaathang {\em custard} {\em powder}=dering=jo arà-kirja. \\ % bf
      \textsc{dist} \textsc{copula} custard powder=\textsc{abl}=\textsc{emph} \textsc{non.past}-make \\
    `This one is such that it is made with custard powder.' (K061026rcp02,K081105eli02)
\z
}


Note that in the great majority of cases, an English loan word is present. This might trigger some patterns containing copulas in the speakers' minds. On the other hand, there are normally no English loan words in the naming use, which is the most frequent one. However, loanwords and names are treated alike in other areas of the grammar as well, namely in that they often co-occur with the quotative \em katha \em \formref{sec:morph:katha}. This connection might merit further research.
To close this section, let's have a look at a short stretch of discourse with three occurrences of \em asdhaathang\em, one as conjunctive participle and two as a copula, for introducing a profession and a name, respectively.




\xbox{16}{
\ea\label{ex:form:copula:special:triple}
\ea
\gll Se=ppe    kaake       \textbf{asàdhaathang} {\em estate} {\em tea} {\em factory} {\em officer}. \\
     \textsc{1s}=\textsc{poss} grandfather \textsc{copula} estate tea factory officer. \\
    `My grandfather was estate tea factory officer.'
\ex
\gll {\em Estate} {\em tea} {\em factory} {\em officer}. Itthusubbath=jo      incayang=yang    siithu anà-braanak \\ % bf
    estate tea factory officer therefore=\textsc{emph} \textsc{3s.polite}=\textsc{acc} there \textsc{past}-be.born. \\
    `Estate tea factory officer. Because of that, he was born here'
\ex
\gll Siithu asà-braanak   incayang=pe      plajaran=nang  incayang  Kandi=nang   anà-dhaathang. \\ % bf
     there \textsc{cp}-be.born \textsc{3s.polite}=\textsc{poss} education=\textsc{dat} \textsc{3s.polite} Kandy=\textsc{dat} \textsc{past}-go. \\
    `After his birth, he went to Kandy for his education.'
\ex
\gll Kandi=nang   \textbf{asà-dhaathang} Kandi=ka    asà-kaaving=apa       itthu=nang blaakang=jo    kithang pada anà-bìssar. \\
      Kandy=\textsc{dat} \textsc{cp}-come Kandy=\textsc{loc} \textsc{cp}-marry=after \textsc{dist}=\textsc{dat} after=\textsc{emph} \textsc{1pl} \textsc{pl} \textsc{past}-big \\
    `After having come to Kandy and after having married, after that we grew up.'
\ex
\gll Kaake=pe      naama  \textbf{asàdhaathang} TN   Salim. \\
      grandfather=\textsc{poss} name \textsc{copula} TN Salim \\
    `My grandfather's name was TN Salim.' (K060108nar02)
\z
\z
}

%
% \xbox{16}{
% \ea\label{ex:form:unreferenced}
% \gll [See anà-pass.{\em out}] abbisdhaathang       {\em University} of Peradeniya=ka\\ % bf
%       \textsc{1s} \textsc{past}-graduate \textsc{copula} University of Peradeniya=\textsc{loc} \\
%     `Where I graduated was the University of Peradeniya.' (K061026prs01)
% \z
%. \\

%
%
% \xbox{16}{
% \ea\label{ex:form:unreferenced}
% \ea
% \gll [Itthu    arà-kirja] abbisdhaathang. \\ % bf
%      \textsc{dist} \textsc{non.past}-make copula. \\
%     `The way you make it is.'
% \ex
% \gll Thullor asà-ambel=apa       baaye=nang  asà-puukul=apa ...\\
%      egg \textsc{cp}-take=after good=\textsc{dat} \textsc{cp}-hit=after. \\
%     `you take eggs and beat them well and ..' (K061026rcp01)
% \z
% \z
%. \\









\section{Personal pronouns}\label{sec:wc:Personalpronouns}
SLM pronouns distinguish   three persons and two numbers. Additionally, there are some additional distinction of politeness and animacy. Unlike in other Malay varieties or in Tamil, there is no inclusive/exclusive distinction in the first person plural.
Table \ref{tab:PersonalPronouns} gives a list of all the words in the class of pronouns.

\begin{sidewaystable}
	\begin{center}
	\begin{tabular}{llllp{2cm}p{3cm}}
	pronoun &  =yang & =nang &=pe	& meaning & origin\\
\hline
	goo 	& goyang   	& godang & goppe 	& 1s.\textsc{familiar} (Southern dialect)& Hokkien \citep{Adelaar1991,Ansaldo2005ms} \\
	see 	& seeyang 	& sedang & seppe	& 1s.\textsc{polite}. \\
	luu 	& lu(u)yang	& ludang & luppe	& \textsc{2s.familiar} & Hokkien \citep{Adelaar1991,Ansaldo2005ms} \\
	dee 	& deeyang 	& dedang & deppe 	& \textsc{3s.inanimate}/ \textsc{3s.familiar} & \\%(eeya)
	diya	& diyayang	& diyanang & diyape	& \\%= dee (meeya)
	siaanu  & siaanuyang   	& siaanunang & siaanupe	& 3s.\textsc{proximal} \\
	incayang  & incayangyang & incayangnang	& incayangpe & \textsc{3s.polite} & \em encik ia \em + addition of velar nasal \formref{sec:phon:Finalvelarizationofnasals} \citep{Slomanson2008ismil} \\
	spaaman & spaamanyang  	& spaamannang 	& spaamanpe & \textsc{3s.polite} & uncle?? (Std, Malay, Javanese \em paman \em \citep[141]{Adelaar1985})\\
	kithang & kithangyang  	& kithannang 	& kithampe &\textsc{1pl} & \lt *kita orang\\
	lorang\footnotemark  	& lorangyang 	& lorangnang &lorampe	& \textsc{2pl}/\textsc{2s.polite} & \lt *lu orang \\
	derang  &  derangyang	& derangnang 	& derampe & \textsc{3pl} &\lt *de orang, neutral\\
	incayang pada & incayang padayang 	& incayang padanang 	& incayang padape & \textsc{3pl.polite} \\
	\end{tabular}
	\end{center}
	\caption[Pronouns]{Pronouns.  The etymology of the plural pronouns includes a contraction of \trs{orang}{man} to \em rang \em \citep{AdelaarEtAl1996,Adelaar2005struct%212
        ,Adelaar1991}%32
 	}
	\label{tab:PersonalPronouns}
\end{sidewaystable}

\footnotetext{\citet[32]{Adelaar1991} has \em lurang\em.}

The standard form to refer to the speaker is \em se \em in the Upcountry. \em Go \em is understood, but its use is frowned upon, making it one of the most stigmatized sociolinguistic features.\footnote{The cognates of \em goo \em are also stigmatized in other Malay varieties  \citep[cf.][]{Ansaldo2009book}.} I have found it in actual use in two households in Badulla, and in Slave Island (not in the Upcountry), but normally it would only be the topic of metalinguistic commentary.

The following two examples show the use of \em se \em and \em go \em with the possessive postposition \em =(p)pe \em and the dative form with \em dang\em.

\xbox{16}{
\ea\label{ex:form:pron:sese}
\gll \textbf{Se}=ppe    {\em profession}=subbath \textbf{se=dang}  siini  mà-pii    su-jaadi. \\
      \textsc{1s}=\textsc{poss} profession=because 1s.\textsc{dat} here \textsc{inf}-go \textsc{past}-become \\
    `I had to come here because of my profession.' (G051222nar01)
\z
}

\xbox{16}{
\ea\label{ex:form:pron:gogo}
\gll \textbf{Go}=ppe      {\em daughter} pada \textbf{go=dang}    makanan        arà-kaasi. \\
     1s.\textsc{familiar} daughter \textsc{pl} 1s.familiar.\textsc{dat} food \textsc{non.past}-give. \\
    `My daughters give me food.' (B060115nar04)
\z
}

One speaker can use both forms in one sentence. This is thought to be an avoidance of \em go \em which is not applied to the whole string.

\xbox{16}{
\ea\label{ex:form:pron:gose}
\gll \textbf{See}=le     pii aada  dhraapa=so duuva thiiga skali \textbf{go}  pii aada. \\
      \textsc{1s}=\textsc{addit} go exist how=many=\textsc{undet} two three time 1s.\textsc{familiar} go exist \\
    `I also went, several times, two or three times I went.' (B060115nar05)
\z
}

This is different for the forms to address the hearer, which shows a clear distinction of familiarity and politeness. \em Lu \em is almost exclusively used to address children, or very intimate friends, whereas in other contexts, \em lorang \em (otherwise second person plural) has to be used and \em lu \em would be considered offensive.

The following two sentences show the use of \em luu \em when singing a song to a child and the use of \em lorang \em when addressing a stranger.



\xbox{16}{
\ea\label{ex:form:pron:lu}
\ea
\gll \textbf{Luu}=le asà-dhaathang kanaapa nya-laaher. \\
     \textsc{2s.familiar}=\textsc{addit} \textsc{cp}-come why \textsc{past}-be.born \\
    `Why did you come to this earth?'
\gll \textbf{Luu} abbis bùssar \textbf{lu}=ppe umma-baapa=nang kaasi thaangang. \\
     \textsc{2s.familiar} finish big \textsc{2s}=\textsc{poss} mother-father=\textsc{dat} give hand. \\
    `When you will have finished growing up, lend a hand to your parents.' (K060116sng01)
\z
\z
}


\xbox{16}{
\ea\label{ex:form:pron:lorang}
\ea
\gll See=yang luppas, Thuan Buruan. \\ % bf
 \textsc{1s}=\textsc{acc} leaave sir bear\\
`Let me go, Mister Bear.'
\gll \textbf{Lorang} se=dang mà-hiidop   thumpath kala-kaasi, \\
    \textsc{2pl} 1s.\textsc{dat} info-live place if-give \\
    `If you spare my life,'
\ex
\gll see \textbf{lorang}=nang  \textbf{lorang}=pe samma duvith=le, baarang pada=le anthi-bale-king. \\
      \textsc{1s} \textsc{2pl}=\textsc{dat} \textsc{2pl}=\textsc{poss} all money=\textsc{addit} goods \textsc{pl}=\textsc{addit} \textsc{irr}-return-caus. \\
    `I will return all your money and goods to you.'    (K070000wrt04)
\z
\z
}


The third person finally shows a greater deal of variation. The most frequent way to refer to a third person topic is probably zero expression, i.e. not mentioning the person, leaving it to the hearer to infer what the the topic of the utterance is. The second most common way is \em incian/incayang, \em which is the most commonly heard overt form in the Upcountry. It seems to be less common in Colombo (Peter Slomanson p.c.). The form \em spaaman \em seems to carry greater prestige, but is not heard very often. \em Dee \em is an impolite form. The plural form \em derang \em can also sometimes be found with singular reference and does not appear to be impolite. \em Siaanu \em finally can only refer to a person that is visible. This morpheme had escaped my attention for a long time, indicating that is not very common.


The text K051205nar02 narrates the story of a thief being captured by a sergeant. The thief is referred to as impolite \em dee\em, whereas the policeman receives respectful marking, \em incayang\em.

\xbox{16}{
\ea\label{ex:form:pron:dee}
\gll \textbf{Dee} buthul jahhath. \\
      3\textsc{s.impolite} very wicked \\
    `He (the thief) was very wicked.' (K051205nar02)
\z
}

\xbox{16}{
\ea\label{ex:form:pron:incayang}
\gll \textbf{Incayang=pe}      naama  Thuan Shaaban. \\
      \textsc{3s.polite}=\textsc{poss} name Thuan Shaaban \\
    `His name (of the sergeant) was Thuan Shaaban.' (K051205nar02)
\z
}



\em Spaaman \em is used in B060115nar05 to refer to a saint.

\xbox{16}{
\ea\label{ex:form:pron:spaaman}
\ea
\gll Seelonka    avuliya pada aada.  \\ % bf
    eylon=\textsc{loc} saint \textsc{pl} exist. \\
    `There are saints in Sri Lanka.'
\ex
\gll \textbf{Spaaman}=pe     naama  Sekiilan avuliya. \\
     \textsc{3s.polite}=\textsc{poss} name Sekiilan saint. \\
    `His name was Avuliya Sekiilan.' (B060115nar05)
\z
\z
}

That saint is later referred to by \em derang\em, technically a plural pronoun, but of clear singular reference here.

\xbox{16}{
\ea
\ea
\gll Sithu=ka=jo kuburan       samma asà-gaali. \\ % bf
      there=\textsc{loc}=\textsc{emph} grave all \textsc{cp}-dig \\
    `They dug his grave right there and'
\ex
\gll Karang itthu    avuliya \textbf{derang}=pe zihaarath aada. \\
      now \textsc{dist} saint 3=\textsc{poss} shrine exist \\
    `now there is his shrine there.' (B060115nar05)
\z
\z
}

The only instance of \em siaanu \em in the corpus is given below.

\xbox{16}{
\ea\label{ex:form:pron:siaanu}
\gll Incayang=yang    \textbf{siaanu}  asà-buunung   thaaro. \\
    \textsc{3s.polite}=\textsc{acc} 3s.prox \textsc{cp}-kill put \\
    `This one has killed him.' (K051220nar01)
\z
}

The plural pronouns have less distinctions. One interesting observation is that despite their being inherently plural, the plural marker \em pada \em can optionally be used on them. Very often, the final nasal of \em kithang, lorang, derang \em then assimilates to the following labial stop in \em pada\em, yielding \em kithampada, lorampada, derampada\em. \em Derampada \em can further be reduced to \em drampada\em.


\xbox{16}{
\ea\label{ex:form:pron:kithangzero}
\gll Basra=ka    hathu  duuva haari \textbf{kithang} \zero{} anà-duuduk. \\
   Basra=\textsc{loc} \textsc{indef} two day \textsc{1pl} { } \textsc{past}-stay \\
    `We stayed one or two days in Basra.' (K051206nar19)
\z
}

\xbox{16}{
\ea\label{ex:form:pron:kithangpada}
\gll \textbf{Kithang} \textbf{pada} siini=jo    arà-duuduk. \\
       \textsc{1pl} \textsc{pl} here=\textsc{emph} \textsc{non.past}-live\\
    `We live here.' (K051206nar07)
\z
}




\xbox{16}{
\ea \label{ex:form:pron:lorangzero:irr1}
\gll Boole lìkkas=ka [see \textbf{lorang}=\zero=yang mliige=nang anthi-panggel]. \\
      can quick=\textsc{loc} \textsc{1s} \textsc{2pl}=\textsc{acc} palace=\textsc{dat} \textsc{irr}=call \\
    `As soon as possible I will call (the two of) you (girls) to the palace.'  (K070000wrt04)
\z
}


\xbox{16}{
\ea\label{ex:form:pron:lorangpada}
\gll \textbf{Lorang} \textbf{pada} pukurjan arà-gijja. \\
     \textsc{2pl} \textsc{pl} work \textsc{non.past}-make. \\
    `(The two of) you (researchers) are working.' (N061031nar01)
\z
}


Whereas there are politeness distinctions in the third person singular, this is not the case for the plural, where \em derang (pada) \em is used for any referent. As with the first and second person plural, the use of \em pada \em is optional.


\xbox{16}{
\ea\label{ex:form:pron:derangzero}
\gll Irish {\em nuns}, \textbf{derang}=\zero=pe {\em English} baaye. \\
     Irish nuns \textsc{3pl}=\textsc{poss} English good. \\
    `The English of the Irish nuns was very good.' (K051222nar06)
\z
}



\xbox{16}{
\ea\label{ex:form:pron:derangpada}
\gll {\em Dutch} {\em period}=ka \textbf{derang} \textbf{pada} dhaathang aada. \\
      Dutch period=\textsc{loc} \textsc{3pl} \textsc{pl} come exist \\
    `They came in the Dutch period.' (K051206nar05)
\z
}


The same plural referent can be referred to in the same utterance once by \em derang \em and another time by \em derang pada\em.

 \xbox{16}{
\ea\label{ex:form:pron:derangpada:double}
   \gll Itthukapang=jo         \textbf{derang} \zero{} nya-thaau   ambel \textbf{derang} \textbf{pada} {\em politic}=nang   suuka katha. \\
    then=\textsc{emph} \textsc{3pl} { } \textsc{past}-know take \textsc{3pl} \textsc{pl} politic=\textsc{dat} like \textsc{quot} \\
`Only then will they come to know that they like politics' (K051206nar12)
\z
}



\section{Interrogative pronouns}\label{sec:wc:Interrogativepronouns}
SLM has a small set of basic interrogative pronouns. These can be combined with postpositions to give a greater array of semantic possibilities. The interrogative pronouns are reduplicated to indicate that exhaustiveness of the answer is required \xref{ex:wc:pron:interr:apaaapa}, and only once to indicate that this is not so \xref{ex:wc:pron:interr:aapaanabilli}.

\xbox{16}{
\ea \label{ex:wc:pron:interr:apaaapa}
\gll Aapa\~{}aapa anà-bìlli? \\
      what\~{}\textsc{red} \textsc{past}-buy\\
    `What all did you buy?'  (K081103eli03)
\z
}

\xbox{16}{
\ea \label{ex:wc:pron:interr:aapaanabilli}
\gll Aapa anà-bìlli? \\
      what \textsc{past}-buy\\
    `What  did you buy?'  (K081103eli03)
\z
}

Besides in interrogative clauses, interrogative pronouns are also used together with clitics to form indefinite expressions \formref{sec:nppp:Nounphrasesbasedoninterrogativepronouns}. Examples are given in    \xref{ex:interr:indef} for an affirmative sentence and in \xref{ex:interr:negindef} for a  negative sentence.


\xbox{16}{
\ea \label{ex:interr:indef}
\gll Thapi \textbf{aapacara=so} itthu samma asà-iilang su-aada. \\
     But how=\textsc{undet} \textsc{dist} all \textsc{cp}-disappeared \textsc{past}-exist. \\
    `But somehow it had all disappeared.'  (K20070920eml01)
\z
}



\xbox{16}{
\ea \label{ex:interr:negindef}
\gll Bannyak haari=dering \textbf{saapa=yang=ke} thàrà-enco-kang katha anà-iingath Andare. \\
      much day=\textsc{abl} who=\textsc{pat=simil} \textsc{neg.past}-fool-\textsc{cause} \textsc{quot} \textsc{past}-think Andare \\
    `Andare thought of all the days that he had not taken anyone for a fool.'  (K070000wrt05)
\z
}


%  \xbox{16}{
%  \ea\label{ex:form:unreferenced}
%    \gll Itthu  maana maana thumpath katha kithang=nang   buthul=nang  mà-biilang    thàrboole. \\
%     \textsc{dist}    which which place \textsc{quot} \textsc{1pl}=\textsc{dat} correct=\textsc{dat} \textsc{inf}-say cannot \\
% `Where all they lived, we cannot tell exactly' (K060108nar02)
% \z
% }


A similar construction, but then with reduplicated interrogative pronouns is used to form maximizing headless relative clauses \formref{sec:nppp:NPscontaininginterrogativepronounsusedforuniversalquantification} like in \xref{ex:interr:whwhso}.


\xbox{16}{
\ea \label{ex:interr:whwhso}
\gll Inni     \textbf{saapa}\Tilde\textbf{saapa}=ka inni  mlaayu pakeyan pada aada\textbf{=so}, lorang  pada ini       mlaayu  pakeyan=samma ini       kaving=nang mà-dhaathang    bannyak uthaama. \\
prox who\Tilde who=\textsc{loc} \textsc{prox} Malay dress \textsc{pl} exist=disj \textsc{2pl} \textsc{pl} \textsc{prox} Malay dress=with \textsc{prox} wedding=\textsc{dat} \textsc{inf}-come much honour\\
`Whoever owns such  Malay dresses, your coming together with this Malay dress to the wedding will be greatly appreciated .' (K060116nar04)
\z
}


Finally, interrogative pronouns can be used with the additive \em =le \em to yield a universal quantifier.



\xbox{16}{
\ea \label{ex:interr:WHle}
\gll Skarang \textbf{maana} aari\textbf{=le} atthu atthu oorang=yang arà-buunung. \\
     now which day=\textsc{addit} one one man=\textsc{acc} \textsc{past}-kill. \\
    `Now people kill each other every day.'  (K051206nar11)
\z
}


The interrogative pronouns can also be used in subordinate clauses \xref{ex:interr:sub1}\xref{ex:interr:sub2} \formref{sec:cls:Subordinateinterrogativeclauses}.

\xbox{16}{
\ea \label{ex:interr:sub1}
\gll Andare raaja=ka su-caanya [inni mà-kirring simpang aada \textbf{aapa}=yang] katha. \\
     Andare king=\textsc{loc} \textsc{past}-ask \textsc{prox} \textsc{inf}-dry keep exist what=\textsc{acc} \textsc{quot} \\
   `Andare inquired from the King  what was it that was left [on the mat] to dry.'  (K070000wrt02)
\z
}


\xbox{16}{
\ea \label{ex:interr:sub2}
\gll [Laayeng       oorang pada \textbf{apcara}  kijja]  se   thàrà-thaau. \\
       different man \textsc{pl} how make \textsc{1s} \textsc{neg}-know\\
    `I don't know how other people do it.'  (B060115rcp02)
\z
}


\subsection{\trs{aapa}{what}}\label{sec:wc:aapa}
\em Aapa \em is used to query inanimate entities. It can be combined with a number of postpositions, e.g. \trs{aapanam}{why, what for}{} or \trs{aapadring}{with what} \xref{ex:form:interr:aapadering}. \em Aapa \em must not be confounded with the postposition \em =apa \em with a short vowel \formref{sec:morph:=apa}.


\xbox{16}{
\ea\label{ex:form:interr:aapa}
\gll \textbf{Aapa}   n-jaadi       mlaayu pada? \\
 what \textsc{past}-become Malay \textsc{pl} \\
`What became of the Malays?' (K051213nar06)
\z
}



\xbox{16}{
\ea\label{ex:form:interr:aapadering}
\gll Ini daging aapa=dering arà-poothong? \\
      \textsc{prox} meat what=\textsc{abl} \textsc{non.past}-cut \\
    `With what should I cut this meat?' (K081105eli02)
\z
}

\em Aapa \em is also often found with the accusative postposition \em =yang \em even if it does not have the semantic role of patient.\footnote{This actually contrasts with a statement in \citet[150]{Slomanson2006cll}: ``\em =nya \em [=\em yang\em] cannot be suffixed to subjects in SLM.''  In example \xref{ex:form:interr:aapa:aapayang}, the NP \em =yang \em attaches to is one part of an equational clause (the other part is a headless relative clause), and the parts of equational clauses should probably be analyzed as subjects in the theory underlying Slomanson's analysis. However, Slomanson does not assume that \em =nya \em and \em =yang \em are identical; but given his etymological argumentation, what is said about the development of \em =nya \em should be true of the development of \em =yang \em as well.}

\xbox{16}{
\ea\label{ex:form:interr:aapa:aapayang}
\gll Andare raaja=ka su-caanya [inni mà-kirring simpang aada \zero{}]  \textbf{aapa=yang} katha. \\
     Andare king=\textsc{loc} \textsc{past}-ask \textsc{prox} \textsc{inf}-dry keep exist { } what=\textsc{acc} \textsc{quot} \\
   `Andare inquired from the King  what was it that was left [on the mat] to dry.'  (K070000wrt02)
\z
}

This is presumably because very often the items queried for by \em aapa \em were patients, so that this form (over)generalized.\footnote{Note that this poses problems for the proposed etymology of \em =yang\em, which should be a patient marker which generalized from patient questions \citep[150]{Slomanson2006cll}. If \em =yang \em  can be used in non-patient questions, this analysis loses credibility.}


\subsection{\trs{saapa}{who}}\label{sec:wc:saapa}
\em Saapa \em is used to query for persons. It can also be combined with a number of postpositions, like \trs{saapanang}{for whom}, \trs{saapayang}{whom}{} or \trs{saapasàsaama}{with whom} \xref{ex:form:interr:saapasesaama}. An additional realization is \phonet{Ta:pa}. %The etymological origin is \em*siapa\em, which is still sometimes heard today.


\xbox{16}{
\ea\label{ex:form:interr:saapa}
\ea
\gll \textbf{Saapa} anà-maathi? \\
      who \textsc{past}-dead \\
    `Who died?'
\ex
\gll Samma mlaayu pada! \\ % bf
      all Malay \textsc{pl} \\
    `All Malays!' (K051213nar07)
\z
\z
}


\xbox{16}{
\ea\label{ex:form:interr:saapasesaama}
\gll \textbf{Saapa}=\textbf{sàsaama} Kluu\u mbu anthi-pii? \\
      who=\textsc{comit} Colombo \textsc{irr}-go \\
    `With whom will you go to Colombo?' (K081105eli02)
\z
}

Like the other interrogative pronouns, \em saapa \em can be used in clauses as above, or on its own as below.

\xbox{16}{
\ea\label{ex:form:saapa:utterance}
\gll \textbf{Saapa}?  See=si? \\
     who \textsc{1s}=\textsc{interr}. \\
    `Who? Me?'  (B060115prs18)
\z
}

The combination of \em saapa \em with a postposition and its indefinite use in a subordinate are illustrated in \xref{ex:form:interr:saapa:subord}.

\xbox{16}{
 \ea\label{ex:form:interr:saapa:subord}
   \gll Lorang pada asà-dhaathang  [\textbf{saapa=pe}=ke          baa   thaangang=ka=jo       pukurjan mà-gijja    athi-jaadi]. \\
\textsc{2pl} \textsc{pl} \textsc{copula} who=\textsc{simil} under hand=\textsc{loc}=\textsc{emph} work \textsc{inf}-make \textsc{irr}-become \\
`You will always have to work under someone's command' (K051206nar07)
\z
}

The use of \em saapa \em in reported questions is given in \xref{ex:form:interr:saapa:reported}.


\xbox{16}{
\ea\label{ex:form:interr:saapa:reported}
\ea
\gll Luu=nya jadi-kang rabbu saapa; lu=ppe nabi pada \textbf{saapa} katha biilang. \\
      \textsc{2s.familiar}=\textsc{acc} become-\textsc{caus} prophet who \textsc{2s}=\textsc{poss} prophet \textsc{pl} who \textsc{quot} say \\
    `Say who the prophet is who made you, who are your prophets.'
\ex
\gll Lu=ppe rabbu \textbf{saapa} katha buthul balas-an asà-biilang, \\
      \textsc{2s}=\textsc{poss} prophet who \textsc{quot} correct answer-\textsc{nmlzr} \textsc{cp}-say \\
    `Having given the correct answer as to who your prophet is,'
\ex
\gll lu=ppe nabi \textbf{saapa} katha buthul balas-an asà-biilang, ... \\
      \textsc{2s}=\textsc{poss} prophet who \textsc{quot} correct answer-\textsc{nmlzr} \textsc{cp}-say \\
    `having given the correct as to who your prophets are, ...'
\z
\z
}

%
% \xbox{16}{
% \ea\label{ex:form:unreferenced}
% \gll Saapa=nang=le ini haadarath masà-thaau. \\
%       who=\textsc{dat}=\textsc{addit} \textsc{prox} ??? must-know \\
%     `Everybody must know this hadarath.' (K061127nar03)
% \z
%. \\
%
%



\subsection{\trs{ma(a)na}{which, what}}\label{sec:wc:mana}
This pronoun is polysemous. It can be used for querying a place \xref{ex:form:interr:mana:where}, or for querying which members of a set have a positive truth value for the proposition \xref{ex:form:interr:mana:which}.

\xbox{16}{
\ea\label{ex:form:interr:mana:where}
\gll \textbf{Mana} nigiri=ka arà-duuduk? \\
 which country=\textsc{loc} \textsc{non.past}-stay\\
`Which country do you live in?' (B060115cvs16)
\z
}

\xbox{16}{
\ea\label{ex:form:interr:mana:which}
\gll Itthu blaakang se=dang karang \textbf{maana}=ke pii thàràboole. \\
      \textsc{dist} after 1s.\textsc{dat} now where=\textsc{simil} go cannot \\
    `Thereafter, I cannot go anywhere now.'  (K061120nar01)
\z
}

Just like the other interrogative pronouns, \em mana \em can combine with postpositions to increase the semantic range of possible queries, like \trs{mana(de)ri}{where from} or \em manaka\em, which highlights the locative reading. An example is given in \xref{ex:form:interr:mana:indef}, which also shows the use for indefinite reference.


\xbox{16}{
\ea\label{ex:form:interr:mana:indef}
\gll Saudi=so, \textbf{mana=ka=so}; athu nigiri=ka. \\
     Saudi.Arabia=\textsc{undet} where=\textsc{loc}=\textsc{undet} \textsc{indef} country=loc. \\
    `In Saudi Arabia or somewhere, in some country.'  (B060115nar02)
\z
}

\subsection{\trs{kaapang}{when}}\label{sec:wc:kaapang}
\em Kaapang \em is used to query for a point in time, like English \em when\em. The interrogative pronoun is nearly homonymous with the postposition \em =kapang \em meaning `then', but the two differ in the length of the vowel. There happens to be no instance of \em kaapang \em used in a question in the corpus. The following example is elicited.


\xbox{16}{
\ea
\gll Kaapang loram pada siini arà-dhaathang? \\
     when \textsc{2pl} \textsc{pl} here \textsc{non.past}-come. \\
    `When are you coming here?' (K081106eli01) 
\z
}


There are some instances of \em kaapang \em being used with clitics to yield a variety of indefinite \xref{ex:form:interr:kaapang:indef} or universal readings \xref{ex:form:interr:kaapang:pon}-\xref{ex:form:interr:kaapang:le2}.


\xbox{16}{
\ea\label{ex:form:interr:kaapang:indef}
\gll See lorang=nang   arà-simpa      \textbf{kaapang=ke}      see lorang=nang ithu    uuthang arà-baayar    katha. \\
    \textsc{1s} \textsc{2pl}=\textsc{dat} \textsc{non.past}-promise when=\textsc{undet} \textsc{1s} \textsc{2pl}=\textsc{dat} \textsc{dist} debt \textsc{non.past}-pay \textsc{quot} \\
    `I promise you that I will pay back that debt some day.' (K070000wrt04)
\z
}

\xbox{16}{
\ea\label{ex:form:interr:kaapang:pon}
\gll Suda itthu    kithang=nang   \textbf{kaapang=pon}   thama-luupa. \\
      thud \textsc{dist} \textsc{1pl}=\textsc{dat} when=any \textsc{neg.irr}-forget \\
    `So, we will never forget this.' (B060115nar02)
\z
}

\xbox{16}{
\ea\label{ex:form:interr:kaapang:le1}
\gll Go  \textbf{kaapang=le}      saala thamau-gijja. \\
     1s.\textsc{familiar} when=\textsc{addit} wrong \textsc{neg.nonpast}-make. \\
    `I never do any wrong.' (B060115nar04)
\z
}

\xbox{16}{
\ea\label{ex:form:interr:kaapang:le2}
\gll {\em Girls} {\em High} {\em School} Kandi=ka se=dang \textbf{kaapang=le}  udahan hatthu arà-kiiring. \\
     girls high school Kandy=\textsc{loc} 1s.\textsc{dat} when=\textsc{addit} invitation \textsc{indef} \textsc{non.past}-send. \\
    `I am always invited to the Girls High School in Kandy.' (K061127nar03)
\z
}

\subsection{\trs{càraapa}{how}}\label{sec:wc:caraapa}
\em Càraapa \em is used to query manner. Its  meaning is composed of \trs{aapa}{what} and \trs{caara}{way}. The inverted form \em a(a)p(a)caara \em is also possible.

There is no example of this pronoun used in a question in the corpus, hence \xref{ex:form:interr:apacaara} gives an elicited example.


\xbox{16}{
\ea\label{ex:form:interr:apacaara}
\gll Aapacara/caraapa ini arà-gijja. \\
     how \textsc{prox} pro-make. \\
    `How do you do that?' (K081103eli03)
\z
}

The corpus does contain uses in subordinate clauses, as in \xref{ex:form:interrr:apacaara:subord1}-\xref{ex:form:interrr:apacaara:subord3}.

\xbox{16}{
\ea\label{ex:form:interrr:apacaara:subord1}
\gll [Laayeng       oorang pada \textbf{apcaara}  kijja]  se   thàrà-thaau. \\
       different man \textsc{pl} how make \textsc{1s} \textsc{neg}-know\\
    `I don't know how other people do it.'  (B060115rcp02)
\z
}


\xbox{16}{
\ea\label{ex:form:interrr:apacaara:subord2}
\gll Derang thàrà-thaau  [ini      pàrhaal pada  \textbf{aapacara} anà-jaadi=so]. \\
      \textsc{3pl} \textsc{neg}-know \textsc{prox} problem \textsc{pl} how \textsc{past}-become=\textsc{undet} \\
    `They did not know how these problems had occurred.' (K051213nar01)
\z
}


\xbox{16}{
\ea\label{ex:form:interrr:apacaara:subord3}
\gll [Cinggala=nang=le {\em Dutch}=nang=le \textbf{aapcara} anà-banthu katha] thaau. \\
       Sinhala=\textsc{dat}=\textsc{addit} Dutch=\textsc{dat}=\textsc{addit} how \textsc{past}-help \textsc{quot} know\\
    `He knows how (the Malays) helped the Dutch and the Sinhalese.' (K051206nar04)
\z
}

Indefinite uses are given in \xref{ex:form:interrr:apacaara:indef1} and \xref{ex:form:interrr:apacaara:indef2}.

\xbox{16}{
\ea \label{ex:form:interrr:apacaara:indef1}
\gll Thapi \textbf{aapacara=so} itthu samma asà-iilang su-aada. \\
     But how=\textsc{undet} \textsc{dist} all \textsc{cp}-disappeared \textsc{past}-exist. \\
    `But somehow it had all disappeared.'  (K20070920eml01)
\z
}


\xbox{16}{
\ea\label{ex:form:interrr:apacaara:indef2}
\gll \textbf{Aapcara=ke}       incayang  ini      ciina oorang Islam=nang   asà-dhaathang,  asà-kaaving=apa,  karang màsiigith=nang  arà-pii. \\
     how=\textsc{simil} \textsc{3s.polite} \textsc{prox} China man Islam=\textsc{dat} \textsc{cp}-come \textsc{cp}-marry=after now mosque=\textsc{dat} \textsc{non.past}-go. \\
    `Somehow he, this Chinaman converted to Islam and married and now goes to the mosque.' (K051220nar01)
\z
}

The `semi-reduplicated' from \em caraapacara \em also exists.

\xbox{16}{
\ea
\gll Kithang caraapacara=kee mà-pii anà-aada thumpath=ka su-sampe. \\
     \textsc{1pl} how=\textsc{simil} \textsc{inf}-go \textsc{past}-exist place=\textsc{loc} \textsc{past}-reach  \\
    `We arrived at our destination  anyhow.'  (K081103eli03)
\z
}


\subsection{\trs{dhraapa}{how much}}\label{sec:wc:dhraapa}
\em Dhraapa \em is used to query an amount or a quantity. As a special case, it can also be used to query the length of a period of time when combined with \trs{laama}{while}, so \trs{dhraapa laama}{how long}.

Example \xref{ex:form:interr:dhraapa:question} shows the use of \em dhraapa \em in a question.

\xbox{16}{
\ea\label{ex:form:interr:dhraapa:question}
\gll \textbf{Dhraapa} thaaun \textbf{dhraapa} buulang lu arà-baapi suusa. \\
      how.many year how.many month \textsc{2s.familiar} \textsc{non.past}-bring sad \\
    `How many years, how many months are you bringing sadness (into my life).'  (K061123sng01)
\z
}

\em Dhraapa \em can also be combined with \em bannyak \em to emphasize the quantity. This can be done in declaratives \xref{ex:form:interr:dhraapa:bannyak:decl} but is mainly done in exclamatives \xref{ex:form:interr:dhraapa:excl1}\xref{ex:form:interr:dhraapa:excl2}.




\xbox{16}{
\ea\label{ex:form:interr:dhraapa:bannyak:decl}
\gll \textbf{Dhraapa=so}  \textbf{bannyak} {\em doctors}  pada=nang   inni     {\em native} {\em treatment}  samma anà-kirja. \\
     how.many=\textsc{undet} many doctors \textsc{pl}=\textsc{dat} \textsc{prox} native treatment all \textsc{past}-make. \\
    `Several doctors, many doctors applied these native treatments.' (K051205nar05)
\z
}


\xbox{16}{
\ea\label{ex:form:interr:dhraapa:excl1}
\gll Spuulu aanak aada  kalu \textbf{dhraapa}   \textbf{bannyak} pasiith  athi-aada. \\
      ten child exist if how.many much trouble \textsc{irr}-exist \\
    `To have ten children, how much trouble there must be!' (B060115nar04)
\z
}


\xbox{16}{
\ea\label{ex:form:interr:dhraapa:excl2}
\gll Inni     {\em two} {\em months}=ka    \textbf{dhraapa}   \textbf{bannyak}    blaajar aada! \\
      \textsc{prox} two months=\textsc{loc} how.many much   learn exist \\
    `How much he has learned in these two months!' (B060115cvs01)
\z
}

The indefinite use of \em dhraapa \em with a clitic  is shown in the following examples.


\xbox{16}{
\ea\label{ex:form:interr:dhraapa:indef1}
\gll See=le     pii aada,  \textbf{dhraapa=so}, duuva thiiga skali go  pii aada. \\
      \textsc{1s}=\textsc{addit} go exist how=many=\textsc{undet} two three time 1s.\textsc{familiar} go  exist \\
    `I also went, several times, two or three times I went.' (B060115nar05)
\z
}


\xbox{16}{
\ea\label{ex:form:interr:dhraapa:indef2}
\gll Kaake=nang  \textbf{dhraapa=so}   bannyak aanak pada,  se=dang kalu blaangang thàrà-thaau. \\
     grandfather=\textsc{dat} how.many=\textsc{undet} much child \textsc{pl} 1s.\textsc{dat} if amount \textsc{neg}-know \\
    `Grandfather had several children, many children, as for me, I do not know the number.' (K051205nar05)
\z
}


% The use of \em dhraapa puukul \em for the time of the day as well as the use with the additive clitic to yield a universal quantifier is given in \xref{ex:form:interr:dhraapa:pukul}.
% 
% \xbox{16}{
% \ea\label{ex:form:interr:dhraapa:pukul}
% \gll \textbf{Dhraapa}   \textbf{puukul=le}  thama-kuthumung. \\
%       how.much hit=\textsc{addit} \textsc{neg.irr}-see \\
%     `You cannot see anything at any hour of the day.' (K051206nar02)
% \z
% } \\

An  example of the universal quantifier is found in \xref{ex:form:interr:dhraapa:universal2}, where the additive clitic attaches to the verb.


\xbox{16}{
\ea\label{ex:form:interr:dhraapa:universal2}
\gll \textbf{Dhraapa} oorang thee\u mbak=\textbf{le}, incayang=nang thama-kìnna. \\
      how.many man shoot=\textsc{addit} \textsc{3s.polite}=\textsc{dat} \textsc{neg.irr}-affect \\
    `How ever many people shot at him, none would touch him.' (K051206nar02)
\z
}


\subsection{\trs{kànaapa}{why}}\label{sec:wc:kanaapa}

This pronoun is used to query a reason and is an alternative to \trs{aapa=nang}{what=\textsc{dat}}.



\xbox{16}{
\ea\label{ex:form:prin:interr:kanaapa}
\gll Sama oorang masà-thaksir [kithang=pe nigiri=ka \textbf{kànaapa} kithang ini bedahan arà-simpang]. \\
     all man must-think \textsc{1pl}=\textsc{poss} country=\textsc{loc} why \textsc{1pl} \textsc{prox} difference \textsc{non.past}-keep. \\
    `All men must think why we maintain these differences in our country.'
\z
}




\section{Deictics}\label{sec:wc:Deictics}
This class comprises words that serve to  encode different distances between the speaker and the entity talked about. The basic forms are proximal \trs{inni}{this} and distal \trs{itthu}{that} as given in the following two examples.


\xbox{16}{
\ea\label{ex:form:deic:inni}
\gll Suda \textbf{inni}     moonyeth pada aapa thaau=si  anà-gijja! \\
      Thus \textsc{prox} monkey \textsc{pl} what know=\textsc{irr} \textsc{past}-make \\
    `So these monkeys do you know what they did !?' (K070000wrt01)
\z
}



\xbox{16}{
\ea\label{ex:form:deic:itthu}
\gll \textbf{Itthu}    aanak pompang duuva=nang   thriima thàrà-kaasi. \\
      \textsc{dist} child female two=\textsc{dat} thank \textsc{neg.past}-give \\
    `He did not thank those two girls.' (K070000wrt04)
\z
}

Both forms can also occur without gemination.


\xbox{16}{
\ea\label{ex:form:deic:ini}
\gll Sdiikith thaaun=nang   duppang see \textbf{ini}      Aajuth=nang su-kìnna  daapath. \\
      few year=\textsc{dat} before \textsc{1s} \textsc{prox} dwarf=\textsc{dat} \textsc{past}-strike get \\
    `Some years ago, I fell prey to this dwarf.' (K070000wrt04)
\z
}


\xbox{16}{
\ea\label{ex:form:deic:ithu}
\gll See lorang=nang   arà-simpa      kaapang=ke      see lorang=nang \textbf{ithu}     uuthang arà-baayar    katha. \\ 
    \textsc{1s} \textsc{2pl}=\textsc{dat} \textsc{non.past}-promise when=\textsc{undet} \textsc{1s} \textsc{2pl}=\textsc{dat} \textsc{dist} debt \textsc{non.past}-pay \textsc{quot} \\
    `I promise you that I will pay back that debt some day.' (K070000wrt04)
\z
}

Note that the four examples above are all taken from written texts by the same speaker, so that the use of geminates in deictics can be both present and absent in the same idiolect.
In the corpus, \em itthu \em (452 instances) is much more common than \em ithu \em (96). The difference is smaller between \em inni \em (224) and \em ini \em (147).

The deictics can occur with all postpositions. As an example, \xref{ex:form:deic:innipe} gives the use of the proximal deictic combined with the possessive marker \em =pe\em.


\xbox{16}{
\ea\label{ex:form:deic:innipe}
\gll Se=dang kalu, suda bannyak thàrà-thaau  \textbf{inni=pe}         atthas mà-biilang=nang. \\
     1s.\textsc{dat} if thus much \textsc{neg}-know \textsc{prox}=\textsc{poss} about \textsc{inf}-say=\textsc{dat} \\
    `So, as for me, I cannot tell you much about this.' (K051205nar04)
\z
}


Deictics can be used either to modify a noun as in the examples \xref{ex:form:deic:inni}-\xref{ex:form:deic:ithu} above, or referentially as in \xref{ex:form:deic:innipe} or below.


\xbox{16}{
\ea\label{ex:form:deic:ref:ini}
\gll \textbf{Ini}      kithang=pe     ruuma. \\
      \textsc{prox} \textsc{1pl}=\textsc{poss} house \\
    `This is our house.' (K051222nar02)
\z
}

\xbox{16}{
\ea\label{ex:form:deic:ref:itthu}
\gll Andare=yang   mà-enco-king=nang   raaja su-biilang  [\textbf{itthu}    paasir] katha. \\
     Andare=\textsc{acc} \textsc{inf}-fool-\textsc{caus}=\textsc{dat} king \textsc{past}-say \textsc{dist} sand \textsc{quot} \\
    `To fool Andare, the king said that that was sand.' (K070000wrt02)
\z
}

The normal function of the deictics is to indicate the spatial location of a referent \funcref{sec:func:Givingthedeicticreferencespacewithregardtospeechactparticipants}. In the following example, two women scream from different sides. One side is indicated with \em itthu\em, the other one with \em inni\em.

\xbox{16}{
\ea\label{ex:form:deic:intro}
\ea
\gll Biini \textbf{itthu}    subla=dering  arà-bitharak; \\
      wife \textsc{dist} side=\textsc{abl} \textsc{non.past}-scream \\
    `(Andare's) wife screamed from that side;'
\ex
\gll puthri \textbf{inni}     subla=dering  arà-bitharak. \\
      queen \textsc{dist} side=\textsc{abl} \textsc{non.past}-scream \\
    `the queen screamed from this side.' (K070000wrt05)
\z
\z
}

Occasionally, proper nouns can also be found together with deictics. In that case, the deictics serve to emphasize the proper noun.

\xbox{16}{
\ea\label{ex:form:deic:propernoun}
\gll See \textbf{ini}      \textbf{Sri}  \textbf{Lanka}=ka    nya-blaajar. \\
      \textsc{1s} \textsc{prox} Sri Lanka=\textsc{loc} \textsc{past}-learn \\
    `I studied here in Sri Lanka.' (K061026prs01)
\z
}

\xbox{16}{
\ea \label{ex:apa:tense}
\gll Suda kithang karang \textbf{inni} \textbf{Gampola}=nang anà-dhaatham=pa  hatthu thuuju thaaun=ka    s-jaadi. \\
 thus \textsc{1pl} now \textsc{prox} Gampola=\textsc{dat} \textsc{past}-come=after \textsc{indef} seven year=\textsc{loc} \textsc{past}-become\\
    `So, now, it has become about seven years after we moved here to Gampola.'  (G051222nar01)
\z
}



There are a number of regular correspondences between the lexemes in this class and a number of adverbs \formref{sec:wc:Spatialadverbs}, which are listed in Table \ref{tab:Deictics}.
 
% For indicating locations, there is a third dimension, \em sanaka \em which indicates that the referent is closer to the hearer than to the speaker.  

\begin{table}
	\begin{center}
	% use packages: array
	\begin{tabular}{lllllll}
	base form & place  	& location	&source & manner& amount	& distance \\
 	\hline
	ini 	& siini 	& siinika 	&sindari&  giini & sgiini 	& close to speaker \\
	ithu 	& siithu 	& siithuka 	&sithari& gitthu& sgiithu 	& away from speaker \\
	\end{tabular}
	\end{center}
	\caption{Deictics}
	\label{tab:Deictics}
\end{table}


There might be a possibility of a third deictic \em anà\em, which was found in one elicitation session but could not be reproduced. The meaning of this deictic is not proximal, and thus different from \em ini\em, but the difference between \em anà \em and \em itthu \em could not be established.

While continental varieties of Tamil only have a twofold contrast, at least Jaffna Tamil has a threefold contrast in deictics (proximal \em i-\em, medial \em u-\em, distal \em a-\em), while Sinhala has a fourfold contrast, adding an anaphoric deictic \citep{Gair1982Isolate,Gair1991deixis}.



\section{Quantifiers}\label{sec:wc:Quantifiers}
There are five quantifiers in SLM: \trs{sdiikith}{few}, \trs{konnyom}{few}, \trs{spaaru}{some}, \trs{bannyak}{many}{} and \trs{sa(a/m)ma}{all}. Another common way to express universal quantification is the  WH\em=le\em-construction discussed in \formref{sec:nppp:NPscontaininginterrogativepronounsusedforuniversalquantification}.

Quantifiers are distinguished from verbs and adjectives by their inability to combine with TAM-morphology. From nouns, they are distinguished by the fact that they are able to modify a predication without having to be derived with \em =nang \em as can be seen from example \xref{ex:quant:nyaanyi}.


\xbox{16}{
\ea \label{ex:quant:nyaanyi}
\gll Incayang bannyak(*=nang) arà-nyaanyi. \\
      \textsc{3s.polite} much(=\textsc{dat}) \textsc{non.past}-sing \\
    `He sings a lot.' (K081103eli02)
\z
}

Normally, quantifiers are used as modifiers  as in \xref{ex:quant:mod:n} for a noun, \xref{ex:quant:mod:adj} for an adjective or \xref{ex:quant:mod:v} for a verb.


\xbox{16}{
\ea \label{ex:quant:mod:n}
\gll Itthu=nam       blaakang \textbf{bannyak} \textbf{oorang} pada siini se-duuduk. \\
     \textsc{dist}=\textsc{dat} after many man \textsc{pl} here \textsc{past}-stay. \\
    `After that, many people lived here.'  (G051222nar03)
\z
}


\xbox{16}{
\ea \label{ex:quant:mod:adj}
\gll Suda inni kaving \textbf{bannya(k)} \textbf{panthas}. \\
 thus \textsc{prox} wedding much beautiful\\
`So this wedding was very beautiful.' (K060116nar04)
\z
}


\xbox{16}{
\ea \label{ex:quant:mod:v}
\gll Itthu    kumpulan=dang      derang=jo     \textbf{bannyak} arà-\textbf{banthu}. \\
      \textsc{dist} association=\textsc{dat} \textsc{3pl}=\textsc{emph} much \textsc{non.past}-help\\
    `It is them who help that association a lot.'  (B060115cvs01)
\z
}

Less often, they are used referentially as in \xref{ex:form:quant:ref}-\xref{ex:form:quant:ref4}.


\xbox{16}{
\ea\label{ex:form:quant:ref}
\ea
\gll \textbf{Samma}=le atthi-maati. \\
 all=\textsc{addit} \textsc{irr}-die\\
`All would die.'
\ex
\gll \textbf{Konnyong} atthi-salba. \\
 little \textsc{irr}-escape\\
`Few would escape.' (K081105eli02)
\z
\z
}

\xbox{16}{
\ea\label{ex:form:quant:ref2}
\gll \textbf{Bannyak} se=dang e-daapath. \\
much 1s.\textsc{dat} \textsc{past}-get \\
`I earned a lot.' (N060113nar04)
\z
}


\xbox{16}{
\ea\label{ex:form:quant:ref3}
\gll Incayang=pe      baa=ka      \textbf{spaaru} aada. \\
3.\textsc{polite}=\textsc{poss} down=\textsc{loc} some exist\\
`Below him, there are (only) few (Malay houses).' (K051213nar05)
\z
}


\xbox{16}{
\ea\label{ex:form:quant:ref4}
\gll \textbf{Spaaru} Indonesia=dering      dhaathang aada. \\
some Indonesia=\textsc{abl} come exist \\
`Some came from Indonesia.' (K060108nar02)
\z
}

Even rarer is the use as a predicate as in \xref{ex:form:quant:pred}.

\xbox{16}{
\ea\label{ex:form:quant:pred}
\gll Buthul \textbf{konnyong}=jo mulbar, hatthu {\em period}. \\
       very few=\textsc{emph} Tamil \textsc{indef} period\\
    `Tamil was very sparse at that time.' (K051222nar06,K081105eli02)
\z
}



Both the referential and the modifying use can be seen in \xref{ex:form:quant:refmod}.

\xbox{16}{
\ea\label{ex:form:quant:refmod}
\gll \textbf{Spaaru} pada \textbf{bannyak} suuka arà-blaajar. \\
      some \textsc{pl} much like \textsc{non.past}-learn\\
    `(Only) some like to study a lot.'  (B060115cvs01)
\z
}

%
% \xbox{16}{
% \ea\label{ex:form:unreferenced}
% \gll Itthu maakang=nang blaakang konnyong nanthok=ke athi-dhaathang. \\
%      \textsc{dist} eat=\textsc{dat} after little sleepy=\textsc{simil} \textsc{irr}-come. \\
%     `After eating this, (you) get a little sleepy.' (K061026rcp04)
% \z
%. \\

Normally quantifiers precede the element they modify \xref{ex:form:quant:mod:pre:direct}, but other material can intervene between the quantifier and the head word, such as \em Muslim \em in \xref{ex:form:quant:mod:pre:intervening}.

\xbox{16}{
\ea \label{ex:form:quant:mod:pre:direct}
\gll Inni     sudaari=pe   femili=ka    \textbf{bannyak} \textbf{oorang} tsunami=da     spuukul su-pii. \\
      \textsc{prox} sister=\textsc{poss} family=\textsc{loc} many people tsunami=\textsc{dat} \textsc{cp}-hit \textsc{past}-go\\
    `In this sister's family, many people were swept away by the tsunami.' (B060115nar02)
\z
}


\xbox{16}{
\ea \label{ex:form:quant:mod:pre:intervening}
\gll \textbf{Bannyak} Muslim oorang pada  arà-duuduk. \\
      many Muslim man \textsc{pl} \textsc{non.past}-exist.\textsc{anim} \\
    `There are many Muslims.' (K061026prs01)
\z
}

Occasionally, quantifiers float in the sentence.  An example for this are given below, where the underscore marks the expected position of the quantifier.





\xbox{16}{
\ea \label{ex:form:quant:float:right}
\gll \_\_ cinggala   su-aada      \textbf{sdiikith}. \\
     { } Sinhala \textsc{past}-exist few. \\
    `There were few Sinhalese.'  (K051222nar06)
\z
}




\section{Numerals}\label{sec:wc:Numerals}
SLM can use native numbers until 999,999 (Table \ref{tab:Numerals}). The words for the powers of 10 have a /s-/ which disappears in their multiples. This is a relic of historical \trs{satthu}{one}.

\begin{table}
	\centering
		  \begin{tabular}{ll|ll|ll}
  			1 & (s/h)a(t)thu 	& 11 & subblas   & 100 		& sraathus \\
  			2 & duuva	& 12 & dooblas   &  200 	& duuva raathus \\
  			3 & thiiga	& 13 & thigablas &   300 	& thiiga raathus \\
  			4 & ùmpath	& 20 & duvapulu	 &  1000	& sriibu\\
  			5 & liima	& 21 & duvapulsatthu&  2000	& duuva riibu \\
  			6 & ìnnam	& 22 & duvapulduuva &   3000	& thiiga riibu\\
  			7 & thuuju	& 23 & duvapulthiiga & 26452	& \multirow{3}{4cm}{duvapulu ìnnam riibu ùmpath raathus limapulu duuva} \\
  			8 & dhlaapan	& 30 & thigapulu	&  \\
  			9 & sbiilan	& 40 & mpathpulu	&  \\
  			10 & spuulu	& 90 & sbilanpulu	&  \\
  		\end{tabular}
 			\caption{Numerals}
  		\label{tab:Numerals}
\end{table}




The following sentences show use of the smaller numbers in natural speech.

%
% \xbox{16}{
% \ea\label{ex:form:unreferenced}
% \gll Se=dang aade pada mpath arà-duuduk. \\
% 1s.\textsc{dat} younger.sibling \textsc{pl} four \textsc{non.past}-stay \\
% `I have four younger siblings.' (nosource)
% \z
% }


% 
% \xbox{16}{
% \ea\label{ex:form:numeral:1}
% \gll Incayang=jo     \textbf{sathu}       vakthu=nang   {\em Malaysia}=dering  dhaathang aada. \\
%       \textsc{3s.polite}=\textsc{emph} one time=\textsc{dat} Malaysia=\textsc{abl} come exist \\
%     `He had come from Malaysia one time.' (K060108nar02)
% \z
% }

\xbox{16}{
\ea\label{ex:form:numeral:1:1:2}
\gll Kithang \textbf{hathu}  {\em week}=nang \textbf{hathu} skaali \textbf{duuva} skaali=ke arà-maakang. \\
     \textsc{1pl} one week=\textsc{dat} one time two time=\textsc{simil} \textsc{non.past}-eat. \\
    `We might eat it one or twice a week.' (K061026rcp04)
\z
}

% \xbox{16}{
% \ea \label{ex:form:numeral:1}
% \gll {\em School}=ka  kithang  mulbar aanak pada  cinggala aanak pada  sraani pada mlaayu pada samma hatthu  samma  \textbf{hatthu}=nang=jo anà-duuduk. \\
%      School=\textsc{loc} \textsc{1pl}  Tamil child \textsc{pl} Sinhala child \textsc{pl} Burgher  \textsc{pl} Malay \textsc{pl} all one all one=\textsc{dat}=\textsc{emph} \textsc{past}-stay\\
%     `In school, the Tamil children, the Sinhalese children, Burghers and Malays, we all stayed in one class.' (K051213nar03)
% \z
%. \\




\xbox{16}{
\ea\label{ex:form:numeral:2}
\gll Pon=pe ruuma=ka=jo thaama \textbf{duuva} \textbf{thiiga} aari atthi-duuduk. \\
     bride=\textsc{poss} house=\textsc{loc}=\textsc{emph} earlier two three days \textsc{irr}-stay. \\
    `They used to stay two or three days in the bride's house.' (K061122nar01)
\z
}



\xbox{16}{
\ea\label{ex:form:numerals:3:1:1}
\ea
\gll Thapi ithu \textbf{thiiga}=nang arà-baagi. \\
     but \textsc{dist} three=\textsc{dat} \textsc{non.past}-divide. \\
 `But you divide it into three (parts).
\ex
\gll \textbf{Thiiga} asà-baagi. \\
      three \textsc{cp}-divide \\
`Having divided it,'
\ex
\gll \textbf{hatthu} kithang=pe mà-use-king=nang; \\
     one \textsc{1pl}=\textsc{poss} \textsc{inf}-use-\textsc{caus}=\textsc{dat} \\
`one (part) is ours to use;'
\ex
\gll \textbf{hatthu} saanak sudaara; \\
     one relative brother \\
`one is for the relatives;'
\ex
\gll hathyang miskiin=pada=nang. \\ % bf
      other poor=\textsc{pl}=\textsc{dat} \\
    `and one is for the poor.' (K060112nar01)
\z\z
}

% \xbox{16}{
% \ea\label{ex:form:numerals:3:4}
% \gll \textbf{Thiiga} \textbf{ùmpath} aada pompang pada. \\
%       three four exist girl \textsc{pl} \\
%     `There are three or four girls.' (K061019nar02)
% \z
% }

\xbox{16}{
 \ea\label{ex:form:numeral:75}
   \gll Incayang  \textbf{thuju-pul-liima} thaaun=sangke incayang  anà-iidop. \\
    \textsc{3s.polite} seven-ty-five year=until \textsc{3s.polite} \textsc{past}-live \\
`He lived up to his 75$^{th}$ year.' (K060108nar02,K081104eli06)
\z
}

\xbox{16}{
\ea\label{ex:form:numeral:26452}
\gll \textbf{Duva-pulu}    \textbf{ìnnam} \textbf{riibu}    \textbf{ùmpath}  \textbf{raathus} \textbf{lima-pulu}    \textbf{duuva} {\em votes}  incayang=nang    anà-daapath. \\
 two-ty six thousand four hundred five-ty two votes \textsc{3s.polite}=\textsc{dat} \textsc{past}-get\\
    `He got 26,452 votes.' (N061031nar01)
\z
}

\xbox{16}{
\ea\label{ex:form:numeral:30000:65000}
\ea
\gll Kandi=ka hathu \textbf{thiga-pulu} \textbf{riibu}=kee mlaayu pada arà-duuduk. \\ % bf
     Kandy=\textsc{loc} \textsc{indef} three-ty thousand=\textsc{simil} Malay \textsc{pl} \textsc{non.past}-exist.\textsc{anim}. \\
    `There are 30,000 Malays in Kandy.'\footnotemark
\ex
\gll Punnu=le Seelong=ka arà-duuduk. \\ % bf
       full=\textsc{addit} Ceylon=\textsc{loc} \textsc{non.past}-exist.\textsc{anim} \\
    `There are a lot in Sri Lanka.'
\ex
\gll Hathu \textbf{ìnnam-pulu} \textbf{liima} \textbf{riibu}=ke. \\
      \textsc{indef} six-ty five thousand=\textsc{simil} \\
    `About 65,000.' (K060108nar02)
\z
\z
}
\footnotetext{This number is overestimated by order of magnitude. The actual number should be around 3,000.}

%
% \xbox{16}{
% \ea\label{ex:form:unreferenced}
% \gll Derang atthu skalli duuva skalli thiiga skalli biilang blaakang=le derang=dering daalang bedahan kalu. \\
%       \textsc{3pl} one time two time three time say after=\textsc{addit} \textsc{3pl}=\textsc{abl} inside difference if \\
%     `After they had said it one, two, three times, if there was difference because of them????.'  (K061122nar01)
% \z
%. \\

 In everyday speech, compound numbers are normally in English and many people find it difficult to correctly construe the Malay numbers. The following example shows a complex number occurring in natural discourse, but this is quite rare, and normally, people would say \em nineteen-fifty-two=ka\em.

\xbox{16}{
\ea\label{ex:form:numeral:1952}
\gll \textbf{Sriibu} \textbf{sbilang}-\textbf{rathus} \textbf{lima}-\textbf{pulu} \textbf{duuva}=ka se nnam {\em class}=ka. \\
 thousand nine-hundred five-ty two=\textsc{loc} \textsc{1s} six class=\textsc{loc} \\
`In 1952, I was in 6$^{th}$ grade.' (K051213nar02)
\z
}

Mixing of small Malay numbers and bigger English number is common, an example is \xref{ex:form:numeral:mixedSLME} (English numbers are written in digits in the examples, not in words).

\xbox{16}{
\ea\label{ex:form:numeral:mixedSLME}
\gll Se=ppe nya-laaher {\em date} \textbf{duuva} \textbf{duuva} {\em 1960}. \\
     \textsc{1s}=\textsc{poss} \textsc{past}-be.born date two two 1960. \\
    `My birthday is 2-2-1960.' (K061019prs01)
\z
}

Powers of ten above 100,000 are often expressed in \em lakhs,\footnote{Sri Lankan English \em lakh \em  $<$ Indian English \em lakh \em $<$ several Indian languages \em la(a)kh \em $<$ Sanskrit \em lak\dotS a. \em Sinhala and Tamil do not use \em lakh \em, but \em lak\dotS aya \em and \em ila\tz cam\em, respectively.} \em a South Asian number meaning 100,000 \xref{ex:form:numeral:lakh}.

\xbox{16}{
\ea\label{ex:form:numeral:lakh}
\gll Itthuka       kithang=nang   \textbf{{\em one}} \textbf{{\em lakh}} nya-daapath. \\
     Then \textsc{1pl}=\textsc{dat} one lakh \textsc{past}-get. \\
    `Then we received  100,000 Rupees.' (K060116nar06)
\z
}



%N061124sng01.trs:sriibu sbiilan raathus thigapulu satthuka
%N061124sng01.trs:dhlapanpulu sbiiLanka


Fractions commonly used include \trs{kaarthu}{quarter} and \trs{(s)thìnnga}{(a) half}.
Percents are formed by using the dative on the numeral for `100', \em sraathus\em. The following example shows the use of both a fraction and a percentage in discussing the amount of ritual almsgiving Muslims have to perform.


\xbox{16}{
\ea\label{ex:form:numeral:percent}
\gll Deram=pe  cari-an=dering \textbf{duuva} \textbf{sthìnnga}, \textbf{sraathus}=\textbf{na},  \textbf{duuva} \textbf{sthìnnga} itthu blaangang derang zakath masà-kaasi. \\
     \textsc{3pl}=\textsc{poss} find-\textsc{nmlzr}=\textsc{abl} two half hundred=\textsc{dat} two half \textsc{dist} among \textsc{3pl} zakat must-give. \\
    `From their earnings, they must give 2 1/2\%, 2 1/2, that amount they must give as \em Zakat\em.' (K061122nar01)
\z
}

% \xbox{16}{
% \ea
% \gll Ini \textbf{duuva}=le laayeng laayeng kithang=pe {\em hill} country Malay {\em club}=le kandy {\em Malay} association=le. \\
%       \textsc{prox} two=\textsc{addit} other other \textsc{1pl}=\textsc{poss} hill country Malay club=\textsc{addit} Kandy Malay association=\textsc{addit}. \\
%     `These two are different, our Hill Country Malay Club and the Kandy Malay Association.' (K060116nar07)
% \z
%. \\








%N061124sng01.trs:sriibu sbiilan raathus thigapulu satthuka
%N061124sng01.trs:suda thigapulu satthuka duuduk=apa thigapulu ìnnam sangke
%N061124sng01.trs:dhlapanpulu sbiiLanka

All cardinals are of Malay origin. The ordinals are derived by prefixing \em ka-\em, thus \em kaduuva, kathiiga, ka-mpath \em etc. The ordinal for \trs{satthu}{one} is an exception: it is \em kàthaama\em.

\xbox{16}{
\ea\label{ex:form:numeral:kathaama}
\gll Thahun baaru muusing=pe     \textbf{kàthaama} aari=ka. \\
      year new season=\textsc{poss} first day=\textsc{loc} \\
    `On the first day of the new year.' (K070000wrt04)
\z
}




\section{Interjections}\label{sec:wc:Interjections}
Interjections are morphemes which can form a whole utterance on their own, and are not morphologically integrated into the grammatical system of the language.

The most frequent interjection is \trs{iiya}{yes}
This interjection is used to give affirmative answers.

\xbox{16}{
 \ea
\gll SN: Butthul {\em sportsman}, bukang? \\ % bf
     { } correct sportsman, \textsc{tag} \\
    `They are god sportsmen, aren't they?'
\ex
\gll SLM: \textbf{Iiya} \textbf{iiya}, {\em sportsman}. \\
     { } yes yes sportsman. \\
    `Yes indeed, they are good sportsmen.'

\el
\ex
\gll SN: {\em Football} aada? \\ % bf
     {  } Football exist. \\
    `Is there football being played in Badulla.'
\ex
\gll SLM: \textbf{Iiya} {\em football} aada. \\ % bf
     {  }  yes football exist \\
    `Yes, there is football.' (B060115cvs01)
\z
}
%\footnote{Some people indicate that \em aha \em is a better word for `yes' but this is a minority, and they actually do not use that word themselves.} \em iya \em is often doubled, especially when the answer is supposed to have been expected. \funcref{sec:func:Answering}

The negative counterpart \em thraa \em is not treated as an interjection, but as a particle in this description \formref{sec:wc:thraa}. This is because \em thraa \em has other meanings than `negative answer'. It is integrated much more tightly into the  grammar and participates in a number of constructions (verbal negation, adjectival negation, locational negation). Furthermore, it can serve as a base for causative derivation. This means that it fails the criteria for interjections, which are not integrated morphologically or syntactically.


The interjection \em saya \em  can be used instead of \em iiya \em and is considered more polite \citep[55]{Saldin2001}. Saldin states that it corresponds to Malaysian \em ia saya\em.


\em Ayyoo\em n is an expression of surprise. It seems to be of Lankan origin. There is no instance of it in the corpus, but it was frequently used in conversations I overheard.
% 
% \subsection{\em Allah\em}\label{sec:wc:Allah}
% This interjection is of course Arabic and can be used in many contexts.
% 
% 
% \xbox{16}{
% \ea\label{ex:form:interj:Allah1}
% \gll \textbf{Allah},\footnotemark diyath-la inni pompang pada dhaathang aada. \\
%       Allah see-\textsc{imp} \textsc{dist} female \textsc{pl} come exist \\
%     Oh, see that woman has come.' (K061019nar02)
% \z
% } \\
% \footnotetext{The final `h' is not necessarily pronounced, but present in the written form.}
% 
% 
% \xbox{16}{
% \ea\label{ex:form:interj:Allah2}
% \gll \textbf{Allah} se=dang thàrà-thaau derang {\em Arabic}=dering arà-caanya. \\
%       Allah 1s.\textsc{dat} \textsc{neg}-know \textsc{3pl} Arabic=\textsc{abl} \textsc{non.past}-ask \\
%     `Jesus, I didn't know. They were speaking in Arabic.' (K061019nar02)
% \z
% } \\
% 

There are at least two interjections of Arabic origin,  \em yammuhaiyaddiin, \em an expression of regret, and  \trs{alhamdulillah}{praise to god}.


\section{Modal particles}\label{sec:wc:Modalparticles}
Modal particles are a small class of words that are  used to indicate   permission, capacity and volition. There are four of them, namely:

\begin{itemize}
 \item \trs{boole}{can} \xref{ex:form:modals:alone:boole} \formref{sec:wc:boole}
 \item \trs{thàrboole}{cannot} \xref{ex:form:modals:alone:therboole} \formref{sec:wc:therboole}
 \item \trs{maau, kamauvan}{want} \xref{ex:form:modals:alone:kamauvan} \formref{sec:wc:(ka)mau(van)}
 \item and \trs{thussa} {\textsc{neg}.want} \xref{ex:form:modals:alone:thussa} \formref{sec:wc:thussa}.
\end{itemize}




\xbox{16}{
\ea \label{ex:form:modals:alone:boole}
\gll Boole. \\
     can \\
    `It is possible/(I/You/He/She/It/one) can.'  (K081103eli03)
\z
}


\xbox{16}{
\ea\label{ex:form:modals:alone:therboole}
\gll Derang [thàràboole]   katha su-biilang. \\ % bf
      \textsc{3pl} cannot \textsc{quot} \textsc{past}-say \\
    `They said ``It is not possible''.' (K060116nar06)
\z
}

 \xbox{16}{
 \ea\label{ex:form:modals:alone:kamauvan}
   \gll [Kamauvan] vakthu=nang,   kithang=nang   itthu    mosthor=nang,   {\em Malaysian} hathu  mosthor=nang  kithang=nang   bole=duuduk. \\ % bf
     want time=\textsc{dat} \textsc{1pl}=\textsc{dat} \textsc{dist} manner=\textsc{dat} Malaysian \textsc{indef} manner=\textsc{dat} \textsc{1pl} can-exist.\textsc{anim} \\
`When we want, we can be like this or in a Malaysian way' (K060108nar02)
\z
}


\xbox{16}{
\ea \label{ex:form:modals:alone:thussa}
\gll Thussa!  \\
     \textsc{neg}.want. \\
    `Don't (you...)!!'  (K081103eli03)
\z
}

There are many similarities in behaviour, but also some slight differences, so that the members of this class are in a relation of family resemblance\footnote{The corresponding `quasi-verbs' or `modal adjectives' in Sinhala are in a similar relationship to each other \citep{Gair1967inflcpos}.} \citep{Wittgenstein1953}. Modal particles are distinguished from verbs by their total lack of verbal morphology, such as prefixes of tense, or causative derivation. Modal particles are distinguished from nouns by their inability to take any nominal modifier, such as adjectives, deictics, the indefiniteness marker, etc.

Modal particles are independent words, which can form an utterance by themselves \xref{ex:form:modals:alone:boole}, and they can be found in different position in the clause \xref{ex:form:modals:enpreclit}.


\xbox{16}{
\ea \label{ex:form:modals:enpreclit}
\ea
\gll Boole=pii. \\
     can=go. \\
    `(X) can go'  (K081103eli03)
\ex
\gll Mà-pii boole. \\
     \textsc{inf}-go can \\
    `(X) can go'  (K081103eli03)
\z
\z
}


Furthermore, they can be used without a verb, as in \xref{ex:form:modals:noverb}, where \em thàrboole \em is  the predicate of the headless relative clause. The event of the relative clause is the same as the one denoted by the verb in the main clause, \trs{ambel}{take}, but this is not overtly realized in the relative clause since it can be inferred.

\xbox{16}{
 \ea\label{ex:form:modals:noverb}
   \gll [Se=dang thàràboole]   pada see thàrà-ambel. \\ % bf
    1s.\textsc{dat} cannot \textsc{pl} \textsc{1s} \textsc{neg.past}-take \\
`What I couldn't take, I didn't take.' (K051213nar01)
\z
}


This distinguishes modal particles from both affixes and clitics expressing modality. Those have a fixed preverbal position and cannot be used by themselves to form a complete utterance. Modal particles are grammaticalizing to clitics and affixes, though, as can be seen in  \xref{ex:form:modals:enpreclit}, where the modal procliticizes to the verb, and can undergo phonological erosion. In the preverbal position, \phonet{bOl} and \phonet{b@r} are also possible, which is not the case for the postverbal position. \em Boole \em is thus becoming very much like the other preverbal affixes, a process that has progressed to a lesser extent for the other modal particles.  \trs{Thàrboole}{cannot} can only be used in clause final position, while \em thussa \em and to a certain extent \em maau \em can also procliticize to a verb.

All modal particles require the dative \xref{ex:modals:dative} \citep[cf][]{Slomanson2008lingua}. Since the dative is very often used to mark lack of volitionality \citep{Ansaldo2005ms}, the use of the dative with modal particles can be explained by the fact that the entity feeling possibility, urge, or desire is normally not wilfully doing so.


\xbox{16}{
\ea\label{ex:modals:dative}
\gll Kithang=\textbf{nang}   mà-pii    thàràboole. \\
      \textsc{1pl}=\textsc{dat} \textsc{inf}-go cannot \\
    `We can't go.'  (K051206nar07)
\z
}

These particles can take nominal \xref{ex:form:modals:nominal} or clausal complements \xref{ex:form:modals:clausal:bare}\xref{ex:form:modals:clausal:inf}. If they take a clausal complement, the verb of the complement clause is either in the bare form \xref{ex:form:modals:clausal:bare} or in the infinitive \xref{ex:form:modals:clausal:inf}. If the modal particle is procliticized, only the bare form is possible \xref{ex:modals:clausal:pre:bare}.



\xbox{16}{
\ea\label{ex:form:modals:nominal}
\gll \textbf{Buula} maau=le          \textbf{paasir}   konnyo  maau. \\
     flour want=\textsc{addit} sugar little want \\
    `You need flour  and you need some sugar.'  (B060115rcp02)
\z
}



\xbox{16}{
\ea \label{ex:form:modals:clausal:bare}
\gll Ithukapang umma baapa su-biilang [[lorang=\textbf{nang} kaaving] \textbf{thàrboole}]. \\
      then father mother \textsc{past}-say \textsc{2pl}=\textsc{dat} marry cannot \\
    `Then the parents said that they could not marry.'  (K051220nar01)
\z
}


\xbox{16}{
\ea\label{ex:form:modals:clausal:inf}
\gll Se=\textbf{dang} karang jaau \textbf{mà}-pii thàrà-boole. \\
     1s=\textsc{dat} now far \textsc{inf}-go cannot. \\
    `I cannot go far.'  (K061120nar01)
\z
}




\xbox{16}{
\ea\label{ex:modals:clausal:pre:bare}
\gll Kithang=\textbf{nang} baaye=nang mulbar {bole}=\zero=baaca. \\
      \textsc{1pl}=\textsc{dat} good=\textsc{dat} Tamil can=read \\
    `We can read Tamil well.'  (K051222nar06)
\z
}




%
%Depending on the progress of grammaticalization, modal particles can have up to  three different forms, one full form, one geminated form and one reduced form \citep[cf.][]{Adelaar1991}. In their geminated and reduced forms, these particles procliticize to the verb \citep[cf.][7]{Slomanson2007}.




\subsection{\trs{boole}{can}}\label{sec:wc:boole}
This particle is used to indicate capacity \funcref{sec:func:Capacity} and permission \funcref{sec:func:Permission} \citep[174]{SmithEtAl2006cll}. When used as a free word, there are no allomorphs, but when procliticized, all \em boole=, bolle= \em and \em bàr(à)= \em can be found \citep[cf.][139]{Slomanson2006cll}.


\xbox{15}{
\ea \label{ex:boole:boole}
\gll [Deram pada=pe  ini     sthri pada    se-baava katha]   \textbf{boole}=biilang. \\
     \textsc{3pl} \textsc{pl}=\textsc{poss} \textsc{prox} wife \textsc{pl} \textsc{past}-bring \textsc{quot} can=say. \\
    `It is possible to say that they brought their wives.'  (K051206nar08 )
\z
}


% \xbox{16}{
% \ea \label{ex:boole:bolle}
% \gll Kitham=pe aanak \textbf{bolle}=biilang. \\
%  \textsc{1pl}=\textsc{poss} child can=say\\
% `Our child will be able to tell (that).' (K051206nar16)
% \z
% }


\em Boole \em is normally used with clausal complements, but it can also be found with nominal complements. In that case, the canonical activity associated with the noun is implied. For instance, \trs{Mlaayu boole}{Malay can} then asserts that the referent is able to speak Malay, without the verb for speaking (\em oomong\em) actually being present. The non-realization of the verb is possible because speaking is the canonical activity one performs with a language (as compared to eating, killing, or hitting). There is thus no big chance of misinterpretation on the part of the hearer.


% \xbox{16}{
% \ea \label{ex:modals:boolejaadi}
% \gll Boole=jaadi. \\ % bf
%      can-become. \\
%     `Maybe.'  (K081103eli03)
% \z
% }
% 
% 
% \xbox{16}{
% \ea \label{ex:boole:dynamic}
% \gll Bas=ka=lle        \textbf{bolle=pii}. \\
%       bus=\textsc{loc}=\textsc{addit} can=go \\
%     `You can also go by bus.'  (B060115cvs08 )
% \z
% }

The functions of \em boole \em are to indicate capacity and permission.

\xbox{16}{
\ea\label{ex:form:modals:boole:deontic}
\gll Itthu=ka  aathi=yang sajja hatthu oorang=nang \textbf{bole=ambel}. \\
      \textsc{dist}=\textsc{loc} liver=\textsc{acc} only one man=\textsc{dat} can-take \\
    `Then only one person can take the liver.'
\z
}

In example \xref{ex:form:modals:boole:deontic}, it is an ontological fact that only one person can take the intact liver, since animals only have one liver.   Two different modal  uses of \em boole \em can be found in the following example.


\xbox{16}{
\ea \label{ex:modals:booleorang}
\gll [[\textbf{Boole}] \textbf{oorang} pada]=nang   siithu \textbf{boole=pii}. \\
     can man \textsc{pl}=\textsc{dat} there can=go. \\
    `All men who are able to go may go.'  (B060115cvs01)
\z
}

Example \xref{ex:modals:booleorang} shows the use of \em boole \em without a verb in the relative clause, indicated by brackets. The (omitted) verb in the relative clause is the same as in the main clause \trs{pii}{go} and can therefore be inferred. At the same time, this sentence shows both the use of \em boole \em to indicate   ability (in the relative clause) and   permission (in the main clause). Finally, we can also see the dative marker \em =nang \em on the overt argument \trs{boole oorang pada}{the man who can}.

Another example of  \em boole \em being used in a relative clause is \xref{ex:form:modals:boole:relative}. Note that there is no special marking on the relative clause. The relativizing meaning is indicated by position only \formref{sec:cls:Relativeclause}.


 \xbox{16}{
\ea\label{ex:form:modals:boole:relative}
\gll Deram pada   [[baae=nang      pìrrang  mà-kijja     \textbf{boole}] \textbf{oorang}]. \\
      \textsc{3pl} \textsc{pl} good=nang war \textsc{inf}-make can man \\
    `They were men who could well make war.'  ( K051213nar06)
\z
}



%\xbox{16}{
%\ea\label{ex:form:unreferenced}
%\gll Kethanang thàrà-jaadi boole. \\
%\textsc{1pl}=nang \textsc{neg}-become can \\
%`???.' (B060115nar02)
%\z
%}
%
%
%\xbox{16}{
%\ea\label{ex:form:unreferenced}
%\gll Se=dang boole me-ambel thàrboole. \\
% 1s.\textsc{dat} can \textsc{inf}-take cannot\\
%`???.' (K051213nar01)
%\z
%}

%
%K060116nar10
% Kluu\u mbu oorang pukijja  boolekang   gijja


\citet{Slomanson2003} discusses the  variable positions of \em boole\em. He states that the categorical status of \em boole \em is ambiguous between a verb and an adjective, but chooses to  treat \em  boole \em as an independent finite verb. He notes certain similarities to Sinhalese \em puluvan\em, which he chooses to call \em modal adjective\em.

\subsection{\trs{thàr(à)boole}{cannot}}\label{sec:wc:therboole}
This particle is the negation of \em boole \em and expresses incapacity \funcref{sec:func:Incapacity} and interdiction \funcref{sec:func:Interdiction}.  \em Thàrboole \em can only be used postverbally and requires   the infinitive with clausal complements \xref{ex:modals:bannyakmabiilang}\xref{ex:modals:sedangkarang}. The infinitive is optionally accompanied by the dative marker \em =nang \em on the verb \xref{ex:modals:derangpadanang}. Since the pronominal argument is also marked by \em =nang\em, this means that there are two instances of \em =nang \em in this sentence, one on the pronoun and one on the verb.


\xbox{16}{
\ea \label{ex:modals:bannyakmabiilang}
\gll Bannyak \textbf{mà}-biilang \textbf{thàrboole}. \\
 much \textsc{inf}-say cannot\\
`(I) can't tell you much.' (K051206nar12)
\z
}

\xbox{16}{
\ea \label{ex:modals:sedangkarang}
\gll Se=dang karang jaau \textbf{mà}-pii \textbf{thàrboole}. \\
     1s.\textsc{dat} now far \textsc{inf}-go cannot. \\
    `I cannot go far.'  (K061120nar01)
\z
}


\xbox{16}{
\ea \label{ex:modals:derangpadanang}
\gll Derang pada=\textbf{nang} atthu=le mà-kijja=\textbf{nang} \textbf{thàràboole}=subbath ...\\
     \textsc{3pl}    \textsc{pl}=\textsc{dat} one=\textsc{addit}  \textsc{inf}-do=\textsc{dat} cannot=because ...\\
`Because they couldn't do anything, ...' (N060113nar01)
\z
}


\em Thàrboole \em is used to indicate  impossibility, which can be either factual
%  \xref{ex:form:modals:therboole:impossibility:train}
\xref{ex:form:modals:therboole:impossibility:cave} or social \xref{ex:form:modals:therboole:impossibility:smoke}.


% \xbox{16}{
% \ea \label{ex:form:modals:therboole:impossibility:train}
% \gll Jaalang asà-thuuthup=siking, lorang=nang koocci=ka Jaapna mà-pii \textbf{thàràboole}. \\
%       path \textsc{cp}-close=because \textsc{2pl}=\textsc{dat} train=\textsc{loc} Jaffna \textsc{inf}-go cannot \\
%     `Because the line is closed, you cannot go to Jaffna by train.' (K08114eli01)
% \z
% } \\

\xbox{16}{
\ea \label{ex:form:modals:therboole:impossibility:cave}
\gll Daalang=ka    {\em light}=le  mà-ambel    baapi thàràboole. \\ % bf
    inside=\textsc{loc} light=\textsc{addit} \textsc{inf}-take bring cannot. \\
    `Inside, you cannot take a light, either.'  (K051206nar02)
\z
}


\xbox{16}{
\ea\label{ex:form:modals:therboole:impossibility:smoke}
\gll {\em Cigarette} mà-miinong thàràboole. \\ % bf
      cigarette \textsc{inf}-drink cannot \\
    `It is forbidden to smoke.' (K060116nar04)
\z
}

%
% \xbox{16}{
% \ea\label{ex:form:modals:therboole:epistemic}
% \gll epistemic thàrboole. \\
%. \\
%     `.' (test)
% \z
%. \\
% incayang itthu asgijja thama aada
% incayang itthu asgijja bolle aada
% incayang itthu asgijja thraa, bolle aada
% *incayang itthu asgijja bolle thraa
% *incayang itthu asgijja thàrboole




\subsection{\trs{maau/(ka)mau(van)}{want}}\label{sec:wc:(ka)mau(van)}
This particle is used to indicate desire \citep[172f]{SmithEtAl2006cll} \funcref{sec:func:Desire}. The three possible forms are \em maau, mauvan \em and \em kamauvan\em, which are derived from the first by the suffix \em -an \em and the  historic circumfix \em ka-...-an\em. All can be used postverbally \xref{ex:form:modals:kamauvan:postverbal:mau}-\xref{ex:form:modals:kamauvan:postverbal:kamauvan}. There might be a possibility to use \em mau= \em preverbally, but elicitation about this topic was inconclusive.


\xbox{16}{
\ea \label{ex:form:modals:kamauvan:postverbal:mau}
\gll Baapa=nang {\em mosque}=nang mà-pii \textbf{maau}. \\
 father=\textsc{dat} mosque=\textsc{dat} \textsc{inf}-go want\\
`Father, you should go to the mosque.' (B060115nar04)
\z
}


\xbox{16}{
\ea \label{ex:form:modals:kamauvan:postverbal:mauvan}
\gll Cumma \textbf{mauvan} thraanang, incayang=nang fithena asà-thaarek=apa  incayang=nang su-thiikam. \\
     idle need without \textsc{3s.polite}=\textsc{dat}  quarrel \textsc{cp}-pull=after \textsc{3s.polite}=\textsc{dat} \textsc{past}-stab \\
    `Unnecessarily, he was dragged into a quarrel  and then stabbed.'  (K051220nar01,K081105eli02)
\z
}

\xbox{16}{
\ea \label{ex:form:modals:kamauvan:postverbal:kamauvan}
   \gll Kithang=nang   hathu  {\em application} mà-sign  \textbf{kamauvan} vakthu=nang=jo,      kithang arà-pii    inni     {\em politicians} pada dìkkath=nang. \\
   \textsc{1pl}=\textsc{dat} \textsc{indef} application \textsc{inf}-sign want time=\textsc{dat}=\textsc{emph} \textsc{1pl} \textsc{non.past}-go \textsc{prox} politicians \textsc{pl} vicinity=\textsc{dat} \\
`When we want to sign an application, we approach these politicians' (K051206nar12)
\z
}
 

Besides the use as a modal particle, \em kamauvan \em also has a lexical meaning of  `desire' \xref{ex:form:modals:mau:lexical:desire}, `need' \xref{ex:form:modals:mau:lexical:need} or `valuable' \xref{ex:form:modals:mau:lexical:valuable}.



\xbox{16}{
\ea\label{ex:form:modals:mau:lexical:desire}
\gll [\textbf{Kamauvan} su-aada] see=yang dhaathang {\em remand}=ka. \\
      desire \textsc{past}-exist \textsc{1s}=\textsc{acc} come remand=\textsc{loc} \\
    `They had the desire that I got remanded.' (K061122nar03)
\z
}


\xbox{16}{
\ea\label{ex:form:modals:mau:lexical:need}
\gll Incayang thàràsìggar vakthu incayang=nang duvith \textbf{kamauvan} su-aada. \\
     \textsc{3s.polite} sick time \textsc{3s.polite}=\textsc{dat} money desire \textsc{past}-exist. \\
    `When he was sick, he had a need for money.' (K060116nar07)
\z
}

\xbox{16}{
\ea\label{ex:form:modals:mau:lexical:valuable}
\gll [Ini {\em records} bannyak \textbf{kamauvan} athi-jaadi katha] kithang arà-iingath. \\
     \textsc{prox} records much valuable \textsc{irr}=become \textsc{quot} \textsc{1pl} \textsc{non.past}-think \\
    `We think that these records will become very valuable in the future.'  (N061124sng01)
\z
}

%K060116nar09.txt: derang nyabiilang    aapa kamauvan


Like the other modal particles, these three particles govern the dative, and clausal complements are in the infinitive.\footnote{\citet[172f]{SmithEtAl2006cll} report that the infinitive marker can be left out, and that the  prefix \em mas(th)i \em can be used on the verb. Dropping the infinitive marker is sometimes done in the Upcountry, but the combination of \em masthi \em and \em maau \em is completely ungrammatical.}
With nominal complements it means that the argument desires to possess the item or property under discussion \xref{ex:form:modals:mau:nominal2}. With clausal complements, it means that the subject wants to perform the action denoted by the clause \xref{ex:form:modals:mau:clausal}.



\xbox{16}{
\ea \label{ex:form:modals:mau:nominal2}
\gll Deran=nang    thumpath \textbf{maau}. \\
     \textsc{3pl}=\textsc{dat} place want. \\
    `They wanted land.'  (N060113nar01)
\z
}

\xbox{16}{
\ea \label{ex:form:modals:mau:clausal}
   \gll Kithang=nang   hathu  {\em application} mà-sign  \textbf{kamauvan} vakthu=nang=jo,      kithang arà-pii    inni     {\em politicians} pada dìkkath=nang. \\
   \textsc{1pl}=\textsc{dat} \textsc{indef} application \textsc{inf}-sign want time=\textsc{dat}=\textsc{emph} \textsc{1pl} \textsc{non.past}-go \textsc{prox} politicians \textsc{pl} vicinity=\textsc{dat} \\
`When we want to sign an application, we approach these politicians' (K051206nar12)
\z
}

 If the complement clause has no subject, it is inferred that the entity mentioned in the  matrix clause wants to perform the action \xref{ex:form:modals:mau:clausal:ss1} \xref{ex:form:modals:mau:clausal:ss2}, whereas if the person wants some other body to perform an action, this second body must be overtly mentioned as in \xref{ex:form:modals:mau:clausal:ds}.



\xbox{16}{
\ea \label{ex:form:modals:mau:clausal:ss1}
\gll Itthusubbath=jo incayang=nang \textbf{maau}, ini {\em Sri} {\em Lankan} {\em Malay} mà-blaajar \textbf{maau}. \\
  therefore=\textsc{emph} 3p.\textsc{polite}=\textsc{dat} want \textsc{prox}  Sri Lankan Malay \textsc{inf}-learn want  \\
`This is why he wants to learn this Sri Lanka Malay.' (B060115prs15)
\z
}

\xbox{16}{
\ea \label{ex:form:modals:mau:clausal:ss2}
\gll [Andare kanabisan=nang anà-mintha] [hathu raaja=ke asà-paake=apa kampong=nang mà-pii \textbf{maau} katha]. \\
    Andare last=\textsc{dat} \textsc{past}-ask \textsc{indef} king=\textsc{simil} \textsc{cp}-dress=after village=\textsc{dat} \textsc{inf}-go want \textsc{quot} \\
    `What Andare wanted as a last wish, was to go to the village dressed up as a king.'  (K070000wrt03)
\z
}

\xbox{16}{
\ea \label{ex:form:modals:mau:clausal:ds}
\gll \textbf{{\em Kithang}}=nang maau kitham=pe mlaayu \textbf{lorang} blaajar, lorang=pe mlaayu \textbf{kitham} blaajar. \\
 \textsc{1pl}=\textsc{dat} want \textsc{1pl}=\textsc{poss} Malay \textsc{2pl} learn \textsc{2pl}=\textsc{poss} Malay \textsc{1pl} learn\\
`We want that you learn our [Sri Lankan] Malay, and we learn your [Malaysian] Malay.' (K060116nar02)
\z
}

 Note that in  \xref{ex:form:modals:mau:clausal:ds}, the second person becomes topical, and that the first person must therefore be reintroduced in the final part of the sentence to get the references right, even if that is normally not necessary if the wanting person and the performing person are identical. Also note that the infinitive marker \em mà- \em is missing on the verbs in the complement clause, which is a particularity of this speaker.

Constructions with \em maau \em can also be used for mild commands \xref{ex:form:modals:mau:mildcommand} \funcref{sec:pragm:Requestingaction}. A particular use of \em maau \em is found in the genre of recipes, where it is used to indicate the needed ingredients \xref{ex:form:modals:mau:recipe}.

\xbox{16}{
\ea \label{ex:form:modals:mau:mildcommand}
\gll Baapa=nang {\em mosque}=nang mà-pii \textbf{maau}. \\
 father=\textsc{dat} mosque=\textsc{dat} \textsc{inf}-go want\\
`Father, you should go to the mosque.' (B060115nar04)
\z
}

\xbox{16}{
\ea \label{ex:form:modals:mau:recipe}
\gll Manis-an=nang mà-thaaro guula \textbf{maau}, gula paasir konnyong \textbf{maau}. \\
 sweet-\textsc{nmlzr}=\textsc{dat} \textsc{inf}-put sugar want sugar sand few want\\
`In order to make it sweet, you need sugar, you need a bit of crystal sugar.' (B060115rcp02)
\z
}


These particles are reasonably common and seem to occur in about equal proportions.


The negation of \em (ka)mauvan \em is \em thàr(ka)mauvan\em. This modal particle can only be used postverbally. The negation of \em maau \em is \em thussa\em, to be discussed in the next section.

\xbox{16}{
\ea \label{ex:form:modals:mau:tharamauvan}
\gll Go {\em size} mà-ambel \textbf{thàràmauvan}. \\
     1s.\textsc{familiar} measure \textsc{inf}-take \textsc{neg}.want. \\
    `I did not (even) need not take measures [for tailoring].'  (B060115nar04)
\z
}
%
% \xbox{16}{
% \ea\label{ex:form:unreferenced}
% \gll thàrkamauvan. \\
%. \\
%     `.'  (test)
% \z
%. \\
%


\subsection{\trs{thussa}{want.not}}\label{sec:wc:thussa}
This particle is the negation of \em maau\em, which is also often used for negative imperatives. This particle can be used  both before \xref{ex:form:modals:thussa:preverbal}  and after the verb \xref{ex:form:modals:thussa:clausal}. As a difference to the other modal particles, the infinitive can also be used when \em thussa \em is in preverbal position \xref{ex:form:modals:thussa:preverbal}. Some speakers also have a procliticized allomorph \em thus=\em.


\xbox{16}{
\ea \label{ex:form:modals:thussa:preverbal}
\gll `\textbf{Thussa} mà-thaakuth', Buruan su-biilang. \\
      neg.\textsc{imp} \textsc{inf}-fear bear \textsc{past}-say\\
    ` ``Don't be afraid,''  said Bear./``I don't want you to be afraid,'' said Bear '  (K070000wrt04)
\z
}


%
% \xbox{16}{
% \ea \label{ex:form:modals:thussa:preverbal2}
% \gll Se=yang \textbf{thussa} mà-liiyath. \\
%      \textsc{ \textsc{1s}=acc} not.want \textsc{inf}-look\\
%     `Don't stare at me!'  (eli12012006)
% \z
% }



\em Thussa \em can also be used with clauses in which the actor is different from the wanter \xref{ex:form:modals:thussa:clausal}.


\xbox{16}{
\ea \label{ex:form:modals:thussa:clausal}
\gll Se=dang lorang ini buuva mà-picca-kang=nang thussa. \\
     1s.\textsc{dat} \textsc{2pl} \textsc{prox} fruit \textsc{inf}-broken-\textsc{caus}=\textsc{dat} \textsc{neg}.want \\
    `I do not want you to pluck this fruit.'  (K081103eli03)
\z
}



Just like \em maau, thussa \em can be used with verbal \xref{ex:form:modals:thussa:preverbal}, nominal \xref{ex:form:modals:thussa:nominal} and with clausal complements \xref{ex:form:modals:thussa:clausal}. Additionally, \em thussa \em can be used on its own as a command, similar to the English \em Don't!\em


\xbox{16}{
\ea \label{ex:form:modals:thussa:nominal}
\gll Se=dang \textbf{paayong}=yang \textbf{thussa}. \\
     \textsc{1s=dat} umbrella=\textsc{dat} \textsc{neg}.want \\
    `I don't want an umbrella, thank you.' (K081103eli03)
\z
}



%\xbox{16}{
%\ea\label{ex:form:unreferenced}
%\gll Se=biilang thussa baapa hatthu=le nya-kirijja. \\
% \textsc{past}-say \textsc{neg}.want father one=\textsc{addit} \textsc{past}-make\\
%`I said ``Don't do anything, Dad''.' (B060115nar04)
%\z
%}

%


%
%\xbox{16}{
%\ea\label{ex:form:unreferenced}
%\gll Deram=nang paasar=nang thussa aada. \\
% \textsc{3pl}=\textsc{dat} shop=\textsc{dat} \textsc{neg}.want exist\\
%`???.' (K051220nar2)
%\z
%}


The function of \em thussa \em is to express lack of desire \funcref{sec:func:Lackofdesire}. It can also be used for interdictions \funcref{sec:func:Interdiction}.
\citet{Adelaar1991} gives the etymon as \em *tra usa \em \citep[also see][60]{Saldin2001}.
%sedang paayongyang thussa 031106=3

\section{Negative Particles}\label{sec:wc:NegativeParticles}
There are two negative particles, \em thraa \em \formref{sec:wc:thraa} and \em bukang \em \formref{sec:wc:bukang}. The latter is used for negating nominal predicates, and for constituent negation. The former is used for negating adjectival and locational predicates, and verbal predicates in the perfect tense.\footnote{Verbal negation is other tenses is not done by a particle, but by the prefixes \em thàrà- \em \formref{sec:morph:thara-} for past and \em thama- \em \formref{sec:morph:thamau-} for nonpast.}\footnote{Sinhala also has two negators, with similar uses \citep{GairEtAl1988argstr}. Continental Tamil is normally described as having only one (\em illai\em), but in Sri Lanka, an additional constituent negator \em alla \em can be found \citep{GairEtAl1981jaffnaverb}.}

\subsection{\em thraa\em}\label{sec:wc:thraa}
\em Thraa \em is the negation of \em aada\em, ($<$ \em tidak aada \em \citep[cf.][26]{Adelaar1991}). There is only one form. This form can be used for giving a negative answer as in \xref{ex:form:thraa:illocution1} \funcref{sec:pragm:Answering}, where it is used to decline an offer, in this case in reported speech.


\xbox{16}{
\ea\label{ex:form:thraa:illocution1}
\gll Ithu=kapang       derang nya-biilang    `\textbf{thraa}, kithang giithu   thama-pii.' \\
      \textsc{dist}-when \textsc{3pl} \textsc{past}-say no \textsc{1pl} like.that \textsc{neg.nonpast}-go \\
    `Then they said ``no, we cannot go like that.'' ' (K051213nar06)
\z
}

The following examples show more instances of \em thraa \em in reported speech.

\xbox{16}{
\ea\label{ex:form:thraa:illocution2}
\gll Oorang pada su-biilang: `\textbf{thraa}, \textbf{thraa}, incayang {\em saint}, incayang avuliya
hatthu jaadi aada' \\
    man \textsc{pl} \textsc{past}-say \textsc{neg} \textsc{neg} \textsc{3s.polite} saint, \textsc{3s.polite}  saint \textsc{indef} become exist.\\
    `The men said ``No, no, he is a saint, he has  become a saint.''\,'  (K051220nar01)
\z
}

\xbox{16}{
\ea\label{ex:form:thraa:illocution3}
\gll `\textbf{Thraa} \textbf{thraa} inni=yang masà-picca-kang' katha biilang. \\
      \textsc{neg} \textsc{neg} \textsc{prox} must-broken-\textsc{caus} \textsc{quot} say \\
    ` ``No, no,'' he said ``You must pick these ones.'' '  (K051220nar01)
\z
}


Additionally, \em thraa \em can be used in all the contexts where \em aada \em could be used in affirmative contexts \funcref{sec:func:Negation}. It is always found in final position then. These contexts are
\begin{itemize}
	\item existential. Negation of both concrete and abstract existence is done with \em thraa\em
	\item locationals. Since locationals are a subtype of existentials, it is little surprising that they are also negated by \em thraa\em.
	\item possession. Predicative possession is also negated by \em thraa\em.
	\item perfect tense of verbal predications. The negation of this tense is formed by changing \em aada \em to \em thraa\em.
\end{itemize}

The first function is the negation of the existential, i.e. predicating that a certain entity does not exist, like \trs{bìrras}{rice} in \xref{ex:form:thraa:nexist:abstract1}.


\xbox{16}{
  \ea\label{ex:form:thraa:nexist:abstract1}
 \gll Bìrras \textbf{thraa}. \\
	raw.rice \textsc{neg} \\
	`There is no rice.' (K051206nar12)
\z
}

Another example is \xref{ex:thraa:nexist:dead}, where the fact is conveyed that the mother does not live anymore and is thus not among the things of which one can say that they exist.


\xbox{16}{
\ea \label{ex:thraa:nexist:dead}
\gll Se=ppe umma \textbf{thraa}. \\
1s=\textsc{poss} mother \textsc{neg} \\
`My mother is dead.' (B060115prs03)
\z
}

 The following example also shows a use of \em thraa \em which does not predicate complete absence of \trs{bedahan}{difference}, but rather the absence of a lot of differences. Some differences remain, but these are a minority.


\xbox{16}{
\ea\label{ex:form:thraa:nexist1}
\gll Punnu bedahan \textbf{thraa}. \\
many difference \textsc{neg} \\
`There are not many differences.'  (K060108nar02)
\z
}

The entity whose non-existence is predicated need not be overtly present, but can be one which is inferable from context. In \xref{ex:form:thraa:nexist:drop}, the entity whose absence \em thraa \em predicates is not overtly realized.


\xbox{16}{
\ea\label{ex:form:thraa:nexist:drop}
\gll Kàthaama su-aada, karang \zero{} \textbf{thraa}. \\
	before \textsc{past}-exist now { } \textsc{neg} \\
	`Before (there) was [rugby], (but) not now.'  (B060115cvs01)
\z
}





% 		\xbox{16}{
% \ea\label{ex:form:unreferenced}
%   \gll Se=dang post bìssar thraa. \\
% 		1s.\textsc{dat} post big \textsc{neg} \\
% 		`I did not have a big post.' (K060116nar23)
% 		\z
% 		}




Absence of an entity from a certain location, rather then general absence as discussed above, is also indicated by \em thraa\em, the second use indicated above. The semantic difference between complete absence and absence from a certain place is mirrored in grammar by the possibility to use \em duuduk \em with animate entities in the latter case but not in the former.
In example \xref{ex:form:thraa:nloc:saudi}, the fact that there are neither cattle nor goats found in Saudi-Arabia is indicated by \em thraa\em. This does not entail that these animals do not exist at all, it merely entails that they are not found in Saudi-Arabia.

\xbox{16}{
\ea\label{ex:form:thraa:nloc:saudi}
\gll Saudika ontha, \textbf{samping} \textbf{kambing} \textbf{thraa}. \\
Saudi.Arabia camel, cattle goat \textsc{neg} \\
`In Saudi Arabia (they offer) camels, there are no cattle or goats.' (K060112nar01)
\z
}


In the following examples, the absence of the Malay language in a certain place and the imputed absence of grammar in the language of the speaker are indicated by \em thraa\em. In both cases, the domain from where the entity is absent is indicated by the locative postposition \em =ka\em, attached to a deictic referring to a concrete place in \xref{ex:form:thraa:nloc:mlaayu} and to a noun referring to the abstract concept \trs{bahasa}{language} in \xref{ex:form:thraa:nloc:grammar}.



\xbox{16}{
\ea\label{ex:form:thraa:nloc:mlaayu}
\gll \textbf{Itthu=ka} \textbf{mlaayu} \textbf{thraa}, bannyak=nang {\em English}=jo aada. \\
	dist=\textsc{loc} Malay \textsc{neg} much=\textsc{dat} English=\textsc{emph} exist \\
	`There is no Malay over there, it is all English which is there.'  (B060115prs15)
\z
}


\xbox{16}{
\ea\label{ex:form:thraa:nloc:grammar}
\gll Kitham pada=pe \textbf{bahasa=ka} ini \textbf{{\em grammar}} \textbf{thraa}. \\
\textsc{1pl} \textsc{pl}=\textsc{poss} language=\textsc{loc} \textsc{prox} grammar \textsc{neg} \\
`There is no grammar in our language.'  (G051222nar02)
\z
}

\em Thraa \em in these uses is underspecified for tense and can be used with past tense reference, as the following example shows, where \trs{itthu muusingka}{back in those days} clearly indicates that we are dealing with reference to the past, but this is not reflected in the behaviour of \em thraa\em, there is no prefix or other device indicating past tense on the predicate.


\xbox{16}{
\ea\label{ex:form:thraa:notense}
\gll Itthu    muusing=ka    cinggala  \textbf{thraa}. \\
      \textsc{dist} time=\textsc{loc} Sinhalese \textsc{neg} \\
    `At that time there was no Sinhala.' (K051222nar06)
\z
}



The third use of \em thraa \em is for negating possession. Just as with affirmative possession with \em aada \em \formref{sec:wc:Existentialverbs:aada}, the possessor is indicated with a postposition. This is normally the dative \em =nang \em as in \xref{ex:thraa:negposs:nang1} or \xref{ex:thraa:negposs:nang2} or its allomorph \em =dang \em used on pronouns as in \xref{ex:thraa:negposs:dang}.


\xbox{16}{
\ea
\label{ex:thraa:negposs:nang1}
\gll Ini aanak pada=\textbf{nang} {\em time} hatthu \textbf{thraa}. \\
prox child \textsc{pl}=\textsc{dat} time \textsc{indef} \textsc{neg} \\
`These children do not have any time.'  (G051222nar01)
\z
}



\xbox{16}{
\ea\label{ex:thraa:negposs:nang2}
\gll Derang pada=nang [itthu mà-kumpul athu mosthor] \textbf{thraa}. \\
\textsc{3pl} \textsc{pl}=\textsc{dat} \textsc{dist} \textsc{inf}-add \textsc{indef} way \textsc{neg} \\
`They have no way to colllect all this.'  (G051222nar01)
\z
}




\xbox{16}{
\ea\label{ex:thraa:negposs:dang}
\gll Suda \textbf{go=dang} hatthu kurang-an \textbf{thraa}. \\
so 1s.\textsc{familiar} \textsc{indef} few=\textsc{nmlzr} \textsc{neg} \\
`So, I am not missing anything.'  (B060115nar04)
\z
}


The fourth use of \em thraa \em is in negated predicates in the perfect. In this construction like in the ones mentioned before, \em thraa \em takes the position of its affirmative counterpart \em aada\em.


\xbox{16}{
\ea\label{ex:form:thraa:perf1}
\gll Itthu=nang blaakang kithang=pe hatthu oorang=le {\em minister} \textbf{jaadi} \textbf{thraa}. \\
dist=\textsc{dat} after \textsc{1pl}=\textsc{poss} \textsc{indef} man=\textsc{addit} minister become \textsc{neg} \\
`After that, not one of our men has become minister again.'  (N061124sng01)
\z
}

\xbox{16}{
\ea\label{ex:form:thraa:perf2}
\gll [Kithang baaye mlaayu arà-oomong katha incayang] \textbf{biilang} \textbf{thraa}. \\
1p good Malay \textsc{non.past}-speak \textsc{quot} \textsc{3s.polite} say \textsc{neg} \\
`He has not said that we speak good Malay.'  (B060115prs15)
\z
}



\xbox{16}{
\ea\label{ex:form:thraa:perf3}
\gll {\em Invitations} daapath \textbf{thraa}. \\
invitations get \textsc{neg} \\
`The invitations had not arrived.' (K060116nar23)
\z
}



Sometimes, \em thraa \em can also be found when negating adjectives \citep[cf.][137]{Slomanson2006cll}.

\xbox{16}{
\ea\label{ex:form:thraa:adj}
\gll Itthu muusing gampang \textbf{thraa}. \\
     \textsc{dist} time easy \textsc{neg} \\
    `It was not easy back then.'  (B060115nar05)
\z
}



In this function, \em thraa \em competes with \em thàrà-\em.
The exact conditions which trigger one or the other adjectival negation are unclear. There seems to be some variation between speakers, and even within idiolects, so that this is difficult to test. This is discussed in more detail under adjectives \formref{sec:pred:Adjectivalpredicate}.

\em Thraa \em also has  a   a verbal reading meaning `disappear, become unavailable' \xref{ex:form:thraa:verbal}. When combined with \em =nang \em it means `without' \xref{ex:form:thraa:thraanang}.


\xbox{16}{
\ea\label{ex:form:thraa:verbal}
\gll Bannyak kithang=nang \textbf{su-thraa}$_{V}$. \\
       much \textsc{1pl}=\textsc{dat} \textsc{past}-\textsc{neg} \\
    `We lost many.' (B060115nar02)
\z
}


\xbox{16}{
\ea\label{ex:form:thraa:thraanang}
\gll Cumma  mauvan  \textbf{thraanang}, incayang=nang fithena asà-thaarek=apa  incayang=nang su-thiikam. \\
     idle need without \textsc{3s.polite}=\textsc{dat}  quarrel \textsc{cp}-pull=after \textsc{3s.polite}=\textsc{dat} \textsc{past}-stab \\
    `Without need, he was dragged into a quarrel  and then stabbed.'  (K051220nar01,K081105eli02)
\z
}

\em Thraa \em can also be the source of a causative derivation, as shown in the following example.

\xbox{16}{
\ea\label{ex:form:thraa:king}
\gll Itthu=nang kithang arà-\textbf{thraa-king} kithang=pe Seelon=pe mosthor=nang. \\
      \textsc{dist}=\textsc{dat} \textsc{1pl} \textsc{non.past}-\textsc{neg}-\textsc{caus} \textsc{1pl}=\textsc{poss} Ceylon=\textsc{poss} manner=\textsc{dat} \\
    `Therefore we are losing our Ceylonese ways.' (B060115rcp01)  
\z
}


% \xbox{16}{
% \ea\label{ex:form:unreferenced}
% \gll {\em Rugger} konnyong muusing=nang thraa=nang aada. \\
%       rugby little time=\textsc{dat} \textsc{neg}=\textsc{dat} exist \\
%     `Rugby had ceased to exist.'  (B060115cvs01)
% \z
%. \\

%
% \xbox{16}{
% \ea\label{ex:form:unreferenced}
% \gll Mà-thuulis=nang=le thraa mà-baaca=nang=le thraa. \\
%      \textsc{inf}-write=\textsc{dat}=\textsc{addit} \textsc{neg} \textsc{inf}-read=\textsc{dat}=\textsc{addit} \textsc{neg} \\
%     `They can't write, they can't read.' (G051222nar02)
% \z
%. \\
%



\subsection{\em bukang\em}\label{sec:wc:bukang}
\em Bukang \em is used for the negation of nominal predicates \funcref{sec:pred:Nominalpredicate} and for the negation of constituents in a predication \funcref{sec:func:Negation}. \em bukang \em occurs after the constituent it negates.



The following examples show the use of \em bukang \em as negation of a nominal ascriptive predicate. \xref{ex:bukang:Islam} shows the use on a simple predication, while in \xref{ex:bukang:sindbad} we have an affirmative ascription \em hatthu Muslim\em, followed by a negative ascription \em mlaayu bukang\em.

\xbox{16}{
\ea \label{ex:bukang:Islam}
\gll Deram Islam oorang \textbf{bukang}. \\
 \textsc{3pl} Islam man \textsc{neg.nonv} \\
`They were not Muslims.' (K051213nar03)
\z
}


\xbox{16}{
\ea \label{ex:bukang:sindbad}
\gll Sindbad  {\em the}  {\em Sailor}     hatthu Muslim, mlaayu \textbf{bukang}. \\
 Sindbad the Sailor \textsc{indef} Muslim, Malay \textsc{neg.nonv} \\
`Sindbad the sailor was a Moor, he was not a Malay.' (K060103nar01)
\z
}

Negation of nominal predicates is the most frequent function \em bukang \em occurs in, but it can also be used for constituent negation, as example \xref{ex:bukang:sepktakraw} shows.


\xbox{16}{
\ea \label{ex:bukang:sepktakraw}
\gll Thaangang=dering  \textbf{bukang}    kaaki=dering          masà-maayeng. \\
 hand=instr \textsc{neg.nonv} foot=instr must-play\\
`You must not play with the hands, but with the feet.' (N060113nar05)
\z
}

In this example, the predicate is \trs{maayeng}{play}, but it is not the predicate which is being negated, but the adjunct giving information about the entitiy which participates in the act of playing as an instrument. This entity happens to be not the hand, as could be expected, but the foot. Hence it is the argument, realized as the constituent \trs{kaakidering}{with feet}, which is negated, and \em bukang \em as the constituent negator must be used. If the verbal negative prefix \em thàrà- \em was used on \trs{maayeng}{play}, it would be the verb which would be negated, which does not correspond to the information structure intended here. It is not the act of playing which is negated, but the instrument of playing, the hand.

A final use of \em bukang \em is to request confirmation, like a question tag as in \xref{ex:bukang:tag}. This example is a nice contrast to the use of \em bukang \em in \xref{ex:bukang:Islam}, where almost the same content is present, but used in a negation context, whereas we are dealing with a tag question here.

\xbox{16}{
\ea \label{ex:bukang:tag}
\gll Deram pada    {\em Moors},  \textbf{bukang}? \\
 \textsc{3pl} \textsc{pl} Moors, \textsc{tag} \\
`The are Moors, aren't they?' (K051206nar12)
\z
}

\em Bukang \em has a transparent etymology  and is cognate to Std. Malay \em bukan\em. It is frequently found as a negator of nominal ascriptive predicates and as a question tag, but is very infrequently found in the constituent negation context, which might have to do with the fact that the communicative need for constituent negation   arises less often.

\section{Other particles}\label{sec:wc:Otherparticles}
In this section I will discuss the remainder of SLM free morphemes, which do not fit into any other class.

\subsection{\trs{suuda}{enough}}\label{sec:wc:suuda}
This is a particle which can  combine with nouns and means that there is a sufficient quantity available \xref{ex:suuda:naasi}. It can thus be glossed as `enough'. It has to be distinguished from the adverb \em suda \em with a short vowel, which means `thus, then, so', which is given for comparison  in \xref{ex:suuda:suda}.


\xbox{16}{
\ea \label{ex:suuda:naasi}
\gll {\em Kithang}=nang bìrras \textbf{suuda}. \\
     \textsc{1pl}=\textsc{dat} raw.rice enough. \\
    `We have enough rice.'  (K081103eli04)
\z
}


\xbox{16}{
\ea \label{ex:suuda:suda}
\gll \textbf{Suda} [puthri=le biini=le arà-caa\u nda aari]=le su-dhaathang. \\
      thus queen=\textsc{addit} wife=\textsc{addit} \textsc{simult}-meet day=\textsc{addit} \textsc{past}-come \\
    `Then came also the day when the queen and the wife were to meet.'  (K070000wrt05)
\z
}

The use of a  particle for this function parallels Sinhala \em \ae ti \em and Tamil \em podum\em.

 
%
% \xbox{16}{
% \ea \label{ex:suda:tharasampe}
% \gll Jleena=pe     dìkkath kapang-duuduk       ini      pohong mà-seereth=nang                   oorang \textbf{thàrà-sampe}. \\
%       window=\textsc{poss} vicinity when-stay \textsc{dist} tree \textsc{inf}-drag=\textsc{dat} man \textsc{neg.past}-reach \\
%     `When he was standing at the window, there were not enough people to drag the tree.' (K051205nar05)
% \z
% }




 % hatthu ke ma biilang thraa=nang suda


\subsection{\trs{kalu}{if}}\label{sec:wc:kalu}
This particle is used to mark conditionals \funcref{sec:func:Conditionals}. It has a free form, which occurs  after the predicate \xref{ex:kalu:after1}\xref{ex:kalu:after3}. Additionally, there is a procliticized form \em kal(a)= \em which occurs only preverbally and blocks the use of other TAM-marking on the verb \xref{ex:kalu:preverbal1}-\xref{ex:kalu:preverbal4} \citep[153]{Slomanson2006cll}. \citet{SmithEtAl2004} also note the form \em ka(ng)-\em, which is not found in the corpus.


\xbox{16}{
\ea \label{ex:kalu:after1}
\gll Nnam thullor ar-\textbf{ambel} \textbf{kalu}. \\
 six eggs \textsc{non.past}-take if\\
`Suppose you take six eggs.' (B060115rcp02)
\z
}

% 
% \xbox{16}{
% \ea\label{ex:kalu:after2}
% \gll Itthu abbis \textbf{maakang} \textbf{kalu}  kithang=nang bole=duuduk hatthu=le jamà-maakang=nang  two duuva two  {\em o'clock}=ke  sangke bole=duuduk. \\
%      \textsc{dist} finish eat if \textsc{1pl}=\textsc{dat} can-stay \textsc{indef}=\textsc{addit} \textsc{neg.nonfin}-eat=\textsc{dat}  two two two o'clock=\textsc{simil} until can-stay. \\
%     `If we eat it up, we can stay up until two o'clock without eating anything.' (K061026rcp04)
% \z
% }
% 

\xbox{16}{
\ea \label{ex:kalu:after3}
\gll Lai     aapa=ke      \textbf{aada}  \textbf{kalu}, se=dang aathi-ka  asà-kluuling bole=caari kaasi {\em Malays} pada=pe atthas. \\
      other what=\textsc{simil} exist if 1s.\textsc{dat} heart=\textsc{loc} \textsc{cp}-roam can-search give Malays \textsc{pl}=\textsc{poss} about \\
    `If there should be something else, I am going to go through my mind, and I will forward it to you' (G051222nar01)
\z
}

% 
% \xbox{16}{
% \ea \label{ex:kalu:after4}
% \gll Giithu \textbf{thraada} \textbf{kalu}. \\
% like.that \textsc{neg}.exist if \\
% `If this is not available, ... .' (K060103rec01)
% \z
% }


\xbox{16}{
\ea \label{ex:kalu:preverbal1}
\ea
\gll Lorang se=dang mà-iidop thumpath \textbf{kala=kaasi}, \\
      \textsc{2pl} 1s.\textsc{dat} \textsc{inf}-stay place if-give\\
    `If you give me a place to stay,'
\ex
\gll see lorang=nang  lorang=pe samma duvith=le  baarang pada=le anthi-bale-king. \\
      \textsc{1s} \textsc{2pl}=\textsc{dat} \textsc{2pl}=\textsc{poss} all money=\textsc{addit} goods \textsc{pl}=\textsc{addit} \textsc{irr}=turn-\textsc{caus} \\
    `I will return all your money and the goods to you.'  (K070000wrt04)
\z
\z
}

\xbox{16}{
\ea \label{ex:kalu:preverbal2}
\ea
\gll See lorang=nang thama=sakith-kang, \\ % bf
      \textsc{1s} \textsc{2pl}=\textsc{dat} \textsc{neg.nonpast}=pain-\textsc{caus} \\
    `I will do you no harm,'
\ex
\gll lorang see=yang diinging=dering \textbf{kala=aapith}. \\
     \textsc{2pl} \textsc{1s}=\textsc{acc} cold=\textsc{abl} if-protect \\
    `if you give me shelter from the cold.'  (K070000wrt04)
\z
\z
}


\xbox{16}{
 \ea\label{ex:kalu:preverbal3}
   \gll Go=ppe     naama Badulla buulath thaau, lorang Mr.  Mahamud  katha \textbf{kala=biilang}. \\
    \textsc{1s}=\textsc{poss} name Badulla     whole   know  \textsc{2pl} Mr Mahamud \textsc{quot} if=say\\
`Whole Badulla knows my name if you say Mr Mahamud' (B060115nar04)
\z
}


\xbox{16}{
\ea\label{ex:kalu:preverbal4}
\gll {\em Important} {\em occasion} pada \textbf{kala=aada} aapa=ke {\em festival} pada laayeng  {\em wedding} {\em ceremony} \el{}  ithu vakthu kithang arà-kirja \\
     important occasion \textsc{pl} if-exist what=\textsc{simil} festival \textsc{pl} other wedding ceremony { } \textsc{dist} time \textsc{1pl} \textsc{non.past}-make. \\
    `If there is an important occasion, some festival or otherwise a wedding ceremony, we will prepare it.' (K061026rcp01)
\z
}

The utterances  \xref{ex:kalu:preverbal1} and \xref{ex:kalu:preverbal2} also show the use of the irrealis markers \em anthi- \em (affirmative) and \em thama- \em (negative) in the apodosis (i.e. the sentence containing the consequence). Those two utterances also show that the order of protasis and apodosis  is free.

In rare cases, \em kal- \em and another TAM-(quasi-)prefix can be found combined, as in \xref{ex:form:kalu:kalsu1}\xref{ex:form:kalu:kalsu2}. The conditions for this are unclear as of now.

\xbox{16}{
\ea\label{ex:form:kalu:kalsu1}
\gll {\em Wicket}=yang      \textbf{kal}=su-picca,         {\em out}. \\
      wicket=\textsc{acc} if-\textsc{past}-fall-broken out. \\
    `When the wicket has fallen, you are out.' (K051201nar02)
\z
}

% \xbox{16}{
% \ea\label{ex:form:kalu:kalsu:xpl}
% \gll {\em Wicket}=ka \textbf{kal-su-}kìnna, {\em out}. \\
%      wicket=\textsc{loc} if-\textsc{past}-strike out. \\
%     `When the ball has hit the wicket, you are out.' (K081105eli02)
% \z
%. \\

\xbox{16}{
\ea\label{ex:form:kalu:kalsu2}
\gll Lorang=nang duvith kal=su-daapath, kitham=pe banthuan thàrà-kamauvan. \\
    \textsc{2pl}=\textsc{dat} money if-\textsc{past}-get \textsc{1pl}=\textsc{poss} help-\textsc{nmlzr} \textsc{neg}-need. \\
    `When you will have got money, you will not need our help (any more).' (K081103eli04)
\z
}

One context in which the stacking of \em kal- \em with another verbal prefix is possible is the expression of negation  \xref{ex:form:kalu:kalthara}.

\xbox{16}{
\ea\label{ex:form:kalu:kalthara}
\gll Lorang=nang duvith kal=thàrà-daapath, kithang anthi-banthu. \\
     \textsc{2pl}=\textsc{dat} money if-\textsc{neg.past}-get \textsc{1pl} \textsc{irr}-help. \\
    `Should you get no money, we would help.' (K081103eli04)
\z
}

However, this combination of \em kal- \em with another verbal prefix is not always possible. \em Anthi- \em in \xref{ex:form:kalu:anthi} cannot be combined with \em kal-\em; the postverbal form \em kalu \em has to be used instead.

\xbox{16}{
\ea\label{ex:form:kalu:anthi}
\gll Lorang=nang duvith anthi-daapath kalu, kithang svaara thraa=nang anthi-duuduk. \\
     \textsc{2pl}=\textsc{dat} money \textsc{irr}-get if \textsc{1pl} noise \textsc{neg}=\textsc{dat} \textsc{irr}-stay. \\
    `If you get your money, we will be finally left in peace.' (K081103eli04)
\z
}

The different combinatorial possibilities of various TAM prefixes and \em kal-/kalu\em, as well as the resulting semantics are in need of further research.

\em Kalu \em can not only be used with verbs, it can also combine with (proper) nouns \xref{ex:form:kalu:noun1}\xref{ex:form:kalu:noun2}\xref{ex:form:kalu:propernoun} and pronouns \xref{ex:form:kalu:pronoun} or deictics \xref{ex:form:kalu:deictic}.

\xbox{16}{
\ea \label{ex:form:kalu:noun1}
\gll Hatthu samping hatthu oorang bolle=kaasi; \textbf{kumpulang} \textbf{kalu}, thuuju oorang. \\
     one cow one man can=give; party cond seven men. \\
    `One man can sacrifice one cow; when people are in a party, seven men are required.'  (K060112nar01,K081105eli02)
\z
}


\xbox{16}{
\ea\label{ex:form:kalu:noun2}
\gll See, \textbf{pukurjan}=nang \textbf{kalu}, thama-pii  ruuma pukurjan asà-kirja ambel ruuma=ka arà-duuduk. \\
    \textsc{1s} work=\textsc{dat} if \textsc{neg.nonpast}-go house work \textsc{cp}-make take house=\textsc{loc} \textsc{non.past}-stay. \\
    `As for work, I don't go to work, I do housework and stay at home.' (B060115prs04)
\z
}

\xbox{16}{
\ea\label{ex:form:kalu:propernoun}
\gll \textbf{Galle}=ka    \textbf{kalu}, se=ppe    {\em cousin} {\em brothers} pada=samma      see anà-jaalang skuul=nang. \\
      Galle=\textsc{loc} if \textsc{1s=poss} cousin brothers \textsc{pl=comit} see \textsc{past}-walk school=\textsc{dat} \\
    `As for when I was in Galle, I walked to school with my cousins.' (K051201nar02)
\z
}

\xbox{16}{
\ea\label{ex:form:kalu:pronoun}
\gll \textbf{Incayang} \textbf{kalu}, duuva skali asà-dhaathang baaye=nang asà-{\em practice} baaye=nang arà-oomong. \\
      \textsc{3s.polite} if two time \textsc{cp}-come good=\textsc{dat} \textsc{cp}-practice good=\textsc{dat} \textsc{non.past}-speak \\
    `As for him, he has come twice and practiced well and (now) speaks well.' (K061030mix01)
\z
}

\xbox{16}{
\ea\label{ex:form:kalu:deictic}
\gll \textbf{Itthu}    \textbf{kalu}, \el{} Dubai=ka    asà-duuduk     laama kar pada kitham  arà-baapi       Iraq  {\em ports}=dang. \\
 \textsc{dist} if \el{} Dubai-\textsc{loc} \textsc{cp}-stay old car \textsc{pl} \textsc{1pl} \textsc{non.past}-take Iraq ports=\textsc{dat} \\
`In that case, we bring old cars from Dubai to the ports in Iraq.' (K051206nar19)
\z
}

In these contexts, it is not a real conditional meaning that emerges, but rather a reading of `given X, Y is likely/required'. Like this, the translations above can be paraphrased as `Given a party, it will be seven people', `Given work, I do not go there', `Given my staying in Galle, I walked to school'  and `Given that state of affairs, we bring them from Dubai to Iraq.' The historically related Indonesian form \em kalau \em is used in similar contexts \citep[242f]{Ewing2005}.

This use is often found when attributing opinions to people. In example \xref{ex:form:kalu:sedang1}\xref{ex:form:kalu:sedang2}, \em kalu \em refers to the speaker. This use cannot be conditional in the strict sense, since the speaker is not apt to being there or not. She is always there when making an utterance. \em Kalu \em in this case rather translates to something like `as for me.'

\xbox{16}{
\ea\label{ex:form:kalu:sedang1}
\gll \textbf{Se=dang} \textbf{kalu} suda bannyak thàrà-thaau  inni=pe         atthas mà-biilang=nang. \\
     \textsc{1s=dat} if thus much \textsc{neg}-know \textsc{prox=poss} about \textsc{inf}-say=\textsc{dat}. \\
    `So, as for me, I cannot tell you much about this.' (K051205nar04)
\z
}


\xbox{16}{
\ea\label{ex:form:kalu:sedang2}
\gll \textbf{Se=dang} \textbf{kalu} bannyak mà-biilang  thàrà-thaau  sdiikith see  athi-biilang. \\
      1s.\textsc{dat} if much \textsc{inf}-say \textsc{neg}-know few \textsc{1s} \textsc{irr}-say \\
    `As for me, I cannot tell you a lot, I will tell you a little.' (K051205nar02)
\z
}

This use indicating topicality can also be found with preverbal allomorphs of \em kalu\em, as in \xref{ex:form:kalu:topical:preverbal}.



\xbox{16}{
\ea\label{ex:form:kalu:topical:preverbal}
\gll Karang se=ppe  {\em family}=yang \textbf{kal}-ambel,  {\em now} se=ppe naama  Thuan Kabir Samath. \\
      Now \textsc{1s}=\textsc{poss} family=\textsc{acc} if-take now \textsc{1s}=\textsc{poss} name Thuan Kabir Samath \\
    `Now, as far as my family is concerned, my name is Thuan Kabir Samath.' (N060113nar03)
\z
}


Furthermore, \em kalu \em can also be combined with modal particles as in \xref{ex:kalu:modal}.


\xbox{16}{
\ea \label{ex:kalu:modal}
\gll Lorang=nang \textbf{kala}-boole, kithang=nang hathu {\em camera} baa. \\
      2s.\textsc{polite}=\textsc{dat} if-can \textsc{1pl}=\textsc{dat} \textsc{indef} camera bring \\
    `Bring us a camera if you can.'  (Letter 26.06.2007)
\z
}

The border between conditional and temporal meaning of \em kalu \em  is often blurred, as the following example shows:



\xbox{16}{
\ea\label{ex:form:kalu:condtemp:border}
\gll Laskalli \textbf{ka}-dhaathang, mari Badulla=nang. \\
 other.time if-come come Badulla=\textsc{dat} \\
`Come to Badulla next time.' (B060115cvs17)
\z
}



In this example, \em ka- \em could have a temporal reading referring to a point of time in the future when the addressee will return to Badulla, or it could have a conditional meaning making a conditional request which is only realized in the event that the addressee actually does return to Badulla.


The boundary is further weakened by the fact that one allomorph of the  temporal prefix \trs{kapang}{when} is \em kam- \em \formref{sec:morph:kapang-}, which is very similar to \em kal(u)\em, so that this leads to further confusion \citep{Slomanson2008ismil}. In the following example, the speaker refers to a past situation, which cannot be interpreted in terms of conditional truth values, because all events have already occurred. Normally, \em kam- \em would be used in this context, but in this example we find \em kal-\em, a further indication that the border between temporal and conditional is fuzzy. This seems to be true especially of the preverbal use. Temporal use of postverbal \em kalu \em is not attested.


\xbox{16}{
\ea\label{ex:form:kalu:temp}
\gll Sebastian            mlaayu mà-oomong    katha \textbf{kala}=biilang,                  butthul asà-suuka su-jaadi. \\
    Sebastian Malay \textsc{inf}-oomon \textsc{quot} cond-say very \textsc{past}-like \textsc{past}-become. \\
    `When (you) told (us) that Sebastian spoke Malay, we became very happy.'  (B060115cvs01)
\z
}

When combined with present tense predicates, \em kalu \em can have a counterfactual  \xref{ex:form:kalu:pres:counterfac} or a realis  meaning \xref{ex:form:kalu:pres:realis}, when combined with perfect predicates, only the counterfactual reading seems possible \xref{ex:form:kalu:perf:counterfac}.\footnote{\citet[46]{Jayawardena2004} states that in Sinhala, there is no difference between a conditional involving a verb in the perfect tense, and a conditional involving a present tense verb. It is unclear, whether in SLM, we are dealing with a similar situation or not.}

\xbox{16}{
\ea\label{ex:form:kalu:pres:counterfac}
\gll Se=dang saayap kala-aada, bole=thìrbang. \\
     1s.\textsc{dat} wing if-exist can=fly. \\
    `If I had wings I could fly.' (K08114eli01)
\z
}


\xbox{16}{
\ea\label{ex:form:kalu:pres:realis}
\gll Lorang asà-caape \textbf{kala=blaajar}, lorang=nang A/L  bole={\em pass}. \\
      \textsc{2pl} \textsc{cp}-tired if-learn \textsc{2pl}=\textsc{dat} A/L can=pass \\
    `If you study hard, you will be able to pass the A/L exams.' (K08114eli01)
\z
}


\xbox{16}{
\ea\label{ex:form:kalu:perf:counterfac}
\gll Sampi! Luu maalas! A/L thàrà-{\em pass}. Lorang baae=nang asà-caape \textbf{asà}-blaajar \textbf{kala=aada}, mà-{\em pass}=nang su-aada \\
      cow \textsc{2s.familiar} lazy A/L \textsc{neg.past}-pass \textsc{2pl} good=\textsc{dat} \textsc{cp}-tired \textsc{cp}-learn if-exist \textsc{inf}-pass=\textsc{dat} \textsc{past}-exist \\
    `Lazybones! If you had studied harder, you would have passed your A/L exams.' (K08114eli01)
\z
}
% \xbox{16}{
% \ea\label{ex:form:kalu}
% \gll Karang arà-biilang    kalu, bunnar=nang, kitham=pe      inni     {\em British} {\em Rule}=ka           anà=duuduk   mlaayu pada baae  thumpath e-aada. \\
%      now \textsc{non.past}-say  cond correct=\textsc{dat} \textsc{1pl}=\textsc{poss} \textsc{prox} British rule=\textsc{loc} past=stay Malay \textsc{pl} good place \textsc{past}-exist\\
%    `Now, to tell the truth, actually, our Malays that stayed (here) during this British rule had a pride of place.'  (N060113nar01)
% \z
%. \\



% K051220nar01.txt:\tx sepe     aanak pada=le         catholic bini=yang       su-biilang     kaapang=pon   see
% K051220nar01.txt:\tx nii\u n\u ggal  kalu see=yang  islaam=nang   kasi  thriima (katha),



\citet{Jayasuriya2002} treats \em kalu \em as a suffix in her theoretical discussion, but in the examples she gives it is written as an independent word, so that the interpretation as particle seems intended.\footnote{The preverbal and negative forms are not discussed.}

% This form is an inherited one that has not undergone phonological change (besides the change in the final vowel in preverbal position).




\subsection{\trs{kalthra}{if.not}}\label{sec:wc:kalthra}
This particle is the negation of \em kalu \em used on NPs, and is used to construct negative conditionals of hypothesized unavailability \xref{ex:kalthraa:postv} \funcref{sec:func:Conditionals}.  It can only be used after the predicate.


\xbox{16}{
\ea\label{ex:kalthraa:postv}
\gll Lorang=ka duvith (*aada)  \textbf{kal=thra}, kithang anthi-banthu. \\
     \textsc{2pl}=\textsc{loc} money exist \textsc{neg}=if, \textsc{1pl} \textsc{irr}-help. \\
    `If you have no money available, we will help you.' (K081103eli04)
\z
}

As given above in \xref{ex:kalthraa:postv}, \em kalthraa \em can be transparently segmented into \trs{kal=thraa}{`if=neg'}. This is less the case when used as a linker on discourse level, when it is used without a proposition. In this case, it is best glossed by `otherwise'\xref{ex:kalthraa:otherwise1}\xref{ex:kalthraa:otherwise2}. Also cooccurrence with \trs{giithu}{like.that} are common \xref{ex:kalthraa:giithu}. This also has the meaning of \em otherwise\em. Given the lexicalized meaning of \em kalthraa \em in this function, it is treated as an particle in its own right here, instead of an analysis as \trs{kal-thraa}{`if-\textsc{neg}'}.



\xbox{16}{
\ea \label{ex:kalthraa:otherwise1}
\gll \textbf{Kalthraa}, kithang arà-baapi       {\em vehicles}. \\
 otherwise \textsc{1pl} \textsc{non.past}-take vehicles\\
`Otherwise, we take vehicles.' (K051206nar19)
\z
}



\xbox{16}{
\ea\label{ex:kalthraa:otherwise2}
\gll Dee maana aari=le      asà-dhaathang, thìngaari vakthu=nang,   \textbf{kalthraa} maalang vakthu=nang. \\
     3 which day=\textsc{addit} \textsc{cp}-come noon time=\textsc{dat} otherwise night time=\textsc{dat} \\
    `He came every day, at noon or otherwise during the night and (attacked).' (K051206nar02)
\z
}

% \xbox{16}{
%  \ea\label{ex:kalthraa:otherwise3}
%    \gll \textbf{Kalthraa},    kithang=nang   athu   {\em connection} hathu  thraa. \\
%     otherwise \textsc{1pl}=\textsc{dat} \textsc{indef} connection \textsc{indef} \textsc{neg} \\
% `Other than that there is no connection between us' (K051206nar12,K081105eli02)
% \z
% }



\xbox{16}{
 \ea\label{ex:kalthraa:giithu}
   \gll Suda itthu=dering=jo        kithang=nang   ini      Indonesia=pe     oorang=si \textbf{giithu}   \textbf{kalthraa}    {\em Malaysian} oorang=si    katha bàrà=thaau ambel. \\
    thus \textsc{dist}=\textsc{abl}=\textsc{emph} \textsc{1pl}=\textsc{dat}  \textsc{prox} Indonesia=\textsc{poss} man=disj that.way otherwise   Malaysian    man=disj \textsc{quot} can-know take \\
`So with that we can come to know whether someone is Indonesian or otherwise Malaysian' (K060108nar02,K081105eli02)
\z
}

The use of \em kalthra \em standing on its own is reasonably common, while the use in negative conditionals is rare.

\subsection{imperative \em mari\em}\label{sec:wc:mari}
This particle is used for commands \funcref{sec:pragm:Requestingaction}.  When used with a verb, the command is to perform the action denoted by that verb \xref{ex:mari:verb1}\xref{ex:mari:verb2}, when used on its own, the action of coming is implied. The goal of coming can either be preverbal \xref{ex:mari:come1} or postverbal \xref{ex:mari:come2}.


\xbox{16}{
\ea \label{ex:mari:verb1}
\gll Mari kuthumu!  \\
 come see\\
`Look!' (K051220nar02)
\z
}


\xbox{16}{
\ea \label{ex:mari:verb2}
\gll \textbf{Mari} maakang. \\
      come eat \\
    `Eat!'  (B060115rcp02)
\z
}

\xbox{16}{
\ea\label{ex:mari:come1}
\gll Kithang=pe ruuma pada=nang \textbf{mari}-\zero{}. \\
      \textsc{1pl}=\textsc{poss} house \textsc{pl}=\textsc{dat} come.\textsc{imp} {} \\
    `Come to our houses!' (B060115cvs08)
\z
}


\xbox{16}{
\ea \label{ex:mari:come2}
\gll Laskalli ka-dhaathang, \textbf{mari}  \zero{} Badulla=nang. \\
 other.time when-come come { } Badulla=\textsc{dat} \\
`Come to Badulla on your next visit!' (B060115cvs17)
\z
}



Both uses are shown in  \xref{ex:form:mari:double}. The first \em mari \em implies the meaning `come', while the second one indicates imperative mood for the verb \trs{oomong}{talk}.

\xbox{16}{
 \ea\label{ex:form:mari:double}
   \gll Mari     {\em darling},  mari oomong. \\
    come.\textsc{imp} darling     \textsc{imp}  talk \\
`Come darling, come and speak' (K060108nar01)
\z
}

It is possible to combine \em mari \em with the imperative suffix \em -la \em as in the following example.


\xbox{16}{
\ea
\ea
\gll Saayang se=ppe thuan \textbf{mari} laari-\textbf{la}. \\
      love \textsc{1s}=\textsc{poss} sir come.\textsc{imp} run-\textsc{imp} \\
    `Come running my beloved gentleman.'
\ex
\gll See=samma kumpul \textbf{mari} thaa\u ndak-\textbf{la}. \\
     \textsc{1s}=\textsc{comit} gather come.\textsc{imp} dance-imp. \\
    `Come and dance with me.' (N061124sng01)
\z
\z
}


The addressee who is asked to perform the action can be overtly specified, as in the following two examples.

\xbox{16}{
\ea
\gll Derang=nang nya-biilang raaja:  \textbf{lorang} \textbf{mari}. \\
     3=\textsc{dat} \textsc{past}-say king \textsc{2pl} come.\textsc{imp} \\
    `The king said to them ``you come!''.' (K051213nar06)
\z
}


\xbox{16}{
\ea
\ea
\gll Lorang pada baaye pìddang=dering,   baaye=nang pìrrang mà-gijja paa\u nde oorang pada. \\ % bf
       \textsc{2pl} \textsc{pl}  good sword=\textsc{abl} good-\textsc{dat} war \textsc{inf}-make brave man \textsc{pl} \\
    `You are good with the sword, you are brave men who know how to wage war.'
\ex
\gll \textbf{Lorang} \textbf{mari} siini kithang=nang. \\
      \textsc{2pl} come here \textsc{1pl}=\textsc{dat} \\
    `Join us.' (K051213nar06)
\z
\z
}

\section{Conjunctions}\label{sec:wc:Conjunctions}
There are no conjunctions. The role that conjunctions fulfill in other languages is taken over by postpositional clitics that attach to a declausal NP \formref{sec:morph:Postpositions}. Coordination is accomplished by clitics \formref{sec:morph:CoordinatingClitics}.

\section{Classifiers}\label{sec:wc:Classifiers}
There are no classifiers, contrary to many other western Austronesian languages \citep[173]{Himmelmann2005typochar}. The only remnant of historical classifiers might be the expression \trs{anak buva}{child-fruit=children}.
