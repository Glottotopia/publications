\chapter{Introduction}
% thanks Anura Jalayath �kman, Mohammed, Kees, Umberto, Lisa Sebastian Drude, Thomas Malten, Eike, Sascha Ebeling, moin, pual trilsbeek, han sloetjes, Faijzal Mohamed, Haspelmath, Tosco, Elena Lancaster, Frayzingier, Antoine Guillaume, Anne Baker, Olga, Rafael, Sunil, Lalith, Sitha, Stephen, Raj, Michael, Walter, Hans, Peter Slomo, Peter Bakker, Tonjes Veenstra, Ulrike Mosel, David Gil, Felix Rau, Hussainmiya, Ian Smith, Scott Paauw, Geoff Haig, John Petersen, Walter Bisang, Uri Tadmoor, Sander Adelaar, Chris Phonology, Diana, Wolfgang, Niels, Eva

\begin{quote}
The island covers a total area of 65.610 km^{[2]}. It stretches
435 km firm south to north at its longest and 225km a t its widest
from east to west. Located at the extreme southern tip of the
Indian Subcontinent [\dots] the island is within latitudes
6\textdegree to 10\textdegree N. \citet[15]{Hussainmiya1990}
\end{quote}

\draftnote{48.000 SLMs, Smithetal2004, SL census 2001}
there are two main population groups, the Sinhalese and the Tamils
who have been around for a long time, at least 98786456 years for
the Sinhalese \src and 2324354 years for the Tamils \src. The
exact dates vary \src but for the present purpose it suffices to
say that both groups were present for at least a millennium when
the main bulk of Malays arrived.

The Tamils can be subdivided into two groups. First the so called Ceylon Tamils, present on the island since many centuries mainly in the north and the east, and, second, the so called Indian Tamils, plantation workers brought from India during the colonial rule, who are to be found mainly in the hill country in the central region. These two groups have divergent dialects, with the latter one's resembling the speech of Tamil Nadu while the Ceylon Tamils speak a variety of Tamil which conserves many archaic features.

The Sinhalese are mainly Buddhists, whereas most of the Tamils are Hindus. 4545.1545\% of the population are Muslims, these are the so called ``Moors''. They claim descendence from Arab traders or South Indian Muslims. Their language is also Tamil.

Malays are identified as a distinct statistical category by the government\citep[17]{Hussainmiya1990}

\section{History of research}
according to \citet[7,22]{Hussainmiya1990}, little research has
been undertaken on the Malay population in Sri Lanka.
Goonetilake's bibliography on Ceylon %\citep{Goonetilake1970}
 only
cites 7 articles whatsoever that treat Malay issues. These are
mostly short (4-5 pages) and not scientific.

References to Malays in general Sri Lankan history books are limited to a few sentences

\section{Speakers}
\begin{quote}
The Sri Lankan Malay are a small community of 47000 (Ministry of Plan Implementation 1988:13), constituting just 0,32\% of the Sri Lankan population. In English they call themselves Sri Lanka Malays [\dots]. They refer to themselves as o:ra\ng{} Ja:wa or o:ra\ng{} Mala:yu [\dots]. By the Sinhalese, they are called \em Ja minissu\em, by the Tamils \em Java Manusar\em, and by the Moors, \em Malai Karar\em, all three meaning `Malay person'\citep[1]{Bichsel}. 
\end{quote}
also \citep[17]{Hussainmiya1990}.

\begin{quote}
The Sri Lankan Malays are Sunni Muslims of the Shafi sect. Since their arrival in Sir Lanka, they have been in close contact with the Moors, who profess the same religion \citep[1]{Bichsel}.
\end{quote}
\citep[cf.][42]{Hussainmiya1990}

Many of the Malays live at a subsistence level. Unemployment is
29\% and thus higher than for any other group. They often do jobs
as clerks, watchmen, office boys and drivers. Some are small
traders. There are no noteworthy Malay
businesmen\draftnote{Jeweller Galle}Living conditions in Slave
island are bad, many Malays move out of town, to
Gampaha\citep[21]{Hussainmiya1990}

\section{Terminology}
The Speakers refer to themselves as \em Orang Java \em and their language is also called \em Java \em by them. An unqualified designation as \em Java \em could mislead the unwarned reader, because the language is not spoken in the island of Java and there is no close relation to Javanese. \citet{Ansaldo2005kirinda} uses \em Kirinda Java \em which is a good solution when dealing with one locally restricted variety. The academic tradition has Sri Lanka Malay (SLM) since \citet{Smith1979}. This usage is a bit unfortunate because it could imply that the language under discussion is in some way a dialect of Malay in the same right as Kuala Lumpur Malay or Jakarta Malay. This is definitely not the case and the difference between the language under discussion and Standard Malay are far bigger than between, say, Pennsylvania German and Standard German. The distance to Standard Malay is indeed so big that it clearly warrants the classification as a language of its own \citep{Adelaar1991}. To be consistent with academic tradition, I will use SLM throughout this work.

The geographic entity where the language is spoken has had several names in history, of which Ceylon and Sri Lanka are the most widely known. The \em Sri \em was however only added to the name of the republic in 1972, so that a reference to the island in earlier periods by \em Sri Lanka \em is not correct. I will use the local term \em Lanka \em to designate the geographic island in all periods. The political entity will be referred to by the name corresponding to the period, i.e. \em Ceylon \em during colonial times and \em Sri Lanka \em after 19xx.


