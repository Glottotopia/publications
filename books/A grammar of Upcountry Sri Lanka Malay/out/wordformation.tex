\chapter{Word formation}\label{sec:form:WordFormation}
This section describes how one or more morphemes discussed in the previous chapters can be combined to a well-formed morphological word.\footnote{For phonological words, see \formref{sec:phon:struct:Words}.}
Several types can be distinguished: Compounding joins two lexical words, while inflection and derivation join lexical and functional morphemes. Conversion changes the category of a lexeme without morphological expression, while reduplication causes a difference in meaning by using the same morpheme twice.
A well-developed word formation process in SLM is the combination of a deictic with an enclitic, which is discussed in a special section \formref{sec:wofo:Deictic+X}. The reason for the inclusion of a process involving a clitic within word formation is that many of these pairs are highly lexicalized.

\section{Compounding}\label{sec:wofo:Compounding}
Compounding is the creation of a new word by combining two or more stems. Compounding is a morphological operation and must be distinguished from syntactic operations (e.g. apposition of two nouns). This distinction is not always easy to make in SLM. One criterion that can be applied is whether the combination yields one phonological word or more. Another criterion is lexicalization of the combination. As an example, the word \trs{kaca+maatha}{mirror'+`eye'=`spectacles} is an instance of compounding, because a) it is parsed into one phonological word (see \formref{sec:phon:Analysisofwordstructure} for a discussion) and b) it has a lexicalized meaning. Combinations which respond to this criterion are normally left-headed in SLM, while those which do not are right-headed, but exceptions exist. The criteria of phonological word, lexicalization and headedness do not align nicely and can be found in different combinations. Since the criterion of headedness is the easiest to apply, the description of combination of several roots is structured along this criterion. Left-headed combinations of two nouns are treated as morphological, while right headed combinations of two nouns are treated as syntactic in this grammar \formref{sec:nppp:NPscontaininganothernoun}.

While presentation according to the other criteria would also have merits, it was felt that this division would be the most straightforward. This practical consideration does not imply any theoretical claim about the necessary association of headedness, lexicalization and phonological structure.

%  where the noun \trs{thundu}{piece} is used to premodify \trs{bambu}{bamboo} and to postmodify \trs{kaayeng}{cloth}
% 
% \xbox{15}{
% \ea\label{ex:constr:NP:NADJ:double2}
% \ea
% \gll  dovulu=pe oorang pada  itthu anà-kirja [hathu  \textbf{thundu} \textbf{bambu}] asà-ambel apa. \\
%     `The people from former times to make this, they took a piece of bamboo and'
% \ex
% \gll  itthu=yang baaye=nang arà-{\em wrap}-kang  [athu \textbf{kaayeng} \textbf{thundu}]=dering. \\
%     `wrapped it well in a piece of  cloth.' (K061026rcp04)
% \z
% \z
% } \\ \kuckn

\subsection{Compounds involving two nouns}\label{sec:wofo:Compoundsinvolvingtwonouns}

Morphological combinations of two nouns (compounds) are always left-headed as in \xref{ex:wordformation:compound:NN:leftheaded}. In this section, compounding will  be indicated with a plus sign (+).

\cbx{N$_{head}$+N$_{mod}$}{N}
 

\xbox{14}{
\ea\label{ex:wordformation:compound:NN:leftheaded}
\gll Seelong=nang  duppang duppang anadhaathang  mlaayu asadhaathang oorang$\curvearrowleft$+ikkang. \\ % bf
 Ceylon=\textsc{dat} before before \textsc{past}-come Malay \textsc{copula} man$\curvearrowleft$+fish\\
`The Malays who had come a long time ago were fishermen' (K060108nar02)
\z
}



% \xbox{14}{
% \ea\label{ex:theepohong}
% \gll {\em the}$\curvearrowright$ pohong. \\
%  tea tree\\
% `tea tree.' (test)
% \z
% }

The combination in \xref{ex:wordformation:compound:NN:rightheaded} is right-headed, and analyzed as syntactic. This is represented without the plus sign.


\xbox{14}{
\ea\label{ex:wordformation:compound:NN:rightheaded}
\gll  gaaram$\curvearrowright$ aayer asà-thuwang=apa. \\ % bf
      salt water \textsc{cp}-pour=after \\
    `Having poured salt water (on the wound).' (K061125nar01)
\z
} \\

Leftheadedness seems to prevail with highly lexicalized lexemes, like \trs{orang+ikkang}{fisherman}, while rightheadedness is more often found in ad-hoc compounds, like \trs{gaaram aayer}{salt water}.

Like in most languages, compounding in SLM  serves the function to modify and further specify a general term, the head noun.


%blaangan aragannang annajuuwal

A special kind of compound is the exocentric N+N-compound.\footnote{This type is common in South Asia \citep{Abbi2001}kuckn and also found in Sinhala \citep[131]{Karunatillake2004} and Tamil \citep[96]{Arden1934}.} An example for this is \trs{ummabaapa}{mother+father=parents}. The resulting compound is neither a modificiation of the first part, nor one of the second part, but a new term meaning the set of the two components. The fact that we are dealing with a morphological operation here, and not simply with a lexicaled zero-marked coordination \formref{sec:constr:UnmarkedNPcoordination}, can be seen from \xref{ex:form:compound:ummabaapa:hatthu}.


 \xbox{14}{
 \ea\label{ex:form:compound:ummabaapa:hatthu}
   \gll  kithang samma \textbf{hatthu} \textbf{umma}+\textbf{baapa}=pe      aanak pada, kithang samma sudaara pada. \\
`We are all the same parents' children, we are all brothers' (B060115cvs01)
\z
}

In this example, \em ummabaapa \em is combined with the indefiniteness marker \em hatthu\em, which indicates that we are dealing with only one entity here, not with several ones. This meaning is difficult to render in English, because the singular does not combine well with parenthood due to the ontological necessity of two people participating in procreation. An approximate translation could be \em We are all the same ancestry's children. \em Example \xref{ex:form:compound:ummabaapa:hatthu} shows that \em ummabaapa \em can be treated as one referent, which rules out its being a zero-marked coordination. For purposes of comparison, the following example shows a similar sentence, but with other kinship terms which cannot enter the exocentric compound pattern.


\xbox{16}{
\ea\label{ex:constr:compund:ummabaapa:contr:neg}
\gll *kithang samma hatthu kaaka dhaatha=pe aade (pada). \\
      \textsc{1pl} all \textsc{indef} elder.brother elder.sister=\textsc{poss} younger.sibling \\
    `We are all an elder sibling's younger sibling.' (test) 4.11.08
\z
} \\
Instead of the impossible exocentric compound in \xref{ex:constr:compund:ummabaapa:contr:neg}, zero coordination of two PPs must be employed, where every NP is individually marked for possessive

\xbox{14}{
\ea\label{ex:constr:compund:ummabaapa:contr:pos}
\gll kithang samma kaaka pada=pe dhaatha pada=pe aade pada. \\
     \textsc{1pl} all elder.brother \textsc{pl}=\textsc{poss} elder.sister=\textsc{poss} younger.sibling \textsc{pl}  \\
    `We are all the younger siblings of elder brother and elder sisters.' (test)4.11.08
\z
} \\

Next to \em ummabaapa, \em \trs{sudaara+sudaari}{brothers and sisters} is another common exocentric compound.

\xbox{14}{
\ea\label{ex:constr:compund:sudaarasudaari}
\gll se=dang \textbf{sudaara}-\textbf{sudaari} pada thuuju oorang aada. \\
      \textsc{1s=dat} brother-sister \textsc{pl} seven man exist \\
    `I have seven brothers and sisters.' (B060115prs14)
\z
} \\


\subsection{Compounds involving two verbs}\label{sec:wofo:Compoundsinvolvingtwoverbs}
SLM has a small number of V+V compounds, which have to be distinguished from serial verbs involving two full verbs \formref{sec:pred:Verbalpredicateswithtwofullverbs} or a verb and a vector verb \formref{sec:pred:Verbalpredicateswithavectorverb}. Verbal compounds consist of one phonological word, whereas the serial verbs consist of two phonological words. This can be seen from the fact that a verbal compound like \xref{ex:constr:compounds:VV:VV} can only have one long vowel, whereas serial verbs can have two \xref{ex:constr:compounds:VV:SVC}. Section \formref{sec:wc:Multi-verbconstructions} contains a more detailed discussion of multi-verb constructions.

\xbox{14}{
\ea\label{ex:constr:compounds:VV:VV}
\gll ini    kavanan=pe  aanak pompa=nya asà-\textbf{k\underline{a}si-k\underline{aa}ving} \rm(compound). \\
    `They gave this groups girls into marriage.' (K051222nar03)
\z
} \\

\xbox{14}{
\ea\label{ex:constr:compounds:VV:SVC}
\gll   itthu   muusing islam igaama nya-\textbf{\underline{aa}jar}    \textbf{k\underline{aa}si} Jaapna Hindu {\em teacher}. \\
      \textsc{dist} time islam religion \textsc{past}-teach give Jaffna Hindu teacher \\
    `Those who taught us Islam back then were Hindu teachers from Jaffna.' (K051213nar03)
\z
} \\
\trs{kasi-kaaving}{given in marriage} in \xref{ex:constr:compounds:VV:VV} has only one long vowel (in \em k\textbf{aa}ving\em) and forms one phonological word. \trs{Aajar kaasi}{educate} in \xref{ex:constr:compounds:VV:SVC} on the other hand has a long vowel in both verbs, which indicates that we are dealing with two phonological words. In this description, we treat the former case (\em kasikaaving\em) as an operation on the word level, whereas the latter case (\em aajar kaasi\em) is an operation on the phrase level and will be treated in more detail in \formref{sec:pred:Verbalpredicateswithavectorverb}. Note that the difference is reflected in orthography by the intervening space for serial verbs, which is not present for verbal compounds.

Furthermore, the two constructions differ in their internal structure, which can be seen from the position of \trs{ka(a)si}{give}. In the compound, it is initial, whereas in the vector verb construction, it is final.

There are not very many verbal compounds in SLM, and all of those which could be found involve \trs{ka(a)si}{give}.

The following examples illustrate these words.

\xbox{14}{
\ea\label{ex:constr:compounds:VV:kasithaau1}
\gll Badulla {\em Kandy} Matale samma   {\em association}=nang \textbf{masa-kasi-thaau}. \\
      Badulla Kandy Matale samma association=\textsc{dat} must-give-know \\
    `Badulla, Kandy, Matale, we must inform all other associations.' (K060116nar06)
\z
}\\

In \xref{ex:constr:compounds:VV:kasithaau1} the TAM-marker precedes the verbal compound \trs{kasithaau}{inform}, indicating that we are dealing with a one-predicate sentence, and not two.\footnote{If we had two predicates, we would expect the conjunctive participle \em asà-\em \formref{sec:morph:asa-}\formref{sec:cls:Conjunctiveparticipleclause}.}
 The vowel in \em kasi \em is short as usual.



The same predicate, \trs{kasithaau}{inform} is also used in \xref{ex:constr:compounds:VV:kasithaau2}, where it occurs between the agent \trs{lorang pada}{you} and \em aada \em indicating perfect tense.

\xbox{14}{
\ea\label{ex:constr:compounds:VV:kasithaau2}
\gll lorang=pe Amsterdam {\em university}=ka athu   aada  katha kithang=nang   lorang pada \textbf{kasi-thaau} aada. \\
     \textsc{2pl}=\textsc{poss} Amsterdam University=\textsc{loc} \textsc{indef} exist \textsc{quot} \textsc{1pl}=\textsc{dat} \textsc{2pl} \textsc{pl} give-know exist  \\
\z
} \\ 
There is no additional material, indicating again that we are dealing with only one predicate and not several ones. Again, the vowel is short in \em kasi\em, but not in \trs{thaau}{know}, corroborating the one-word-analysis.

The verbal compounds found so far do not carry aspectual or attitudinal information, which distinguishes them from serial verb constructions involving vector verbs \formref{sec:pred:Verbalpredicateswithavectorverb}, where the  purpose of the vector verb is precisely to convey that kind of information. Another difference is that the semantics of verbal compounds is transparent as in the examples above, where `giving knowledge' is equated with `inform' and `giving marriage', with `marry off'. The semantics of the compound is thus transparent and reflects the semantics of its parts.  This distinguishes verbal compounds again from serial verb constructions with vector verbs, where the lexical meaning of the vector verb can be bleached.
As an example of a bleached meaning of a vector verb in a serial verb construction (not a verbal compound), consider \xref{ex:form:compound:VV:SVC:contrast}.

\xbox{14}{
\ea\label{ex:form:compound:VV:SVC:contrast}
\gll incayang=pe kepaala=ka anà-aada thoppi=dering moonyeth pada=nang su-\textbf{buwang} \textbf{puukul}. \\
    `He took the hat from his head and threw it violently at the monkeys.'  (K070000wrt01)
\z    
}\\ 

In this example, the meaning of \trs{puukul}{hit} is bleached and only the violent aspect remains, whereas the simultaneous contact between agent, instrument and patient, which is obligatory for the literal use of \trs{puukul}{hit} is not respected, since we are dealing with throwing, where contact between agent and instrument and instrument and recipient are successive and not simultaneous.

The only instances of verbal compounds found up to date all involve the verb \trs{ka(a)si}{give}. The small size of the set of possible verbs points to a closed class, so that it might be the  case that the constructions with \em kasi-V \em are actually fossilized instances of valency-changing derivations (cf. similar uses in other varieties of Malay \citep{AdelaarEtAl1996, Bakker2006}).


\subsection{Compounds of V+N}\label{sec:wofo:CompoundsofV+N}
These compounds are not found in SLM, with two exceptions, \trs{kappal+thìrbang}{ship'+`fly'=`airplane} and \trs{kappal+thìrbang}{ship'+`swim'=`submarine}. It is very likely that these are actually unanalyzed loans  from Malaysian Malay. Both the left-headedness and the compounding of a noun and a verb make this compound a very unlikely innovation on Lankan soil. Futhermore, it is highly unlikely that the somewhat peculiar semantics of this compound (`ship' for `plane') originated independently in two different places.
It can also be excluded that this word came to Sri Lanka with the immigrants because at that point in time neither airplane nor submarine had been invented yet.


\section{Inflection}\label{sec:wofo:Inflection}
After compounding, inflection is another productive process of word formation.
There are only two word formation constructions in SLM involving inflection, a prefixal one which combines verbal (quasi-)prefixes with a verb \xref{cb:constr:infl:pref} and a suffixal one with the imperatives \em -la \em or \em -de \em attached to a verb \xref{cb:constr:infl:suf}.

\cb[\label{cb:constr:infl:pref}]{$\left\{\begin{array}{l} \rm VPREF-\\ \rm  QUASIPREF\div  \end{array}\right\}$ V}
\cb[\label{cb:constr:infl:suf}]{V$\left\{\begin{array}{l} \rm -la\\ \rm -de \end{array}\right\}$}
Example \xref{ex:infl:pref} shows the use of an inflection prefix, \xref{ex:infl:quasipref} shows the use of an inflectional quasi-prefix, while \xref{ex:infl:suffix} shows the use of the imperative suffix, an inflection as well.
 

\xbox{14}{
\ea\label{ex:infl:pref}
\gll baapa=le       aanak=le      guula \textbf{su}$_{pref}$-maakang$_V$. \\ % bf
     father=\textsc{addit} child=\textsc{addit} sugar \textsc{past}-eat  \\
    `Father and child ate the sugar.' (K070000wrt02)
\z
} \\
\xbox{14}{
\ea\label{ex:infl:quasipref}
\gll kithang caabe \textbf{ara}$\div_{quasipref}$maakang$_V$. \\ % bf
     \textsc{1pl} spicy \textsc{non.past}-eat  \\
    `We eat spicy.'  (B060115rcp01)
\z      
}\\ 

\xbox{16}{
\ea\label{ex:infl:suffix}
\gll diya$_V$-\textbf{la}$_{suf}$ {\em uncle}, se=ppe SSC {\em exam} mà-ambel=nang duwa-pul aari su-aada. \\ % bf
     see-\textsc{imp} uncle \textsc{1s}=\textsc{poss} SSC exam \textsc{inf}-take=\textsc{dat} two-ty day \textsc{past}-exist  \\
    `See, uncle, there are (only) twenty days left until I will have take the SSC exam.'  (K051220nar01)
\z      
}\\ 

\section{Derivation}\label{sec:wofo:Derivation}

Derivation is not very productive in SLM. There are three general patterns of derivational word formation, and three only used for numeral expressions.

\subsection{Nominalization}\label{sec:wofo:Nominalization}

A noun can be formed by attaching the suffix \em -an \em to adjectives or verbs.

\cbx[\label{constr:nominalization}]{ $\left\{\begin{array}{l} \rm V\\\rm ADJ\end{array}\right\}$ \textit{-an}}{N}
An example of this is given in \xref{ex:constr:deriv:an1}, where \trs{oomong}{speak} is not used predicatively, but as a term used in the predication headed by \trs{biilang}{say}, which is why it must be nominalized by \em -an\em, a derivational process yielding \trs{omongan}{speech}.

\xbox{14}{
\ea\label{ex:constr:deriv:an1}
\gll  se=ppe laayeng \textbf{omong-an} samma see anà-biilang. \\
      \textsc{1s}=\textsc{poss} other speak-\textsc{nmlzr} all \textsc{1s} \textsc{past}-say \\
\z
} \\
An example with deadjectival noun is given below.

\xbox{14}{
\ea\label{ex:constr:deriv:an}
\gll [manis-an maakang]=nang go suuka bannyak. \\ % bf
 sweet-\textsc{nmlzr} eat=\textsc{dat} \textsc{1s.familiar} like much\\
\z
}\\

% \xbox{14}{
% \ea\label{ex:constr:unreferenced}
% \gll Derang=pe umma derang=nang jaith\textbf{-an}=le, jaarong pukurjan=le su-aajar. \\
%       \textsc{3pl}=\textsc{poss} mother \textsc{3pl}=\textsc{dat} sew-\textsc{nmlzr}=\textsc{addit}, needle work=\textsc{addit} \textsc{past}-teach \\
% \z      
% }\\  

% Sinhala and Tamil do not make use of suffixes for nominalization. Sinhala uses derivation involving ablaut, the infinitive, adclausal derivation involving \em eka \em or adclausal conversion. Tamil only uses adclausal derivation involvinng \em adu\em.

Nominalization is common, and the suffix \em -an \em is inherited unchanged from former stages of the language \citep{Adelaar1991}.

\subsection{Causativization}\label{sec:wofo:Causativization}
The second general suffixal derivation pattern is causativization by \em king/kang \em \formref{sec:morph:-king}. This pattern is possible with verbal or adjectival hosts.

\cbx[\label{constr:verbalization}]{$\left\{\begin{array}{l} \rm V\\\rm ADJ\end{array}\right\}\left\{\begin{array}{l} -king\\-kang\end{array}\right\}$}{V}
\xref{ex:constr:deriv:king:adj} shows the use of \em -king \em on an adjectical host. 
\xref{ex:constr:deriv:king:verb} shows the use of \em -king \em on a verbal host.

\xbox{14}{
\ea\label{ex:constr:deriv:king:adj}
\gll  itthuka asà-thaaro, itthu=yang arà-\textbf{panas}$_{adj}$-\textbf{king}. \\ % bf
      \textsc{dist}=\textsc{loc} \textsc{cp}-put \textsc{dist}=\textsc{acc} \textsc{non.past}-hot-\textsc{caus} \\
\z      
}\\ 


\xbox{14}{
\ea\label{ex:constr:deriv:king:verb}
\gll baaye meera caaya kapang-jaadi, \textbf{thurung}$_{intr}$\textbf{-king}. \\ % bf
     good red colour when-become, descend-\textsc{caus}  \\
    `When  [the food] has  turned to a nice rose colour, remove (it) [from the fire].'  (K060103rec02)
\z      
}\\

Recursive causativization (`cause someone to cause someone to do something') is generally considered weird. The informants did not really know what to do with it. Solutions like \trs{thiithi-king-kang}{cause someone to cause someone to feed someone} with two causatives were proposed by some but ultimately rejected afters discussion.

%\xbox{14}{
%\ea\label{ex:constr:unreferenced}
%\gll butthul ruuma pada arà-\textbf{picca-kang}. \\
%correct house \textsc{non.past}-broken-cause \\
%`Many houses were demolished.' (nosource)
%\z
%}
%
%\xbox{14}{
%\ea\label{ex:constr:unreferenced}
%\gll itthu    {\em ports}=ka     laama kar asà-baapi        skaarang derang Iraq Government  su-\textbf{binthi-kang}. \\
% \textsc{dist} ports=\textsc{loc} old car \textsc{cp}-bring now \textsc{3pl} Iraq Government \textsc{past}-stand-caus\\
%`Having brought the old cars to those ports, now the Iraq government stops them.' (K051206nar19)
%\z
%}


%
%\xbox{14}{
%\ea\label{ex:constr:unreferenced}
%\gll samma se-kaving-king. \\
% all \textsc{past}-marry-caus\\
%`.' (B060115nar04.35)
%\z
%}
%
%\xbox{14}{
%\ea\label{ex:constr:unreferenced}
%\gll samma oorang massa {\em married}-king. \\
% all man must marrie-caus\\
%`.' (B06011nar04.34)
%\z
%}

\subsection{Involitive derivation}\label{sec:wofo:Involitivederivation}
The only derivational prefix is \em kana-\em, used to derive an involitive verb.

\cbx{kana-V}{V}

\em Kana- \em is different from inflectional (quasi-)prefixes and does not compete for the same slot, as shown in \xref{ex:constr:deriv:kana:verb}, where we find both a quasi-prefix (\em arà- \em) and \em kana-\em.

\xbox{14}{
\ea\label{ex:constr:deriv:kana:verb}
\gll se=dang naasi arà-kana-maakang. \\
     \textsc{1s=dat} rice \textsc{non.past}-\textsc{invol}-eat  \\
    `I compulsively ate (all) the rice.' (nosource)
\z
} \\


\subsection{Derivation of numerals}\label{sec:wofo:Derivationofnumerals}
Numeral derivation comprises \em -blas \em to derive the numbers from 13 to 19, \em -pulu \em to derive  multiples of 10 from 20 to 90, and \em ka- \em to derive ordinals.

\cbx{(ka-) NUM $\left\{\begin{array}{l} \rm -pulu ~(NUM)\\\rm -blas\end{array}\right\}$?}{NUM}



\section{Conversion}\label{sec:wofo:Conversion}
When a lexeme can be used in syntactic positions prototypically associated with different word classes, the question arises whether this is to be treated as double class-membership or as conversion (zero-derivation). `Conversion' as a term is only used when the process is fully generalized across a word class \citep{abc}. Double class-membership is reserved for lexemes that accidentally can be employed in other functions, while most of the members of the class they are in cannot. Under this definition, all adjectives in SLM can convert to nouns or to verbs, but no other word class can convert. Isolated elements of other word classes can have double class membership, though, for instance \trs{jaalang}{walk(V)} and \trs{jaalang}{street}.

Conversion on a lexical level has to be distinguished from referential use of entire clauses at phrase level. It is actually possible to find verbs in referential position, prototypically associated with nouns. While this is indeed an instance of conversion, it is not on the word level, but on the phrase level. In those cases, the conversion is not 
 V+\zero$\to$N, but S+\zero$\to$NP, S consisting only of a verb. See \formref{sec:nppp:Nounphrasesbasedonaclause} for discussion.
 
\subsection{Adjectives to nouns or verbs}\label{sec:wofo:Adjectivestonounsorverbs}
All members of the adjective class can convert to verbs or nouns. The converted adjectives can then be used in all the constructions where other verbs/nouns could be used.

\cbx{ADJ-\zero}{V}
\cbx{ADJ-\zero}{N}

The following examples show the use of the adjective \trs{bìssar}{big}{} as an adjective \xref{ex:constr:conversion:ADJADJ}, a noun \xref{ex:constr:conversion:ADJN} and a verb \xref{ex:constr:conversion:ADJV}.


\xbox{14}{
\ea\label{ex:constr:conversion:ADJADJ}
\gll   se=ppe      \textbf{bìssar} lai   ruuma aada. \\
       \textsc{1s}=\textsc{poss}     big    other  house exist\\
    `There is another big house of mine.'  (B060115cvs09.7)
\z      
}\\ 


\xbox{16}{
\ea\label{ex:constr:conversion:ADJN}
\gll incayang=pe      {\em wife}=le       {\em wife}=pe baapa=le       masigith=pe \textbf{bìssar}$_{N}$     mas-panggel. \\
3s.polite=\textsc{poss}   w.=\textsc{addit} w=\textsc{poss}  father=\textsc{addit} mosque=\textsc{poss} chief must=call \\
    `His wife and her father had to call the mosque's head.'  (K051220nar01.56)
\z      
}\\ 


\xbox{14}{
\ea\label{ex:constr:conversion:ADJV} 
 \gll  aanak pada asà-\textbf{bìssar}  skuul=nang anà-pii. \\
	      child \textsc{pl} \textsc{cp}-big school=\textsc{dat} \textsc{past}-go \\
\z      
}\\ 


%\xbox{14}{
%\ea\label{ex:constr:unreferenced}
%\gll incayang=pe {\em wife}=le      baae  {\em Malay}      {\em costumes}=n\E{} baae  thaaro. \\
% \textsc{3s.polite}=\textsc{poss} wife=\textsc{addit} good Malay costumes=\textsc{dat} good put\\
%`.' (K060116nar03)
%\z
%}

\section{Reduplication}\label{sec:wofo:Reduplication}
There are three productive processes of reduplication in SLM to form new words:

\begin{enumerate}
 \item reduplication of quantifiers to get an intensive reading
 \item reduplication of adjectives to get a prototypical/focal reading
 \item reduplication of a verb to form the simultaneous conjunctive participle
\end{enumerate}

Other word classes, like  nouns, cannot be reduplicated, but see below for fossilized nominal reduplications.

%``\el plurality in Sri Lankan Malay \el is, however, not indicated by reduplication byt by a plural marker `\em pada\em'  Adelaar (1991:32) states that [SLM] has borrowed this feature from Jakartanese where `\em pada\em' precedes the predicate and indiactes plurality of the subject. Jakartanese, however, has borrowed this feature from Javanese.''\citet[]{Jayasuriya2001}

\subsection{Quantifier reduplication}\label{sec:wofo:Quantifierreduplication}
Quantifier reduplication is reasonably common.
The quantifiers \trs{konnyom}{few}{} and \trs{bannyak}{many}{} are often reduplicated to intensify the meaning.

\cb{ $QUANT\sim QUANT$}

\xbox{14}{
\ea\label{ex:redupl:quant}
\gll deram  pada arà-duuduk    \textbf{konnyom}\~{}\textbf{konnyom} kiccil kiccil kawanang=ka. \\
 \textsc{3pl} \textsc{pl} \textsc{non.past}-stay little\~{}\textsc{red} small small family=loc\\
\z
}

\subsection{Adjectival reduplication}\label{sec:wofo:Adjectivalreduplication}
Adjectival reduplication is not found often. When adjectives are reduplicated, they yield a focal meaning.


\cb{$ADJ\sim ADJ$}

\trs{kiccil}{small}{} and \trs{pullam}{slow}{} are found reduplicated, but \trs{bìssar}{big}{} and \trs{kìrras}{fast}{} are not. For an example, see \xref{ex:redupl:quant} above.

\subsection{Reduplicated roots}\label{sec:wofo:Reduplicatedroots}
There are a number of words which are clearly the result of historical reduplication but whose elements do not occur any more on their own. Examples for this are

\trs{lavalaava}{cobweb},
\trs{cumicuumi}{squid},
\trs{thangathaanga}{ladder},
\trs{bathabaatha}{bricks}.

A somewhat different case is \trs{barambaarang}{furniture}, where the non-reduplicated form \em baarang \em does exist, but has a less specific meaning, `item, good'. 


\subsection{Verbal reduplication}\label{sec:wofo:Verbalreduplication}
Verbal reduplication  is an operation to derive the simultaneous conjunctive participle.

\cb{ $V\sim V$}

An example is given in \xref{ex:redupl:cp}, where the flight of the rabbit happens in a jumping manner. This verbal modification of the verb \trs{pii}{go} has to be in the simultaneous conjunctive participle, which is formed by reduplicating the verb \trs{lompath}{jump}.

\xbox{14}{
\ea\label{ex:redupl:cp}
\gll kancil \textbf{lompath}\~{}\textbf{lompath} arà-pii. \\
     rabbit jump\~{}\textsc{red}      \textsc{non.past}-run \\
    `The rabbit runs away jumping.'  (test)4.11.08
\z      
}\\ 

A similar case is found in \xref{ex:redupl:cp:laugh}, where the laughing of the Andare happens in a bursting manner, and \trs{lompath}{jump} is reduplicated to indicate this modification of the predicate \trs{thathaawa}{laugh} by the verb \trs{lompath}{jump}. Note that the English translation suggests an aspectual value of suddenness which is not there in the SLM example.

\xbox{14}{
\ea\label{ex:redupl:cp:laugh}
\gll  Siini    Andare=atthas \textbf{lompath}\~{}\textbf{lompath} arà-thathaawa. \\
      here Andare=about jump\~{}\textsc{red} \textsc{non.past}-laugh \\
\z      
}\\

A final example of a predicate (in this case \trs{maakang}{eat}) being modified by another predicate (\trs{biilang}{say}) is \xref{ex:redupl:cp:say}, where the reduplicated modifier also has a quite elaborate complement. 

\xbox{14}{
\ea\label{ex:redupl:cp:say}
\gll [lu=ppe muuluth=ka=le paasir, se=ppe muuluth=ka=le paasir  katha] \textbf{biilang}\~{}\textbf{biilang} baaye=nang baapa=le aanak=le guula su-maakang. \\
      \textsc{2s.familiar}=\textsc{poss} mouth=\textsc{loc}=\textsc{addit} sand \textsc{1s}=\textsc{poss} mouth=\textsc{loc}=\textsc{addit} sand \textsc{quot} say\~{}\textsc{red} good=\textsc{dat} father=\textsc{addit} child=\textsc{addit} sugar \textsc{past}-eat  \\
    `Saying ``There is sand in your mouth and there is sand in my mouth'' both father and son ate the sugar.'  (K070000wrt02)
\z      
}\\


This construction has to be distinguished from  constructions with \trs{kapang-}{when}, \trs{=kapang}{when} and \trs{watthu}{time, while}. These forms introduce a temporal adjunct, while the reduplicated construction is not an adjunct, but a modifier of the verbal predicate. The meaning of \xref{ex:redupl:cp:say} is thus not `While saying ..., they ate' but rather `They ate in a talking manner'.


%kandi=ka=jo kiccilccil Hire  Short Hire samma  lalalaari mas-duuduk

% Verbal reduplication is not only possible with pure verbs, but also with converted adjectives, as \xref{ex:redupl:convadj} shows. However, it is not clear from the example whether the adjective is used in a stative sense here (\em quickly\em) or in a dynamic sense, which would be typical of an adjective converted to a verb (\em becoming quicker and quicker\em).
% 
% \xbox{14}{
% \ea\label{ex:redupl:convadj}
% \gll Itthu=nang blaakang inni oorang \textbf{likkas}\~{}\textbf{likkas} thoppi pada=yang asà-kumpul ambel sithu=ka=dering su-pii. \\
% \textsc{dist} after \textsc{prox} man fast\~{}\textsc{red} hat \textsc{pl}=\textsc{acc} \textsc{cp}-collect take there=\textsc{loc}=\textsc{abl} \textsc{past}-go\\
% \z
% }\\ 

\section{Deictic+X}\label{sec:wofo:Deictic+X}
Due to the absence of conjunctions and the scarcity of adverbs, SLM has to resort to other means to fulfill the linking functions taken care of by these elements in other languages. Very often, this involves a combination of the distal deictic \em itthu \em with one other element, normally an enclitic. These combinations are often lexicalized and not necessarily analyzed by the speakers. This is the reason why they are treated with  word formation processes, rather than on the phrase level, where they historically originated.

\cbx{ \textit{itthu} $\left\{\begin{array}{l} \rm=POSTP\\\rm=COORDINATING CLT \\\end{array}\right\}$}{ADV}
Of these combinations, the most salient are \trs{itthunam}{therefore}, \trs{itthusubbath}{because of that}, \trs{itthukapang}{then} and  \trs{itthule}{but}.
 
\subsection{itthunam}\label{sec:wofo:itthunam}

The combination of \em itthu \em with the dative marker \em =nang \em is used in discourse to mark that the following event was a consequence of the preceding one. The use of the dative in this construction might have to do with the fact that \em =nang \em is generally used to indicate purpose \funcref{sec:func:Purpose} and consequence, and that this use has generalized to this discourse marker.
 
\xbox{16}{
\ea\label{ex:constr:unreferenced:itthunam1}
\gll \textbf{itthunam},       samma baae  oorang pada inni     nigiri=ka=jo        deram  sa-duuduk. \\
therefore all good man \textsc{pl} \textsc{prox} country=\textsc{loc}=\textsc{foc} \textsc{3pl} \textsc{past}-stay \\
\z
}

\xbox{14}{
\ea\label{ex:constr:unreferenced:itthunam2}
\gll \textbf{itthunam},      mlaayu pada=pe     {\em tradition}=ke      aada. \\
 therefore Malay \textsc{pl}=\textsc{poss} tradition=\textsc{simil} exist\\
`That's why there are the Malays' traditions.' (N060113nar01)
\z
}

\subsection{itthusubbath}\label{sec:wofo:itthusubbath}
The combination of \em itthu \em with the causal postposition \em =subbath \em is used to indicate a reason, similar to English \em therefore\em. The causal component is stronger in \em itthusubbath \em than in \em itthunam\em.

\ea
\gll   \textbf{Itthusubbath}=jo incayang=yang    siithu anabraanak. \\
  therefore=\textsc{foc} \textsc{3s.polite}=\textsc{acc} there \textsc{past}-be.born   \\
\z


\subsection{itthuka(apa)ng}\label{sec:wofo:itthuka(apa)ng}

The combination of \em itthu \em and \trs{kaapang}{when}{} is frequently used to advance to a following event in discourse. In the recipe in \xref{ex:constr:itthukapang:recipe}, the list of ingredients is structured by means of \em itthukang\em.
 
\xbox{14}{
\ea\label{ex:constr:itthukapang:recipe}
 \ea
\gll duuwa raathus lima-pulu    {\em gram} buula     maau. \\ % bf
 two hundred five-ty gram flour want \\
`You need 250g flour.' 
\ex
\gll \textbf{itthukang}  nni      santhang     maau. \\
	then \textsc{prox} coconut.milk want\\
 `And then you need this coconut milk.' (B060115rcp02.06)
 \z
\z
}
 
\subsection{itthule}\label{sec:wofo:itthule}

The combination of \em itthu \em with the additive clitic \em =le \em is used to cancel implications \funcref{sec:disc:Cancelingimplicatures}, like English \em but \em, or \em this being the case, ... nevertheless ...\em.\footnote{The same combination of distal deictic and additive clitic is also used in Sinhala \citep[55]{Jayawardena2004}.} Example \xref{ex:constr:itthule} shows the use of this marker, where its occurrence in the second clause cancels the implicature that the ransom referred to in the first clause would cause the speaker to be returned in its original state. Instead, and contrary to expectations,  he is turned into a bear.
 
\xbox{14}{
\ea\label{ex:constr:itthule}
\ea
\gll se=ppe baapa incayang=nang ummas su-kaasi. \\ % bf
     \textsc{1s}=\textsc{poss} father 3w.polite-\textsc{dat} gold \textsc{past}-give \\
    `My father gave him the gold.'   
\ex
\gll \textbf{Itthule}, [see=yang mà-kiiring=nang duppang] incayang see=yang hathu Buruan mà-jaadi su-bale-king. \\
     But \textsc{1s}=\textsc{acc} \textsc{inf}-send=\textsc{dat} before \textsc{3s.polite} \textsc{1s}=\textsc{acc} \textsc{indef} bear \textsc{inf}-become \textsc{past}-turn-cause  \\
\z
\z      
}\\

This use must not be confounded with the use of \em =le \em as an additive clitic meaning \em also, too, as well \em on a deictic.


\xbox{14}{
\ea\label{ex:constr:itthule:contrast}
\gll \textbf{itthu=le} oorang mlaayu=pe baaye hathu {\em traditional} {\em food} hatthu. \\
      \textsc{dist}=\textsc{addit} man Malay=\textsc{poss} good \textsc{indef} tradtional food \textsc{indef} \\
\z
} \\
In example \xref{ex:constr:itthule:contrast}, we are not dealing with the adversative use of \em itthule \em but rather with a transparent combination of \em itthu\em, refering to a dish mentioned earlier, and \em =le\em, which indicates that the mentioned dish is tasty just like the other dishes mentioned before.
