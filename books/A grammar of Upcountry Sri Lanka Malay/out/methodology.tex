\chapter{Methodology}
\section{Theroretical assumptions}
The presentation in this grammar is intended to be as theory-neutral as possible. No knowledge of any particular theoretical framework is assumed in order to keep the book accessible for a maximum number of people, scholars and laymen alike. Theory-neutralness is also advisable with regard to future generations of linguists who might not be any more aware of the major linguistic trends in the beginning of the 21st century than we are aware of the formalisms of tagmemics that saw their heydays half a century ago.

Being theory-neutral does not mean that this grammar is atheoretical. There are many decisions in the presentation of the data that were guided by theoretical considerations. Structuring data is impossible without a solid theory that lays the fundament on which the grammatical findings can be grounded. But in a house, it is not necessary to see plans of the fundament in every room, and in a grammar book it is not necessary to cast everything into a powerful yet complicated formalism when it can also be said in simple, maximally theory-neutral words. The material for grounding this grammar on is mainly taken from Radical Construction Grammar \citep{Croft2001rcg} and Functional Discourse Grammar \citep{HengeveldEtAl2008fdg}. However, the reader will rarely notice this anymore than he notices the fundament of a house he happens to enter. Only on closer inspection of the architecture will he find signs of the fundament, e.g. in the specific disposition of pillars or chapters. Care has been taken to avoid theory-specific jargon and formalisms as much as possible. When formalisms are used, they are explained in detail. Normally, standard descriptive nomenclature has been used. Every nomenclature in itself relies on theoretical considerations, the nomenclature here is based on Basic Linguistic Theory \citet{Dryer2006blt, Dixon1997riseandfall}, which should be accessible to every reader with a basic knowledge of typology.

\section{Data collection}
The field work for this grammar was done in X field trips between november 2005 and 200X, of about 2 months each. During the trips, the focus was on data gathering, while the periods between them where devoted to analysis. The findings were cross-checked during the following field-trip. Focus was on natural discourse, monological as well as narration. In the first field trip, I did quite a lot of elicitation to get a feeling for the possible constructions in the languages so I could identify them when I heard/transcribed them. The phonological system is not very complicated and was already analyzed in the great lines \citep{Bichsel}, so I encountered little problems in deciding how to segment words or how to represent them orthographically. With a good preliminary orthography at hand, I was able to get into morphosyntax quite quickly. SLM has no complicated allophonic rules, and the set of allomorphs is also very restraint. This made it quite easy to segment morphemes and to identify the same morpheme in different contexts. From the first day on, I spoke as much Malay as possible\footnote{I have no command whatsoever of Standard Indonesian or Standard Malaysian. When I am talking of Malay in the following, Sri Lankan Malay is always intended.}. This was of course quite limited in the beginning, but my consultants got used to talking to me in Malay from the very beginnining. In this way, elicitation session were only a little different from normal interaction, and I hope this has taken away some of their artificial character.

The Malays all have a good command of English, so English was also used in conversation, not only for difficult topics, but also when the speakers just felt like it. In order to maintain a relaxed and casual atmosphere, I did not interfere with that. Maybe some day in the future a researcher wants to do research on code-switching and will be happy to have this material as is.

The examples in this book are mainly taken from natural discourse. For every example, its source is indicated. The source file follows certain conventions that permit to identify the text genre. The first letter indicates the town, the following number indicate the date (YYMMDD). The three following letters indicate the genre, for instance nar for narrative or eli for elicitation (see Table \ref{tab:genres} for a list). The transcriptions as well as the audio and possibly video files are available on the internet. The primacy of natural discourse over elicitation means that there are not many example that stem from an eli-file. These are mainly for negative judgments, which never occur in normal discourse, and marginal phenomena that could not be found in natural discourse but are generally considered of linguistic interest. A third reason to use elicited examples was the desire to keep the context constant, e.g. in the elicitation of paradigms.

\begin{table}
\begin{center}
% use packages: array,tabularx
\begin{tabular}{ll}
nar & narrative \\ 
cvs & conversation \\ 
sng & song \\ 
rcp & recipe \\ 
wrt & written text \\ 
prs & presentation \\ 
eml & email \\ 
\end{tabular}
\caption{Genre codes used in text examples}
\label{tab:genres}
\end{center}

\end{table}

\section{Data gathering techniques}
\subsection{Elicitation}
Elicitation. Let me be frank: I do not value elicitation very highly. The elicitor's wits, charm, subterfuge and creativity (or lack thereof) in making up contexts  determine a big deal more the grammaticality judgment of a given sentence than the consultant's intuition. Consultants like any other person you meet want the meeting to be enjoyable and one way of making it enjoyable is giving the linguist the things he wants to hear, or at least not refusing him the things he longs for that much. 
The linguist on the other hand might suppose that a certain construction would not exist anyway and not bother to check every context. After all there are many other things to be done that seem more promising, and time is limited.
This being said, a linguist is not interested in random sentences as they occur, but wants to investigate a certain topic and thus has to stimulate the informant to produce sentences that are of interest for the particular topic. Unfortunately, humans' intuition about their linguistic behaviour is mediocre \citep{Labov1975fact, datengwm}, so direct questions are best avoided.
One thing that surely triggers politeness and accomodation is the famous sentence \em Can you say X? \em I do not even talk about its variations like \em And you can't say Y, can you? \em and \em So you can also say Y? \em Of course people \em can \em say a lot of things. In the absence of schooling under prescriptive teachers, they can even say a lot more. There is a good lot of variation in many oral languages, and SLM is no exception. As such, speakers are used to a lot of variation in their input and will only reject a sentence if it seriously hinders comprehension. This grammar has the aim to provide an account of how Sri Lanka Malay is used. It is not an account of what people can be man\oe uvered into saying. Direct questions were thus only used as a technique of second language acquisition on the part of the researcher. If ever a grammatical phenomenon is illustrated by examples with \em eli \em in their source, caution should be exercised. It is very probable that the phenomenon under discussion is marginal and was only included for sake of completeness.

But how could specific grammatical phenomena be tested without direct questions? As a general principle, the methods involving less target language production by the linguist were preferred over others. The normal way of prompting informants to produce a certain sentence was to establish a certain context that would logically lead to an utterance with the desired meaning. When checking for specificational/identificational constructions for instance, the following context was presented: 

\ea Suppose you come home and you see that someone has broken in and stolen the TV, some money and so on. You do not know who that was. But later you learn that Farook was the thief. How would you then tell a friend that Farook was the thief? \z

Normally consultants would not just give the last sentence, but they would actually repeat the story with the final climax that Farook was found to be the thief. This sentence is \em Farook-jo maaling \em, with a focus clitic indicating that it is not a normal ascription.
Given that there was a whole story around, the consultants would not really know what I was looking at and could not really second-guess which information I was keen on and accomodate my desires.

If I suspected that there were other possibilities (I normally did), I asked precisely that.

\ea  Could you also say something else?  \z 

Note that this is already a greater commitment by the linguist because it suggests that there \em is \em something else, and consultants might try to be polite. If the construction I was looking for still did not show up, I would try to modify the sentence without revealing what my preferred structure would be. Something like 

\ea \ea Can you start with another word? \\
\\
 or \\
 
\ex Can you leave out something, and does that change the meaning? \z\z
 
come to mind. Still no target language material by the linguist. One step further is 

\ea Could you also start with X?\z

This involves production by the linguist, but it is very limited and the consultant still has to utter the sentence, instead of just nodding. Consultants easily accept doubtful sentences, but they rarely utter them themselves. The last level would be 

\ea And if I say X Y Z, how does that sound?\z

This last sentence is already pushing the linguist's production quite far and is susceptible to be accepted as correct even if no native speaker would ever utter it.
The second part of the sentence is crucial because the consultant cannot just say \em yes\em. She has to give a qualitative interpretation between something like \em very good \em and \em You can't say that. That does not mean anything! \em If the sentence X Y Z is accepted only at this late stage, it is mandatory to do two things. First, the consultant has to be asked to repeat the sentence. Many people can say that incomplete sentences are fine, for instance, but the dropped material will invariably show up when they are asked to repeat. The second thing is to ask which one of the sentences is `better', the one that the consultant gave first, or the one she finally accepted after direct prompt. The answer \em Yours is better \em clearly indicates politeness effects. Fortunately, I did not come across this. The answer \em both same \em is OK and indicates that there seems to be no great difference in acceptability between the two sentences. Still, it has to be noted down which sentence came first. The answer \em The second one is a bit odd, no? \em or \em You can say that in fast speech \em indicates that the second sentence would never be said in that way by the consultant and is thus ungrammatical.
A good technique is also starting the sentence in a doubtful tone and leaving it suspended in the middle, like looking for a word. Often, the consultant will join in then and finish the sentence.

One thing that is difficult with proceeding in this way is that it might lead to parrot responses at a certain point. With parrot responses I mean that the structure of the priming language is completely taken over in the target language. But prompting in the target language is worse as far as influence from the priming language is concerned because the linguist is very likely to impose certain structures of his language on the language he tries to describe. Be this as it may, consultants that would remarkably often copy the English order of constituents were asked some simple sentences in English and some comparable sentences in Sinhala, where the order of constituents is different from English. When their responses were like in English in the first case, but like in Sinhala in the second, this was ascribed to the parrot effect and caution was taken in interpreting these consultants' data. When the order of constituents remained consistent under different priming material, it was concluded that the given sentences were authentic.\footnote{
One remarkable feat in this domain is NH, who is not only able to repeat the same sentence with the same constituents a random number of times. Other people tend to replace NP by pronouns for instance, or offer alternatives. She is furthermore not influenced at all by the structure of the priming languages and will deliver very complex postverbal English subordinates invariably in preverbal position. All other informants would leave them as they are as they go along the sentence constituent for constituent.}

\subsection{Conversation and narratives}
While I tried not to speak SLM during elicitation sessions in order to not distort the data, I tried to avoid speaking English in conversation and when collecting narratives. People tend to give answers in the languages they are asked in, so asking in English would give stories in English, which was not desired. 

The inhabitants of Lanka are generally very sociable and love to chat. The Malays are no exception to this. They integrate foreigners very well, and so there is some linguistic production of mine to be found in the cvs-files. As far as I can see, this has not altered the speak of the Malays in general. I do not think that they accomodated their speech beyond reducing speed and articulating better (and even that they did not always do). I thus feel that examples from these conversations can be used to illustrate language use as well as any other file. In fact, the analysis of intonation is based on a conversation between several Malays and me where they inquire about my place of living etc. 

Sometimes my consultants would become involved in a conversation and forget about my presence. In that case, I often withdrew from the conversation and let them speak among themselves. These data are about the most naturalistic as one can get. Unfortunately, this is not only true for morphosyntax, intonation, discourse structure etc., but also for what is generally known as common ground. Given that I did not participate in the conversations, no need was felt to explain concepts and relations I was not aware of, and I must say that very often I had not the slightest clue what a conversation was about. This is partly due to specialized vocabulary, fast speech and so on, but also to the fact that I did not know the people, places and events that were interesting. Add this ignorance of the common ground to the general South Asian tendency to drop known material and you get a completely lost linguist.

Songs were also recorded, but the examples from there were not used to make points because songs often allow special constructions.
%  It might be interesting to do analyses of syllable structure and stress using songs \citep[cf.][]{Birch} but this has not been done yet.

\subsection{Translation}
The ``elicitation'' sessions were mostly spontaneous. I had noted down some phenomena I was interested  in and would improvise contexts for variations as we went along. However, sometimes I would use prefabricated English or Sinhala sentences to systematically check some phenomenon. 
Normally, I wrote up sentences for three or four different topics which I asked in random order. This was done to prevent entrenchment of the constructions. When people are asked the very same construction twenty times in a row with only marginal variation, they tend to rate the 21st construction like the others. Mixing in some completely different constructions can prevent this feeling of ``assembly chain elicitation''.

I also used examples from a learner's grammar of Spoken Sinhala\footnote{There are no textbooks for Spoken Sri Lanka Tamil.}\citep{Karunatillake2004}. This was done to see whether some categories that are grammaticalized in Sinhala are grammaticalized in SLM, too. For instance, I had the suspicion that SLM would have a grammaticalized evidential marker because this is an areal feature. However, asking to translate English sentences with some evidential reading like \em It is said/it seems that X \em would always yield \em arabiilang ...\em which is the lexical word ``to say''. It was impossible to construct a context in English where speakers were led to mark evidentiality. Using Sinhala primes proved an instant success, though. The Sinhala evidential clitic \em =lu \em in some sentence was translated as the particle \em kiyang\em. \em kiyang \em is the evidential marker that was impossible to elicit through English. Given this success, I copied all sentences from the learner's grammar and read them to the Malays, a source of big fun. From their translations, I could see whether there was some more grammaticalized material in SLM that was very difficult to elicit through English. Of course, this technique cannot be used as evidence for semantics or word order because of interference effects. Where it can be used succesfully is to establish the existence of spurious grammatical morphemes.


\subsection{Word lists}
Word collection was largely done in a spontaneous and associative way. Swadesh lists and extended Swadesh lists for South Asia taken from \citet{Abbi2001} were also used. Abbi also provides questionnaires for some typical areal phenomena, but it was found that they were too schematic and would not yield good responses.

Every now and then, I checked parts of my dictionary with consultants to see whether I had correctly written down what they had told me. This was especially necessary for the different coronal stops. I found it very hard to hear the difference between dental and alveolar stops (the latter are allophones of retroflex stops), so cross checking was necessary. The Sinhala alphabet has different graphemes for each one of these stops, so I could ask the consultants to write down the Malay word in Sinhala script and see which grapheme they would use\footnote{Sri Lanka has a literacy rate of 95\%, and the Malays are among the best educated ethnic groups so this technique is culturally adequate. Also, the Sinhala alphabet is phonemic and the pronounciation of a given grapheme is always predictable. My informants happened to be more familiar with the Sinhala script, but the Tamil script could also have been used.} 
This method relies on writing because when asked which letter they would write, they would either say \phonet{\dentt aj@n@} or \phonet{\tz aj@n@}, the Sinhala names of the letters. This leaves the initial problem of whether the stop was dental or alveolar. But \phonet{\dentt aj@n@} is represented in Latin script by \graphem{th}, so I found out that I could ask the informants whether a certain word should be spelled with \graphem{t} or \graphem{th}. The answer would indicate whether the stop was dental (th) or alveolar (t).
Other contexts where this asking to write down a word was useful was vowel quantity, the presence or absence of a labial approximant before [u] and the distinction between prenasalized consonants and combinations of nasal+homorganic stop.

\section{Recording}
\subsection{Primary recording}
Most sessions were recorded on audio with permission of the consultants. Sensitive material was deleted on request or not recorded. Recording devices are a bit obtrusive, and speakers were a bit uneasy at first. I have the impression that after some time the consultants got used to the recording device and continued talking like normal.

Some sessions were also recorded on video with a very small camcorder. This is more obtrusive than audio recording, and the data are less naturalistic. Some people were perfectly fine with being taped while others seemed to be a bit uneasy. In the latter case, the video camera was not used much. I found that using a tripod improves the technical quality, but decreases the linguistic quality of the data because people tend to fix the camera. This is why I did not use it much. I would normally sit with the camera on my lap, the LCD display turned 45^{o} so I could see it. This permitted to keep eye contact with the consultant and to use mimics and gesture to signal that I was keeping track of the story. At the same time, I could occasionally shed a glance to the LCD to see how the camera was doing, whether the tape was full or the battery empty etc.

All audio data were recorded as linear pcm at 44,1kHz on flash memory cards using a Mayah Flashman ZXCVBNM and a SONY QWER microphone. Video was taken with a SONY ASDFG and a gun microphone. The videos seen in the archive have their audio track replaced by the Flashman audio track because the latter's quality is superior.

\subsection{Postprocessing}
Every evening, the audio data were copied to the laptop, and video was captured with the program Premiere. Video was converted to mpeg1 and mpeg2 with the program Tsunami\footnote{It is a somewhat awkward feeling to use a program called Tsunami in Sri Lanka on an everyday basis.}, while audio was left as pcm/wav. The audio files were segmented into thematic chunks with the program transcriber and cut into smaller sesion files with the export function of the same program. Mpeg1, mpeg2, master audio files and session files were stored in a folder that was burnt on two DVDs when it reached 4 GB. One DVD was sent by snail mail to Amsterdam while the other DVD served as a local backup. 

The data collected during one trip amount to about 50 GB, which is too much for my laptop's hard disk. An external USB-disk was used for extra storing capacity. This disk was also used as an additional backup place as long as space permitted.

The very first narrow phonetic transcriptions were done using transcriber. I wanted to use the program Keyman, which permits easy insertion of IPA symbols into any program. But I had to find that Keyman and transcriber would not work together. I finally used transcriber as if keyman worked. This means for instance typing e= to get a schwa. In any other program, the conversion is done instantly, but not in transcriber. I stored the file with e= and later search\&replaced the e= by \E. After I had worked out an orthography, this was no longer necessary because the orthography only draws on the standard latin alphabet.

Transcriptions were also backed up on the same DVDs as the media files. I wrote a small perl script to convert the transcriber files into Toolbox files, keeping the time code information in a way that ELAN would later be able to import my Toolbox files. Toolbox was used to interlinearize the transcriptions. As a final step, the Toolbox file was imported into ELAN and aligned with video if that was available (the audio file was already aligned through the time code I had kept).
