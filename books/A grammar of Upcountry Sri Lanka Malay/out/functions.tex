


\begin{motto}
Es gibt kein sprachliches Mittel f\"ur jede Situation, aber f\"ur  jede Situation ein Mittel\citep[202]{Keller1990}
\end{motto}


In the last chapter we discussed how SLM can construct well-formed sentences. For every morpheme and every construction, it was shortly described what functions it could fulfill. In this chapter we will switch viewpoint and investigate what means SLM uses to encode universal discourse functions.\footnote{The ordering of sections is loosely inspired by Functional Discourse Grammar \citep{HengeveldEtAl2008fdg}}. Since form and function are closely interrelated, there will also be a lot of references to the formal chapter. These references are indicated by a boxed arrow $\boxdotright$. References to other functional domains are indicated by a circled arrow $\circledotright$.

In this chapter we will discuss  functional domains \funcref{sec:func:FunctionalDomains} \citep{Givon2001a,Givon2001b}.

\chapter{Functional domains}\label{sec:func:FunctionalDomains}

\section{Particpants}\label{sec:func:Particpants}
The participants of a proposition are the entities which participate in the state or event. In the sentence \em Peter gives Mary the book\em, \em Peter, Mary \em and \em the book \em are the three participating entities. Entities can be of different orders, which will be discussed in \funcref{sec:func:Participantsofdifferententityorders}. These participants can fulfill different roles in the proposition, like Agent, Patient or Instrument. This will be discussed in \funcref{sec:func:Participantroles}. Normally, the number of participants and semantic roles are equal and one participant has one role. Sometimes this is not the case, for instance in \em Peter cut himself\em, there is only one participant, \em Peter,\em but there are two roles, Agent and Patient. These cases are discussed in \funcref{sec:func:Mismatchesbetweennumberofsemanticrolesandnumberofsyntacticarguments}. In yet other cases, one wants to include participants in the proposition whose reference is not established, like \em someone \em or \em whoever\em. These cases are discussed in \funcref{sec:func:Unknownparticipants}. Finally, participants can be modified in a number of ways. These modifications are discussed in \funcref{sec:func:Modifyingparticipants}.

\subsection{Participants of different entity orders}\label{sec:func:Participantsofdifferententityorders}
We can distinguish between the following entity orders

\begin{itemize}
 \item individuals x
 \item properties f
 \item states-of-affairs e
 \item propositions p
 \item speech acts
\end{itemize}

In SLM, all of these can be participants. We will treat them in turn.

\subsubsection{Individuals}\label{sec:func:Individuals}
The prototypical participant is an individual. Participants of the individual type are encoded as a noun phrase, which is headed by a noun or a pronoun, as the case may be. In example \xref{ex:func:ptcpt:ent:indiv}, there are two individual participants, a thief, who is encoded by a noun phrase headed by the third person pronoun \em dee \em and the victims of the thief, who are also head of a noun phrase.


\xbox{16}{
\ea \label{ex:func:ptcpt:ent:indiv}
\gll  [\textbf{dee}] [\textbf{oorang} \textbf{pada}]=nang arà-cuuri. \\
    `He steals from the people.'
% \ex
% \gll oorang=pe baarang pada samma arà-cuuri. \\
%     man=\textsc{poss} goods \textsc{pl} all \textsc{non.past}-steal\\
%  `He steals all the peoples' goods. (K051205nar02)
% \z
\z
}


\subsubsection{Properties}\label{sec:func:Properties}
Properties are distinguished from individuals by not having spatial extension\kuckn. They are furthermore distinguished from states-of-affairs by not being located in space and time.

Properties can be used as participants. An example is the property `amount', which participates in the state of ignoring, \em thàràthaau\em, in \xref{ex:func:ptcpt:prop}.

\xbox{16}{
\ea\label{ex:func:ptcpt:prop}
\gll se=dang kalu [\textbf{blaangang}] thàràthaau. \\
    \textsc{1s=dat} if number ignore   \\
    `As for me, I ignore the number (of the descendants).' (K051205nar05)
\z
} \\
Often, properties are derived with the nominalizer \em -an \em when they are to be used as participants, as in \xref{ex:func:ptcpt:prop:an:tsunami}.

\xbox{16}{
\ea\label{ex:func:ptcpt:prop:an:tsunami}
\gll tsunaami anà-jaadi     \textbf{cupath-an}=nang kithang=nang   thàràthaau. \\
      tsunami \textsc{past}-become quick-\textsc{nmlzr} \textsc{1s=dat} ignore \\
\z
} \\
Note that in \xref{ex:func:ptcpt:prop:an:tsunami}, the property is modified by \trs{tsunaami anà-jaadi}{with which the tsunami came}. While the coming of the tsunami is of course localizable in space and time, the property \trs{cupathan}{speed} is not. It is impossible to say \em the speed was at the coast at 7am\em.



%
% K060103nar01.txt:\tx ini      baarang pada=yang      asà-baapi   laayeng nigiri=ka     anà-juwal
% K060103nar01.txt:\tx blaangan arga=nang        anà-juwal



\subsubsection{States-of-affairs}\label{sec:func:States-of-affairs}
States of affairs a second-order entities which have temporal extension. The morphosyntactic expression of a state-of-affairs as a referent depends on its realization status. If the state-of-affairs is realized, a finite subordinate clause is used, as in \xref{ex:ptcpt:ent:soa:cl:realis}, where the writing of the letter has already happened. The past tense prefix \em anà- \em indicates that the subordinate clause is finite.

\xbox{16}{
\ea \label{ex:ptcpt:ent:soa:cl:realis}
\gll   [[lorang suurath=yang  mlaayu=dering \textbf{ana}-thuulis]=nang bannyak arà-suuka]. \\
      \textsc{1pl}=\textsc{poss} father \textsc{past}-say \textsc{2pl} letter=\textsc{acc} Malay=\textsc{abl} past=write=\textsc{dat} much simult-like \\
    `He liked very much that you wrote the letter in Malay.'  (Letter 26.06.2007)
\z

If the state-of-affairs is not realized yet, the subordinate clause will not be finite but be marked with the infinitive \em mà-\em.\footnote{Or the negative infinitive \em jamà-\em.} In example \xref{ex:ptcpt:ent:soa:cl:irrealis}, the predicate \trs{suuka}{like} has two participants, a liker and a likee. The liker is an individual, which is encoded by the first person pronoun \em se(dang)\em, while the wantee is a state-of-affairs, meeting the wife. This state-of-affairs is not realized yet. As a consequence, the verbal predicate carries the infinitive marker \em mà-\em.

}\\
\xbox{16}{
\ea \label{ex:ptcpt:ent:soa:cl:irrealis}
\gll se=dang Andare=pe biini=yang \textbf{mà-caanda} suuka. \\
      \textsc{1s=dat} Andare=\textsc{poss} wife=\textsc{acc} \textsc{inf}-meet like \\
\z
} \\

For some conventional states-of-affairs a lexical solution (i.e. a noun like \trs{pìrrang}{war})  is available.




% A state-of-affairs like `war' can also be expressed by a noun, in this case \em pirrang\em.
%
% \xbox{16}{
% \ea \label{ex:ptcpt:ent:soa:n}
% \gll derang bannyak \textbf{pirrang}=ka=jo arà-buunung. \\
%       \textsc{3pl} many war=\textsc{loc}=\textsc{foc} \textsc{non.past}-kill\\
% \z
% } \\





% \xbox{16}{
% \ea \label{ex:ptcpt:ent:soa:subord}
% \gll itthusubbath=jo incayang=nang, ini sri Lankan {\em Malay} mà-blaajar maau. \\
%   therefore=\textsc{foc} 3p.polite=\textsc{dat} \textsc{prox} Sri Lankan Malay \textsc{inf}-learn want \textsc{quot} \\
% \z
% }



\subsubsection{Propositional content}\label{sec:func:Propositionalcontent}
The difference between states-of-affairs and propositions is that the  former are located in space in time, but have no truth value, while the latter are not located in space and time, but do have a truth value. Propositional content can be true or false, and asserted and denied; this is not possible for states-of-affairs. Propositional content is normally indicated by the quotative \em katha\em, as in the following example, where people discover that they like politics. However, their liking of politics could be an illusion, and the positive truth value asserted here could turn out to be negative. We are thus dealing with propositional content here, and not with a state-of-affairs.

 \xbox{16}{
\ea \label{ex:ptcpt:ent:prop:nclause:politic}
   \gll  itthukapang=jo         \textbf{derang} \zero{} nya-thaau   ambel \textbf{derang} \textbf{pada} {\em politic}=nang   suuka katha. \\
`Only then will they come to know that they like politics' (K051206nar12)
\z
}

The quotative marker is not necessary, as is shown in the following example, where the speaker asserts that the president had not send the subscription money, but again, he could be mistaken and the negative truth value he assigns to the proposition could turn out to be false.

\xbox{16}{
\ea \label{ex:ptcpt:ent:prop:nclause:president}
\gll see su-diya [kithang=pe {\em president}  {\em subscription}=yang thàrà-kiiring]$_{propositional argument}$. \\ % bf
      \textsc{1s} \textsc{past}-see \textsc{1s=poss} president   subscription=\textsc{acc} \textsc{neg.past} send \\
\z
} \\
% Another example of a property being used as a participant is \trs{suuka}{wishing}  in \xref{ex:func:ptcpt:prop:an:sukahan}. This property is used referentially, as is clear from its possessive premodification \trs{Andarepe}{Andare's}. Again, the nominalizer \em -an \em is used here to make the word \em suuka \em denoting a property fit in the argument structure required by the predicate \trs{abbisking}{fulfill}.
%
% \xbox{16}{
% \ea\label{ex:func:ptcpt:prop:an:sukahan}
% \gll Suda raaja=le Andare=pe \textbf{suka-han}=yang mà-abbis-king=nang baaye katha su-biilang.  \\
%       thus king=\textsc{addit} Andare=\textsc{poss} suka-\textsc{nmlzr}=\textsc{acc} \textsc{inf}-finish-\textsc{caus} good \textsc{quot} \textsc{past}-say \\
% \z
% } \\


\subsubsection{Utterances}\label{sec:func:Utterances}
Utterances can also play the role of participants in SLM.   While propositional content can be true or false, utterances do not have a truth value. On the other hand, utterances have illocutionary force and can be judged as to their pragmatic adequacy and felicity, which is not possible for propositional content. In SLM, utterances are marked by \em katha \em in most of the cases. In example \xref{ex:ptcpt:ent:utt}, the content of the scolding of the dwarf is reported. This content can be judged as to whether it was appropriate or not, but it cannot be evaluated for its truth value.

\xbox{16}{
\ea \label{ex:ptcpt:ent:utt}
\gll  [incayang=pe jeengoth=yang asà-thaarek=apa  incayang=nang su-sakith-kang katha]$_{utt}$ anà-maaki. \\ % bf
      \textsc{3s}=\textsc{poss} beard=\textsc{acc} \textsc{cp}-pull=after  \textsc{3s}=\textsc{dat} \textsc{past}-pain-\textsc{caus} \textsc{quot} \textsc{past}-scold\\
\z

}\\

Also non-declarative utterances can be used as participants, an example is a reported question in \xref{ex:ptcpt:ent:utt:interr}.

\xbox{16}{
\ea \label{ex:ptcpt:ent:utt:interr}
\gll [Aashik=nang hathu {\em soldier} mà-jaadi suuka=si katha]$_{utt}$ arà-caanya. \\
     Aashik=\textsc{dat} \textsc{indef} soldier \textsc{inf}-become like=\textsc{interr} \textsc{quot} \textsc{non.past}-ask  \\
\z
} \\
We see that \em katha \em is used both for propositional content and for utterances used as entities. The distinction between these two levels does  therefore not seem to be very relevant in SLM grammar.

% The fact that we are indeed dealing with an utterance and not with a clause is evident from \xref, where the quotative marker \em katha \em has scope over two clauses
%
%
% \xbox{16}{
% \ea\label{ex:func:unreferenced}
% \ea
% \gll [[sama oorang masa-thaksir kithang=pe nigiri=ka kanaapa kithang ini bedahan arà-simpang]$_{CLS}$. \\
%      all man must-think \textsc{1pl}=\textsc{poss} country=\textsc{loc} why \textsc{1pl} \textsc{prox} difference \textsc{non.past}-keep  \\
% \ex
% \gll  [cinggala laayeng mulbar laayeng {\em Moor} laayeng mlaayu laayeng sraani laayeng]$_{CLS}$]$_{UTT}$ \textbf{katha}. \\
%       Sinhala different Tamil different Moor different Malay different Burgher different \textsc{quot} \\
% \ex
% \gll  [itthu caara igaama pada=ka {\em catholic} laayeng {\em protestant} laayeng {\em hindu} igaama laayeng buddha igaama laayeng islaam pada laayeng]$_{UTT}$ \textbf{katha}. \\
%       \textsc{dist} way religion \textsc{pl}=\textsc{loc} catholic different protestant different Hindu religion different Buddhis different Islam \textsc{pl} different \textsc{quot} \\
% \z
% \z
% } \\

\subsection{Participant roles}\label{sec:func:Participantroles}
The participants of a proposition can have different semantic roles. The sentence \em Mary cut the bread with the knife on the balcony \em has four participants, \em Mary, the bread, the knife \em and \em the balcony\em. In this example, Mary has the semantic role of agent, the bread is patient, the knife instrument, and the balcony, location. In the following, we will discuss a number of common semantic roles and how they are expressed in Sri Lanka Malay.

With the exceptions of utterances, all entities discussed in the previous section can be marked for different semantic roles. Utterances are always of the semantic role \textsc{theme}, which is never marked morphosyntacitcally in their case.

Participant marking is done by means of postpositions.  If the role of a participant is inferable from context or general knowledge of the world, the marking can be left out (\citet[26]{Ansaldo2005ms}, \citet[31]{Ansaldo2008genesis}). In example \xref{ex:ptcpt:role:unmarkedpat}, the first sentence establishes that killings are common nowadays. This is done with a reciprocal construction, which bears the accusative postposition \em =yang \em indicating the patient, and no postposition for the agent. In the second sentence, a recent incident is related to the common ground established in the preceding sentence: 7 or 8 more people were involved in a killing the day before. The semantic role of these people is not marked, but it can be infered from context that 7 or 8 people were killed by an unknown number of people. The inverse reading ``7 or 8 people killed an unknown number of people'' is unlikely, first because the amount of victims is easier to establish than the number of killers, and second because this stretch of discourse speaks about dangers for the population and thus takes the viewpoint of the victims. The second sentence contributes additional information about the victimhood of the population, introduced as a topic in the first sentence. The speaker trusts the hearer that the latter will understand this and does not mark the victims with the accusative marker \em =yang \em in the second sentence.

\xbox{16}{
\ea \label{ex:ptcpt:role:unmarkedpat}
\ea
\glll maana aari=le      atthu  atthu   oorang=\textbf{yang}   arà-buunung. \\
     which day=\textsc{addit} \textsc{indef} \textsc{indef} man=\textsc{acc} \textsc{non.past}-kill   \\
    `Everyday people kill each other.'
\ex
\glll kumaareng=le      thuuju=so        dhlaapan=so        oorang=\zero{} asà-buunung. \\ % bf
yesterday=\textsc{addit} seven=\textsc{undet} eight=\textsc{undet} man \textsc{cp}-kill\\
{ }  { } patient { }\\
 `Just yesterday, seven or eight people were killed .'(K051206nar11)
\z
\z
} \\
Leaving out the participant role is very often done for topical participants in initial position, like  in \xref{ex:ptcpt:role:unmarkedtopic}. In this case, the speaker trusts the hearer that the latter will be able to retrieve the participant role of \trs{thumpath}{place(s)} from context, \textsc{location} for this lexeme. This is not too difficult here, since \trs{thumpath}{place(s)} is unlikely to have any other role.

\xbox{16}{
\ea \label{ex:ptcpt:role:unmarkedtopic}
\gll Seelong samma thumpath=\zero{} mlaayu aada. \\
 Ceylon all place Malay exist\\
    `In Sri Lanka, there are Malays everywhere.' (K051222nar04)
\z
} \\
Sometimes, marking of participant roles is less straightforward. In the following example, the undergoers of \trs{thiikam}{stab} are not marked for semantic role while the undergoers of \trs{theembak}{shoot} are.

\xbox{16}{
\ea \label{ex:ptcpt:role:double:thiikam}
\ea
\gll itthukapang      oorang pada=\zero{} thiikam=apa. \\
      then man \textsc{pl} stab=after \\
    `Then people were stabbed'
\ex
\gll  oorang pada=\textbf{nang}   theembak=apa. \\
      man \textsc{pl}=\textsc{dat} shoot=after \\
    `and people were shot'
\ex
\gll  se=dang bannyak creeweth pada su-aada. \\ % bf
      \textsc{1s=dat} much trouble \textsc{pl} \textsc{past}-exist \\
\z
\z
} \\

\subsubsection{Agent}\label{sec:func:Agent}
The agent is normally \zero-marked \citep[19]{Ansaldo2005ms}. An example of this is \trs{kithang}{1pl} in \xref{ex:ptcpt:role:agent}.


\xbox{16}{
\ea \label{ex:ptcpt:role:agent}
\gll ithu=kapang lorang=pe leher=yang kithang=\zero{}$_{ag}$ athi-poothong. \\ % bf
     \textsc{dist}=when \textsc{2pl}=\textsc{poss} neck=\textsc{acc} \textsc{1pl} \textsc{irr}=cut  \\
\z
} \\
If the agent is not a natural person, but an institution, the instrumental marker \em =dering \em is found, as in the following two examples, where the Government and the police are said to be agents of capturing and interrogating.\footnote{Instrumental marking of institutional actors is also found in Sinhala (\citet[31]{GairEtAl1997}, \citet[791]{Gair2003}).}

\xbox{16}{
\ea\label{ex:ptcpt:role:agen:institution1}
\gll {\em British}  {\em Government}=\textbf{dering}   {\em Malaysia} Indonesia,  inni nigiri pada    samma peegang. \\
    British Government=\textsc{abl}  Malaysia Indonesia \textsc{prox} country \textsc{pl}  all catch\\
   `The British Government captured Malaysia and Indonesia,  those countries.' (K051213nar06)
\z
}\\


\xbox{16}{
\ea\label{ex:ptcpt:role:agen:institution2}
\gll see=yang \textbf{{\em police}=dering} nya-preksa. \\
     \textsc{1s}=\textsc{acc} police=\textsc{abl} \textsc{past}-enquire  \\
\z
} \\

\subsubsection{Patient}\label{sec:func:Patient}
The patient of transitive clauses can be marked either with \em =yang \em or with \em =nang\em, but zero-marking is also found \citep{Ansaldo2008genesis}. The  semantics of the verb seem to influence the choice of the postposition, but lexical subcategorization seems to play a part as well. The verb \trs{puukul}{hit}  subcategorizes for \em =nang\em, as do  \trs{theembak}{shoot}\xref{ex:func:ptcpt:semrole:pat:theembak} and \trs{thiikam}{stab}\xref{ex:func:ptcpt:semrole:pat:thiikam}.

\xbox{16}{
\ea\label{ex:func:ptcpt:semrole:pat:puukul}
\gll {\em Dutch}=\textbf{nang} mà-\textbf{puukul}=jo cinggala raaja pada pii aada. \\
      Dutch=\textsc{dat} \textsc{inf}-hit=\textsc{foc} Sinhala king go exist\\
\z
}\\



\xbox{16}{
\ea\label{ex:func:ptcpt:semrole:pat:theembak}
\gll  oorang pada\textbf{=nang}   \textbf{theembak}=apa se=dang bannyak creeweth pada su-aada. \\
      man \textsc{pl}=\textsc{dat} shoot=after \textsc{1s=dat} much trouble \textsc{pl} \textsc{past}-exist \\
\z
} \\

\xbox{16}{
\ea\label{ex:func:ptcpt:semrole:pat:thiikam}
\gll incayang=\textbf{nang}    su-thiikam. \\
     3=\textsc{dat}  \textsc{past}-stab\\
    `They stabbed him.' (K051220nar01)
\z
} \\
The verb \trs{poothong}{cut} on the other hand governs the accusative marker \em =yang\em.

\xbox{16}{
\ea\label{ex:func:ptcpt:semrole:pat:poothong:yang}
\gll ithu=kapang lorang=pe leher=\textbf{yang} kithang athi-\textbf{poothong}. \\
     \textsc{dist}=when \textsc{2pl}=\textsc{poss} neck=\textsc{acc} \textsc{1pl} \textsc{irr}=cut  \\
\z
} \\
\xbox{16}{
\ea\label{ex:func:ptcpt:semrole:pat:poothong:zero}
\gll derang hathu  papaaya=\textbf{yang}   asà-\textbf{poothong}=apa. \\
      3 \textsc{indef} papaya=\textsc{acc} \textsc{cp}-cut=after \\
\z
} \\

It appears that dative marking is more likely for highly affected participants, like the victims of hitting, stabbing and shooting, whereas less dynamic verbs like \trs{ambel}{take} and \trs{thaaro}{put} govern \em =yang\em. Table \ref{tab:func:semrole:pat:yangnang} gives an overview over different verbs.

\begin{table}
\begin{center}
% use packages: array
\begin{tabular}{lll|l}
 & =yang &  & =nang \\
\hline
\trs{kuthumung}{see} & \trs{biilang}{say} &\trs{preksa}{interrogate} &\trs{puukul}{hit} \\
\trs{diyath}{see} & \trs{thaaro}{put} &\trs{caari}{search/find} & \trs{thiikam}{stab} \\
\trs{admit-kang}{admit} &\trs{angkath}{lift} &  \trs{abbis-king}{finish(trs)} &  \\
\trs{ambel}{take} & \trs{caanda}{meet} &\trs{iingath}{think} & \\
\trs{panggel}{call} & \trs{buunung}{kill} &\trs{sakithkang}{hurt} &\\
\trs{uubar}{chase} &\trs{kiiring}{send} & \trs{luppas}{let go} &  \\
\trs{thaanam}{plant} & \\
\end{tabular}
\caption[Transitive verbs which govern the accusative  or the dative]{Transitive verbs which govern the accusative \em =yang \em or the dative \em =nang \em for the undergoer argument.}
\label{tab:func:semrole:pat:yangnang}
\end{center}
\end{table}

% K070000wrt01a.txt~: Oorang su  baawung , thoppi pada yang ana caari.
%
% K070000wrt02a.txt: Andare yang ma enco king nang raaja su biilang itthu paasir katha.
%
% K070000wrt04a.txt~: <91>See lorang nang thama sakith kang
%
% K070000wrt04a.txt~: Derang incayang yang su salbaking.
%
% incayang pe jeengoth yang asa thaarek apa incayang nang su sakith kang
%
% K070000wrt04a.txt~: Aajuth yang buurung ma angkath baapi su diyath.


% \ea\label{ex:func:unreferenced} Alaudin Farook=yang aramaati\z
% \ea\label{ex:func:unreferenced} Alaudin Farooko=nang arapuukul\z

If there is no risk of confounding agent and patient, the patient is commonly not marked by a postposition. This is most notably the case
\begin{itemize}
 \item if the patient is much lower on the animacy hierarchy than the agent,
 \item if the patient has  plural reference,
 \item if the action performed on the patient does not affect him much (see Theme below),
 \item if the patient is not individuated,
 \item if the patient is not topical,
 \item if the other argument is experiencer rather than agent,
 \item if the other argument is a sentential complement
\end{itemize}

Very often, the factors given above conspire, and the occurence of \em =yang \em in a sentence can be argued for or against based on more than one of these factors. Table \ref{tab:func:semrole:pat:yang:noyang} gives an overview of which examples can be used to illustrate the factors given above.



\begin{table}
% use packages: array
\begin{tabular}{lp{1cm}p{1cm}p{1cm}p{1cm}p{1cm}p{1cm}p{1cm}}
 	& +anim & +pl 	& +aff 	& +indiv & +top & +exp 	& +sent \\
-yang 	&
%anim
   \xref{ex:semrole:pat:unmarked:affected}	&
%pl
 \xref{ex:semrole:pat:unmarked:plurality}	&
%aff
   \xref{ex:semrole:pat:unmarked:animacy} 	&
%indiv
   \xref{ex:semrole:pat:unmarked:animacy}
 \xref{ex:semrole:pat:unmarked:affected}
\xref{ex:semrole:pat:unmarked:exp}	&
%top
\xref{ex:semrole:pat:unmarked:topical}  	&
%exp
   \xref{ex:semrole:pat:unmarked:affected}
\xref{ex:semrole:pat:unmarked:exp}
\xref{ex:semrole:pat:unmarked:sententialcomplement}	&
%sent
\xref{ex:semrole:pat:unmarked:sententialcomplement}
       \\
\end{tabular}


\begin{tabular}{lp{1cm}p{1cm}p{1cm}p{1cm}p{1cm}p{1cm}p{1cm}}
 	& -\textsc{anim} & -\textsc{pl} 	& -aff 	& -indiv & -top & -exp 	& -sent \\
-yang 	&
%anim
   \xref{ex:semrole:pat:unmarked:animacy}
 \xref{ex:semrole:pat:unmarked:plurality}
 \xref{ex:semrole:pat:unmarked:individuated}
\xref{ex:semrole:pat:unmarked:topical}
\xref{ex:semrole:pat:unmarked:exp}	&
%pl
  \xref{ex:semrole:pat:unmarked:animacy}
 \xref{ex:semrole:pat:unmarked:affected}
\xref{ex:semrole:pat:unmarked:individuated}
\xref{ex:semrole:pat:unmarked:topical}
\xref{ex:semrole:pat:unmarked:exp}&
%aff
  \xref{ex:semrole:pat:unmarked:plurality}
\xref{ex:semrole:pat:unmarked:affected}
\xref{ex:semrole:pat:unmarked:individuated}
\xref{ex:semrole:pat:unmarked:exp}	&
%indiv
  \xref{ex:semrole:pat:unmarked:plurality}
\xref{ex:semrole:pat:unmarked:individuated}
\xref{ex:semrole:pat:unmarked:topical}	&
%top
  \xref{ex:semrole:pat:unmarked:animacy}
\xref{ex:semrole:pat:unmarked:plurality}
  \xref{ex:semrole:pat:unmarked:affected}
\xref{ex:semrole:pat:unmarked:individuated}
\xref{ex:semrole:pat:unmarked:exp}	&
%exp
  \xref{ex:semrole:pat:unmarked:animacy}
\xref{ex:semrole:pat:unmarked:plurality}	&
%sent
   \xref{ex:semrole:pat:unmarked:animacy}
 \xref{ex:semrole:pat:unmarked:plurality}
 \xref{ex:semrole:pat:unmarked:affected}
\xref{ex:semrole:pat:unmarked:exp}   \\

\end{tabular}
\caption[Zero-marking of the patient]{Zero-marking of the patient as a function of a number of factors. Numbers refer to the examples below.}
\label{tab:func:semrole:pat:yang:noyang}
\end{table}

\xbox{16}{
\ea \label{ex:semrole:pat:unmarked:animacy}
\gll   itthu=kapang baapa  derang=pe     kubbong=ka  hatthu pohong]$_{PAT}$=\zero{} nya-poothong. \\ % bf
       \textsc{dist}=when father \textsc{3pl}=\textsc{poss} estate=\textsc{loc} \textsc{indef} tree \textsc{past}-cut\\
\z
} \\
\xbox{16}{
\ea \label{ex:semrole:pat:unmarked:plurality}
\gll  [oorang anà-baawa samma thoppi=pada]$_{PAT}$=\zero{} asà-ambel. \\ % bf
      man \textsc{past}-bring all hat=\textsc{pl} \textsc{cp}-take\\
\z
}\\

\xbox{16}{
\ea \label{ex:semrole:pat:unmarked:affected}
\gll  ithukapang       oorang pada    [hathu  oorang]$_{PAT}$=\zero{} asà-kuthumung. \\ % bf
then man \textsc{pl} \textsc{indef} man \textsc{cp}-see \\
\z
} \\

\xbox{16}{
\ea \label{ex:semrole:pat:unmarked:individuated}
\gll  baapa=le       aanak=le      [guula]$_{PAT}$=\zero{} su-maakang. \\ % bf
      father=\textsc{addit} son=\textsc{addit} sugar \textsc{past}-eat \\
    `.' (K070000wrt02)
\z
} \\
\xbox{16}{
\ea \label{ex:semrole:pat:unmarked:topical}
\ea
\gll  itthu=nang [aayer]$_{PAT}$=\zero{} asà-thaaro=apa. \\ % bf
       \textsc{dist}=\textsc{dat} water \textsc{cp}-put=after\\
    `Having added water to it'
\ex
\gll  [\textbf{itthu} aayer]$_{PAT}$=yang baaye=nang arà-{\em boil}-kang. \\
       \textsc{dist} water=\textsc{acc} good=\textsc{dat} \textsc{non.past}-boil-caus\\
\z
\z
} \\
\xbox{16}{
\ea \label{ex:semrole:pat:unmarked:exp}
\gll [kìrras pinthu=nang arà-thatti hathu swaara]$_{PAT}$=\zero{} su-dìnggar. \\ % bf
     strong door=\textsc{dat} \textsc{simult}-hammer \textsc{indef} noise] \textsc{past}-hear \\
\z
}\\


\xbox{16}{
\ea \label{ex:semrole:pat:unmarked:sententialcomplement}
\gll Blaakang=jo incayang anà-kuthumung [moonyeth pada thoppi asà-ambel pohong atthas=ka arà-maayeng]$_{PAT}$=\zero{}. \\ % bf
     after=\textsc{foc} 3s.polite \textsc{past}-see monkey \textsc{pl} hat \textsc{cp}-take tree top=\textsc{loc} \textsc{simult}-play  \\
\z
}\\

It is common that several of these factors conspire. For instance, \em thoppi \em in example \xref{ex:semrole:pat:unmarked:plurality} is at the same time unindividuated, plural reference, low on the animacy hierarchy and relatively unaffected (cf. Table \ref{tab:func:semrole:pat:yang:noyang}).

If on the other hand, there is a risk of confounding agent and patient, \em =yang \em is used more often. This is notably the
case under the following conditions.

\begin{itemize}
 \item   the agent is less animate than the patient \xref{ex:semrole:pat:marked:revanim}
 \item   the patient is  individuated \xref{ex:semrole:pat:marked:indiv}.
\end{itemize}


\xbox{16}{
\ea \label{ex:semrole:pat:marked:revanim}
\gll  Aajuth=\textbf{yang} buurung mà-angkath baapi su-diyath. \\
      dwarf=\textsc{acc} bird \textsc{inf}-lift take.away \textsc{past}-try \\
    `The bird tried to carry the dwarf away.' (K070000wrt04a)
\z
} \\
In this example, the dwarf is comparatively little affected, but he is individuated, and, more importantly, the normal order of more animate entities acting upon less animate ones is inverted here.
Again these factors conspire very often, and in example \xref{ex:semrole:pat:marked:indiv} the speaker and his persecutors are close on the animacy hierarchy, and the speaker is individuated.

\xbox{16}{
\ea \label{ex:semrole:pat:marked:indiv}
\gll Se=ppe oorang pada [see saapa] katha thàràthaau subbath \textbf{see=yang} su-uubar. \\
     \textsc{1s=poss} man \textsc{pl} \textsc{1s} who \textsc{quot} ignore because \textsc{1s}=\textsc{acc} \textsc{past}-chase\\
\z
}\\

It should be noted that no hard and fast rule for the occurence of \em =yang \em can be given; its occurence depends on multiple factors, and even then there are unexplainable sentences. An examples is \xref{ex:semrole:pat:double} where two things are bought, but one is marked with \em =yang \em and the other one is not.

\xbox{16}{
\ea \label{ex:semrole:pat:double}
\gll Lai se {\em computer}=nang baaru {\em optical} {\em mouse} atthu=\zero=le {\em Encarta2006} {\em software}=\textbf{yang}=le su-bìlli. \\
     other \textsc{1s} computer=\textsc{dat} new optical mouse \textsc{indef}=\textsc{addit} Encarta2006 software=\textsc{acc}=\textsc{addit} \textsc{past}-buy  \\
\z
}\\

The semantic factors sketched above have an influence, but also idiolectal and possibly style factors. For every simple determinant factor for the occurence of \em =yang\em, counterexamples can be found, but a combination of several of the criteria mentioned above could possibly yield good predictions. Such a multivariate analysis is far beyond the scope of this grammatical description.

%
%
% K070000wrt04a.txt~: Kaaki ka gaandas kang ambel ana duuduk Aajuth yang sanke luppas hathu pollu dering Rose-red buurung nang su puukul.
%
% K070000wrt04a.txt~:Aajuth thaakuth ka su naangis , <93>See yang luppas , Thuan Buruan.
%
%
% K070000wrt04a.txt~: Itthule see yang ma kiiring nang duppang incayang see yang hathu Buruan ma jaadi su baleking.
%

% \xbox{16}{
% \ea\label{ex:func:unreferenced}
% \gll oorang arà-buunung    samma. \\
%      man \textsc{non.past}-kill all  \\
%     `He killed everybody.' (K051205nar02)
% \z
% } \\

The affectedness of a patient can additionally be highlighted by the vector verb \em thaaro \em \formref{sec:wc:thaaro}, as in the following examples.

\xbox{16}{
\ea\label{ex:semrole:pat:thaaro1}
\gll    incayang=yang    siaanu  asà-buunung   \textbf{thaaro}=apa. \\
    3s.polite=\textsc{acc} 3s.prox \textsc{cp}-kill put=after \textsc{quot}   \\
    `This one has killed him.' (K051220nar01)
\z
} \\
\xbox{16}{
\ea\label{ex:semrole:pat:thaaro2}
\gll see=yang dhaathang {\em remand}=ka mà-thaarek \textbf{thaaro}=nang thàràboole su-jaadi. \\
     \textsc{1s}=\textsc{acc} come remand=\textsc{loc} \textsc{inf}-pull put=\textsc{dat} cannot \textsc{past}-become  \\
\z
} \\
\citet{SmithEtAl2004} claim that accusative (patient) and dative (recipient) are conflated in SLM, but the base for their claim is unclear \citep{Ansaldo2005ms}. The forms \em =yang \em and \em =nang \em are clearly distinct (with the exception of a very infrequent \em =nya\em, which might be an allomorph of either, and for whose occurrence no explanation can be given as of now). The verbs also clearly subcategorize either for the one or the other, so that there are neither morphophonological nor distributional reasons to assume a conflation of patient marking and recipient marking.

\subsubsection{Theme}\label{sec:func:Theme}
Themes are either marked by \zero{} or by \em =yang\em.

\xbox{16}{
\ea\label{ex:semrole:pat:theme:zero}
\gll se=ppe    baapa  incayang=nang    ummas=\zero{} su-kaasi. \\ % bf
      \textsc{1s=poss} father 3s.polite=\textsc{dat} gold \textsc{past}-give\\
    `My father gave him gold.'  (K070000wrt04)
\z
}\\

\xbox{16}{
\ea\label{ex:semrole:pat:theme:yang}
\gll derang=pe perhaal  pada=\textbf{yang}   mà-thuulis    ambel=nang. \\
      \textsc{3pl}=\textsc{poss} report \textsc{pl}=\textsc{acc} \textsc{inf}-write take=\textsc{dat} \\
\z
} \\
Since themes are by definition less affected than patients, marking with \em =yang \em is less common. Themes are never marked by \em =nang\em.
Themes of psych verbs or utterance verbs can also be marked by \trs{atthas}{about}, as in \xref{ex:semrole:pat:theme:atthas}.

\xbox{16}{
\ea\label{ex:semrole:pat:theme:atthas}
\gll se=ppe \textbf{atthas} laskalli masa-biilang=si. \\
      \textsc{1s=poss} about again must-say=\textsc{interr} \\
\z
} \\
An interesting case is \xref{ex:semrole:pat:theme:venerate}, where the them is marked by \em =yang\em, which is possibly to underscore the individuated nature and anthropomorphic image of the cow venerated by Hindus. I have tried to render this by capitalization in English.

\xbox{16}{
\ea\label{ex:semrole:pat:theme:venerate}
\gll  Hindu pada sampi=\textbf{yang}   arà-muuji. \\
      Hindu \textsc{pl} cow=\textsc{acc} \textsc{non.past}-venerate \\
\z
} \\
\subsubsection{Recipient}\label{sec:func:Recipient}
Recipients are always coded with the postposition \em =nang\em. \xref{ex:semrole:rec} gives an example.

\xbox{16}{
\ea\label{ex:semrole:rec}
\gll se=ppe    baapa  incayang=\textbf{nang}    ummas su-kaasi. \\
      \textsc{1s=poss} father 3s.polite=\textsc{dat} gold \textsc{past}-give\\
    `My father gave him gold.'  (K070000wrt04)
\z
}\\



\subsubsection{Experiencer}\label{sec:func:Experiencer}
The semantic role of experiencer is indicated by \em =nang\em \citep{Ansaldo2005ms,Ansaldo2008genesis}, as common in South Asia \citep{Masica1976,Sridhar1976cls,Sridhar1976sils,Sridhar1979,VermaEtAlEd1990,BhaskararaoEtAlEd2004I,BhaskararaoEtAlEd2004II}.\kuckn The following sentences give examples for experiencers of mental predicates \xref{ex:ptcpt:semrole:exp:mental1}-\xref{ex:ptcpt:semrole:exp:mental4}, sensory  predicates \xref{ex:ptcpt:semrole:exp:sensory1}\xref{ex:ptcpt:semrole:exp:sensory2} and bodily predicates\xref{ex:ptcpt:semrole:exp:bodily1}.



\xbox{16}{
\ea \label{ex:ptcpt:semrole:exp:mental1}
\gll   {\em tailoring} go\textbf{dang}    baaye=nang   \textbf{thaau}$_{mental}$. \\
       tailoring \textsc{1s=dat} good=\textsc{dat} know\\
    `I know tailoring very well.' (B060115nar04)
\z
} \\

\xbox{16}{
\ea \label{ex:ptcpt:semrole:exp:mental2}
\gll Inni     oorang=\textbf{nang}   itthu    \textbf{thàràthaau}$_{mental}$. \\
      \textsc{prox} man=\textsc{dat} \textsc{dist} ignore \\
\z
} \\



\xbox{16}{
\ea \label{ex:ptcpt:semrole:exp:mental3}
\gll derang pada=\textbf{nang}   karang {\em Malay} arà-\textbf{luupa}$_{mental}$. \\
     \textsc{3pl} \textsc{pl}=\textsc{dat} now Malay \textsc{non.past}-forget  \\
\z
} \\


\xbox{16}{
\ea \label{ex:ptcpt:semrole:exp:mental4}
\gll  Suda Andare=\textbf{nang}=le        buthul \textbf{suuka}$_{mental}$. \\
      thus Andare=\textsc{dat}=\textsc{addit} very like \\
    `So Andare also liked it a lot.' (K070000wrt05)
\z
} \\



\xbox{16}{
\ea \label{ex:ptcpt:semrole:exp:sensory1}
\gll   Derang=\textbf{nang} byaasa swaara hatthu su-\textbf{dìnggar}$_{sensory}$. \\
      \textsc{3pl}=\textsc{dat} habit noise \textsc{indef} \textsc{past}-hear\\
\z
} \\

\xbox{16}{
\ea \label{ex:ptcpt:semrole:exp:sensory2}
\gll Itthu haari=ka=jo aanak pompang duuwa=\textbf{nang} ... kiccil jillek Aajuth hatthu=yang su-\textbf{kuthumung}$_{sensory}$ \\
     \textsc{dist} day=\textsc{loc}=\textsc{foc} child female two=\textsc{dat} ... small ugly dwarf \textsc{indef}=\textsc{acc} \textsc{past}-see  \\
\z
} \\



\xbox{16}{
\ea \label{ex:ptcpt:semrole:exp:bodily1}
\gll  go\textbf{dang}    karang bannyak \textbf{thàràsiggar}$_{bodily}$. \\
      1s.familiar=dat now lot sick \\
    `I am very sick now.' (B060115nar04)
\z
} \\


%
%
% \xbox{16}{
% \ea \label{ex:ptcpt:semrole:exp:social1}
% \gll se=ppe bungkus su iilang. \\
%      \textsc{1s=poss} purse \textsc{past}-disappear  \\
%     `I lost my purse.' (test)
% \z
% } \\
%
% \xbox{16}{
% \ea
% \gll *se=dang suiilang. \\
%        \\
%     `.' (nosource)
% \z
% } \\
%




%
% \xbox{16}{
% \ea\label{ex:func:unreferenced}
% \gll Aanak pompang duuwa su-thaakuth. \\
%      child woman two \textsc{past}-fear  \\
%     `The two girls were scared.'  (K070000wrt04)
% \z
% }\\

The fact that the modals \trs{maau}{want}, \trs{thussa}{not.want}, \trs{boole}{can} and \trs{thàrboole}{cannot} mark the argument can also be explained by the fact that the person having the desire or ability (or lack thereof) is normally not actively responsible for that in the moment of speaking.  True, a person who can drive a car has actively acquired that knowledge at some point in time, but at the time of speaking, she is passive.

%
%
% \xbox{16}{
% \ea\label{ex:func:unreferenced}
% \gll cinggala mulbar thàrà thaau. \\
%  sinhala tamil \textsc{neg} know\\
% `Sinhalese do not know Tamil.' (K051220nar02 )
% \z
% }
%
%


\subsubsection{Source}\label{sec:func:Source}
Source can be either marked by \em =dering \em \xref{ex:ptcpt:semrole:src:dring} or  \em sduuduk \em  \xref{ex:ptcpt:semrole:src:sduuduk1}.


\xbox{16}{
\ea \label{ex:ptcpt:semrole:src:dring}
\ea
\gll kitham=pe      oorang thuuwa pada  bannyak dhaathang aada {\em Malaysia}=\textbf{dering}. \\
     \textsc{1pl}=\textsc{poss} man old \textsc{pl} many come exist Malaysia=\textsc{abl}  \\
      `Many of our ancestors came from Malaysia,'
\ex
\gll spaaru Indonesia=\textbf{dering}      dhaathang aada. \\
     some Indonesia=\textsc{abl} come exist\\
  `some came from Indonesia.' (K060108nar02.73)
\z
\z
} \\


\xbox{16}{
\ea \label{ex:ptcpt:semrole:src:sduuduk1}
\gll  suda see {\em Trinco}=ka  \textbf{asaduuduk} Kluumbu=nang   su-dhaathang. \\
      So \textsc{1s} Trincomalee=\textsc{loc} from Colombo=\textsc{dat} \textsc{past}-come \\
\z
} \\
Since \em sduuduk \em is the conjunctive participle of the animate existential verb \em duuduk\em, it can only be used to indicate the source of  animate entitities. The stones in \xref{ex:ptcpt:semrole:src:dring} are not animate, and \em sduuduk \em cannot be used there to indicate the origin of the stones.


The following examples show some more uses of \em =dering \em to indicate source

\xbox{16}{
\ea \label{ex:ptcpt:semrole:src:dering:extra1}
\gll  Hathu haari, hathu oorang thoppi mà-juwal=nang kampong=\textbf{dering} kampong=nang su-jaalang pii. \\
     \textsc{indef} day \textsc{indef} man hat \textsc{inf}-sell=\textsc{dat} village=\textsc{abl} village=\textsc{dat} \textsc{past}-walk go  \\
\z
}\\


%
% \xbox{16}{
% \ea\label{ex:func:unreferenced}
% \gll 1967     asaduuduk {\em Kandy}=ka    su-duuduk. \\
%       1967 from Kandy=\textsc{loc} \textsc{past}-stay \\
% \z
% } \\
\xbox{16}{
\ea \label{ex:ptcpt:semrole:src:extra2}
\ea
\gll see asà-{\em retire} aada {\em police}=\textbf{dering}. \\
      \textsc{1s} \textsc{cp}-retire after police=\textsc{abl} \\
\ex
\gll karang {\em retire}=apa. \\ % bf
     now retire=after  \\
    `after having retired'
\ex
\gll lima-blas duwa-pulu thaaun arà-jaadi. \\ % bf
     five-teen two-ty year \textsc{non.past}-become  \\
    `fifteen, twenty years ago ...' (B060115prs14)
\z
\z
} \\
\xbox{16}{
\ea \label{ex:ptcpt:semrole:src:extra3}
\gll {\em Kandy} {\em Malay} {\em Association}=\textbf{dering} hatthu hatthu oorang pada arà-lompath {\em Hill} {\em Country}=nang. \\
     Kandy Malay association=\textsc{abl} \textsc{indef} \textsc{indef} man \textsc{pl} \textsc{non.past}-jump hill country=\textsc{dat}  \\
\z
} \\
The use of \em sduuduk \em is exemplified by the sentences below.

\xbox{16}{
\ea \label{ex:ptcpt:semrole:src:sduuduk:extra1}
\gll kithan       nya-pii Anuradhapura=dang, Katunaayaka     \textbf{sduuduk}. \\
 \textsc{1pl} \textsc{past}-go Anuradhapura=\textsc{dat} Katunayaka from\\
\z
}


\xbox{16}{
\ea \label{ex:ptcpt:semrole:src:sduuduk:extra2}
\gll Katugastota \textbf{asaduuduk} St.Anthony's=nang      arà-jaalang \\
     Katugastota from St.Anthony's=\textsc{dat} \textsc{non.past}-walk  \\
\z
} \\
% The participant marked by \em sduuduk\em, does not have to undergo a change of location, even if the etymology of \em asà-duuduk \em (having sat) suggests otherwise. In the following example, the angels do not change their position, they remain immobile on both sides of the grave, but the provenance of the questions is from those sides. One can either argue that in \xref, \em sduuduk \em refers to the questions, which do indeed `travel' from the mouth to the ear as it were, and thereby change location. This is a bit strange, though, since inanimate participants, such as questions, can normally not combine by \em sduuduk\em. It would also be possible to see
%
% \xbox{16}{
% \ea\label{ex:func:unreferenced}
% \ea
% \gll  mayyeth arà-kubuur-kang wakthu. \\
%       corpse \textsc{non.past}-bury-\textsc{caus} time \\
%     `When the corpse is buried'
% \ex
% \gll  mayyeth=pe kubur-an paapang=nya arà-thuuthup wakthu=le. \\
%       corpse=\textsc{poss} bury-nlmzr pole=\textsc{acc} \textsc{non.past}-close time=\textsc{addit} \\
% \ex
% \gll mà-liyath thaakuth. \\
%      \textsc{inf}-watch fear   \\
%     `To watch the angel.'
% \ex
% \gll mleekath karakiye  duwa subla=le asaduuduk percayahan arà-caanya. \\
%       angel ??? two side=\textsc{addit} from question \textsc{non.past}-ask\\
%     `The angel ??? will ask questions from both sides.' (K060116sng02)
% \z
% \z
% } \\

\subsubsection{Goal}\label{sec:func:Goal}
Goal is normally coded by \em =nang\em (\citet[19,22]{Ansaldo2005ms},\citet[24]{Ansaldo2008genesis}), which attaches to the place name \xref{ex:ptcpt:semrole:goal:placename}, the spatial enitity \xref{ex:ptcpt:semrole:goal:spatialnoun} or a relator noun \xref{ex:ptcpt:semrole:goal:relatornoun}  but deictics do not take this marker when used as goal \xref{ex:ptcpt:semrole:goal:deictic}.


\xbox{16}{
\ea \label{ex:ptcpt:semrole:goal:placename}
\gll  soojer pada  incayang=sàsaama Seelon=\textbf{nang} asà-dhaathang. \\
European \textsc{pl}  3s.polite with Ceylon=\textsc{dat} \textsc{cp}-come\\
\z
}


\xbox{16}{
\ea \label{ex:ptcpt:semrole:goal:spatialnoun}
\gll suda lorang=yang ruuma=\textbf{nang} anthi-aaji.baapi \\
    thus \textsc{2pl}=\textsc{acc} house=\textsc{dat} \textsc{irr}=bring.anim   \\
\z
} \\


%
% \xbox{16}{
% \ea\label{ex:func:unreferenced}
% \gll  Hathu haari , hathu oorang thoppi mà-juwal=nang kampong=dering kampong=\textbf{nang} su-jaalang pii. \\
%      \textsc{indef} day \textsc{indef} man hat \textsc{inf}-sell=\textsc{dat} village=\textsc{abl} village=\textsc{dat} \textsc{past}-walk go  \\
% \z
% }\\



%
%  \xbox{16}{
%  \ea\label{ex:func:unreferenced}
%    \gll  blaakang see angkath   thaangangnang   subiilang     Badulla  thamadhaathangsi. \\
%     \textsc{neg.nonpast}- come      -Q \\
% (B060115cvs01)
% \z
% }

%   00033
% f K051206nar20
%  suda see Trincoka         asaduuduk Kluumbunang   sudhaathang
%  So I went from Trincomalee to Colombo


\xbox{16}{
\ea \label{ex:ptcpt:semrole:goal:relatornoun}
\gll  ithukang ithu bambu giithu=jo luwar=\textbf{nang} arà-dhaathang. \\
      then this bamboo like.that=\textsc{foc} outside=\textsc{dat} \textsc{non.past}-come\\
\z
} \\

\xbox{16}{
\ea \label{ex:ptcpt:semrole:goal:deictic}
\gll se=ppe    {\em profession}=subbath se=dang  siini\textbf{=\zero} mà-pii    su-jaadi. \\
      \textsc{1s=poss} profession=because \textsc{1s=dat} here \textsc{inf}-go \textsc{past}-become \\
    `I had to come here because of my profession.' (G051222nar01)
\z
} \\


If the goal of motion is human, the relator noun \em dìkkath \em is obligatory, as shown in \xref{ex:ptcpt:semrole:goal:human} \citep[cf.][]{SmithEtAl2004}.
\xbox{16}{
\ea \label{ex:ptcpt:semrole:goal:human}
   \gll  kithang=nang   hathu  {\em job} hatthu mà-ambel=nang      kithang=nang   hathu  {\em application} mà-sign  kamauwan wakthu=nang=jo      kithang arà-pii    inni     {\em politicians} pada \textbf{dìkkath=nang}. \\ % bf
    \textsc{1pl}=\textsc{dat} \textsc{indef} job \textsc{indef} \textsc{inf}-take=\textsc{dat} \textsc{1pl}=\textsc{dat} \textsc{indef} application \textsc{inf}-sign want time=\textsc{dat}=\textsc{foc} \textsc{1pl} \textsc{non.past}-go \textsc{prox} politicians \textsc{pl} vicinity=dat\\
\z
}



\em =nang \em can attach to proper nouns refering to non-human entities, even if they are not spatial in the strict sense of the term, such as in \xref{ex:ptcpt:semrole:goal:association}, where the goal of motion is an association.

\xbox{16}{
\ea \label{ex:ptcpt:semrole:goal:association}
\gll {\em Kandy} {\em Malay} {\em Association}=dering hatthu hatthu oorang pada ara lompath \textbf{{\em Hill}} \textbf{{\em Country}=nang}. \\
    `More and more people stepped over from the KMA to the Hill Country Malay Association.' (K060116nar07)
\z
} \\
Goal of motion can also be marked by the locative marker \em =ka\em \xref{ex:ptcpt:semrole:goal:ka:government}\xref{ex:ptcpt:semrole:goal:ka:hathuhathu} \citep{SmithEtAl2004}.
\xbox{14}{
\ea \label{ex:ptcpt:semrole:goal:ka:government}
\gll {\rm government}=pe     hathu  thumpath=\textbf{ka}    asà-pii   pukurjan bole=girja \\
     government=\textsc{poss} \textsc{indef} place=\textsc{loc} \textsc{cp}-go work can-make  \\
\z
} \\

\xbox{14}{
\ea \label{ex:ptcpt:semrole:goal:ka:hathuhathu}
\gll derang samma oorang hatthu hatthu thumpath pada=ka asà-pii pukurjan su-gijja \\
      \textsc{3pl} all man \textsc{indef} \textsc{indef} place \textsc{pl=loc} \textsc{cp}-go work \textsc{past}-make \\
\z
} \\


\subsubsection{Path}\label{sec:func:Path}
Path is coded by \em =dering\em. In the following example, the bird's trajectory has an unknown source and an unknown goal, but leads over the girls' heads. This is indicated by the ablative marker \em =dering\em.

\xbox{16}{
\ea \label{ex:ptcpt:semrole:path}
\gll bìssar hathu buurung derang=pe atthas=\textbf{dering} su-thìrbang. \\
      big \textsc{indef} bird \textsc{3pl}=\textsc{poss} top=\textsc{abl} \textsc{past}-fly \\
\z
}\\

\subsubsection{Instrument}\label{sec:func:Instrument}
Instrument is indicated by \em=dering\em\formref{sec:morph:=dering}. The instrument can be used by a sentient being, as in \xref{ex:ptcpt:semrole:instr:sent}, but \em =dering \em can also be used in a wider sense for participants not manipulated by a sentient being as in \xref{ex:ptcpt:semrole:instr:nonsent}.

\xbox{16}{
\ea\label{ex:ptcpt:semrole:instr:sent}
\gll thaangang=dering bukang kaaki=\textbf{dering} masa-maayeng. \\
      hand=\textsc{abl} \textsc{neg.nonv} leg=\textsc{abl} must-play \\
\z
} \\
\xbox{16}{
\ea \label{ex:ptcpt:semrole:instr:nonsent}
\gll \textbf{daawon=dering}     thuuthup ada  gaaja    hatthu asdhaathang. \\
     leaf=instr close exist elephant \textsc{indef} \textsc{cp}-come  \\
\z
} \\
% Languages can also be conceived as an instrument for communication and are marked with \em =dering \em as well.
%
% \xbox{16}{
% \ea\label{ex:func:unreferenced}
% \gll arà-biilang mlaayu=dering. \\
%  \textsc{non.past}-speak Malay=abl\\
% `Speak in Malay.' (K060103nar01)
% \z
% }

The semantic role of instrument is normally always marked overtly. An exception to this rule is found in \xref{ex:ptcpt:semrole:instr:drop}, where the ears could be said to be the instruments of hearing, yet they are not marked by \em =dering\em. Since ears are so prototypically associated with hearing, marking of their semantic role does not seem necessary.

\xbox{16}{
\ea \label{ex:ptcpt:semrole:instr:drop}
\gll  Andare ruuma=nang asà-pii biini=nang su-biilang puthri=nang \textbf{kuuping} arà-dìnggar kuurang katha. \\
    Andare house=\textsc{dat} \textsc{cp}-go wife=\textsc{dat} \textsc{past}-say princess=\textsc{dat} ear \textsc{non.past}-hear little \textsc{quot}  \\
\z
} \\


% \xbox{16}{
% \ea
% \gll Muralipe boolayang wicketka sukìnna. \\
%        \\
%     `.' (nosource)
% \z
% } \\

\subsubsection{Beneficiary}\label{sec:func:Beneficiary}
Beneficiary is normally coded by \em =nang\em \citep{Ansaldo2005ms,Ansaldo2008genesis} \xref{ex:ptcpt:semrole:ben:overt1}\xref{ex:ptcpt:semrole:ben:overt2}, but can also be marked with a relator noun \trs{bagiyan}{behalf}. As with the other roles, the overt marking can be dropped if the role is clear from context \xref{ex:ptcpt:semrole:ben:dropped}.

\xbox{16}{
\ea \label{ex:ptcpt:semrole:ben:overt1}
\gll   cinggala   raaja\textbf{=nang} deram  pada banthu aada. \\
 sinhala king=\textsc{dat} \textsc{3pl} \textsc{pl} help exist\\
\z
}

\xbox{16}{
\ea \label{ex:ptcpt:semrole:ben:overt2}
\gll  Derang=pe umma derang=\textbf{nang} jaith-an=le, jaarong pukurjan=le su-aajar. \\
       \textsc{3pl}=\textsc{poss} mother \textsc{3pl}=\textsc{dat} sew-\textsc{nmlzr}=\textsc{addit} needle work=\textsc{addit} \textsc{past}-teach\\
\z
}\\


% B060115cvs01.txt- itthu    kumpulan=dang      derang=jo     bannyak arà-banthu

\xbox{16}{
\ea \label{ex:ptcpt:semrole:ben:dropped}
\gll  kettha  drampada=\zero{}      bannyak banthu. \\ % bf
      \textsc{1pl} \textsc{3pl} much help \\
\z
} \\
% \xbox{16}{
% \ea\label{ex:func:unreferenced}
% \gll Derang derang=pe umma=nang butthul saayang=kee=jo samma ruuma pukurjan=nang=le anà-banthu. \\
%       \textsc{3pl} \textsc{3pl}=\textsc{poss} mother=\textsc{dat} correct love=\textsc{simil}=\textsc{foc} all house work=\textsc{dat}=\textsc{addit} \textsc{past}-help  \\
% \z
% }\\

% Very often, the role of beneficiary is found with three-place  predicates, but it is also possible to find it in two place predicates as in \xref{ex:ptcpt:semrole:ben:twoplace}.
%
% \xbox{16}{
% \ea\label{ex:ptcpt:semrole:ben:twoplace}
% \gll itthu=nang blaakang [kithang=\textbf{nang}   anà-daapath    {\em government}]=ka incayang=yang    uthaama mlaayu=nang   anà-duuduk. \\
%       \textsc{dist}=\textsc{dat} after \textsc{1pl}=\textsc{dat} \textsc{past}-get government=\textsc{loc} 3s.polite=\textsc{acc} honour Malay=\textsc{dat} \textsc{past}-exist.\textsc{anim}  \\
% \z
% } \\
Beneficiary can also be added to another predication which does not require a beneficiary strictly speaking, if the action turned out to be beneficial. This is shown in \xref{ex:ptcpt:semrole:ben:constr}.

\xbox{16}{
\ea\label{ex:ptcpt:semrole:ben:constr}
\gll  giini   duuduk     bannyak  [\textbf{kithang=pe} \textbf{oorang} \textbf{pada}]=nang  anà-caape. \\
    `Being there, he toiled a lot for our people.' (N061031nar01)
\z
} \\
The beneficial nature of an action can addtionally be highlighted by the vector verb \trs{kaasi}{give} as in \xref{ex:ptcpt:semrole:ben:kaasi:drop}. Note that the benefitting persons are not overtly realized because they have been established as a topic before.


\xbox{16}{
\ea\label{ex:ptcpt:semrole:ben:kaasi:drop}
\gll  itthu muusing  Islam igaama  nya-\textbf{aajar} \textbf{kaasi} Jaapna  Hindu {\em teacher}. \\
    `At that time, those who taught Islamic religion were Hindu teachers from Jaffna.' (K051213nar03)
\z
} \\
It is also possible to overtly realize the benefitting participant when \em kaasi \em is used, as in \xref{ex:ptcpt:semrole:ben:kaasi:overt}.

\xbox{16}{
\ea\label{ex:ptcpt:semrole:ben:kaasi:overt}
   \gll  kithang=pe     ini      {\em younger} {\em generation}=nang=jo     konnyong masa-\textbf{biilang} \textbf{kaasi}, masa-aajar. \\
 `It is to the younger generation that we must explain it, must teach it.'(B060115cvs01)
\z
}

If the action is beneficial to the agent itself, instead of \trs{kaasi}{give}{}, \trs{ambel}{take}{} is used.

\xbox{16}{
\ea\label{ex:ptcpt:semrole:ben:ambel}
\gll [Tony Hassan {\em uncle}=nang asà-kaasi duwith] athi-\textbf{mintha} \textbf{ambel}=si? \\
    `Shall I ask for the money you gave to uncle Tony Hassan?' (K071011eml01)
\z
} \\
In this case, the person performing the action of asking would also profit from it since he has a chance of getting the money.

In rare cases, \em ambel \em can also be used if the action profits other entities than the agent, as in \xref{ex:ptcpt:semrole:ben:ambel:cross}.

\xbox{16}{
\ea\label{ex:ptcpt:semrole:ben:ambel:cross}
\gll   see=yang lorang=susamma diinging muusing sanke-habbis anà-\textbf{simpang} \textbf{ambel}. \\
    `You have kept me together with you until the cold season was over.' (K070000wrt04a)
\z
} \\
In this case of a family providing shelter for a bear, it is the bear who profits, not the family, but still \em ambel \em is used in this sentence.

The following four examples show the use of \em ambel \em and \em kaasi \em for indicating self-benefactive \xref{ex:func:semrole:ben:ambel:proto} or alter-benefactive \xref{ex:func:semrole:ben:kaasi:proto}.


\xbox{16}{
\ea
\gll se=dang su-mirthi. \\
     \textsc{1s=dat} \textsc{past}-understand  \\
\z
} \\


\xbox{16}{
\ea\label{ex:func:semrole:ben:ambel:proto}
\gll se su-mirthi ambel. \\
     \textsc{1s} \textsc{past}-understand take  \\
\z
} \\

\xbox{16}{
\ea
\gll se su-mirthi-king. \\
     \textsc{1s} \textsc{past}-understand-\textsc{caus}  \\
\z
} \\
\xbox{16}{
\ea\label{ex:func:semrole:ben:kaasi:proto}
\gll se su mirthiking kaasi. \\
     \textsc{1s} \textsc{past}-understand-\textsc{caus} give  \\
\z
} \\


\subsubsection{Comitative}\label{sec:func:Comitative}
The comitative is coded by the postposition \em (=se) saama\em\formref{sec:morph:=sesaama}. Care must be taken to distinguish the comitative \em saama \em from the universal quantifier \trs{samma}{all}. This is especially important when the comitative is realized as \em sama \em or \em samma\em.

\xbox{16}{
\ea\label{ex:ptcpt:semrole:comit:intro}
\gll se=ppe mma-baapa=le=\textbf{sàsaama}=jo arà-duuduk. \\
 \textsc{1s=poss} mother-father=\textsc{addit}=\textsc{comit}=\textsc{foc} \textsc{non.past}-stay\\
\z
}


\em se \em in \em se saama \em does not necessarily refer to the speaker, even if the shape of \trs{see}{1s} would suggest that. The next example shows this, where \em sàsaama \em indicates the comitative role of Sindbad the Sailor (refered to by the pronoun \em incayang\em).


\xbox{16}{
\ea\label{ex:ptcpt:semrole:comit:3rdperson}
\ea
\gll soojer pada  \textbf{incayang=sàsaama} Seelon=nang asà-dhaathang. \\
     Europeans \textsc{pl} \textsc{3s}=\textsc{comit} Ceylon=\textsc{dat} \textsc{cp}-come \\
\ex
\gll inni daganan=yang derang=le        anà-blaajar. \\ % bf
	prox trade=\textsc{acc} \textsc{3pl}=\textsc{addit} \textsc{past}-learn\\
\z
\z
} \\
The use of the \em se\em-part is optional, as the following example shows.


\xbox{16}{
\ea\label{ex:ptcpt:semrole:comit:dropse}
\gll Soore=ka , Snow-white=le Rose-red=le derang=pe umma=\zero=\textbf{samma} appi dìkkath=ka arà-duuduk ambel.  \\
      Evening=\textsc{loc} Snow.white=\textsc{addit} Rose.Red=\textsc{addit} \textsc{3pl}=\textsc{poss} mother=\textsc{comit} fire vicinity=\textsc{loc} \textsc{simult}-sit take \\
\z
}\\

In a wider sense, the comitative can be used for parties engaged on different sides of a fight. While they do not fight together, they engage jointly in the act of fighting.

\xbox{16}{
\ea\label{ex:ptcpt:semrole:comit:fight}
\ea
\gll cinggala  raaja=nang=le          anà-banthu. \\ % bf
     Sinhala king=\textsc{dat}=\textsc{addit} \textsc{past}-help  \\
\ex
\gll mà-{\em fight}=nang       {\em British} \textbf{saama}. \\
    `to fight with the British.'
\ex
\gll {\em British} oorang pada=nang=le na-banthu. \\ % bf
     British man \textsc{pl}=\textsc{dat}=\textsc{addit}  \\
    `and they helped the British men'
\ex
\gll      mà-{\em fight}=nang cinggala  raaja \textbf{saama}. \\
      \textsc{inf}-fight=\textsc{dat} Sinhala king with \\
    `to fight with the Sinhala king.' (K051206nar04.11)
\z
\z
} \\
Very often, simple coordination with \em =le \em is used instead of the comitative.

\subsubsection{Purpose}\label{sec:func:Purpose}
Purpose is expressed by either an infinitive clause \xref{ex:ptcpt:semrole:purp:ma1}- \xref{ex:ptcpt:semrole:purp:ma3}, or the postposition \em =nang\em \xref{ex:ptcpt:semrole:purp:nang}, or a combination thereof\xref{ex:ptcpt:semrole:purp:ma+nang1}-\xref{ex:ptcpt:semrole:purp:ma+nang3}.


\xbox{16}{
\ea \label{ex:ptcpt:semrole:purp:ma1}
\gll Blaakang Andare [Kandi=ka asduuduk Dikwella  arà-pii jaalang]=ka aayer \textbf{ma}-miinong Udamalala kampong=ka su-birthi. \\
      after Andare Kandy=\textsc{loc} from Dikwella \textsc{non.past}-go road=\textsc{loc} water \textsc{inf}-drink Udamalala village=\textsc{loc} \textsc{past}-stop\\
\z
}\\

\xbox{16}{
\ea \label{ex:ptcpt:semrole:purp:ma2}
\gll de laaye hathu nigiri=nang anà-baapi, buunung-king=\textbf{nang}. \\
      3\textsc{s.impolite} other \textsc{indef} country=\textsc{dat} \textsc{past}-bring kill-\textsc{caus}=dat\\
\z
}\\


\xbox{16}{
\ea \label{ex:ptcpt:semrole:purp:ma3}
\gll  {\em freedom}=yang   \textbf{ma}-daapath=\textbf{nang}  kithang=nang   bannyak caape aada. \\
      freedomg=\textsc{acc} \textsc{inf}-get=\textsc{dat} \textsc{1pl}=\textsc{dat} much tired exist \\
\z
} \\
\xbox{16}{
\ea \label{ex:ptcpt:semrole:purp:nang}
\gll karang masiigith=nang  arà-pii  liima wakthu sbaayang=\textbf{nang}. \\
  now mosque=\textsc{dat} \textsc{non.past}-go five time pray=\textsc{dat}     \\
\z
} \\

\xbox{16}{
\ea \label{ex:ptcpt:semrole:purp:ma+nang1}
\gll  Hathu haari , hathu oorang thoppi \textbf{ma}-juwal=\textbf{nang} kampong=dering kampong=nang su-jaalang pii. \\
     \textsc{indef} day \textsc{indef} man hat \textsc{inf}-sell=\textsc{dat} village=\textsc{abl} village=\textsc{dat} \textsc{past}-walk go  \\
\z
}\\


\xbox{16}{
\ea \label{ex:ptcpt:semrole:purp:ma+nang2}
\gll derang [dìkkathka aada laapang]=nang \textbf{ma}-maayeng=\textbf{nang} su-pii. \\
     \textsc{3pl} vicinity=\textsc{loc} exist ground=\textsc{dat} \textsc{inf}-play=\textsc{dat} \textsc{past}-go  \\
\z
}\\

\xbox{16}{
\ea \label{ex:ptcpt:semrole:purp:ma+nang3}
\gll arà-blaajar mlaayu ini buk \textbf{ma}-thuulis kiyang. \\
      \textsc{non.past}-learn Malay \textsc{prox} book \textsc{inf}-write \textsc{evid} \\
\z
}\\

These strategies can be found in the same idiolects. The following example show the use of \em mà- \em with and without \em =nang \em in purposive clauses in one stretch of discourse by one speaker.

\xbox{16}{
\ea \label{ex:ptcpt:semrole:purp:double}
\ea
\gll Kandi=pe     raaja=nang   kitham=pe  inni     banthu-an  asà-kamauwan      se-aada. \\ % bf
     Kandy=\textsc{poss} king=\textsc{dat} \textsc{1pl}=\textsc{poss} \textsc{prox} help-nmlz \textsc{cp}-want \textsc{past}-exist  \\
\ex
\gll inni raaja=yang   \textbf{ma}-jaaga=\zero{}. \\
    \textsc{prox} king=\textsc{acc} \textsc{inf}-protect\\
\ex
\gll [itthu   \textbf{ma}-jaaga\textbf{=nang} anà-baa melayu]=dering   satthu oorang=jo    se. \\
 \textsc{dist} \textsc{inf}-protect=\textsc{dat} \textsc{past}-bring Malay]=\textsc{abl} one man=\textsc{foc} 1s\\
\z
\z
} \\
While purpose is most often marked on clauses, it is also possible to mark it on NPs, as is the case in the following example, where the noun \trs{thakuthan}{fear} hosts the postposition \em =nang\em.


\xbox{16}{
\ea \label{ex:ptcpt:semrole:purp:NP}
\gll  {\em second} {\em world} {\em war}  {\em time} ka  Kluumbu nang   {\em Japanese} arà-{\em bomb}-king [thakuth-an]=\textbf{nang}. \\
      second world war time=\textsc{loc} Colombo=\textsc{dat} Japanese \textsc{non.past}-bomb-cause fear-\textsc{nmlzr}=\textsc{dat} \\
\z
} \\


\subsubsection{Cause and reason}\label{sec:func:Causeandreason}
Cause is coded by the postposition \em subbath\em, which follows an NP based on a noun \xref{ex:ptcpt:semrole:caus:n}, a pronoun \xref{ex:ptcpt:semrole:caus:pron}, a deictic \xref{ex:ptcpt:semrole:caus:deic} or a  clause \xref{ex:ptcpt:semrole:caus:cl}.

\xbox{16}{
\ea \label{ex:ptcpt:semrole:caus:n}
\gll [non-muslims pada]=\textbf{subbath} kithang muuka konnyong arà-cunjikang siini. \\
      non-muslims \textsc{pl}=because \textsc{1pl} face little \textsc{non.past}-show here \\
    `Because of the non-muslims we do not wear the veil.' (K061026prs01)
\z
} \\
\xbox{16}{
\ea \label{ex:ptcpt:semrole:caus:pron}
\gll Lorang [see]=\textbf{subbath} ithu Aajuth=yang su-salba-king. \\
      \textsc{2pl} \textsc{1s}=because \textsc{dist} dwarf=\textsc{acc} \textsc{past}-safe-cause \\
\z
}\\

\xbox{16}{
\ea \label{ex:ptcpt:semrole:caus:deic}
\gll suda \textbf{itthusubbath}=jo,   se laile        {\em Marine} {\em Engineering}                asà-kijja ambel arà-pii. \\
   so therefore=\textsc{foc} \textsc{1s} again Marine Engineering \textsc{cp}-make take \textsc{non.past}-go    \\
\z
}\\


\xbox{16}{
\ea \label{ex:ptcpt:semrole:caus:cl}
\gll [Ini oorang giini kapang-jaalang pii caape]=\textbf{subbath} jaalang hathu piingir=ka anà-aada hathu pohong \textbf{baawa}=ka su seender. \\
    \textsc{prox} man this.way then-walk go tired=because road \textsc{indef} border=\textsc{loc} \textsc{past}-exist.inanim \textsc{indef} tree down=\textsc{loc} \textsc{past}-rest   \\
\z
}\\

Another possibility is the use of \trs{lanth(r)an}{because}, which can attach to a clause \xref{ex:ptcpt:semrole:reason:lanthran:clause}  or to an NP \xref{ex:ptcpt:semrole:reason:lanthran:NP}.


\xbox{16}{
\ea \label{ex:ptcpt:semrole:reason:lanthran:clause}
\gll derang hathu suurath nya-kiiring [see ini Kandi Mlaayu {\em Association}=dering nya-kiisar]=\textbf{lanthran}. \\
      \textsc{3pl} \textsc{indef} letter \textsc{past}-send \textsc{1s} \textsc{prox} Kandy Malay associaion=\textsc{abl} \textsc{past}-go.aside=\textbf{because} \\
\z
} \\
\xbox{16}{
\ea\label{ex:ptcpt:semrole:reason:lanthran:NP}
\ea
\gll beeso luusa lubaarang arà-dhaathang. \\ % bf
     tomorrow later.in.the.future festival \textsc{non.past}-come  \\
    `The day after tomorrow is the festival.'
\ex
\gll  itthu=\textbf{lanthran} kithang=pe ruuma see arà-cuuci. \\
       \textsc{dist}=because \textsc{1pl}=\textsc{poss} house \textsc{1s} \textsc{non.past}-clean\\
\z
\z
} \\

\subsubsection{Value}\label{sec:func:Value}
The value of an item is indicated by \em =nang\em. In \xref{ex:ptcpt:semrole:value}, the value of the stones indicated earlier in discourse is said to be \trs{bae lakuwan}{a good price}. This NP carries \em =nang\em.

\xbox{16}{
\ea\label{ex:ptcpt:semrole:value}
\gll [baaye  lakuwan]=\textbf{nang}    anà-juuwal. \\
     good wealth=\textsc{dat} \textsc{past}-sell  \\
\z
} \\


% \xbox{16}{
% \ea\label{ex:func:unreferenced}
% \gll See lorang=nang arà-simpa kaapang=ke see lorang=nang ithu uuthang arà-baayar katha. \\
%     \textsc{1s}  \textsc{2pl}=\textsc{dat} \textsc{non.past}-promise when=\textsc{simil} \textsc{1s} \textsc{2pl}=\textsc{dat} \textsc{dist} debt \textsc{non.past}-pay \textsc{quot} \\
% \z
% } \\

\subsubsection{Portion}\label{sec:func:Portion}
When something is divided into portions, the fraction is also indicated by \em =nang\em, as is the case in dividing meat into three portions in the example below

\xbox{16}{
\ea\label{ex:ptcpt:semrole:portion}
\gll thapi ithu \textbf{thiiga=nang} arà-baagi. \\
     but \textsc{dist} three=\textsc{dat} \textsc{non.past}-divide  \\
\z
} \\

\subsubsection{Set domain}\label{sec:func:Setdomain}

The superset among which a member is chosen is indicated in English by \em among\em. In SLM, this is either indicated by the locative \em =ka \em \xref{ex:ptcpt:semrole:setdomain:ka1}  or the ablative \em =dering \em \xref{ex:ptcpt:semrole:setdomain:dering1}-\xref{ex:ptcpt:semrole:setdomain:dering3}.


\xbox{16}{
\ea \label{ex:ptcpt:semrole:setdomain:ka1}
\gll  \textbf{mlaayu=ka}=jo  bannyak awuliya Seelong=\textbf{ka}   aada. \\
      Malay=\textsc{loc}=\textsc{foc} many saint Ceylon=\textsc{loc} exist \\
    `It is among the Malays that there are many saints in Sri Lanka.' (K060108nar02)
\z
} \\
\xbox{16}{
\ea \label{ex:ptcpt:semrole:setdomain:dering1}
\ea
\gll karang inni     {\em Kandy} nigiri=ka aada  mlaayu awuliya pada. \\ % bf
     now \textsc{prox} Kandy town=\textsc{loc} exist Malay saint \textsc{pl}  \\
    `Now there are Malay saints here in Kandy.'
\ex
\gll derang=\textbf{dering}  hatthu awuliya  dhaathang aada sini=ka dìkkath. \\
     \textsc{3pl}=\textsc{abl} \textsc{indef} saint come exist here=\textsc{loc} vicinity  \\
\z
\z
} \\
\xbox{16}{
\ea \label{ex:ptcpt:semrole:setdomain:dering2}
   \gll  [itthu    mà-jaaga=nang        anà-baa      mlaayu]=\textbf{dering}    satthu      oorang=jo    see. \\
 \textsc{dist} \textsc{inf}-protect=\textsc{dat} \textsc{past}-bring Malay=\textsc{abl}  one man=\textsc{foc} 1s\\
\z
}

\xbox{16}{
\ea \label{ex:ptcpt:semrole:setdomain:dering3}
\gll Seelong=pe  makanan=\textbf{dering}         aapa makanan        suuka? \\
     Ceylon food=\textsc{abl} what food like  \\
    `What food do you like within Sri Lankan cuisine?' (B060115cvs02)
\z
} \\
\subsubsection{Temporal domain}\label{sec:func:Temporaldomain}
The temporal domain in which an event occurs is indicated by \em =nang\em. In the following examples, this is the week and the day. There is usually some quantifying element elsewhere in the clause indicating the relation, like \trs{duuwa skalli}{two times}{} or \trs{empathblas kaayu}{fourteen miles}.

\xbox{16}{
\ea \label{ex:ptcpt:semrole:tempdomain1}
\gll suda \textbf{hathu}  \textbf{{\em week}=nang}   duuwa skali  arà-dhaathang    {\em daughter}. \\
     so one week=\textsc{dat} two time \textsc{non.past}-come daughter  \\
    `Thus my daughter comes twice a week.' (K051201nar01)
\z
} \\
\xbox{16}{
\ea \label{ex:ptcpt:semrole:tempdomain2}
\gll kithang \textbf{hathu}  \textbf{{\em week}=nang}  hathu skali duwa skaali=ke arà-maakang. \\
     \textsc{1pl} one week=\textsc{dat} one time two time=\textsc{simil} \textsc{non.past}-eat  \\
\z
} \\
\xbox{16}{
\ea \label{ex:ptcpt:semrole:tempdomain3}
\gll derang su-jaalang  empath-blas kaayu \textbf{hathu} \textbf{aari=nang}. \\
    `They walked fourteen miles a day.' (K051213nar03)
\z
} \\

\subsubsection{Duration}\label{sec:func:Duration}
Duration is normally not overtly marked. In the following two examples, the time spans \trs{thuuju thaaun}{seven years}{} and \trs{spuulu thaaun}{ten years}{} do not receive any special marking

\xbox{16}{
\ea \label{ex:ptcpt:semrole:duration1}
\gll  see  [thuuju thaaun]=\zero{} luwar nigiri=ka asà-duuduk  karang abbis dhaathang aada. \\
      \textsc{1s} seven year outside country=\textsc{loc} \textsc{cp}-exist.\textsc{anim} now finish coming exist \\
\z
} \\

\xbox{16}{
\ea \label{ex:ptcpt:semrole:duration2}
\gll see [spuulu thaaun]=\zero{} siini sri Lanka=ka pukurjan nya-kirja. \\
     \textsc{1s} ten year here Sri Lanka=\textsc{loc} work \textsc{past}-do  \\
    `I  worked here in Sri Lanka for ten years.' (K061026prs01)
\z
} \\
If the duration shall be emphasized, \em =dering \em can be used.


\xbox{16}{
\ea \label{ex:ptcpt:semrole:duration:emph}
\gll [Bannyak aari]=\textbf{dering} saapa=yang=ke thàrà-enco-kang. \\
      many day=\textsc{abl} who=\textsc{acc}=\textsc{simil} \textsc{neg.past}-fool-cause \\
\z
} \\
\subsubsection{Role???}\label{sec:func:Role???}

\xbox{16}{
\ea\label{ex:func:unreferenced}
\gll  Seelong=nang  {\em exiles} \textbf{caara} nya-kiiring. \\
    `The soldiers were sent to Sri Lanka as exiles.' (K051213nar06)
\z
} \\

\subsubsection{Manner???}\label{sec:func:Manner???}

\xbox{16}{
\ea
\gll   Itthu    wakthu=ka    hathu  bìssar beecek caaya  Buruan mlaarath=\textbf{ka}     uuthang=dering  luwar=nang     su-dhaathang. \\
      \textsc{dist} time=\textsc{loc} \textsc{indef} big brown colour bear difficulty=\textsc{loc} forest=\textsc{abl} outside=\textsc{dat} \textsc{past}-come \\
\z
} \\
\subsubsection{Summary}\label{sec:func:Summary}
Table \ref{tab:semroles} gives an overview of the distribution of semantic roles on different morphemes


\begin{table}[h]
\begin{center}
% use packages: array
\begin{tabular}{cccccc}
sduuduk		&	=dering 	&	=\zero{} 	&	=yang	& 	=nang 	& 	=ka 		\\
		&\framebox[2cm]{INSTR}	&  \multicolumn{3}{c}{\framebox[6.8cm]{PAT}} 		&\framebox[2cm]{LOC}	\\
		&\multicolumn{2}{c}{\framebox[3.85cm]{AGENT}}	&  		&\multicolumn{2}{c}{\framebox[4.4cm]{GOAL}}\\
\multicolumn{2}{c}{\framebox[3.85cm]{SRC}}&\multicolumn{2}{c}{\framebox[4.4cm]{THEME}}& \framebox[2cm]{REC}  		\\
		&\multicolumn{2}{c}{\framebox[3.85cm]{TEMP DOMAIN}}&		& \framebox[2cm]{BEN}	 		\\
		&\framebox[2cm]{PATH} 	&			& 		& \framebox[2cm]{PURP}  		\\
		&\framebox[2cm]{~SET DOM.\begin{picture}(0,0)(0,0)
					\put(1,6){\line(1,0){218}}
					\put(1,4){\line(1,0){218}}
					\end{picture}
					}&			&		& 		&\framebox[2cm]{SET DOM.}\\
		&			&			& 		& \framebox[2cm]{EXP}    		\\
		&			&			& 		& \framebox[2cm]{PORTION} 		\\
		&			&			& 		& \framebox[2cm]{VALUE} 		\\
\end{tabular}
\end{center}
\caption[Repartition of semantic roles on morphemes]{Repartition of semantic roles on morphemes. \em =nang \em is used to express eight different semantic roles, \em =dering \em is used for six, while the other morphemes have lower number of semantic roles they can express. Six semantic roles can be expressed by more than one morpheme. The two morphemes which can be used for the role \textsc{set domain} are linked by a line because for typographical reasons they could not be made contiguous. The semantic roles of \textsc{comitative} and \textsc{reason} are left out of the graphic.}
\label{tab:semroles}
\end{table}



\subsection{Mismatches between number of semantic roles and number of syntactic arguments}\label{sec:func:Mismatchesbetweennumberofsemanticrolesandnumberofsyntacticarguments}
It is possible that the predicate demands more semantic roles than there are participants in the discourse. This is most notably the case for reflexives and reciprocals. The opposite is that a semantic role applies to more than one participant. In that case, the participants are coordinated.


\subsubsection{Reciprocals}\label{sec:func:Reciprocals}
In a reciprocal situation, the participants taking part in the reciprocity are both actor and undergoer (or recipient) and their double involvement in the event is indicated by two occurrences of the indefiniteness marker \em hatthu \em \formref{sec:morph:Indefinitenessclitic}. The following three examples show this for nominal predications.

\xbox{16}{
\ea\label{ex:func:reciproc:muhabbath}
\gll  oorang pada hatthunang hatthu muhabbath. \\
      man \textsc{pl} \textsc{indef}=\textsc{dat} \textsc{indef} love \\
\z
} \\
\xbox{16}{
\ea\label{ex:func:reciproc:maara}
\gll  oorang pada hatthunang hatthu maara. \\
      man \textsc{pl} \textsc{indef}=\textsc{dat} \textsc{indef} love \\
\z
} \\

\xbox{16}{
\ea\label{ex:func:reciproc:precaaya}
\gll  oorang pada hatthunang hatthu buthul percaaya. \\
     man \textsc{pl} \textsc{indef}=\textsc{dat} \textsc{indef} correct trust  \\
\z
} \\
With verbal predications, the vector verb \em ambel \em can also be used to convey reciprocal semantics \xref{ex:func:reciproc:ambel1}\xref{ex:func:reciproc:ambel2}.


\xbox{16}{
\ea\label{ex:func:reciproc:ambel1}
\gll  ini  nigiri=pe oorang pada bannyak arà-buunung \textbf{ambel}. \\
      \textsc{prox} country=\textsc{poss} man \textsc{pl} much \textsc{non.past}-kill take \\
    `The people in this country kill each other.' (nosource)6.11.08
\z
} \\
\xbox{16}{
\ea\label{ex:func:reciproc:ambel2}
\gll  Faarokle Oomarle su buunung \textbf{ambel}. \\
      Faarok=\textsc{addit} Oomar=\textsc{addit}  kill take \\
    `Farook and Oomar killed each other.' (nosource)6.11.08
\z
} \\
Both possibilities (\em hatthunang hatthu \em and \em ambel \em) can be combined \xref{ex:func:reciproc:ambel1nang1}.





\xbox{16}{
\ea\label{ex:func:reciproc:ambel1nang1}
\gll  oorang pada \textbf{hatthu=nang} \textbf{hatthu} arà-ciong \textbf{ambel}. \\
    `People kiss each other.' (nosource)6.11.08
\z
} \\

The verb \trs{ciong}{kiss} in \xref{ex:func:reciproc:ambel1nang1} normally governs the dative, which aligns nicely with the dative case marker \em =nang \em found on \em hatthu \em in \xref{ex:func:reciproc:ambel1nang1}. However, even verbs which normally govern the accusative, like \trs{buunung}{kill} in \xref{ex:func:reciproc:ambel1nang:acc} take \em hatthunang hatthu \em in the reciprocal construction, and not \em *hatthuyang hatthu\em, with the accusative marker \em =yang \em instead of \em =nang\em.

\xbox{16}{
\ea\label{ex:func:reciproc:ambel1nang:acc}
\gll  oorang pada hatthu=\textbf{nang} hatthu arà-buunung ambel. \\
      man \textsc{pl} \textsc{indef}=\textsc{dat} \textsc{indef} \textsc{non.past}-kill take \\
\z
} \\

If through the use of  a modal the set of entities performing the reciprocal action gets case marked, as in \xref{ex:func:reciproc:ambel1nang:acc:boole}, the verb returns to assigning its normal role, accusative in \xref{ex:func:reciproc:ambel1nang:acc:boole}.

\xbox{16}{
\ea\label{ex:func:reciproc:ambel1nang:acc:boole}
\gll  oorang pada=nang hatthu (*nang) hatthu=yang bole-buunung. \\
      man \textsc{pl}=\textsc{dat} \textsc{indef} \textsc{dat} \textsc{indef}=\textsc{acc} can-kill \\
\z
} \\


If the reciprocal involves other roles than actor and undergoer, such as recipient in \xref{ex:func:reciproc:ambel1nang:rec}, the position of the case marker is on the second item, not on the first as above in \xref{ex:func:reciproc:ambel1nang1} and \xref{ex:func:reciproc:ambel1nang:acc}.

\xbox{16}{
\ea\label{ex:func:reciproc:ambel1nang:rec}
\gll oorang pada hatthu=\zero{} hatthu oorang=\textbf{nang} duwith arà-\textbf{kaasi} \textbf{ambel}. \\
    `People give money to each other.' (nosource)6.11.08
\z
} \\
Note the occurence of the vector verb \trs{ambel}{take} together with the full verb \trs{kaasi}{give  in \xref{ex:func:reciproc:ambel1nang:rec}}. These normally have opposite semantics, but the semantic content of \em ambel \em is bleached in this construction, and it fulfills a grammatical function instead of contributing semantic content to the propositions.

If the case marker which would be used in a non-reciprocal clause is neither zero nor \em =yang \em nor \em =nang\em, it is omitted in the reciprocal construction. The verb \trs{thaanya}{ask} normally assigns the locative \em =ka \em to the askee. This is not possible in the reciprocal construction, and hence there is no case marker found in \xref{ex:func:reciproc:ambel1nang:loc}.

\xbox{16}{
\ea\label{ex:func:reciproc:ambel1nang:loc}
\gll oorang pada hatthu hatthu arà-thaanya ambel. \\
     man \textsc{pl} \textsc{indef} \textsc{indef} \textsc{non.past}-ask take  \\
\z
} \\
An exception to this rule might be the ablative case marker \em =dering\em, which seems to be able to be used prenominally in a reciprocal construction. This is very suprising, given that it is normally a \em post\em position. This example should be taken with a grain of salt, but might be a worhtwhile starting point for future investigations of reciprocity.

\xbox{16}{
\ea\label{ex:func:reciproc:ambel1nang:dering}
\gll (dering) oorang pada hatthu hatthu duwith are mintha ambel. \\
     \textsc{abl} man \textsc{pl} \textsc{indef} \textsc{indef} money \textsc{non.past}-ask take  \\
\z
} \\
The reciprocal must be distinguished from joint reflexive action as in \xref{ex:func:reciproc:contr:refl}.

\xbox{16}{
\ea\label{ex:func:reciproc:contr:refl}
\gll Farook=le Oomar=le derang derang=yang su-buunung ambel. \\
      Farook=\textsc{addit} Oomar=\textsc{addit} 3 3=\textsc{acc} \textsc{past}-kill take \\
\z
} \\


% \xbox{16}{
% \ea
% \gll ini nigiripe oorang pada butthul saayang sukahaannang ara baapi haari. \\
%        \\
%     `.' (nosource)6.11.08
% \z
% } \\


%
% \xbox{16}{
% \ea
% \gll Farookle Oomarle derampe saala deringjo hatthu {\em car} {\em accident} asa jaadi sumniinggal. \\
%        \\
%     `Farook and Oomar died in a car accident due to their own fault.' (nosource)6.11.08
% \z
% } \\




% \xbox{16}{
% \ea
% \gll oorang pada hatthu hatthuyang ara buunung. \\
%        \\
%     `.' (nosource)
% \z
% } \\

\subsubsection{Reflexive}\label{sec:func:Reflexive}
If only one term is both \textsc{actor} and \textsc{undergoer}, there exist three possibilities:  Use the vector verb \em ambel\em, use the focus clitic \em =jo\em, separate the term into two non-identical subterms.

The first possibility is to use the vector verb \em ambel \em to express reflexive action, as in \xref{ex:func:refl:ambel:buunung} and \xref{ex:func:refl:ambel:banthu}. The case assigned by the main verb remains unaffected by this (accusative by \trs{buunung}{kill}, dative by \trs{banthu}{help}). The focus marker can optionally be used.


\xbox{16}{
\ea\label{ex:func:refl:ambel:buunung}
\gll incayang incayang=\textbf{yang}(=jo) su-buunung \textbf{ambel}. \\
      3s.polite 3s.polite=\textsc{acc}=\textsc{foc} \textsc{past}-kill take \\
\z
} \\

\xbox{16}{
\ea\label{ex:func:refl:ambel:banthu}
\gll incayang incayang=\textbf{nang}=(jo) su-banthu \textbf{ambel}. \\
      3s.polite 3s.polite=\textsc{acc}=\textsc{foc} \textsc{past}-kill help \\
\z
} \\

In some possessive contexts, the use of \em ambel \em is not possible. In those cases, \em =jo \em is obligatory. The focus clitic can attach either on the possessor or on the possesse, entailing slight changes in meaning, rendered by `he himself' and `his own' in English.



\xbox{16}{
\ea
\gll \textbf{incayang=pe=jo} ruuma=yang incayang su-ronthok-king. \\
      3s.polite=\textsc{poss}=\textsc{foc} house=\textsc{acc} 3s.polite \textsc{past}-demolished.caus \\
    `He demolished his own house.' (nosource)6.11.08
\z
} \\

\xbox{16}{
\ea
\gll incayang=pe ruuma=yang \textbf{incayang=jo} su-ronthok-king. \\
     3s.polite house=\textsc{acc} 3s.polite=\textsc{foc} \textsc{past}-demolished-\textsc{caus}  \\
\z
} \\


The third possibility is to use an additional term to express the undergoer in a more precise manner, thus making it different from a real reflexive construction. The additional term is normally in a meronymic relationship to the main term, like \trs{diiri}{body} to the speaker in \xref{ex:func:refl:diiri:yang} and \xref{ex:func:refl:diiri:nang}. The normal case markers are used in this construction as assigned by the verb.
Even in this case, \em ambel \em and \em =jo \em are normally present.


\xbox{16}{
\ea\label{ex:func:refl:diiri:yang}
\gll se se=ppe diiri yang(=jo) su-poothong ambel. \\
     \textsc{1s} \textsc{1s=poss} body=acc(=foc) past cut take  \\
\z
} \\

\xbox{16}{
\ea\label{ex:func:refl:diiri:nang}
\gll se se=ppe diiri nang(=jo) su-puukul ambel. \\
     \textsc{1s} \textsc{1s=poss} body \textsc{dat}=\textsc{foc} \textsc{past}-hit take  \\
\z
} \\


% \xbox{16}{
% \ea
% \gll incayang pe haal nang incayang su dhaathang. \\
%        \\
%     `He came by his own wish.' (nosource)6.11.08
% \z
% } \\




\subsubsection{Self-benefactive}\label{sec:func:Self-benefactive}
It is possible that an agent performs an action of which he himselfs benefits. In English, a sentence like \em John bought himself a car \em comes close to this, since \em John bought a car \em normally already pragmatically implies, in absence of other context, that John will be the owner of the car. This fact can be highlighted by \em himself \em in English. In SLM, a different construction is used for actions which profit the agent, namely the vector verb \em ambel\em.


\xbox{16}{
\ea\label{ex:ptcpt:mismatch:selfben:ambel}
\gll [Tony Hassan {\em uncle}=nang asà-kaasi duwith] athi-mintha \textbf{ambel}=si? \\ % bf
     Tony Hassan uncel=\textsc{dat} \textsc{cp}-give money \textsc{irr}-ask take=\textsc{interr}  \\
\z
} \\
In example \xref{ex:ptcpt:mismatch:selfben:ambel}, the use without \em ambel \em would be perfectly fine, but the use with \em ambel \em highlights the fact that the speaker would perform an action which would profit himself. It is difficult to argue for this example that it changes the valency of the verb \trs{mintha}{ask}. The verb is still trivalent, but the emphasis is more on the asker than on the askee or the theme.

Note that overt coding of the self-benefactive is optional, as shown by the following example without \em ambel\em.


\xbox{16}{
\ea\label{ex:ptcpt:mismatch:selfben:noambel}
\gll   derang pada arà-mintha \zero{}   nigiri. \\ % bf
      \textsc{3pl} \textsc{pl} \textsc{non.past}-ask { } country \\
\z
} \\

%
% \xbox{16}{
% \ea
% \gll se se=dang(jo) hatthu kaar su bìlli (ambel). \\
%        \\
%     `.' (nosource)6.11.08
% \z
% } \\
%
%
% \xbox{16}{
% \ea
% \gll  seeyang {\em university} ka su ambel bìrrath. \\
%        \\
%     `.' (nosource)6.11.08
% \z
% } \\
\em Ambel \em can only be used if the beneficiary is actively involved. For \trs{daapath}{find} in \xref{ex:ptcpt:mismatch:selfben:daapath}, this is not the case. From the context, we do not learn that the speaker was actively involved in getting a new job, hence \em ambel \em cannot be used.

\xbox{16}{
\ea\label{ex:ptcpt:mismatch:selfben:daapath}
\gll se=dang baaru hatthu idopan su-daapath (*ambel). \\
    \textsc{1s=dat} new \textsc{indef} job \textsc{past}-get take   \\
    `I got a new job.' (nosource)6.11.08
\z
} \\
If the speaker has found a new job after being actively involved in finding it, \em ambel \em is possible \xref{ex:ptcpt:mismatch:selfben:sucaariambel}.


\xbox{16}{
\ea\label{ex:ptcpt:mismatch:selfben:sucaariambel}
\gll se baaru hatthu idopan su-caari  ambel. \\
     \textsc{1s} new \textsc{indef} job \textsc{past}-find take  \\
    `I found a new job.' (nosource)6.11.08
\z
} \\
On the other hand, if the process of finding a job is not finished yet, \em ambel \em cannot be used. In \xref{ex:ptcpt:mismatch:selfben:aracaariambel}, we find the non-past marker \em arà-\em, indicating that at the time of speaking the process of finding/searching was still ongoing. The job has not been found yet, hence no benefit is materialized, and \em ambel \em cannot be used.\footnote{The difference between searching and finding is not marked lexically in SLM, but only aspectually. If searching is completed, this implies finding.}

\xbox{16}{
\ea\label{ex:ptcpt:mismatch:selfben:aracaariambel}
\gll se baaru hatthu idopan arà-caari (*ambel). \\
     \textsc{1s} new \textsc{indef} job non.\textsc{past}-find take  \\
    `I am looking for a new job.' (nosource)6.11.08
\z
} \\



\subsubsection{More than one entity in a term}\label{sec:func:Morethanoneentityinaterm}
If a term consists of more than one entity, these entities are conjoined by a semantically appropriate coordinating construction\formref{sec:constr:Coordinatingconstructions}. The zero coordination \xref{ex:ptcpt:mismatch:coord:zero} and the coordination with a clitic \xref{ex:ptcpt:mismatch:coord:le} are shown below.

%
% \xbox{16}{
% \ea\label{ex:func:unreferenced}
% \gll se=ppe umma\textbf{=le}  aade\textbf{=le} aade=pe duuwa aanak\textbf{=le} kitham=pe baapa=pe aade\textbf{=le} baapape aade\textbf{le} aanak\textbf{=le} thiiga aanak\textbf{=le}, bannyak aanak pada anà-niinggal. \\
% \textsc{1s=poss} mother=\textsc{addit} younger.sibling=\textsc{addit} younger.sibling=\textsc{poss} two child=\textsc{addit} \textsc{1pl}=\textsc{poss} father=\textsc{poss} younger.sibling=\textsc{addit} father=\textsc{poss} younger.sibling=\textsc{addit} child=\textsc{addit} three child=\textsc{addit} many child \textsc{pl} \textsc{past}-die\\
% \z
% }



\xbox{16}{
\ea\label{ex:ptcpt:mismatch:coord:zero}
\ea
\gll mlaayu pada  duuduk=apa. \\ % bf
 Malay \textsc{pl} stay=after\\
 `After the Malays had settled down'
\ex
 \gll spaaru  mlaayu pada   \textbf{singapur=\zero}       \textbf{indonesia=\zero}  \textbf{{\em Malaysia}=\zero} anà-pii. \\
 some Malay \textsc{pl} Singapur Indonesia Malaysia \textsc{past}-go\\
` (only) some Malays went (back) to Singapur, Indonesia or Malaysia.' (K051213nar07.22)
\z
\z
}


\xbox{16}{
\ea\label{ex:ptcpt:mismatch:coord:le}
\gll Snow-white=nang=\textbf{le} Rose-red=nang=\textbf{le} ini hatthu=ke thàrà-mirthi. \\
     Snow.white=\textsc{dat}=\textsc{addit} Rose.Red=\textsc{dat}=\textsc{addit}  \textsc{prox} \textsc{indef}=\textsc{simil} \textsc{neg.past}-understand\\
\z
 }\\

%
% \xbox{16}{
% \ea\label{ex:func:unreferenced}
% \gll  Derang=pe umma derang=nang jaith-an=le, jaarong pukurjan=le su-aajar. \\
%        \textsc{3pl}=\textsc{poss} mother \textsc{3pl}=\textsc{dat} sew-\textsc{nmlzr}=\textsc{addit} needle work=\textsc{addit} \textsc{past}-teach\\
% \z
% }\\
%
%
% \xbox{16}{
% \ea\label{ex:func:unreferenced}
% \gll Lorang se=dang mà-iidop thumpath kala-kaasi see lorang=nang  lorang=pe samma duwith=le baarang pada=le anthi-bale-king. \\
%        \textsc{2pl} \textsc{1s=dat} \textsc{inf}-stay place if-give \textsc{1s} \textsc{2pl}=\textsc{dat} \textsc{2pl}=\textsc{poss} all money=\textsc{addit} goods   \textsc{pl}=\textsc{addit} \textsc{irr}=return-caus\\
% \z
% } \\
% \xbox{16}{
% \ea\label{ex:func:unreferenced}
% \gll ketham pada     makanan pada\textbf{=si }    pakeyan   pada\textbf{=si } su-baawang. \\
%  \textsc{1pl} \textsc{pl} food \textsc{pl}=disj clothing \textsc{pl}=disj \textsc{past}-bring\\
% \z
% }
%
%
% \xbox{16}{
% \ea\label{ex:func:unreferenced}
% \gll {\em doctors} pada=so {\em police} a.s.p=so  {\em judge}=so, samma oorang thaau see=yang \\
%      doctors \textsc{pl}=\textsc{undet} police a.s.p=\textsc{undet} judge=\textsc{undet} all man know \textsc{1s}=\textsc{acc}  \\
%     `Whether they be doctors, police a.s.p.s or judges, all men know me.'  (B060115nar04)
% \z
% }\\
%
% \xbox{16}{
% \ea\label{ex:func:unreferenced}
% \gll saudi=so, mana=ka=so; athu nigiri=ka. \\
%      Saudi.Arabia=\textsc{undet} where=\textsc{loc}=\textsc{undet} \textsc{indef}=country=\textsc{loc}  \\
% \z
% }\\
%
%
% \xbox{16}{
% \ea\label{ex:func:unreferenced}
% \gll suda go buthul baaye=nang bole=thaau, mà-jaaith=so mà-poothong=so. \\
%      thus \textsc{1s} correct good=\textsc{dat} can=know \textsc{inf}-sew=\textsc{undet} \textsc{inf}-cut=\textsc{undet}\\
% \z
% }\\

\subsection{Unknown participants}\label{sec:func:Unknownparticipants}
If a predicate semantically requires an argument, but the precise referential nature of this argument is not known, there are several possibilities: the referent is not important, the referent is important and individuated, the referent is important and categorial, the referent is important and generic. A someone different possibility is that the speaker wants the hearer to provide the referent, by forming a question.

\subsubsection{Unimportant referent}\label{sec:func:Unimportantreferent}


Unimportant referents which might be required semantically do not have to be expressed in morphosyntax.
In example \xref{ex:ptcpt:unknown:drop1}, the author of the history book is not known and irrelevant. Therefore, no reference to the author is made.\footnote{It would be possible to read this sentence as `A Sinhala history book has written something', but since books rarely engage in activities such as writing, no hearer will seriously be tempted by this interpretation.}

\xbox{16}{
\ea \label{ex:ptcpt:unknown:drop1}
\gll \zero{} cinggala {\em history} {\em book} atthu thuulis aada. \\ % bf
 {} sinhala history book atthu written exist\\
`There is a Sinhala history book written (about that).' (K051213nar06)
\z
}



\subsubsection{Individuated referent}\label{sec:func:Individuatedreferent}
If the referent is important and  individuated, but further information is not available (like English \em somebody, something\em), the WH=\em so\em-construction is used to yield an indefinite pronoun.

\xbox{16}{
\ea\label{ex:func:ptcpt:unknown:indiv:saapa}
\gll  \textbf{saapa=so} {\em Malay} {\em exam} arà-girja. \\
      who=\textsc{addit} Malay exam \textsc{non.past}-make \\
    `Someone was taking a Malay exam.' (K060103nar01)
\z
} \\
\xbox{16}{
\ea\label{ex:func:ptcpt:unknown:indiv:aapacara}
\gll Thapi \textbf{aapacara=so} itthu samma asà-iilang su-aada. \\
     But how=\textsc{undet} \textsc{dist} all \textsc{cp}-disappeared \textsc{past}-exist  \\
\z
}\\

% K061019sng01.trs:dhaathang thurus police oorang
% K061019sng01.trs:inni aapa katha biilang aanak pe saala peegang saapa so biilang aada nang
%
%
%
% \xbox{16}{
% \ea\label{ex:func:unreferenced}
% \gll uuthang=ka asà-pii apa see picakang, ithu aapa=so, ithu daawong pada. \\
%       jungle=\textsc{loc} \textsc{cp}-go after \textsc{1s} break \textsc{dist} what=\textsc{undet} \textsc{dist} leaf \textsc{pl} \\
% \z
% } \\

\subsubsection{Unknown categorial referents}\label{sec:func:Unknowncategorialreferents}
In distinction to individuated referents, which are instantiated by a specific entity in discourse, categorial referents do not refer to a specific entity, but to any member of the indicated category.

In the following example, Andare wants to be dressed like a king. In this case, this does not refer to a specific king, but to any member of the class of kings. This is indicated by the indefinite article \em hathu\em.


\xbox{16}{
\ea\label{ex:func:ptcpt:unknown:categor:hatthu}
\gll [Andare kanabisan=nang anà-mintha] [\textbf{hathu} raaja=ke asà-paake=apa kampong=nang mà-pii maau katha]. \\
    Andare last=\textsc{dat} \textsc{past}-ask \textsc{indef} king=\textsc{simil} \textsc{cp}-dress=after village=\textsc{dat} \textsc{inf}-go want \textsc{quot}   \\
\z
}\\


An example where the categorial nature is emphasized would be  English   \em Do whatever you like\em, where \em whatever \em does not refer to an individuated referent, but to any member of the category of liked things that the hearer wants to instantiate the referent with. These unknown categorial referents are formed with the WH\~ WH ... \em=so \em  construction in SLM \formref{sec:nppp:NPscontaininginterrogativepronounsusedforuniversalquantification}. An example is
\xref{ex:func:ptcpt:unknown:categorial}, where a predication is made about the category of people who have a Malay dress. The precise referents are unknown to the speakers. They are not individuated nor specific. Any member of the set of possessors of Malay dresses is invited to wear them at the wedding.

\xbox{16}{
\ea\label{ex:func:ptcpt:unknown:categorial}
\gll       \textbf{saapa}\Tilde\textbf{saapa}=ka inni  mlaayu pakeyan pada aada\textbf{=so}, lorang  pada ini       mlaayu  pakeyan samma ini       kaving=nang mà-dhaathang    bannyak uthaama. \\
 who\Tilde who=\textsc{loc} \textsc{prox} Malay dress \textsc{pl} exist=disj \textsc{2pl} \textsc{pl} \textsc{prox} Malay dress with \textsc{prox} wedding=\textsc{dat} \textsc{inf}-come much \\
\z
}


\subsubsection{Unknown referents, generic}\label{sec:func:Unknownreferents,generic}
The last possibility for unknown referents is to be generic, i.e. no particular referent is intended, but the predicate is thought to be true of any referent. An example in English would be \em one \em as in \em one has to be kind to strangers \em or generic \em you \em as in \em you must obey the law\em. In SLM, generic reference is not expressed overtly. An example is the \xref{ex:ptcpt:unknown:generic} about the general rules of Sepaktakraw. While in the SLM example, no referent is expressed, in English the use of \em you \em is mandatory, alternatively the use of \em one \em or a passive construction.


\xbox{16}{
\ea \label{ex:ptcpt:unknown:generic}
\gll \zero{} kaaki=dering masthi-maayeng. \\
     { } leg=\textsc{abl}  must-play\\
    `You have to play with your feet.' (N060113nar05)
\z
} \\
\subsubsection{Queried referents}\label{sec:func:Queriedreferents}
Referents are queried for by replacing them with the interrogative pronoun   \formref{sec:wc:Interrogativepronouns} which corresponds to the semantic category of the referent (\trs{aapa}{what}, \trs{mana}{where}, \trs{kaapang}{when}, etc), plus postpositions if needed for a more precise indication of semantic role (\trs{aapa=nang}{for what}, \trs{mana=dering}{from where}, etc). Referents are divided into humans, which are queried for by \em saapa \em and others, which are queried for by \em aapa\em, unless a more precise interrogative pronoun is available.

\subsection{Modifying participants}\label{sec:func:Modifyingparticipants}
In the preceding sections we have seen how participants are encoded in SLM. In this section, we will see how participants can be modified. This can be done in various number of ways: modifications pertaining to quality, such as size or colour, quantitative modications, indicating a possessor or a location. Temporal modification of a partcipant is also possible, an English example would be \em ex-wife\em.

We will discuss these different types of modifications in turn in this section, identifying which constructions are used for them and which kinds of participants can be modified in such a way.  As a general principle, nouns are much more accessible to modification than pronouns or even clauses.

% In general, modifiers of participants are found on the left of their head word, which contrasts sharply with general Austronesian typology as described by \citet[141]{Himmelmann2005typochar}: ``Adpositions are generally prepositions in western Austronesian languages \el Auxiliaries generally precede main verbs. Negators also generally precede the negated constituent \el possessors generally follow the possessum \el numbers generally follow the head noun \el otherwise, adnominal modifiers generally follow the head, with demonstratives being placed at the very end of an NP.

\subsubsection{Quality}\label{sec:func:Quality}
Participants can be modified by indicating their quality. This can either refer to a property they have (with dimension being a special case), or by indicating their similarity to some other entity.


\paragraph{Property}
Indication of property can done by either a noun or an adjective, which can be either pre- or postposed, or a relative clause, which can only be preposed \formref{sec:nppp:Thefinalstructureofthenounphrase}. Examples \xref{ex:ptcpt:mod:qual:nom:pre} and \xref{ex:ptcpt:mod:qual:nom:post} show modification by pre- and postposed nouns, while examples \xref{ex:ptcpt:mod:qual:adj:pre} and \xref{ex:ptcpt:mod:qual:adj:post} show the same for pre- and postposed adjectives.


\xbox{16}{
\ea \label{ex:ptcpt:mod:qual:nom:pre}
\gll {\em the}$\curvearrowright$ pohong. \\ % bf
 tea tree\\
`tea tree.'
\z
}

\xbox{16}{
\ea \label{ex:ptcpt:mod:qual:nom:post}
\gll orang  $\curvearrowleft$ikkang. \\ % bf
 man fish\\
`fisherman'
\z
}

\xbox{16}{
\ea \label{ex:ptcpt:mod:qual:adj:pre}
\gll \textbf{baaru}$^\curvearrowright$ \textbf{oorang} pada massa-thaaro. \\
 new man \textsc{pl} must put\\
` (We) must put new people.' (K060116nar11)
\z
}

\xbox{16}{
\ea \label{ex:ptcpt:mod:qual:adj:post}
\gll se=ppe       \textbf{oorang} $^\curvearrowleft$\textbf{thuuwa} pada    anà-biilang [kitham pada {\em Malaysia}=dering    anà-dhaathang    katha]. \\
 \textsc{1s=poss} man old \textsc{pl} \textsc{past}-say \textsc{1pl} \textsc{pl} Malaysia=\textsc{abl} \textsc{past}-come quot\\
\z
}

The adjectival predications may be internally complex, as the negated word for `good' in \xref{ex:ptcpt:mod:qual:adj:compl}.


\xbox{16}{
\ea \label{ex:ptcpt:mod:qual:adj:compl}
\gll go  \textbf{thàrà-baaye}   pukujan  thama-gijja. \\
     \textsc{1s} \textsc{neg}-good work \textsc{neg.irr}-do  \\
\z
} \\

Dimension is expressed by the adjectives \trs{bìssar}{big} and \trs{kiccil}{small}. There are no diminutives or augmentatives.

\xbox{16}{
\ea \label{ex:ptcpt:mod:qual:dimension}
\ea
\gll \textbf{bìssar} atthu  kumpulan    thraa. \\
good one association neg\\
`There is no big association.'
\ex
\gll \textbf{kiccil} kumpulan    pada=jo. \\
 small association \textsc{pl}=foc\\
`The associations are small indeed.' (N060113nar01.58)
\z
\z
}

% \xbox{16}{
% \ea\label{ex:func:unreferenced}
% \gll deram  pada arà-duuduk    konnyom konnyom \textbf{kiccil} \textbf{kiccil} kawanang=ka. \\
% `They live in few small groups.' (N060113nar01.52)
% \z
% }


Colour is expressed by a colour adjective, which is often found combined with the noun \trs{caaya}{colour}.\footnote{This is an influence from the adstrates, where colour terms need to be supported by the noun for `colour', cf. the words for `blue' in Sinhala (\em nil paa\tz a\em) and Tamil (\em niil ni\textsubbar{r}am \em), where \em ni(i)l \em means `blue' and the other word means `colour'.}


\xbox{16}{
\ea \label{ex:ptcpt:mod:qual:colour}
\gll Hatthu komplok bannyak=jo \textbf{puuthi} \textbf{caaya}, hathyeng=yang \textbf{meera}=jo \textbf{meera} \textbf{caaya}. \\
    `One bush was very white, the other one was of the reddest red.'  (K070000wrt04)
\z
}\\

Some colour terms are not adjectives, but nouns, as is the case for the word for `brown' which is derived from the word \trs{beecek}{mud}.

\xbox{16}{
\ea \label{ex:ptcpt:mod:qual:colour:noun}
\gll   Sithu=ka, hathu bìssar \textbf{beecek} \textbf{caaya} Buruan su-duuduk.\\
    `There was a big brown bear.' (K070000wrt04a)
\z
} \\
Table \ref{tab:func:colourterms} gives the terms for common colours.

\begin{table}
\begin{center}
% use packages: array
\begin{tabular}{llllll}
SLM & gloss  & original meaning & SLM  & gloss  & original meaning \\
iitham & black & black & iijong & green & green \\
puuthi & white & white & kuunyith & yellow & turmeric\footnotemark \\
meera & red & red & beecek & brown & mud \\
niila & blue & blue &  &  &  \\
\end{tabular}
\end{center}
\caption[Colour terms]{Colour terms in SLM, with their original meaning if different.}
\label{tab:func:colourterms}
\end{table}

\footnotetext{The original word \em kuuning \em can also be heard, but because of the phonetic resemblance of \em kuuning \em and \trs{kuunyith}{turmeric}, the latter makes inroads into the domain of colours, also because Sinhala (\em kaha \em) and Tamil (\em ma\ny ca\lz \em)  use the word for `turmeric' to refer to `yellow' as well.}




\paragraph{Similarity}\label{sec:func:mod:Similarity}
Similarity is expressed by the clitic \em =ke \em on the item the term is similar to\formref{sec:morph:=ke}.

\xbox{16}{
\ea\label{ex:func:simil}
\gll se=dang \textbf{baapa=ke} {\em soldier} mà-jaadi suuka. \\
     \textsc{1s=dat} father=\textsc{simil} soldier \textsc{inf}-become like  \\
    `I want to become a soldier like daddy.' (B060115prs10)
\z
} \\
%\xbox{16}{
%\ea\label{ex:func:unreferenced}
%\gll itthu=nam       mlaayu pada=pe     {\em Tradition}=ke      aada. \\
% therefore Malay \textsc{pl}=\textsc{poss} Tradition=simil\\
%`That's why there are the Malays' traditions.' (N060113nar01.85)
%\z
%}




\subsubsection{Quantity}\label{sec:func:Nmod:Quantity}
NPs based on nouns and under restricted circumstances NPs  based on plural pronouns (and very rarely even clauses) can be modified for quantity.

\paragraph{Plurality} \label{sec:func:mod:Plurality}
The following examples show the presence and absence of the plural marker on the word \trs{mlaayu}{Malay}{} in very similar contexts.

\xbox{16}{
\ea\label{ex:func:ptcpt:mod:quant:pl:pada}
\gll bannyak mlaayu \textbf{pada} Hambanthota=ka    arà-duuduk. \\
	much Malay \textsc{pl} Hambantota=\textsc{loc} \textsc{non.past}-exist.\textsc{anim}\\
\z
}

\xbox{16}{
\ea\label{ex:func:ptcpt:mod:quant:pl:zero}
\gll cinggala=\zero{}   aada mlaayu=\zero{} aada {\em Moor}=\zero{} aada  mulbar=\zero{} aada. \\ % bf
     Sinhala exists Malay exist Moor exist Tamil exist  \\
    `There are Sinhalas, Malays, Moors and Hindus.' (G051222nar04.13)
\z
} \\
Plurality need not be expressed when a numeral is mentioned in the clause, as in \xref{ex:ptcpt:mod:quant:num:nopada}, but nothing precludes using \em pada \em nevertheless, as in \xref{ex:ptcpt:mod:quant:num:pada}.



\xbox{16}{
\ea \label{ex:ptcpt:mod:quant:num:nopada}
\gll kithang hathu  {\em week}=nang hathu skali \textbf{duwa} skaali=\zero=ke arà-maakang. \\
     \textsc{1pl} one week=\textsc{dat} one time two time={ }=\textsc{simil} \textsc{non.past}-eat  \\
\z
	} \\

\xbox{16}{
\ea \label{ex:ptcpt:mod:quant:num:pada}
  se=dang duppang lai  \textbf{thiiga} generation \textbf{pada} ini nigiri=ka     anà-duuduk\\
 \textsc{1s=dat} before other three generation \textsc{pl} \textsc{prox} country=\textsc{loc} \textsc{past}-exist\\
\z
}


Plurality can be emphasized on plural pronouns by adding \em pada \em as well. In the following example, the plurality of the word \trs{kitham}{1pl}{} is emphasized by adding \em pada\em.

\xbox{16}{
\ea\label{ex:func:ptcpt:mod:quant:pl:kithampada}
\gll  \textbf{kitham}  \textbf{pada}=pe     baasa=ka       nni {\em grammar} thraa. \\
      \textsc{1pl} \textsc{pl}=\textsc{poss} language=\textsc{loc} \textsc{prox} . neg\\
\z
} \\
Unlike the example above, the following example has \em kitham \em in it without \em pada\em.


\xbox{16}{
\ea\label{ex:func:ptcpt:mod:quant:pl:kithamnopada}
\gll ini \textbf{kitham}=\zero=pe nigiri su-jaadi. \\
 \textsc{prox} \textsc{1pl}=\textsc{poss} country \textsc{past}-become\\
\z
}

% The distributedness of an item is also expressed by \em pada\em. For instance, the mass noun \trs{duwith}{money}{} can be combined with \em pada \em to yield the distributive meaning `funds'.
%
%
% \xbox{16}{
% \ea \label{ex:ptcpt:mod:pl:distr}
% \gll duwith \textbf{pada} aada. \\
%  money \textsc{pl} exist\\
% `Funds were available/Coins and banknotes were available.' (K060116nar09)
% \z
% }

Plurality can also be expressed with \em pada \em on finite headless relative clauses, as in \xref{ex:ptcpt:mod:pl:relc}.

\xbox{16}{
\ea \label{ex:ptcpt:mod:pl:relc}
\gll [[Seelon=nang anà-dhaathang] \textbf{pada}] mlaayu \textbf{pada}. \\
 Ceylon=\textsc{dat} \textsc{past}-come \textsc{pl} Malay pl\\
\z
}


Plurality might actually be a special case of collectiveness \citep{Rijkhoff2002}, which emphasizes the collective character of a set noun. This is evident from the following example, where a group of three gentlemen should give interviews. The three gentlemen are named and coordinated, but that plural marker is added.



\xbox{16}{
\ea \label{ex:ptcpt:mod:pl:collective}
\gll Mr Dole=pe Mr Samath=pe Mr Yusu \textbf{pada}=le {\em interview}=nya thraa. \\
Mr Dole=\textsc{poss} Mr Samath=\textsc{poss} Mr Yusu \textsc{pl}=\textsc{addit} interview=\textsc{dat} neg\\
`Mr Dole, Mr Samath and Mr Yusu were not selected for the interview.' (K060116nar05)
\z
}

It is clear that there is only one specimen of each of the named persons, and the group only exists once, so \em pada \em cannot indicate cardinality greater than 1 in the strict sense. Rather, it emphasizes that we are dealing with a collectivity, the group is not seen as monolithic, but as composed of several members, and the cardinality of the members is greater than one. This interpretation as collective actually fits well with the optionality of \em pada \em according to Rijkhoff's presentation of \em set nouns\em.\kuckn

\paragraph{Definite quantity}\label{sec:func:mod:DefiniteQuantity}
Absolute number can be indicated by cardinal numbers on nominal or pronominal NPs. Numerals precede or follow the noun \xref{ex:ptcpt:num:pre}\xref{ex:ptcpt:num:post}, but always follow the pronoun\xref{ex:ptcpt:num:postpronominal}.

\xbox{16}{
\ea \label{ex:ptcpt:num:pre}
\gll \textbf{thiiga} oorang, \textbf{thiiga} oorang=le, \textbf{thiiga} oorang pada=jo itthu ini {\em volleyball} arà-{\em play}-king=kee. \\
 three man, three man=\textsc{addit}, three man \textsc{pl}=\textsc{foc} \textsc{dist} \textsc{prox} volleyball \textsc{non.past}-play-\textsc{caus}=\textsc{simil}       \\
\z
}\\

\xbox{16}{
\ea \label{ex:ptcpt:num:post}
\gll [[panthas$\curvearrowright$ [[rooja$\curvearrowright$  kumbang]$\curvearrowright$  [pohong $\curvearrowleft$komplok]]] $\curvearrowleft$\textbf{duuwa}] asà-jaadi su-aada. \\
      beautiful rose flower tree bush two \textsc{cp}-grow \textsc{past}-exist \\
    `Two beautiful rose bushes had grown.'  (K070000wrt04)
\z
}\\



\xbox{16}{
\ea \label{ex:ptcpt:num:postpronominal}
\gll Mr    Sebastian            aada, se aada, \textbf{kitham}  \textbf{duuwa} are-oomong. \\
 Mr Sebastian exist \textsc{1s} exist \textsc{1pl} two \textsc{non.past}-speak\\
`You are here, I am here, the two of us are talking.' (K060116nar05.86)
\z
}

The plural particle \em pada \em is often present when numerals modify nouns, but not obligatory \xref{ex:ptcpt:mod:defquant:nopada}. It is not present when the numeral modifies a pronoun \xref{ex:ptcpt:num:postpronominal}.


\xbox{16}{
\ea \label{ex:ptcpt:mod:defquant:pada}
\gll thuuju  {\em generation}  \textbf{pada}  asà-biilang. \\
 seven generation \textsc{pl} \textsc{cp}-say\\
\z
}


\xbox{16}{
\ea \label{ex:ptcpt:mod:defquant:nopada}
\gll mlaayu thigapuluthuuju  baasa \zero{}   aada. \\ % bf
 Malay 37 language { } exist\\
`There are 37 Malay languages.' (K060116nar02.102)
\z
}

For nominal NPs, the numeral may also follow, but is more often put in front of the noun. \em pada \em is normally present.

Measures normally do not take \em pada\em. In example \xref{ex:ptcpt:mod:defquant:measure}, the word \trs{kaayu}{mile} is used without \em pada\em.

\xbox{16}{
\ea \label{ex:ptcpt:mod:defquant:measure}
\gll duuwa kaayu \zero{}  kithang masa-pii. \\
      two mile { } \textsc{1pl} mast-go \\
\z
} \\

\paragraph{Definite order}\label{sec:func:mod:DefiniteOrder}
Definite order is expressed by an ordinal derived from the cardinal by \em ka-\em.
Ordinal numbers \formref{sec:morph:ka-} can only precede the noun. They do not seem to be used with pronouns or clauses.

\xbox{16}{
\ea \label{ex:ptcpt:mod:deford:measure}
\gll se   asdhaathangpa kitham=pe      femili=ka    \textbf{ka-duuwa}
aanak \\
 \textsc{1s} cop \textsc{1pl}=\textsc{poss} familiy=\textsc{loc} card-two child\\
`I am the second child in our family.' (K060108nar01)
\z
}



\paragraph{Indefinite quantity}\label{sec:func:mod:IndefiniteQuantity}
Indefinite quantity is expressed by quantifiers\formref{sec:wc:Quantifiers}. It is mainly used  to modify nouns \xref{ex:ptcpt:mod:quant:indef:canonical}, but can also be found on pronouns\xref{ex:ptcpt:mod:quant:indef:pron}.



\xbox{16}{
\ea \label{ex:ptcpt:mod:quant:indef:canonical}
\gll      \textbf{bannyak} \textbf{mlaayu} pada Hambanthota=ka    arà-duuduk. \\
    `There are many Malays in Hambantota.' (B060115nar02)
\z
} \\


\xbox{16}{
\ea \label{ex:ptcpt:mod:quant:indef:pron}
\gll   suda kithang=nang   boole mosthor \textbf{kithang} \textbf{samma} oorang asà-kumpul. \\
    `So all of us gathered as we could.' (B060115nar02)
\z
} \\





%
% \xbox{16}{
% \ea \label{ex:ptcpt:mod:quant:indef:floated2}
% \gll   oorang=pe baarang pada \textbf{samma} arà-cuuri. \\
%        man=\textsc{poss} goods \textsc{pl} all \textsc{non.past}-stay\\
%     `He steals all the people's goods.' (K051205nar02)
% \z
% } \\

% \xbox{16}{
% \ea\label{ex:func:unreferenced}
% \gll cinggala su-aada sdiikith. \\
%      Sinhala \textsc{past}-exist few   \\
%     `There were few Sinhalese.' (K051222nar06)
% \z
% } \\
%
% \xbox{16}{
% \ea\label{ex:func:unreferenced}
% \gll thiiga umpath aada pompang pada. \\
%       three four exist girl \textsc{pl} \\
% \z
% } \\

% Table \ref{tab:QuantityAdverbs} gives a list of common quantity adverbs.
%
% \begin{table}
% 	\begin{center}
% 	% use packages: array
% 	\begin{tabular}{ll|ll}
% 	SLM & gloss & SLM &  gloss \\
% 	konnyom & few  & spaaru & some\\
%  	sdiikith & few & bannyak & many\\
% 	\end{tabular}
% 	\end{center}
% 	\caption{Common quantity adverbs}
% 	\label{tab:QuantityAdverbs}
% \end{table}



Additionally, a lexical solutions like  \trs{punnu}{full}  can be used \xref{ex:ptcpt:mod:quant:indef:lexical:punnu}. Using \trs{guunung}{mountain} for this was also overheard, but could not be verified in the corpus and needs to be checked for verification.

\xbox{16}{
\ea \label{ex:ptcpt:mod:quant:indef:lexical:punnu}
\gll \textbf{punnu}   mlaayu oorang=nang=le        cinggala  mà-blaajar    thàr-suuka=nang  derang laayeng nigiri  pada=nang   su-pii. \\
     full Malay man \textsc{pl}=\textsc{dat}=\textsc{addit} Sinhala \textsc{inf}-learn \textsc{neg}-like=\textsc{dat} \textsc{3pl} other country \textsc{pl}=\textsc{dat} \textsc{past}-go  \\
\z
} \\
% \paragraph{Relative quantity}\label{sec:func:mod:RelativeQuantity}
% Relative quantity is normally expressed by a relative clause involving the adjectives \trs{liiwath}{more} or  \trs{kuuram}{few,less}.
%
% % \xbox{16}{
% % \ea \label{ex:ptcpt:mod:quant:relative:libbi1}
% % \ea
% % \gll \textbf{spaaru} oorang pada polis  armi. \\
% % some man \textsc{pl} police army\\
% % `Some man in the police, the army.'
% % \ex
% % \gll  derang \textbf{libbi} oorang pada samma siini  arà-duuduk. \\
% %  \textsc{3pl} more man \textsc{pl} all here \textsc{non.past}-stay\\
% % `More than those (living there) are living here.' (K051213nar07)
% % \z
% % \z
% % }
% %
% % \xbox{16}{
% % \ea \label{ex:ptcpt:mod:quant:relative:libbi2}
% % \gll \textbf{libbi} melayu samma Klumbu=ka arà-duuduk. \\
% %  more Malay all Colombo=\textsc{loc} \textsc{non.past}-stay\\
% % \z
% % }
%
% \xbox{16}{
% \ea \label{ex:ptcpt:mod:quant:relative:kuurang}
% \gll  ini \textbf{kuurang} arà-duuduk    laayeng  kumpulan    pada=yang   mà-{\em represent}-kang=nang. \\
%       \textsc{prox} few \textsc{non.past}-exist.\textsc{anim} other group \textsc{pl}=\textsc{acc} \textsc{inf}-represent-\textsc{caus}=\textsc{dat} \\
% \z
% } \\


\paragraph{Indefinite order}\label{sec:func:mod:IndefiniteOrder}
To indicate indefinite order of an entity, the ablative \em =dering \em is used together with the relator noun \trs{daalang}{inside}. A precise quantity like \trs{liima}{five} can be used \xref{ex:ptcpt:mod:order:relative:liima}, or a collective noun like \trs{kumpulan}{group} in \xref{ex:ptcpt:mod:order:relative:liima}.


\xbox{16}{
\ea\label{ex:ptcpt:mod:order:relative:liima}
\gll  se perthaama *(liima) oorang=dering daalang {\em race}=yang su-abbis-king. \\
      \textsc{1s} first five man=\textsc{abl} inside race=\textsc{acc} \textsc{past}-finish-\textsc{caus} \\
\z
} \\

\xbox{16}{
\ea\label{ex:ptcpt:mod:order:relative:kumpulan}
\gll  se perthaama kumpulan=dering daalang {\em race}=yang su-abbis-king. \\
      \textsc{1s} first group=\textsc{abl} inside race=\textsc{acc} \textsc{past}-finish-\textsc{caus} \\
\z
} \\
\subsubsection{Possession}\label{sec:func:mod:Possession}
Terms based on nouns can get assigned a possessor, which is marked with \em =pe\em\formref{sec:morph:=pe}. The possessor is always preposed when it is used as a modifier. Possession can also be expressed predicatively, this is discussed in section \funcref{sec:func:Possession}.


\xbox{16}{
\ea \label{ex:ptcpt:mod:poss}
\gll kithang sama oorang \textbf{lorang=pe} \textbf{suurath}=yang daapath wakthu=ka kithang sama oorang bannyak su-suuka. \\
    `All of us were very happy when we received your letter.'  (Letter 26.06.2007)
\z
}\\

In example \xref{ex:ptcpt:mod:poss}, it is not because of any letter that the speakers are happy, but the set of letters they are happy about is restricted to the one written by the adressee.

\subsubsection{Location}\label{sec:func:Nmod:Location}
A term can be modified to entities at a certain location. This is done by preposing a locative expression. This is technically indistinguishable from both an adjunct and a relative clause construction.

% \xbox{16}{
% \ea\label{ex:func:unreferenced}
% \ea
% \gll  se [meeja=ka]_{MOD} maangga arà-maakang. \\
%       \textsc{1s} table=\textsc{loc} mango \textsc{non.past}-eat\\
%     `I am eating the mango on the table.' (tes)
% \ex se [meejaka]_{ADJCT} maangga aramaakang\\
% `I am eating the mango, on the table.' (tes)
% \ex se [meejaka]_{RELP} maangga aramaakang\\
% `I am eating the mango which is on the table.' (test)
% \z
% \z
% } \\
\xbox{16}{
\ea \label{ex:ptcpt:mod:loc}
\gll sithu=ka     aada  bìssar oorang pada=yang   asà-{\em attack}-kang     {\em mail}=nya    asà-cuuri \textbf{{\em mail}=ka}    duwith arà-baapi. \\
     there=\textsc{loc} exist big man \textsc{pl}=\textsc{acc} \textsc{cp}-attack-\textsc{caus} mail=\textsc{acc} \textsc{cp}-steal mail=\textsc{loc} money \textsc{non.past}-bring  \\
\z
} \\

\subsubsection{Time}\label{sec:func:Nmod:Time}
Terms may be marked as refering to a past referent by means of the adjective \trs{laama}{former}{}.



\xbox{16}{
\ea \label{ex:ptcpt:mod:time}
\gll se=ppe laama ruuma. \\
      \textsc{1s=poss} old house \\
    `My old house.' (test)6.11.08
\z
} \\

Furthermore, the relevant period may be used as a possessive modifier with \em =pe\em. Both adverbs like \trs{karang}{now}{} and temporal nouns like \trs{muusing}{time}{} are possible here.


\xbox{16}{
\ea \label{ex:ptcpt:mod:time:pe:karang}
\gll \textbf{karam=pe} mosthor=nang, mpapulu aari=ka=jo sunnath=le arà-kijja. \\
now=\textsc{poss} manner=\textsc{dat} forty   day=\textsc{loc}=\textsc{foc} circumcision=\textsc{addit} \textsc{non.past}-make  \\
\z
}\\

\xbox{16}{
\ea \label{ex:ptcpt:mod:time:pe:dovulu}
\gll \textbf{dovulu=pe}     oorang pada. \\
      before=\textsc{poss} man \textsc{pl} \\
\z
} \\

\xbox{16}{
\ea \label{ex:ptcpt:mod:time:pe:muusing}
\gll giithu=jo      derang pada, \textbf{itthu}  \textbf{muusing=pe}     mlaayu pada. \\
     like.that=\textsc{foc} \textsc{3pl} \textsc{pl} \textsc{dist} time=\textsc{poss} Malay pl\\
\z
} \\


% \xbox{16}{
% \ea \label{ex:ptcpt:mod:time:pe:muusing}
% \gll itthu    muusing=pe     {\em British}  government. \\
%     \textsc{dist} time=\textsc{poss} British government   \\
%     `The British government of that time.' (N061031nar01)
% \z
% }


%
% \xbox{16}{
% \ea\label{ex:func:unreferenced}
% \gll siini baapa ka jona aada samma record pada laama muusing ka ana aada. \\
%      here father=\textsc{loc}=\textsc{jona} exist all record \textsc{pl} old time=\textsc{loc} \textsc{past}-exist  \\
% \z
% } \\
% \xbox{16}{
% \ea\label{ex:func:unreferenced}
% \gll incayang=pe      muusing=ka kithang=pe     {\em Malays} pada atthu oorang=nang=le        [{\em parliament}=nang  mà-dhaathang=nang]      thumpath thàrà-daapath. \\
%      3s.polite=\textsc{poss} time=\textsc{loc} \textsc{1pl}=\textsc{poss} Malays \textsc{pl} \textsc{indef} man=\textsc{dat}=\textsc{addit} parliament=\textsc{dat} \textsc{inf}-come=\textsc{dat} place \textsc{neg.past}.get  \\
% \z
% } \\
% \subsubsection{Similarity}\label{sec:func:Similarity}
% Similarity can be expressed with the clitic \em =ke(e)\em.
%
% \xbox{16}{
% \ea\label{ex:func:similarity}
% \gll se=dang baapa=\textbf{ke} {\em soldier} mà-jaadi suuka. \\
%      \textsc{1s=dat} father=\textsc{simil} soldier \textsc{inf}-become like  \\
%     `I want to become a soldier like daddy.' (B060115prs10)
% \z
% }

\subsubsection{Restriction and characterization}\label{sec:func:Restrictionandcharacterization}
Modifications can have two different readings, restrictive and characterizing. Restrictive modifications reduce the number of entities the term refers to. \em Rich people are unhappy \em refers to less entities than \em People are unhappy\em. The referents are restricted to those people who also have the characteristic of being rich.

Characterization on the other hand does not reduce the number of entities the term refers to, but gives additional information. \em My loving father \em does not specify which subset of set comprising all the enities x with my.father(x)   the term refers to, there is only one father of mine anyway. \em Loving \em in this case rather gives additional information about the term it refers to, without restricting its referential power. This can be paraphrased as \em My father, who happens to be loving.\em

In SLM, all modifications can be used to restrict or characterize terms based on NPs. Terms based on pronouns can only be characterized. The use of restrictive modification has been exempliefied amply in the preceding sections, so that only the less common characterization will be discussed in this section.

In \xref{ex:func:restmod:n:dwarf}, the dwarf has been talked about before in the story, and this time he reappears in the claws of a bird. There is only one dwarf in the story, and it is absolutely clear that it is this dwarf that is being talked about. In \xref{ex:func:restmod:n:dwarf}, the dwarf is nevertheless modified by  a preposed relative clause, which gives more information about his location. This information does not restrict the set of all possible dwarfs to those being held by birds, it rather gives additional information about the dwarf already identified before, it characterizes him.


\xbox{16}{
\ea\label{ex:func:restmod:n:dwarf}
\gll [Kaaki=ka gaandas-kang ambel anà-duuduk]$_{characterizing}$ Aajuth=yang sangke=luppas hathu pollu=dering Rose-red buurung=nang su-puukul. \\
     leg=\textsc{loc} tie-\textsc{caus} take \textsc{past}-stay dwarf=\textsc{acc} until=leave \textsc{indef} stick=\textsc{abl} Rose-red bird=\textsc{dat} \textsc{past}-hit  \\
\z
} \\
Another example is the characterization of the Almighty God as the creator of the addressee in \xref{ex:func:restmod:n:allah}. This is clearly not restritive but rather characterizing.



\xbox{16}{
\ea\label{ex:func:restmod:n:allah}
\gll [luu nya-laher-kang Allah-thaala]$_{characterizing}$=nang liima wakthu mà-sbaayang=nang. \\
     \textsc{2s.familiar}=\textsc{acc} be.born-\textsc{caus} Allah-almighty=\textsc{dat} five time \textsc{inf}-pray=\textsc{dat}  \\
    `To pray five time to Almighty God, who created you.' (K060116sng01)
\z
} \\

%
% \xbox{16}{
% \ea\label{ex:func:unreferenced}
% \ea
% \gll  luu=nya jadi-kang rabbu saapa lu=ppe nabi pada saapa katha biilang. \\
%       \textsc{2s.familiar}=\textsc{acc} become-\textsc{caus} ??? who \textsc{2s}=\textsc{poss} prophet \textsc{pl} who \textsc{quot} say \\
%     `Say who the RABBU is who made you, who are your prophets.'
% \ex
% \gll lu=ppe rabbu saapa katha buthul balas-an asà-biilang. \\
%       \textsc{2s}=\textsc{poss} ??? who \textsc{quot} correct answer-\textsc{nmlzr} \textsc{cp}-say \\
% \ex
% \gll  lu=ppe nabi saapa katha buthul balas-an asà-biilang. \\
%       \textsc{2s}=\textsc{poss} prophet who \textsc{quot} correct answer-\textsc{nmlzr} \textsc{cp}-say \\
% \ex
% \gll  thumpath=yang baaye sorga=nang bunnang-la. \\
%       place=\textsc{acc} good heaven=\textsc{dat} ???-\textsc{imp} \\
% \z
% \z
% } \\

\subsubsection{Modification of modification}\label{sec:func:Modificationofmodification}
The modifiers of a participant can in turn be modified, e.g. for intensity, as in English \em a very big horse\em, compared to \em a big horse\em. In SLM, this is done lexically, via , \trs{butthul}{correct} or \trs{bannyak}{a lot}, which are all given with their primary meaning here, although in the context of secondary modification, they would probably all rather be glossed as `very'. The following sentences give examples of these words first in their primary meaning and then used as modifiers of modifiers.

The word \em but(t)hul \em has `correct' as its primary meaning, this is shown in \xref{ex:func:ptcpt:mod:modmod:butthul:lit}. The intensive reading is shown in \xref{ex:func:ptcpt:mod:modmod:butthul:intens}. Note that the intensive reading in \xref{ex:func:ptcpt:mod:modmod:butthul:intens} has no geminated stop, which might be indicative of its being not a lexical, but a functional morpheme.

\xbox{16}{
\ea\label{ex:func:ptcpt:mod:modmod:butthul:lit}
\gll  lu=ppe nabi saapa katha \textbf{buthul} balas-an asà-biilang. \\
      \textsc{2s}=\textsc{poss} prophet who \textsc{quot} correct answer-\textsc{nmlzr} \textsc{cp}-say \\
\z
} \\

\xbox{16}{
\ea\label{ex:func:ptcpt:mod:modmod:butthul:intens}
\gll   dee \textbf{buthul} jahhath. \\
      3\textsc{s.impolite} very wicked \\
    `He was very wicked.' (K051205nar02)
\z
} \\

% \xbox{16}{
% \ea\label{ex:func:ptcpt:mod:modmod}
% \gll buthul konnyong=jo mulbar hatthu {\em period}. \\
%      very few=\textsc{foc} Tamil \textsc{indef} period  \\
%     `There was little Tamil at a time.' (K051222nar06)
% \z
% } \\

% \xbox{16}{
% \ea\label{ex:func:unreferenced}
% \gll buthul eenak. \\
%       very tasty \\
%     `It is very tasty.' (K061026rcp02)
% \z
% } \\

The second intensifier is \trs{bannyak}{much}, whose literal meaning is given in \xref{ex:func:ptcpt:mod:modmod:bannyak:lit}, while the intensifier meaning is shown in \xref{ex:func:ptcpt:mod:modmod:bannyak:intens}.

\xbox{16}{
\ea\label{ex:func:ptcpt:mod:modmod:bannyak:lit}
\gll \textbf{bannyak} Muslim oorang pada araduuduk. \\
     many Muslim man \textsc{pl} \textsc{non.past}-exist.\textsc{anim}  \\
\z
} \\
\xbox{16}{
\ea\label{ex:func:ptcpt:mod:modmod:bannyak:intens}
\gll \textbf{bannyak} thuuwa oorang, nya-blaajar oorang. \\
     much old man \textsc{past}-learn man  \\
    `A very old man, an educated man.' (K060116nar07)
\z
} \\
For positive predicationg, \trs{baaye}{good} can also be used as an intensifier.


\xbox{16}{
\ea\label{ex:func:ptcpt:mod:modmod:baae:intens1}
\gll \textbf{baaye} meera caaya kapang-jaadi, thurung-king. \\
     good red colour when-become descend-\textsc{caus} \\
\z
} \\
\xbox{16}{
\ea\label{ex:func:ptcpt:mod:modmod:baae:intens2}
\gll dee aracuuri   \textbf{baaye} kaaya      oorang pada=dering. \\
      3\textsc{s.impolite} good rich man \textsc{pl}=\textsc{abl}   \\
`He stole from very rich people.'
\z
}

% The third intensifier is \trs{punnu}{full}, with the literal reading exemplified in \xref and the intensifier reading, in \xref.
%
% punnu   oorang pada anadhaathang
% K060108nar02.txt:\tx thee
% K060108nar02.txt:\tx thee daawong maa....       thaanamnang

As for decreasing intensity, it appears that \em sdiikith \em or \em konnyong \em are used, both having `few' as their literal meaning. Unfortunately, there are no instances of \em konnyong \em and \em sdiikith \em used for modification of modification in the corpus, only for modification of an adjectival predicate. These are given here for reference.


\xbox{16}{
\ea\label{ex:func:ptcpt:mod:modmod:konnyong}
\gll \textbf{konnyong} thàràsiggar    go=dang. \\
     few sick \textsc{1s=dat}  \\
    `I am a little sick.' (B060115nar04)
\z
} \\

\xbox{16}{
\ea\label{ex:func:ptcpt:mod:modmod:sdiikith}
\gll  Se=ppe biini \textbf{sdiikith} thuuli. \\
      \textsc{1s=poss} wife few deaf \\
    `My wife is a little bit deaf.' (K070000wrt05)
\z
} \\

%
% \xbox{16}{
% \ea\label{ex:func:unreferenced}
% \gll [[[ini      {\em British} government=samma pii=apa     mà-oomong] kithang=pe     {\em statesmen}  pada]=ka  Dr  Jaayah=le      bannyak kethaama hathu  bergaada]   katha  bole=biilang. \\
%        \textsc{prox} British government=\textsc{comit} go=after \textsc{inf}-talk \textsc{1pl}=\textsc{poss} statesmen \textsc{pl}=\textsc{loc} Dr Jaayah much first \textsc{indef} ??? \textsc{quot} can-say \\
%     `You can say that Dr Jaayah was also a very prominent BERGAADA among our statesmen who had gon to negotiate with the British Government.' (N061031nar01)
% \z
% } \\



\section{Predication}\label{sec:func:Predication}
Different things  can be said about participants. One can say about them that they are in a certain state, that they take part  in a certain event, that they are a member of a certain semantic class or that they possess things.

These different types of semantic predicates are encoded by different constructions in the languages of the world, and SLM is no exception.

\subsection{States}\label{sec:func:States}
States can be expressed in SLM by adjectival predicates as in \xref{ex:soa:states:adj} or by verbal predicates as in \xref{ex:soa:states:v}. Some states like colours are expressed by nominal predicates \xref{ex:soa:states:n}.

\xbox{16}{
\ea \label{ex:soa:states:adj}
\gll samma oorang \textbf{baae}$_{ADJ}$. \\
all man good \\
`All men are good.' (B060115cvs13)
\z
}

\xbox{16}{
\ea \label{ex:soa:states:v}
\gll  itthu pada [sraathus binthan pada arà-kiilap]=ke su-\textbf{kiilap}$_{V}$. \\
     \textsc{dist} \textsc{pl} 1000 star \textsc{pl} \textsc{simult}-shin=\textsc{simil} \textsc{past}-shine  \\
\z
}\\


\xbox{16}{
\ea \label{ex:soa:states:n}
\gll Hatthu komplok bannyak=jo puuthi caaya, hathyeng=yang meera=jo [\textbf{meera} \textbf{caaya}]$_{N}$. \\
    `One bush was very white, the other one was of the reddest red.'  (K070000wrt04)
\z
}\\

The defective verbs \trs{suuka}{like} and \trs{thaau}{know} also encode states.


\subsection{Class membership}\label{sec:func:Classmembership}
Membership of an entity in a class of entities is expressed by nominal predicates.


\xbox{16}{
\ea \label{ex:func:pred:class:hatthu}
\gll Sindbad  {\em the}  {\em Sailor}     hatthu Muslim, mlaayu bukang. \\
 Sindbad the Sailor \textsc{indef} muslim, Malay \textsc{neg.nonv}\\
`Sindbad the sailor was a Moor, he was not a Malay.' (K060103nar01)
\z
}


Optionally, the copula \em (asa)dhaathang(apa) \em can be used.



\xbox{16}{
\ea \label{ex:func:pred:class:cop}
\gll se \textbf{asdhaathang} hatthu butthul {\em moderate} Muslim atthu. \\
 \textsc{1s} \textsc{copula} one very moderate Muslim one\\
\z
}



\subsection{Identity}\label{sec:func:Identity}
The predication of identity of two referents is done by the equational construction, which juxtaposes two NPs. This construction often uses the copula \em asadhaathang \em \xref{ex:func:pred:ident:cop} or the focus clitic \em =jo \em \xref{ex:func:pred:ident:jo}.



\xbox{16}{
\ea \label{ex:func:pred:ident:cop}
\gll baapa=pe      umma   \textbf{asadhaathang} kaake=pe           aade. \\
    father=\textsc{poss} mother \textsc{copula} grandfather=\textsc{poss} younger.sibling  \\
    `My paternal grandmother was my grandfather's younger sister.' (K051205nar05)
\z
} \\
\xbox{16}{
\ea \label{ex:func:pred:ident:jo}
\gll suda [itthu    kaake=pe aade=pe                aanak]$_{reftop}$=\textbf{jo}    baapa$_{refident}$. \\ % bf
      thus \textsc{dist} grandfather=\textsc{poss} younger.sibling=\textsc{poss} child=\textsc{foc} father \\
    `So that grandfather's younger sister's child is my father.' (K051205nar05)
\z
}


\subsection{Location}\label{sec:func:Location}
Location is expressed by a locational predicate with \em =ka\em.

\xbox{16}{
\ea \label{ex:func:loc}
   \gll  kithang=pe     oorang thuuwa pada samma Seelong\textbf{=ka}. \\
   \textsc{1pl}=\textsc{poss} man old \textsc{pl} all Ceylon=\textsc{loc} \\
\z
}

Addtionally, location can be expressed by an existential construction with specification of the semantic role of \textsc{location} on one participant, in the following example \em Dubai\em. The existenial predicate differs from the pure locational predicate by the existence of an existential verb, in this case \em duuduk\em.

\xbox{16}{
\ea \label{ex:func:loc:exist}
\gll  se=ppe      dhaatha=pe           thiiga aanak=le      Dubai=\textbf{ka}     arà-\textbf{duuduk}. \\
 \textsc{1s=poss} elder.sister=\textsc{poss} three child=\textsc{addit} Dubai=\textsc{loc} \textsc{non.past}=stay\\
\z
}


\subsection{Possession}\label{sec:func:pred:Possession}
Possessive predicates are expressed by a locational predicate of the possessive subtype with either the dative, used for permanent possession, or the locative, used for temporary possession. For more details on possession, see \funcref{sec:func:Possession}.


\xbox{16}{
\ea \label{ex:func:poss:nang}
\gll  \textbf{se=dang} liima anak  klaaki pada \textbf{aada}. \\
      \textsc{1s=dat} five child male \textsc{pl} exist \\
    `I have five sons.' (K060108nar02)
\z
} \\

\xbox{16}{
\ea \label{ex:func:poss:ka}
\gll incayang\textbf{=ka} ... bìssar beecek caaya hathu {\em bag} su-\textbf{aada}. \\
     3s.polite=\textsc{loc} ... big mud colour \textsc{indef} bag \textsc{past}-exist  \\
    `He had a big brown bag with him.' (K070000wrt04a)
\z
} \\
\subsection{Events}\label{sec:func:Events}
Events are characterized by being [+dynamic]. They  are always encoded as verbs \xref{ex:soa:events:v}. An example is given in \xref{ex:soa:events:v}.

\xbox{16}{
\ea \label{ex:soa:events:v}
\gll  baapa=le       aanak=le      guula su-maakang. \\
      father=\textsc{addit} child=\textsc{addit} sugar \textsc{past}-eat \\
    `Father and son ate sugar.' (K070000wrt02)
\z
} \\
It is possible to use lexemes from the adjective class to denote events and not states. In this case, they undergo conversion to verbs, take verbal morphology and refer to  the process of the state denoted by the adjective coming into being. The adjective \trs{bìssar}{big} in \xref{ex:soa:events:adj} is used with verbal morphology and then denotes not the state of being big, but the event of becoming big, i.e. growing up.

\xbox{16}{
\ea \label{ex:soa:events:adj}
\gll itthu=nam blaakang=jo, kitham pada \textbf{anà-bìssar}. \\
 \textsc{dist} after=\textsc{foc} \textsc{1pl} \textsc{pl} \textsc{past}-big\\
\z
}

The use of an adjective in a verbal frame changes its aktionsart from static to dynamic. It is important to distinguish aktionsart from grammatical aspect in this regard. Aktionsart namely does not interfere with the marking of aspect. The following examples show that dynamic aktionsart can combine with the past tense (with in this case perfective semantics) \xref{ex:soa:events:adj:su}, the conjunctive participle \xref{ex:soa:events:adj:s} also giving a perfective reading, but also with \em arà- \em in \xref{ex:soa:events:adj:ara}, giving an imperfective reading.



\xbox{8.5}{
\ea \label{ex:soa:events:adj:su}
hatthu spuulu liimablas     thaaun=nang jaalang blaakang,     inni kumpulan    \textbf{sa}-mampus\\
indef ten fifteen year=\textsc{dat} go after \textsc{prox} association \textsc{past}-dead. \\
`About ten, fifteen years after that, the association became defunct.
\z
 }

\xbox{16}{
\ea \label{ex:soa:events:adj:s}
\gll aanak pada \textbf{asa}-bìssar, skuul=nang anà-pii. \\
 child \textsc{pl} \textsc{cp}-big school=\textsc{dat} \textsc{past}-go\\
\z
}


\xbox{8.5}{
\ea \label{ex:soa:events:adj:ara}
\gll  ruuma \textbf{ara}-kiccil. \\
      house \textsc{non.past}-small\\
    `The houses are getting small.'  (K051222nar04)
\z
}\\

The use of adjectives in a verbal frame is thus not a case of grammatical  inchoative or ingressive aspect, but a change in lexical aspect from [-dynamic] to [+dynamic].

%
%\xbox{16}{
%\ea\label{ex:func:unreferenced}
%\gll {\em patients} pada  nyaakith  oorang pada s-pii      thaangan are-cuuci    nni      saarong samma bassa. \\
% \\
%`.' (nosource)
%\z
%}

The change of aktionsart from static to dynamic through the use in a verbal frame is also possible with loan words, as shown in \xref{ex:soa:events:adj:loan}

\xbox{16}{
\ea \label{ex:soa:events:adj:loan}
\gll  se=dang \textbf{arà-{\em late}} bukang, see arà-dhaathang. \\
      \textsc{1s=dat} \textsc{non.past}-late \textsc{tag} \textsc{1s} \textsc{non.past}-come \\
\z
} \\
Example \xref{ex:soa:events:adj:doubleconversion} is more involved. In this example, the adjective \trs{lummas}{soft} is first converted to a verb `become soft', which in turn is the only constituent of the NP to which the postpositiong \em =nang \em attaches. Example \xref{ex:soa:events:adj:doubleconversion} thus shows two instances of conversion, one from adjective to verb on the morphological level, and one from clause to NP on the syntactic level.


\xbox{16}{
\ea \label{ex:soa:events:adj:doubleconversion}
\gll [[[[\zero{} lummas$_{ADJ}]$ $-\zero]_{verb}$]]$_{CLS}$ =\zero]$_{NP}$ =nang blaakang minnyak klaapa=ka inni=yang gooreng. \\ % bf
      { } soft -vblzr =\textsc{nmlzr} =\textsc{dat} after coconut.oil coconut=\textsc{loc} \textsc{prox}=\textsc{acc} fry \\
\z
} \\

While adjectives can undergo this change in lexical aspect by zero-derivation, periphrases have to be used for states denoted by other word classes. States denoted by nouns have to take \trs{jaadi}{become}{} to get a dynamic reading, which allows for the application of perfective aspect, as in \xref{ex:soa:events:n:dyn:jaadi:su}, where the past tense marker \em su- \em conveys a perfective reading. Before, the country was not theirs, but a change of state took place, and now the country is theirs. This change of state implies dynamic aktionsart, which is not a possibility for nouns. Therefore, the periphrasis with the verb \trs{jaadi}{become}{} is used, which allows for the expression of dynamic aktionsart.

\xbox{16}{
\ea \label{ex:soa:events:n:dyn:jaadi:su}
\gll ini kitham=pe \textbf{nigiri} su-\textbf{jaadi}. \\
prox 1pl:poss country \textsc{past}-become \\
`This became our country.' (K051222nar04)
\z
}

Things are similar with the following example, where at a time $t_0$, the person was not a  cancer patient, but at a time $t_1$ had become a cancer patient. This again has to be expressed by \em jaadi\em.

\xbox{16}{
\ea \label{ex:soa:events:n:dyn:jaadi:se}
\gll incayang  \textbf{{\em cancer}}  \textbf{patient} se-\textbf{jaadi}. \\
      3s.polite cancer patient \textsc{past}-become \\
    `He became a cancer patient.' (K060116nar15)
\z
} \\

For the word \trs{enco}{fooled}{}, the same holds true.

\xbox{16}{
\ea \label{ex:soa:events:n:dyn:jaadi:thara}
\gll Thapi=le Andare thàrà-\textbf{jaadi} \textbf{enco}. \\
    `But Andare did not get fooled.' (K070000wrt02)
\z
} \\

% \xbox{16}{
% \ea\label{ex:func:unreferenced}
% \gll  itthu=nang      blaakang, [kithang=pe hatthu oorang=le      {\em minister} jaadi  thraa] kithang=nang   nya-aada. \\
%       \textsc{dist}=\textsc{dat} after \textsc{1pl}=\textsc{poss} \textsc{indef} man=\textsc{addit} minister become \textsc{neg} \textsc{1pl}=\textsc{dat} \textsc{past}-exist \\
% \z
% } \\
This process is also possible for mass nouns, as shown in \xref{ex:soa:events:n:dyn:jaadi:mass} for the mass nouns \trs{daara}{blood} and \trs{suusu}{milk}.

\xbox{16}{
\ea \label{ex:soa:events:n:dyn:jaadi:mass}
\gll oorang pada kapang-laari   dhaathang, ini daara sgiithu=le  \textbf{suusu} \textbf{su-jaadi}. \\
    `When people came running, the blood had turned into milk.' (K051220nar01)
\z
} \\

The same construction can be used for modal predications. Modal predications are of static aktionsart, but if for some reason a change of state (from possible to impossible or the other way round) takes place, a verbal periphrasis with \em jaadi \em must be employed to give the dynamic reading.\footnote{See \citep[224]{Karunatillake2004} for a related construction in Sinhala.} In \xref{ex:soa:events:mod:dyn:jaadi:su:modal1}, the high commissioner was first able to come, but finally became unavailable. This change of state is again expressed by \trs{jaadi}{become}.


\xbox{16}{
\ea \label{ex:soa:events:mod:dyn:jaadi:su:modal1}
\gll itthu    blaakang=jo,    kitham=pe {\em AGM}=nang  duppang,  {\em high} {\em commissioner} {\em cultural} {\em show}=nang mà-dhaathang=nang        \textbf{thàràboole} \textbf{s-jaadi}. \\
    After that, before our Annual General Meeting, it became impossible for the High Commissioner to attend the cultural show. (K060116nar23.39)
\z
} \\
A similar constellation obtains in \xref{ex:soa:events:mod:dyn:jaadi:su:modal2}.

\xbox{16}{
\ea \label{ex:soa:events:mod:dyn:jaadi:su:modal2}
\gll see=yang dhaathang {\em remand}=ka mà-thaarek thaaro=nang \textbf{thàràboole} \textbf{su-jaadi}. \\
    `it became impossible to remand me.' (K061122nar03)
\z
} \\

The following sentences give some more examples of this use.


\xbox{16}{
\ea \label{ex:soa:events:mod:dyn:jaadi:extra1}
\gll Aashik=nang \textbf{hathu} \textbf{{\em soldier}} \textbf{mà-jaadi} suuka=si katha arà-caanya. \\
    `He asks if you want to become a soldier, Ashik.' (B060115prs10)
\z
} \\

\xbox{16}{
\ea \label{ex:soa:events:mod:dyn:jaadi:extra2}
\gll baaye \textbf{meera} \textbf{caaya} kapang-\textbf{jaadi}, thurung-king. \\
    `When it turns into a nice red colour, remove it (from the fire).' (K060103rec02)
\z
} \\

% \xbox{16}{
% \ea\label{ex:func:unreferenced}
% \gll ithu=ka sakith aathi asà-jaadi aada. \\
%       \textsc{dist}=\textsc{loc} sick heart \textsc{cp}-become exist \\
%     `On that, discontent had arisen.' (K060116nar04)
% \z
% } \\



Nominal predicates construed with experiencers have the possibility to mark change of state with the verb \trs{pii}{go}, as in \xref{ex:soa:events:mod:dyn:pii}, where the experiencer \em incayang \em is marked with the dative marker \em =nang \em which has `allative' as an additional meaning. This makes it possible to use it with the verb \trs{pii}{go}, entailing that the sematic role changes from experiencer to goal.

\xbox{16}{
\ea\label{ex:soa:events:mod:dyn:pii}
\gll Incayang=nang baaye=nang maara su-pii. \\
     3s.polite=\textsc{dat} good=\textsc{dat} angry \textsc{past}-go  \\
    `He became really angry (Anger went upon him).'  (K070000wrt01)
\z
}\\


Finallly, verbs contrued with experiencers, like \trs{thathaawa}{laugh} can also use this periphrasis with \em pii \em to emphasize the change of state.


\xbox{16}{
\ea\label{ex:soa:events:mod:dyn:pii:exp}
\gll Itthu wakthu=ka Andare asà-maathi anà-duuduk mosthor kuthumung=apa raaja=nang \textbf{thathaawa} su-\textbf{pii}. \\
   \textsc{dist} time=\textsc{loc} Andare \textsc{cp}-dead \textsc{past}-stay manner see=after king=\textsc{dat} laugh \textsc{past}-come    \\
    `When he then saw the way that Andare had died and lay there, the king started to laugh.' (K070000wrt03)
\z
} \\
There is one example where a word likely to be a noun is used without \em jaadi\em, but rather in the verbal frame discussed above, marked by a TAM-prefix. This is the loan word \em pension \em in \xref{ex:soa:events:mod:dyn:jaadi:contr}, but it could actually be argued that \em pension \em in SLM does in fact mean `retired', and therefore is an adjective, which can undergo conversion to become a verb for dynamic reading, as usual.

\xbox{16}{
\ea\label{ex:soa:events:mod:dyn:jaadi:contr}
\gll derang pada samma konnyong aari pukurjan asà-gijja, \textbf{su-{\em pension}}. \\
      \textsc{3pl} \textsc{pl} all few day work \textsc{cp}-make \textsc{past}-pension \\
\z
} \\


\subsection{Causation}\label{sec:func:Causation}

There are two ways to indicate causation: the causative morpheme \em -king \em\formref{sec:morph:-king}  and a construction using a verb of saying like \trs{biilang}{say}.


% \xbox{16}{
% \ea \label{ex:caus:king:adj}
% \gll  itthuka asà-thaaro, itthu=yang arà-\textbf{panas-king}. \\
%       \textsc{dist}=\textsc{loc} \textsc{cp}-put \textsc{dist}=\textsc{acc} \textsc{non.past}-hot-\textsc{caus} \\
% \z
% }\\


\xbox{16}{
\ea \label{ex:func:caus:king}
\gll baaye meera caaya kapang-jaadi, \textbf{thurung-king}. \\
     good red colour when-become, descend-\textsc{caus}  \\
    `When  [the food] has  turned to a nice rose colour, remove (it) [from the fire].'  (K060103rec02)
\z
}\\

% \xbox{16}{
% \ea \label{ex:caus:king:trans}
% \gll kitham  arà-\textbf{mirthi-kang}       Kluumbu {\em confederation}=nang     kithang=nang daapath {\em latest} twelve=ka. \\
%      \textsc{1pl} \textsc{non.past}-understand-cause Colombo c.=\textsc{dat} \textsc{1pl}=\textsc{dat} l. twelfth=\textsc{loc} \\
% \z
% }\\

Verbally causing someone to do something can be expressed with an utterance verb in the main clause and an infinitive clause with the action ordered.

\xbox{16}{
\ea \label{ex:func:caus:biilang}
\gll oorang pada=nang     \textbf{mà-dhaathang}     \textbf{katha} asà-\textbf{biilang}. \\
     man \textsc{pl}=\textsc{dat} \textsc{inf}-come \textsc{quot} \textsc{cp}-tell  \\
\z
} \\

%\section{Giving extra information beyond the nuclear predication}\label{sec:func:Givingextrainformationbeyondthenuclearpredication}
%Next to the basic information about predicate, arguments, space and time, there are some other, more peripheral, pieces of information people want to communicate about. These have received various names in the literature, like circumstantials\citep{}, satellites\citep{}, or XXX\citep{}. These have in common that they are not central or necessary for the predication. The predication would still be complete without them. Rather, they provide additional information, like reasons, conditions, purpose,

% \subsubsection{Quantity?}\label{sec:func:Quantity?}
%
% \xbox{16}{
% \ea\label{ex:func:unreferenced}
% \gll ini      swaara liiwath. \\
%       \textsc{prox} noise much \\
%     `There is a lot of noise there.' (K051222nar04)
% \z
% } \\

\section{Modification}\label{sec:func:Modification}
Different modification strategies exist in SLM. These depend fully on the lexical category of the head noun, regardless of its semantic class. The different possibilities are   discussed in \formref{sec:nppp:Thefinalstructureofthenounphrase} for referential phrases (NPs) and \formref{sec:pred:Verbalpredicates} for predicate phrases..

% \subsection{Modifying events}\label{sec:func:Modifyingevents}
% Modification of events can be done in several ways. The first one is to use an NP with a dative marker, as in \xref{ex:func:soa:event:mod:nang}. The second one is to use a suitable vector verb, which for the time being would be only \trs{thaaro}{hit} to add a meaning of `violent' to the event.
%
% \xbox{16}{
% \ea\label{ex:func:soa:event:mod:nang}
% \gll {\em brass}-iyang \textbf{baae=nang} \textbf{cuuci}. \\
% `Wash the rice well!' (K060103rec01.1)\z
% }
%
% \xbox{16}{
% \ea\label{ex:func:soa:event:mod:thaaro1}
% \gll [incayang=pe kepaala=ka anà-aada] thoppi=dering moonyeth pada=nang su-buwang \textbf{puukul}. \\
%       3s.polite=\textsc{poss} head=\textsc{loc} \textsc{past}-exist hat=\textsc{abl} monkey \textsc{pl}=\textsc{dat} \textsc{past}-throw hit \\
% \z
% }\\
%
% The third way to modify an event is to use the simultaneous participle, which is formed by reduplication. An example of the event of running being modified by such a participle, which indicates the jumping manner in which the running happens, is given in \xref{ex:func:soa:event:mod:redup}.
%
% \xbox{16}{
% \ea\label{ex:func:soa:event:mod:redup}
% \gll kancil \textbf{lompath}\~{}\textbf{lompath} arà-pii. \\
%      rabbit jump\~{ }jump      \textsc{non.past}-run \\
%     `The rabbit runs away jumping.'  (test)6.11.08
% \z
% }\\
%
% A fourth way, which might be more marginal, is the modification of an event by indicating what did \em not \em happen at the same time. This arguable only constitues one event, not two, so that the main event is modified by the (non-existent) second one.
%
%
% \xbox{16}{
% \ea\label{ex:func:soa:event:mod:jama}
% \gll    kithang=nang bole=duuduk hatthu=le \textbf{jamà-maakang=nang}  two duwa 2 o' {\em clock}=ke  sangke bole=duuduk. \\
%        \textsc{1pl}=\textsc{dat} can-stay \textsc{indef}=\textsc{addit} \textsc{neg.nonfin}-eat=\textsc{dat}  two two two o'clock=\textsc{simil} until can-stay  \\
% \z
% } \\
% The indication what indeed \em did \em happen at the same time would probably not be a modification of the first event, but rather information about a second event.
%
% % \xbox{16}{
% % \ea\label{ex:func:soa:event:mod:thaaro2}
% % \gll Ithu=kapang ithu moonyeth pada=le [anà-maayeng duuduk thoppi] pada=dering inni oorang=nang su-\textbf{bale-king} \textbf{puukul}. \\
% %     `Then the monkeys threw back the hats with which they had been playing.' (K070000wrt01)
% % \z
% % } \\
% %
% % \xbox{16}{
% % \ea\label{ex:func:unreferenced}
% % \gll  ini [kuurang arà-duuduk]        laayeng kumpulan  pada=yang   mà-{\em represent}-kang=nang. \\
% %       \textsc{prox} few \textsc{non.past}-exist.\textsc{anim} other group \textsc{pl}=\textsc{acc} \textsc{inf}-represent-\textsc{caus}=\textsc{dat} \\
% % \z
% % } \\
%
% % \xbox{16}{
% % \ea\label{ex:func:jama:negcp1}
% % \ea
% % \gll liiwath aayer \textbf{jamà}-jaadi=\textbf{nang}. \\
% %      much water \textsc{neg.nonfin} become=\textsc{dat}  \\
% %     `Without putting too much water' (K060103rec01)
% % \ex
% % \gll itthu aayer=yang hathu blaangan=nang luppas. \\
% %      \textsc{dist} water=\textsc{acc} \textsc{indef} amount=\textsc{dat} leave  \\
% % \z
% % \z
% % } \\

\section{Space}\label{sec:func:Space}
Participants, states and events are always located in space and time. This section discusses the different possibilities to give spatial information in SLM. The next section will deal with the counterpart of space, time. Some concepts, like figure and ground are important for both domains.

We can distinguish three different kinds of spatial reference: absolute (non-deictic) reference (\em in India\em), deictic reference with reference to the speaker (\em in front of the tree\em, i.e. between the tree and the speaker) and deictic reference with regard to other entities (\em between a rock and a hard place\em). Finally, events can also have an inherent directionality, like \em come and bring \em or \em go and take \em in English.

\subsection{Giving the non-deictic reference space}\label{sec:func:Givingthenon-deicticreferencespace}
The absolute reference space is normally indicated  by the locative marker \em =ka \em \formref{sec:morph:=ka} attached to an NP with local reference.  This can be a common noun as in \xref{ex:space:nond:cn} or a proper noun as in \xref{ex:space:nond:pn}. In rare cases, the locative marker is not present on proper nouns, as in \xref{ex:space:nond:noka}.


\xbox{16}{
\ea \label{ex:space:nond:cn}
\gll se   m-blaajar      \textbf{{\em estate}=ka}. \\
 \textsc{1s} \textsc{past}-learn estate=loc\\
\z
}


\xbox{16}{
\ea \label{ex:space:nond:pn}
\gll [\textbf{Kandi=ka}    arà-duuduk]      {\em military} rejimen melayu. \\
     Kandy=\textsc{loc}  \textsc{non.past}-stay . regiment Malay \\
    `The Military regiment Malays who stayed in Kandy.' (K060108nar02.27)
\z
} \\

\xbox{16}{
\ea\label{ex:space:nond:noka}
\gll  Seelon  samma thumpath=\zero{} mlaayu aada. \\ % bf
      Ceylon all place Malay exist\\
    `There are Malays all over Sri Lanka.' (K051222nar04)
\z
} \\


\subsection{Giving the deicitc reference space with regard to speech act participants}\label{sec:func:Givingthedeicitcreferencespacewithregardtospeechactparticipants}
The spatial location of referents with regard to the speaker can be given by the deictics \trs{ini}{proximal}{} and \trs{itthu}{distal}.



\xbox{16}{
\ea \label{ex:space:inni}
\gll \textbf{inni}     sudhaari=pe   femili=ka    bannyak oorang tsunami=da     spuukul. \\
     \textsc{prox} sister=\textsc{poss} familiy=\textsc{loc} many man tsunami=\textsc{dat}  \textsc{cp}-hit\\
    `Of this sister's family, many members were hit by the Tsunami.' (B060115nar02.17)
\z
} \\

\xbox{16}{
\ea \label{ex:space:itthu}
\gll \textbf{itthu}    {\em ports}=ka    laama kar asà-baapi. \\
     \textsc{dist} p.=\textsc{loc} old car \textsc{cp}-bring  \\
    `We brought the old cars to those ports.' (K051206nar19.7)
\z
} \\
The spatial location of events with regard to the speaker can be indicated by the spatial adverbs \em siini \em and \em siithu\em. These can optionally bear a locative clitic as in \xref{ex:space:siini:ka}\xref{ex:space:siithu:ka}, or be used without one, as in \xref{ex:space:siini:noka}\xref{ex:space:siithu:noka}.

\xbox{16}{
\ea \label{ex:space:siini:ka}
\gll   \textbf{siini=ka}    {\em settle} daapath, itthu=nam       blaakang bannyak oorang pada \textbf{siini} se-duuduk. \\
      here=\textsc{loc} s. get, \textsc{dist}=\textsc{dat} after many man \textsc{pl} here \textsc{past}-stay\\
    `They settled down here, after that many people came to stay.' (G051222nar03.9)
\z
} \\

\xbox{16}{
\ea \label{ex:space:siithu:ka}
\gll \textbf{siithu=ka}    se   cinggala   oorang=pe cinggala   em-blaajar. \\
 dem.loc.dist=\textsc{loc} \textsc{1s} sinhala man=\textsc{poss} sinhala \textsc{past}-learn \\
\z
}



\xbox{16}{
\ea \label{ex:space:siini:noka}
\gll  spaaru siini=\zero{} su-duuduk. \\ % bf
       few here \textsc{past}-stay\\
    `Few stayed here.' (K051205nar04)
\z
} \\

\xbox{16}{
\ea \label{ex:space:siithu:noka}
\gll  incayang  siithu=\zero{} asà-kaaving. \\ % bf
       3s.polite there \textsc{cp}-marry\\
    `He married there.' (K051206nar18)
\z
} \\
While marking of the locative is optional, marking of the allative is impossible for deictics \xref{ex:space:siini:all} and the marking of the ablative is obligatory\xref{ex:space:siini:abl}.


\xbox{16}{
\ea \label{ex:space:siini:all}
\ea   *siini=nang
\ex *siithu=nang
\z
\z
} \\
\xbox{16}{
\ea\label{ex:space:siini:abl}
\gll see arà-sumpa  paanas muusing dhaathang=thingka see siini=\textbf{dering} arà-pii. \\
     \textsc{1s} \textsc{non.past}-promise hot time come=middle \textsc{1s} here=\textsc{abl} \textsc{non.past}-go  \\
\z
} \\


The spatial character can be highlighted by the used of \trs{subla}{side}, as is the case in the following example.


\xbox{16}{
\ea\label{ex:space:subla}
\ea
\gll Biini itthu    \textbf{subla}=dering  arà-bithàràk. \\
      wife \textsc{dist} side=\textsc{abl} \textsc{non.past}-scream \\
\ex
\gll   Puthri inni     \textbf{subla}=dering  arà-bithàràk. \\
      princess \textsc{dist} side=\textsc{abl} \textsc{non.past}-scream \\
\z
\z
} \\

Spatial location close to the hearer can be indicated by \em sanaka\em, but this is rarely done. Normally \em  siini \em is used, if the hearer is thought to be close to the speaker, or \em siithu \em if the hearer happens to be farther away.

Among third person pronouns, \em siaanu \em can only be used for persons close to the speaker.

Deictic and non-deictic spatial reference can be combined, as in \xref{ex:space:deicnondeic}, where we find non-deictic \em Sri Lankaka \em and deictic \em siini\em.

\xbox{16}{
\ea\label{ex:space:deicnondeic}
\gll see spuulu thaaun \textbf{siini} \textbf{Sri} \textbf{Lanka=ka}  pukurjan nya-kirja. \\
    `I worked here in Sri Lanka for ten years.' (K061026prs01)
\z
} \\

More precise indications of spatial location with regard to speech act participants can be given by using one or more of them as ground in a figure-ground relation. This will be explained in the next section.


% \xbox{16}{
% \ea\label{ex:func:unreferenced}
% \gll  see ini      Sri  Lanka=ka    nya-blaajar. \\
%       \textsc{1s} \textsc{prox} Sri Lanka=\textsc{loc} \textsc{past}-learn \\
% \z
% } \\



\subsection{Figure-ground relations}\label{sec:func:Figure-groundrelations}
Besides the non-deictic reference space and the reference space with regard to speech act participants, it is possible to give the spatial relation between two or more arbitrary entities. The locative marker \em =ka \em can be used to convey a generic figure-ground relation, where nothing about the precise disposition of figure and ground in implied. More precise constellations can be obtained by using relator nouns\formref{sec:wc:Relatornouns} \citep[25][cf.]{Adelaar1991}. These relator nouns indicate stative information (top, bottom, front, back). Lative information (from the inside, to the middle) can be obtained by combining them with the relevant postpositions \trs{=ka}{essive}{}, \trs{=nang}{allative}{} \trs{=dering}{ablative}{}.
The most common case is to combine two nouns in such a figure ground relation. The following examples show a number of relator nouns

\xbox{16}{
\ea\label{ex:space:figground:atthas}
\gll Ini pohong$_{ground}$ \textbf{atthas}=ka [moonyeth hathu kawanan]$_{figure}$ su-aada. \\ % bf
     \textsc{prox} top=\textsc{loc}  monkey \textsc{indef} group \textsc{past}-exist\\
    `On top of this tree was a group of monkeys.'   (K070000wrt01)
\z
}\\


% \xbox{16}{
% \ea\label{ex:space:figground:daalng}
%    \gll  kiccil wavvaal pada daalang=ka  arà-duuduk. \\
%      small  bat     \textsc{pl}   inside=\textsc{loc} \textsc{non.past}-exist.\textsc{anim} \\
% \z
% }



\xbox{16}{
\ea\label{ex:space:figground:baawa}
\gll Andare$_{ground}$ [hathu pohong]$_{figure}$=pe \textbf{baawa}=ka kapang-duuduk. \\ % bf
     Andare \textsc{indef} tree=\textsc{poss} bottom=\textsc{loc} when-sit  \\
    `When Andare sat down uncer a tree.' ((K070000wrt03))
\z
} \\

\xbox{16}{
\ea\label{ex:space:figground:dikkath}
\gll Soore=ka, [Snow-white=le Rose-red=le]$_{figure}$ derang=pe umma=samma appi$_{ground}$ \textbf{dìkkath}=ka arà-duuduk ambel.  \\ % bf
      Evening=\textsc{loc} Snow.white=\textsc{addit} Rose.Red=\textsc{addit} \textsc{3pl}=\textsc{poss} mother=\textsc{comit} fire vicinity=\textsc{loc} \textsc{simult}-sit take \\
\z
}\\

% \xbox{16}{
% \ea\label{ex:space:figground:blaakang}
% \gll ruma  saakith Suvasevana {\em hospital}=pe     \textbf{blaakang}=ka. \\
%       house sick Suvasena hospital=\textsc{poss} back=\textsc{loc} \\
% \z
% } \\

Pronouns can be used in this construction as well. This is true for third person pronouns \xref{space:figgr:pron:3} as well as for speech act participants as in \xref{space:figgr:pron:1}

\xbox{16}{
\ea \label{space:figgr:pron:3}
\gll  [incayang]$_{ground}$=pe      baa=ka      [spaaru]$_{figure}$ aada. \\
3.polite=\textsc{poss} down=\textsc{loc} some exist\\
`There are some [Malays] down the hill from  where he lives.' (K051213nar05.144)
\z
}


\xbox{16}{
\ea \label{space:figgr:pron:1}
\gll kithang lorampe atthas anà-oomong. \\
      \textsc{1pl} \textsc{2pl}=\textsc{poss} about \textsc{past}-talk \\
\z
} \\
%
% \xbox{16}{
% \ea \label{space:figgr:pron:1}
% \gll  se=ppe diiri=ka maanjur aada. \\
%        \\
%     `.' (test)6.11.08
% \z
% } \\
The interpretation of the relator noun need not be literal. In \xref{space:figgr:leeway}, the neighbour from above does not live directly on top of the speakers, but rather on the next floor. Still, the use of \trs{ruuma}{house} or \trs{thattu}{roof} is not possible, and the simple pronoun \em kitham \em is used instead.


\xbox{16}{
\ea \label{space:figgr:leeway}
\gll Marian kitham=pe (*ruuma) atthas (*thattu)=ka araduuduk. \\
     Marian \textsc{1pl}=\textsc{poss} house top roof =\textsc{loc} \textsc{non.past}-live  \\
\z
} \\

% \xbox{16}{
% \ea
% \gll  kuuthu se=ppe paala=ka aada. \\
%      louse \textsc{1s=poss} head=\textsc{loc} exist  \\
%     `There are lice on my head.' (nosource)6.11.08
% \z
% } \\
Relative indications of location can be expressed by \trs{thangang naasi subla}{hand'+`rice'+`subla'=`right hand side} and \trs{thangang kiiri subla}{left hand side} ( The latter has no transparent meaning). The following four examples show this, as well as using \em thìnnga \em for `in the middle' and \em duppang \em for `opposite'.

\xbox{16}{
\ea
\gll Izi se=ppe \textbf{thangang} \textbf{naasi} subla=ka arà-duuduk. \\
    `Izi is sitting at my right hand side.' (nosource)6.11.08
\z
} \\

\xbox{16}{
\ea
\gll Sebastian se=ppe \textbf{thangang} \textbf{kiiri} subla=ka araduuduk. \\
    `Sebastian is sitting at my left hand side.' (nosource)6.11.08
\z
} \\

\xbox{16}{
\ea
\gll see thinnga=ka arà-duuduk. \\
     \textsc{1s} middle=\textsc{loc} \textsc{non.past}-sit  \\
\z
} \\

\xbox{16}{
\ea
\gll  Sebastian se=ppe duppang=ka araduuduk. \\
       Sebastian \textsc{1s=poss} front=\textsc{loc} \textsc{non.past}-sit  \\
\z
} \\


The essive has been exemplified by the sentences above.   Allative \xref{ex:space:figgr:allative} and ablative \xref{ex:space:figgr:ablative} are exemplified by the following examples.


\xbox{16}{
\ea \label{ex:space:figgr:allative}
\gll  kithang arà-pii    inni     {\em politicians} pada dìkkath=\textbf{nang}. \\
      \textsc{1pl} \textsc{non.past}-go \textsc{prox} politicians \textsc{pl} vicinity=\textsc{dat} \\
\z
} \\
\xbox{16}{
\ea \label{ex:space:figgr:ablative}
\ea
\gll Biini itthu    subla=\textbf{dering}  arà-bithàràk. \\
      wife \textsc{dist} side=\textsc{abl} \textsc{non.past}-scream \\
\ex
\gll   Puthri inni     subla=\textbf{dering}  arà-bithàràk. \\
      princess \textsc{dist} side=\textsc{abl} \textsc{non.past}-scream \\
\z
\z
} \\

The perlative is also formed with \em =dering\em, as shown in the following example.

\xbox{16}{
\ea \label{ex:space:figgr:perlative}
\gll bìssar hathu buurung derang=pe atthas=\textbf{dering} su-thìrbang. \\
      big \textsc{indef} bird \textsc{3pl}=\textsc{poss} top=\textsc{abl} \textsc{past}-fly \\
\z
}\\


Reciprocal grounding is also possible as in \xref{ex:space:figgr:reciprocal}, where the two places are close to each other. This is highlighted by the use of the dative marker \em =nang \em on both, as well as the use of two additive markers \em =le\em.


\xbox{16}{
 \ea \label{ex:space:figgr:reciprocal}
   \gll   spaaman anà-niinggal \textbf{thumpath=nang=le}        Passara   katha arà-biilang    \textbf{nigiri=nang=le} \textbf{dìkkath}. \\
 	`The place where he died and the village called Passara are close to each other.'(B060115nar05)
\z
}


\begin{table}
	\centering
		\begin{tabular}{ll}
			on & atthas\\
			under & baawa\\
			in front & duppang\\
			behind &blaakang\\
			next to & dìkkath\\
			in & daalang\\
  			out & luuwar\\
			around\\
			in the middle& thingga\\
			thangang naasi & right\\
 			thangang kiiri & left\\
		\end{tabular}
	\caption[Spatial orientation]{Spatial orientation}
	\label{tab:SpatialOrientation}
\end{table}




\subsection{Indicating the spatial orientation of an event}\label{sec:func:Indicatingthespatialorientationofanevent}
The spatial orientation of an event is indicated by attaching the clitics for the semantic roles of source, goal and location \funcref{sec:func:Source} \funcref{sec:func:Goal} to the  NP expressing the place.
\xbox{16}{
\ea \label{ex:space:orientation}
\gll itthu    baathu=yang    incayang Seelong=\textbf{dering} laayeng nigiri=\textbf{nang} asà-baapi. \\
 \textsc{dist} stone=\textsc{acc} 3s.polite Ceylon-\textsc{abl} other country=\textsc{dat} \textsc{cp}-bring\\
\z
}


The deictic adverbs \em siini \em and \em siithu \em  are never combined with \em =nang \em to indicate  goal.

\xbox{16}{
\ea \label{ex:space:orientation:nodeic}
\gll {\em second} {\em world} {\em war}=watthu siini\textbf{=zero} dhaathang aada atthu  kappal. \\
      second world war=time here come exist \textsc{indef} ship \\
    `During the second world war a ship came here.' (K051206nar07.80)
\z
} \\

\xbox{16}{
\ea \label{ex:space:orientation:nodeic2}
\gll [boole oorang pada]=nang   siithu\textbf{=zero} boole=pii. \\
     can man \textsc{pl}=\textsc{dat} there can=go  \\
    `All men who are able to go may go.'  (B060115cvs01.87)
\z
}\\

They can be combined with the ablative marker \em =dering\em, as shown in the following example.

\xbox{16}{
\ea \label{ex:space:orientation:deic:dering}
\gll oorang mlaayu siithu=\textbf{dering}  dhaathang=apa cinggala  raaja=nang=le anà-banthu. \\
      man Malay there=\textsc{abl} come=after Sinhala king=\textsc{dat}=\textsc{addit} \textsc{past}-help \\
\z
} \\
For the spatial orientation of motion with regard to a figure, the lexical solutions \trs{pii}{go}, \trs{dhaathang}{come} and \trs{baalek}{return} are available. If a container is involved, \trs{maasok}{enter} and \trs{kuluwar}{exit} are used. For vertical motion, the verbs \trs{naaek}{ascend} and \trs{thuurung}{descend} are available. The verb \trs{bawung}{get up} also includes a spatial component.  \xref{ex:space:orientation:thuurung} gives an example of the verb \trs{thuurung}{descend}.


\xbox{16}{
\ea \label{ex:space:orientation:thuurung}
\ea
\gll Ithu oorang=nang baaye=nang nanthok pii=nang blaakang. \\ % bf
     \textsc{dist}=\textsc{dat} man=\textsc{dat} good=\textsc{dat} sleep go=\textsc{dat} after \\
    `After the man had well fallen asleep'
\ex
\gll  pohong=dering baawa=nang asà-\textbf{thuurung}. \\
       tree=\textsc{abl} down=\textsc{dat} \textsc{cp}-descend\\
\ex
\gll  [oorang anà-baawa] samma thoppi=pada asà-ambel. \\ % bf
      man \textsc{past}-bring all hat=\textsc{pl} \textsc{cp}-take\\
\ex
\gll  mà-maayeng=nang su-mulain. \\ % bf
      \textsc{inf}-play=\textsc{dat} \textsc{past}-start \\
\z
\z
}\\


\section{Time}\label{sec:func:Time}


Just like space, time can be anchored in a non-deictic way. The temporal expression \em in 1984 \em corresponds to the spatial expression \em in India\em.\footnote{It is true that \em in 1984 \em is actually anchored in a figure-ground relation with the conventional  birth date of Jesus Christ, but speakers normally do not conceptualize this birth when they use such a time reference.}
As a second possibility, time can be indicated relative to the speech act. We can distinguish the periods before, during and after the speech act.
The temporal relation between a figure and a ground can be expressed in very much the same fashion as for space, as in \em before the war\em, which corresponds to \em before the church\em. This is true for SLM, too. Taking into account the start time, end time, duration and overlap of the two events, we can distinguish a number of situations, such as precedence, subsequence, point coincidence, etc.

The location of time of an event contrasts with its internal temporal structure. We can distinguish phasal information (start, progression, end) from aspectual information (habitual, iterative).






\subsection{Giving non-deictic reference time}\label{sec:func:Givingnon-deicticreferencetime}
The means SLM uses to indicate non-deictic time resemble the means used for absolute space. Temporal reference is marked by means of the postposition \em=ka \em \xref{ex:time:abs:ka}. This postposition is frequently dropped in rapid speech \xref{ex:time:abs:noka}.


\xbox{16}{
\ea \label{ex:time:abs:ka}
\gll Soore\textbf{=ka}, Snow-white=le Rose-red=le derang=pe umma=samma appi dìkkath=ka arà-duuduk ambel.  \\
      Evening=\textsc{loc} Snow.white=\textsc{addit} Rose.Red=\textsc{addit} \textsc{3pl}=\textsc{poss} mother=\textsc{comit} fire vicinity=\textsc{loc} \textsc{simult}-sit take \\
\z
}\\

There are no morphosyntactic limitations for lexical expressions of time, but due to the impossibility of having more than one verbal prefix \citep{Slomanson2007cll}, the presence of a non-tense prefix (i.e. aspectual, phasal, modal, infinitive) blocks the expression of tense on the verb. The resulting verb can then have any time reference.



\xbox{16}{
\ea \label{ex:time:abs:noka}
\gll kithang anapii    pugi \rm {\em week}=\zero{}. \\ % bf
     \textsc{1pl} \textsc{past}-go last week  \\
\z
} \\

Years are indicated by their number in Common Era. For this, English numbers are normally used \xref{ex:time:abs:year}, but Malay numbers can also be found \xref{ex:time:abs:month}. Also months are indicated by their English names \xref{ex:time:abs:month}.

\xbox{16}{
\ea \label{ex:time:abs:year}
\gll suda \textbf{\em nineteen-{\em ninety}-four}=ka        se=ppe    {\em husband} su-nniinggal. \\
      thus nine-teen-ninety-four=\textsc{loc} \textsc{1s=poss} husband \textsc{past}-die \\
\z
} \\
\xbox{16}{
\ea \label{ex:time:abs:month}
\gll limapulunnam=ka, \textbf{\rm April}  buulang=dìkka. \\
56=\textsc{loc} april month=vicinity \\
`In '56, around the month of April.' (K060108nar01.13)
\z
}

% \xbox{16}{
% \ea\label{ex:func:unreferenced}
% \gll mampus    blaakang sriibus  sbiilan-raathus     nnampuulu     thuuju thaaun watthu laskalli inni     Melayu Kumpulan    di Kandi=yang         kumpulkang. \\
% croak  after thousand nine-hundred  sixty seven year time other.time \textsc{dist} Melayu Kumpulan Di Kandi-yang add-caus\\
% `.' (K060116nar02.4)
% \z
% }



For the days of the week as given in Table \ref{tab:DaysOfTheWeek}, both native and English words are used.

\begin{table}
\centering
 \begin{tabular}{ll}
Monday & (h)ari sinnen\\
Tuesday&(h)ari slaasa\\
Wednesday& (h)ari rubbo \\
Thursday& (h)ari k(h)uumis \\
Friday&(h)ari jumahath \\
Saturday&(h)ari satthu \\
Sunday&(h)ari ahath/ahadh \\
 \end{tabular}
 \caption[Days of the week]{Days of the week. The letters in parentheses might be prefered by speakers in the orthography, but are not present phonemically.}
 \label{tab:DaysOfTheWeek}
\end{table}


The hours of the clock are indicated by preposing \em pu(ku)l \em to the number.

\xbox{16}{
\ea\label{ex:func:time:abs:hour}
\gll  pukul ìnnam. \\ % bf
 hit six\\
`6 o'clock.' (B060115cvs09)
\z
}

Special words exist for `noon' (\em thinggaari\em) and `midnight'(\em thinggamaalang\em)
Fractions of the hour are expressed by \trs{spaaru}{half} and \trs{kaarthu}{quarter}.

\subsection{Giving the reference time with regard to the speech act}\label{sec:func:Givingthereferencetimewithregardtothespeechact}
Besides non-deictic time reference, events can be situated with regard to the speech act, to wit, before, during and after the speech act.
We can distinguish lexical solutions  from grammaticalized solutions.  (Cf. Table \ref{tab:TemporalAdverbs}.

 \begin{table}
	\centering
		\begin{tabular}{ll}
			itthu muusingka & in former times\\
			kumareng dovulu & the day before yesterday\\
			kumareng & yesterday\\
			nyaari & today\\
			beeso(na) & tomorrow\\
			luuso(na) & (the/some) day after tomorrow\\
			pugi\footnotemark X & last X\\
			hatthiya X & next X\\
		\end{tabular}
	\caption{Temporal adverbs}
	\label{tab:TemporalAdverbs}
\end{table}\footnotetext{This is a development from historical \trs{*piggi}{go}, \citep[214]{Adelaar2005struct}.}


As for grammaticalized solutions, SLM features a developed TAM system, which is used to convey temporal meaning.
There are no morphosyntactic limitations for lexical expressions of time, but due to the impossibility of having more than one verbal prefix \citep{Slomanson2007cll}, the presence of a non-tense prefix (i.e. aspectual, phasal, modal, infinitive) blocks the expression of tense on the verb. The resulting verb can then have any time reference.
Both lexical and grammatical solutions will be discussed in more detail for the different logical possibilities.

\subsubsection{Before speech act}\label{sec:func:Beforespeechact}
The first logical possible constellation is that the reference time precedes the time of the speech act. This constellation can be expressed by both lexical and grammatical means. Among the lexical means, we number temporal adverbs like \trs{kumaareng}{yesterday}{} or less specific \trs{(ka)thaama}{earlier}{}, as well as constructions involving the adverb \trs{pugi}{last}{} or the distal deictic \em itthu\em.

\xbox{16}{
\ea \label{ex:func:time:deic:before:kumaareng}
\gll \textbf{kumaareng}=le thuuju=so dhlaapan=so oorang asà-buunung, ... \\
      yesterday=\textsc{addit} seven=\textsc{undet} eight=\textsc{undet} man \textsc{cp}-kill  ... \\
    `Again yesterday, seven or eight men were killed .'  (K051206nar11)
\z
}\\

\xbox{16}{
\ea \label{ex:func:time:deic:before:kethaama}
\gll   \textbf{kethaama} kithang kiccil muusing=ka   inni     Peeradheniya jaalang=ka samma an-aada      mlaayu. \\
       before \textsc{1pl} small time=\textsc{loc} \textsc{prox} Peradeniya road=\textsc{loc} all \textsc{past}-exist Malay\\
\z
} \\
\xbox{16}{
\ea \label{ex:func:time:deic:before:pugi}
\gll kithang anapii    \textbf{pugi} \rm {\em week}. \\
    `We went last week.' (K051206nar07)
\z
} \\
\xbox{16}{
\ea \label{ex:func:time:deic:before:itthumuusing}
\gll \textbf{itthu}    \textbf{muusing=ka}    cinggala  thraa. \\
     \textsc{dist} time=\textsc{loc} Sinhala \textsc{neg}  \\
    `At that time, there was no Sinhala.' (K051222nar06)
\z
} \\


\em (ke)thaama \em can also be used to indicate the amount of time that lies between the event and the speech act.

% \xbox{16}{
% \ea\label{ex:func:unreferenced}
% \gll itthu awuliya karang sraathus liimapuulu thaaun=nang thaama. \\
%  \textsc{dist} saint now hundred fifty year=\textsc{dat} earlier\\
% `That saint (lived) 150 years ago.' (nosource)
% \z
% }
%
% \xbox{16}{
% \ea\label{ex:func:unreferenced}
% \gll karang atthu  spuulu thaaun=nang=ke thaama. \\
%  now one ten years=\textsc{dat}=\textsc{simil} earlier\\
% `About ten years ago.' (nosource)
% \z
% }

\xbox{16}{
\ea \label{ex:func:time:deic:before:thaama}
   \gll  \textbf{s-raathus} \textbf{limapulu}    \textbf{thaaun=nang}   \textbf{thaama}   incayang  bannyak igaama=pe atthas sbaayang naaji. \\
`150 years  ago,  he recited a lot about religion,' (K051220nar01)
\z
}

The absolute use and the relative use of \em (ke)thaama \em are exemplified at the same time in the following passage.

\xbox{16}{
\ea \label{ex:func:time:deic:before:thaama:double}
\ea
\gll se=ppe    baapa  dhaathangapa {\em government} {\em servant}. \\ % bf
      \textsc{1s=poss} father cop government servant \\
    `My father was a government servant.'
\ex
\gll inni     {\em railway} {\em department}=ka    {\em head} {\em guard} hatthu \textbf{kethaama}. \\
      \textsc{prox} railway department=\textsc{loc} head guard \textsc{indef} before \\
    `He was a head guard in the railway department before.'
\ex
\gll itthu=nang      \textbf{kethaama} incayang  {\em second} {\em world} {\em war}=ka    CLI Seelon {\em lightning} {\em infantery}     katha athu   {\em soldier}. \\
    \textsc{dist}=\textsc{dat} before 3s.polite second world war=\textsc{loc} CLI Ceylon lightning infantery \textsc{quot} \textsc{indef} soldier   \\
\z
\z
} \\
If time reference is inferable from discourse, non-verbal predicates need not carry overt indication of time reference. In example \xref{ex:time:before:nocoding}, the  preceding sentence  has established past time reference, and the nominal predication in \xref{ex:time:before:nocoding} need not be marked for time reference.

\xbox{16}{
\ea \label{ex:time:before:nocoding}
\gll  Itthu bannyak laama hathu ruuma. \\ % bf
      \textsc{dist} very old \textsc{indef} house \\
    `That one was a very old house.'  (K070000wrt04)
\z
}\\


Lexical solutions are available for both verbal and non-verbal sentence types. Grammatical solutions on the other hand can only be employed with verbs (or converted adjectives).


Verbal predications with past reference are normally indicated by the prefixes \em anà- \em \formref{sec:morph:anà-} and \em su-\em \formref{sec:morph:su-}.

\xbox{16}{
\ea \label{ex:time:before:verb:ana}
\gll 58=ka=jo \textbf{anà-}mulain. \\
      58=\textsc{loc}=\textsc{foc} \textsc{past}-start \\
\z
}\\

\xbox{16}{
\ea \label{ex:time:before:verb:su}
\gll aanak pompang duuwa=nang slaamath katha \textbf{su-}biilang. \\
     child girl two=\textsc{dat} goodbye \textsc{quot} \textsc{past}-say  \\
\z
}\\


Location in time before the speech act can also be expressed by  the  perfect construction \formref{sec:wc:Theperfecttenses}. In distinction to the English present perfect, this does not imply relevance for the time of speaking. Example \xref{ex:time:before:perfect:perfect} shows the use of the perfect tense in a context relevant to the time of speaking, while  \xref{ex:time:before:perfect:past} gives an example of the perfect construction being used in a context without relevance to the speech situation.


\xbox{16}{
\ea \label{ex:time:before:perfect:perfect}
\gll see=le     \textbf{pii} \textbf{aada}  dhraapa=so duuwa thiiga skalli. \\
    `I myself have been there, how many, maybe two or three times.' (B060115nar05)
\z
} \\
\xbox{16}{
\ea \label{ex:time:before:perfect:past}
\gll {\em Dutch} {\em period}=ka derang pada \textbf{dhaathang} \textbf{aada}. \\
`They came during the Dutch period' (K051206nar05)
\z
}


Events  which are asserted to not have happened in the past take the negation marker \em thàrà- \em \xref{ex:time:before:past.neg} or postverbal \em thraa \em in the perfect \xref{ex:time:before:perfect:neg}. As with the affirmative, this does not entail any relevance for the time of speaking, unlike English present perfect. \xref{ex:time:before:past.neg} refers to a time long ago and has no relevance for the speech act, while \xref{ex:time:before:past.neg:relevantpresent} refers to a point in time just before the speech act, and has a consequence for the speech act, namely that the story will be retold in Malay.

\xbox{16}{
\ea \label{ex:time:before:past.neg}
\gll S. {\em subscription} \textbf{thàrà}-kiiring \\
    `S. did not send the subsription.' (K060116nar10)
\z
} \\

\xbox{16}{
\ea \label{ex:time:before:perfect:neg}
\gll kithang baaye mlaayu arà-oomong katha incayang \textbf{biilang} \textbf{thraa}. \\
    `He has not said that we speak good Malay.'  (B060115prs15)
\z
}\\

 \xbox{16}{
\ea \label{ex:time:before:past.neg:relevantpresent}
\gll  inni=yang see mlaayu=dering \textbf{thàrà}-biilang. \\
      \textsc{prox}=\textsc{acc} \textsc{1s} Malay=\textsc{abl} \textsc{neg.past}=say \\
\z
} \\
Non-verbal predications do not show any special negative marking for the past. This is exemplified for an existential predication in \xref{ex:time:before:nonv:loc:neg}.


\xbox{16}{
\ea \label{ex:time:before:nonv:loc:neg}
\gll oorang pada=nang   hathu  oorang=nang   creeweth \textbf{thraa}. \\
     man \textsc{pl}=\textsc{dat} \textsc{indef} man=\textsc{dat} trouble \textsc{neg}  \\
\z
} \\
Subordinate clauses which carry an indication of relative tense (simultaneous (\em arà-\em) or anterior \em asà-\em) cannot express absolute tense. They then inherit the tense meaning of the matrix clause. In example \xref{ex:time:before:simult1}, the verb \trs{kuthumung}{see}{} in the matrix clause is in the past tense, indicated by \em anà-\em. The verb \trs{ambel}{take}{} in the subordinate clause has the anterior prefix \em asà-\em, while \trs{maayeng}{play}{} in the subordinate clause has the prefix \em arà-\em, indicating the simultaneity of seeing and playing and the anteriority of taking with respect to seeing. Neither of the verbs in the subordinate clause is marked for past tense, but the past tense meaning is inherited from the matrix clause.

\xbox{16}{
\ea \label{ex:time:before:simult1}
\gll Blaakang=jo incayang \textbf{ana}-kuthumung [moonyeth pada thoppi \textbf{asa}-ambel pohong atthas=ka \textbf{arà-}maayeng]. \\
     after=\textsc{foc} 3s.polite \textbf{past}-see monkey \textsc{pl} hat \textbf{anterior}-take tree top=\textsc{loc} \textbf{simult}-play  \\
\z
}\\


Another example is \xref{ex:time:before:simult2} where the act of hearing is simulatenous to the crying. Note that the conjunctive participle does not imply anteriority in this case, but rather coordination. Further note that we are dealing with past reference.

\xbox{16}{
\ea \label{ex:time:before:simult2}
\gll [Banthu-an asà-mintha \textbf{ara}-naangis] swaara hatthu derang=nang su-dìnggar. \\
      help-\textsc{nmlzr} \textsc{cp}-beg \textsc{simult}-cry sound \textsc{indef} \textsc{3pl}=\textsc{dat} \textsc{past}-hear\\
\z
}\\


Durative or habitual actions in the past can also be coded by the progressive marker \em arà- \em. This overrides the expression of past tense. The following two examples show durative marking \xref{ex:time:before:override:durative} and habitual marking \xref{ex:time:before:override:habitual}.

\xbox{16}{
\ea \label{ex:time:before:override:durative}
\gll itthu    kumpulan=dang      derang=jo     bannyak \textbf{ara}-banthu. \\
      \textsc{dist} association=\textsc{dat} \textsc{3pl}=\textsc{foc} much \textsc{non.past}-help\\
\z
}\\

\xbox{16}{
\ea \label{ex:time:before:override:habitual}
\gll itthu    kalu, [...] Dubai=ka    asà-duuduk     laama kar pada kitham  \textbf{ara}-baapi       Iraq  {\em ports}=nang  \\
 \textsc{dist} if [...] Dubai-\textsc{loc} \textsc{cp}-stay old car \textsc{pl} \textsc{1pl} \textsc{non.past}-take Iraq ports=dat\\
\z
}

The use of \em arà- \em in past contexts is clearly seen in \xref{ex:time:before:override:habitual:double}, where we have a non-deictic indication of time (1958), but the verb is not marked with a past tense prefix, but with \em arà-\em, which marks habitual in this case.

\xbox{16}{
\ea \label{ex:time:before:override:habitual:double}
\gll muula     perthaama, Badulla  ruuma saakith=ka    \textbf{s-riibu}   \textbf{sbiilan} \textbf{raathus} \textbf{lima-pulu}    \textbf{dhlaapan=ka} pukurjan \textbf{ara}-gijja    wakthu. \\
    `Before, when I was working in Badulla in 1958.' (K051213nar01)
\z
} \\

%
% \xbox{16}{
% \ea\label{ex:func:unreferenced}
% \ea\label{ex:func:unreferenced}
% \gll kitham=pe      oorang thuwa pada  bannyak dhaathang aada {\em Malaysia}=dring. \\
%      \textsc{1pl}=\textsc{poss} man old \textsc{pl} many come exist Malaysia=\textsc{abl}  \\
%       `Many of our ancestors came from Malaysia,'
% \ex
% \gll spaaru indonesia=dring      dhaathang aada. \\
%      some Indonesia=\textsc{abl} come exist\\
%        `some came from Indonesia.' (K060108nar02.73)
% \z
% \z
% } \\
% Past time reference for non-verbal predications is normally not expressed. When necessary, the adverb \trs{thaama}{earlier} can be used. \em thaama \em must not be confused with \trs{thamau}{\textsc{neg.nonpast}}\formref{sec:form:thamau}.
%
%
% \xbox{16}{
% \ea\label{ex:func:unreferenced}
% \gll Seelon=ka {\em English} thaama aada. \\
%  Ceylon=\textsc{loc} English earlier BE\\
% `There is/will be no English in Sri Lanka.' (nosource)
% \z
% }



Lexical and grammatical indication of time can be combined, as in \xref{ex:time:before:lexgram:double}, where we find the lexical marker \em kathaama \em and the grammatical marker \em nya-\em.

\xbox{16}{
\ea \label{ex:time:before:lexgram:double}
\gll see \textbf{kathaama} pukurjan \textbf{nya}-kirja. \\
     \textsc{1s} earlier work \textsc{past}-do  \\
    `I used to work in former times.' (B060115prs01)
\z
} \\

\subsubsection{Simultaneous to speech act}\label{sec:func:Simultaneoustospeechact}
Temporal situation conceived of as simulatneous to the speech act can be expressed by a limited number of lexical means, or by the verb prefix \em arà-\em\citet[164]{SmithEtAl2007}. Lexical solution include the use of \trs{(s)kaarang}{now}{}, \trs{nyaari}{today} or \trs{ini X}{this X}, where X is a temporal noun like \trs{thaaun}{year} or \trs{muusing}{time, period}. The following sentences are examples of simultaneity to the speech act expressed lexically.


\xbox{16}{
\ea \label{ex:func:time:simult:skaarang}
\gll suda \textbf{skaarang}    kitham=pe      aanak pada    laaeng   pukurjan pada    arà-girja. \\
      thus now \textsc{1pl}=\textsc{poss} child \textsc{pl} other work \textsc{pl} \textsc{non.past}-make \\
\z
} \\

\xbox{16}{
\ea \label{ex:func:time:simult:nyaari}
\gll itthusubbath=jo    \textbf{nyaari}            go  laile  arà-duuduk. \\
     therefore=\textsc{foc} today \textsc{1s.familiar} again \textsc{non.past}-stay  \\
    `That's why I still stay (here).' (B060115nar04)
\z
} \\

\xbox{16}{
\ea \label{ex:func:time:simult:inimuusing}
\gll ini muusing=dika oorang ikkang Hambanthota ka ara duuduk. \\
     \textsc{prox} time=\textsc{loc} man fish Hambantota=\textsc{loc}   \\
    `Presently, the fishermen are at Hambantota.' (test)6.11.08
\z
} \\


Events taking place in the time frame of the speech act are coded by \em arà- \em for verbal predicates \xref{ex:func:time:simult:ara}\formref{sec:morph:ara-}, and \zero-coded at all for other predicates \xref{ex:func:time:simult:zero}.


\xbox{16}{
\ea \label{ex:func:time:simult:ara}
\gll  suda itthu    oorang pada=le      Seelong=ka   \textbf{ara}-duuduk. \\
      thus \textsc{dist} man \textsc{pl}=\textsc{addit} Ceylon=\textsc{loc} \textsc{non.past}-stay \\
\z
} \\


\xbox{16}{
\ea \label{ex:func:time:simult:zero}
\gll suda, se=ppe      femili \zero{} itthu=jo. \\
 thus \textsc{1s=poss} family { } \textsc{dist}=foc\\
\z
}

Simultaneity can be conceived in a wide sense\citet[164]{SmithEtAl2007}. In example \xref{ex:func:time:simult:wide} about a Chinese converted to Islam, it is not clear whether he is going to the mosque right now. Simultaneity to the time of speaking is conceived in a wider sense here, meaning that at the time of speaking the convert has the habit of going to the mosque, without vouching for his going at the present moment.



\xbox{16}{
\ea \label{ex:func:time:simult:wide}
\ea
\gll  ini   ciina oorang Islam=nang   asà-dhaathang. \\ % bf
   \textsc{prox} China man Islam=\textsc{dat}  \textsc{cp}-go  \\
    `That Chinaman came to Islam and'
\ex
\gll  ini  asà-kaaving=apa. \\ % bf
 \textsc{prox} \textsc{cp}-marry=after      \\
\ex
\gll  \textbf{karang} masiigith=nang  \textbf{ara}-pii  liima wakthu sbaayang=nang. \\
  now mosque=\textsc{dat} \textsc{non.past}-go five time pray=\textsc{dat}     \\
\z
\z
} \\


% The time frame for simultaneity can be conceived wider or narrower. In example \xref{ex:dehrampadayang}, it is not sure whether at the time of speaking, someone actually send relatives abroad. But if the speech situation is conceived as encompassing a greater period, like a month or so, the event of speaking and sending abroad coincide.
%
%
% \xbox{16}{
% \ea\label{ex:func:unreferenced}
% \gll deram pada=yang laayeng nigiri pada=nang arà-kiiring. \\
%  \textsc{3pl} \textsc{pl}=\textsc{acc} other country \textsc{pl}=\textsc{dat} \textsc{non.past}-send\\
% \z
% }


Negated predicates with time reference simultaneous to the speech act are marked with \em thamau\em\xref{ex:func:time:simult:neg:thamau}\formref{sec:morph:thamau-} for verbs. All other predication types take their standard negation, which is the same regardless of time reference.



\xbox{16}{
\ea \label{ex:func:time:simult:neg:thamau}
\gll kitham=pe      aanak pada \textbf{thama}=oomong. \\ % bf
 \textsc{1pl}=\textsc{poss} child \textsc{pl} \textsc{neg.nonpast} speak \\
\z
}

\subsubsection{After speech act}\label{sec:func:Afterspeechact}
Temporal reference to a point in time after the speech act can be made with some lexical means, and grammatical means for verbal predications. Lexical solutions include \trs{beeso}{tomorrow}{} or \trs{(beeso) luusa}{after tomorrow}.

\xbox{16}{
\ea \label{ex:time:after:beeso}
\gll  ruuma birsi=nang arà-simpang. \textbf{Beeso}=nang kithang arà-mnaaji \\
      house clean=\textsc{dat} \textsc{non.past}-keep. Tomorrow=\textsc{dat} \textsc{1pl} \textsc{non.past}-pray \\
\z
} \\
\xbox{16}{
\ea \label{ex:time:after:beesoluusa}
\gll \textbf{beeso} \textbf{luusa} lubaarang arà-dhaathang. \\
    `The day after tomorrow is the festival.'
\z
} \\

Additionally, \trs{duppang}{future} can be used to indicate temporal reference to after the speech act. Note that the relator noun \trs{duppang}{before}, constructed with \em =nang\em, is used for past reference, while the full noun \trs{duppang}{future}  is used for future refenrence. This is similar to German, where the preposition \trs{vor}{ago} is used to indicate a temporal distance in the past, while the future is said to lie before us \em vor uns\em) as well, whereas the past lies behind (\em hinter\em).


\xbox{16}{
\ea \label{ex:time:after:duppang:contrast}
\gll  kithang=\textbf{nang} \textbf{duppang}$_{reln}$ lai duuwa bàrgaada asà-dhaathang aada. \\
`Before us, there were two other families.' (K060108nar02)
\z
}


\xbox{16}{
\ea
\gll  \textbf{duppang} \textbf{muusing}$_{fullnoun}$=ka=le Dodangwela aapacara=le thama-bìssar. \\
    `Even in the future, Dodangwela [a village close to Kandy]  will not be big.' (test)6.11.08
\z
} \\
\xbox{16}{
\ea
\gll \textbf{duppang} \textbf{muusing}$_{fullnoun}$=ka oorang ikkang Negombo=nang anthi-pii. \\
    `In the future, the fishermen will go to Negombo.' (nosource)6.11.08
\z
} \\
As for grammatical solutions, verbal predications which are thought to take place after the speech act are either marked with \em arà- \em \xref{time:after:ara} or \em anthi-\em\xref{time:after:anthi}.


% \xbox{16}{
% \ea\label{ex:func:unreferenced}
% \gll laskalli arà-maakang. \\
%  other.time \textsc{non.past}-eat\\
% `You will come and eat another time.' (B060115cvs16.40)
% \z
% }

\xbox{16}{
\ea \label{time:after:ara}
\gll paanas muusing dhaathang=thingka see siini=dering \textbf{ara}-pii. \\
      hot season come=middle \textsc{1s} here=\textsc{abl} \textsc{non.past}-go\\
\z
}\\

\xbox{16}{
\ea \label{time:after:anthi}
\gll ithu=kapang lorang=pe leher=yang kithang \textbf{athi}-poothong. \\
     \textsc{dist}=when \textsc{2pl}=\textsc{poss} neck=\textsc{acc} \textsc{1pl} \textsc{irr}-cut  \\
\z
} \\
Non-verbal predicates technically do not have to be marked for future reference. However, asserting the future truth of a proposition often implies pragmatically that it is not true at the time of speaking, so that a construction conveying this change of state from false to true is prefered. This can be done by using adjectives in a verbal predication \xref{ex:func:time:after:adj}, or by using a construction involving \trs{jaadi}{become}\xref{ex:func:time:after:jaadi}\formref{sec:wc:Specialconstructionsinvolvingverbalpredicates}. \em Jaadi \em will bear the irrealis marker \em anthi- \em then.

\xbox{16}{
\ea \label{ex:func:time:after:adj}
\gll ithukapang gaathal \textbf{anthi}-kuurang. \\
	 then itching \textsc{irr}-less\\
    `Then the itching will become less.'  (K060103cvs02)
\z
}\\


\xbox{16}{
\ea \label{ex:func:time:after:jaadi}
\gll incayang hatthu guru athi/*arà-jaadi. \\
      3s.polite \textsc{indef} teacher \textsc{irr}/\textsc{non.past}-become \\
    `He will become a teacher.' (test)6.11.08
\z
} \\
Considering negation for propositions refering to a point in time after the speech act, \em thamau- \em is employed for verbs \xref{ex:func:time:after:thamau:v} and adjectives \xref{ex:func:time:after:thamau:adj}, while other predicates will use a periphrasis to indicate that the future state will not come into being\xref{ex:func:time:after:thamajaadi}. In the rare case that the speaker wants to assert that a state not true at the time of speaking will not be true in the future either, this can be done by a normal non-verbal predication, but lexical material is required to make the time reference clear \xref{ex:time:after:nonpresnonfut}.

\xbox{16}{
\ea \label{ex:func:time:after:thamau:v}
\gll  See lorang=nang   \textbf{thama-}sakith-kang. \\
      \textsc{1s} \textsc{2pl}=\textsc{dat} \textsc{neg.nonpast}-sick-\textsc{caus} \\
\z
} \\
\xbox{14}{
\ea \label{ex:func:time:after:thamau:adj}
\gll inni pukuran=yang mà-gijja thamau-gampang \\
     \textsc{prox} work=\textsc{acc} \textsc{inf}-make \textsc{neg.irr}-easy  \\
\z
} \\

\xbox{16}{
\ea \label{ex:func:time:after:thamajaadi}
\gll se hatthu guru thamajaadi. \\
      \textsc{1s} \textsc{indef} teacher \textsc{neg.irr}-become \\
\z
} \\

\xbox{16}{
\ea \label{ex:time:after:nonpresnonfut}
\gll  see \textsc{innam} \textsc{blaakang} hatthu aanak bukang. \\
    `I will never be a child again.' (nosource)6.11.08
\z
} \\

% \xbox{16}{
% \ea
% \gll  se hatthu guru ma jaadi thàrà suuka. \\
%        \\
%     `.' (nosource)
% \z
% } \\

\subsubsection{General truth}\label{sec:func:Generaltruth}
General truth independent of time frame is expressed with the progressive marker \em arà- \em (\em thamau- \em in the negative) for verbal predicates and is not coded overtly for non-verbal predicates.

% \xbox{16}{
% \ea \label{ex:CeylonAirportyang}
% \gll Seelong {\em Airport}=yang duwa-pulu-empath wakthu=le asà-bukka arà-simpang kiyang. \\
%      Ceylon Airport=\textsc{acc} two-ty-four hour=\textsc{addit}  \textsc{cp}-open \textsc{non.past}-stay evid\\
%     `The Ceylon Airport will stay open 24h, it seems. (lit: They will open it and keep it [like that])'  (Letter 26.06.2007)
% \z
% }\\
%
In example \xref{ex:func:time:general:tony}, the speaker's name Tony is true irrespective of the time of speaking. In example \xref{ex:func:time:general:sunnath}, the circumcision at the fortieth day is a general truth as well which is true about circumcisions in the past, present and future alike. Both are coded by \em arà-\em.

\xbox{16}{
\ea \label{ex:func:time:general:tony}
\gll see=yang    Tony katha \textbf{ara}-panggel. \\
1s=\textsc{acc}  Tony \textsc{quot} \textsc{non.past}-call\\
\z
}

\xbox{16}{
\ea \label{ex:func:time:general:sunnath}
\gll karam=pe mosthor=nang, mpapulu aari=ka=jo sunnath=le \textbf{ara}-kijja. \\
 now=\textsc{poss} manner=\textsc{dat} forty   day=\textsc{loc}=\textsc{foc} circumcision=\textsc{addit} \textsc{non.past}-make  \\
\z
}\\

General truth can sometimes be difficult to distinguish from habitual aspect. In this grammar, I distinguish general truth, coded by \em arà-\em, from habitual, which can be coded by \em arà- \em or \em anthi-\em, but this is somewhat arbitrary. See Section \ref{sec:func:asp:Habitual} for a discussion of habitual aspect.

%
% \xbox{16}{
% \ea\label{ex:func:unreferenced}
% \gll hatthu {\em chess}  {\em {\em team}}=nangìnnam arà-duuduk. \\
%  one Chess team=\textsc{DAT} six \textsc{non.past}-stay\\
% `There are six (players) in one chess team.' (B060115prs20.28)
% \z
% }
%
% Also chess teams consist of six players, regardless of time reference.
%
% pompang pada  derang muuka ara thuuthup
%
% Definitions are also coded by \em arà- \em as a general truth. This is the case in the following example.
%
%
% \xbox{16}{
% \ea\label{ex:func:unreferenced}
% \gll {\em majority} katha arabiilang    hathu  liiwath blaangan {\em votes}=dering   su-bunnang. \\
%      mahority \textsc{quot} \textsc{non.past}-say \textsc{indef} more amount votes=\textsc{abl} \textsc{past}-win \\
% \z
% } \\
The normal negation of \em arà- \em is \em thamau-\em. This prefix is also used to negate propositions with general time reference. Example \xref{ex:func:time:general:neg} shows general truth of a negative predication: Hindus generally do no eat beef as a rule, which again is independent of time reference to present past or future.  This is coded by \em thuma\em, an allomorph of \em thamau\em.

\xbox{16}{
\ea \label{ex:func:time:general:neg}
	\ea
 \gll Hindu \textbf{ara}-maakang kambing. \\
	Hindu \textsc{non.past}-eat goat\\
	`Hindus eat goat.'
	\ex samping \textbf{thuma}-maakang \\
		beef \textsc{neg.nonpast}-eat\\
	` (they) don't eat beef.' (K060112nar01)
	\z
\z
}


General truth of non-verbal predications does not receive special marking. In example \xref{ex:func:time:general:nonv} the general truth of the speaker's profession being engineering is conveyed by the copula, which can also be used with other time references.


\xbox{16}{
\ea \label{ex:func:time:general:nonv}
\gll  se=ppe percariyan asdhaathang {\em engineering}. \\
      \textsc{1s=poss} earning \textsc{copula} engineering \\
    `My profession is engineer.' (K061026prs01)
\z
} \\
\subsection{Figure and ground in the temporal domain}\label{sec:func:Figureandgroundinthetemporaldomain}
Just like two entities can be related to each other in space, this is also possible in time. Depeding on start point, end point, duration and overlap of the two events, a certain number of semantic constellations are possible.\footnote{This list is inspired by \citep[330]{Givon2001}.}  In SLM, all of them can be expressed lexically, and some can also be expressed grammatically. Very often, relator nouns are employed to indicate the precise relationship.

Any point in time can be refered to anaphorically by the distal deictic \em itthu\em. The proximal deictic \em ini \em  on the other hand specifies the time of the speech act as temporal grounding. The precise kind of relation is then indicated by a relator noun like \trs{blaakang}{after}, with  dative marker \em =nang \em added to \em ini\em. In the process \em ini=nang \em is contracted to \em ìnnam\em (not to be counfounded with \trs{ìnnam}{six}).

\xbox{16}{
\ea
\gll  see innam blaakang hatthu aanak bukang. \\
     \textsc{1s} \textsc{prox.dat} after \textsc{indef} child \textsc{neg.nonv} \\
\z
} \\


\subsubsection{Precedence}\label{sec:func:Precedence}
Precedence obtains when the time frame of the figure precedes the time frame of the ground.

\ea $\stackrel{fig}{\multimap}~~\stackrel{gr}{\multimapdotinv}$\z

In SLM, it is coded by \trs{duppang}{before} or \trs{(ke)thaama}{before}.


\em Duppang \em is used as an adverb in this function. The ground is coded with the dative marker \em =nang\em, while the figure does not receive special marking. The figure can be expressed by a noun, as in \xref{time:precedence:duppang:adv:n}, or by a verb in the infinitive, as in \xref{time:precedence:duppang:adv:v}


\xbox{16}{
\ea \label{time:precedence:duppang:adv:n}
\gll itthu    blaakang=jo,    [\textbf{kitham=pe} \textbf{AGM}]$_{ground}$=nang  duppang,  [{\em high} {\em commissioner} {\em cultural} {\em show}=na mà-dhaathang=nang        \textbf{thàràboole} s-\textbf{jaadi}]$_{figure}$. \\
    After that, before our Annual General Meeting, it became impossible for the High Commissioner to attend the cultural show.    (K060116nar23.39)
\z
} \\

\xbox{16}{
\ea \label{time:precedence:duppang:adv:v}
\gll Itthule [see=yang mà-kiiring=nang]$_{ground}$ duppang [incayang see=yang hathu Buruan mà-jaadi su-bale-king]$_{figure}$. \\ % bf
     but \textsc{1s}=\textsc{acc} \textsc{inf}-send=\textsc{dat} before 3s.polite \textsc{1s}=\textsc{acc} \textsc{indef} bear \textsc{inf}-become \textsc{past}-turn-\textsc{caus}  \\
\z
} \\
If the ground is not given, the time of the speech situtation is taken as a default. In that case, the amount of time separating the event from the speech act can be indicated by \em =nang\em, as in \xref{time:precedence:duppang:adv:speechsituation}.


\xbox{16}{
\ea \label{time:precedence:duppang:adv:speechsituation}
\gll  \zero$_{ground}$ Sdiikith thaaun=nang duppang, [see ini Aajuth=nang su-kìnna daapath]$_{figure}$. \\ % bf
     { }  few year=\textsc{dat} before \textsc{1s} \textsc{prox} dwarf=\textsc{dat} \textsc{past}-patfoc get \\
\z
} \\

The situation is similar with \em (ke)thaama\em. Here as well, the ground is marked by \em =nang. \em The following example also shows that  a deictic can serve as ground.


\xbox{16}{
\ea \label{time:precedence:kethaama}
\gll {}[inni]$_{ground}$=nang       kethaama [see Nawalapitiya=ka    duuduk aada]$_{figure}$. \\ % bf
      \textsc{prox}=\textsc{dat} before \textsc{1s} Nawalapitiya=\textsc{loc} stay exist \\
    `Before that, I was in Nawalapitiya.' (G051222nar01)
\z
} \\

\subsubsection{Subsequence}\label{sec:func:Subsequence}
Subsequence obtains when the time frame of the ground precedes the time frame of the figure.


\ea $\stackrel{gr}{\multimapdot}~~\stackrel{fig}{\multimapinv}$\z

Subsequence of events need not be coded but can be inferred by the succession of utterances, which is iconic.
Subsequence can be coded lexically by \trs{itthuka(apa)ng}{and then} or \trs{blaakang}{after}. As for grammatical means, the conjunctive participle \em asà- \em very often indicates subsequence.


The non-coding of succeeding events is exemplified by \xref{ex:time:subseq:non}, where the getting up is followed by the search, but no grammatical or lexical element indicates the temporal relation between the two events. This is then inferred to be subsequence as the default case.

\xbox{16}{
\ea \label{ex:time:subseq:non}
\ea
\gll Oorang su-baawung. \\ % bf
      man \textsc{past}-get.up \\
    `The man got up'
\ex
\gll \zero{} thoppi pada yang anà-caari. \\ % bf
      { } hat \textsc{pl} \textsc{pat} past search \\
\z
\z
}\\

A lexical solution for subsequence is \trs{blaakang}{after}. As with precedence, the ground is indicated by \em =nang\em.

\xbox{16}{
\ea \label{ex:time:subseq:blaakang}
\gll { } [itthu]$_{ground}$=nam \textbf{blaakang} [bannyak oorang pada siini se-duuduk]$_{figure}$. \\ % bf
     { } \textsc{dist}=\textsc{dat} after many man \textsc{pl} here \textsc{past}-stay  \\
\z
} \\

Just like \em duppang\em, \em blaakang \em can also be used without overt grounding, in which case it just emphasizes that the event is subsequent to whatever was there before.

\xbox{16}{
\ea \label{ex:time:subseq:blaakang:noground}
\gll   [\zero]$_{ground}$  Blaakang, [incayang=nang    baaye=nang   nanthok su-pii]$_{figure}$. \\  % bf
     { }  after 3s.polite=\textsc{dat} good=\textsc{dat} sleepy \textsc{past}-go  \\
    `Then you add the raw rice.' (K070000wrt01)
\z
} \\

%
% \xbox{16}{
% \ea\label{ex:func:unreferenced}
% \gll Dr Draaman \textbf{duuwa} \textbf{thaaun} \textbf{blaakang} incayang se-nniinggal. \\
% `Dr Draman died two years later.' (nosource)
% \z
% }


The combination of a deictic and a postposition \trs{ithukapang}{then}{}  and the particle \trs{suda}{then} are rather discourse markers\funcref{sec:wofo:Deictic+X}, but also convey a meaning of subsequence.


\xbox{16}{
\ea \label{ex:time:subseq:ithukapang}
\gll \textbf{ithu=kapang}       umma-baapa  su-biilang     lorang=nang   kaaving thàrboole. \\
      \textsc{dist}=then mother-father \textsc{past}-say 2pl.polite=\textsc{dat} marry cannot \\
    `Then the parents said, you cannot marry.' (K051220nar01)
\z
} \\
 \xbox{16}{
\ea \label{ex:time:subseq:suda}
   \gll  \textbf{suda} derang=nang   hathyang muusing sangke mà-duuduk  su-jaadi. \\
    then \textsc{3pl}=\textsc{dat} other time=until \textsc{inf}-stay \textsc{past}-become \\
`So they had to wait until another time'(K051220nar01)
\z
}


There exist three grammaticalized means to express subsequence, the conjunctive participle \em asà- \em\formref{sec:morph:asa-}\citep[cf][]{Slomanson2008lingua}, the postposition \em =apa \em \formref{sec:morph:=apa} and the (plu)perfect construction. \em asà- \em and \em =apa \em are frequently combined, but this is not necessary. The following examples show the exclusive use of \em asà- \em \xref{ex:time:subseq:s} and  \em =apa \em \xref{ex:time:subseq:apa}, and the combination thereof \xref{ex:time:subseq:asa+apa}.


\xbox{16}{
\ea \label{ex:time:subseq:s}
\ea
\gll pirrang=nang anà-dhaathang oorang pada \textbf{asà-}pirrang. \\
 war=\textsc{dat} \textsc{past}-come man \textsc{pl} \textsc{cp}-wage.war\\

\ex
\gll derang=nang \textbf{asà-}banthu. \\
3pl=\textsc{dat} \textsc{cp}-help\\

\ex
\gll siini=jo se=ciinggal. \\ % bf
here=\textsc{foc} \textsc{past}-settled\\
\z
\z
}


\xbox{16}{
\ea \label{ex:time:subseq:apa}
\ea
\gll mlaayu pada  duuduk \textbf{aapa}. \\
 Malay \textsc{pl} stay after\\
 `After the Malays had settled down'
\ex
 \gll spaaru  mlaayu pada   Singapur       Indonesia  {\em Malaysia} anà-pii. \\ % bf
 some Malay \textsc{pl} Singapur Indonesia Malaysia \textsc{past}-go\\
` (only) some Malays went (back) to Singapur, Indonesia or Malaysia.' (K051213nar07.22)
\z
\z
}

\xbox{16}{
\ea \label{ex:time:subseq:asa+apa}
\ea
\gll   thàrà-kalu [ini oorang thoppi arà-kumpul]=yang \textbf{asa}-kuthumung=\textbf{apa}. \\
       \textsc{neg}-if \textsc{prox} man hat \textsc{simult}-collect=\textsc{acc} \textsc{cp}-see after\\
\ex
\gll    moonyeth pada=le asà-dhaathang creeweth  athi-kaasi katha. \\ % bf
       monkey \textsc{pl}=\textsc{addit} \textsc{cp}-come trouble \textsc{irr}=give quot\\
\z
\z
}\\

The pluperfect with \em asà- \em is given in \xref{ex:time:subseq:asa+aada:pluperf1}-\xref{ex:time:subseq:asa+aada:pluperf2}. The theoretical possibility of forming the pluperfect with \em =apa \em was not found in the corpus.


\xbox{14}{
\ea\label{ex:time:subseq:asa+aada:pluperf1}
\gll Incayang mliiga=nang kapang-pii, Raaja hathu thiikar=ka guula \textbf{asa}-siibar mà-kirring simpang \textbf{su}-aada.  \\
     3s.polite palace=\textsc{dat} when-go king \textsc{indef} man=\textsc{loc} sugar \textsc{cp}-spread \textsc{inf}-dry keep \textsc{past}-exist  \\
    `When he was going to the palace, the King had sprinkled sugar in a mat and had left it to dry.'
\z
}

\xbox{14}{
\ea \label{ex:time:subseq:asa+aada:pluperf2}
\gll Itthu bannyak laama hathu ruuma, itthule ruuma duuwa subla=ka panthas rooja kumbang pohong komplok duuwa \textbf{asa}-jaadi \textbf{su}-aada.  \\
      \textsc{dist} very old \textsc{indef} house, but house two side=\textsc{loc} beautiful rose flower tree bush two \textsc{cp}-become \textsc{past}-exist \\
    `That was a very old house, but still on the two sides of the house, there had grown two beautiful rose bushes.'
\z
}


Next to the pluperfect, the normal perfect can also be used to refer to the past of the past, the marking of past tense on the existential \em aada \em is optional, as the following two examples show.

\xbox{16}{
\ea \label{ex:time:subseq:asa+aada:perf1}
\gll  Kanabisan=ka=jo duwa oorang=le anà-thaau ambel [Andare duwa oorang=yang=le \textbf{asa}-enco-kang \zero{}-\textbf{aada} katha]. \\
       last=\textsc{loc}=\textsc{foc} two man=\textsc{addit} \textsc{past}-know take Andare two man=\textsc{acc}=\textsc{addit} \textsc{cp}-fool-\textsc{caus} exist quot\\
\z
}



\xbox{16}{
\ea \label{ex:time:subseq:asa+aada:perf2}
\ea
\gll  derang pada kathahan thama-thuukar. \\
      \textsc{3pl} \textsc{pl} word \textsc{neg.nonpast}-change \\
\ex
\gll  karang  {\em British} {\em government}=nang  derang kathahan \textbf{asa}-kaasi \zero-\textbf{aada}. \\
      now British Government=\textsc{dat} \textsc{3pl} word \textsc{cp}-give exist \\
\z
\z
} \\

%
% \xbox{16}{
% \ea\label{ex:func:unreferenced}
% \ea\label{ex:func:unreferenced}figure
% \gll nyaakith  oorang pada \textbf{s-}pii. \\
%  sick man \textsc{pl} \textsc{cp}-go\\
% \ex
% \gll thaangan are-cuuci. \\
% hand \textsc{non.past}-wash\\
% `wash their hands.' (nosource)
% \z
% \z
% }

% \xbox{16}{
% \ea\label{ex:func:unreferenced}
% \ea\label{ex:func:unreferenced}
% \gll \textbf{asà-}girijja incayang, su-{\em pass}. \\
%  \textsc{cp}-do 3.polite \textsc{past}-pass\\
% `Having done that, I passed (the exam).'
% \ex
% \gll \textbf{asà-}{\em pass}, kaarang, {\em Advanced} {\em Level} arà-girijja skaarang. \\
% \textsc{cp}-pass now . . \textsc{non.past}-make now\\
% `Having passed the exam, then, I did the Advance Level then.'
% \ex
% \gll karang, {\em Advanced} {\em Level} \textbf{asà-}girijja, 1978=ka. \\
% `Then, having done the Advanced Level exames, in 1978,'
% \ex
% \gll {\em Late} {\em Exam} anthi-girijja. \\
% . . \textsc{irr} make\\
% `I would make a late exam.' (K051222nar08.4)
% \z
% \z
% }

The following passage of a narrative shows the use of four different strategies to encode subsequence: zero coding, \em blaakang, ithukapang \em and \em asà-\em.

\xbox{16}{
\ea \label{ex:time:subseq:quadruple}
\ea
\gll  Incayang=nang baaye=nang maara su-pii. \\ % bf
      3s.polite=\textsc{dat} good=\textsc{dat} anger \textsc{past}-go \\
    `He got very angry.'
\ex
\gll \textbf{\zero} incayang=pe kepaala=ka anà-aada thoppi=dering moonyeth pada=nang su-buwang puukul. \\
     { } 3s.polite=\textsc{poss} head=\textsc{loc} \textsc{past}-exist hat=\textsc{abl} monkey \textsc{pl}=\textsc{dat} \textsc{past}-throw hit \\
\ex
\gll \textbf{Ithu=kapang} ithu moonyeth pada=le anà-maayeng duuduk thoppi pada=dering inni oorang=nang su-bale-king puukul. \\
dist=then \textsc{dist} monkey \textsc{pl}=\textsc{addit} \textsc{past}-play exist.\textsc{anim} hat \textsc{pl}=\textsc{abl} \textsc{prox} man=\textsc{dat} \textsc{past}-return-\textsc{caus} hit\\
\ex
\gll Itthu=nang \textbf{blaakang} inni oorang likkas\~likkas thoppi pada=yang \textbf{asà-}kumpul ambel sithu=ka=dering su-pii. \\
dist after \textsc{prox} man fast\~red hat \textsc{pl}=\textsc{acc} \textsc{cp}-collect take there=\textsc{loc}=\textsc{abl} \textsc{past}-go\\
\z
\z
}\\


% \xbox{16}{
% \ea\label{ex:func:unreferenced}
% \gll  see   {\em leave} asà-abbis ambel=apa     see nya-pii    {\em Middle} {\em East}. \\
%       \textsc{1s} leave \textsc{cp}-finish take=after \textsc{1s} \textsc{past}-go Middle East \\
% \z
% } \\

% \xbox{16}{
% \ea\label{ex:func:unreferenced}
% \gll  incayang  mniinggal blaakang Zahir  Lahiyang anà-thaaro. \\
%       3s.polite die after Zahir Lahiyang \textsc{past}-put \\
%     `After he had died, they put Zahir Lahiyang).' (K051213nar08)
% \z
% } \\

\subsubsection{Simultaneity}\label{sec:func:Simultaneity}
Simultaneity obtains when two states-of-affairs are true at the same point in time.

\ea $\begin{array}{c}\stackrel{fig}{----}\\\stackrel{gr}{----}\end{array}$\z

Simultaneity is lexically coded by \trs{watthu}{time} or \trs{kaapang}{when}. This is used as a postposition on an NP, which can be based on a noun \xref{ex:time:simult:n}, a deictic\xref{ex:time:simult:deic} or a clause
%\xref{ex:time:simult:cl1}
\xref{ex:time:simult:cl2}.

\xbox{16}{
\ea \label{ex:time:simult:n}
\gll [{\em world} {\em war}]=\textbf{watthu}. \\
     world war time \\
    `During the world war.' (K051206nar07)
\z
} \\
\xbox{16}{
\ea \label{ex:time:simult:deic}
\gll [itthu]=\textbf{watthu} {\em Malaysia}=ka anà-duuduk    {\em Military}  {\em Regiment}. \\
      \textsc{dist} time Malaysia=\textsc{loc} \textsc{past}-stay Military Regiment \\
\z
} \\
% \xbox{16}{
% \ea \label{ex:time:simult:cl1}
% \gll suda [kitham pada  arà-{\em escort}    dhaathang]=\textbf{watthu}. \\
%       so, \textsc{1pl} \textsc{pl} \textsc{simult}-escort come=time\\
% \z
% } \\
\xbox{16}{
\ea \label{ex:time:simult:cl2}
\gll [kiccil]=\textbf{kapang} kithang sudaara pada samma {\em cricket} arà-maayeng. \\
     small when \textsc{1pl} siblings \textsc{pl} all cricket \textsc{non.past}-play  \\
    `When we were small, us children used to play cricket.'  (K051201nar02)
\z
}\\


In subordinate clauses, the prefix \em arà- \em codes simultaneity. In \xref{ex:time:simultaneity:ara}, the acts of seeing and playing are simultaneous, which is encoded by \em arà- \em on the verb \trs{maayeng}{play}. Taking (\em ambel \em) precedes the playing, but can be construed as simultaneous to seeing, or as preceding seeing.

\xbox{16}{
\ea \label{ex:time:simultaneity:ara}
\gll Blaakang=jo incayang \textbf{ana}-kuthumung [moonyeth pada thoppi \textbf{asa}-ambel pohong atthas=ka \textbf{arà-}maayeng]. \\
     after=\textsc{foc} 3s.polite \textbf{past}-see monkey \textsc{pl} hat \textbf{anterior}-take tree top=\textsc{loc} \textbf{simult}-play  \\
\z
}\\

Furthermore, simultaneity of two events can be expressed by putting one of the events in a reduplication construction \xref{ex:time:simult:redup:intro} \formref{sec:wofo:Verbalreduplication}.


\xbox{16}{
\ea \label{ex:time:simult:redup:intro}
\gll kithang nyaanyi\~{}nyaanyi su-thaandak. \\
     \textsc{1pl} sing\~{}red \textsc{past}-dance  \\
    `We danced while singing/we sang and danced.' (nosource)14.11.08
\z
} \\
A naturalistic example of this construction being used for simultaneity is \xref{ex:time:simult:redup:naturalistic}.


\xbox{16}{
\ea \label{ex:time:simult:redup:naturalistic}
\gll [lu=ppe muuluth=ka=le paasir, se=ppe muuluth=ka=le paasir  katha \textbf{biilang\~{}biilang}] baaye=nang baapa=le aanak=le guula su-maakang. \\
      \textsc{2s.familiar}=\textsc{poss} mouth=\textsc{loc}=\textsc{addit} sand \textsc{1s=poss} mouth=\textsc{loc}=\textsc{addit} sand \textsc{quot} say\~{}red good=\textsc{dat} father=\textsc{addit} child=\textsc{addit} sugar \textsc{past}-eat  \\
    `Saying ``There is sand in your mouth and there is sand in my mouth'' both father and son ate the sugar.'  (K070000wrt02)
\z
}\\

\subsubsection{Point coincidence}\label{sec:func:Pointcoincidence}
Point coincidence obtains when the end point of the first event coincides with the beginning of the second event.

\ea $\multimapdot\hspace{-.14cm}\multimapinv$\z
It is expressed  by the verbal prefix \em kam-/ka-/kapang-\em. In example \xref{ex:time:pc:kapang1}, the  pushing of the bamboo commences exactly at that point in time when the steam comes out.

\xbox{16}{
\ea \label{ex:time:pc:kapang1}
\gll aavi luwar=nang \textbf{kapang}-dhaathang,  itthu bambu=yang giini angkath=apa  pullang arà-thoolak. \\
      steam outside=\textsc{dat} when-come \textsc{dist} bamboo=\textsc{acc} like.this lift=after slow \textsc{non.past}-push \\
    `When the steam comes out, lift the bamboo like this and push it slowly.' (K061026rcp04)
\z
} \\

In example \xref{ex:time:pc:kapang2}, the habit of gathering is lost as soon as the people take up work.


\xbox{16}{
\ea \label{ex:time:pc:kapang2}
\ea
\gll inni     pukurjan=nang  \textbf{kam}-pii. \\
      \textsc{prox} work=\textsc{dat} when-go \\
    `When they go to this work.'
\ex
\gll  deram pada               itthu mà-kumpul     hatthu mosthor thraa. \\
3pl \textsc{dist} \textsc{inf}-add \textsc{indef} habit neg\\
\z
\z
} \\

% \xbox{16}{
% \ea \label{ex:time:pc:kam1}
% \gll kam-pii   siithu=ka    hatthu maccang hatthu duuduk aada jaalang=ka. \\
%      when-go there=\textsc{loc} \textsc{indef} tiger \textsc{indef} sit exist street=\textsc{loc}  \\
% \z
% } \\
%
% \xbox{16}{
% \ea\label{ex:func:unreferenced}
% \gll laskalli ka-dhaathang,    mari. \\
%      other.time when-come come.imp \\
%     `Visit us when you come again.' (B060115cvs17)
% \z
% } \\



% \xbox{16}{
% \ea\label{ex:func:unreferenced}
% \gll itthu thaaun=jo Mahathma Gandhi arà-buunu thaaun. \\
% \textsc{dist} year=\textsc{foc} Mahathma Gandhi \textsc{non.past}-kill year \\
% `That was the year Mahathma Gandhi was killed.' (nosource)
% \z
% }


% \xbox{16}{
% \ea
% \gll se ruuma=dering kapang-kuluwar se su-jaatho. \\
%      \textsc{1s} house=\textsc{abl} when-exit \textsc{1s} \textsc{past}-fall  \\
% \z
% } \\
%
% \xbox{16}{
% \ea
% \gll Se {\em Point} Pedro dìkkath=dering kapang-pii, lawut=nang su-jaatho. \\
%      \textsc{1s} Point Pedro vicinity=\textsc{abl} when-go sea=\textsc{dat} \textsc{past}-fall  \\
% \z
% } \\
Even if the second state-of-affairs becomes true immediately on the completion of the first one, \trs{blaakang}{after} is an alternative to \em kapang-\em. In \xref{ex:time:pc:blaakang}, the speaker became an orphan as soon as is parents died, not after that. Still \em blaakang \em is used.

\xbox{16}{
\ea \label{ex:time:pc:blaakang}
\gll se=ppe umma=le baapa=le mniinggal=nang blaakang, se hatthu yathiim su-jaadi. \\
     \textsc{1s=poss} mother=\textsc{addit} father=\textsc{addit} die=\textsc{dat} after \textsc{1s} \textsc{indef} orphan \textsc{past}-become  \\
\z
} \\

\subsubsection{Terminal boundary}\label{sec:func:Terminalboundary}

The configuration of a ground point in time indicating the terminal boundary of the figure can be schematized as follows.

\ea $\stackrel{fig}{\multimap }$\z
 \vspace{-0.6cm}\hspace{1.5cm} $\uparrow_{gr} $

Terminal boundary is expressed by the adposition \trs{sangke}{until}, which can be a postposition \xref{ex:time:tb:postnom1}- \xref{ex:time:tb:postnom3} or a preposition\xref{ex:time:tb:prenom}.



\xbox{16}{
 \ea \label{ex:time:tb:postnom1}
   \gll  suda derang=nang   [hathyang muusing]=\textbf{sangke} mà-duuduk  su-jaadi. \\
    thus \textsc{3pl}=\textsc{dat} other time=until \textsc{inf}-stay \textsc{past}-become\\
`So they had to wait until another time'(K051220nar01)
\z
}


\xbox{16}{
\ea \label{ex:time:tb:postnom2}
\gll incayang  [thujupul     liima thaaun]=\textbf{sangke} incayang  anà-iidop. \\
     3s.polite seventy five year=until 3s.polite \textsc{past}-live  \\
    `He lived until seventy-five.' (K060108nar02)
\z
} \\

\xbox{16}{
\ea \label{ex:time:tb:postnom3}
\gll suda thiga-pulu    satthu=ka duudukapa         thiga-pulu    ìnnam=\textbf{sangke}. \\
      so three-ty one=\textsc{loc} from three-ty six until \\
    `So, from '31 to '36.' (N061031nar01)
\z
} \\


\xbox{16}{
\ea \label{ex:time:tb:prenom}
\gll [Kaaki=ka gaandas-kang ambel anà-duuduk Aajuth=yang \textbf{sangke}=luppas] hathu pollu=dering Rose-red buurung=nang su-puukul. \\
     leg=\textsc{loc} tie-\textsc{caus} take \textsc{past}-stay dwarf=\textsc{acc} until=leave \textsc{indef} stick=\textsc{abl} Rose-red bird=\textsc{dat} \textsc{past}-hit  \\
\z
} \\
Like the other postpositions, \em sangke \em can also attach to a clause \xref{ex:time:tb:clause}. This is not possible for the prepositional use.


\xbox{16}{
\ea \label{ex:time:tb:clause}
\gll [Buruan diinging abbis]=\textbf{sangke} siithu su-siinggal. \\
      bead cold finish=until there \textsc{past}-stay \\
    `The bear stayed there until the cold season was over.' (K070000wrt04a)
\z
} \\
The terminal boundary can coincide with the moment of speaking, but does not imply that the state-of-affairs will not continue afterwards, as in the following example.

\xbox{16}{
\ea \label{ex:time:tb:speechsituation}
\gll nyaari=\textbf{sangke} se   inni ruuma=ka=jo       arà-duuduk. \\
 today until \textsc{1s} \textsc{prox} house=\textsc{loc}=\textsc{foc} \textsc{non.past}-stay\\
\z
}

In the example above, the terminal boundary is the time of speaking, but this does not imply that the speaker moves out of his house. It just indicates that the terminal boundary of the event `living', which is still progressing, is the time of speaking for the moment.

% \xbox{16}{
% \ea\label{ex:func:unreferenced}
% \gll itthu muusing asà-duuduk sampe nyaari se pukuran arà-gijja. \\
%  \textsc{dist} time \textsc{cp}-stay reach today \textsc{1s} work \textsc{non.past}-make\\
% `From that time on until today I have been working.' (K060108nar01.22)
% \z
% }



\subsubsection{Initial boundary}\label{sec:func:Initialboundary}
The initial boundary of a ground point in time, after which the period covered by the figure begins, can be schematized as follows.

\ea $\stackrel{fig}{\multimapinv }$\z
  \vspace{-0.5cm}\hspace{1.2cm}  $\uparrow_{gr} $\\

Initial boundary is expressed by means of \em asduuduk \em \xref{ex:time:ib:asduuduk}. Whereas source in a spatial sense can be expressed by both \em asduuduk \em and \em =dering\em, the latter is not possible for temporal relations. An alternative realization of \em asduuduk \em is \em duuduk=apa \em \xref{ex:time:ib:duudukapa}.


\xbox{16}{
\ea\label{ex:time:ib:asduuduk}
\gll itthu muusing \textbf{asduuduk} sangke=nyaari se pukuran arà-gijja. \\
 \textsc{dist} time from until==today \textsc{1s} work \textsc{non.past}-make\\
`From that time on until today I have been working.' (K060108nar01.22)
\z
}
%
% \xbox{16}{
% \ea\label{ex:func:unreferenced}
% \gll thirteen     \textbf{asduuduk}     go          blaajar aada {\em tayloring}. \\
%  thirteen \textsc{cp}-stay \textsc{1s.familiar} learn exist tayloring\\
% `From thirteen (years) onwards, I had learned tayloring.' (B060115nar04.79)
% \z
% }



\xbox{16}{
\ea\label{ex:time:ib:duudukapa}
\gll suda thiga-pulu    satthu=ka \textbf{duudukapa}         thiga-pulu    ìnnam=sangke. \\
      so three-ty one=\textsc{loc} from three-ty six until \\
    `So, from '31 to '36.' (N061031nar01)
\z
} \\
%
% \xbox{16}{
% \ea
% \gll ka-thiiga january asduudukapa, {\em current} blaangan arà-liiwath. \\
%     card-three january from electricity amount \textsc{non.past}-more  \\
%     `Starting January 3rd, the electricity prices will go up.' (nosource)14.11.08
% \z
% } \\
Initial boundary can also be expressed by \trs{kapang}{when}, as in \xref{ex:time:ib:kapang} , or by \trs{watthu=ka}{at the time of} as in \xref{ex:time:ib:watthuka}. In these examples, we are not dealing with point coincidence, since the blackness of the speaker did not come about on completion of the process of birth. This contrasts with becoming an orphang \xref{ex:time:pc:blaakang}, which is only true as soon as both parents have died.

\xbox{16}{
\ea\label{ex:time:ib:kapang}
\gll  see \textbf{kapang}-laaher, bannyak iitham. \\
      \textsc{1s} when-be.born much dark \\
    `When I was born, I was very dark.' (nosource)14.11.08
\z
} \\


\xbox{16}{
\ea\label{ex:time:ib:watthuka}
\gll  see=yang anà-braanak \textbf{watthu=ka} asaduuduk, see  iitham. \\
     \textsc{1s}=\textsc{acc} \textsc{past}-bear time=\textsc{loc} from \textsc{1s} dark  \\
\z
} \\


\subsubsection{Intermediacy}\label{sec:func:Intermediacy}

\ea $\begin{array}{c}\stackrel{fig}{\multimapinv\hspace{-.14cm}\multimap}\\\stackrel{gr}{\multimapdotinv----\multimapdot}\end{array}$\z

Intermediacy is also expressed by \trs{kaapang}{when}, as in \xref{ex:time:im:quake}, where the time of the  earth quake entirely falls within the stay of the speaker in Pakistan.
\draftnote{givon2001:330}


%
% kithang=nang   duppang lai     duuwa bergaada   dhaathang
% K060108nar02.txt: aada

%
% \xbox{16}{
% \ea
% \gll  lorang see=yang  diinging=dering  kala-aapith. \\
%       \textsc{2pl} \textsc{1s}=\textsc{acc} cold=\textsc{abl} if-??\\
% \z
% } \\

\xbox{16}{
\ea\label{ex:time:im:quake}
\gll Se Pakistan=ka kapang-duuduk hatthu buumi ginthar-an anà-jaadi. \\
      \textsc{1s} Pakistan=\textsc{loc} when-stay \textsc{indef} earth shiver-\textsc{nmlzr} \textsc{past}-become \\
\z
} \\

\subsection{Phasal information}\label{sec:func:Phasalinformation}
Besides being embedded in the time frame of the outside world, events also have an internal temporal structure. For expository reasons, we distinguish phasal information, which is concerned with the begin, progress and end of an event from aspectual information, which is concerned with boundedness of the event. Depending on theoretical orientation, linguists may or may not agree with this choice, but it makes the presentation more straightforward. At this point, I do not want to endorse any particular theoretical relation between phasal and aspectual information, the separation is purely due to practical reasons.

Phasal information is concerned with indicating the progress that the event has made in its completion. We distinguish the begin, the progress and the end.

\paragraph{Begin}
To indicate that the action is in its inception, SLM uses the verb \trs{mulain}{start}. The main verb is marked with the infinitive prefix \em mà- \em and normally precedes \em mulain\em.

\xbox{16}{
\ea\label{ex:func:time:phase:begin:mulain1}
\gll
[sini=pe      raaja=nang     ma=banthu] \textbf{anà-mulain}\\
here=\textsc{poss} king=\textsc{dat} \textsc{inf}-help \textsc{past}-start\\
\z
} \\
\xbox{16}{
\ea\label{ex:func:time:phase:begin:mulain2}
\gll siithu=jo see [mà-blaajar] anà-\textbf{mulain}. \\
      there=\textsc{foc} \textsc{1s} \textsc{inf}-learn \textsc{past}-start \\
\z
} \\
\paragraph{Progression}
Progression is normally not coded when refering to the present. In the past, progressive in subordinates can be coded by \em arà-\em, used instead of the normal marking for past tense (\em su-/anà-\em). However, this could be analyzed as simultaneous relative tense as well.

\xbox{16}{
\ea\label{ex:func:time:asp:prog}
\gll Blaakang=jo incayang anà-kuthumung [moonyeth pada thoppi asà-ambel pohong atthas=ka \textbf{ara}-maayeng]. \\
     after=\textsc{foc} 3s.polite \textsc{past}-see monkey \textsc{pl} hat \textsc{cp}-take tree top=\textsc{loc} \textsc{simult}-play  \\
\z
}\\

Another possibility to code progressive is the use of \em duuduk\em.

\xbox{16}{
\ea\label{ex:func:time:asp:prog:duuduk1}
\gll [dee arasbuuni   \textbf{duuduk}     {\em cave}] asaraathang  sini=ka    asaduuduk hathu  3  {\em miles} cara  jaau=ka. \\
      3\textsc{s.impolite} \textsc{simult}-hide sit cave] \textsc{copula} here=\textsc{loc} from \textsc{indef} 3 miles way far=\textsc{loc} \\
\z
} \\

\xbox{16}{
\ea\label{ex:func:time:asp:prog:duuduk2}
\gll incayang suda [aapa=ke hathu pukurjan] mà-girja arà-diyath \textbf{duuduk}. \\
      3s.polite thus what=\textsc{simil} \textsc{indef} word \textsc{inf}-make \textsc{non.past}-try sit \\
\z
} \\
\xbox{16}{
\ea\label{ex:func:time:asp:prog:duuduk3}
\gll loram pada anà-dhaathang wakthu=dika se spaathu ana/anthi-gijja \textbf{duuduk}. \\
       \textsc{2pl} \textsc{pl} \textsc{non.past}-come time=\textsc{loc} \textsc{1s} shoe \textsc{past/irr}-make stay  \\
\z
} \\

\paragraph{Continue}
To indicate that an event is still going on, \em laile \em is used.


\xbox{16}{
\ea\label{ex:func:time:phase:continue:laile}
\ea
\gll Seelong {\em independent} {\em state} anà-jaadi=nang aapa. \\ % bf
     Ceylon independent state \textsc{past}-become=\textsc{dat} after  \\
    `After Sri Lanka had become independent,'
\ex
\gll  kithang=nang independence anà-daapath=nang aapa. \\ % bf
      \textsc{1pl}=\textsc{dat} independence \textsc{past}-get=\textsc{dat} after \\
    `after we had obtained the independence,'
\ex
\gll \textbf{laile} derang anà-duuduk {\em under} {\em the} {\em Commonwealth}. \\
      still \textsc{3pl} \textsc{past}-stay under the Commonwealth \\
\z
\z
} \\
\em Laile \em is used in both positive and negative contexts, as the following two examples show.

\xbox{14}{
\ea
\gll  se \textbf{laile} lorang=nang saaya. \\
      \textsc{1s} still \textsc{2pl}=\textsc{dat} love \\
    `I still love you.' (nosource)
\z
} \\

\xbox{14}{
\ea
\gll  se=dang \textbf{laile} piisang \textbf{thàrà}-daapath. \\
      \textsc{1s=dat} still banana \textsc{neg.past}-get \\
    `I still have not got any bananas.' (nosource)
\z
} \\
It is important to distinguish \trs{laile}{still}, \trs{lai}{more}, \trs{laskalli}{again} and \trs{laayeng}{different}, as illustrated in the following examples.

\xbox{16}{
\ea
\gll  naasi \textbf{laile} asà-maatham thraa. \\
      rice still \textsc{cp}-cooked \textsc{neg} \\
\z
} \\

\xbox{16}{
\ea
\gll \textbf{lai} masa-rubbus. \\
     more must-boil   \\
    `Cook it some more.' (nosource)
\z
} \\
\xbox{16}{
\ea
\gll \textbf{laskali} masa-rubbus. \\
     again must-boil  \\
    `Cook it again.' (nosource)
\z
} \\

\xbox{14}{
\ea
\gll se=dang \textbf{laayeng} naasi maau. \\
     \textsc{1s=dat} different rice want  \\
    `  I want different rice.' (nosource)
\z
} \\


% more
%     se=dang lai naasi maau
%         Sinh tawa
%         I want more rice
%     se=dang laayeng naasi maau
%         I want different rice
%     se=dang lai piisang maau
%         I want more bananas
%     se=dang laayeng piisang maau
%         I want different bananas
%     laile
%         still
%         se=dang laile piisang thàràdaapath
%             I still have not got bananas
%         se laile lorangnang saaya
%             I still love you

% \xbox{16}{
% \ea
% \gll  naasi laile asa maatham thraa. \\
%       still \\
%     `The rice is not cooked yet.' (nosource)
% \z
% } \\
%
% \xbox{16}{
% \ea
% \gll lai masa rubbus. \\
%      more   \\
%     `Cook it some more.' (nosource)
% \z
% } \\
% \xbox{16}{
% \ea
% \gll laskali masa rubbus. \\
%      again  \\
%     `Cook it again.' (nosource)
% \z
% } \\
% \xbox{16}{
% \ea
% \gll incayang laile asa dhaathang thraa. \\
%       still \\
%     `He has not yet come.' (nosource)
% \z
% } \\



\paragraph{End}
To express that the event has reached its completion \em ab(b)is \em is used. Completive is not always easy to distinguish from subsequent events\formref{sec:func:Subsequence} , since the inception of the second event often implies the completion of the first.

\xbox{16}{
\ea\label{ex:func:time:phase:end}
\gll  kaaving \textbf{abbis}    derang pada=nang=le        aanak aada. \\
     marry finish \textsc{3pl} \textsc{pl}=\textsc{dat}=\textsc{addit} child exist  \\
\z
} \\
\subsection{Aspectual structure}\label{sec:func:Aspectualstructure}
Phasal information discussed in the last section is used to highlight a portion of an event, be it the beginning, the progress or the end.  Aspectual information, which will be discussed now, is concerned with the internal structure of the event: is it conceived of as including its temporal boundaries or not, is it one event or several, like iterative, etc.

\paragraph{Perfective and imperfective}\label{sec:func:asp:Imperfective}
Perfectivity and Imperfectivity are normally not expressed in SLM, but subsequent events marked by a chain of \em asà-\em marked verbs, or events marked for completive with \em abis \em are normally perfective \citet[171]{SmithEtAl2007}). All other TAM (quasi-)prefixes are underspecified for (im)perfectivity. The two past markers \em anà- \em and \em su- \em were tested for aspect distinctions, to no avail.

%
% \paragraph{Ingressive}
% Ingressive aspects focusses on the coming into being of a state-of-affairs. For different predicate types, different strategies are employed. For nominal predicates, \trs{jaadi}{become} is used.
%
%
% \xbox{16}{
% \ea\label{ex:func:time:asp:ingr:n}
% \gll oorang pada kapang-laari   dhaathang ini      daara sgiithu=le            \textbf{suusu} su-\textbf{jaadi}. \\
%     man \textsc{pl} when-run come \textsc{prox} blood that.much=\textsc{addit} milk \textsc{past}-become   \\
%     `When people came running, the blood had turned into milk.' (K051220nar01)
% \z
% } \\
%
% For adjectival predicates, TAM-marking  implies ingressive, as in the following example.
%
%
% \xbox{16}{
% \ea\label{ex:func:time:asp:ingr:adj}
% \gll itthu=nam blaakang=jo, kitham pada \textbf{anà-bìssar}. \\
%  \textsc{dist} after=\textsc{foc} \textsc{1pl} \textsc{pl} \textsc{past}-big\\
% \z
% }
%
% For verbs, ingressive aspect need not be marked, but can be highlighted by the vector verb \em ambel\em.
%
% \xbox{16}{
% \ea\label{ex:func:time:asp:ingr:v:ambel}
% \gll Kanabisan=ka=jo duwa oorang=le anà-\textbf{thaau} \textbf{ambel} [Andare duwa oorang=yang=le asà-enco-kang aada] katha. \\
%     `At the very end, both women understood that Andare had fooled both of them.' (K070000wrt05)
% \z
% } \\
% For modal predications, \em jaadi \em is used, similar to nominal predications.
%
%
% \xbox{16}{
% \ea\label{ex:func:time:asp:ingr:mod:jaadi}
% \gll itthu    blaakang=jo,    kitham=pe {\em AGM}=nang  duppang,  {\em high} {\em commissioner} {\em cultural} {\em show}=nang mà-dhaathang=nang        \textbf{thàràboole} \textbf{s-jaadi}. \\
%     After that, before our Annual General Meeting, it became impossible for the High Commissioner to attend the cultural show. (K060116nar23.39)
% \z
% } \\
% For ingressive aspect of locational predications, a verb of motion is used.
% First, the entity is not at a certain place, and then it is. This is ingressive aspect, but for locational predicates, this is normally coded lexically by a verb of motion, like \trs{pii}{go}{} in \xref{ex:func:time:asp:ingr:loc}.
%
% \xbox{16}{
% \ea\label{ex:func:time:asp:ingr:loc}
% \gll incayang itthusubbath {\em Malaysia}=nang asà-\textbf{pii} aada. \\
%      3s.polite therefore Malaysia=\textsc{dat} \textsc{cp}-go exist  \\
% \z
% } \\


\paragraph{Habitual}\label{sec:func:asp:Habitual}
Habitual can be marked by either  the present tense quasi-prefix \em ara= \em \xref{ex:time:aspect:habit:ara} \citep{Ansaldo2009} or the irrealis quasi-prefix \em anthi-\em \xref{ex:time:aspect:habit:irr}. In the negative, habitual is always marked by \em thamau-\em \xref{ex:time:aspect:habit:ara}.

\xbox{16}{
\ea \label{ex:time:aspect:habit:ara}
  \ea
 \gll Hindu \textbf{ara}-maakang kambing. \\
	Hindu \textsc{non.past}-eat goat\\
	`Hindus eat goat.'

	\ex samping \textbf{thuma}-maakang \\
		beef \textsc{neg.nonpast} eat\\
	` (they) don't eat beef.' (K060112nar01)
	\z
\z
}

\xbox{16}{
\ea \label{ex:time:aspect:habit:irr}
\gll saudi=ka ontha \textbf{anthi}-kaasi. \\
 Saudi.Arabia=\textsc{loc} camel \textsc{irr}-give\\
`In Saudi Arabia, they give camels.' (K060112nar01)
\z
}

Irrealis marking for habitual context can also be used with past reference. The following example has the adverb \trs{kethaama}{before}, which indicates past reference, but still uses \em anthi- \em to convey the habitual reading.

\xbox{16}{
\ea \label{ex:time:aspect:habit:irr:past}
\gll     punnu   mlaayu pada    \textbf{kethaama} {\em English}=jona        \textbf{anthi}-oomong. \\
       many Malay \textsc{pl} before English=\textsc{jona} \textsc{irr}-speak\\
\z
} \\

Things are the same with the following example, where an explicit reference to the past (`There once was a time') is combined with the irrealis marker to convey habituality.


\xbox{16}{
\ea \label{ex:time:aspect:habit:past:neg2}
\ea
   \gll  samma oorang {\em school}=nang   asà-pii   arà-blaajar    cinggala. \\  % bf
    all   man   school=\textsc{dat} \textsc{cp}-go \textsc{non.past}-learn Sinhala \\
\ex
   \gll  samma kithang=pe     mlaayu, hathu  muusing su-aada,      samma cinggala=dering=jo      \textbf{athi}-oomong. \\
    all   \textsc{1pl}=\textsc{poss} Malay  \textsc{indef}  time    \textsc{past}-exist all   Sinhalese=\textsc{abl}=\textsc{foc} \textsc{irr}-talk \\
\z
\z
}

\paragraph{Iterative}
There are no means to indicate iterative or repetitive aspect in the strict sense.
The repetition of an event can be indicated by \trs{laskalli}{again}.

\xbox{16}{
\ea \label{ex:time:aspect:iter:laskalli1}
\gll  asà-kumpul,     blaakang \textbf{laskalli} asà-{\em beat}-king, inni daalang=ka    kithang aayer  massa    me-libbi-king. \\
      \textsc{cp}-add after again \textsc{cp}-beat-\textsc{caus} \textsc{prox} inside=\textsc{loc} \textsc{1pl} water must \textsc{inf}-remain-\textsc{caus} \\
\z
} \\
\em Laskalli \em can be reduplicated as in \xref{ex:time:aspect:iter:laskalli:red}.

\xbox{16}{
\ea \label{ex:time:aspect:iter:laskalli:red}
\gll se=ppe pake-yan pada bannyak koothor. Itthu=subbath laskalli laskalli mà-cuuci su-jaadi \\
     \textsc{1s=poss} dress-\textsc{nmlzr} \textsc{pl} much dirt \textsc{dist}=because again again \textsc{inf}-wash \textsc{past}-become \\
\z
} \\
Another construction which comes close in meaning  and involves reduplication of an infinitive is given in \xref{ex:time:aspect:iter}.

\xbox{16}{
\ea \label{ex:time:aspect:iter}
\gll itthu=nang      blaakang karang aari \textbf{mà-pii}    \textbf{m-pii}      karang kitham=pe      mlaayu arà-mulain. \\
      \textsc{dist}=\textsc{dat} after now day \textsc{inf}-go \textsc{past}-go now \textsc{1pl}=\textsc{poss} Malay \textsc{non.past}-start \\
    `Now, after that, as days went by, our Malay is beginning.' (K051222nar03)(test)6.11.08
\z
} \\

% \xbox{16}{
% \ea \label{ex:time:aspect:iter:laskalli2}
% \gll deram pada asà-pii=apa  kettha=nam  \textbf{laskalli} asà-dhaathang bannyak thriima  kaasi. \\
%     \textsc{3pl} \textsc{pl} \textsc{cp}-go after \textsc{1pl}=\textsc{dat} again \textsc{cp}-come lot thanks give   \\
% \z
% } \\

\paragraph{Distributive}

Distributive is indicated by the reduplication of a numeral, the head noun and the plural marker.

\cb{NUM\~{}NUM N PL \NP* PRED}

This is possible for numbers greater than one \xref{ex:time:aspect:distr:2}, but also for \trs{hatthu}{one} \xref{ex:time:aspect:distr:1a} \xref{ex:time:aspect:distr:1b}.

\xbox{16}{
\ea \label{ex:time:aspect:distr:2}
\gll duuwa duuwa oorang pada su-pii. \\
    two two man \textsc{pl} \textsc{past}-go   \\
\z
} \\
\xbox{16}{
\ea \label{ex:time:aspect:distr:1a}
\gll hatthu hatthu oorang pada supii. \\
     one one man \textsc{pl} \textsc{past}-go  \\
\z
} \\



\xbox{16}{
\ea \label{ex:time:aspect:distr:1b}
\gll {\em Kandy} {\em Malay} {\em Association}=dering \textbf{hatthu} \textbf{hatthu} oorang \textbf{pada} arà-lompath {\em Hill} {\em Country}=nang. \\
    `More and more people stepped over from the KMA to the Hill Country Malay Association.' (K060116nar07)(test)6.11.08
\z
}



% \xbox{16}{
% \ea
% \gll oorang padaka hatthu hatthu ara thaanya ambel. \\
%        \\
%     `I ask people questions one by one.' (nosource)6.11.08
% \z
% } \\
If the numbers are more important, a periphrastic construction is prefered \xref{ex:time:aspect:distr:37}.


\xbox{16}{
\ea \label{ex:time:aspect:distr:37}
\gll  hatthu skaali=ka thiga-pul-thuuju oorang pada su-pii. \\
     one time=\textsc{loc} three-ty-seven man \textsc{pl} \textsc{past}-go  \\
\z
} \\

% \xbox{16}{
% \ea\label{ex:func:ptcpt:mismatch:reciprocal}
% \ea
% \gll \textbf{hatthu} \textbf{hatthu} oorang=yang anà-buunung thraa. \\
% 	`before, they were not killing people one by one.'
% \ex
% \gll skarang maana aari=le \textbf{atthu} \textbf{atthu} oorang=yang arà-buunung. \\
%     `But now, every day, one by one, people are killed.'  (K051206nar11)
% \z
% \z
% }\\
%
% \xbox{16}{
% \ea
% \gll {\em police} oorang hatthu hatthuyang su buunung. \\
%        \\
%     `the police killed the people one by one.' (nosource)6.11.08
% \z
% } \\
% \xbox{16}{
% \ea
% \gll   oorang pada hatthu hatthuyang su buunung. \\
%        \\
%     `the people killed each other one by one.' (nosource)6.11.08
% \z
% } \\

\subsection{Temporal frequency}\label{sec:func:Temporalfrequency}
Generally, temporal frequency can be expressed lexically by a temporal noun like \trs{aari}{day}{} or \trs{watthu}{time}{} which is modified by a quantifier \funcref{sec:wc:Quantifiers}, such as \trs{spaaru watthu}{some time}. The reader is referred to the section following this one for more discussion of this pattern involving quantifiers. The temporal domain \funcref{sec:func:Temporaldomain} (like `per week', `per year') is indicated by \em =nang\em.

Zero frequency is expressed by a negated predicate. There is no special word for \em never\em, but the negation can be reinforced by \trs{kaapang}{when}{} with the enclitics \trs{=pon}{any} or \trs{=le}{additive}, as in example \xref{ex:func:time:tempfreq:neg:kaapangpon} and \xref{ex:func:time:tempfreq:neg:kaapangle}.

\xbox{16}{
\ea\label{ex:func:time:tempfreq:neg:kaapangpon}
\gll suda itthu kithang=nang \textbf{kaapang=pon} thama=luupa. \\
     thus \textsc{dist} \textsc{1pl}=\textsc{dat} when=any \textsc{neg.nonpast}=forget  \\
\z
}\\

\xbox{16}{
\ea\label{ex:func:time:tempfreq:neg:kaapangle}
\gll go  \textbf{kaapang=le}      saala thamau-gijja. \\
     \textsc{1s.familiar} when=\textsc{addit} wrong \textsc{neg.nonpast}-make  \\
    `I never do any wrong.' (B060115nar04)
\z
} \\
The opposite of this is total frequency (\em always\em). This can be expressed either lexically by \em subbang watthu \em or by again \trs{kaapangle}{when}\xref{ex:func:time:tempfreq:aff:kaapangle}.


\xbox{16}{
\ea\label{ex:func:time:tempfreq:aff:kaapangle}
\gll  Girls' {\em High} {\em School} Kandi=ka se=dang \textbf{kaapang=le}  udahan hatthu arà-kiiring. \\
     girls high school Kandy=\textsc{loc} \textsc{1s=dat} when=\textsc{addit} invitation \textsc{indef} \textsc{non.past}-send  \\
\z
} \\
Finally, the WH=\textsc{clt}-construction with a temporal noun can be used as well.

\xbox{16}{
\ea\label{ex:func:time:tempfreq:aff:WHle}
\gll dee \textbf{maana} aari=\textbf{le}      asà-dhaathang, thìngaari wakthu=nang   kalthraa maalang wakthu=nang. \\
     3 which day=\textsc{addit} \textsc{cp}-come noon time=\textsc{dat} otherwise night time=\textsc{dat}  \\
\z
} \\
\section{Quantity}\label{sec:func:Quantity}
Quantity of entities can be expressed by adverbs like \trs{bannyak}{many}, by WH-words combined with boolean clitics (\trs{saapa=so}{someone}, \trs{kaapang=ke}{some day} \trs{kaapang=le}{always}, \trs{kaapang=pon}{never}).
Normally, there is a set of referents over which the quantity ranges, like \em aanak \em in \xref{ex:func:quant:intro:aanak:set}, but it is also possible to express general quantities which range over all imaginable referents\xref{ex:func:quant:intro:aanak:noset}.


\xbox{16}{
\ea\label{ex:func:quant:intro:aanak:set}
\gll  \textbf{bannyak} \textbf{aanak} pada karang mlaayu  thama-oomong. \\
    `Many children do not speak Malay now.' (G051222nar02)
\z
} \\
\xbox{16}{
\ea\label{ex:func:quant:intro:aanak:noset}
\gll  \textbf{bannyak} mà-biilang thàrboole. \\
 much \textsc{inf}-say cannot\\
`(I) can't tell you much.' (K051206nar12)
\z
}



\subsection{Zero proportion}\label{sec:func:Zeroproportion}
A proportion of zero is indicated by a negated predicate, mostly of the verbal, existential or adjectival type. Zero proportion is mostly not marked by any other marker than the negation of the predicate. In \xref{ex:func:quant:zero:thama}, the verbal negator \em thamau- \em suffices to imply that none of the fighters changed their words.


\xbox{16}{
\ea\label{ex:func:quant:zero:thama}
\gll  derang pada \zero{} katahan thama-thuukar.  \\ % bf
      \textsc{3pl} \textsc{pl} { } word \textsc{neg.irr}-change \\
\z
} \\
Similar things can be said about \xref{ex:func:quant:zero:thraa}, where the absence of any Malay is only marked by the existential negator \em thraa\em.

\xbox{16}{
\ea\label{ex:func:quant:zero:thraa}
\gll itthuka \zero{} mlaayu \textbf{thraa}, bannyak=nang {\em English}=jo aada. \\
	dist=\textsc{loc} { }  Malay \textsc{neg} much=\textsc{dat} English=\textsc{foc} exist \\
	`There is no Malay over there, it is all English which is there.'  (B060115prs15)
\z
}\\

If the total absence of applicable reference shall be emphasized, the enclitic \em =pon \em can be used.
This is the case in \xref{ex:func:quant:zero:pon} where the set to which the negative predicate applies is stated by \trs{oorang}{men}{} and the totality of the absence is indicated by a following \em =pon\em. Note that \em oorang \em is in this case accompanied by the indefinite article \em hatthu\em. In this construction, the indefinite article precedes the noun, whereas in other constructions, it can also follow the noun.

\xbox{16}{
\ea \label{ex:func:quant:zero:pon}
\gll   kithang \textbf{hatthu}=oorang=\textbf{pon} thàrà-iinggath. \\
       \textsc{1pl} \textsc{indef}=man=any \textsc{neg.past} think\\
\z
}\\

Total absence of any referent of any set is applied by adding \em =pon \em to the indefinite article \em atthu \em as in \xref{ex:func:quant:zero:atthupon}. The missing mention of domain (cf. \em oorang \em just above) indicates that the negation ranges over any and all possible referents, there is not a single thing which you can do on the bus.

\xbox{16}{
\ea \label{ex:func:quant:zero:atthupon}
\gll {\em bus}=ka \zero{} \textbf{hatthu=pon} mà-kirja thàràboole. \\
    `You can't do anything on the bus.' (K061125nar01)
\z
} \\
\subsection{More than one entity}\label{sec:func:Morethanoneentity}
Cardinality greater than one need not be specified, but can be expressed by a numeral \xref{ex:func:quant:1+:num}, the plural word \em pada \em \xref{ex:func:quant:1+:pada}, or both, if necessary \xref{ex:func:quant:1+:numpada}.

\xbox{16}{
\ea \label{ex:func:quant:1+:num}
\gll \textbf{duwa-pulu}    \textbf{ìnnam} \textbf{riibu}    \textbf{empath}  \textbf{raathus} \textbf{lima-pulu}    \textbf{duuwa} {\em votes} \zero{}  incayang=nang    anà-daapath. \\
    `He got 26,452 votes.' (N061031nar01)
\z
}\\



\xbox{16}{
\ea \label{ex:func:quant:1+:pada}
\gll itthu    watthu=ka    itthu    nigiri  \textbf{pada}=ka    arà-duuduk. \\
     \textsc{dist} time=\textsc{loc} \textsc{dist} land \textsc{pl}=\textsc{loc} \textsc{non.past}-stay \\
\z
}\\

\xbox{16}{
\ea \label{ex:func:quant:1+:numpada}
\gll se=dang \textbf{liima} anak  klaaki \textbf{pada} aada. \\
      \textsc{1s=dat} five child male \textsc{pl} exist \\
    `I have five sons.' (K060108nar02)
\z
} \\
Furthermore, the following sections also treat different possibilities to indicate sets of cardinality greater than one.

\subsection{Low proportion}\label{sec:func:Lowproportion}
A low proportion of referents, like English \em few, \em is indicated by \em sdiikith \em or \em konnyom \em and the affirmative or negative predicate, as the case may be.

\xbox{16}{
\ea \label{ex:func:quant:low:sdiikith}
\gll siithu Dubai=ka sri Lanka=pe oorang mlaayu pada \textbf{sdiikith} arà-duuduk. \\
      there Dubai=\textsc{loc} Sri Lanka=\textsc{poss} man Malay \textsc{pl} few \textsc{non.past}-saty \\
    `There in Dubai, there are few Sri Lankan Malays.' (K061026prs01)
\z
} \\
\xbox{16}{
\ea \label{ex:func:quant:low:konnyong}
\gll \textbf{konnyong} mlaayu=jo Seelong=ka thiinngal aada. \\
     few Malay=\textsc{foc} Ceylon=\textsc{loc} settle exist  \\
    `Few Malays have settled down in Sri Lanka.' (K051222nar06)
\z
} \\
% \xbox{16}{
% \ea\label{ex:func:unreferenced}
% \gll Indian oorang pada arà-duuduk  Pakistan oorang pada arà-duuduk  Sri Lankan oorang pada \textbf{sdiikith}=jo arà-duuduk. \\
%      Indian man \textsc{pl} \textsc{non.past}-exist.\textsc{anim} Pakistan man \textsc{pl} \textsc{non.past}-exist.\textsc{anim} Sri Lankan man \textsc{pl} few=\textsc{foc} \textsc{non.past}-exist.\textsc{anim}   \\
% \z
% } \\


\subsection{Unclear proportion}\label{sec:func:Unclearproportion}
To indicate that the exact proportion is unclear, but not negligible, \em spaaru \em is used.
\em Spaaru \em indicates a slightly higher proportion than \em sdiikith \em or \em konnyom\em. It can refer to about half of the set, which is not possible for \em sdiikith \em or \em konnyom\em. In this respect, \em spaaru \em resembles English \em some\em, which covers more important proportions than \em few\em.

\xbox{16}{
\ea \label{ex:func:quant:some1}
\gll \textbf{spaaru} oorang pada su-pii, \textbf{spaaru} oorang pada su-birthi. \\
    some man \textsc{pl} \textsc{past}-go some man \textsc{pl} \textsc{past}-stop   \\
\z
} \\
\xbox{16}{
\ea \label{ex:func:quant:some2}
\gll \textbf{spaaru} oorang pada thàrà-pii sindari. \\
     some man \textsc{pl} \textsc{neg.past}-go from.here  \\
\z
} \\
\subsection{High proportion}\label{sec:func:Highproportion}
A high proportion is indicated by \em bannyak\em.


\xbox{16}{
\ea \label{ex:func:quant:high1}
\gll itthu=pe        pada=jo    \textbf{bannyak} mlaayu pada karang siini aada. \\
     \textsc{dist=poss} \textsc{pl=foc} much Malay \textsc{pl} now here exist  \\
\z
} \\
\xbox{16}{
\ea \label{ex:func:quant:high2}
\gll  oorang pada  thiikam apa,  oorang pada=nang theembak apa,  se=dang \textbf{bannyak}  creeweth pada su-aada. \\
      man \textsc{pl} stab after man \textsc{pl}=\textsc{dat} shoot after \textsc{1s=dat} much trouble \textsc{pl} \textsc{past}-exist \\
\z
} \\
\subsection{Totality}\label{sec:func:Totality}
To indicate that all entities of the set are included in the predication, either \trs{samma}{each/evety/all}{} or the WH=\em le\em-construction is used.

Example \xref{ex:func:quant:all:samma} shows the use of bare \em samma \em to indicate the totatlity of the people knowing the speaker.

\xbox{16}{
\ea \label{ex:func:quant:all:samma}
\gll {\em doctors} pada=so {\em police} a.s.p=so  {\em judge}=so, \textbf{samma} oorang thaau see=yang \\
     doctors \textsc{pl}=\textsc{undet} police a.s.p=\textsc{undet} judge=\textsc{undet} all man know \textsc{1s}=\textsc{acc}  \\
    `Whether they be doctors, police assistant superintendents of police or judges, all men know me.'  (B060115nar04)
\z
}\\

More emphasis on the totality can be given by adding \em =le \em to either the entity itself \xref{ex:func:quant:all:sammanle} or \em samma\em\xref{ex:func:quant:all:sammale}.

\xbox{16}{
\ea \label{ex:func:quant:all:sammanle}
\gll \textbf{samma} oorang=\textbf{le}      saanak. \\
      all man=\textsc{addit} relative \\
    `All people are relatives.' (K051206nar07)
\z
} \\
\xbox{16}{
\ea \label{ex:func:quant:all:sammale}
\gll suda incayang=pe aanak pada \textbf{samma=le} {\em musicians} pada=jo. \\
     thus 3s.polite=\textsc{poss} child \textsc{pl} every=\textsc{addit} musicians \textsc{pl}=\textsc{foc}  \\
    `So all his children are musicians.'  (G051222nar01)
\z
}\\

Finally, the combination of an appropriate WH-word with again \em =le \em yields a universal quantifier as well. In this case, \em =le \em can be attached to the entity itself as in \xref{ex:func:quant:all:whle:n} or to the predicate as in \xref{ex:func:quant:all:whle:predicate}


\xbox{16}{
\ea \label{ex:func:quant:all:whle:n}
\gll skarang \textbf{maana} aari\textbf{=le} atthu atthu oorang=yang arà-buunung. \\
     now which day=\textsc{addit} one one man=\textsc{acc} \textsc{past}-kill   \\
\z
}\\

\xbox{16}{
\ea\label{ex:func:quant:all:whle:predicate}
\gll  lai     \textbf{saapa} mlaayu kuthumung=\textbf{le} aapa=ke      {\em connection} hatthu aada. \\
     other who Malay see=\textsc{addit} what=\textsc{simil} connection \textsc{indef} exist \\
    `If you see any other Malay, there will always be some kind of connection.' (K051206nar07)
\z
} \\


% \xbox{16}{
% \ea\label{ex:func:unreferenced}
% \gll mlaayu=pe samma criitha pada=le. \\
%      Malay=\textsc{poss} all story \textsc{pl}=\textsc{addit}  \\
%     `All the Malays' stories.' (N061031nar01)
% \z
% } \\
\section{Modality}\label{sec:func:Modality}
Speakers do not only exchange absolute truths. They also convey their estimation of the likelihood and desirablility of a situation. This is the domain of modality. Modality is divided into XYZ\kuckn.

\subsection{Deontic modality}\label{sec:func:Deonticmodality}
Deontic modality covers the external or internal need to perform or not a certain action. It can semantically be divided in event modality, where the need applies to anybody, and participant modality, where the need only applies to the named entities. In English, the sentence \em It is impossible to walk on water \em is event modality, because it the impossibility applies to anybody. The sentence \em John must go to school \em on the other hand is participant modality, since the obligation only applies to John, and other persons are not covered by this statement.

In SLM, the distinction between event modality and participant modality has no major morphosyntactic repercussions. In case of event modality, no participant is mentioned, whereas in case of participant modality, the participant is mentioned and typically bears a dative postposition.

In example \xref{ex:func:modality:deont:event}, we are dealing with event modality: no one can go far into that cave. Hence no participant is mentioned. In \xref{ex:func:modality:deont:participant} on the other hand, we are dealing with participant modality, because it is only the speaker(s) who are unable to go. This is indicated by mentioning \trs{kithang}{1pl} and adding the dative, in this case \em =nang\em.


\xbox{16}{
\ea\label{ex:func:modality:deont:event}
 bannyak jaau mà-pii \textbf{thàrboole},  itthu=ka   \\
 much far \textsc{inf}-go cannot \textsc{dist}=\textsc{loc}      \\
    `You/one cannot go far there/It is impossible to go far into that cave.' (K051206nar02)
\z
}\\

\xbox{16}{
\ea\label{ex:func:modality:deont:participant}
\gll \textbf{se=dang} karang jaau mà-pii \textbf{thàràboole}. \\
     \textsc{1s=dat} now far \textsc{inf}-go cannot  \\
    `I cannot go far now.'  (K061120nar01)
\z
}\\


For expository reasons, we will use the following order of different types of modality below: Permission, ability, impossibility, interdiction, desire, obligation, necessity. These are not all necessary semantically on the same level of modality, but this order will allow for a better understanding of the relations.

\subsubsection{Permission}\label{sec:func:Permission}
Permission is coded by the modal particle \em boole \em \formref{sec:wc:boole}.

\xbox{16}{
\ea\label{ex:func:modality:permission}
\gll lai   sdiikith       ari=jo            go=dang \textbf{bolle}=duuduk. \\
 other little day=\textsc{foc} \textsc{1s.familiar} can=stay\\
\z
}



% \xbox{16}{
% \ea
% \gll Municipal kithang nang {\em permission} ana kaasi siking, kithang nang siinika ruuma hatthu bole ikkath. \\
%       Municipal \textsc{1pl}=\textsc{dat} permission  \\
%     `.' (nosource)14.11.08
% \z
% } \\

\subsubsection{Capacity}\label{sec:func:Capacity}
Capacity is also coded by \em boole\em. There is no difference between physical \xref{ex:func:modality:ability:physical}  and mental ability\xref{ex:func:modality:ability:mental}.


\xbox{16}{
\ea\label{ex:func:modality:ability:physical}
\gll [\textbf{boole} oorang pada]na     siithu \textbf{boole} pii. \\
 can man \textsc{pl}=\textsc{dat} dem.loc.dist can go\\
`The men who can go may go.' (B060115cvs01.87)
\z
}


\xbox{16}{
\ea\label{ex:func:modality:ability:mental}
\gll cinggala  bahasa   saapa=nang=le        \textbf{bole}=bicaara      siini. \\
     Sinhala language who=\textsc{dat}=\textsc{addit} can-talk here  \\
    `Anybody can talk Sinhala here.' (K051206nar14)
\z
} \\


Example \xref{ex:func:modality:ability:physical} shows the use of \em boole \em in an abilitative context (first occurrence) and in a permissive context (second occurence). This different interpretation of \em boole \em in the two clauses makes the sentence non-tautological. In return, speakers normally do not utter tautologies, so that the fact that \xref{ex:func:modality:ability:physical} was uttered proves that the two tokens of \em boole \em must indeed have different interpretations.

\subsubsection{Incapacity}\label{sec:func:Incapacity}
Incapacity is expressed  by \em thàrboole\em\formref{sec:wc:therboole}.


\xbox{16}{
\ea\label{ex:func:modality:imposs1}
\gll  bannyak mà-biilang \textbf{thàrboole}. \\
 much \textsc{inf}-say cannot\\
` (I) can't tell you much.' (K051206nar12)
\z
}


\xbox{16}{
\ea\label{ex:func:modality:imposs2}
\gll derang pada=nang atthu=le mà-kijja=nang  \textbf{thàràboole}=subbath ....\\
 	3pl         \textsc{pl}=\textsc{dat} one=\textsc{addit}  \textsc{inf}-do=\textsc{dat} cannot=because\\
`Because they couldn't do anything.' (N060113nar01.11)
\z
}


\xbox{16}{
\ea
\gll  Lorang=nang aayer=ka appi mà-mnyala-king thàràboole. \\
      \textsc{2pl}=\textsc{dat} water=\textsc{loc} fire \textsc{inf}-light-\textsc{caus} cannot \\
    `You cannot light fire in water.' (nosource)14.11.08
\z
} \\

\subsubsection{Interdiction}\label{sec:func:Interdiction}
Deontic interdiction is normally expressed by a periphrasis involving the adjective \trs{thàràbaae}{not.good}. Another possibility is the use of \trs{thàrboole}{cannot}\formref{sec:wc:therboole}. The interdiction is inferred from the primary meaning of incpacity.

\xbox{16}{
\ea\label{ex:func:modality:interd}
\gll {\em cigarette} mà-miinong \textbf{thàràboole}. \\
      cigarette \textsc{inf}-drink cannot \\
    `It is forbidden to smoke.' (K060116nar04)
\z
} \\

\xbox{16}{
\ea
\gll Maalang=nang mà-swara-kang thàràboole. \\
     night=\textsc{dat} \textsc{inf}-noise-\textsc{caus} cannot  \\
\z
} \\

The speech act of interdiction is not deontic and is  covered under prohibition \funcref{sec:pragm:Requestingaction}.

\subsubsection{Obligation}\label{sec:func:Obligation}
Obligation can be expressed by the quasi-prefix  \em masthi\em\formref{sec:morph:masthi-}. This is an exception to the rule in that it  does not take the dative, but the nominative. Still, it can be used for participant modality as in \xref{ex:func:modality:obl:part} or for event modality as in \xref{ex:func:modality:obl:event}.

\xbox{16}{
\ea\label{ex:func:modality:obl:part}
\ea
\gll  luu=\zero{} baaye=nang \textbf{masa}-blaajar baaye=nang \textbf{masa}-mnaaji. \\
      \textsc{2s.familiar} good=\textsc{dat} must-learn good=\textsc{dat} must-recite \\
    `You have to learn well and you have to recite well.'
\ex
\gll lu=ppe umma-baapa=nang baaye=nang \textbf{masa}-kaasi thaangang. \\
      \textsc{2s}=\textsc{poss} mother-father=\textsc{dat} good=\textsc{dat} must-give hand \\
    `You must lend a hand to your parents.' (K060116sng01)
\z
\z
} \\
\xbox{16}{
\ea\label{ex:func:modality:obl:event}
\gll hathu oorang kala-pasiith, hathu {\em chance} \textbf{masa}-kaasi ithu  oorang=nang=le. \\
    `When one has trouble, we must give him a chance.' (K060116nar07)
\z
} \\
Periphrastic constructions involing the existential \em aada \em and \trs{jaadi}{become} are also possible\formref{sec:wc:Specialconstructionsinvolvingverbalpredicates}.

\xbox{16}{
\ea\label{ex:func:modality:obl:jaadi}
\gll se=ppe    {\em profession}=subbath \textbf{se=dang}  siini \textbf{ma}-pii    su-\textbf{jaadi}. \\
      \textsc{1s=poss} profession=because \textsc{1s=dat} here \textsc{inf}-go \textsc{past}-become \\
    `I had to come here because of my profession.' (G051222nar01)
\z
} \\
\xbox{16}{
\ea\label{ex:func:modality:obl:aada}
\gll  \textbf{se=dang} \textbf{aada} ini {\em army} pada=yang \textbf{ma}-salba-kang=\textbf{nang}. \\
    `I had to save these soldiers.' (K051213nar01)
\z
} \\

The construction with \em aada \em indicates a general obligation, whereas the construction with \em jaadi \em indicates an obligation which was caused by a change of circumnstances. In example \xref{ex:func:modality:obl:jaadi}, the need to move to another town was caused by a change in professional status, whereas in \xref{ex:func:modality:obl:aada}, no such change happened, it was just the general human obligation to help the soldiers which obtained. The obligation to help people in need is not brought about by a change in environment, but is a moral value (Even if in this case it is applied to a concrete situation). This general obligation irrespective of circumstances is indicated by \em aada\em.

% \xbox{16}{
% \ea\label{ex:func:unreferenced}
% \gll  se=dang karang ruuma=nang masa-pii. \\
%       \textsc{1s=dat} now house=\textsc{dat} must-go \\
%     `I have to go home now.' (B060115cvs08)
% \z
% } \\
The following  example show the difference between the constructions with \em jaadi \em and with \em aada\em. The need to swim in case of a flood is indeed caused by the circumstances, which is why \em jaadi \em is used. That need is not caused by authority, which is why it is impossible to use \em aada \em there.

\xbox{16}{
\ea\label{ex:func:modality:obl:disc:flood}
\gll Hatthu baanjir anà-dhaathang=siking, se=dang \textbf{ma}-bìrnang su-\textbf{jaadi}/*su-\textbf{aada}. \\
    \textsc{indef} flood \textsc{past}-come=because \textsc{1s=dat} \textsc{inf}-swim \textsc{past}-become/\textsc{past}-exist   \\
\z
} \\
The inverse situation holds in \xref{ex:func:modality:obl:disc:prayers}, where the need to pray five time a day is imposed by religious authority and not by the circumstances. This is why \em aada \em is possible, and \em jaadi \em is not.

% \xbox{16}{
% \ea
% \gll lorang hatthu aari nang liima skali masa sbaayang. \\
%        \\
%     `.' (nosource)14.11.08 normal
% \z
% } \\

\xbox{16}{
\ea\label{ex:func:modality:obl:disc:prayers}
\gll lorang=nang hatthu aari=nang liima skali \textbf{ma}-sbaayang \textbf{aada/*arà-jaadi}. \\
      \textsc{2pl}=\textsc{dat} one day=\textsc{dat} five time \textsc{inf}-pray exist/\textsc{non.past}-become \\
    `You have to pray five times a day.' (nosource)14.11.08 duty
\z
} \\


% \xbox{16}{
% \ea
% \gll *se=dang liima skali masbaayang nang sujaadi. \\
%        \\
%     `.' (nosource)14.11.08
% \z
% } \\
%
% \xbox{16}{
% \ea
% \gll se naapas ara ambel. \\
%        \\
%     `I am breathing.' (nosource)14.11.08
% \z
% } \\
%
% \xbox{16}{
% \ea
% \gll se naapas masa ambel. \\
%        \\
%     `I have to take a breath.' (nosource) when at dump and longing for fresh air
% \z
% } \\
%
% \xbox{16}{
% \ea
% \gll  se=dang naapas ma ambel maau. \\
%        \\
%     `I need/want to breathe.' (nosource)14.11.08
% \z
% } \\
%
% \xbox{16}{
% \ea
% \gll  *se=dang naapas ma ambel su aada. \\
%        \\
%     `.' (nosource)
% \z
% } \\
Finally, there are some cases where the need to do something is permanent, and not caused by authority or exceptional circumstances. One of these cases is breathing. Then, only the quasi-prefix \em masa- \em is possible.

\xbox{16}{
\ea\label{ex:func:modality:obl:disc:breath:masa}
\gll  mà-iidop=nang, naapas \textbf{masa}-ambel. \\
      \textsc{inf}-live=\textsc{dat} breath must-take \\
    `You must breathe in order to survive.' (nosource)14.11.08
\z
} \\

\xbox{16}{
\ea\label{ex:func:modality:obl:disc:breath:aada}
\gll *mà-iidop=nang naapas \textbf{ma}-ambel \textbf{aada}. \\
      \textsc{inf}-live=\textsc{dat} breath \textsc{inf}-take exist \\
    `(You are compelled to breathe in order to survive).' (nosource)14.11.08
\z
} \\

\subsubsection{Necessity}\label{sec:func:Necessity}
Necessity is expressed by \em (ka)mau(van)\em. This particle is also used for desire.

\xbox{16}{
\ea\label{ex:func:modality:obl:nec:mau}
\gll thullor maau. \\  % bf
 egg want  \\
`You need eggs [to make a dessert].' (B060115rcp02)
\z
}

\xbox{16}{
\ea\label{ex:func:modality:obl:nec:kamauvan}
   \gll  kithang=nang   hathu  {\em application} mà-sign  \textbf{kamauwan} wakthu=nang=jo,      kithang arà-pii    inni     {\em politicians} pada dìkkath=nang. \\
   \textsc{1pl}=\textsc{dat} \textsc{indef} application \textsc{inf}-sign want time=\textsc{dat}=\textsc{foc} \textsc{1pl} \textsc{non.past}-go \textsc{prox} politicians \textsc{pl} vicinity\\
\z
}

The borderline between obligation, necessity and desire is generally not clear cut in the Lankan languages. Sri Lankan Malay has a more clear cut distinction between obligation conveyed by \em masthi- \em and desire conveyed by \em (ka)maau(van)\em, but this seems to be eroding, and indiscriminate use of \em maau \em for both desire and obligation is becoming more common, paralleling the Sinhala and Tamil semantics. Example \xref{ex:func:modality:obl:nec:blur} shows the use of \em maau \em to convey mild obligation. This particle is normally used to express desire (see above).

\xbox{16}{
\ea\label{ex:func:modality:obl:nec:blur}
\gll baapa=nang {\em mosque}=nang mà-pii maau. \\ % bf
 father=\textsc{dat} mosque=\textsc{dat} \textsc{inf}-go want\\
\z
}

\subsubsection{Lack of necessity}\label{sec:func:Lackofnecessity}
Lack of necessity has to be distinguished from lack of desire (see below). Lack of necessity is coded by \em ther(ka)mauvan\em, while lack of desire is coded by \em thussa\em.

\xbox{16}{
\ea\label{ex:func:modality:lackofnecessity:maau}
\gll  se.dang se=ppe pukujan pada=nang baaru hatthu kaar \textbf{maau} \\
    `I need a new car for my work.' (nosource)14.11.08
\z
} \\
\xbox{16}{
\ea\label{ex:func:modality:lackofnecessity:tharmauvan}
\gll loram pukujan=nang baaru hatthu kaar \textbf{thàràmauvan}. \\
     \textsc{2pl} work=\textsc{dat} new \textsc{indef} car not.need  \\
    `You do not need a new car for your work.' (nosource)14.11.08
\z
} \\
My informants furthermore added, that \em thàrà\textbf{j}amauvan \em was a `common mistake', so that that form seems to have a certain frequency as well, even if I have not come across it personally.

\subsection{Volitive modality}\label{sec:func:Volitivemodality}


\subsubsection{Desire}\label{sec:func:Desire}
Desire is coded by the modal particle \em (ka)mau(van)\em, which derives from the lexical word \trs{kamauvan}{desire}, which can also be used to convey the notion of desire. The meaning of \em kamauvan \em is ambiguous between `need' and `desire' \xref{ex:func:modality:desire:intro}.\footnote{This ambiguity only exists in affirmative contexts, in negative contexts, lack of need is coded by \em thàrkamauvan\em \xref{ex:func:modality:lackofnecessity:tharmauvan}, and lack of desire is coded by \em thussa \em \xref{ex:func:modality:lackofdesire}.}

\xbox{16}{
\ea\label{ex:func:modality:desire:intro}
\gll se=dang baaru hatthu kaar (ka)maau(van). \\
      \textsc{1s=dat} new \textsc{indef} car want \\
    `I want/need a new car.' (nosource)14.11.08
\z
} \\
The following three examples show the use of \em maau \em for a desired object \xref{ex:func:modality:desire:ent} and a desired state \xref{ex:func:modality:desire:soa}.

\xbox{16}{
\ea\label{ex:func:modality:desire:ent}
\gll deran=nang    thumpath \textbf{maau}. \\
     3pl=dat place want  \\
    `They wanted land.'  (N060113nar01)
\z
}\\

\xbox{16}{
\ea\label{ex:func:modality:desire:soa}
\gll itthusubbath=jo incayang=nang  ini Sri Lankan {\em Malay} mà-blaajar \textbf{maau}. \\
  therefore=\textsc{foc} 3p.polite=\textsc{dat} \textsc{prox} Sri Lankan Malay \textsc{inf}-learn want  \\
\z
}

% \xbox{16}{
% \ea\label{ex:func:modality:desire:soas}
% \gll kitha=nang \textbf{maau} kitham=pe mlaayu loorang blaajar lorang=pe mlaayu kitham blaajar. \\
%  \textsc{1pl}=\textsc{dat} want \textsc{1pl}=\textsc{poss} Malay \textsc{2pl} learn \textsc{2pl}=\textsc{poss} Malay \textsc{1pl} learn\\
% `We want that you learn our [Sri Lankan] Malay, and we learn your [Malaysian] Malay.' (K060116nar02.100)
% \z
% }



% \xbox{16}{
% \ea\label{ex:func:modality:desire:lexical}
% \gll lorang [se=dang \textbf{kamauvan}] pada=yang gijja kaasi. \\
%      \textsc{2pl} \textsc{1s=dat} desire \textsc{pl}=\textsc{acc} make give  \\
% \z
% }\\


\subsubsection{Lack of desire}\label{sec:func:Lackofdesire}
Lack of desire is expressed by \em thussa\em. This is thus different from the coding of lack of necessity, which is \em thàrkamauvan \em \xref{ex:func:modality:lackofnecessity:tharmauvan}.

\xbox{16}{
\ea\label{ex:func:modality:lackofdesire}
\gll se=dang baaru hatthu kaar thussa. \\
      \textsc{1s=dat} new \textsc{indef} car not.want \\
    `I do not want a new car.' (nosource)14.11.08
\z
} \\





\subsection{Dynamic modality}\label{sec:func:Dynamicmodality}
Dynamic modality is concerned with the possibility to do something, not because someone allowed it or one has the ability to do so, but because relevant dispositions are available, such as the bus in \xref{ex:func:modality:dyn}. In \xref{ex:func:modality:dyn}, it is not the case that the person in need of transport has the permission or the exceptional ability to take the bus. It is rather the case that a  bus service operates, which the person can use to get from A to B. Dynamic modality is coded by \em boole\em.
%
% \xbox{16}{
% \ea\label{ex:func:modality:dyn}
% \gll bas=ka=lle        \textbf{bolle}=pii. \\
%       bus=\textsc{loc}=\textsc{addit} can=go \\
%     `You can also go by bus.'  (B060115cvs08 )
% \z
% }\\

% K051222nar05.txt:  thapi English kalablaajar    Ceylong samma thumpath bolekluuling


\xbox{16}{
\ea\label{ex:func:modality:dyn}
\gll  Lorang=nang bas=ka Jaapna(=nang) \textbf{bole}=pii. \\
      \textsc{2pl}=\textsc{dat} bus=\textsc{loc} Jaffna=\textsc{dat} can=go \\
    `You can go to Jaffna by bus.' (nosource)14.11.08
\z
} \\
On the other hand, the railway line to Jaffna is interrupted. This negative dynamic modality is coded by thàrboole. Again, this is not a question of permission or capacity, but of the existence of the relevant facilities.


\xbox{16}{
\ea
\gll  Jaalang asà-thuuthup=siking, lorang=nang koocci=ka Jaapna mà-pii \textbf{thàràboole}. \\
      path \textsc{cp}-close=because \textsc{2pl}=\textsc{dat} train=\textsc{loc} Jaffna \textsc{inf}-go cannot \\
    `Because the line is closed, you cannot go to Jaffna by train.' (nosource)14.11.08
\z
} \\
%
% \xbox{16}{
% \ea
% \gll Lorang nang {\em Maldives} nang koocci ka ma pii thàràboole. \\
%        \\
%     `.' (nosource)
% \z
% } \\
\subsection{Epistemic modality}\label{sec:func:Epistemicmodality}
Epistemic modality is concerned with the likelihood, probability and faith that speakers have in the truth of the propositions they state.
\citet[213]{FoleyEtAl1984} give the following continuum for epistemic modality

\ea real $leftarrow$ necessary -- probable -- possible $rightarrow$ unreal \z

In SLM, five levels of certainty can be distinguished. This was tested with a small made up setting. Some relatives have left for Badulla, and the question is whether at the time of speaking they have reached Badulla or not. The informants have spontaneously proposed percentages of probability for the different constructions, and this is repeated there.

\begin{itemize}
	\item 100\%: derang Badullana sampe aada
	\item 75\%: derang Badullana sampe anthi=jo aada
	\item 50\%: derang Badullana sampe anthi=aada
	\item 25\%: derang Badullana sampe thama=jo aada
	\item 0\%: derang Badullana sampe thama=aada
	\item: \trs{derang}{3pl},\trs{Badullana}{to Badulla},\trs{sampe}{reach},\trs{aada}{exist},\trs{anthi}{IRR},\trs{thama}{IRR.NEG},\trs{jo}{EMPH}.
	\item `They have \{surely / probably / possibly\} (not) arrived in Badulla (yet).'
\end{itemize}

We see that the perfect construction with \em aada \em conveys certainty, while its negation \em thama-aada \em conveys certainty of the contrary. The irrealis marker \em anthi- \em conveys the most incertain estimation (50\%). By attaching the focus clitic \em =jo \em to either \em anthi- \em or \em thama- \em a somewhat greater likelihood of the event is expressed.

Other means to express epistemic modality, like adverbs or particles do not exist. The use of the construction given above does not occur in the corpus, with the exception of \xref{ex:func:modality:epist}.


\xbox{16}{
\ea\label{ex:func:modality:epist}
\gll bìssar aanak asà-dhaathang \textbf{anthi-aada} ruuma=nang. \\
      big child \textsc{cp}-come \textsc{irr}=exist house=\textsc{dat} \\
\z
} \\
The following sentence is from an unrelated elicitation session, but illustrates the epistemic use of \em thama-aada\em nicely.

\xbox{16}{
\ea
\gll incayang hatthu thookal, itthule spaaru wakthu=nang, incayang {\em Maldives}=nang \textbf{asà-birnang} \textbf{thama-aada}. \\
    `He may be a fool, but never would he have tried to swim to the Maldives.' (nosource)14.11.08
\z
} \\
\subsection{Evidential modality}\label{sec:func:Evidentialmodality}
Speakers can mark information about the source of the information: is it first-hand knowledge or hearsay?

In SLM, first-hand knowledge does not receive special marking, while non-first-hand information can be indicated by the evidential marker \em kiyang\em\formref{sec:morph:kiyang}. The use of \em kiyang \em is optional, though, and not very frequent.

\xbox{16}{
\ea \label{ex:func:modality:evid:kiyang}
\gll Seelong {\em Airport}=yang duwa-pulu-empath wakthu=le asà-bukka arà-simpang \textbf{kiyang}. \\
     Ceylon Airport=\textsc{acc} two-ty-four hour=\textsc{addit}  \textsc{cp}-open \textsc{non.past}-stay evid\\
    `The Ceylon Airport will stay open 24h, it seems.'  (Letter 26.06.2007)
\z
}\\

In example \xref{ex:func:modality:evid:kiyang}, the speaker cannot vouch for the truth of the information he provides about the opening hours of the Colombo airport. He has only second hand knowledge of this, presumably taken from the media. This lack of first-hand knowledge is indicated by \em kiyang\em.

A lexical solution to convey evidential modality is to use \trs{biilang}{say}


\xbox{16}{
\ea \label{ex:func:modality:evid:biilang}
\gll  itthu    nya-aada     katha=le      \textbf{arà-biilang}. \\
      \textsc{dist} \textsc{past}-exist \textsc{quot}=\textsc{addit} \textsc{non.past}-say \\
\z
} \\

Both possibilities are exemplified below.


\xbox{16}{
\ea
\gll  {\em Maldives}=ka baaru hatthu {\em president} aada katha arà-biilang. \\
      Maldives=\textsc{loc} new \textsc{indef} president exist \textsc{quot} \textsc{non.past}-say \\
\z
} \\

\xbox{16}{
\ea
\gll {\em Maldives}=ka baaru hatthu {\em president} aada kiyang. \\
     Maldives=\textsc{loc} new \textsc{indef} president exist \textsc{evid}  \\
    `It is said that there is a new president in the Maldives.' (nosource)14.11.08
\z
} \\
Finally, a third possibility involves the `undetermined' clitic \em =so \em and indicates that the speaker is not completely sure about the propositional content he is conveying.

\xbox{16}{
\ea
\gll {\em President}=yang asa/su/anà-buunung=so mana=so. \\
     president=\textsc{acc} \textsc{cp}-/\textsc{past}-/\textsc{past}-kill=\textsc{undet} which=\textsc{undet}  \\
\z
} \\

% \xbox{16}{
% \ea
% \gll  {\em President} yang asa/su/ana  buunung kiyang. \\
%        \\
%     `.' (nosource)14.11.08
% \z
% } \\


\section{Conditionals}\label{sec:func:Conditionals}
Condition is expressed by the particle \em kalu \em \formref{sec:wc:kalu} (affirmative) or \em kalthra \em (negative)\formref{sec:wc:kalthra} in the conditional clause. The consequence is normally marked with the irrealis marker \em anthi- \em in the affirmative or \em thamau- \em in the negative. \footnote{Talking about conditions, especially counterfactuals, is very rare and the aim of elicitation sessions on couterfactuals was not always clear to the informants.}


\xbox{16}{
\ea\label{ex:func:condtionals:intro}
\ea
\gll See lorang=nang \textbf{thama}=sakith-kang. \\
      \textsc{1s} \textsc{2pl}=\textsc{dat} \textsc{neg.irr}=pain-\textsc{caus} \\
\ex
\gll lorang see=yang diinging=dering \textbf{kala}-aapith. \\
     \textsc{2pl} \textsc{1s}=\textsc{acc} cold=\textsc{abl} if-look.after \\
\z
\z
}\\


\xbox{16}{
\ea\label{ex:func:condtionals:wick}
\ea
\gll derang hathu  papaaya=yang   asà-poothong=apa. \\ % bf
      \textsc{3pl} \textsc{indef} papaw=\textsc{acc} \textsc{cp}-cut=after\\
\ex
\gll papaaya=ka    suumbu hatthu \textbf{kal}-thaaro. \\
     papaw=\textsc{loc} wick \textsc{indef} when-put \\
    `and put a wick in the papaya.'
\ex
\gll aayer=ka    \textbf{kal}-thaaro    \textbf{thama}=myaalak. \\
     water=\textsc{loc} when-put \textsc{neg.nonpast}=burn  \\
    `and put it into water, it would not burn.'
\ex
\gll minnyak     \textbf{kal}-thaaro    \textbf{anthi}-myaalak. \\
     coconut.oil when-put \textsc{irr}-burn  \\
    `When the put coconut oil, it would burn.' (source){test}
\z
\z
} \\
If a modal prefix or particle is used, this supersedes the use of \em anthi-\em, as in the following examples, where \em masthi- \em and \em bole= \em takes the preverbal position and blocks \em anthi- \em from occuring there.

\xbox{16}{
\ea\label{ex:func:condtionals:masthi}
\gll hathu oorang \textbf{kala}-pasiyeth, hathu {\em chance} \textbf{masa}-kaasi ithu  oorang=nang=le. \\
    `When one man suufers, we must give him a chance.' (K060116nar07)
\z
} \\
\xbox{16}{
\ea\label{ex:func:condtionals:boole1}
\gll  go=dang    asà-poothong        \textbf{kala}-kaasi   lorang=nang   \textbf{bole}=jaaith. \\
      1s.fam=dat \textsc{cp}-cut if-give \textsc{2pl}=\textsc{dat} can-sew \\
    `If I cut it and give it to you, you can sew it.' (B060115nar04)
\z
} \\

\xbox{16}{
\ea\label{ex:func:condtionals:boole2}
\gll siini=jo incayang=yang \textbf{kala}-baawa, \textbf{bole}=thaau ambel. \\
     here=\textsc{foc} 3s.polite=\textsc{acc} if-bring can=know take  \\
    `If you bring him here, he can come to know.' (K061030mix01)
\z
} \\
Conditionals of unavailability can be formed by adding  \em kalthraa \em to the NP expressing the lacking substance \xref{ex:func:condtionals:noun:neg}.

\xbox{16}{
\ea\label{ex:func:condtionals:noun:neg}
\gll lorang=ka duwith (*aada)  \textbf{kalthra}, kithang anthi-banthu. \\
     \textsc{2pl}=\textsc{loc} money exist, \textsc{1pl} \textsc{irr}-help  \\
\z
} \\

In the affirmative, the use of the existential \em aada\em, which is lacking in \xref{ex:func:condtionals:noun:neg}, is obligatory \xref{ex:func:condtionals:noun:aff}.

\xbox{16}{
\ea\label{ex:func:condtionals:noun:aff}
\gll se=dang saayap \textbf{kala-aada}, bole=thìrbang. \\
     \textsc{1s=dat} wing if-exist can=fly  \\
    `If I had wings I could fly.' (nosource)14.11.08
\z
} \\
The conditional marker attaching to a present tense predicate indicates possibility \xref{ex:func:condtionals:realis}, while its being used on a perfect tense predication indicates counterfactuality \xref{ex:func:condtionals:counterfactual}.

\xbox{16}{
\ea\label{ex:func:condtionals:realis}
\gll lorang asà-caape \textbf{kala-blaajar}, lorang=nang A/L  bole={\em pass}. \\
      \textsc{2pl} \textsc{cp}-tired if-learn \textsc{2pl}=\textsc{dat} A/L can=pass \\
\z
} \\

\xbox{16}{
\ea\label{ex:func:condtionals:counterfactual}
\gll Sampi! Luu maalas! A/L thàrà-pass. Lorang baae=nang asà-caape \textbf{asa}-blaajar \textbf{kala-aada}, mà-{\em pass}=nang su-aada \\
      cow \textsc{2s.familiar} lazy A/L \textsc{neg.past}-pass \textsc{2pl} good=\textsc{dat} \textsc{cp}-tired \textsc{cp}-learn if-exist \textsc{inf}-pass=\textsc{dat} \textsc{past}-exist \\
\z
} \\

However, if the context is clear, counterfactuals can also be formed with the present tense. In  \xref{ex:func:condtionals:counterfactual:contr} it is clear that humans will not grow wings, and that we are dealing with a counterfactual condition.

\xbox{16}{
\ea\label{ex:func:condtionals:counterfactual:contr}
\gll se=dang saayap kala-aada, bole=thìrbang. \\
     \textsc{1s=dat} wing if-exist can=fly  \\
    `If I had wings I could fly.' (nosource)14.11.08
\z
} \\
% Negative conditionals with \textsc{past}-tense verbal predicates have to be formed with a stacking of \em kal- \em and the relevant past tense form, as in \xref{ex:func:caus:biilang}
%
%
% \xbox{16}{
% \ea\label{ex:func:condtionals:past}
% \gll lorang=nang duwith kal-thàrà-daapath, kithang anthi-banthu. \\
%      \textsc{2pl}=\textsc{dat} money if-\textsc{neg.past}-get \textsc{1pl} \textsc{irr}-help  \\
% \z
% \z
% } \\
Another possibility to express conditionals, which neither further research, is the use of the `undetermined' clitic on a past tense verb in the main clause, as given in \xref{ex:func:condtionals:so}.

\xbox{16}{
\ea\label{ex:func:condtionals:so}
\gll {\em wicket}=ka su-kìnna=so, {\em out}. \\
     wicket=\textsc{loc} \textsc{past}-trike=\textsc{undet} out  \\
\z
} \\

% \xbox{16}{
% \ea
% \gll {\em wicket}=ka kalsukìnna, {\em out}. \\
%        \\
%     `.' (nosource)5.11.08
% \z
% } \\



%
% \subsubsection{Potentialis and Irrealis}\label{sec:func:PotentialisandIrrealis}
% Potentialis and irrealis are not distinguished,
%
% \xbox{16}{
% \ea\label{ex:func:unreferenced}
% \gll [sirikaya mà-maasakh=dang]      kalu inni thullor  mau. \\
%  Wattalapam \textsc{inf}-cooked-\textsc{dat} if \textsc{prox} egg want\\
% `If you want to prepare Wattalapam, you have to use these eggs.' (B060115rcp02.9)
% \z
% }
%
%
% \xbox{16}{
% \ea\label{ex:func:unreferenced}
% \gll  [ìnnam  thùllor arà-ambe]      kalu. \\
%       six egg \textsc{non.past}-take if \\
%     `If you take six eggs, ... .' (B060115rcp02)
% \z
% } \\
%
%
% \xbox{16}{
% \ea\label{ex:func:unreferenced}
% \gll [kumpulang] kalu tuuju oorang. \\
%       party if seven man \\
%     `In case of a party [sacrifice], (you) (need) seven men.' (K060112nar01)
% \z
% } \\
%
%
%
%
% \xbox{16}{
% \ea\label{ex:func:unreferenced}
% \ea\label{ex:func:unreferenced}
% \gll {\em voting} m-ambel thàràbole. \\
% v. \textsc{inf}-take cannot\\
% `You cannot hold a vote.'
% \ex
% \gll [lorang pada voting an-ambel]     kalu two {\em weeks} {\em notice} kaasi apa. \\
% \textsc{2pl} \textsc{pl} v. \textsc{past}-take if t. w. n. give after \\
% \z
% \z
% } \\
% Negated conditions can be formed by adding \em kalu \em to a negated clause, or by the contracted form \em kalthra\em.
%
% \xbox{16}{
% \ea\label{ex:func:unreferenced}
% \gll minnyak      klaapa, gitthu    thraada  kalu, [...] matheega=ka    goreng \\
%      coconut.oil coconut, like.that  nexist if [...] ghee=\textsc{loc} fry \\
%     ` (Fry them in) coconut oil. If that is not available, fry them in ghee.' (K060103rec01.79)
% \z
% } \\
%
%
% \xbox{16}{
% \ea\label{ex:func:unreferenced}
% \gll inni=yang samma minnyak      klaapa  giithu   kalthra inni, aapa=yang, matheega=ka    goreng. \\
%      \textsc{prox}=\textsc{acc} with coconut.oil coconut like.that if.not this what=\textsc{acc} ghee=\textsc{loc} fry   \\
%     `Together with coconut oil, otherwise fry it in what, with ghee.' (K060103rec01.92)
% \z
% } \\


\section{Comparison}\label{sec:func:Comparison}
\subsection{Equation}\label{sec:func:Equation}

Equation is formed by adding \em keejo \em to the second element of the equation. The first element is normally marked by \em =le\em. Neither of the elements of the equation receives special case marking.

% \xbox{16}{
% \ea
% \gll se=ppe ruuma loram=pe\textbf{=kee=jo} bìssar. \\
%      \textsc{1s=poss} house \textsc{2pl}=\textsc{poss}=\textsc{simil}=\textsc{foc} big  \\
%     `My house is as big as yours.' (nosource)(test)6.11.08
% \z
% } \\
\xbox{16}{
\ea
\gll incayang*(=le) see=kee=jo kaaya. \\
     3s.polite=\textsc{addit} \textsc{1s}=\textsc{simil}=\textsc{foc} righ  \\
\z
} \\

\subsection{Superiority}\label{sec:func:Superiority}
Superiority is marked either by \trs{liiwath}{more} or \trs{libbi}{remain}. The standard is marked with the dative.

\xbox{16}{
\ea
\gll lorang  se\textbf{dang} \textbf{libbi} kaaya. \\
    `You are  more wealthy than me.' (nosource)14.11.08
\z
} \\
There appear to be some particular semantics with regard to the use of \em libbi \em and \em liiwath.\em Depending on the adjective, one or the other might be better \xref{ex:func:comp:sup:libbiliiwath:liiwath} \xref{ex:func:comp:sup:libbiliiwath:libbi}.

\xbox{16}{
\ea\label{ex:func:comp:sup:libbiliiwath:liiwath}
\gll se=ppe ruuma lorampe nang liiwath/*libbi bìssar. \\
     \textsc{1s=poss} house \textsc{2pl}=\textsc{poss}=\textsc{dat} more/remain big  \\
    `My house is bigger than yours.' (nosource)(test)6.11.08
\z
} \\
\xbox{16}{
\ea\label{ex:func:comp:sup:libbiliiwath:libbi}
\gll se=ppe ruuma lorampe dang *liiwath/libbi kiccil. \\
    \textsc{1s=poss} house \textsc{2pl}=\textsc{poss}=\textsc{dat} more/remain small  \\
    `My house is smaller than yours.' (nosource)(test)6.11.08
\z
} \\
The above examples seem to point to a distinction between positive qualities, graded by \em liiwath \em and negative qualities graded by \em libbi\em. However, things are more difficult, as the following examples show, where both \trs{mlaarath}{difficult} and \trs{gampang}{easy} are graded by \em liiwath. \em This aspect is in need of further research.


\xbox{14}{
\ea
\gll {\em Japanese} {\em English}=nang liiwath mlaarath. \\
     Japanese English=\textsc{dat} more difficult  \\
    `Japanese is more difficult than English.' (nosource)6.11.08 14
\z
} \\

\xbox{14}{
\ea
\gll {\em Japanese} {\em English} na liiwath gampang. \\
     Japanese English=\textsc{dat} more easy  \\
    `Japanese is easier than English.' (nosource)6.11.08 14
\z
} \\


\subsection{Inferiority}\label{sec:func:Inferiority}
Inferiority is formed like superiority, but \trs{kuurang}{few, less} is added after the property word.

\xbox{16}{
\ea
\gll lorang  se=dang libbi kaaya \textbf{kuurang}. \\
     \textsc{2pl} \textsc{1s=dat} remain rich few  \\
\z
} \\
\subsection{Superlative}\label{sec:func:Superlative}
The superlative is expressed by \em anà- \em \xref{ex:func:comp:superl:ana}, \em =jo \em \xref{ex:func:comp:superl:jo}, or a combination thereof \xref{ex:func:comp:superl:anajo}.

\xbox{16}{
 \ea\label{ex:func:comp:superl:ana}
\gll Bill Gates duniya ka anà-kaaya oorang. \\
    Bill Gates world=\textsc{loc} superl-rich man   \\
    `Bill Gates is the richest man of the world.' (nosource)3.11.08
\z
} \\
\xbox{16}{
\ea\label{ex:func:comp:superl:jo}
\gll anjing oorang=pe baae=jo thumman. \\
      dog man=\textsc{poss} good=\textsc{foc} friend \\
    `The dog is man's best friend.' (nosource)4.11.08
\z
} \\

% \xbox{16}{
%  \ea\label{ex:func:comp:superl:jo}
% \gll Seelon=ka bìssar=jo  pohong. \\ % bf
%      Ceylon=\textsc{loc} big=\textsc{foc} tree  \\
%     `The biggest tree in Sri Lanka.'  (test)3.11.08
% \z
% }\\


\xbox{16}{
 \ea\label{ex:func:comp:superl:anajo}
   \gll  anà-muuda=jo       anak  klaaki. \\
    superl-young=\textsc{foc} child male \\
`the youngest one is a son' (K060108nar02)(test)3.11.08
\z
}



%
%
% \xbox{16}{
% \ea
% \gll  Seelonka ana bìssar pohong. \\
%        \\
%     `.' (nosource)4.11.08
% \z
% } \\




% \xbox{16}{
% \ea
% \gll duniyaka anà-kaaya=jo oorang Bill Gates. \\
%        \\
%     `.' (nosource)3.11.08
% \z
% } \\
% \xbox{16}{
% \ea
% \gll Bill Gatesjo duniyaka anà-kaayajo oorang. \\
%        \\
%     `.' (nosource)3.11.08
% \z
% } \\
%
% \xbox{16}{
% \ea
% \gll se=ppe ruuma subla oorangjo Kandika anà-miskin=jo. \\
%        \\
%     `.' (nosource)3.11.08
% \z
% } \\
%
% \xbox{14}{
% \ea
% \gll bannyak kìrras ana/ara laari oorang pompangjo Susanthika. \\
%        \\
%     `.' (nosource)
% \z
% } \\
\subsection{Elative}\label{sec:func:Elative}
The elative is expressed by \em =jo\em.

\xbox{16}{
\ea \label{ex:jo:elative}
\gll Hatthu komplok \textbf{bannyak=jo} puuthi caaya. \\
     one bush very=\textsc{foc} white colour \\
    `One bush was very, very white.'  (K070000wrt04)
\z
}\\

\subsection{Abundantive}\label{sec:func:Abundantive}
There is no special form for abundantive. The normal intensifiers \em bannyak \em or \em buthul \em can be used.

\xbox{16}{
\ea
\gll incayang buthul gummuk. \\
     3s.polite correct fat  \\
    `He is too/very fat.' (nosource)14.11.08
\z
} \\
The excess can be additionally marked by \trs{liiwath}{more}, but this is optional.

\xbox{16}{
\ea
\gll itthu {\em bag}=yang lorang=nang bannyak bìrrath (\textbf{liiwath}). \\
     \textsc{dist} bag=\textsc{acc} \textsc{2pl}=\textsc{dat} much heavy more  \\
\z
} \\

\subsection{Insufficientive}\label{sec:func:Insufficientive}
To indicate that the degree is not sufficient as compared to a non-expressed standard, \trs{thàrà-sampe}{does not reach} is used. \em thàràsampe \em can be used with nouns or with adjectives \xref{ex:func:comp:insuf:kaake}.

%
% \xbox{16}{
% \ea
% \gll  Itthu gaaji thàràsampe. \\
%       \textsc{dist} salary insufficient \\
%     `That salary is not enough.' (nosource)14.11.08
% \z
% } \\
%
%
% \xbox{16}{
% \ea
% \gll ini pliitha=pe thìrrang thàràsampe. \\
%     \textsc{prox} lamp=\textsc{poss} light insufficient   \\
%     `This lamp is not bright enough.' (nosource)
% \z
% } \\
%
% \xbox{16}{
% \ea
% \gll Ini kameeja=pe/=ka puuthi thàràsampe. \\
%      \textsc{prox} shirt=\textsc{poss}/=\textsc{loc} white insufficient  \\
%     `This shirt is not white enough.' (nosource)
% \z
% } \\

\xbox{16}{
\ea\label{ex:func:comp:insuf:kaake}
\gll se=ppe kaakenang siggar/sigaran thàràsampe. \\
     \textsc{1s=poss} grandfather healthy/health insufficient  \\
    `My grandfather is not well enough.' (nosource)
\z
} \\
The purpose for which the degree is insufficient can be indicated by an infinitive clause.

\xbox{16}{
\ea\label{ex:func:comp:sufficient}
\gll \textbf{pohong} \textbf{mà-seereth=nang} oorang thàrà-sampe. \\
    `There were not enough men to drag the tree.' (K051205nar05)
\z
} \\
For adjectives, the normal negation can be used, complemented by \trs{punnu}{full} and an intensifier.

\xbox{16}{
\ea
\gll Ini makanan \textbf{bannyak/giithu} \textbf{punnu} puddas \textbf{thraa}. \\
    `This food is not spicy enough.' (nosource)14.11.08
\z
} \\
\subsection{Correlative}\label{sec:func:Correlative}
Correlative comparison is formed by a deduplicated infinitive.


\xbox{14}{
\ea
\gll  caabe mà-thaaro mà-thaaro puddas liiwath. \\
      chili \textsc{inf}-put \textsc{inf}-put spicy more \\
    `The more chili you put, the spicier (the food becomes).' (nosource)14.11.08 3.11.08
\z
} \\

\xbox{14}{
\ea
\gll haari mà-pii mà-pii laambath liiwath. \\
     day \textsc{inf}-goo \textsc{inf}-go delay more  \\
    `Days pass by and the delay is caused/The more days pass, the more delay is caused.' (nosource)14.11.08
\z
} \\
% \xbox{16}{
% \ea\label{ex:constr:pred:unreferenced}
% \gll caabe mathaaro mathaarojo, puddas ara liiwath. \\
%        \\
%     `.'  (test)4.11.08
% \z
% }\\








%
% \xbox{16}{
% \ea
% \gll  ?lorang se=dang liiwath kaaya. \\
%        \\
%     `.' (nosource)14.11.08 sociolinguistics
% \z
% } \\
%
% \xbox{16}{
% \ea
% \gll  ?lorang se=dang liiwath kaaya kurang. \\
%        \\
%     `.' (nosource)14.11.08
% \z
% } \\





\section{Possession}\label{sec:func:Possession}
Within the realm of possession, we can distinguish three different constellations (assertions in boldface): either the possessee is asserted (\em I have \textbf{a car}\em), or the possessor is asserted (\em the car is \textbf{mine}\em), or they are both part of the presupposition (\em My car \textbf{broke down}\em). Furthermore, permanent and temporary possession can be distinguished, which is relevant for SLM, as are differences in animacy, while differences in alienability are not relevant.

There is no verb meaning `to have', in line with general South Asian \citep[166]{Masica1976}  and Austronesian typology \citep[139]{Himmelmann2005typochar}. There is a verb \trs{puunya}{possess}, but this is hardly ever used.

\subsection{Assertion of the possessee}\label{sec:func:Assertionofthepossessee}
The assertion of the possessee's being in a possessive relationship with the possessor is done with a combination of the postpositions \em =nang \em or \em =ka \em with the existentials \em aada \em or \em duuduk\em. \em =nang \em is used for permanent possession, while \em =ka \em is used for temporary possession. \em duuduk \em can only be used for animate possessees (which are almost always kin), while \em aada \em can be used for any possessee. The following sections show various combinations of these parameters.

\begin{center}
% use packages: array
\begin{tabular}{p{4cm}p{4cm}p{4cm}p{4cm}}
 	  & animate & inanimate & abstract \\
permanent & =nang +aada \xref{ex:poss:perm:anim:aada} , =nang +duuduk\xref{ex:poss:perm:anim:duuduk1} \xref{ex:poss:perm:anim:duuduk2}
					& =nang + aada \xref{ex:poss:perm:anim:aada}\\
temporary & n/a
			& =ka+aada \xref{ex:poss:temp:inanim:aada:ka}
					& =nang + aada\xref{ex:temp:perm:abstr:aada}
\end{tabular}
\end{center}

\subsection{Permanent possession of animates}\label{sec:func:Permanentpossessionofanimates}
Possessed animates like kin are always construed with the dative marker \em =nang \em on the possessor. The existential can be either \em duuduk \em or \em aada\em.

% \xbox{16}{
% \ea
% \gll thiiga klaaki aade=le hatthu pompang aade=le \textbf{se=dang} arà-\textbf{duuduk}. \\
%      three male younger.sibling=\textsc{addit} one female younger.sibling \textsc{1s=dat} \textsc{non.past}-exist.\textsc{anim}  \\
% \z
% } \\

\xbox{16}{
\ea\label{ex:poss:perm:anim:duuduk1}
\gll se=dang hathu maven arà-\textbf{duuduk}. \\
      \textsc{1s=dat} \textsc{indef} son \textsc{non.past}-exist.\textsc{anim} \\
\z
} \\
\xbox{16}{
\ea\label{ex:poss:perm:anim:duuduk2}
\gll se=dang duuwa pompang aade=le hathu klaaki aade=le anà-\textbf{duuduk}. \\
     \textsc{1s=dat} two female younger.sibling=\textsc{addit} one male younger.sibling=\textsc{addit} \textsc{past}-exist.\textsc{anim}  \\
\z
} \\
\xbox{16}{
\ea \label{ex:poss:perm:anim:aada}
\gll  \textbf{se=dang} liima anak  klaaki pada \textbf{aada}. \\
      \textsc{1s=dat} five child male \textsc{pl} exist \\
    `I have five sons.' (K060108nar02)
\z
} \\

\subsection{Temporary possession of animates}\label{sec:func:Temporarypossessionofanimates}


\xbox{16}{
\ea \label{ex:poss:temp:anim}
\gll Seeka hatthu buurung aada. \\
    \textsc{1s}=\textsc{loc} \textsc{indef} bird exist   \\
\z
} \\
\subsection{Permanent possession of inanimates}\label{sec:func:Permanentpossessionofinanimates}
Inanimates can never be construed with \em duuduk\em. \em Aada\em always has to be used.

\xbox{16}{
\ea \label{ex:poss:perm:inanim:aada}
\gll Mr. Yusuf, karang, incayang=\textbf{nang} [ini {\em private} {\em bank}=ka aada duwith pada] \textbf{aada} \\
    `Mr. Yusuf owned the money which was deposited in this private bank.' (K060116nar09)
\z
} \\

% \xbox{16}{
% \ea\label{ex:func:unreferenced}
% \gll Seelong=le     kithang=pe     mlaayu=nang=le        hathu  bagiyan an-aada
%   \\
%        \\
%     `.' (K051222nar04)
% \z
% } \\

\xbox{16}{
\ea
\gll se=dang hatthu ruuma aada. \\
     \textsc{1s=dat} \textsc{indef} house exist  \\
\z
} \\

\subsection{Temporary possession of inanimates}\label{sec:func:Temporarypossessionofinanimates}
Temporary possession of inanimates is construed with the locative \em =ka \em and the inanimate existential \em aada\em.

\xbox{16}{
\ea \label{ex:poss:temp:inanim:aada:ka}
\gll incayang=\textbf{ka} ... bìssar beecek caaya hathu {\em bag} su-\textbf{aada}. \\
     3s.polite=\textsc{loc} ... big mud colour \textsc{indef} bag \textsc{past}-exist  \\
    `He had a big brown bag with him.' (K070000wrt04a)
\z
} \\

\xbox{16}{
\ea \label{ex:poss:temp:inanim:aada:ka2}
\gll see=\textbf{ka}=jo bannyak ini panthong \zero{}. \\
     \textsc{1s}=\textsc{loc}=\textsc{foc} much \textsc{prox} song  \\
    `I possess a lot of these songs [on sheets].' (K060116nar04)
\z
} \\
% B060115nar04.txt: karang itthu    textile       goka     aada

\subsection{Possession of abstract concepts}\label{sec:func:Possessionofabstractconcepts}
Abstract concepts can never be construed with \em =ka \em or \em duuduk\em. \em =nang \em and \em aada \em have to be used.

\xbox{16}{
\ea \label{ex:temp:perm:abstr:aada}
\gll se\textbf{dang} bannyak creeweth pada su-\textbf{aada}. \\
     \textsc{1s=dat} lot trouble \textsc{pl} \textsc{past}-exist  \\
\z
} \\
\subsection{Assertion of the possessor}\label{sec:func:Assertionofthepossessor}
If the possessee is established, but the possessor not, an equative construction is used. This construction assigns the possessum to the class of items possessed by the possessor. This class is indicated by marking the possessor with the possessive postposition \em =pe\em.

% \xbox{16}{
% \ea\label{ex:func:unreferenced}
% \gll ini ankel=\textbf{pe}? \\
%  \textsc{prox} uncle=poss\\
% `Is this the uncle's?' (nosource)
% \z
% }\\


\xbox{16}{
\ea\label{ex:func:poss:asspr1}
\gll  itthu    muusing bannyak {\em teacher} pada] [\textbf{Jaapna=pe}]. \\
      \textsc{dist} time many teacher \textsc{pl} Jaffna=\textsc{poss} \\
\z
} \\
\xbox{16}{
\ea\label{ex:func:poss:asspr2}
\gll se=ppe    saayang jiiwa biilang [\textbf{se=ppe}] katha. \\
     \textsc{1s=poss} love life say \textsc{1s=poss} \textsc{quot}  \\
\z
} \\
\subsection{Presupposition of the possessor and the possessee}\label{sec:func:Presuppositionofthepossessorandthepossessee}
If the possessive relation is presupposed, the possessive postposition \em =pe \em is used \xref{poss:presup:n}. If the possessor is expressed by a monosyllabic pronoun, \em =ppe \em is used \xref{poss:presup:pron12}. Other pronouns take the normal form \xref{poss:presup:pron}.


\xbox{16}{
\ea \label{poss:presup:n}
\gll   kithang=pe \textbf{baapa=pe} naama Mahamud. \\
       \textsc{1pl}=\textsc{poss} father=\textsc{poss} name Mahamud\\
    `Our father's name is Mahamud.' (B060115nar03)
\z
} \\

\xbox{16}{
\ea \label{poss:presup:pron12}
\gll lu=\textbf{ppe} muuluth=ka=le paasir, se=\textbf{ppe} muuluth=ka=le paasir. \\
      \textsc{2s.familiar}=\textsc{poss} mouth=\textsc{loc}=\textsc{addit} sand \textsc{1s=poss} mouth=\textsc{loc}=\textsc{addit} sand    \\
    `There is sand in your mouth and there is sand in my mouth.'   (K070000wrt02)
\z
}\\

\xbox{16}{
\ea \label{poss:presup:pron}
\gll  \textbf{kitham=pe}     ruuma dìkkath=ka. \\
      \textsc{1pl}=\textsc{poss} house vicinity=\textsc{loc}  \\
    `Close to our house.' (K051220nar01)
\z
} \\
\section{Questions}\label{sec:func:Questions}
Question do not have as a goal to transmit propositional information to the hearer, but rather to incite the hearer to provide some information himself.

\subsection{Yes-no questions}\label{sec:func:Yes-noquestions}
SLM Yes-no-questions are formed by adding the interrogative clitic \em=si \em to the questioned element. This is accompanied by rising intonation. If it is the predicate that is questioned, then the interrogative particle is sometimes left out and the question is solely indicated by intonation. See \formref{sec:phon:Intonation} for some intonation curves. If some constituent is questioned \em =si \em cannot be left out.

It is possible to question the predicate as in \xref{ex:func:quest:si:pred}, or a constituent as in \xref{ex:func:quest:si:const}.

\xbox{16}{
\ea \label{ex:func:quest:si:pred}
\gll se=pe uumur massa-\textbf{biilan=si}? \\
 1=\textsc{poss} age must-tell=\textsc{interr}\\
`Do I have to tell my age?' (B060115prs01)
\z
}

\xbox{16}{
\ea \label{ex:func:quest:si:const}
\gll saapa? \textbf{se=si}? \\
 who \textsc{1s}=\textsc{interr}\\
`Who? Me?' (B06015prs18)
\z
}

\subsection{Alternative questions}\label{sec:func:Alternativequestions}
are formed like yes-no-questions, but with \em =si \em attached to all the alternative elements.


\xbox{16}{
\ea
\gll piisang\textbf{=si} maangga\textbf{=si} maau? \\
     banana=\textsc{interr} mango=\textsc{interr} want  \\
    `Is it plantain or mango that you want?'  (test)5.11.08
\z
}\\



\subsection{Content questions}\label{sec:func:Contentquestions}
\subsubsection{Person}\label{sec:func:q:Person}

\trs{Saapa}{who} is the interrogative pronoun used for persons. It can take the whole array of postpositions to indicate its syntactic and semantic role\formref{sec:wc:saapa}.

\xbox{16}{
\ea \label{ex:func:quest:content:person}
\gll \textbf{saapa}  anà-maath?. \\
 who \textsc{past}-die\\
`Who died?' (K051213nar07)
\z
}

\subsubsection{Animal}\label{sec:func:q:Animal}
There is no special interrogative pronoun to query for animals. One can use \trs{mana binaathang}{which animal}. This differs from the adstrate Sinhala, where a specialized interrogative pronouns for animals exists, \em mokaa \em \citep[259]{Karunatillake2004}.


\xbox{16}{
\ea
\gll mana binaathang lorang yang anà-giigith? \\
     which animal \textsc{2pl}=\textsc{acc} \textsc{past}-bite  \\
\z
} \\

\subsubsection{Things}\label{sec:func:q:Things}
\trs{Aapa}{what}\formref{sec:wc:aapa} and \trs{mana}{which}\formref{sec:wc:mana} are use to query for things in the widest sense.   Like \em saapa\em, they can take any postposition to indicate the syntactic and semantic role.

\xbox{16}{
\ea \label{ex:func:quest:content:things:aapa}
\gll \textbf{aapa}   anà-jaadi       mlaayu pada? \\
 what \textsc{past}-become Malay pl\\
`What became of the Malays?' (K051213nar06)
\z
}

\xbox{16}{
\ea \label{ex:func:quest:content:things:mana}
\gll \textbf{mana} nigiri=ka arà-duuduk? \\
 which country=\textsc{loc} \textsc{non.past}-stay\\
\z
}

\subsubsection{Quantity}\label{sec:func:q:Quantity}
Quantity is asked for with \trs{dhraapa}{how much}?


\xbox{16}{
\ea \label{ex:func:quest:content:quant}
\gll \textbf{dhraapa} thaaun \textbf{dhraapa} buulang lu arà-baapi suusa? \\
      how.many year how.many month \textsc{2s.familiar} \textsc{non.past}-bring sad \\
\z
}\\



\xbox{16}{
\ea
\gll birras hatthu kilo dhraapa? \\
     raw.rice one kilo how.much  \\
    `How much is one kilo of rice?' (nosource)
\z
} \\

\subsubsection{Location}\label{sec:func:q:Location}
Questions for locations are formed with the interrogative pronoun \trs{mana}{where}, which can be used for stative location \xref{ex:func:quest:content:loc:ka} or goal of motion \xref{ex:func:quest:content:loc:nang}. For source, \trs{maana dering, manari}{whence} has to be used {ex:func:quest:content:loc:dering}, an alternative is \em mana asduuduk \em {ex:func:quest:content:loc:asduuduk}.


\xbox{16}{
\ea \label{ex:func:quest:content:loc:ka}
\gll maana(=ka) se=ppe thoppi? \\
     where=\textsc{loc} \textsc{1s=poss} hat \\
\z
} \\
\xbox{16}{
\ea \label{ex:func:quest:content:loc:nang}
\gll  maana(=nang) arà-pii? \\
      where=\textsc{dat} \textsc{non.past}-go \\
\z
} \\
\xbox{16}{
\ea \label{ex:func:quest:content:loc:dering}
\gll maana*(=dering) inni arà-dhaathang? \\
     where=\textsc{abl} \textsc{prox} \textsc{non.past}-come  \\
\z
} \\
\ea \label{ex:func:quest:content:loc:asduuduk}
\gll maana asduuduk anà-dhaathang? \\
     where from \textsc{past}-come  \\
    `where do you come from?' (nosource)(test)6.11.08
\z
} \\

If the query is about location in one of a given array of items, \em aapa=ka \em or   can also be used, both meaning `in which X, at which X etc.'


\xbox{16}{
\ea
\gll itthu thoppi=yang aapa=ka aada? \\
      \textsc{dist} hat=\textsc{acc} what=\textsc{loc} exist \\
    `On what is that hat?' (nosource)(test)
\z
} \\

% \xbox{16}{
% \ea
% \gll itthu thoppi=yang siini aada. \\
%     \textsc{dist} hat=\textsc{acc} here exist    \\
%     `That hat is here.' (nosource)6.11.08 better with yang
% \z
% } \\


\subsubsection{Time}\label{sec:func:q:Time}
General temporal reference can be queried with \trs{kaapang}{when}.


\xbox{16}{
\ea
\gll kaapang loram pada siini arà-dhaathang? \\
     when \textsc{2pl} \textsc{pl} here \textsc{non.past}-come  \\
\z
} \\

An amount of time can be queried with \trs{dhraapa laamar}{how long}(Literally \em how much while.\em)

\xbox{16}{
\ea
\gll ini pukuran=nang dhraapa laama athi-ambel? \\
     \textsc{prox} work=\textsc{dat} how.much while \textsc{irr}-take  \\
    `How long will it take for this work?' (nosource)(test)6.11.08
\z
} \\

The time can be queried with \trs{pul-dhraapa}{At what time}(Literally \em at how much o'clock \em).


\xbox{16}{
\ea
\gll skaarang wakthu pukul dhraapa=ke boole=aada? \\
     now time o'clock  how.many=\textsc{undet} can=exist  \\
    `What time will it be now?' (nosource)(test)6.11.08
\z
} \\

\xbox{16}{
\ea
\gll  se pukul dhraapa=nang masa-dhaathang? \\
      \textsc{1s} o'clock how.much-\textsc{dat} must-come \\
    `At what time should I come.' (nosource)6.11.08
\z
} \\

\subsubsection{Manner}\label{sec:func:q:Manner}
\em Caraapa \em is used to query for manner. The inverted form \em ap(a)caara \em also exists.

\xbox{16}{
\ea
\gll See ini koolang=yang arà-langka aapacara/caraaapa. \\
      \textsc{1s} \textsc{prox} river=\textsc{acc} \textsc{non.past}-cross how \\
\z
} \\
% \xbox{16}{
% \ea \label{ex:func:quest:content:manner}
% \gll see thaau ambel \textbf{aapacara} katha. \\
%      \textsc{1s} know take how \textsc{quot}  \\
%     `I learnt how.' (K051206nar07)
% \z
% } \\



\subsubsection{Reason, cause and purpose}\label{sec:func:q:Reasoncauseandpurpose}
These questions are formed by \trs{aapa}{what} followed by  the benfactive postposition \em =nang \em \xref{ex:func:quest:content:purp:aapa}. If the cause is a person, \trs{saapa}{who} can be used instead of \em aapa\em \xref{ex:func:quest:content:purp:saapa}.

\xbox{16}{
\ea \label{ex:func:quest:content:purp:aapa}
\gll lorang=nang inni aapa=nang? \\
     \textsc{2pl}=\textsc{dat} \textsc{prox} what=\textsc{dat}  \\
\z
} \\


\xbox{16}{
\ea \label{ex:func:quest:content:purp:saapa}
\gll ini saapa=nang arà-baapi? \\
      \textsc{prox} who=\textsc{dat} \textsc{non.past}-take.away \\
\z
} \\

A specialized interrogative pronoun to query for reason only is \em kanaapa\em.


\xbox{16}{
\ea
\gll kanaapa itthu anà-billi? \\
     why \textsc{dist} \textsc{past}-buy  \\
\z
} \\




\subsubsection{Other}\label{sec:func:q:Other}
Other questions are formed by combining the interrogative pronouns with the relevant postpositions\formref{sec:wc:Interrogativepronouns}.

\xbox{16}{
\ea
\gll ini peena saapa=pe? \\
     \textsc{prox} pen who=\textsc{poss}  \\
    `Whose pen is this?' (nosource)6.11.08
\z
} \\
\xbox{16}{
\ea
\gll  Farook saapa=ke. \\
      Farook who=\textsc{simil} \\
\z
} \\

\section{Commands}\label{sec:func:Commands}
% Commands do not convey propositional content to the hearer, but invite him to (not) perform  a certain action. Commands are formed by imperative clauses. See the discussion there. Additionally, it is common to formulate requests as declaratives with the modal \trs{boole}{can}. `You can also come later' the means `please come later.' There are no examples of this pattern in the corpus, but I found it very frequently in personal interaction.
% Commands are formed by using the imperative construction, which consist of the bare verb \formref{sec:form:ImperativeClause}.
%
% \xbox{16}{
% \ea\label{ex:func:commands:pos:zero}
% \gll Aajuth thaakuth=ka su-naangis, ``See=yang luppas''. \\
%      dwarf fear=\textsc{loc} \textsc{past}-cry \textsc{1s}=\textsc{acc} leave  \\
% \z
% }\\
%
% An optional enclitic \em =la \em can be added to the verb. This is considered more polite.
%
% \xbox{16}{
% \ea\label{ex:func:commands:pos:la}
% \gll  allah, diyath-\textbf{la} inni pompang pada dhaathang aada. \\
%       Allah see-\textsc{imp} \textsc{dist} female \textsc{pl} come exist \\
% \z
% } \\
%
% Another possibility is to use the preverbal particle \trs{mari}{come}.
%
% \xbox{16}{
% \ea\label{ex:func:commands:pos:mari}
% \gll \textbf{mari} maakang. \\
%  come eat\\
% `Eat!' (B060115rcp02.63)
% \z
% }
%
% If the command is about coming, then \em mari \em can be used alone.
%
% \xbox{16}{
% \ea\label{ex:func:commands:pos:maricome}
% \gll laskalli \textbf{mari}. \\
%  other time come\\
% `Come again!' (G051222nar01.25)
% \z
% }
%
% \em mari \em and \em -la \em can be combined.
%
% \xbox{16}{
% \ea\label{ex:func:commands:pos:marila}
% \ea
% \gll  saayang se=ppe thuan \textbf{mari} laari-\textbf{la}. \\
%       love \textsc{1s=poss} sir come.imp run-\textsc{imp} \\
% \ex
% \gll see=samma kumpul \textbf{mari} thaandak-\textbf{la}. \\
%      \textsc{1s}=\textsc{comit} gather come.imp dance-\textsc{imp}  \\
%     `Come and dance with me.' (N061124sng01)
% \z
% \z
% } \\
% Suggestions can be formed by using the imperative with \em mari \em followed by the interrogative clitic \em =si\em. \em=la \em is not possible in this context.
%
%
% \xbox{16}{
% \ea
% \gll mara-maakang. \\
%        \\
%     `Let's eat.' (nosource)(test)6.11.08
% \z
% } \\
%
% \xbox{16}{
% \ea
% \gll kithang mara maakang si. \\
%        \\
%     `Shall we eat.' (nosource)6.11.08
% \z
% } \\
%
% \xbox{16}{
% \ea
% \gll see ma maakang si. \\
%        \\
%     `Shall I eat' (nosource)6.11.08
% \z
% } \\
Prohibtions can be formed with \em jamà- \em\formref{ex:func:commands:neg:jama} or with \em thussa\em\formref{ex:func:commands:neg:thussa}.

% \xbox{16}{
% \ea\label{ex:func:commands:neg:jama}
% \gll biilang, maalu, \textbf{jamà-maalu}. \\
%      say shy \textsc{neg.nonfin} shy  \\
%     `Speak! You are shy, don't be shy.' (B060115prs07)
% \z
% } \\
\xbox{6}{
\ea\label{ex:func:commands:neg:jama}
\gll see=yang jamà-liiyath. \\
     \textsc{1s}=\textsc{acc} \textsc{neg.imp}-look  \\
\z
}
\xbox{6}{
\ea\label{ex:func:commands:neg:thussa}
\gll se=yang \textbf{thussa} mà-liiyath. \\
     \textsc{ \textsc{1s}=acc} not.want \textsc{inf}-look\\
    `Don't stare at me!'  (eli12012006)6.11.08
\z
}


\section{Negation}\label{sec:func:Negation}
Negation varies with predication type, clause type, tense,  and information structure. Every different type of predicate (verbal, existential, modal, nominal, adjectival, circumstantial) has a corresponding negation pattern. The verbal predicate has an additional negation type used in subordinate clauses. In the domain of tense, we can distinguish past, perfect, non-past and future as relevant tense domains. As for information structure, predicate negation is different from constituent negation. Table \ref{tab:func:negation} gives an overview of the different negation patterns. Please see section \ref{sec:pred} and the relevant subsections for more discussion and examples.

\begin{table}
\begin{center}
% use packages: array
\begin{tabular}{r|c|c|c|c|}
 & past & perfect & present & future \\\hline
\hline
predicate negation&&&&\\\hline
verbal&&&&\\
finite clause & thàrà-V & V thraa &  \multicolumn{2}{|c|}{thama-V}  \\\hline
infinite clause &  \multicolumn{4}{|c|}{jamà}  \\\hline
existential & \multicolumn{4}{|c|}{thraa}  \\\hline
modal&&&& \\ \hline
~(ka)maau(van) & \multicolumn{4}{|c|}{thàrkamauvan/thussa}  \\\hline
~boole & \multicolumn{4}{|c|}{thàrboole}   \\\hline
nominal &  \multicolumn{3}{|c|}{bukang} & thama-jaadi/bukang \\\hline
adjectival1 &  \multicolumn{3}{|c|}{ADJ thraa} & thama-ADJ \\\hline
adjectival1 &  \multicolumn{3}{|c|}{thàrà-ADJ} & thama-ADJ \\\hline
circumstantial &  \multicolumn{3}{|c|}{bukang} &  \\\hline
locational & \multicolumn{4}{|c|}{thraa}\footnotemark  \\\hline
locational & thàràduuduk & duuduk thraa & \multicolumn{2}{|c|}{thama-duuduk}\\\hline
\hline
\hline
constituent negation & bukang &  &  &
\end{tabular}
\end{center}
\caption[Negation patterns for various predicate types and tenses]{Negation patterns for various predicate types and tenses.}
\label{tab:func:negation}
\end{table}
\footnotetext{This suppletive negation of the existential is an exception to the generalization formulated by \citet[138]{Himmelmann2005typochar}, that the existential is negated by the common verbal negator.}

Investigating Table \ref{tab:func:negation}, we observe a number of generalizations. For instance, negation of circumstantial predication and constituent negation are both done by \em bukang\em. This could point to circumstantial predicates not being predicates in their own rights, but rather constituents with an unexpressed overt predicate.

Another item which occurs frequently is \em thraa\em, which is used for perfect tense negation of verbs, negation of locational predicates and negation of adjectives. The first two of this can be explained by the presence of \em aada \em in the affirmative counterpart. Since \em thraa \em is the negative form of \em aada\em, the occurrence of \em thraa \em in perfect and locational predicates where \em aada \em is used in the affirmative is not surprising. The negation of adjectives by \em thraa \em cannot be explained in this manner. Note that speakers differ as to the negation patterns of adjectives. Depending on speaker and individual lexeme, the pattern with \em thraa \em or the pattern with \em thàrà \em can be found.

When \em thàrà- \em is used to negate adjectives, it can be used in all tenses, while in its other use for negating verbs, it necessarily has past tense reference.

For negation of non-verbal predications refering to the future, some special periphrases exist. All these involve the non.past negative marker \em thama- \em and an extra verb, like \trs{jaadi}{become} and \trs{pii}{go}.

Conjunctive participle clauses and purposive clauses are marked by \em jamà- \em when negated. Other subordinate clauses like relative clauses or argument clauses take the same negation as main clauses.

Indicating that none of the possible referents would yield a positive truth value (\em no student came\em) and indicating that for all referents, the truth value is negative (\em The students did not come\em) is done in a very similar fashion in SLM. The only difference between the two is that in the former case the indefinite article \em atthu \em is used before  the noun and a suitable coordinating clitic like \em =le\em, \em =ke \em or \em =pon \em is used after the NP.

\xbox{16}{
 \ea\label{ex:func:neg:le1}
\gll   kithang=pe \textbf{hatthu} oorang=\textbf{le}      {\em minister} jaadi  \textbf{thraa}. \\
        \textsc{1pl}=\textsc{poss} \textsc{indef} man=\textsc{addit} minister become \textsc{neg}  t \\
\z
} \\
\xbox{16}{
 \ea\label{ex:func:neg:le3}
   \gll  derang=nang   Kluumbu=pe    samma {\em association}=le      {\em support}; Kluumbu=pe    \textbf{hatthu} {\em association}=\textbf{le}   kithang=nang   \textbf{thàrà}-{\em support}. \\
    \textsc{3pl}=\textsc{dat} Colombo=\textsc{poss} all association=\textsc{addit} support Colombo=\textsc{poss} \textsc{indef} association=\textsc{addit} \textsc{1pl}=\textsc{dat} \textsc{neg.past}-support after\\
\z
}



\xbox{16}{
 \ea\label{ex:func:neg:pon}
\gll   kithang \textbf{hatthu}=oorang=\textbf{pon} \textbf{thàrà}-iinggath. \\
    `We cannot think of any person.'  (B060115nar02)
\z
}\\

Rarely, the indefinite article and the clitic are both found after then noun.

\xbox{16}{
 \ea\label{ex:func:neg:pon:inv}
\gll see pukaran=\textbf{hatthu=pon} \textbf{thama}=gijja, ruuma=ka arà-duuduk. \\
    `I don't do any work, I stay at home.'  (B060115prs03)
\z
}\\

The clitic attaches after the postposition, as can be seen from the following two examples, where the postposition \em =nang \em intervenes between the noun \trs{oorang}{man} and the clitic \em =le\em.

\xbox{16}{
 \ea\label{ex:func:neg:postp1}
\gll incayang=pe      muusing=ka kithang=pe     {\em Malays} pada \textbf{atthu} \textbf{oorang=nang=le}        [{\em parliament}=nang  mà-dhaathang=nang      thumpath] \textbf{thàrà}-daapath. \\
    `During his time, no man of our Malays got a place to go to parliament (i.e. a seat).' (N061031nar01)
\z
} \\
\xbox{16}{
 \ea\label{ex:func:neg:postp2}
\gll incayang=le       kithang=pe     mlaayu pada \textbf{hatthu} oorang\textbf{=nang=le} thumpath \textbf{thàrà}-kaasi. \\
     3s.polite=\textsc{addit} \textsc{1pl}=\textsc{poss} Malay \textsc{pl} \textsc{indef} man=\textsc{dat}=\textsc{addit} place \textsc{neg.past}-give  \\
\z
} \\
It is also possible to use the combination of \em atthu \em and a clitic without a noun. In this case, the clitic \em =ke \em was also found in the corpus, next to \em =le \em and \em =pon\em.


\xbox{16}{
 \ea\label{ex:func:neg:zeronoun:le}
\gll  laayeng   \textbf{hatthu=\zero=le}      thraa. \\
      different \textsc{indef}=\textsc{addit} \textsc{neg} \\
\z
} \\

\xbox{16}{
 \ea\label{ex:func:neg:zeronoun:pon}
\gll {\em bus}=ka \textbf{hatthu=\zero=pon} mà-kirja thàràboole. \\
      bus=\textsc{loc} \textsc{indef}=\textsc{neg} \textsc{inf}-make cannot \\
\z
} \\

\xbox{16}{
 \ea\label{ex:func:neg:zeronoun:ke}
\gll Snow-white=nang=le Rose-red=nang=le ini \textbf{hatthu=\zero=ke} \textbf{thàrà}-mirthi. \\
    `Snow White and Rose Red did not understand a thing.'  (K070000wrt04)
\z
}\\

\section{Kin}\label{sec:func:Kin}
The SLM kinship system distinguishes between males and females, generations, and relative age. Siblings and cousins are not distinguished, but there are many different types of uncles and aunts, depending on whether they are male or female, maternal or paternal, elder or younger. Parents' elder siblings and their spouses are only distinguished for sex. Parents' younger siblings are also distinguished for sex of the parent, and spouses are different from consanguineous relatives.  Grandparents and grandchildren are only distinguished by sex, while greatgrandparents and greatgrandchildren are not distinguished by sex either, but the word for greatgrandparents includes the morpheme for the grandparents.

The terms \trs{maven}{son, nephew, younger man} and \trs{mavol}{daughter, niece} can also be heard, but it was not clear to which individuals they could refer. When addressing older members of the community, younger speakers use \em uncle, auntie \em (English) or \trs{kaake, neene}{grandfather, grandmother}, depending on the age. When elder speakers address younger ones, they can use \em maven \em or \em mavol\em. Among peers, the sibling terms \trs{kaaka}{elder male}, \trs{dhaatha}{elder female} or \trs{aade}{younger sibling} are used. These follow the proper name, so \trs{Imi kaaka}{Elder brother Imi}.


\begin{sidewaysfigure}
% \begin{figure}
 \centering

\begin{tabular}{c@{\hspace{-0.2cm}}c@{\hspace{-0.2cm}}c@{\hspace{-0.2cm}}c@{\hspace{-0.2cm}}c@{\hspace{-0.2cm}}c@{\hspace{-0.2cm}}c@{\hspace{-0.2cm}}c@{\hspace{-0.2cm}}c@{\hspace{-0.2cm}}}
&
\multicolumn{3}{c}{\pile{0.8}{}{\dreieck}{moyang\\\footnotesize kaake}\hspace{-0.3cm}=\hspace{-0.3cm}\pile{0.8}{}{\kreis}{moyang\\\footnotesize neene}}  &
&
&
\multicolumn{3}{l}{\pile{0.8}{}{\dreieck}{moyang\\\footnotesize kaake}\hspace{-0.3cm}=\hspace{-0.3cm}\pile{0.8}{}{\kreis}{moyang\\\footnotesize neene}}
\\
 &
\pile{0.8}{}{\dreieck}{kaake}\hspace{-0.3cm}=\hspace{-0.3cm}\pile{0.8}{$\mid$\vln{-1}{8}{105}}{\kreis}{neene} &
\pile{0.8}{$\mid$}{\dreieck}{kaake} &
\pile{0.8}{$\mid$}{\dreieck}{kaake}\hspace{-0.3cm}=\hspace{-0.3cm}\pile{0.8}{}{\kreis}{neene} &
 &
\pile{0.8}{}{\dreieck}{kaake}\hspace{-0.3cm}=\hspace{-0.3cm}\pile{0.8}{$\mid$\vln{-1}{8}{109}}{\kreis}{neene} &
\pile{0.8}{$\mid$}{\kreis}{neene} &
\pile{0.8}{$\mid$}{\dreieck}{kaake}\hspace{-0.3cm}=\hspace{-0.3cm}\pile{0.8}{}{\kreis}{neene} &
\\\\
\pile{0.8}{$\mid$\vln{-1}{8}{264}}{\dreieck}{muuda}\hspace{-0.3cm}=\hspace{-0.3cm}\pile{0.8}{}{\kreis}{maami} &
\pile{0.8}{$\mid$}{\kreis}{maami}\hspace{-0.3cm}=\hspace{-0.3cm}\pile{0.8}{}{\dreieck}{muuda} &
\pile{0.8}{$\mid$}{\dreieck}{\textbf{uuva}}\hspace{-0.3cm}=\hspace{-0.3cm}\pile{0.8}{}{\kreis}{uuvama} &
\pile{0.8}{$\mid$\hln{9}{8}{20}}{\kreis}{\textbf{uuvama}}\hspace{-0.3cm}=\hspace{-0.3cm}\pile{0.8}{}{\dreieck}{uuva} &
%
\pile{0.8}{$\mid$}{\dreieck}{baapa}\hspace{-0.3cm}=\hspace{-0.3cm}\pile{0.8}{$\mid$\vln{-1}{8}{238}}{\kreis}{umma} &
%
\pile{0.8}{$\mid$\hln{12}{8}{20}}{\dreieck}{maama}\hspace{-0.3cm}=\hspace{-0.3cm}\pile{0.8}{}{\kreis}{maami} &
\pile{0.8}{$\mid$}{\kreis}{\parbox{0.8cm}{biibi/\\\vspace{-0.3cm}caaci}}\hspace{-0.3cm}=\hspace{-0.3cm}\pile{0.8}{}{\dreieck}{muuda} &
\pile{0.8}{$\mid$}{\dreieck}{\textbf{uuva}}\hspace{-0.3cm}=\hspace{-0.3cm}\pile{0.8}{}{\kreis}{uuvama} &
\pile{0.8}{$\mid$}{\kreis}{\textbf{uuvama}}\hspace{-0.3cm}=\hspace{-0.3cm}\pile{0.8}{}{\dreieck}{uuva}
\\
\\
\multicolumn{2}{l}{\hln{34}{8}{40} \footnotesize uncles' ch same as siblings} &
\pile{0.8}{$\mid$}{\kreis}{aade\\\footnotesize pompang}\hspace{-0.3cm}=\hspace{-0.3cm}\pile{0.8}{}{\dreieck}{kiccil\\\footnotesize aabang} &
\pile{0.8}{$\mid$\hln{15}{8}{20}}{}{EGO\\~} =\parbox{1cm}{\footnotesize laaki\male\\~\\ \footnotesize  biini\female}  &
\pile{0.8}{$\mid$}{\dreieck}{\textbf{kaaka}\\~}\hspace{-0.3cm}=\hspace{-0.3cm}\pile{0.8}{}{\kreis}{umbo\\~} &
\pile{0.8}{$\mid$}{\kreis}{\textbf{dhaatha}\\~}\hspace{-0.3cm}=\hspace{-0.3cm}\pile{0.8}{}{\dreieck}{aabang\\~} &
\multicolumn{2}{r}{\hln{90}{8}{40}\footnotesize aunts' ch same as siblings}
\\\\
\multicolumn{4}{r}{\pile{0.8}{}{\dreieck}{manthu\\}\hspace{-0.3cm}=\hspace{-0.3cm}\pile{0.8}{$\mid$\vln{-1}{8}{120}}{\kreis}{aanak\\\footnotesize klaaki}} &
 &
\multicolumn{4}{l}{\pile{0.8}{$\mid$\hln{-60}{8}{40}}{\dreieck}{aanak\\\footnotesize pompang}\hspace{-0.3cm}=\hspace{-0.3cm}\pile{0.8}{}{\kreis}{manthu\\}}
\\\\
&
&
\pile{0.8}{$\mid$\vln{-1}{8}{68}}{\dreieck}{cuucu}&
\pile{0.8}{$\mid$\hln{-6}{8}{20}}{\kreis}{ciici}&
&
\pile{0.8}{\vln{1}{8}{68}\hln{16}{8}{20}$\mid$}{\dreieck}{cuucu}&
\pile{0.8}{$\mid$}{\kreis}{ciici}&
&
\\
&
 &
\pile{0.8}{\hln{-1}{0}{12}}{\kreis\dreieck}{cangawaari}&
\pile{0.8}{\hln{-1}{0}{12}}{\kreis\dreieck}{cangawaari}&
 &
\pile{0.8}{\hln{-1}{0}{12}}{\kreis\dreieck}{cangawaari}&
\pile{0.8}{\hln{-1}{0}{12}}{\kreis\dreieck}{cangawaari}&
 &
\\\\
\end{tabular}

 % kin.eps: 141954512x146425912 pixel, 300dpi, 120.881.50x1239739.38 cm, bb14 14 1021 422
 \caption[Kinship relations]{Kinship relations in SLM. Bold face denotes elder siblings, where applicable. Normal font denotes younger siblings, if there is a distinction with bold face in the same generation. Relative age of spouses of parents' siblings is not relevant.}
 \label{fig:func:kin}
\end{sidewaysfigure}
% \end{figure}
 

\citep{Kekulawala1982} first studied the kinship system of SLM (cited in \citep{Bichsel}) and analyzed it as lineo-bifurcate collateral type
