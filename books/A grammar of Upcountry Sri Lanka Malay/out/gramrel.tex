
\chapter{Grammatical relations}\label{sec:gramrel}
While in English and many other languages, there is one argument which is more prominent than the other ones, the subject, this is not the case in SLM \citep[cf.][16]{Ansaldo2005ms}. It is very difficult to point out criteria to distinguish the syntactic prominence of the arguments in a clause. Commonly used tests like agreement, word order, case marking, relative clauses or realization of pronouns do not yield any systematic differences. Furthermore, there is no operation  which would change the status of an argument form more prominent to less prominent, or from less prominent to more prominent, as would be the case for passivization in English. After English passivization, the formerly more prominent argument is less prominent, and the less prominent argument is more prominent.
This lack of a difference in prominence between the arguments is easy to formulate from a descriptive point of view, but might be more difficult to incorporate into some current linguistic theories. This grammatical description being theory-neutral, the theoretical predilections of some formalism do not matter in principle, yet given the importance of `subject' in most of linguistic theorizing it is important to provide the emprical groundings on which the absence of this category was arrived at. The criteria which will now be discussed are case marking, agreement, word order, pro-drop, conjunction reduction, equi-NP deletion, control, resumptive pronouns, passivization, binding and obligatoriness.


\kuckn
\citet[152]{Himmelmann2005typochar} lists the following criteria commonly used in Western Austronesian languages to establish grammatical relations (also see \citep{Schachter1995}).\draftnote{Croft1985}

\begin{itemize}
 \item case marking
 \item agreement
 \item word order
 \item pro-drop
 \item conjunction reduction
 \item control
 \item resumptive pronouns
 \item passivization
 \item binding
 \item raising
 \item obligatoriness
 \item relativization
\end{itemize}

Following \citet[250ff]{VanValinEtAl1997}, `subjects' must be instances of 'restricted neutralizations'. `Neutralization' means means that two or more argument receive an identical treatment in syntax although they have different semantic roles. An example is agreement in English, which can be with the actor with active voice verbs, or with the undergoer in passive voice verbs. `Restriction' means that the neutralization must not be generalized: some semantic roles must be excluded from the neutralization. \textsc{instrument} for instance can never trigger agreement in English, so that agreement in English is a neutralization of at least actor and undergoer, but this neutralization is restricted since it excludes the role of \textsc{instrument}, among others.

As far as SLM is concerned, for the majority of the domains named above, we are dealing with unrestricted neutralization, i.e. all semantic roles are treated alike. Cases in point are agreement, word order, pro-drop, conjunction reduction, obligatoriness, and relativizatoin. Some other domains cannot be applied because the relevant morphosyntactic operation does not exist in SLM. This applies to passivization, binding, and raising. As argued for in Section \ref{sec:argstr}, case marking does not neutralize semantic distinctions either. The only domain where we might possibly dealing with restrictive neutralization is `control', but even there, the case is far from clear. The different domains will now be discussed in turn. 

 

\section{Case marking}\label{sec:grel:Casemarking}
In many languages, the subject of a transitive sentence is case-marked identically to the only argument of an intransitive sentence. In German, the A argument and the S argument are both marked for nominative, in Basque, the P argument and the S argument are both marked for absolutive. 
As discussed in \formref{sec:argstr:Summaryofargumentrstructure}, SLM cases are assigned on a semantic basis, and the syntactic nature and valency of the arguments' predicate matters little. The only argument of an intransitive predicate (S) can be marked for dative, instrumental, accusative, or zero. It is thus not possible to decide which one of the two arguments of a transitive predication is case-marked like S. Under a strict view, none of the arguments of the transitive clause has the exactly the same case-marking distribution as S, since A is never marked for accusative, and P is never marked for instrumental. Under a more loose view, both A and P are marked like S in that they can take dative and zero marking. Case marking thus does not provide a good criterion to decide which one of the two arguments of a transitive clause is more like the only argument of an intransitive clause, in other words, `subject' cannot be determined based on the criterion of case marking.

\section{Agreement}\label{sec:grel:Agreement}
Another feature which can be used to determine the subject is the agreement pattern. If only one argument triggers agreement on the verb, this is the subject. In SLM, there is no generalized agreement, and the cases of agreement found so far \formref{sec:cls:Agreement} are very weak and only found in some idiolects. For the time being, it seems safe to assume that SLM has no agreement as such, so that this feature cannot be used to determine subjecthood of arguments.

\section{Word order}\label{sec:grel:Wordorder}
In some language, word order gives cues as to the subjecthood of the argument, e.g. in English, where the subject must precede the verb.  Two prominent positions are to be investigated: the position immediately preceding the verb and the position of the first argument in the clause. Furthermore, the distribution has to be controlled for the status of the arguments as nouns or pronouns, since these might behave differently with regard to placement. We will limit the discussion to the position of A. The following example shows that A can be found in a position which is neither the first one, nor the closest to the the verb. In this example, A is pronominal.

\xbox{14}{
\ea \label{ex:cls:grel:wo:A:position:middle}
\gll [itthu    baathu=yang]$_P$    [\textbf{incayang}]$_A$  [Seelong=dering]         [laayeng    nigiri=nang] asà-baapi. \\
 \textsc{dist} stone=\textsc{acc} \textsc{3s.polite} Ceylon=\textsc{abl} other country=\textsc{dat} \textsc{cp}-bring\\
\z 
}

Example \xref{ex:cls:grel:wo:A:position:middle} shows that a likely argument for subjecthood, the topical \trs{incayang}{3s.polite}, is not found in initial position. It is not adjacent to the verb either. This suggests that \em incayang\em, the topical agent, does not have to occur in any particular position, unlike English, where \em he \em would have to occur before \em brought\em, there is no other possibility in English.

A similar situation obtains in \xref{ex:cls:grel:wo:position:A:middle2}, where again the pronominal agent is sandwiched between the other arguments.

\xbox{14}{
\ea \label{ex:cls:grel:wo:position:A:middle2}
\gll  [itthu    thumpath=yang]$_P$   [incayang]$_A$  [giithu]=jo      [awuliya=nang]=jo      su-kaasi. \\
      \textsc{dist} place=\textsc{acc} \textsc{3s.polite} like.that=\textsc{foc} saint=\textsc{dat}=\textsc{foc} \textsc{past}-give \\
\z
} \\
The above examples treated pronominal arguments. The `sandwich' position of a nominal argument is given in \xref{ex:cls:grel:wo:position:A:middle:noun}


\xbox{14}{
\ea \label{ex:cls:grel:wo:position:A:middle:noun}
   \gll  [kithang=pe     oorang thuuwa pada=yang]$_P$   [{\em Dutch}]$_A$   [Seelong=nang]  anà-aaji     baa. \\
    man old \textsc{pl}=\textsc{acc} Dutch  Ceylon=\textsc{dat} \textsc{past}-bring bring  \\
\z
}

The three examples \xref{ex:cls:grel:wo:A:position:middle}-\xref{ex:cls:grel:wo:position:A:middle:noun} show that the A-argument does not have to occur in a privileged position. This is true of both nominal and pronominal arguments. These examples also show, that a nominal P-argument can occur in the first position of the clause. We still have to show, that a pronominal P-argument can occur in the first position of a clause \xref{ex:cls:grel:wo:position:P:initial:pronoun} and that P-arguments can occur in other positions in the clause than the initial one.
Example \xref{ex:cls:grel:wo:position:P:initial:pronoun} shows the use of a pronominal P in initial position.

\xbox{14}{
\ea \label{ex:cls:grel:wo:position:P:initial:pronoun}
\gll    [incayang=yang]$_P$ [siaanu]$_A$ asà-buunung   thaaro=apa. \\
    \textsc{3s.polite}=\textsc{acc} 3s.prox \textsc{cp}-kill put=after \textsc{quot}   \\
    `This one has killed him.' (K051220nar01)(test)3.11.08
\z
} \\
Example \xref{ex:cls:grel:wo:position:P:sandwich:noun} shows the use of a nominal P-argument sandwiched between two other arguments, and thus in no privileged position. Example \xref{ex:cls:grel:wo:position:P:sandwich:pronoun} shows the same for a pronominal argument.

\xbox{14}{
\ea \label{ex:cls:grel:wo:position:P:sandwich:noun}
\gll [Aanak pompang duuwa=le,    derang=pe    umma=le]$_A$   [Buruan=yang]$_P$  [ruuma daalang=nang]   su-panggel. \\
     child female two \textsc{3pl}=\textsc{poss} mother=\textsc{addit} bear=\textsc{acc} house { }  inside=\textsc{dat} \textsc{past}-call  \\
\z
} \\

\xbox{14}{
\ea \label{ex:cls:grel:wo:position:P:sandwich:pronoun}
\gll boole likkas=ka [see]$_A$ [lorang=yang]$_P$ [mliige=nang] anthi-panggel]. \\
      can quick=\textsc{loc} \textsc{1s} \textsc{2pl}=\textsc{acc} palace=\textsc{dat} \textsc{irr}=call \\
\z
}\\

Up to now, we have seen that there is no special relative order between the A argument and the P argument. We now turn to arguments marked with the dative. The following two examples show that the relative order of the zero-marked argument and the dative-marked argument is free. \xref{ex:cls:grel:wo:position:zerodat} has the zero-marked argument in first position, while \xref{ex:cls:grel:wo:position:datzero} has the dative occurring first.


\xbox{14}{
\ea \label{ex:cls:grel:wo:position:zerodat}
\gll  [se=ppe    baapa=\zero{}]  [incayang=nang]$_{dat}$    ummas$_{\zero}$ su-kaasi. \\ % bf
      \textsc{1s}=\textsc{poss} father \textsc{3s.polite}=\textsc{dat} gold \textsc{past}-give \\
    `.' (K070000wrt04)
\z
} \\

\xbox{14}{
\ea \label{ex:cls:grel:wo:position:datzero}
\gll [se=dang]$_{dat}$ laiskali  [se=ppe    bìnnar mosthor]$_{\zero}$ anà-jaadi. \\ % bf
     \textsc{1s=dat} again \textsc{1s}=\textsc{poss} real way \textsc{past}-become \\
    `I became my old self again [because the curse was broken].' (K070000wrt04)
\z
} \\
Given that the distribution of zero/dative on pronoun/noun  is the same in \xref{ex:cls:grel:wo:position:zerodat} and \xref{ex:cls:grel:wo:position:datzero}, but the order is different, no privileged position seems to exist.

To sum up, word order does not seem to give a clue as to the relative prominence of arguments in 2+-place-predicates. Word order is pragmatically conditioned as common in South Asia \citep{Bickel2004syntexp}.\footnote{See \citet[62]{Bayer2004} and references therein for a more extensive list of preferences influencing word order in languages with free word order.}




%  \xbox{14}{
%  \ea\label{ex:constr:unreferenced}
%    \gll  cinggala  incayang=nang    thàràthaau. \\
%     Sinhalese \textsc{3s.polite}=\textsc{dat} ignore \\
% `He didn't know Sinhala' (K060108nar02)
% \z
% }



\section{Pro-drop}\label{sec:grel:Pro-drop}
In some languages, the possibility to drop certain arguments but not others can be used as a test for subjecthood. In Standard Spanish, for instance, only subjects can be dropped, but objects can not. If the realization of an argument is optional in Spanish, this is a clear indicator that that argument must be the subject.

SLM differs from Spanish in that any argument can be dropped \citep{SmithEtAl2004}.\footnote{This is common in both western Austronesian \citep[171]{Himmelmann2005typochar} and Sinhala \citep[813]{Gair2003} and Tamil \citep[367]{Lehmann1989}.} \xref{ex:cls:grel:prodrop:agent} shows the dropping of an agent.
\xbox{14}{
\ea \label{ex:cls:grel:prodrop:agent}
\gll   itthu=nang      blaakang, \zero{}$_A$ dee=yang   su-peegang     siini Seelong=ka. \\ % bf
      \textsc{dist}=\textsc{dat} after { } 3=\textsc{acc} \textsc{past}-catch here Ceylon=\textsc{loc} \\
\z
} \\
The dropping of the agent and the dropping of postpositions are independent of each other. The following example shows a dropped agent and a zero-marked patient/theme, which could easily lead to confusion.

\xbox{14}{
\ea \label{ex:cls:grel:prodrop:agent:casedrop}
\gll \zero{}$_A$  derang=\zero{}=le      su-baawa. \\ % bf
      { } \textsc{3pl}=(postp)=\textsc{addit} \textsc{past}-bring \\
\z
} \\
From the context, it is clear that the persons refered to as \trs{derang}{they} are undergoing the action of bringing rather than performing it.

While we have seen dropping of the agent above, the following examples show the dropping of patient or theme. Example \xref{ex:cls:grel:prodrop:patient1} is about the speaker and his parents, and the event of being sent to an uncle's house for education. It is clear from cultural knowledge that it is necessarily the parents who were sending the child to the uncle, and not the other way round. This means that the parents are agent and the speaker in his childhood is theme, but it is the speaker that is dropped.

\xbox{14}{
\ea \label{ex:cls:grel:prodrop:patient1}
\gll  hathu thaaun  [se=ppe umma-baapa] \zero$_{theme}${}  [se=ppe maama=pe ruuma=nang] su-kiiring. \\ % bf
      \textsc{indef} year \textsc{1s}=\textsc{poss} mother-father { } \textsc{1s}=\textsc{poss} uncle=\textsc{poss} house=\textsc{dat} \textsc{past}-send \\
\z
} \\
This speaker's educational carreer continues and he is admitted to college in \xref{ex:cls:grel:prodrop:patient2}. Again, the agent is mentioned (\trs{derang}{they}) but not the patient/theme.

\xbox{14}{
\ea \label{ex:cls:grel:prodrop:patient2}
\gll s-riibu   sbiilan raathus empathpulu   thuuju thaaun,   derang \zero{}$_{pat/theme}$ Badulla=ka    Dharmadhuutha {\em College}=nang   anà-{\em admit}-kang. \\ % bf
     one-thousand nine hundred four-ty seven year \textsc{3pl} { } Badulla=\textsc{loc} Dharmadhuutha College=\textsc{dat} \textsc{past}-admit-\textsc{caus}  \\
\z
} \\
Interestingly, the same speaker also uttered a sentence where the agent is dropped and the theme realized, but the theme is not case marked (cf. discussion of \xref{ex:cls:grel:prodrop:agent:casedrop} above).

\xbox{14}{
\ea \label{ex:cls:grel:prodrop:patient3}
\gll [mà-blaajar=nang] \zero{}$_A$ see=\zero{}$_{theme}$ anà-kiiring [se=ppe maama hatthu=pe ruuma=nang]. \\  % bf
     \textsc{inf}-learn=\textsc{dat} { } \textsc{1s} \textsc{past}-send \textsc{1s}=\textsc{poss} uncle \textsc{indef}=\textsc{poss} house=\textsc{dat}  \\
\z
} \\
% Dropping of both agent and patient is also possible, as shown in the following example about the insertion of a talisman in a leg for protection against bullets. There is a lot of dropping in this example, but we will concentrate on the last clause, the closing of the wound of the leg by suture. In that clause, neither the agent of the stitching nor the patient (person, leg or wound) is mentioned, but all of them can be inferred from context and knowledge of the world. Note that the patient of the stitching is introduced in the fourth clause (\trs{kaaki}{leg}), while in the fifth clause, the patient changes to the talisman (expressed by zero), before a zero anaphora is used in the sixth clause to refer again to the leg. The hearer is expected to infer that the event of stitching could not possibly include a talisman in the role of patient, so that the hearer should retrieve another suitable patient, in that case the leg.
% 
% \xbox{14}{
% \ea \label{ex:cls:grel:prodrop:agpat1}
% \ea
% \gll jiimath  hatthu thaaro \zero{}$_{agent}$ kal-aada. \\ % bf
%      talisman \textsc{indef} put { } when-put  \\
%     `When he is wearing a talisman,'
% \ex
% \gll \zero{}$_{agent}$ jiimath=yang    thaaro kalu. \\ % bf
%      { } talisman=\textsc{acc} put if  \\
%     `if he wears the talisman'
% \ex
% \gll   inni     kaaki daalang=ka=jo   \zero{}$_{agent}$    bìlla          abbis    thaaro aada. \\ % bf
%        \textsc{prox} leg inside=\textsc{loc}=\textsc{foc} { } chop finish put exist\\
%     `he had cut open the leg and put the talisman inside'
% \ex
% \gll \zero{}$_{agent}$  kaaki poothong. \\ % bf
%       { } leg cut \\
%     `cut that leg'
% \ex
% \gll  \zero{}$_{agent}$ \zero{}$_{{\em talisman}}$ daalang=ka    thaaro=apa. \\ % bf
%      { } { } inside=\textsc{loc} put=after \\
%     `having put (the talisman) inside'
% \ex
% \gll \zero{}$_{agent}$ \zero{}$_{{\em leg}}$ jaaith aada. \\
%      { } { } sew exist  \\ % bf
%     `and sewed it.' (K051206nar02)(test)5.11.08
% \z
% \z
% } \\  redo indices
Simultaneous dropping of A and P is also possible, as shown in  \xref{ex:cls:grel:prodrop:agpat2}, where both \trs{kithang}{we} and \trs{aanak pada}{children} are dropped in the final clause.


\xbox{14}{
\ea \label{ex:cls:grel:prodrop:agpat2}
\ea
\gll luwar   nigiri  [kithang]$_i$=nang   mà-pii    thàrà-suuka. \\ % bf
 outside country \textsc{1pl}=\textsc{dat} \textsc{inf}-go \textsc{neg}-like\\
\ex
\gll  nni      nigiri=ka=jo     [kitham=pe     aanak buwa pada]$_j$=yang   asà-simpang. \\ % bf
      \textsc{prox} country=\textsc{loc}=emph \textsc{1pl}=\textsc{poss} child fruit \textsc{pl}=\textsc{acc} \textsc{cp}-keep \\
\ex
\gll  inni  {\em schools} pada=nang \zero{}$_i$ \zero{}$_j$  asà-kiiring. \\ % bf
      \textsc{prox} schools \textsc{pl}=\textsc{dat} { } { } \textsc{cp}-send \\
\z
\z
} \\
Finally, the following example shows the dropping of all three arguments of the verb \trs{juuwal}{sell}, the agent \em Sindbad\em, the recipient \trs{soojor}{Europeans} and the theme \trs{lakuan bathu}{gems}.

 \xbox{14}{
\ea \label{ex:cls:grel:prodrop:triple}
 \ea
   \gll  incayang$_i$  [ini      Seelong=ka  anà-aada    lakuan   baathu]$_j$ asà-caari. \\ % bf
    \textsc{3s.polite} \textsc{prox} Seelon=\textsc{loc} \textsc{past}-exist wealth stone \textsc{cp}-find \\ % bf
 \ex
   \gll  [soojor   pada]$_k$ [incayang=sasaama]$_i$  [Seelong=nang]  asà-dhaathang. \\ % bf
    European \textsc{pl} \textsc{3s.polite}=\textsc{comit} Ceylon=\textsc{dat} \textsc{cp}-come \\
\ex
   \gll  \zero$_{i}$ \zero$_{j}$ \zero$_{k}$ blaangan arga=nang        anà-juwal. \\ % bf
    	{ } { } { } amount   expensive=\textsc{dat} \textsc{past}-sell \\
\z
\z
}

To sum up, dropping of arguments is not influenced by their syntactic status. Dropping cannot be used to distinguish between more and less prominent arguments.

\section{Conjunction reduction}\label{sec:grel:Conjunctionreduction}
Related to the topic of pro-drop is the phenomenon of conjunction reduction. If in the non-initial part of coordinated sentences, an argument can be dropped it must be the subject. This test can be used in English for example, where the sentence \em Mary hit John and \zero{} ran away \em can only mean that Mary ran away, hence \em Mary \em is the subject. 

The test of conjunction reduction works well with English because English does not allow pro-drop in the first place. As discussed in the preceding section, pro-drop of any argument is wide-spread in SLM, and it is difficult to decide whether a given sentence should be analyzed as pro-drop or conjunction reduction.\footnote{Cf. \citet[171]{Himmelmann2005typochar}}
The following examples show instances of what in English would be conjuction reduction, but it might as well be analyzed as simple pro-drop in SLM, parallel to the examples in the preceding section.

The subordinate clause of example \xref{ex:cls:grel:conjred:agpat1} introduces both agent (\trs{kithang}{we}) and patient \trs{mlaayu makanan}{Malay food}. In the main clause, neither of them is repeated, both are dropped/reduced. Note that in the English translation \em it \em refering to the food is obligatory while this is not the case in SLM.



\xbox{14}{
\ea \label{ex:cls:grel:conjred:agpat1}
\ea
\gll  kithang$_i$ puaasa  wakthuka [mlaayu makan-an]$_j$        asà-gijja,. \\ % bf
       \textsc{1pl} fasting time=\textsc{loc} Malay eat-\textsc{nmlzr} \textsc{cp}-make  \\
\ex  
\gll   \zero{}$_i$ \zero{}$_j$   ruuma pada=nang   arà-kaasi. \\ % bf
       { } { } house \textsc{pl}=\textsc{dat} \textsc{non.past}-give\\
\z
\z
} \\
A similar case is found in \xref{ex:cls:grel:conjred:agpat2}, where the fourth and final clause consists of the predicate \em arà-juuwal, \em which takes two arguments, one seller and one produce. The seller \trs{incayang}{he} is introduced in the first clause, the produce \trs{itthu samma}{all that}(refering to vegetables mentioned earlier) is introduced in the third clause. Both of them are not overtly realized in the final clause (Note again the obligatory \em it \em in the English translation).


\xbox{14}{
\ea \label{ex:cls:grel:conjred:agpat2}
\ea
\gll incayang=jo$_i$ asà-pii \\ % bf
 \textsc{3s.polite}=\textsc{foc} \textsc{cp}-go\\
 \ex
 \gll  paasar=dang asà-pii \\ % bf
 shop=\textsc{dat} \textsc{cp}-go\\
 \ex
 \gll  itthu samma$_j$ asà-baa \\ % bf
 \textsc{dist} all \textsc{cp}-bring \\
 `brings all that and'
 \ex
  \gll \zero$_i${} \zero$_j${} arà-juuwal. \\ % bf
	`sells it.' (nosource)
\z
\z
}

To sum up, conjunction reduction, if it is indeed a useful concept in SLM at all, could not be used to distinguish between A and P and hence does not allow to judge the relative prominence of these arguments.


\section{Control}\label{sec:grel:Control}
The only instance where we can find  a glimpse of a privileged argument in SLM are control structures. Whereas generally in SLM, a non-expressed argument can have any role, in control structures with \trs{kamauvan}{want/need}, it must be the agent. As a test, the verb \trs{ciong}{kiss} was taken, because desire can be expressed to be the acting or the undergoing part in the act of kissing, which is more difficult for other verbs, like eating or killing. In a predication `X wants to be part in the act of kissing', will X be associated with the kisser, the kissed, or either? To test this the context of the fairy tale ``The Frog Prince'' was taken, where a frog wants to be kissed to regain is former self, a prince. The pragmatic context thus forces an association of the wanter with the undergoer. However, this is not what we find. The sentence as in \xref{ex:cl:grel:control:kiss:subj} can only mean that the frog wants to be the actor in the act of kissing.


\xbox{14}{
\ea\label{ex:cl:grel:control:kiss:subj}
\gll se=dang mà-ciong kamauvan. \\ % ciom
     \textsc{1s=dat} \textsc{inf}-kiss want  \\
\z
} \\
If the, pragmatically more appropriate, reading of the frog wanting to be the undergoer needs to be expressed, \xref{ex:cl:grel:control:kiss:obj} must be used, where an indefinite pronoun \trs{saapake}{somebody} instantiates the actor. This entails that the frog cannot be the actor anymore, and hence is the undergoer.


% \xbox{14}{
% \ea
% \gll koodok su biilang: se=dang saapake massa ciong katha. \\
%        \\
%     `frog wants to be kissed.' (nosource)
% \z
% } \\
\xbox{14}{
\ea\label{ex:cl:grel:control:kiss:obj}
\gll se=dang$_i$ [saapa=ke \zero{}$_i$ mà-ciong] maau. \\
     \textsc{1s=dat} who=\textsc{simil} \textsc{inf}-kiss want   \\
\z
} \\
This shows that \trs{kamauvan}{want} is an agent-controller in SLM. The wanter is automatically assigned the agent role in the complement clause. This is the case even if the the wanter is assigned dative case in the main clause, and the undergoer (as in the case of \trs{ciong}{kiss}) would also be assigned dative case (\em se=dang\em). Agent-control is thus a stronger force than the desire to associate arguments which take the same case marking. This control relation need not be syntactic (`subject-control', `syntactic pivot'), however. It is also possible that we are dealing with an `actor control' relation (`semantic pivot') cf . \citet[257]{VanValinEtAl1997}.

\section{Resumptive pronouns}\label{sec:grel:Resumptivepronouns}
In some languages, resumptive pronouns can be used to identify whether an argument is subject or not. SLM uses no resumptive pronouns, so that this test cannot be applied.

\section{Passivization}\label{sec:grel:Passivization}
Passivization can be used in some languages to identify arguments. This test is useful if an argument cannot be used for a certain operation before passivization, but after passivization, this is possible. That argument has then been promoted to subject status. As an example, \em Mary$_i$ hit John$_j$ and \zero{}$_i$ ran away \em can only mean that the hitter (\em Mary \em in this case) ran away. If we want the victim to run away, we can use passivization to promote the former patient to subject: \em John$_i$  was hit by Mary$_j$ and \zero{}$_i$ ran away\em.

SLM has no such operation. There is the vector verb \em kìnna \em and the (probably etymologically related prefix) \em kana-\em, but they do not change the syntactic status of the arguments. What they do is to add an element of adversity and surprise, but this is semantic and has no syntactic repercussions.

The following two examples show that the argument structure of verbs does not change through the use of \em kìnna\em. In \xref{ex:cl:grel:kinna}, \em kìnna \em is present and we find the patient marked with zero and the agent/recipient marked with the dative.


\xbox{14}{
\ea\label{ex:cl:grel:kinna}
\gll Sdiikith thaaun=nang duppang see ini Aajuth=nang su-kìnna daapath. \\ % bf
      few year=\textsc{dat} before \textsc{1s} \textsc{prox} dwarf=\textsc{dat} \textsc{past}-kìnna get \\
\z      
}\\

Marking of arguments is exactly identical in \xref{ex:cl:grel:kinna:contr}, but  \em kìnna \em is optional there, because entering a relationship with a man is not necessarily surprising or undesirable, whereas being captured is.

\xbox{14}{
\ea\label{ex:cl:grel:kinna:contr}
\gll se=ppe mavol ini oorang=nang su-(kìnna) daapath. \\
     \textsc{1s}=\textsc{poss} daughter \textsc{prox} man=\textsc{dat} \textsc{past}-kìnna get  \\
\z
} \\

This shows that \em kìnna \em does not change the syntactic status of arguments, but instead adds a semantic element. This can be obligatory if the expressed proposition forces an adversative reading, but if the proposition is neutral with regard to its being undesirable, \em kìnna \em is optional. If the proposition is clearly not a case of an adversative event, \em kìnna \em cannot be used. To conclude, \em kìnna \em is not a passivization operation which changes the prominence of arguments. It can therefore not be used to test for subjecthood.

\section{Binding}\label{sec:grel:Binding}
Binding of reflexive anaphora can be used to establish subjects in some languages. SLM has no reflexive pronouns, so that this test cannot be used.

\section{Raising}\label{sec:grel:Raising}
There are no raising verbs in SLM, so that this test cannot be used. The function fulfilled by raising verbs in English is coded by the enclitic \em =ke \em or the evidential marker \em kiyang\em, which are located in the same clause as the predicate.

\section{Obligatoriness}\label{sec:grel:Obligatoriness}
In some languages, a subject must be syntactically present even if there is none semantically. Cases in point are meteorological verbs like \em it is raining \em in English, where \em it \em is a dummy subject. In SLM, clauses without arguments are perfectly fine, like \trs{\zero{} arà-uujang}{it is raining}, so that obligatoriness cannot be used to establish the category of subject in SLM.

\section{Relativization}\label{sec:grel:Relativization}
In some languages, only certain arguments can be relativized on. In SLM, any argument can be relativized on, so that this test is not useful either. See \formref{sec:cls:Relativeclause} for an extensive discussion of relativization.

\section{Conclusion}\label{sec:grel:Conclusion}
It is very difficult to establish grammatical relations in SLM. The only instance where the notion of `subject' seems to matter is in control structure, like with \trs{kamauvan}{want}. For the following discussion of clause types, the notion of `subject' is not relevant (nor is the notion of `object'). In discussing the structure of clauses, we will only refer to \em NP\em, without implying a particular syntactic role. This means that for instance word order in the main clause is described as \em NP NP V \em (or \em NP* V\em), rather than \em SOV\em.

% K061019prs01.trs:piisang le ara thaaro
% K061019prs01.trs:itthu nang blaakang
% K061019prs01.trs:kithang asa mnaaji
% K061019prs01.trs:samma ruuma pada nang
% K061019prs01.trs:ara kaasi

 

% 
% \xbox{14}{
% \ea\label{ex:constr:unreferenced}
% \gll \zero{} see=yang {\em stage}=ka asà-thaaro, \zero{}   mic asà-thaaro, \zero{} mà-biilang arà-kaasi. \\
%       { }   \textsc{1s}=\textsc{acc} stage=\textsc{loc} \textsc{cp}-put { } { } mic \textsc{cp}-put { }  \\
% \z
% } \\
 
