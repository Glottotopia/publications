\chapter{Pragmatics}\label{sec:func:Pragmatics}
After we have seen which are the elements and constructions of SLM grammar, and how different meanings can be expressed by them, we now turn to the different communicative goals that speakers may want to achieve through the use of these functions. We will first discuss different types of speech acts (requesting confirmation, requesting action, answering) and then turn to 

\section{Speech acts}\label{sec:pragm:Speechacts}


\subsection{Providing information}\label{sec:pragm:Providinginformation}
If a speaker wants to provide information, he uses assertion.
Assertion has been dealt with extensively in the preceding chapters, and all relevant information can be taken from there.

Speakers do not only want to provide information. Sometimes, they also want the hearers to do something, e.g. provide missing information or perform a certain action. They might also want to express their agreement with the hearer or someone else performing a certain action. This will be dealt with in the following chapters.

\subsection{Requesting information}\label{sec:pragm:Requestinginformation}
Requesting information can be divided into two subtypes: requesting information about the truth of a certain proposition, and requesting information about the content of a certain proposition. The former type is called `polar question' and the latter, `content question'.

Content questions are formed in SLM by means of an interrogative clause for content \formref{sec:cls:InterrogativeclauseWH}, which resemble a normal clause, but replace the NP or PP queried for by an interrogative pronoun like \trs{saapa}{who}. It is common to ask content questions.

The second possibility are polar questions, which are formed by a YN-interrogative clause \formref{sec:cls:Interrogativeclauseclitic}, or by a declarative clause followed by the tag \trs{bukang}{isn't it?}

The clitic \em =si \em is neutral as to the expected answer. In example \xref{ex:pragm:reqinf:neutral}, the speaker is not sure whether she has the obligation to tell her age or not. If a positive answer is expected, the declarative sentence followed by  \trs{bukang}{isn't it} is used\xref{ex:pragm:reqinf:bukang1}\xref{ex:pragm:reqinf:bukang2}.


\xbox{16}{
\ea\label{ex:pragm:reqinf:neutral}
\gll se=pe uumur massa-biilan=si? \\ % bf
 1=\textsc{poss} age must-tell=\textsc{interr}\\
`Do I have to tell my age?' (B060115prs01)
\z
}

\xbox{16}{
\ea\label{ex:pragm:reqinf:bukang1}
\gll itthu    saala, bukang?\\ % bf
\textsc{dist} wrong \textsc{neg.nonv} \\
\z
}

\xbox{16}{
\ea\label{ex:pragm:reqinf:bukang2}
\gll  Kandi=ka ithu Thuan Skiilan aada, bukang? \\ % bf
      Kandy=\textsc{loc} \textsc{dist} Thuan Skiilan exist \textsc{tag} \\
\z
} \\ 

When expecting a negative answer, a negative declarative sentence is used, followed by \em =si\em.

\xbox{16}{
\ea\label{ex:pragm:reqinf:neg=si}
\gll puaasa muusing thàrà-duuduk=si? \\ % bf
fasting season \textsc{neg}-stay=\textsc{interr}\\
`You were not here during the fasting season, were you?' (B060115cvs03)
\z
} 

A third type of information that can be requested is metalinguistic information about a translation of a word into another language. For this, the dative marker is used. The interrogative pronoun is either \em aapa \em as in \xref{ex:pragm:reqinf:metaling:aapa} or \em aapayang \em as in \xref{ex:pragm:reqinf:metaling:aapayang}.

\xbox{16}{
\ea\label{ex:pragm:reqinf:metaling:aapa}
\gll {\em ``simple''}=nang aapa arà-biilang. \\ % bf
     simple=\textsc{dat} what \textsc{non.past}-say  \\
    `What do you say for "simple"?' (B060115cvs01)
\z
} \\

\xbox{16}{
\ea\label{ex:pragm:reqinf:metaling:aapayang}
\gll  {\em ``terrorist''}=nang   apayang arà-biilang,    mlaayu=dering. \\ % bf
      terrorist=\textsc{dat} what \textsc{non.past}-say Malay=\textsc{abl} \\
\z
} \\
\subsection{Requesting confirmation}\label{sec:pragm:Requestingconfirmation}
Requesting confirmation of the expected truth value has been discussed above in the section just preceding, `Requesting information'. Another kind of confirmation which can be requested is whether the hearer is actually following the words of the speaker. This can be done by \trs{thaau=si}{you know?}. The hearer can then confirm verbally or non-verbally and the speaker can continue his turn.

\xbox{16}{
\ea\label{ex:pragm:reqconf:thausi}
\gll bannyak pukurjan anà-kirja    soojor pada-saama,      \textbf{thaau=si}  soojor? \\
     much work \textsc{past}-do European \textsc{pl}=\textsc{comit} know=\textsc{interr} European  \\
    `He worked together with ``soojors''. You know ``soojor''?' (K061026prs01)
\z
} \\
Another possibility to request confirmation is \em =jona \em \formref{sec:morph:=jona}.


\xbox{16}{
\ea\label{ex:pragm:reqconf:jona}
\gll   punnu mlaayu pada kethaama {\em English}=\textbf{jona} anthi-oomong. \\
       many Malay \textsc{pl} earlier English=\textsc{jona} \textsc{irr}=speak \\
\z      
}\\ 


\subsection{Requesting permission}\label{sec:pragm:Requestingpermission}
Permission is requested by adding the interrogative clitic \em =si \em to a verb marked by either the infinitive \em mà- \em or the adhortative \em marà-\em.


\xbox{16}{
\ea
\gll see \textbf{mà}-maakang=\textbf{si}? \\
     \textsc{1s} \textsc{inf}-eat=\textsc{interr}  \\
\z
} \\

\xbox{16}{
\ea
\gll kithang marà-maakang=si? \\
      \textsc{1pl} adhort-eat=\textsc{interr} \\
\z
} \\
 
\subsection{Requesting action}\label{sec:pragm:Requestingaction}
Next to providing and requesting information, speakers also sometimes want other people to perform certain actions, or to refrain from performing certain actions.
The most straightforward way to request action is an imperative construction \funcref{sec:cls:Imperativeclause}. Two examples are given here for positive and negative imperatives.


\xbox{16}{
 \ea\label{ex:pragm:reqact:pos}
   \gll  Binthan {\em auntie}=ka    caanya, binthan {\em auntie}=yang   konnyong panggel. \\  % bf
    Binthan auntie=\textsc{loc} ask, Binthan auntie=\textsc{acc} few call \\
`Ask auntie Binthan, call Binthan auntie' (K060116nar06)
\z
}

\xbox{16}{
 \ea\label{ex:pragm:reqact:neg}
\gll `Thussa mà-thaakuth', Buruan su-biilang. \\ % bf
      \textsc{neg.imp} \textsc{inf}-fear bear \textsc{past}-say\\
\z      
}

Another more subtle way is to use \trs{lorangnang ... (thàr)boole}{you can(not) ...}\xref{ex:pragm:reqact:therboole}, or the verbal prefix \trs{masthi}{must} with an unexpressed referent \xref{ex:pragm:reqact:masthi}. Both imply that it would be preferable that the action be (not) performed, but lack coercive power (in the case of \em (thàr)boole\em) or the direct expression of the addresse (in the case of \em masthi\em).

\xbox{16}{
 \ea\label{ex:pragm:reqact:therboole}
\gll ithu-kapang umma-baapa  su-biilang     lorang=nang   kaaving \textbf{thàrboole}. \\ % bf
     \textsc{dist}=when mother-father \textsc{past}-say \textsc{2pl}=\textsc{dat} marry cannot  \\
    `Then the parents said: ``You cannot marry/Don't marry.''\,' (K051220nar01)
\z
} \\
\xbox{16}{
 \ea\label{ex:pragm:reqact:masthi}
\gll  minnyak klaapa ini raambuth=dering \textbf{masa}-goosok. \\ % bf
      cocnut.oil coconut \textsc{prox} hair=\textsc{abl} must-rub \\
`You must rub coconut oil (over the itching) with (human) hair.'
\z
} \\

\xbox{16}{
 \ea\label{ex:pragm:reqact:maau:mosque}
\gll baapa=nang {\em mosque}=nang mà-pii \textbf{maau}. \\
 father=\textsc{dat} mosque=\textsc{dat} \textsc{inf}-go want\\
\z
}

\xbox{16}{
 \ea\label{ex:pragm:reqact:maau:recipe}
\gll manis-an=nang mà-thaaro guula \textbf{maau} gula paasir konnyong \textbf{maau}. \\
 sweet-\textsc{nmlzr}=\textsc{dat} \textsc{inf}-put sugar want sugar sand few want\\
\z
}





\subsection{Answering}\label{sec:pragm:Answering}
Answering affirmatively is done by \em iiya \em \xref{ex:pragm:answer:yes}, negative answers are given by \em thraa\em, regardless of predicate type.\xref{ex:pragm:answer:no}.


\xbox{16}{
 \ea\label{ex:pragm:answer:yes}
\gll SN: butthul {\em sportsman}, bukang? \\ % bf
     { } correct sportsman, \textsc{tag} \\
\ex
\gll SLM: \textbf{iiya} \textbf{iiya}, {\em sportsman}.    \\
    `Yes indeed, they are good sportsmen.'  \\
\el
\ex
\gll SN: {\em football} aada? \\ % bf
     {  } Football exist  \\
    `Is there football being played in Badulla.'  
\ex
\gll SLM: \textbf{iiya} {\em football} aada. \\ % bf
     {  }  yes football exist \\
    `Yes, there is football.' (B060115cvs01)
\z
} \\

% 
% \xbox{16}{
% \ea\label{ex:pragm:answer:no}
% \gll  ithu=kapang       derang nya-biilang:    \textbf{thraa}, kithang giithu   thama-pii. \\
%       \textsc{dist}-when \textsc{3pl} \textsc{past}-say no \textsc{1pl} like.that \textsc{neg.nonpast}-go \\
% \z
% } \\

It is also posssible to repeat the  predicate of the question for affirmative purposes, or to repeat a verbal predicate and add a negative prefix (\em thàrà-/thama-\em) as a negative answer. This can be seen for example in \xref{ex:pragm:answer:no}, where the predicate \trs{kaaving}{marry} is repeated with the negative prefix \em thama- \em in the answer (Why the non-past prefix is chosen here is unclear, though).

\xbox{16}{
 \ea\label{ex:pragm:answer:no}
\ea
\gll SLM!: Sebastian su-kaaving=si? \\ % bf
     { } Sebastian \textsc{past}-marry=\textsc{interr}?  \\
    `Is Sebastian/Are you married?' 
\ex
\gll  SLM2: thraa thraa,  Sebastian  thama-kaaving. \\ % bf
      { }  no no Sebastian \textsc{neg.nonpast}-marry \\
    `No, no, Sebastian is not married.' (B060115cvs03)
\z
\z
} \\
As for answering content questions, just giving an NP/PP specifying the queried referent is sufficient.
If the answer to a question reasserts a known referent, it is often repeated. This is the case in \xref{ex:pragm:answer:repeat}, where the string \em dìkkath \em is present in the question and hence active in discourse. In the answer, this string is present twice, indicating that the speculation of the hearer as to the vicinity of the school has been understood and is indeed correct.

\xbox{16}{
\ea\label{ex:pragm:answer:repeat}
\ea
\gll Q: skuul dìkkath=si? \\ % bf
 { } school vicinity=\textsc{interr} \\
\ex
\gll dìkkath dìkkath. \\ % bf
vicinity vicinity\\
`Oh yeah, it is close indeed.' (nosource)4.11.08
\z
\z
}

\subsection{Allowing}\label{sec:pragm:Allowing}
Allowing is normally done with the modal \trs{boole}{can}

\xbox{16}{\ea
\ea
\gll dadaa, se=dang sinema mà-liiyath=nang bolle=pii=si? \\
     Daddy \textsc{1s=dat} cinema \textsc{inf}-see=\textsc{dat} can=go-\textsc{interr}  \\
    `Daddy, can I go to the cinema?' 
\ex
\gll  iiya, Izi=nang bolle=pii. \\
      yes, Izi=\textsc{dat} can=go \\
    `Yes, you can go, Izi.' (nosource)4.11.08
\z
\z
} \\
The lexical verb for `to allow' is \em luppas\em, which also means  `leave (behind)'.

\xbox{16}{
\ea
\gll see Izi=yang arà-luppas sinema mà-liyath=nang mà-pii. \\
     \textsc{1s} Izi=yang \textsc{non.past}-leave cinema \textsc{inf}-watch=\textsc{dat} \textsc{inf}-go  \\
\z
} \\
\subsection{Wishing}\label{sec:pragm:...makeawish}
Wishes can be made by invoking Allah. Interestingly, the complement of wishes is marked with the pseudo-prefix \trs{masa-}{must}, instead of an irrealis marker or an infinitive, which could have been expected.


\xbox{16}{
\ea
\gll  Yaa, Allah, se arà-mintha se=dang itthu pukuran masa-daapath katha. \\
      Yeah, Allah, \textsc{1s} \textsc{non.past}-wish \textsc{1s=dat} \textsc{dist} job must-get \textsc{quot} \\
\z
} \\
% 
% \xbox{16}{
% \ea
% \gll se aramintha lorang se=dang masa-daapath katha. \\
%        \\
%     `.' (nosource)4.11.08
% \z
% } \\
% 
% \xbox{16}{
% \ea
% \gll se ara mintha lorang se=ppe massa/?anthi jaadi katha. \\
%        \\
%     `.' (nosource)4.11.08
% \z
% } \\

Wishes can also be indicated by \trs{incalla}{if God wills}.


\xbox{16}{
\ea\label{ex:pragm:wish:incalla}
\gll incalla   [lai     thaau sudaara sudaari pada]=ka    bole=caanya    ambel [nya-gijja    lai     saapa=kee  aada]=si    katha. \\
      Hopefully other know brother sister \textsc{pl}=\textsc{loc} can-ask take \textsc{past}-make other who=\textsc{simil} exist=\textsc{interr} \textsc{quot} \\
\z
} \\

% \subsection{... tell stories}\label{sec:pragm:...tellstories}
% 
% 
% \xbox{16}{
% \ea\label{ex:form:unreferenced}
% \gll  karang liyath, kithang=pe tsunami (pe) atthas[=ka] ini aapa ara(ana)-jaadi katha. \\
%       now see.imp \textsc{1pl}=\textsc{poss} thusnami about=\textsc{loc} \textsc{prox} what \textsc{non.past}-become \textsc{quot} \\
% \z
% } \\


\subsection{Promises}\label{sec:pragm:...promise}
Promises are made with the irrealis marker \em anthi- \em \xref{ex:pragm:promise:anthi} or with the non.past marker \em arà- \em \xref{ex:pragm:promise:sumpa}.

\xbox{16}{
\ea\label{ex:pragm:promise:anthi}
\gll kithang lorang=nang baaye mliiga \textbf{athi}-kaasi,  mà-kaaving panthas pompang pada \textbf{athi}-kaasi,  duwith \textbf{athi}-kaasi. \\
      \textsc{1pl} \textsc{2pl}=\textsc{dat} good palace \textsc{irr}-give \textsc{inf}-marry beautiful girl \textsc{pl} \textsc{irr}-give money \textsc{irr}-give \\
\z
} \\
\xbox{16}{
\ea\label{ex:pragm:promise:sumpa}
\gll see arà-\textbf{sumpa}  paanas muusing dhaathang=thingka see siini=dering \textbf{ara}-pii. \\
     \textsc{1s} \textsc{non.past}-promise hot time come=middle \textsc{1s} here=\textsc{abl} \textsc{non.past}-go  \\
\z
} \\ 
\subsection{Explanations}\label{sec:pragm:...explain}
Explanations are often structured along the pattern of statement, rhetorical question, repetition of statement. The following stretch of discourse is in English, but the pattern is also found in SLM, even though it is not in the corpus.

 
\xbox{16}{
\ea
\ea
\gll two miles we have to walk. \\
       \\
\ex
\gll how many miles? \\
       \\ 
\ex
\gll {\em  two} {\em mile} duuwa kaayu kithang masa-pii. \\
     two mile two mile \textsc{1pl} must-go  \\
    `Two miles we had to walk.' (K051213nar03)
\z
\z
} \\
\subsection{Greeting people}\label{sec:pragm:...greetpeople}
While the other native ethnic groups in Sri Lanka do normally not express greetings verbally, this is different for the Malays.\footnote{Tamil \em vanakkam \em and Sinhala \em Aayu boovan \em are only used for special occasions. One Sinhala informant reported that he could not remember when he had last used the greeting. Things are a bit different for Moors, who use \em Salam aleikum \em on a regular basis.}
The most common greeting is \em slaamath\em, which can be uttered at arrival or departure, and also on the telephone. It can be complemented by the period of the day \trs{slaamath paagi}{good morning}{},\trs{slaamath soore}{good afternoon/evening}{},\trs{slaamath maalang}{good night}{}. More informal greetings do not exist, but English \em hello \em can be used for such purposes.

More formal is \trs{slaamath dhaathang}{welcome}, which can be uttered at the same occasions as its English counterpart.
% \draftnote{sec:pragm:givon2001:319}

\subsection{Taking leave}\label{sec:pragm:...takeleave}
When leaving, one can use \em slaamath \em as described above, or one can use \trs{spi dhaathang}{come and go}. This is modelled on analogous construction in Sinhala and Tamil. The rationale for this is that it is considered unfortunate to say \em I leave \em because that might mean departure from this world. Therefore, one has to specify that one will return, and this is done by adding the \trs{dhaathang}{come} part. Optionally, the \em spi-\em part can be left out, which leaves the curious situation that you can say \em I am coming \em to indicate you are leaving, as in the following example.

\xbox{16}{
\ea\label{ex:pragm:unreferenced}
\gll  se=dang arà-{\em late}, bukang, \textbf{see} \textbf{arà-dhaathang}. \\
    `I am getting late, aren't I, goodbye.' (B060115cvs08)
\z
} \\
\subsection{Thanking}\label{sec:pragm:...thank}
Just as with greetings, Sinhalese and Tamils do not use verbal means for this function, but Malays do. The normal way of thanking is \trs{thriima kaasi}{thank give}.

 
\xbox{16}{
\ea
\gll  (bannyak) thriima kaasi. \\
      much thank give \\
    `Thank you very much.' (nosource)
\z
} \\
% \subsection{... insist}\label{sec:pragm:...insist}
% \subsection{... get angry}\label{sec:pragm:...getangry}
% 
% 
% \xbox{16}{
% \ea
% \gll. \\
%        \\
%     `.' (nosource)
% \z
% } \\


\subsection{Asking a favour}\label{sec:pragm:...askafavour}
Asking a favour is normally done with an interrogative construction involving \em =si\em. Given that the favour is typically beneficial to the asker, the vector verb \trs{kaasi}{give} is often used then.

\xbox{16}{
\ea\label{ex:pragm:askfavour:ambel}
\gll mamaa, se=dang itthuyang anthi-ambe \textbf{kaasi=si}? \\
      mum \textsc{1s=dat} \textsc{dist}=\textsc{acc} \textsc{irr}-take give=\textsc{interr} \\
\z
} \\

\xbox{16}{
\ea
\gll mamaa, se=dang nyaari daaging athi-maasak \textbf{kaasi=si}?   \\
     mum, \textsc{1s=dat} today meat \textsc{irr}-cooked give=\textsc{interr}  \\
    `Mama, can you cook beef for me today?' (nosource)4.11.08
\z
} \\
Favours are accompanied by polite pronouns. It is even possible to use double marked plural forms like \em loram pada \em with singular reference then.

\xbox{16}{
\ea\label{ex:pragm:askfavour:GFG}
\gll Saayang, \textbf{loram} \textbf{pada}=sesama nyaari {\em Galle} {\em Face} mà-pii arà-dhaathang=si? \\
    `Darling, shall I come and take you to the Galle Face Green [a favourite hangout for young couples in Colombo]?' (nosource)4.11.08
\z
} \\


% \xbox{16}{
% \ea
% \gll {\em Auntie}, se ara mintha incayang se=dang massa daapath katha. \\
%        \\
%     `.' (nosource)4.11.08
% \z
% } \\
\xref{ex:pragm:askfavour:auntie} gives a request about a third person.

\xbox{16}{
\ea\label{ex:pragm:askfavour:auntie}
\gll {\em Auntie}, se arà-mintha incayang se=ppe masa-jaadi katha. \\
    auntie \textsc{1s} \textsc{non.past}-ask 3s.polite \textsc{1s=poss} must-become \textsc{quot}   \\
\z
} \\


\subsection{Granting a favour}\label{sec:pragm:...grantafavour}
Granting a favour is done by \trs{iiya}{yes} or \trs{butthul}{correct, OK}. \xref{ex:pragm:grantfavour:ambel} is the positive answer for request \xref{ex:pragm:askfavour:ambel}

\xbox{16}{
\ea\label{ex:pragm:grantfavour:ambel}
\gll iiya, butthul, se athi-ambel kaasi. \\
     yes correct \textsc{1s} \textsc{irr}-take give  \\
\z
} \\
The positive reply to \xref{ex:pragm:askfavour:GFG} is \xref{ex:pragm:grantfavour:GFG}. Note the use of the terms \trs{saayang}{love} for the girl and \trs{jiiwa}{life} for the boy.

\xbox{16}{
\ea\label{ex:pragm:grantfavour:GFG}
\gll  iiya, jiiwa, se  thi-dhaathang. \\
      yes life \textsc{1s} \textsc{irr}-come \\
\z
} \\


\subsection{Declining a favour}\label{sec:pragm:...declineafavour}
The negative answer to \xref{ex:pragm:askfavour:ambel} is \xref{ex:pragm:declinefavour:ambel}, the negative answer to \xref{ex:pragm:askfavour:GFG} is \xref{ex:pragm:declinefavour:GFG}, the negative answer to \xref{ex:pragm:askfavour:auntie} is \xref{ex:pragm:declinefavour:auntie}.

\xbox{16}{
\ea\label{ex:pragm:declinefavour:ambel}
\gll thraa, se loran=nang thama-ambel kaasi. \\
    \textsc{neg} \textsc{1s} \textsc{2pl}=\textsc{dat} \textsc{irr}-take give   \\
\z
} \\

% \xbox{16}{
% \ea
% \gll thraa, se(re)ka duwith thraa. \\
%        \\
%     `.' (nosource)4.11.08
% \z
% } \\

\xbox{16}{
\ea\label{ex:pragm:declinefavour:GFG}
\gll  thookal=si? \\
      fool=\textsc{interr} \\
\z
} \\
\xbox{16}{
\ea\label{ex:pragm:declinefavour:auntie}
\gll thookal, itthu kalu, thama-jaadi. \\
     fool \textsc{dist} if \textsc{neg.irr}-become  \\
    `You fool! As for that, that will never happen!.' (nosource)4.11.08
\z
} \\




\subsection{Telling your age}\label{sec:pragm:...tellyourage}
There are two ways to tell your age, one modelled on the European pattern and one on the Lankan pattern. The European pattern is stating the number of your last birthday, while the Lankan way is stating the number of your next birthday with the verb \trs{gìnnap}{complete}. Both cases put the argument in the dative case, in the examples \em sedang\em.



\xbox{16}{
\ea\label{ex:pragm:speechact:age:european}
\gll karang se=dang ìnnam-pulu liima (thaaun) uumur. \\ % bf
     now \textsc{1s=dat} six-ty five age  \\
    `Now I'm sixty-five.' (N060113nar03)
\z
} \\

\xbox{16}{
\ea\label{ex:pragm:speechact:age:lankan}
\gll   sangke=nyaari se=dang  limapulu  thaaun arà-gìnnap. \\
 until=today \textsc{1s=dat} fifty year \textsc{non.past}-complete\\
`I am in the course of my fiftieth year on earth (= I am 49).' (K060108nar01.15)
\z
}
 





\subsection{Introducing yourself}\label{sec:pragm:...introduceyourself}


\xbox{16}{
\ea 
\gll sudaara sudaari se=ppe naama Wahida Jamaldiin. \\ % bf
     brother sister \textsc{1s=poss} name Wahida Jamaldeen  \\
    `Brothers and sisters, my name is Wahida Jamaldeen.' (B060115prs05)
\z
} \\


\subsection{Expressing  annoyance}
\xbox{16}{
\ea
\gll cumma duuduk-la. \\
     idle stay-\textsc{imp}  \\
    `Mind your own business!' (nosource)5.11.08
\z
} \\

\subsection{Ending a conversation}\label{sec:pragm:...endaconversation}
An explicit way to end a conversation is given in \xref{ex:pragm:end:explicit}. The sentences in \xref{ex:pragm:end:implicit1} and \xref{ex:pragm:end:implicit2} are heard very often, but I am not very sure about their pragmatic implications. They might be a polite hint to stop now, or, on the contrary, they might signal desire to continue.

\xbox{16}{
\ea\label{ex:pragm:end:explicit}
\gll skaarang marà-birthi-king, se=dang lai pukuran aada. \\
     now adhort-stop-caus, \textsc{1s=dat} more work exist  \\
    `Let's stop now, I still have some work to do.' (nosource)4.11.08
\z
} \\

\xbox{16}{
\ea\label{ex:pragm:end:implicit1}
\gll  lai aapa aada?\\
      more what exist \\
    `what else?' (nosource)4.11.08
\z
} \\

\xbox{16}{
\ea\label{ex:pragm:end:implicit2}
\gll lai aapa=ke (mà-oomong) aada=si? \\
     more what=\textsc{simil} \textsc{inf}-talk exist=\textsc{interr}  \\
\z
} \\




\section{Blending in the social tissue}\label{sec:pragm:Blendinginthesocialtissue}
This section is mainly concerned with politeness. The main expression of politeness are the personal pronous (see Table \ref{tab:PersonalPronouns} on page \pageref{tab:PersonalPronouns}). By choosing the polite form, distance is conveyed, whereas the intimate form conveys closeness.

\subsection{Terms of address}
When speaking to  relatives, the term for the relation may be used instead of the 2nd or 3rd person pronoun. \xref{ex:pragm:address:kin:2:baapa} and \xref{ex:pragm:address:kin:2:aanak} show the use of kin terms as terms of address.


\xbox{16}{
\ea\label{ex:pragm:address:kin:2:baapa}
\gll hatthu=le jamà-gijja, \textbf{baapa} ruuma=ka duuduk. \\
      \textsc{indef}=\textsc{addit} \textsc{neg.nonfin} make father house=\textsc{loc} stay \\
\z
} \\

\xbox{16}{
\ea\label{ex:pragm:address:kin:2:aanak}
\ea
\gll  Andare aanak=nang su-biilang:\\ % bf
      Andare child=\textsc{dat} \textsc{past}-say \\
\ex
\gll \textbf{Aanak} \\
    ` ``Son, '' '
\ex
\gll 'lu=ppe        umma su-maathi'     katha  bithàràk=apa   asà-naangis   mari. \\ % bf
      \textsc{2s}=\textsc{poss} mother \textsc{past}-die \textsc{quot} scream=after \textsc{cp}-weep come.imp \\
    ` ``come and cry and weep `My mother has died!' '' '
\z
\z
} \\

The use of kin terms when refering to third person is shown in  \xref{ex:pragm:address:kin:3}, a stretch of discourse about the father of the speaker, who, despite of being introduced early on, is mostly refered to as \em baapa\em. The zeroes indicate instances where the referent is not overtly realized. Also note the reference to \trs{aade}{younger.sibling} in a similar sense.

\xbox{16}{
\ea\label{ex:pragm:address:kin:3}
\ea
\gll itthu    blaakang \textbf{baapa}  sajja anà-duuduk. \\
     \textsc{dist} after father only \textsc{past}-saty  \\
    `The father stayed at home alone.'  
\ex
\gll baapa  asà-duuduk suda giithu samma oorang=pe samma aanak pada=pe  ruuma  \textbf{baapa}  arà-pii. \\
     father \textsc{cp}-stay thus like.that all man=\textsc{poss} all child \textsc{pl}=\textsc{poss} house father \textsc{non.past}-go  \\
    `Father stayed there and then like that father went to all folks', all children's homes.'
\ex
\gll giithu   asaduuduk     kithang {\em transfer} su-dhaathang     {\em Kandy}=nang. \\ % bf
     like.that cp.stay/from \textsc{1pl} transfer \textsc{past}-come Kandy=\textsc{dat}  \\
    `From there, we were transfered to Kandy.' 
\ex
\gll Kandy=nang   dhaathang=jo    \textbf{baapa}=le       su-dhaathang     {\em Kandy}=nang. \\
     Kandy=\textsc{dat} come=\textsc{foc} father=\textsc{addit} \textsc{past}-come Kandy=\textsc{dat}  \\
\ex
\gll  \zero{} kithang=pe     ruuma=ka=le   anà-duuduk, \textbf{aade}=pe  ruuma=ka=le    anà-duuduk. \\
      { } \textsc{1pl}=\textsc{poss} house=\textsc{loc}=\textsc{addit} \textsc{past}-stay younger.sibling=\textsc{poss} house=\textsc{loc}=\textsc{addit} \textsc{past}-stay \\
\ex
\gll \zero{} itthu samma         ruuma=ka    asà-duuduk  \\ % bf
     { } \textsc{dist} all house=\textsc{loc} \textsc{cp}-stay  \\
\ex
\gll {\em last}=ka  \textbf{baapa}  Badulla=nang   anà-pii \\
     last=\textsc{loc} father Badulla=\textsc{dat} \textsc{past}-go  \\
\z
\z
} \\
Next to kin terms, proper names can also be used as terms of address \xref{ex:pragm:address:propername1}\xref{ex:pragm:address:propername2}.



\xbox{16}{
\ea\label{ex:pragm:address:propername1}
\gll Andare=nang asà-panggel anà-biilang `se=dang \textbf{Andare}=pe biini=yang mà-caanda suuka.' \\
      Andare=\textsc{dat} \textsc{cp}-call \textsc{past}-say \textsc{1s=dat} Andare=\textsc{poss} wife=\textsc{acc} \textsc{inf}-meet want\\
\z
} \\

\xbox{16}{
\ea\label{ex:pragm:address:propername2}
\gll \textbf{Aashik}=nang hathu {\em soldier} mà-jaadi suuka=si katha arà-caanya. \\
     Aashik=\textsc{dat} \textsc{indef} soldier \textsc{inf}-become like=\textsc{interr} \textsc{quot} \textsc{non.past}-ask  \\
\z
} \\
 
% \xbox{16}{
% \ea\label{ex:form:unreferenced}
% \gll  Sebastian puddas arà-maakang=si. \\
%       Sebastian spicy \textsc{non.past}-eat=\textsc{interr} \\
% \z
% } \\

Elder relatives may also be adressed \xref{ex:pragm:address:uncle1} - \xref{ex:pragm:address:uncle3} as or refered  to with the English terms \em uncle \em and \em auntie\em. This does not imply that the person is actually a sibling of a parent.
 

\xbox{16}{
\ea\label{ex:pragm:address:uncle1}
\gll \textbf{{\em uncle}}, kitham=pe  umma   anà-biilang  giini. \\
     uncle \textsc{1pl}=\textsc{poss} mother \textsc{past}-say like.this  \\
    `Uncle, our mother said the following: ... .' (K051220nar01)
\z
} \\

\xbox{16}{
\ea\label{ex:pragm:address:uncle2}
\gll  diya-la,  \textbf{{\em uncle}} se=ppe  SSC {\em exam} mà-ambel=nang duwa-pul aari su-aada. \\
     see-\textsc{imp} uncle \textsc{1s=poss} SSC exam \textsc{inf}-take=\textsc{dat} two-ty day \textsc{past}-exist \\
    `See, uncle, there are (were) only twenty days left until my SSC exam.' (K051220nar01)
\z
} \\
\xbox{16}{
\ea\label{ex:pragm:address:uncle3}
\ea
\gll  Thuan Kuddhuus, Thuan Thungku, Thuan Skiilan, Thuan Idriis,  \\
      Thuan Kuddhuus Thuan Thungku Thuan Skiilan Thuan Idriis \\
    `Thuan Kuddhuus, Thuan Thungku, Thuan Skiilan, Thuan Idriis,'  
\ex
\gll Thuan -- \textbf{{\em uncle}}, hathyang naama pada saapa ? \\
     Thuan { } uncle other name \textsc{pl} who  \\
    `Thuan ... --- what's the names of the others again, uncle?' (K060108nar02)
\z
\z
} \\ 
\subsection{Distantiating}\label{sec:pragm:Distantiating}
Distantiating oneself from the provided content can be done by \trs{se=dang kalu}{as for me}, \trs{se=dang thaau mosthornang}{as far as I know} or \trs{boolebiilang}{you could say that ...}


\xbox{16}{
\ea\label{ex:pragm:distantiating:sedangkalu}
\gll kaake=nang  dhraapa=so   bannyak aanak pada,  \textbf{se=dang} \textbf{kalu} blaangang thàràthaau. \\
    `Grandfather had several children, many children, as for me, I do not know the number.' (K051205nar05)
\z
} \\
\xbox{16}{
\ea\label{ex:pragm:distantiating:sedangthaaumosthor}
\gll \textbf{se=dang} \textbf{thaau} \textbf{mosthor=nang} inni=jo        inni     kithang=pe inni     Seelong=nang {\em political} {\em news}. \\
    `As far as I know, this is the political news for Sri Lanka.' (N061031nar01)
\z
} \\
\xbox{16}{
\ea\label{ex:pragm:distantiating:bolebiilang}
\gll  \textbf{bole-biilang}    se=ppe    kaake=le            asadhaathang hathu  {\em army} {\em officer} Badulla=ka. \\
      can-say \textsc{1s=poss} grandfather=\textsc{addit} \textsc{copula} \textsc{indef} army officer Badulla=\textsc{loc} \\
\z
} \\
Direct attribution of the source is also possible, as in \xref{ex:pragm:distantiating:spaaruoorang}, where the source of the information \trs{spaaru oorang pada}{some people} is explicitely mentioned.

\xbox{14}{
\ea\label{ex:pragm:distantiating:spaaruoorang}
\gll spaaru oorang pada arà-biilang Seelong=nang  {\em English} anà-aaji.baa katha \\
     some man \textsc{pl} \textsc{non.past}-say Seelon=\textsc{dat} English \textsc{past}-bring.anim \textsc{quot}  \\
 English who brought them to sri lanka.'  
\z
}

\subsection{Reinforcing}\label{sec:pragm:Reinforcing}
\citet{Slomanson2008ismil} reports the use of the string \em apa kata k$@$m-bilang \em (his orthography) for reinforcing purposes. The following is an example he gives.



\xbox{16}{
\ea
\gll Apa  kata   k$@$m-bilang, April ka  jo      e-datang (aDa) \\
     what word when-say    April  in FOC ASP-come  AUX  \\
    `What I say to you is that it has come in April.' \citep[7(17a)]{Slomanson2008ismil}
\z
} \\ 
% \section{Speech genres}\label{sec:pragm:Speechgenres}
% \section{Conversation}\label{sec:pragm:Conversation}
% How is conversation organized?
% 
% \subsection{Turn taking}\label{sec:pragm:Turntaking}
% How is turn taking organized?
% bukang indicates that another turn is to follow



% \xbox{16}{
% \ea
% \gll kithang cumma araduuduk \\
%        \\
%     `We are just waiting (bored).' (nosource)
% \z
% } \\
