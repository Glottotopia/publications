

\chapter{Clauses}\label{sec:clauses}
We have discussed how morphemes can be combined into words and how words can be combined into NPs, PPs and predicates. We will now address the next higher unit and discuss how several phrases can be combined to form a clause

A clause must obligatorily have a predicate (PRED), and it can have one or more nominal phrases (NPs) which encode the arguments of the predicate. These NPs often carry postpositions (POSTP). Additional material can also be put into the clause  in the form of adjuncts. All clauses are normally predicate final (with some exceptions for relative clauses), so that the normal structure of the clause is as follows.



\cbx{
\NP* PRED
}{CLS}


\section{Main clauses}\label{sec:cls:Mainclauses}
The main clause normally consists of a predicate in final positions, and a number of adjuncts and arguments to the left. The order of those arguments and adjuncts is free. As argued in \formref{sec:gramrel}, SLM grammar has no privileged argument, like `subject' or `object'. Rather, all the arguments have the same prominence. This can be represented in the following Tree:

\ea \label{ex:clause:tree:intro}
\Tree   [.S  NP NP NP NP ... ] 
\z

This tree actually resembles Hale's analysis of Japanese clause structure\footnote{This analogy between SLM and Japanese is also found in the domain of relative clauses \formref{sec:cls:Relativeclause}.} \citep{Hale1982japconf}, in which he denies the existence of a VP in Japanese and argues that Japanese is non-configurational. Also Mohanan's (1982) analysis of Malayalam, a sister language of Tamil, \nocite{Mohanan1982} comes to the same conclusion. \citet[177]{Lehmann1989} finally states for Tamil:
\begin{quote}
We assume that the verb and all its argument NPs, as well as adverbial adjuncts, are immediate consitutents of the sentence. This means that in Tamil there is no syntactic bond \el{} which affects the verb and the object NPs but not the subject NP. That is to say, there is no verb phrase (VP) constituent in Tamil. Subjects, objects, and the verb are all immediate constituents of S.
\end{quote}

Just like Japanese, Malayalam, or Tamil,  SLM seems to lack  rigid constituency within the clause, with the exception of the position of the verb, which must be the last or second-to-last element (see below) in the clause.

In SLM, it is very common for arguments to be dropped if they are inferable from context. It appears that Sri Lankan Malay is similar to Late Archaic Chinese in this respect, for which  \citet{Li1997zero} said: ``What demands explanation is not zero-anaphora \el{} but the appearance of of a referential expression, whether a pronoun or a full fledged noun phrase.'' While this chapter will discuss many clauses with overt arguments, it should be born in mind that the normal case for Sri Lankan Malay is to have most of the arguments dropped. This is discussed in more detail in Chapter \ref{sec:Informationflow}.



As for clauses, we can distinguish a number of subtypes, discussed in the   sections given:

\begin{itemize}
 \item Declarative clause \formref{sec:cls:Declarativeclause}
\item Copular clause \formref{sec:cls:Copularclause}
\item Two types of interrogative clauses \formref{sec:cls:Interrogativeclauseclitic}, \formref{sec:cls:InterrogativeclauseWH}
\item Imperative clause \formref{sec:cls:Imperativeclause}.
\end{itemize} 

\subsection{Declarative clause}\label{sec:cls:Declarativeclause}
The declarative clause is the most simple type. It consists of between zero and three arguments followed by a predicative phrase. See the section on predicate types for examples \formref{sec:pred}.


\cbx{
\NP* PRED
}{CLS}

The order of the arguments is free, but the leftmost argument is normally the topic.\footnote{See \citet[62]{Bayer2004} and references therein for a more extensive list of prefered orders in languages with free word order.} An example of this is given in \xref{ex:clause:decl:topic}. The case marker of the topic is occasionally dropped as in   \xref{ex:clause:decl:ppdrop1}\xref{ex:clause:decl:ppdrop2}
 
\xbox{16}{
\ea\label{ex:clause:decl:topic}
\gll  [[ini kaaving=nang aada] haarath-saarath pada]$_{top}$ kitham=pe bannyak oram pada arà-kijja. \\
     \textsc{prox} wedding=\textsc{dat} exist   traditions \textsc{pl} 1p=\textsc{poss} many people \textsc{pl} \textsc{non.past}-make\\
\z      
}\\     
 

\xbox{16}{
\ea\label{ex:clause:decl:ppdrop1}
\gll [luwar  nigiri]$_{top}$=\zero{}    kithang=nang   mà-pii    thàrà-suuka. \\
 outside country \textsc{1pl}=\textsc{dat} \textsc{inf}-go \textsc{neg}-like\\
\z
}


\xbox{16}{
\ea\label{ex:clause:decl:ppdrop2}
\gll [samma thumpath]$_{top}$=\zero{}  mlaayu aada. \\
     every place Malay exist  \\
    `In every place, there are Malays.'  (K051222nar04)
\z      
}\\ 
 

%\xbox{16}{
%\ea\label{ex:constr:unreferenced}
%\gll se=dang se=ppe biini arà-iingath. \\
%      \textsc{1s=dat} \textsc{1s}=\textsc{poss} wife pro-think \\
%\z      
%}\\ 



Sometimes, material is found to the right of the predicate \citep[cf.][137ff.]{Slomanson2007cll}\citet[14]{Ansaldo2005ms}. This is frequently done for arguments encoding location and goal.

\cbx{NP(=POSTP)_{TOPIC} \NP* PRED PP$_{loc}$}{CLS}

%\xbox{16}{
%\ea\label{ex:constr:unreferenced}
%\gll itthu    baathu=yang    incayang Seelong=dering             \textbf{laayeng} \textbf{nigiri=nang} asà-baapi. \\
%`He broght those stones from Ceylon to other countries.' (nosource)
%\z
%}

Example \xref{ex:clause:decl:right:local1} shows a local argument (GOAL) following the predicate \trs{nyabaawa}{brought}.
% , while \xref{ex:clause:decl:right:local2} shows two local arguments   following the predicate.

\xbox{16}{
\ea\label{ex:clause:decl:right:local1}
\gll pìrrang {\em Time}=ka {\em British} {\em Government} nya-baawa \textbf{inni} \textbf{nigiri=nang}. \\
`During the war, the British Government brought (them) to this country.' (nosource)
\z
}

% \xbox{16}{
% \ea\label{ex:clause:decl:right:local2}
% \gll kithan       nya-pii \textbf{Anuradhapura=dang}, \textbf{Katunaayaka}     \textbf{sduuduk}. \\
%  \textsc{1pl} \textsc{past}-go Anuradhapura=\textsc{dat} Katunayaka from\\
% \z there is a tendency to have topical arguments at the leftmost position. The frequent initial position of pronouns and human arguments is a consequence of this. It is very common for arguments to be dropped if they are inferable from context. It appears tha
% }

When right dislocation of non-spatial arguments occurs, these are often in focus.

\cb{ 	NP(=POSTP)_{TOPIC} \NP* PRED \NP$_{foc}$}
In the following three examples, the argument following the predicate is in contrastive focus to other referents mentioned before in discourse (other ethnic groups, other food, other languages).

\xbox{16}{
\ea\label{ex:clause:decl:right:foc1}
\gll nni Peradeniya jaalang=ka samma n-aada \textbf{mlaayu}. \\
 \textsc{prox} Peradeniya road=\textsc{loc} all \textsc{past}-exist Malay\\
`Everybody in this Peradeniya Rd is MALAY.' (K051222nar04)
\z
}

\xbox{16}{
\ea\label{ex:clause:decl:right:foc2}
\gll Hindu arà-maakang \textbf{kambing}. \\
 Hindu \textsc{non.past}-eat goat\\
`Hindus eat GOAT.' (K060112nar01)
\z
}

\xbox{16}{
\ea\label{ex:clause:decl:right:foc3}
\gll itthukapang kitham arà-blaajar \textbf{mulbar}. \\
then \textsc{1pl} \textsc{non.past}-learn Tamil \\
\z
}


Also, very heavy NPs can be dislocated to the right

\xbox{16}{
\ea\label{ex:clause:decl:right:heavy}
\gll kitha=nang$_{NP}$ maau$_{PRED}$ [\textbf{kitham=pe} \textbf{mlaayu} \textbf{lorang} \textbf{blaajar} \textbf{lorang=pe} \textbf{mlaayu} \textbf{kitham} \textbf{blaajar}]$_{NP}$. \\
`We want that you learn our Malay and that we learn your Malay.' (K060116nar02.100)
\z
}

%soocera padanang annajuuwal



%\xbox{16}{
%\ea\label{ex:constr:unreferenced}
%\gll kuttumu aada nni      CD-ya. \\
% see exist \textsc{prox} CD-yang\\
%`.' (nosource)
%\z
%}

But also arguments fulfilling none of the conditions mentioned above can be found after the verb, as shown in the following four examples.

\xbox{16}{
\ea\label{ex:clause:decl:right:ordinary1}
\gll suda hathu  {\em week}=nang   duuwa skali  arà-dhaathang    \textbf{{\em daughter}}. \\
     so one week=\textsc{dat} two time \textsc{non.past}-come daughter  \\
    `Thus my daughter comes twice a week.' (K051201nar01)
\z
} \\
 \xbox{16}{
\ea\label{ex:clause:decl:right:ordinary2}
   \gll  derang pada arà-mintha    \textbf{nigiri}. \\
    3     \textsc{pl} \textsc{non.past}-ask country\\
\z
}

\xbox{16}{
\ea\label{ex:clause:decl:right:ordinary3}
\gll  itthu=nang      blaakang su-dhaathang     \textbf{Hambanthota} \textbf{mlaayu} \textbf{pada}. \\
    `After that came the Hambantota Malays.' (K051206nar14)
\z
} \\
\xbox{16}{
\ea\label{ex:clause:decl:right:ordinary4}
\gll  itthu muusing  Islam igaama  nya-aajar kaasi \textbf{Jaapna}  \textbf{Hindu} \textbf{{\em teacher}}. \\
    `At that time, those who taught Islamic religion were Hindu teachers from Jaffna.' (K051213nar03)
\z
} \\
It could be possible to analyze the last three examples as sentences similar to an English pseudo-cleft. The first part would be a headless relative clause (`What they are asking for',  `Who came then',  `Who taught Islam'), and the second part the instantiation thereof (`a country', `Hambantota Malaya', `Jaffna teachers'.). This structure is common in Tamil and could be at the origin of a similar structure in Sinhala as well \citep{Gair1985calque}.

If we summarize our findings about argument positions we can say that an arbitrary number of arguments can occur before the verb, but only one can occur after the verb.


\cbx{
\NP* PRED (\NP)
}{CLS}

This resembles very much the structure of the NP as given in \formref{sec:nppp:Thefinalstructureofthenounphrase}. The amount and order of elements preceding the head is quite free, while following the head, there is only one position, which has some restrictions to it. As for the noun phrase, these are more absolute (no deictic, possessive or relative clause can ever be used after the noun), while for the clause, they are more lax, more like tendencies (non-spatial arguments tend not to occur after the verb). Analogous to \xref{ex:np-pp:tree:final} in \formref{sec:nppp:TheSLMNPasappositional}, we can represent the SLM declarative clause as in \xref{ex:clause:tree:final}

\ea \label{ex:clause:tree:final}
\Tree   [.S
	 [.pre  NP NP {...}	 ]
	   [.V ]
	   [.post NP ] 
	]
\z




\subsection{Copular clause}\label{sec:cls:Copularclause}
Some predicates can be supported by the copula \em (asa)dhaathang(apa) \em \formref{sec:wc:Copula}. This is most often done for predicates of naming \xref{ex:cl:copula:name} or of class-membership, either in an ethnic group \xref{ex:cl:copula:class:ethn}, a profession \xref{ex:cl:copula:class:profession} or kin \xref{ex:cl:copula:class:kin}.  English material in the sentence seems to favour the use of the copula.

\cb{NP COPULA name}
\cb{NP COPULA class}


\xbox{16}{
\ea\label{ex:cl:copula:name}
\gll se=ppe    baapa  dhaathangapa \textbf{Jinaan} \textbf{Samath}. \\
    `My father was Jinaan Samath.' (N060113nar03)
\z
} \\

\xbox{16}{
\ea\label{ex:cl:copula:class:ethn}
\ea
\gll se=ppe    {\em daughter-in-{\em law}}=pe {\em mother} asadhaathang \textbf{bingaali}. \\
      \textsc{1s}=\textsc{poss} daughter-in-law=\textsc{poss} mother \textsc{copula} Bengali \\
    `My daughter-in-law's mother is Bengali.'   
\ex
\gll ithukapang       {\em daughter-in-{\em law}}=pe     {\em father} asadhaathang \textbf{mlaayu}. \\
      then daughter-in-law=\textsc{poss} father \textsc{copula} Malay \\
    `Then my daughter-in-law's father is Malay.' (K051206nar08)
\z
\z
} \\

\xbox{16}{
\ea\label{ex:cl:copula:class:profession}
\gll umma=pe       baapa  dhaathangapa  hathu  \textbf{{\em inspector}}          \textbf{\em of}  \textbf{{\em police}}. \\
     mother=\textsc{poss} father \textsc{copula} \textsc{indef} inspector of police  \\
\z
} \\


\xbox{16}{
\ea\label{ex:cl:copula:class:kin}
\gll baapa=pe      umma   asadhaathang  \textbf{kaake=pe}           \textbf{aade}. \\
    father=\textsc{poss} mother \textsc{copula} grandfather=\textsc{poss} younger.sibling  \\
    `My paternal grandmother was my grandfather's younger sister.' (K051205nar05)
\z
} \\
Note that the use of the copula is optional in all these cases. The following examples give sentences contrasting with the examples above in the absence of the copula.


\xbox{16}{
\ea\label{ex:cl:copula:name:contr}
\gll sudaara sudaari se=ppe naama \textbf{Wahida} \textbf{Jamaldiin}. \\
    `Brothers and sisters, my name is Wahida Jamaldeen.' (B060115prs05)
\z
} \\

\xbox{16}{
\ea\label{ex:cl:copula:class:ethn:contr}
\gll Sindbad  {\em the}  {\em Sailor}     \textbf{hatthu} \textbf{Muslim}. \\
`Sindbad the Sailor was a Moor, he was not a Malay.' (K060103nar01)
\z
}


\xbox{16}{
\ea\label{ex:cl:copula:class:prof:contr}
\gll  \zero{}  karang \textbf{{\em Dialog}} \textbf{\em GSM}=ka   \textbf{{\em junior}} \textbf{{\em executive}} \textbf{hatthu}. \\
    `She is now junior executive at Dialog GSM [phone company].'
\z
} \\
\xbox{16}{
\ea\label{ex:cl:copula:class:kin:contr}
\gll  se=ppe    neene       itthu  \textbf{kaake=pe}           \textbf{aade}. \\
     \textsc{1s}=\textsc{poss} grandmother \textsc{dist} grandfather=\textsc{poss} younger.sibling  \\
    `My grandmother is that grandfather's younger sibling.' (K051205nar05)
\z
} \\
There are occasionally other semantic classes introduced by the copula, please refer to \formref{sec:wc:Copula} for a discussion.

The copular predication is often used for equational predicates. This is especially true for the naming use, where we assert that referent X is the same as referent Y. We do not assert of any predicate that it is true of my father; rather, we assert that two referents are identical.

Because equation is a symmetric predicate (\em My father is John \em and \em John is my father \em have the same truth values), the position of the equated referents is occasionally swapped, as in \xref{ex:cl:copula:swap1} and \xref{ex:cl:copula:swap2}.
 

\xbox{16}{
\ea\label{ex:cl:copula:swap1}
\gll [se=ppe    kaake]       asadhaathang [{\em estate} {\em tea} {\em factory} {\em officer}]. \\ % bf
      \textsc{1s}=\textsc{poss} grandfather \textsc{copula} estate tea factory officer \\
    `My grandfather was estate tea factory officer.' (K060108nar02)
\z
} \\
\xbox{16}{
\ea\label{ex:cl:copula:swap2}
\gll [{\em estate}=pe  {\em field} {\em officer}] asadhaathangapa  [kithang=pe     kaake]. \\ % bf
     estate=\textsc{poss} field officer \textsc{copula} \textsc{1pl}=\textsc{poss} grandfather  \\
\z
} \\
In very rare occasions, the order of the copula and the two NPs can be mixed up, as in \xref{ex:cl:copula:chaos}.

\xbox{16}{
\ea\label{ex:cl:copula:chaos}
\gll asadhaathangapa incayang  {\em army} {\em captain}. \\ % bf
     \textsc{copula} \textsc{3s.polite} army captain  \\
\z
} \\

\subsection{Interrogative clause, clitic}\label{sec:cls:Interrogativeclauseclitic}
A third clause type is formed by adding the interrogative clitic \em =si \em to the element of a declarative clause one wishes to question,  normally the predicate. This is shown for verbal predicates in \xref{ex:cl:interr:cl:v} and for a nominal predicate in \xref{ex:cl:interr:cl:n}. Questioning a constituent is shown in \xref{ex:cl:interr:const}. Right dislocation does not seem to be possible in interrogative clauses.

\cbx{CLS=si}{CLS}
\cbx{\NP NP=si PRED}{CLS}


\xbox{16}{
\ea\label{ex:cl:interr:cl:v}
\gll se=pe uumur massa-biilan=\textbf{si}? \\
 1=\textsc{poss} age must-tell=\textsc{interr}\\
`(Do I) have to tell my age?' (B060115prs01.4)
\z
}

\xbox{16}{
\ea\label{ex:cl:interr:cl:n}
\gll  Lorang=nang see=yang ingath-an=\textbf{si}. \\
     \textsc{2pl}=\textsc{dat} \textsc{1s}=\textsc{acc} think-\textsc{nmlzr}=\textsc{interr} \\
\z      
}\\ 

\xbox{16}{
\ea\label{ex:cl:interr:const}
\gll \textbf{daging baabi=si} anà-bìlli. \\
     pork=\textsc{interr} \textsc{past}-buy  \\
\z      
}\\  
 

This is also the Sinhala and Tamil way of forming interrogative clauses.


\subsection{Interrogative clause, WH}\label{sec:cls:InterrogativeclauseWH}
A fourth clause type is the interrogative clause involving an interrogative pronoun.
The interrogative pronoun substitutes the queried element. It is normally found \em in situ\em, but since the word order to the left of the verb is quite free anyway, this  also means that \em in situ \em position cannot be distinguished from initial position. Technically speaking, this is just a special case of the declarative clause, where one or several NPs are instantiated by interrogative pronouns, instead of nouns, prounouns, etc.


\cbx{WH(=POSTP) \NP PRED}{CLS}

The following three examples show the use of a WH-pronoun in initial position.

\xbox{16}{
\ea\label{ex:cl:interr:wh:initial1}
\gll \textbf{mana} nigiri=ka arà-duuduk? \\
 which country=\textsc{loc} \textsc{non.past}-stay\\
\z
}


\xbox{16}{
\ea\label{ex:cl:interr:wh:initial2}
\gll \textbf{saapa}  m-maati? \\
 who \textsc{past}-die\\
`Who died?' (K051213nar07.14)
\z
}

\xbox{16}{
\ea\label{ex:cl:interr:wh:initial3}
\gll  \textbf{aapa}   arà-biilang    itthu? \\
      what \textsc{non.past}-say \textsc{dist} \\
\z      
}\\ 

Non-initial position of the WH-pronoun is found in the following two examples.

\xbox{16}{
\ea\label{ex:cl:interr:wh:situ1}
\gll lorang naama kapang-biilang, baapa \textbf{saapa} umma \textbf{saapa} katha. \\
      \textsc{2pl} name when-say father who mother who quot\\
    `When you tell your name, (also tell) who (are) (your) parents.'  (B060115nar04)
\z      
}\\ 

% \xbox{16}{
% \ea\label{ex:cl:interr:wh:situ2}
% \gll itthu=nang      blaakang \textbf{aapa} nya-gijja. \\
%  \textsc{dist}=\textsc{dat} after   what \textsc{past}-do\\
% `Then, what did we do?' (K051206nar07)
% \z
% }


\xbox{16}{
\ea\label{ex:cl:interr:wh:situ3}
\gll incayang=pe naama \textbf{aapa}, {\em sir}=pe naama? \\
      \textsc{3s.polite}=\textsc{poss} name what sir=\textsc{poss} name \\
    `What's his name, the teacher's name?' (K060103cvs01)
\z
} \\


The interrogative pronoun can be doubled. This indicates that an exhaustive list is expected as the answer. A single item will not do. Examples \xref{ex:cls:interr:aapaanabilli} and \xref{ex:cls:interr:apaaapa} illustrate this.

\xbox{16}{
\ea \label{ex:cls:interr:aapaanabilli}
\gll  aapa anà-bìlli. \\
      what \textsc{past}-buy\\
    `What  did you buy.'  (test)3.11.08
\z
}\\


\xbox{16}{
\ea \label{ex:cls:interr:apaaapa}
\gll aapa\~{}aapa anà-bìlli. \\
      what\~{}red \textsc{past}-buy\\
    `What all did you buy.'  (test)3.11.08
\z
}\\



%  \xbox{16}{
% \ea \label{ex:interr:apaaapa:naturalistic}
% \gll \textbf{aapa} \textbf{aapa} kitham Kandi=pe {\em cultural} {\em show} atthu=le thaaro? \\
% `What did we also put on a Kandy Cultural show?' (K060116nar11)
% \z
% }
% 
% 

\subsection{Imperative clause}\label{sec:cls:Imperativeclause}
The fifth main clause type is the imperative clause. It consists of between 0 and 2 arguments, none of them agent, and a verb at the right edge, which optionally carry the particle \em mari \em to its left or the imperative suffixes \em -la \em pr \em -de \em to its right, or both. The verb cannot carry any further TAM marking.

\cb{ \NP* (\textit{mari}) V $\left(\begin{array}{r}-la\\-de\end{array} \right)$}
Examples \xref{ex:cl:imp:nothing1} and \xref{ex:cl:imp:nothing2} show the use of a bare verb in the imperative clause,  \xref{ex:cl:imp:mari} shows the use of \em mari\em, \xref{ex:cl:imp:la} shows the use of \em -la\em, while \xref{ex:cl:imp:marila} has both \em mari \em and \em -la\em.

\xbox{16}{
\ea\label{ex:cl:imp:nothing1}
\gll Aajuth thaakuth=ka su-naangis, ``See=yang \zero-luppas-\zero''. \\ % bf
     dwarf fear=\textsc{loc} \textsc{past}-cry \textsc{1s}=\textsc{acc} leave  \\
\z      
}\\

\xbox{16}{
\ea\label{ex:cl:imp:nothing2}
   \gll  Binthan {\em auntie}=ka    \zero-caanya-\zero{}, binthan {\em auntie}=yang   konnyong \zero-panggel-\zero{}. \\ % bf
    Binthan auntie=\textsc{loc} ask, Binthan auntie=\textsc{acc} few call \\
`Ask auntie Binthan, call Binthan auntie' (K060116nar06)
\z
}
 
\xbox{16}{
\ea\label{ex:cl:imp:mari}
\gll \textbf{mari} maakang. \\
      come eat \\
    `Eat!'  (B060115rcp02)
\z      
}\\ 

\xbox{16}{
\ea\label{ex:cl:imp:la}
\gll allah \textbf{diyath-la} inni pompang pada dhaathang aada. \\
      Allah watch-\textsc{imp} \textsc{prox} female \textsc{pl} come exist \\
\z
} \\

\xbox{16}{
\ea\label{ex:cl:imp:marila}
\ea
\gll  saayang se=ppe thuan \textbf{mari} laari-\textbf{la}. \\
      love \textsc{1s}=\textsc{poss} sir come.imp run-\textsc{imp} \\
\ex
\gll see=samma kumpul \textbf{mari} thaandak-\textbf{la}. \\
     \textsc{1s}=\textsc{comit} gather come.imp dance-\textsc{imp}  \\
    `Come and dance with me.' (N061124sng01)
\z
\z
} \\
\xbox{16}{
\ea
\gll pii! ... pii!! ... pii-de!!! \\
     go go go-imp.impolite  \\
    `Go! Go now! Bugger off!!!' (nosource)
\z
} \\


The imperative prefixes \trs{marà-}{adhort} and \trs{jamà-}{neg.imp} \citep[cf.][]{Slomanson2008lingua} are used in a different construction.
\cb{NP* $\begin{array}{r}mar\grave{a}-\\jam\grave{a}- \end{array}$}

\xbox{16}{
\ea\label{ex:cl:imp:neg:jama}
\gll hatthu=le \textbf{jamà}-gijja baapa ruuma=ka duuduk. \\ % bf
      \textsc{indef}=\textsc{addit} \textsc{neg.imp}-do father house=\textsc{loc} stay \\
\z      
}\\ 
\xbox{16}{
\ea\label{ex:cl:imp:neg:thussa}
\gll Thussa mà-thaakuth, Buruan su-biilang. \\ % bf
      \textsc{neg.imp} \textsc{inf}-fear bear \textsc{past}-say\\
\z      
}\\ 


The adhortative combined with the interrogative particle \em =si\em forms an adhortative like English \em shall we?\em.

\cb{ \NP* marà-V=\textit{si}}

\xbox{16}{
\ea\label{ex:cl:imp:adhort}
\gll marà-pii=si. \\
     adhort-go=\textsc{interr}  \\
    `Shall we go?'  (test)5.11.08
\z      
}\\ 


% \xbox{16}{
% \ea
% \gll *maripiisi/ *nyaanyi la si. \\
%        \\
%     `.' (nosource)
% \z
% } \\

\section{Relative clause}\label{sec:cls:Relativeclause}
 The relative clause is characterized by a different word order as compared to the main clause. 

\cb[\label{cb:form:relc:intro}]{
... 
$
	\left[
		\NP 
		PRED
	\right]_{RELC}
$ 
NP(=POSTP)$_{main}$ ... PRED$_{main}$
}

Relative clauses are only indicated by position.\footnote{\citet{SmithEtAl2004} note that relative clauses ``are headed by a verbal adjective'', yet none of the examples they give has a form glossed as `verbal adjective'; it appears that all verbs in the examples they give are simply finite verbs.} Any declarative main clause can be turned into a relative clause by putting it before a nominal. That nominal can have any semantic role.\footnote{This is thus very different from other Western Austronesian languages, where only subjects can be relativized \citep[161]{Himmelmann2005typochar}.} This simple principle covers all there is to say about relative clauses, still the different possibilities will be discussed in detail.

The Relative Clause Construction is used for relativization properly speaking (\em The money that John left surprised me\em), but also for fact clauses (\em The fact that John left surprised me\em). These two constructions are semantically different in that in the former, we are dealing with an individual (\em money\em) modified by a proposition, whereas in the latter, we are dealing with a state-of-affairs (\em fact\em), whose content is given by the fact clause. While these two types are semantically different, this semantic difference is not mirrored in SLM syntax: both use the construction given in \xref{cb:form:relc:intro}.\footnote{See \citet{Matsumoto1997} for a comparable analysis of Japanese relative clauses and fact clauses.}

We will first discuss the occurence of different TAM-markers in relative clauses, to show that they are fully finite\footnote{Unlike Sinhala and Tamil, which have non-finite relative clauses.}\formref{sec:cls:TAMintherelativeclause}. We will then turn to different predicate types in relative clauses \formref{sec:cls:Predicatetypesintherelativeclause} and finally discuss the different semantic roles on which one can relativize \formref{sec:cls:Semanticrolesintherelativeclause}.

\subsection{TAM in the relative clause}\label{sec:cls:TAMintherelativeclause}
There are no restrictions on the TAM which can appear in the relative clause. The following sections give examples of relative clauses in the different tenses.

\subsubsection{Past \em anà-\em}\label{sec:cls:pastana}

\xbox{16}{
\ea\label{ex:cl:relc:ana1}
\gll [incayang=pe kepaala=ka \textbf{ana}-aada] thoppi=dering moonyeth pada=nang su-buwang puukul. \\
      \textsc{3s.polite}=\textsc{poss} head=\textsc{loc} \textsc{past}-exist hat=\textsc{abl} monkey \textsc{pl}=\textsc{dat} \textsc{past}-throw hit \\
\z      
}\\


% \xbox{16}{
% \ea\label{ex:cl:relc:ana3}
% \gll bannyak thuuwa oorang nya-blaajar oorang. \\
%      much old man \textsc{past}-learn man  \\
%     `An old man, an educated man.' (K060116nar07)
% \z
% } \\

\subsubsection{Past \em su-\em}\label{sec:cls:su}


\xbox{16}{
\ea\label{ex:cl:relc:su}
\gll [Ruuma duuwa subala=ka   \textbf{su}-aada      rooja pohong  komplok duuwa]=yang   asà-baa=apa   mliige=pe     duuwa subla=ka su-thaanam. \\
      house two side=\textsc{loc} \textsc{past}-exist rose tree bush two=\textsc{acc} \textsc{cp}-bring after palace=\textsc{poss} two side=\textsc{loc} \textsc{past}-plant \\
\z
} \\
% 
% \xbox{16}{
% \ea\label{ex:cl:relc:unreferenced}
% \gll  sa-{\em mix} {\em salad}. \\
%       \textsc{past}-mix salad \\
%     `A mixed salad.' (B060115nar03)
% \z
% } \\


 

% \xbox{16}{
% \ea\label{ex:cl:relc:unreferenced}
% \gll [Aanak raaja=pe perkathahan=yang Snow-white=nang=le Rose-red=nang=le sukahan=dering \textbf{su}-punnu hathu hidopan] su-thunjiking. \\
%       child king=\textsc{poss} word=\textsc{acc} Snow-white=\textsc{dat}=\textsc{addit} Rose-red=\textsc{dat}=\textsc{addit} desire=\textsc{abl} \textsc{past}-full \textsc{indef} dwelling \textsc{past}-show \\
% \z
% } \\




\subsubsection{Perfect with \em aada\em}\label{sec:cls:perfectwithaada}
The perfect tense can be found in the relative clause.

\xbox{16}{
\ea\label{ex:cl:relc:perf:aada1}
\gll [Seelon=nang dhaathang \textbf{aada}] melayu oorang ikkang. \\
 Ceylon-\textsc{dat} come exist Malay man fish\\
`The Malays who came to Sri Lanka were fishermen.' (K060108nar02.23)
\z
}



 \xbox{16}{
\ea\label{ex:cl:relc:perf:aada2}
   \gll  itthu    asadhaathang [baaye=nang waasil-king \textbf{aada}]  {\em dagger} hatthu. \\
     \textsc{dist} \textsc{copula} good=\textsc{dat} blessed-\textsc{caus} exist dagger \textsc{indef} \\
\z
}


\subsubsection{Non-past \em arà-\em}\label{sec:cls:ara}


\xbox{16}{
\ea\label{ex:cl:relc:ara:simult1}
\gll Suda [puthri=le biini=le \textbf{arà}-caanda haari]=le su-dhaathang. \\
      so princess=\textsc{addit} wife=\textsc{addit} \textsc{simult}-meet day=\textsc{addit} \textsc{past}-come \\
\z      
}\\


This event is located in the past, yet \em arà- \em is used, which can then not have the non.past meaning, but rather the meaning of `simultaneous to the time of the matrix sentence' i.e. the coming of the day and the meeting are treated as refering to the same time frame.

The same thing can be said about the next example, where the hearing and the knocking are conceived of as simultaneous. It is not possible for the knocking to refer to non.past, since it is impossible to have heard something in the past which had not occured yet then. Therefore, the non.past reading of \em arà- \em is not an option here.


\xbox{16}{
\ea\label{ex:cl:relc:ara:simult2}
\gll [kìrras pinthu=nang \textbf{arà}-thatti hathu swaara] su-dìnggar. \\
     strong door=\textsc{dat} \textsc{simult}-hammer \textsc{indef} noise] \textsc{past}-hear \\
\z      
}\\ 

\subsubsection{Irrealis \em anthi-\em}\label{sec:cls:anthi}

\xbox{16}{
\ea\label{ex:cl:relc:unreferenced}
\gll relative anthi. \\
       \\
    `.'  (lack)
\z      
}\\ 


\subsubsection{Conjunctive participle \em asà-\em}\label{sec:cls:asa}
The conjunctive participle prefix can be used on its own in the relative clause, as in \xref{ex:cl:relc:asa:main}, or on non-final verbs if there are more verbs in the relative clause\xref{ex:cl:relc:asa:chain}.



\xbox{16}{
\ea\label{ex:cl:relc:asa:main}
\gll [Tony Hassan {\em uncle}=nang \textbf{asa}-kaasi duwith] athi-mintha ambel=si? \\
     Tony Hassan uncel=\textsc{dat} \textsc{cp}-give money \textsc{irr}-ask take=\textsc{interr}  \\
\z
} \\

\xbox{16}{
\ea\label{ex:cl:relc:asa:chain}
\gll [Banthu-an \textbf{asa}-mintha arà-naangis] swaara hatthu derang=nang su-dìnggar. \\
      help-\textsc{nmlzr} \textsc{cp}-beg \textsc{simult}-cry sound \textsc{indef} \textsc{3pl}=\textsc{dat} \textsc{past}-hear\\
\z      
}\\ 
 


\subsubsection{Infinitive/purposive \em mà-\em}\label{sec:cls:ma}

The infinitive prefix \em mà- \em can be used in two different contexts in subordinate clauses. In the first case, matrix clause and subordinate clause share two arguments, as in \xref{ex:cl:relc:ma:sharetwo}, where the adressee is argument of both giving and marrying, and the girls are also arguments of giving and marrying.



\xbox{16}{
\ea\label{ex:cl:relc:ma:sharetwo}
\gll  kithang lorang=nang baaye mliiga athi-kaasi, (\zero) \textbf{ma}-kaaving panthas pompang pada athi-kaasi,  { } duwith athi-kaasi. \\
      \textsc{1pl} \textsc{2pl}=\textsc{dat} good palace \textsc{irr}-give { } \textsc{inf}-marry beautiful female \textsc{pl} \textsc{irr}-give (\zero) money \textsc{irr}-give \\
\z
} \\
The argument of the main clause which is not the head noun of the relative clause participates in the event denoted by the matrix clause in order to engage in the event denoted by the subordinate clause with the referent denoted by the head noun of the relative clause. In this case, the adressees participate in the event of giving together with the girls in order to engage in the event of marrying together with the girls.

This differs from \xref{ex:cl:relc:ma:shareone}, where only one argument is shared between matrix clause, \em reason\em. This argument participates in both the events of telling and coming, but it is not the reason for the event depicted in the matrix clause.

\xbox{16}{
\ea\label{ex:cl:relc:ma:shareone}
\gll [inni     mlaayu pada Sri  Lanka=nang    \textbf{ma}-dhaathang=nang {\em reason}]  aapa   katha arà-biilang. \\
     \textsc{prox} Malay \textsc{pl} Sri Lanka=\textsc{dat} \textsc{inf}-come=\textsc{dat} {} what \textsc{quot} \textsc{non.past}-tell  \\
\z      
}\\

While in \xref{ex:cl:relc:ma:sharetwo} the reason for giving the girls is marrying, in \xref{ex:cl:relc:ma:shareone}, the reason for telling is not coming. The interpretation `I tell the reason so that it enganges in an act of coming' is not possible. This is different from \xref{ex:cl:relc:ma:sharetwo}, which can be paraphrased as `I give you girls so that you engage in the event of marrying with the girls.'

Two more examples of the use of \em mà- \em in relative clauses can be found below.

\xbox{16}{
\ea\label{ex:cl:relc:ma:extra1}
\gll Lorang se=dang [\textbf{ma}-hiidop thumpath] kala-kaasi, ...  \\
     \textsc{2pl} \textsc{1s=dat} \textsc{inf}-stay place if-give, ...  \\
\z      
}\\ 


\xbox{16}{
\ea\label{ex:cl:relc:ma:extra2}
\gll itthu=nang aada  [{\em divorce} \textbf{ma}-kijja=nang hatthu prentha oorang]. \\
    `For that there is a lawyer to make the divorce.'  (K061122nar01)
\z      
}\\ 

\subsubsection{Negative past \em thàrà-\em}\label{sec:cls:negative}

\xbox{16}{
\ea\label{ex:cl:relc:thara}
\gll 
derang pada panggel=nang blaakang [\textbf{thàrà}-dhaathang oorang pada]=nang nya-force-kang kiyang \\
  \textsc{3pl} \textsc{pl} call=\textsc{dat} after \textsc{neg.past}-come man \textsc{pl}=\textsc{dat} \textsc{past}-force-\textsc{caus} \textsc{evid}    \\
\z      
}\\ 

\subsection{Predicate types in the relative clause}\label{sec:cls:Predicatetypesintherelativeclause}
All predicate types can be found in relative clauses.

\subsubsection{Verbal predicate}\label{sec:cls:verbal}

\xbox{16}{
\ea\label{ex:cl:relc:verbal}
\gll itthu [se arà-\textbf{kirijja}] mosthor=jo. \\
 \textsc{dist} \textsc{1s} \textsc{non.past}-make manner=foc\\
\z
}   

\subsubsection{Modal predicate}\label{sec:cls:modal}
Modal predicates can be used in relative clauses. This is especially frequent for \trs{boole}{can}, as given in \xref{ex:cl:relc:modal1}.

 
\xbox{16}{
\ea\label{ex:cl:relc:modal1}
\gll [deram pada \el{} baae=nang pirrang mà-kijja \textbf{boole}] oorang. \\
 \textsc{3pl} \textsc{pl} { } good=\textsc{dat} war \textsc{inf}-make can man\\
\z
}

Normally, the head noun takes the case the matrix clause requires, as in \xref{ex:cl:relc:modal1} (where it is zero), but it also occurs that the head noun takes the case required by the relative clause instead of the one required by the matrix clause. In \xref{ex:cl:relc:modal2}, the matrix clause would not assigns any case to its only argument, but \em boole \em in the relative clause normally assigns dative, which is then also marked on the head noun by the postposition \em =nang\em.

\xbox{16}{
\ea\label{ex:cl:relc:modal2}
\gll  [itthu    baaye mosthor=nang \textbf{bole}=kirja    oorang mlaayu]=nang sajja=jo. \\
      \textsc{dist} good manner=\textsc{dat} can-make man Malay=\textsc{dat} only=\textsc{foc} \\
\z
} \\
The negation \em thàrboole \em can also be used in a relative clause.

\xbox{16}{
\ea\label{ex:cl:relc:modal:therboole}
\ea
\gll [boole oorang   pada] samma dhaathang. \\
     can man \textsc{pl} all come   \\
    `The people who could came.' 
\ex
\gll  [\textbf{thàràboole}   oorang pada] su-biilang:     kithang=nang   mà-dhaathang    thàràboole. \\
      cannot man \textsc{pl} \textsc{past}-say \textsc{1pl}=\textsc{dat} \textsc{inf}-come cannot \\
\z
\z
} \\
% 
%  \xbox{16}{
%  \ea\label{ex:cl:relc:unreferenced}
%    \gll  {\em Malaysia}=ka    anà-duuduk        mosthor, [anà-boole    mosthor], kithang itthu=yang   itthu    mosthor=nang   arà-baapi. \\
%     Malaysia=\textsc{loc} \textsc{past}-exist.\textsc{anim} manner \textsc{past}-can manner \textsc{1pl} \textsc{dist}=\textsc{acc} \textsc{dist} manner=\textsc{dat} \textsc{non.past}-bring \\
% \z
% }

Besides \em boole\em, \em maau \em has also been found occuring in a relative clause.



\xbox{16}{
\ea\label{ex:cl:relc:modal:mau}
\gll  [derang=nang \textbf{maau} mosthor] baalas katha nya-biilang. \\
      \textsc{3pl}=\textsc{dat} want manner answer \textsc{quot} \textsc{past}-say \\
\z
} \\



\subsubsection{Nominal and adjectival predicates}\label{sec:cls:nominalandadjectival}

Relative clauses based on nominal predicate clauses and adjectival predicate clauses are formally indistinguishable from nouns modified by a bare noun \xref{ex:cl:relc:nominal} or a bare adjective \xref{ex:cl:relc:adjectival} instead of a a clause.

\xbox{16}{
\ea\label{ex:cl:relc:nominal}
\gll [moonyeth]$_{N/RELC}$  hathu kawanan su-aada. \\ % bf
     monkey \textsc{indef}=group \textsc{past}-exist\\
    `There was a   monkey group.'\\
    `There was a group which consisted of monkeys.'  (K070000wrt01)
\z      
}\\ 


\xbox{16}{
\ea\label{ex:cl:relc:adjectival}
\gll ini [laama]$_{ADJ/RELC}$ {\em car} pada=jo kithang arà-baapi. \\ % bf
      \textsc{prox} old car \textsc{pl}=\textsc{foc} \textsc{1pl} \textsc{non.past}-bring \\
    `It is these cars which are old that we take [to Iraq].' (K051206nar19)
\z      
}\\ 

When the relative clause can be distinguihed from a nominal premodification is in the case that there are arguments to it, as in \xref{ex:relc:pred:nom:kamauwan}, where \trs{se=dang}{1s=dat} is an argument of \trs{kamauwan}{wish/want}. It is not clear, however, whether \em kamauwan \em is used as a noun here, or rather as a modal particle.

\xbox{16}{
\ea\label{ex:relc:pred:nom:kamauwan}
\gll lorang [se=dang kamauvan pada]=yang gijja kaasi. \\ % bf
     \textsc{2pl} \textsc{1s=dat} desire \textsc{pl}=\textsc{acc} make give  \\
\z      
}\\ 

\subsubsection{Circumstantial predicates}\label{sec:cls:circumstantial}
% \xbox{16}{
% \ea\label{ex:cl:relc:circ}
% \gll [[\textbf{sithu=ka}     \textbf{aada}]  bìssar oorang pada]=yang   asà-attack-kang     mail=nya    asà-cuuri [\textbf{{\em mail}=ka}]    duwith arà-baapi. \\
%      there=\textsc{loc} exist big man \textsc{pl}=\textsc{acc} \textsc{cp}-attack-\textsc{caus} mail=\textsc{acc} \textsc{cp}-steal mail=\textsc{loc} money \textsc{non.past}-bring  \\
% \z
% } \\
Circumstantial predicates are normally found  with overt marking of the existential, as in the following two examples.

\xbox{16}{
\ea\label{ex:cl:relc:circ:aada1}
\gll [incayang=pe kepaala=ka \textbf{anà-aada}] thoppi=dering moonyeth pada=nang su-buwang puukul. \\
      \textsc{3s.polite}=\textsc{poss} head=\textsc{loc} \textsc{past}-exist hat=\textsc{abl} monkey \textsc{pl}=\textsc{dat} \textsc{past}-throw hit \\
\z      
}\\ 


\xbox{16}{
\ea\label{ex:cl:relc:circ:aada2}
\gll derang [dìkkath=ka \textbf{aada}] laapang=nang mà-maayeng=nang su-pii. \\
     \textsc{3pl} vicinity=\textsc{loc} exist ground=\textsc{dat} \textsc{inf}-play=\textsc{dat} \textsc{past}-go  \\
\z      
}\\ 


%K051206nar04.txt:  pirrang=nang     baae  aada oorang pada=jo


\subsubsection{Relative clauses based on utterances}\label{sec:cls:Utterance}
Besides clauses, it is also possible to use utterances to premodify a noun. The following examples show this for a short reported string \xref{ex:cl:relc:utterance:name1}\xref{ex:cl:relc:utterance:name2} and a long reported string giving the content of \trs{habbar}{news}\xref{ex:cl:relc:utterance:clause}.

\xbox{16}{
 \ea\label{ex:cl:relc:utterance:name1}
   \gll  baawa=ka  [Kaasim katha]$_{RELC}$ [hatthu]$_{INDEF}$ {\em family}$_{N}$. \\
     bottom=\textsc{loc} Kaasim \textsc{quot} \textsc{indef} family \\
\z
}

\xbox{16}{
 \ea\label{ex:cl:relc:utterance:name2}
\gll itthukang     anà-aada      [Mr  Janson  katha]$_{RELC}$  [hathu]$_{INDEF}$  oorang$_{N}$. \\
     then \textsc{past}-exist Mr Janson \textsc{quot} one man. \\
    `Then there was a certain Mr Janson.' (K051206nar04)
\z
} \\
\xbox{16}{
\ea\label{ex:cl:relc:utterance:clause}
\gll  se=dang habbar, [[ini      laama {\em car} pada samma inni     {\em suicide} {\em bombers} asà-dhaathang {\em car}=yang, aapa,    arà-paavicci     katha] habbar]. \\
      \textsc{1s=dat} news \textsc{prox} old cat \textsc{pl} all \textsc{prox} suicide bombers \textsc{cp}-come car=\textsc{acc} what \textsc{non.past}-use(Sinh.) \textsc{quot} news \\
    `I have information, information ``these suicide bombers come and take all these old cars and, what, and use them.'' ' (K051206nar19)
\z
} \\
\subsection{Semantic roles in the relative clause}\label{sec:cls:Semanticrolesintherelativeclause}
There is no restriction on the semantic roles that can be relativized on. This will be shown for the different semantic roles in the individual sections below.

\subsubsection{Agent}\label{sec:cls:Agent}
The verb \trs{dhaathang}{come} subcategorizes for Agent.

\xbox{16}{
\ea\label{ex:cl:relc:agent}
\gll [Seelon=nang dhaathang aada] melayu oorang ikkang. \\ % bf
 Ceylon-\textsc{dat} come exist Malay man fish\\
`The Malays who came to Sri Lanka were fishermen.' (K060108nar02.23)
\z
}


%\xbox{16}{
%\ea\label{ex:cl:relc:unreferenced}
%\gll [Banthu-an asà-mintha arà-naangis] swaara hatthu derang=nang su-dìnggar. \\
%      help-\textsc{nmlzr} \textsc{cp}-beg \textsc{non.past}-cry sound \textsc{indef} \textsc{3pl}=\textsc{dat} \textsc{past}-hear\\
%\z      
%}\\ 


\subsubsection{Patient}\label{sec:cls:Patient}
Patients of both intransitive clauses (\trs{mniinggal}{die} in\xref{ex:cl:relc:patient:intr})
and transitive clauses (\trs{baa}{bring} in \xref{ex:cl:relc:patient:tr1}\xref{ex:cl:relc:patient:tr2}) can be relativized on.

\xbox{16}{
\ea\label{ex:cl:relc:patient:intr}
\gll [{\em north}-pe    pirrang-ka mniinggal mlaayu pada]=nang bìssar hatthu {\em religious} {\em function}. \\ % bf
north=\textsc{poss} war=\textsc{poss} die Malay \textsc{pl}=\textsc{dat} big \textsc{indef} religious function\\
`A religious function for the Malays who had died in the Northern war.' (K060116nar11)
\z
}
 


\xbox{16}{
\ea\label{ex:cl:relc:patient:tr1}
\ea 
\gll ka-duuwa     anà-dhaathang    {\em slaves}  pada. \\ % bf
     card-two \textsc{past}-come slaves \textsc{pl}  \\
    `The second to come were slaves,'  
\ex
\gll [{\em soldier} pada        na-baa     oorang pada]. \\ % bf
     soldier \textsc{pl} \textsc{past}-bring man \textsc{pl}  \\
\z
\z
}


\xbox{16}{
\ea\label{ex:cl:relc:patient:tr2}
\gll [[itthu  mà-jaaga=nang] anà-baa melayu]=dring   satthu oorang=jo    se. \\ % bf
 \textsc{dist} \textsc{inf}-protect=\textsc{dat} \textsc{past}-bring malay]=\textsc{abl} one man=\textsc{foc} 1s\\
\z 
}\\   
 

% \xbox{16}{
% \ea\label{ex:cl:relc:patient:tr2}
% \glll [[hathu_{NUM_i} duuri pohong]_{NP}=nang [puuthi   paanjang jeenggoth]_{NP}=yang anà-kana-daapath kìnna]_{RElC} hathu_{NUM_j} kiccil_{ADJ} jillek_{ADJ} Aajuth_{HEAD} hatthu_{NUM_j} yang su-kuthumung. \\
%       [{} {} {} {} {} {} {} {} {}]_{RELC} NUM ADJ ADJ N NUM {} {}\\
%       \textsc{indef} thorn tree=\textsc{dat} white long beard=\textsc{acc} r\textsc{past}-\textsc{invol}-get befall \textsc{indef} small ugly dwarf indef]=\textsc{acc} past see\\
%     `They saw a small ugly dwarf who had his beard got stuck in a thorn tree.'  (K070000wrt04)
% \z      
% }\\ 
% 


\subsubsection{Theme}\label{sec:cls:Theme}
For the semantic role of theme, relativization is possible for all the different shades of meaning, be they a non-affected undergoer as in \xref{ex:cl:relc:theme:theme1} or \xref{ex:cl:relc:theme:theme2}, a stimulus as in \xref{ex:cl:relc:theme:stimulus1} of \xref{ex:cl:relc:theme:stimulus2}, or an item which is said to be located at some place \xref{ex:cl:relc:theme:exist1}\xref{ex:cl:relc:theme:exist2}.

\xbox{16}{
\ea\label{ex:cl:relc:theme:theme1}
   \gll  incayang  [ini      Seelong=ka  anà-aada    lakuan   baathu] asà-caari. \\ % bf
    \textsc{3s.polite} \textsc{prox} Seelon=\textsc{loc} \textsc{past}-exist wealth stone \textsc{cp}-find \\
\z
}

 \xbox{16}{
 \ea\label{ex:cl:relc:theme:theme2}
   \gll   [spaaman anà-niinggal thumpath]=nang=le        [Passara   katha arà-biilang    nigiri]=nang=le dìkkath. \\ % bf
         \textsc{3s} \textsc{past}-die place=\textsc{dat}=\textsc{addit} Passara \textsc{quot} \textsc{non.past}-say country=\textsc{dat}=\textsc{addit} vicinity\\
\z
}


\xbox{16}{
\ea\label{ex:cl:relc:theme:stimulus1}
\gll [kìrras pinthu=nang arà-thatti hathu swaara] su-dìnggar. \\ % bf
     strong door=\textsc{dat} \textsc{simult}-hammer \textsc{indef} noise] \textsc{past}-hear \\
\z      
}\\ 




\xbox{16}{
\ea\label{ex:cl:relc:theme:stimulus2}
\gll [se=dang   thaau mosthor]=nang karang=nang   ka-dhlaapan     {\em generation} arà-pii. \\ % bf
     \textsc{1s=dat} know way=\textsc{dat} now=\textsc{dat} card-eight generation \textsc{non.past}-go  \\
    `As far as I know, we are now in the eighth generation.'  (K060108nar02)
\z      
}\\ 

\xbox{16}{
\ea\label{ex:cl:relc:theme:exist1}
\gll derang [ini kaaving=nang aada] haarath saarath pada, kitham=pe bannyak ooram pada arà-kijja. \\ % bf
     \textsc{3pl} \textsc{prox} wedding=\textsc{dat} exist   traditions \textsc{pl} 1p=\textsc{poss} many people \textsc{pl} \textsc{non.past}-make\\
\z      
}\\
  
\xbox{16}{
\ea\label{ex:cl:relc:theme:exist2}
\gll  suda karang [kithang=nang   aada  {\em problem}] dhaathangapa kithang=pe     aanak  pada mlaayu thama-oomong. \\ % bf
      thus now \textsc{1pl}=\textsc{dat} exist problem \textsc{copula} \textsc{1pl}=\textsc{poss} child \textsc{pl} Malay \textsc{neg.nonpast}-speak \\
\z
} \\
 
% 
% \xbox{16}{
% \ea\label{ex:cl:relc:unreferenced}
% \gll [Andare kanabisan=nang anà-mintha] hathu raaja=ke asà-paake=apa kampong=nang mà-pii maau katha. \\
%     Andare last=\textsc{dat} \textsc{past}-ask \textsc{indef} king=\textsc{simil} \textsc{cp}-dress=after village=\textsc{dat} \textsc{inf}-go want \textsc{quot}   \\
% \z      
% }\\ 


% 
% \xbox{16}{
% \ea\label{ex:cl:relc:unreferenced}
% \gll [sithu=ka     aada]$_{RELC}$  [bìssar]$_{ADJ}$ oorang$_{N}$ pada=yang   asà-attack-kang     mail=nya    asà-cuuri {\em mail}=ka    duwith arà-baapi. \\
%      there=\textsc{loc} exist big man \textsc{pl}=\textsc{acc} \textsc{cp}-attack-\textsc{caus} mail=\textsc{acc} \textsc{cp}-steal mail=\textsc{loc} money \textsc{non.past}-bring  \\
% \z
% } \\

\subsubsection{Experiencer}\label{sec:cls:Experiencer}
The person experiencing the ability is relativized on in \xref{ex:cl:relc:exp}.

 
\xbox{16}{
\ea\label{ex:cl:relc:exp}
\gll deram pada ... [baae=nang pirrang mà-kijja boole oorang] \\ % bf
 \textsc{3pl} { } \textsc{pl} good=\textsc{dat} war \textsc{inf}-make can man\\
\z
}
    
\subsubsection{Recipient}\label{sec:cls:Recipient}
The recipient of the verb \trs{kaasi}{give} is relativized on in \xref{ex:cl:relc:rec}.

\xbox{16}{
\ea\label{ex:cl:relc:rec}
\gll se duwith anà-kaasi oorang su-iilang. \\
     \textsc{1s=dat} money  \textsc{past}-give man \textsc{past}-disappear  \\
    `The man I gave money to disappeared.'  (test)5.11.08
\z
}\\ 

\subsubsection{Possessor}\label{sec:cls:Possessor}
The husband formerly `possessed' by the woman is relativized on in \xref{ex:cl:relc:poss}

\xbox{16}{
\ea\label{ex:cl:relc:poss}
\gll    [laaki anà-mniinggal hathu pompang]. \\ % bf
     husband \textsc{past}-die \textsc{indef} woman  \\
    `A woman whose husband had died.'  (K070000wrt04)
\z      
}\\ 


\subsubsection{Location}\label{sec:cls:Location}
Locations can be relativized on. In \xref{ex:cl:relc:loc1}, the location of staying is at them same time the location of falling. The locative argument \trs{thumpath}{place} is found as the semantic role relativized on. The same is true of \em thumpath \em in \xref{ex:cl:relc:loc2}

\xbox{16}{
\ea\label{ex:cl:relc:loc1}
   \gll siithu [nya-duuduk    thumpath]=ka baapa  su-jaatho. \\ % bf
     there \textsc{past}-stay place=\textsc{loc} father \textsc{past}-fall  \\
    `There, at the place he was staying, my father fell.' (K051205nar05)
\z
} 

\xbox{16}{
 \ea\label{ex:cl:relc:loc2}
   \gll   [spaaman anà-niinggal thumpath]=nang=le        [Passara   katha arà-biilang    nigiri]=nang=le dìkkath. \\ % bf
         \textsc{3s} \textsc{past}-die place=\textsc{dat}=\textsc{addit} Passara \textsc{quot} \textsc{non.past}-say country=\textsc{dat}=\textsc{addit} vicinity\\
\z
}

Locations can also be relativized on in headless relative clauses \formref{sec:nppp:Headlessrelativeclauses}, as in \xref{ex:cl:relc:loc3}, where the place of graduating (\em passing out \em in Sri Lanka) is relativized upon.

\xbox{16}{
\ea\label{ex:cl:relc:loc3}
\gll  [see anà-{\em pass.out} \zero{}] abbisdhaathang       {\em University} of Peradeniya=ka\\ % bf
    `Where I graduated was the University of Peradeniya.' (K061026prs01)
\z
} \\
\subsubsection{Time}\label{sec:cls:Time}
Points in time can be relativized on, such as \trs{haari}{day} in \xref{ex:cl:relc:time1}, or \trs{thaaun}{year} in \xref{ex:cl:relc:time2}

\xbox{16}{
\ea\label{ex:cl:relc:time1}
\gll Suda [puthri=le biini=le arà-caanda haari]=le su-dhaathang. \\ % bf
      so princess=\textsc{addit} wife=\textsc{addit} \textsc{simult}-meet day=\textsc{addit} \textsc{past}-come \\
\z      
}\\ 

  
\xbox{16}{
\ea\label{ex:cl:relc:time2}
\gll itthu thaaun=jo [Mahathma Gandhi arà-buunu thaaun]. \\ % bf
dist year=\textsc{foc} Mahathma Gandhi \textsc{non.past}-kill year \\
`That year was the year that Mahathma Gandhi was killed.' (K051213nar02)
\z
}


% \xbox{16}{
% \ea\label{ex:cl:relc:time3}
% \gll  se=ppe [nya-laaher {\em date}] duuwa duuwa 1960. \\ % bf
%      \textsc{1s}=\textsc{poss} \textsc{past}-date two two 1960  \\
% \z
% } \\
\subsubsection{Instrument}\label{sec:cls:Instrument}
The following sentences shows that the instrument for collecting can be found in a relative clause.


\xbox{16}{
\ea\label{ex:cl:relc:instr}
\gll incayang=ka [[bannyak panthas ummas baarang pada=le   bathu inthan pada=le anà-punnu-kang]$_{RELC}$ bìssar beecek caaya hathu {\em bag}$_{head noun}$] su-aada. \\ % bf
       \textsc{3s}=\textsc{loc} many beautiful gold good \textsc{pl}=\textsc{addit} stone value \textsc{pl}=\textsc{addit} \textsc{past}-lot-\textsc{caus} big mud colour \textsc{indef} bag \textsc{past}-exist\\
\z      
}\\ 

\subsubsection{Manner}\label{sec:cls:Manner}
The relativized argument can have the role of Manner.

\xbox{16}{
\ea\label{ex:cl:relc:manner}
\gll itthu [se arà-kirija] mosthor=jo. \\ % bf
 \textsc{dist} \textsc{1s} \textsc{non.past}-make manner=foc\\
\z
}   


\subsubsection{Purpose}\label{sec:cls:Purpose}
If the relativized role is Purpose, the relative clause will be in the infinitive.

\xbox{16}{
\ea\label{ex:cl:relc:purpose1}
\gll derang pada=nang [itthu mà-kumpul athu mosthor] thraa. \\ % bf
      \textsc{3pl} \textsc{pl}=\textsc{dat} \textsc{dist} \textsc{inf}-add \textsc{indef} way neg\\
\z      
}\\ 


\xbox{16}{
\ea\label{ex:cl:relc:purpose2}
\gll itthu=nang aada  [{\em divorce} mà-kijja=nang hatthu prentha oorang]. \\ % bf
      \textsc{dist}=\textsc{dat} exist divorce \textsc{inf}-make=\textsc{dat} \textsc{indef} law man \\
\z      
}\\

\subsubsection{Fact}\label{sec:cls:Fact}

\xbox{16}{
\ea\label{ex:cl:relc:fact}
\gll  se=dang habbar, [ini  laama {\em car} pada samma inni {\em suicide} bombers asà-dhaathang {\em car}=yang, aapa,   arà-paavicci     katha habbar]. \\ % bf
      \textsc{1s=dat} news \textsc{prox} old cat \textsc{pl} all \textsc{prox} suicide bombers \textsc{cp}-come car=\textsc{acc} what \textsc{non.past}-use \textsc{quot} news \\
    `I have information, information that these suicide bombers come and take all these old cars and, what, and use them.' (K051206nar19)
\z
} \\

%K060116nar10.3  itthuka    aada hamma mlaayu kaaving  aapacare jaalang katha

Typologically speaking,  SLM is of the Japanese type, and not of the Dravidian type like Sinhala or Tamil \citep[50f,70f]{Lehmann1984}. Malayic relative clauses are postposed and use a relative pronoun, so there is no Malayic influence here.


Relative clauses are frequently used by anybody. They have been described by \citet{Slomanson2007cll}


Relative clauses can also occur without a head they modify. These headless relative clauses are treated in \formref{sec:nppp:Headlessrelativeclauses}.



\section{Conjunctive participle clause}\label{sec:cls:Conjunctiveparticipleclause}

For sequence of events, a hypotactical construction exists, which consists in using the conjunctive participle (infinite) in all but the last clause.  The finite clause has to be the last event. 

\cb{ 
$
	\left[ 
		\NP *
	 	\left\{\begin{array}{c}as\grave{a}-\\jam\grave{a}-\end{array}\right\}-V
	\right]
$* MAIN CLAUSE
}

The subordinate clauses may only have verbal predicates. This is normally no drawback since all [+dynamic] states-of-affairs are coded by verbs, and the occurrence of [-dynamic] states-of-affairs in a sequence is limited. If another state-of-affair is to be used in a sequence, it must necessarily get a dynamic reading (see \funcref{sec:func:Events} for strategies for this).

Example \xref{ex:cl:cp:intro} shows a typical instance of this construction. There are three clauses, of which the first two contain a verb marked by \em asà- \em, while the last one contains a verb in another tense. Note that in this case, the referent \trs{oorang pada}{the people} is introduced in the first clause, while in English, it would be introduced in the matrix clause, in this case the last one.


\xbox{16}{
\ea\label{ex:cl:cp:intro}
 \ea 
 \gll oorang pada \textbf{asà-}pirrang. \\
	 man \textsc{pl} \textsc{cp}-wage.war  \\
	\ex
 \gll derang=nang \textbf{asà-}banthu. \\
	3pl=\textsc{dat} \textsc{cp}-help\\
	\ex
	\gll siini=jo se-ciinggal. \\ % bf
	dem.loc.prox=\textsc{foc} \textsc{past}-settle\\
	\z
\z
}

If one of the subordinate clauses is in the negative, \em jamà- \em \formref{sec:morph:jama-} is used instead of \em asà-\em.
 
\xbox{16}{
\ea\label{ex:cl:cp:jama}
\ea 
\gll liiwath aayer \textbf{jamà}-jaadi=\textbf{nang}. \\
     much water \textsc{neg.nonfin}-become=\textsc{dat}  \\
    `Without putting too much water(=having not put too much water)'  
\ex
\gll itthu aayer=yang hathu blaangan=nang luppas. \\ % bf
     \textsc{dist} water=\textsc{acc} \textsc{indef} amount=\textsc{dat} leave  \\
\z
\z
} \\
There are no restrictions to the roles that the arguments of the conjoined clauses may have. Other languages allow conjunctive participle constructions only if the subject (in those languages) are identical. This is the case for most South Asian languages \citep[108]{Masica1976}, but not for SLM  as \xref{ex:goasniinggal} shows:

\xbox{16}{
\ea\label{ex:goasniinggal}
\ea
\gll  go \textbf{asa}-niinggal, \\
1s.familiar \textsc{cp}-die \\
`I having died'
\ex 
\gll alla  go=nya   \textbf{asa}-dhaathang, \\
Allah \textsc{1s.familiar}=\textsc{acc} \textsc{cp}-come\\
\ex 
\gll kuburan     \textbf{asa}-gaali, \\
 grave \textsc{cp}-dig \\
`The grave having been dug'
\ex 
\gll go=nya   kubur-king!    \\ % bf
      \textsc{1s.familiar}=\textsc{acc} bury-\textsc{caus}   \\
      Bury me!' \\
    `I  die and Allah  comes for me and the grave will be dug and they will have me buried.' (B060115nar05)
\z
\z
} \\
The only argument of the first clause is \trs{go}{I}, while the agent/`subject' of the second one is \em Allah\em. The arguments of the first and the second clause are not identical. One could argue that the restriction on coordination only holds for zero-anaphors. This is also not the case in SLM, as the third and fourth clause show. If this restriction held, \em Allah \em would be the subject of these clauses, but this is obviously not the case.


Another example of reference changing between the \em asa\em-clauses and the main clause is \xref{ex:cl:cp:ds}, where the first three lines are about the parents and the last two are about the children. Nevertheless, all but the last clause are marked by \em asà-\em.


\xbox{16}{
\ea\label{ex:cl:cp:ds}
\ea
\gll  nni      nigiri=ka=jo    \zero$_{i(ag)}$ kitham=pe     [aanak buwa pada]$_j$=yang   asà-simpang. \\ % bf
      \textsc{prox} country=\textsc{loc}=\textsc{emph} { } \textsc{1pl}=\textsc{poss} child fruit \textsc{pl}=\textsc{acc} \textsc{cp}-keep \\
\ex
\gll \zero$_i$ \zero$_j$ inni     {\em schools} pada=nang   asà-kiiring, samma asà-kirja. \\ % bf
     { } { }  \textsc{prox} schools \textsc{pl}=\textsc{dat} \textsc{cp}-send all \textsc{cp}-make\\
\ex
\gll   karang \zero$_j$ asà-blaajar, \zero$_j$   pukurjan asà-kirja   ambel. \\ % bf
       now { } \textsc{cp}-learn { } work \textsc{cp}-make take \\
\ex
\gll  skarang \zero$_{i,j}$ siini=jo  arà-duuduk. \\ % bf
      now { } here=\textsc{foc} \textsc{non.past}-live \\
\z
\z
} \\
\section{Purposive clauses}\label{sec:cls:Purposiveclauses}
The purposive clause resembles the main clause, with the exception that the verb is in the infinitive and that the purposive clause cannot stand on its own but must have a matrix clause it attaches to. Purposive clauses are often additionally marked with \em =nang\em, but this is optional \citep[cf.][139f]{Slomanson2007cll}.

\cb{$\left[\NP* \left\{\begin{array}{l}m\grave{a}-\\jam\grave{a}-\end{array}\right\} V\right]$ MAINCLAUSE}
\cb{MAINCLAUSE $\left[\NP* \left\{\begin{array}{l}m\grave{a}-\\jam\grave{a}-\end{array}\right\}V\right]$}

The purposive clause can be center-embedded in the main clause as in \xref{ex:subord:purp:center} or follow the predicate of the main clause as in \xref{ex:subord:purp:right}

\xbox{16}{
\ea\label{ex:subord:purp:center}
\gll derang [dìkkath=ka aada laapang]=nang [mà-maayeng]$_{purp}$=nang su-pii. \\ % bf
     \textsc{3pl} vicinity=\textsc{loc} exist ground=\textsc{dat} \textsc{inf}-play=\textsc{dat} \textsc{past}-go  \\
\z      
}\\ 

\xbox{16}{
\ea\label{ex:subord:purp:right}
\gll itthu   {\em cave}=nang kithang=le pii aada [mà-liyath]$_{purp}$=nang. \\ % bf
 \textsc{dist} cave=\textsc{dat} \textsc{1pl}=\textsc{addit} go exist \textsc{inf}-look=\textsc{dat}    \\
\z
}\\

The use of   \em jamà- \em to yield negative purposive clauses is dubious, as shown in \xref{ex:subord:purp:jama}.


\xbox{16}{
\ea\label{ex:subord:purp:jama}
\gll  ??se incayang=nang duwith anà-kaasi jamà-oomong katha. \\
      \textsc{1s} \textsc{3s.polite}=\textsc{dat} money \textsc{past}-give \textsc{neg.inf}-talk \textsc{quot} \\
\z
} \\

\section{Subordinate interrogative clauses}\label{sec:cls:Subordinateinterrogativeclauses}
Interrogative pronouns can replace NPs in subordinates just as they can in main clauses. This is done if the speaker reports a question as in \xref{ex:subq:repq1}\xref{ex:subq:repq2}, or if he confesses his ignorance as in \xref{ex:subq:ign1}\xref{ex:subq:ign2}, or if he indicates who knows the answer\xref{ex:subq:answer}.

\xbox{16}{
\ea\label{ex:subq:repq1}
\gll Andare raaja=ka su-caanya [inni mà-kirring simpang aada \textbf{aapa=yang}] katha. \\
     Andare king=\textsc{loc} \textsc{past}-ask \textsc{prox} \textsc{inf}-dry keep exist what=\textsc{acc} \textsc{quot} \\
\z      
}\\

\xbox{16}{
\ea\label{ex:subq:repq2}
\gll   incayang  arà-caari     [inni     awuliya \textbf{aapa} \textbf{mosthor}]  katha. \\
    `He is looking for how that saint was.' (B060115cvs04)
\z
} \\

\xbox{16}{
\ea\label{ex:subq:ign1}
\gll [ithu \textbf{maana\~{}maana} thumpath] katha kithang=nang buthul=nang mà-biilang thàrboole. \\
     \textsc{dist} which\~{}red place \textsc{quot} \textsc{1pl}=\textsc{dat} correct=\textsc{dat} \textsc{inf}-say cannot  \\
\z      
}\\ 

\xbox{16}{
\ea\label{ex:subq:ign2}
\gll Se=ppe oorang pada [see \textbf{saapa}] katha thàràthaau subbath see=yang su-uubar. \\
     \textsc{1s}=\textsc{poss} man \textsc{pl} \textsc{1s} who \textsc{quot} ignore because \textsc{1s}=\textsc{acc} \textsc{past}-chase\\
\z      
}\\ 


\xbox{16}{
\ea\label{ex:subq:answer}
\gll incayang=nang thaau itthu=pe mosthor=atthas punnu, cinggala \textbf{aapacara} anà-banthu katha. \\
      \textsc{3s.polite}=\textsc{dat} know \textsc{dist}=\textsc{poss} way=about  much Sinhala how \textsc{past}-help \textsc{quot} \\
\z      
}\\



The five examples above treat content questions, which are marked by \em katha\em. \em Katha \em can be missing, as in the following example.


\xbox{16}{
\ea\label{ex:subq:nokatha}
\gll derang thàràthaau [\textbf{aapacara} anà-jaadi] \zero{}. \\
     \textsc{3pl} ignore  how \textsc{past}-happen { } \\
\z      
}\\ 



It is also possible to have truly embedded questions, which are marked by \em =so. \em This is possible for content questions as in as in \xref{ex:subq:so:content} and for polar questions \xref{ex:subq:so:polar}.

\xbox{16}{
\ea\label{ex:subq:so:content}
\ea
\gll itthu see arà-gijja mosthor=jo. \\ % bf
      \textsc{dist} \textsc{1s} \textsc{non.past}-make way=\textsc{foc}  \\
\ex
\gll  See ini arà-biilang. \\ % bf
      \textsc{1s} \textsc{prox} \textsc{non.past}-say \\
\ex
\gll  [Laayeng oorang pada \textbf{aapacara} \textbf{arà-gijja=so}] thàràthaau. \\
    `I do not know how other people do it.'  (B060115rcp02)
\z
\z
}\\ 

\xbox{16}{
\ea\label{ex:subq:so:polar}
\gll [inni samma anthi-oomong=\textbf{so}] thàràthaau. \\
     \textsc{prox} all \textsc{irr}-say=\textsc{undet} ignore  \\
    `I do not know what they say about all that.'  (G051222nar02)
\z      
}\\ 


In this function, \em =so \em competes with \em =si\em, which is found in   \xref{ex:subq:si1} and \xref{ex:subq:si2}. 

\xbox{16}{
\ea\label{ex:subq:si1}
\gll [Aashik=nang hathu {\em soldier} mà-jaadi suuka]=\textbf{si} \textbf{katha} arà-caanya. \\
    `He asks if you want to become a soldier, Ashik.' (B060115prs10)
\z
} \\

\xbox{16}{
\ea\label{ex:subq:si2}
\gll incalla   [lai     thaau sudaara sudaari pada]=ka    bole=caanya    ambel [[nya-gijja    lai     saapa=kee  aada]=\textbf{si}    \textbf{katha}]. \\
      Hopefully other know brother sister \textsc{pl}=\textsc{loc} can-ask take \textsc{past}-make other who=\textsc{simil} exist=\textsc{interr} \textsc{quot} \\
\z
} \\
Note that in these two examples, \em katha \em is found, so that the clause containing \em =si \em can be analyzed as a reported main clause. In \xref{ex:subq:so:content} and \xref{ex:subq:so:polar}, on the other hand, \em katha \em is not found, so that we are not dealing with a reported string, but rather with a true subordinate clause. The utterance containing \em =si \em thus has its own illocution (question), which is reported in the matrix clause with assertive illocution. There are thus two utterances, and two illocutions. When \em =so \em is used on the other hand, we are not dealing with an utterance, but only with a clause, which cannot have illocutionary force. This can also be seen from the absence of \em katha\em, which can only attach to utterances, but not to clauses. The \em =si\em- construction  thus contains two utterances, whereas the \em =so\em-construction only contains one. The embedded element in the \em =si\em-construction is an utterance, whereas it is a clause in the \em =so\em-construction

The difference can be captured in the following two patterns.

\cbx{
\fbox{  \NP* \fbox{\fbox{\NP* V}_{cls}=si}_{utt} ~katha \NP*	 V}_{cls}}
{utt}
 
\cbx{
\fbox{ \NP* \fbox{\NP* V=so}_{cls} \NP*  V}_{cls}}

\section{Supraordination}\label{sec:cls:supraordination}
An interesting feature of SLM syntax is the possibility of the matrix clause occuring \em within \em the embedded clause. I will call this structure `supra-ordination'. This is very often done with exlamations of surprise with the aim to make the adressee aware of the exceptional nature of the content. An answer is normally not expected. A typographical means to render this meaning in English is the combined use of the question mark and the exclamation mark as in \em Do you know what my boss just said !?!?\em. In the source language, I will indicate supraordination by inverted brackets ]...[ as in the following examples.

\xbox{16}{
\ea\label{ex:cl:supraord:rhet1}
\gll 
Suda inni moonyeth pada ]\textbf{aapa} \textbf{thaau=si}[ anà-gijja  !\\
    `Do you know what these monkeys then did!?'  (K070000wrt01)
\z      
}\\ 

\xbox{16}{
\ea\label{ex:cl:supraord:rhet2}
\gll suda derang pada siini derang itthu    oorang ]\textbf{aapa} \textbf{thaau=si}[   anà-gijja. \\
    `Do you know what these men did!?' (K051206nar07)
\z
} \\

\xbox{16}{
\ea\label{ex:cl:supraord:rhet3}
\gll itthu=nang      blaakang derang pada ]\textbf{aapa} \textbf{thaau}[ anà-gijja? \\
    `After that do you know what they did?' (K051206nar15)
\z
} \\

In the preceding examples, we are dealing with rhetorical questions where the answer is not provided yet. These sentences have a verb in the final position. In \xref{ex:cl:supraord:norhet}, things are a little bit different in that at the end of the utterance all the content is provided, i.e. the hearer does not have to try to find the answer to the rhetorical question because the answer (\trs{jiimath}{talisman}) is already provided.

 \xbox{16}{
\ea\label{ex:cl:supraord:norhet}
   \gll  kithang arà-thaaro ]\textbf{aapa} \textbf{thaau=si}[ jiimath. \\
`We put you know what?, a talisman!' (K051206nar02)
\z
}


% \xbox{16}{
% \ea\label{ex:cl:supraord}
% \gll  itthu=nang      aapa diya   asà-dhaathang. \\
%       \textsc{dist}=\textsc{dat} what see \textsc{cp}-come \\
%     `See what happened then!' (K051206nar02)
% \z
% } \\


% 
% 
% K061127nar03.trs:karang liyath
% K061127nar03.trs:kithang pe tsunami atthas ka ini aapa ara jaadi katha


\section{The position of adjuncts}\label{sec:cls:Thepositionofadjuncts}
Adjuncts are formed either by adverbs or by postpositional phrases. They can occur anywhere before, between or after NPs and predicates. The following schema illustrates this for a main clause with right-dislocation.

\cb{ (ADJCT) NP (ADJCT) NP (ADJCT) PRED (ADJCT) NP (ADJCT))}



The following three examples show the use of the adjunct \trs{karang}{now} in initial, medial and final position.

\xbox{16}{
\ea\label{ex:cl:adjct:karang:initial}
\gll \textbf{karang} ini kitham=pe nigiri su-jaadi. \\
 now \textsc{prox} \textsc{1pl}=\textsc{poss} country \textsc{past}-become\\
\z
}


\xbox{16}{
\ea\label{ex:cl:adjct:karang:medial}
\gll go=dang            \textbf{karang} konnyong  thàràsiggar. \\
     1s.familiar=dat now little ill  \\
    `I am a bit sick these days.'  (B060115nar04.72)
\z      
}\\ 



\xbox{16}{
\ea\label{ex:cl:adjct:karang:final}
\gll {\em Associations}  pada Bahasa Indonesia  Bahasa {\em Malaysia}=le       {\em introduce}-kang   aada \textbf{karang}. \\
  associations \textsc{pl} Bahasa Indonesia Bahasa Malaysia=\textsc{addit} introduce-\textsc{caus} exist now    \\
    `Associations have now introduced Bahasa Indonesia and Bahasa Malaysia.' (G051222nar03.4)
\z      
}\\ 


\section{Reported speech}\label{sec:cls:Reportedspeech}
Reported speech is indicated by the particle \em katha \em which is put at the end of the reported string. \em katha \em is used regardless of the  reported string being a clause, a subclausal unit, or an utterance. 
Example \xref{ex:reportedspeech:clause} shows a reported clause, \xref{ex:reportedspeech:interj} shows the use of \em katha \em on an interjection, which does not meet the criteria to be a clause. See \formref{sec:morph:katha} for more discussion and examples.


\xbox{16}{
\ea\label{ex:reportedspeech:clause}
\gll se=ppe      oorang thuuwa pada    anà-biilang [kitham pada {\em Malaysia}=dering    anà-dhaathang]$_{CLS}$    katha. \\ % bf
 \textsc{1s}=\textsc{poss} man old \textsc{pl} \textsc{past}-say \textsc{1pl} \textsc{pl} Malaysia=\textsc{abl} \textsc{past}-come quot\\
\z
}

\xbox{16}{
\ea\label{ex:reportedspeech:interj}
\gll [{\em yes}]$_{INTERJ}$  katha m-biilang. \\ % bf
 Yes \textsc{quot} \textsc{past}-say\\
\z
}

The reported utterance can precede \xref{ex:cl:report:precede} or follow \xref{ex:cl:report:follow1}\xref{ex:cl:report:follow2} the clause containing the verb of utterance or cognizance.

\xbox{16}{
\ea\label{ex:cl:report:precede}
\gll  [luu=nya jadi-kang rabbu saapa lu=ppe nabi pada saapa katha] biilang. \\ % bf
      \textsc{2s.familiar}=\textsc{acc} become-\textsc{caus} prophet who \textsc{2s}=\textsc{poss} prophet \textsc{pl} who \textsc{quot} say \\
    `Say who the prophet is who made you, who are your prophets.'
\z
} \\
\xbox{16}{
\ea\label{ex:cl:report:follow1}
\gll biilang [\zero{} se=ppe]    katha. \\ % bf
     say {} \textsc{1s}=\textsc{poss} \textsc{quot}  \\
\z
} \\ 
\xbox{16}{
\ea\label{ex:cl:report:follow2}
\gll spaaru oorangpada    arà-biilang    [\zero{} Seelong=nang  {\em English} anà-aaji     baa]   katha. \\ % bf
     some man \textsc{pl} \textsc{non.past}-say Ceylon=\textsc{dat} English past bring bring \textsc{quot}  \\
\z
} \\

\cb{
  \NP
	$
	\left\{   
		\begin{array}{l}
					\rm V_{say}\\\rm	V_{cognize}
		\end{array}
	\right\}$											[		UTTERANCE		\textit{katha}	]
}

\cb{
  \NP	[		UTTERANCE		\textit{katha}	] (NP)
	$
	\left\{
		\begin{array}{l}
					\rm V_{say}\\\rm	V_{cognize}
		\end{array}
	\right\}$										
}


While \em katha \em operates on the level of the utterance, and not on the level of the clause, deictic reference is still readjusted to the present speech situation. This means that tenses and pronouns change between the original string and the reported string. For instance, in \xref{ex:cl:report:follow1}, a fragment taken from a song, the loved one is asked to say that he belongs to the speaker, as conveyed by the translation `Say that you are \textbf{mine}'. Here, the deictic reference is readjusted since the literal string would have been `I am \textbf{yours}.' as in `Say: ``I am \textbf{yours}.'' ' Another example is \xref{ex:cl:report:deic:lu}, where in a double-embedded utterance, the child is supposed to say `Your mother has died', but the intended reading is of course that the child's mother had died, not the father's. This is obvious from the use of the familiar second person pronoun \em luu\em, which a father can use when addressing his child, but which would be impossible if the child was to address the father.

\xbox{16}{
\ea\label{ex:cl:report:deic:lu}
\ea
\gll Andare aanak=nang su-biilang:. \\ % bf
      Andare child=\textsc{dat} \textsc{past}-say \\
\ex
\gll Aanak. \\
     child  \\
    ` ``Son, '' '
\ex
\gll `\textbf{lu}=ppe        umma su-maathi'     katha  bithàràk=apa  asà-naangis   mari. \\
      \textsc{2s}=\textsc{poss} mother \textsc{past}-die \textsc{quot} scream=after \textsc{cp}-weep come.imp \\
    ` ``come and cry and weep `My (=your) mother has died!' '' (to fool the king)'
\z
\z
} \\
The deictic center is also readjusted in the temporal domain. In example \xref{ex:cl:report:deic:temp} the speaker reports expressing her being pleased. The literal string would have been `We are pleased' but in reported speech, this is changed to `We were pleased', because at the time of speaking, the event has already passed.


\xbox{16}{
\ea\label{ex:cl:report:deic:temp}
\gll [suda kithang=le \textbf{su}-suuka]     katha anà-biilang. \\ % bf
     thus \textsc{1pl}=\textsc{addit} \textsc{past}-like \textsc{quot} \textsc{past}-say  \\
\z
} \\
However, this is not always the case. In \xref{ex:cl:report:deic:temp:no}, a narrative about past events (identifiable by the past marker \em su-\em) contains an embedded quotation in non-past tense (\em arà-\em). However, \em arà- \em in this sentence could also be analyzed as simultaneous tense \formref{sec:morph:ara-}, so that it is difficult to decide whether temporal readjustment does indeed not take place.


\xbox{16}{
\ea\label{ex:cl:report:deic:temp:no}
\gll  Andare ruuma=nang asà-pii biini=nang \textbf{su}-biilang [puthri=nang kuuping \textbf{arà}-dìnggar kuurang katha]. \\
    Andare house=\textsc{dat} \textsc{cp}-go wife=\textsc{dat} \textsc{past}-say princess=\textsc{dat} ear \textsc{simult}-hear little \textsc{quot}  \\
\z
} \\

Furthermore, imperative clauses in direct speech are rendered as infinital clauses in reported speech.


\xbox{16}{
\ea
\gll oorang padanang   \textbf{mà-}dhaathang katha asabiilang. \\ % bf
      ma \textsc{pl}=\textsc{dat} \textsc{inf}-come \textsc{quot} \textsc{cp}-say \\
\z
} \\

 
% \xbox{16}{
% \ea\label{ex:constr:unreferenced}
% \gll karang [inni     aapa  n-jaari]       katha arà-biilang. \\
%  now \textsc{prox} what \textsc{past}-become \textsc{quot} \textsc{non.past}-say\\
% \z
% }





Sinhala and Tamil use their quotative constructions in very similar situations, but in both languages, the quotative is homophonous to the conjunctive participle of the word meaning `to say'. This is not the case in SLM, where \trs{asà-biilang}{\textsc{cp}-say} is different from \em katha\em.


%command
%    se Farook  n bilang V kata
%    se sepe aanakna  subiila  Mekkana  (*se)  (m )pii kata
%        son goes
%    se sepe aankana subiila  Mekkana  anti/ar  pii kata
%        father goes
%promise
%    se Farro a promiski  V kata




%B060115rcp02.txt: laeng       oorang pada apcara  kijja
%K051220nar02.txt:  biilang baae  apcara thaaro mandi     appa
%K060116nar10.1  karang inni     aapa  njaari       katha  arà-biilang
%K060116nar10.3  itthuka       aada hamma mlaayu kaaving  aapacare jaalang katha



\section{Agreement}\label{sec:cls:Agreement}
Although SLM does not show agreement on the verb as such, we are presently witnessing an incipient grammaticalization process which uses resumptive pronouns just before or after the verb if the anaphora is not immediately adjacent.\footnote{Whether this already counts as agreement is subject to debate and theoretical orientation \citep[cf.][99f.]{Corbett2006}.}
This might very well evolve into an agreement system.\footnote{The mechanism has been described by \citet{Givon1976tpga}.} As of now, however, there is no general agreement to be found in SLM.



\xbox{16}{
\ea\label{ex:cl:agr:spaaman}
\gll \textbf{spaaman} awuliya su-jaadi=\textbf{spaaman}. \\
 \textsc{3s.polite} saint \textsc{past}-become=3s.polite\\
`he became a saint.' (B060115nar05)
\z
} 

The source for this pattern is possibly Tamil, where the agreement suffix is very often very similar to the pronoun (\trs{n\textbf{iinga\dotl} poor\textbf{inga\dotl}}{You(pl.) are going.}) and etymologically related. However, some speakers use resumptive pronouns before the verb \xref{ex:cl:agr:incayang}-\xref{ex:cl:agr:derang}, not after the verb as Tamil influence would predict.


\xbox{16}{
\ea\label{ex:cl:agr:incayang}
\gll \textbf{Dr} \textbf{Draaman} duuwa thaaun=nang blaakang \textbf{incayang}=su-mniinggal. \\
`After two years, Dr Draman died.' (K051213nar08)
\z
}

\xbox{16}{
\ea\label{ex:cl:agr:deram}
\gll itthunam       \textbf{samma} \textbf{baae}  \textbf{oorang} \textbf{pada} inni     nigiri=ka=jo        \textbf{deram}=sa-duuduk. \\
`It is because of that that all good men settled down in this country.' (N060113nar01)
\z
}

\xbox{16}{
\ea\label{ex:cl:agr:derang}
\gll  suda \textbf{deram}  inni     {\em political} {\em promise}  hatthu \textbf{derang}=eng-kaasi     1958=ka. \\
       so \textsc{3pl} \textsc{prox} political promise \textsc{indef} \textsc{3pl}=\textsc{past}-give 1958=loc\\
\z      
}\\ 

As of now, we cannot speak of SLM having a generalized agreement system.
If in the future such an agreement system emerged, it is most likely that it would be A and S which would agree with the verb, so that we would deal with a nominative-accusative language, but this has not taken place yet.
