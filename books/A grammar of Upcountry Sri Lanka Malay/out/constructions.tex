\chapter{Beyond the clause}\label{sec:form:SuperclausalConstructions}
When morhpemes are combined into words, words into phrases and phrases into clauses, this is still not enough to produce coherent discourse. The relation between the different clauses must also be indicated. This is done in SLM with linking adverbials. Another construction, tail-head-linkage, also assure discourse continuity. Coordinating construction can be used to link elements on different levels. Finally, we will take a look at some of the idioms encountered in SLM.

\section{Linking adverbials}\label{sec:constr:Linkingadverbials}
SLM does not have a category of conjunctions. Instead, a number of adverbs or lexicalized combinations of deictic+postposition are used. The resulting structure is paratactic rather than hypotactic. 


\cb{ CLAUSE $\left\{\begin{array}{c}ADV\\DEIC+CLT\end{array}\right\}$ADV CLAUSE }

The following sentences show the use of the adverbial
\trs{suda}{thus}\xref{ex:beyond:link:suda}, or deictics+X
\trs{itthule}{but}\xref{ex:beyond:link:itthule},
\trs{ikang}{then}\xref{ex:beyond:link:ikang}.


\xbox{16}{
\ea\label{ex:beyond:link:suda}
\ea 
\gll se=ppe blaakang arà-dhaathang se=pe {\em fifth} {\em generation} se=ppe anak klaaki pada. \\ % bf
     \textsc{1s}=\textsc{poss} back \textsc{non.past}-come \textsc{1s}=\textsc{poss} fifth generation \textsc{1s}=\textsc{poss} child boy \textsc{pl}  \\
    `After me are coming the fifth generation, my sons.'   
\ex
\gll \textbf{suda} se=ppe thuuwa anak klaaki asadhaathang dhlapan-blas thaaun. \\
     thus \textsc{1s}=\textsc{poss} old child boy  \textsc{copula} eight-teen year \\
    `Thus, my eldest son is eighteen.'  (K060108nar02)
\z   
\z   
}\\ 


\xbox{16}{
\ea\label{ex:beyond:link:itthule}
\gll spaaru mlaayu pada arà-oomong \textbf{itthule} mulbar arà-oomong. \\
     some Malay \textsc{pl} \textsc{non.past}-speak but Tamil \textsc{non.past}-speak  \\
\z      
}\\ 


\xbox{16}{
\ea\label{ex:beyond:link:ikang}
\ea 
\gll Kandi=ka {\em Malay} {\em mosque}=pe blaakang=ka incayang=pe zihaarath aada. \\ % bf
     Kandi=\textsc{loc} Malay mosque=\textsc{poss} back=\textsc{loc} \textsc{3s.polite}=\textsc{poss} shrine exist  \\
\ex
\gll \textbf{ikang} derang=pe sudara pompang=jo aada Hanthane=ka. \\
     then \textsc{3s}=\textsc{poss} sibling sister=\textsc{foc} exist Hantane=\textsc{loc} \\
    
\z    
\z  
}\\ 

Both the adverbial strategy and the deictic+X strategy are found in the small fragment below.


\xbox{16}{
\ea\label{ex:beyond:link:double}
\ea 
\gll \textbf{itthule} kithang=pe muusing=ka=jo anà-baa. \\
      but \textsc{1pl}=\textsc{poss} time=\textsc{loc}=\textsc{foc} \textsc{past}-bring \\
\ex
\gll itthu kithang hathu wakthunang kithang=nang duppang dhaathang athi-aada. \\ % bf
     \textsc{dist} \textsc{1pl} \textsc{indef} time=\textsc{dat} \textsc{1pl}=\textsc{dat} before come \textsc{irr}-exist  \\
\ex
\gll {\em exiles} pada=jo anà-baa. \\ % bf
      exiles \textsc{pl}=\textsc{foc} \textsc{past}-bring \\
\ex
\gll \textbf{itthu=nang=jo} see anà-biilang. \\
      \textsc{dist}=\textsc{dat}=\textsc{foc} \textsc{1s} \textsc{past}-say \\
\ex
\gll oorang soojor anà-baa katha. \\ % bf
      man European \textsc{past}-bring \textsc{quot} \\
\ex
\gll iiya. \\ % bf
      yes \\
    `Yeah.'        
\ex
\gll \textbf{suda} itthu oorang pada=le Seelong=ka arà-duuduk. \\
     thus \textsc{dist} man \textsc{pl}=\textsc{addit} Ceylon=\textsc{loc} \textsc{non.past}-stay  \\
\z   
\z   
}\\ 

\section{Tail-head linkage}\label{sec:constr:Tail-headlinkage}
SLM makes extensive use of tail-head linkage. This means that the end of a clause is repeated at the beginning of the following clause. This is normally accomplished with a conjunctive participle. Example \xref{ex:beyond:tailhead1} shows this pattern over several clauses.


\xbox{16}{
\ea\label{ex:beyond:tailhead1}
\ea 
\gll asà-girijja incayang, su-\textbf{{\em pass}}$_i$. \\ % bf
 \textsc{cp}-do 3.polite \textsc{past}-pass\\
`Having done that, I passed (the exam).'
\ex
\gll asà-\textbf{{\em pass}}$_i$, kaarang,  {\em Advanced}  {\em Level} arà-\textbf{\em girijja}$_j$ \textbf{skaarang}. \\ % bf
cp-pass now . . \textsc{non.past}-make now\\
`Having passed the exam, then, I did the Advance Level then.'
\ex
\gll karang, {\em Advanced} {\em Level} asà-\textbf{girijja}$_j$, 1978=ka. \\ % bf
now . . \textsc{cp}-make, 1978=loc\\
`Then, having done the Advanced Level exames, in 1978,'
\ex
\gll {\em Late} {\em Exam} anthi-girijja. \\ % bf
late exam \textsc{irr}-make\\
`I would make a late exam.' (K051222nar08.4)
\z
\z
}\\

Example \xref{ex:beyond:tailhead2} shows another instance of this.

\xbox{16}{
\ea\label{ex:beyond:tailhead2}
\ea
\gll  mintha daaging=yang \textbf{cuuci}$_i$. \\ % bf
      raw beef=\textsc{acc} wash \\
    `Wash the raw beef.'
\ex
\gll asà-\textbf{cuuci}$_i$, laada=le gaaram=le bathu giling-an=ka  \textbf{giiling}$_j$. \\ % bf
      \textsc{cp}-wash pepper=\textsc{addit} salt=\textsc{addit} stone grind=\textsc{nmlzr} grind \\
    `Having washed it, grind salt and pepper in a grinding stone.'
\ex
\gll asà-\textbf{giiling}$_j$, daaging muntha=yang baathu=ka asà-thaaro, giccak. \\ % bf
     \textsc{cp}-grind beef raw=\textsc{addit} stone=\textsc{loc} \textsc{cp}-put smash  \\
\z
\z
} \\
\section{Coordinating constructions}\label{sec:constr:Coordinatingconstructions}
Coordination can be done on a variety of levels of linguistic analysis: words, phrases, clauses. In SLM, either zero-coordination or clitic coordination is used  for nearly all these cases.

\subsection{Unmarked coordination}\label{sec:constr:Unmarkedcoordination}
Two items can be simply juxtaposed to express coordination.

\cb{ X=\zero{} Y=\zero}

The coordinated items can either be  NPs, PPs, modifiers or clauses.

\subsubsection{Unmarked NP coordination}\label{sec:constr:UnmarkedNPcoordination}
Unmarked coordination of NPs can be found in PPs which contain more than one noun to which the postposition applies. In the following example, the postposition \em =dering \em applies to the three mentioned countries at the same time.

\xbox{16}{
\ea\label{ex:beyond:coord:NP:zero:postp}
\gll mlaayu pada anà-dhaathang  [[nigiri mlaayu]$_{NP}$  [{\em Malaysia}]$_{NP}$ [Indonesia]$_{NP}$ [{\em Singapore}]$_{NP}$=dering]$_{PP}$. \\ % bf
      Malay \textsc{pl} \textsc{past}-come country Malay Malaysia Indonesia Singapore=\textsc{abl} \\
\z
} \\ 

While in \xref{ex:beyond:coord:NP:zero:postp}, it was the postposition which indicated that the NPs are coordinated, in the following example, we find the countries of origin in  coordinated NPs without a postposition.

\xbox{16}{
\ea\label{ex:beyond:coord:NP:zero:zero}
\gll {\em British} {\em government} [{\em Malaysia}]$_{NP}$ [Indonesia]$_{NP}$ [ini nigiri pada]$_{NP}$ samma anà-peegang ambel. \\ % bf
     British government Malaysia Indonesia \textsc{prox} country \textsc{pl} all \textsc{past}-catch take  \\
    `The British government captured all these countries.' (K051213nar06)
\z
} \\
Human referents can also be coordinated by such a construction, as the following example shows.

\xbox{16}{
\ea\label{ex:beyond:coord:NP:zero:human}
\gll itthukapang [se=ppe baapa]$_{NP}$, [se=ppe kaake]$_{NP}$, [kaake=pe baapa]$_{NP}$, [kithang samma oorang]$_{NP}$ Seelong=pe oorang pada. \\ % bf
   then \textsc{1s}=\textsc{poss} father \textsc{1s}=\textsc{poss} grandfather grandfather=\textsc{poss} father \textsc{1pl}  all man  Ceylon=\textsc{poss} man \textsc{pl} \\
\z
}

The next example is very similar to the preceding one. An enumeration of NPs refering to different ethnic groups are strung together by unmarked coordination.

\xbox{16}{
\ea\label{ex:beyond:coord:NP:zero:groups}
\gll {\em school}=ka  [kithang]$_{NP}$  [mulbar aanak pada]$_{NP}$  [cinggala aanak pada]$_{NP}$  [sraani pada]$_{NP}$ [mlaayu pada]$_{NP}$,  samma  hatthu=nang=jo anà-duuduk. \\ % bf
     School=\textsc{loc} \textsc{1pl}  Tamil child \textsc{pl} Sinhala child \textsc{pl} Burgher  \textsc{pl} Malay \textsc{pl}  all one=\textsc{dat}=\textsc{foc} \textsc{past}-stay\\
\z
} \\

\subsubsection{Unmarked PP coordination}\label{sec:constr:UnmarkedPPcoordination}

Just like NPs, PPs can be strung together by unmarked coordination.

\xbox{16}{
\ea\label{ex:beyond:coord:PP:zero}
   \gll  kamauwan wakthu=nang,   kithang=nang   [itthu    mosthor=nang]$_{NP}$,   [{\em Malaysian} hathu  mosthor=nang]$_{NP}$   kithang=nang   bole=duuduk. \\ % bf
     want time=\textsc{dat} \textsc{1pl}=\textsc{dat} \textsc{dist} manner=\textsc{dat} Malaysian \textsc{indef} manner=\textsc{dat} \textsc{1pl} can-exist.\textsc{anim} \\
\z
}

\subsubsection{Unmarked predicate coordination}\label{sec:constr:Unmarkedpredicatecoordination}
It might be the case that there is unmarked predicated coordination in SLM, but this cannot be established.
Given the possibility to drop referents retrievable for the hearer, it is impossible to decide whether there is unmarked predicate coordination or whether we are rather dealing with clause coordination where all arguments of the second clause are not overtly realized. This unclear status is indicated by \zero{} between parentheses. The question whether the element is there, but not overtly realized (clause coordination), or whether the element is not there (predicate coordination) cannot be decided.\footnote{\citet[206]{Karunatillake2004} explicitely states for the related Sinhala construction: ``Verbs that go in pairs like [read/write] can be coordinated by simple juxtaposition.''}
 
\xbox{16}{
\ea\label{ex:beyond:coord:pred}  
\gll kithang=nang baaye=nang mulbar bole=baaca (\zero$_{agent}$) (\zero$_{theme}$) bole=thuulis. \\ % bf
      \textsc{1pl}=\textsc{dat} good=\textsc{dat} Tamil  can=read { } { } can=write \\
    `We can read and write Tamil well.'  (K051222nar06)
\z      
}\\

This analytical problem  arises again in \xref{ex:beyond:coord:pred2}, where it cannot be decided whether the assertions are coordinated, or whether we are dealing with clause coordination with dropped agent.


\xbox{16}{
\ea\label{ex:beyond:coord:pred2}
\gll  kithang lorang=nang baaye mliiga athi-kaasi, (\zero) mà-kaaving panthas pompang pada athi-kaasi,  (\zero) duwith athi-kaasi. \\ % bf
      \textsc{1pl} \textsc{2pl}=\textsc{dat} good palace \textsc{irr}-give { } \textsc{inf}-marry beautiful female \textsc{pl} \textsc{irr}-give { }  money \textsc{irr}-give \\
\z
} \\

\subsubsection{Unmarked modifier coordination}\label{sec:constr:Unmarkedmodifiercoordination}
While unmarked coordination of NPs and PPs is common, unmarked coordination of modifiers is found less often. Example \xref{ex:beyond:coord:mod1}  restricts the set of referents for \trs{kuwathan}{strenght} to the strength of the arms and the legs, by unmarked coordination.

\xbox{16}{
\ea\label{ex:beyond:coord:mod1}  
\gll kaaki$_{mod}$ thaangan$_{mod}$ kuwathan$_{head}$ thraada. \\
      foot hand strong-\textsc{nmlzr} nexist \\
    `You have no strength in your arms or legs.' (K060116sng01)
\z
} \\ 
Example \xref{ex:beyond:coord:mod2} coordinates two towns of origin which restrict the reference of \trs{oorang}{man} to people hailing from these towns. The two towns do not carry an overt mark of coordination.

\xbox{16}{
\ea\label{ex:beyond:coord:mod2} 
\gll  thraa, kithang samma {\em Kandy}$_{mod}$ Nawalapitiya$_{mod}$ oorang$_{head}$. \\
       \textsc{neg} \textsc{1pl} all Kandy Nawalapitiya man\\
\z
} \\

A complex modification is found in \xref{ex:beyond:coord:mod3}, where the coordination is actually not of the same type as above, since it is a joint Sinhala-Tamil problem, and not a problem which the Sinhalese and the Tamils happen to share. This contrasts with \xref{ex:beyond:coord:mod2} , where the referents comprise all people with origin Kandy and all people with origin Nawalapitiya. \xref{ex:beyond:coord:mod3} does not refer to the conjuction of all Sinhala problems and all Tamil problems, but it refers to precisely the one problem which is characterized by Sinhalese and Tamils playing a role in it.

\xbox{16}{
\ea\label{ex:beyond:coord:mod3}
\gll duwa {\em week} didaalang cinggala$_{mod}$ mulbar$_{mod}$ reepoth$_{head}$. \\
     two week within Sinhala Tamil problem  \\
    `Within two weeks, there was the Sinhala-Tamil problem.' (K051213nar01)
\z
}

\subsubsection{Unmarked clause coordination}\label{sec:constr:Unmarkedclausecoordination}
A chain of indepedendent clauses can be formed without any material indicating coordonation. In \xref{ex:beyond:coord:clause:zero}, four existential clauses are joined without overt indication of coordination.


\xbox{16}{
\ea\label{ex:beyond:coord:clause:zero}
\gll  [cinggala   aada]$_{CLAUSE1}$ [mlaayu aada]$_{CLAUSE2}$ [{\em Moor}  aada]$_{CLAUSE3}$ [mulbar aada]$_{CLAUSE4}$. \\ % bf
 Sinhala exist Malay exist Moor exist Tamil exist\\
`There are Sinhalese, there are Malays, there are Moors, there are Hindus.' (G051222nar04)
\z
}

Five adjectival predicate clauses are conjoined without overt marking in \xref{ex:beyond:coord:caluse:zero:adj}.

\xbox{16}{
\ea\label{ex:beyond:coord:caluse:zero:adj}
\ea 
\gll  kithang=pe nigiri=ka bedahan aada.\\ % bf
      \textsc{1pl}=\textsc{poss} country=\textsc{loc} difference exist \\
    `There are differences in our country.'
\ex
\gll  [Mulbar laayeng]$_{CLAUSE1}$ [cinggala laayeng]$_{CLAUSE1}$ [{\em Moor} laayeng]$_{CLAUSE1}$ [mlaayu laayeng]$_{CLAUSE1}$ [sraani pada laayeng]$_{CLAUSE1}$. \\  % bf
     Tamil different Sinhala different Moor different Malay different burgher \textsc{pl} different  \\
    ` Hindus, Sinhalese, Moors, Malays, Burghers, all are different.' (K061127nar03)
\z
\z
} \\
\subsection{Coordination with clitics}\label{sec:constr:CoordinationwithBooleanclitics}
SLM makes use of coordinating clitics \formref{sec:morph:CoordinatingClitics} for all purposes of coordination. These attach to all of the coordinated items, which can be arguments, adjuncts or predicates. Coordination    of clauses with this construction  seems to be possible, but is very seldom found. This is expressed by parentheses in \xref{cb:beyond:coord}.

The use of an `A-and B-and structure' \citep{Haspelmath2004coord} is typically South Asian.


\cb[\label{cb:beyond:coord}]{
$\left\{
\begin{array}{l}
\rm NP\\\rm PRED\\\rm ADJ\\\rm (CLAUSE)
\end{array}
\right\}$
=COORDINATING CLITIC
$\left\{
\begin{array}{l}
\rm NP\\\rm PRED\\\rm ADJ\\\rm (CLAUSE)
\end{array}
\right\}$
=COORDINATING CLITIC
}

\subsubsection{Argument coordination}\label{sec:constr:Argumentcoordination}
Argument coordination is a regular instance of coordination. The following examples show the use of conjunctive and disjunctive coordination of arguments.

%\xbox{16}{
%\ea\label{ex:beyond:unreferenced}
%\gll se=ppe umma\textbf{=le}  aade\textbf{=le} aade=pe duuwa aanak\textbf{=le} kitham=pe baapa=pe aade\textbf{=le} baapape aade\textbf{=le} aanak\textbf{=le} thiiga aanak\textbf{=le}, bannyak aanak pada anà-niinggal. \\
%1s=\textsc{poss} mother=\textsc{addit} younger.sibling=\textsc{addit} younger.sibling=\textsc{poss} two child=\textsc{addit} \textsc{1pl}=\textsc{poss} father=\textsc{poss} younger.sibling=\textsc{addit} father=\textsc{poss} younger.sibling=\textsc{addit} child=\textsc{addit} three child=\textsc{addit} many child \textsc{pl} \textsc{past}-die\\
%\z
%}

\xref{ex:beyond:coord:bool:NP} shows the use of overt coordination on NPs unmarked for case, whereas \xref{ex:beyond:coord:bool:PP} shows the use of overt coordination on PPs.

\xbox{16}{
 \ea\label{ex:beyond:coord:bool:NP}
\ea
   \gll  thiiga klaaki aade=\textbf{le}  hatthu pompang aade=\textbf{le}   se=dang arà-duuduk. \\
    three male younger.sibling=\textsc{addit} one female younger.sibling \textsc{1s=dat} \textsc{non.past}-exist.\textsc{anim}\\
\z
\z
}


\xbox{16}{
 \ea\label{ex:beyond:coord:bool:PP}
\gll Snow-white=nang\textbf{=le} Rose-red=nang\textbf{=le} ini hatthu=ke thàrà-mirthi. \\
     Snow.white=\textsc{dat}=\textsc{addit} Rose.Red=\textsc{dat}=\textsc{addit}  \textsc{prox} \textsc{indef}=\textsc{simil} \textsc{neg.past}-understand\\
\z      
}\\

An utterance with two instances of coordination is \xref{ex:beyond:coord:bool:double}.

\xbox{16}{
 \ea\label{ex:beyond:coord:bool:double}
\ea
\gll see anà-blaajar \textbf{mulbar=le} \textbf{{\em English}=le} itthu muusing. \\
    `I studied in Tamil and English back then.'  
\ex
\gll derang karang arà-blaajar \textbf{cinggala=le} \textbf{{\em English}=le}. \\
    `They study in Sinhala and English now.' (K051222nar06)
\z
\z
} \\
Disjunctive coordination of arguments is given in \xref{ex:beyond:coord:bool:disj}.

\xbox{16}{
\ea\label{ex:beyond:coord:bool:disj}
\gll ketham pada     makanan pada\textbf{=si}    pakeyan   pada\textbf{=si } su-baawang. \\
 \textsc{1pl} \textsc{pl} food \textsc{pl}=disj clothing \textsc{pl}=disj \textsc{past}-bring\\
\z
}

%Snow-white Aanak raaja yang le , Rose-red incayang pe sudaara yangle su kaaving (K070000wrt04)


\subsubsection{Adjunct coordination}\label{sec:constr:Adjunctcoordination}
The distinction between arguments and adjuncts is not clear-cut in SLM \formref{sec:argstr}, but if such a distinction is assumed, example \xref{ex:beyond:coord:adjct} shows that  local adjuncts can be overtly coordonated.

\xbox{16}{
\ea\label{ex:beyond:coord:adjct}
\gll Matale=ka\textbf{=le}  Kluumbu=ka\textbf{=le} su-aada. \\ % bf
     Matale=\textsc{loc}=\textsc{addit} Colombo=\textsc{loc}=\textsc{addit} \textsc{past}-exist  \\
\z      
}\\ 


\subsubsection{Predicate coordination}\label{sec:constr:Predicatecoordination}
Predicate coordination is normally not done by clitics, but rather by the anterior or simultaneous conjunctive participles \formref{sec:morph:asa-}\formref{sec:wofo:Verbalreduplication}.

\subsubsection{Modifier coordination}\label{sec:constr:Modifiercoordination}
There is no instance of overt modifier coordination.

\subsubsection{Clause coordination}\label{sec:constr:Clausecoordination}
Clause coordination differs from the other kinds of coordination in that the coordinating clitic does not attach to the right edge of the items, but rather to the right edge of one of the arguments of the clause. In example \xref{ex:coord:clause:muuluth}, it is clear that it is not the mouths which are coordinated, but the clauses containing the predications since the predicate \trs{paasir}{sand} is present twice. Still, the coordinating clitic \em =le \em does not attach to the predicate, but rather to \em muuluthka \em in both of the clauses. There is thus a mismatch between semantic scope (clause) and morphological locus (PP).

\xbox{16}{
\ea\label{ex:coord:clause:muuluth}
\gll lu=ppe \textbf{muuluth=ka=le} paasir, se=ppe \textbf{muuluth=ka=le} paasir. \\
      \textsc{2s.familiar}=\textsc{poss} mouth=\textsc{loc}=\textsc{addit} sand \textsc{1s}=\textsc{poss} mouth=\textsc{loc}=\textsc{addit} sand  \\
    `There is sand in your mouth and there is sand in my mouth.' (K070000wrt02)
\z      
}\\ 

A similar observation can be made about \xref{ex:coord:clause:banthu}, where the coordination concerns two acts of helping, one towards the Sinhalese king and one towards the British. But the coordinating clitics do not attach to the predicate \em mà-fight=nang \em or  the right edges of the clauses (\em saama\em), but to the argument expressing the beneficiary. True, the crucial difference between the two coordinands is the identity of the beneficiary, but these beficiaries participate in different propositions. It is not about helping (A and B) to do X, but rather about helping A to do X and B to do Y. Again, the morphological marking of coordination does not reflect its semantic scope.

\xbox{16}{
\ea\label{ex:coord:clause:banthu}
\ea
\gll cinggala  raaja=nang=\textbf{le}  na-banthu. \\
     Sinhala king=\textsc{dat}=\textsc{addit} \textsc{past}-help  \\
\ex
\gll mà-{\em fight}=nang       {\em British}=saama. \\ % bf
     \textsc{inf}-fight=\textsc{dat} British=with \\
    `to fight with the British.'
\ex
\gll {\em British} oorang pada=nang=\textbf{le} na-banthu. \\
     British man \textsc{pl}=\textsc{dat}=\textsc{addit}  \\
    `and they helped the British men'
\ex
\gll      mà-{\em fight}=nang cinggala  raaja=saama. \\ % bf
      \textsc{inf}-fight=\textsc{dat} Sinhala king=with \\
    `to fight with the Sinhala king.' (K051206nar04)
\z
\z
} \\

There seems to be a tendency to avoid coordinating clitics on predicates,\footnote{Also found in Sinhala,   (\citet[46]{GairEtAl1997}, \citet[808]{Gair2003}) and Tamil  \citep[162]{Lehmann1989}, but see \citet[70]{Karunatillake2004}.} as exemplified above. this tendency, however, is not universal, as \xref{ex:beyond:coord:bool:pred:contr} shows, where the two predicates \trs{niinggal}{die} both carry the additive clitic \em =le\em.

\xbox{16}{
 \ea\label{ex:beyond:coord:bool:pred:contr}
   \gll  ithukang       bini-yang       \textbf{niinggal=le}       umma-yang        \textbf{niinggal=le}  aanak pada    caari  dhaathang. \\
    Then wife-hyperch die=\textsc{addit} mother-hyperch die=\textsc{addit} child \textsc{pl} search come \\
`So then the wife died and the mother died and the children came to look' (K051220nar01)
\z
}


% 
% \xbox{16}{
% \ea\label{ex:beyond:unreferenced}
% \ea\label{ex:beyond:unreferenced}
% \gll see=le nya-laaher badulla. \\
%      \textsc{1s}=\textsc{addit} \textsc{past}-be.born Badulla  \\
% \ex
% \gll see nya-blaajar=le badulla=ka. \\
%      \textsc{1s} \textsc{past}-learn=\textsc{addit} Badulla=\textsc{loc}  \\
% \z
% \z
% } \\
% Note, however, that we are not dealing with an \em X=le Y=le \em structure here, where \em =le \em would be translated as `and', but with a use of \em =le \em only once per clause, which corresponds to English `also, too, as well'.

 

 
%  
% \xbox{16}{
% \ea\label{ex:beyond:unreferenced}
% \ea\label{ex:beyond:unreferenced}
% \gll se=ppe {\em family}=ka duuwa aanak pompang arà-duuduk. \\
%       \textsc{1s}=\textsc{poss} family=\textsc{loc} two child girl \textsc{non.past}-exist.\textsc{anim} \\
% \ex
% \gll derang duwa oorang=le asà-kaaving. \\
%       \textsc{3pl} two man=\textsc{addit} \textsc{cp}-marry \\
% \ex
% \gll derang=nang=le aanak pada aada. \\
%      \textsc{3pl}=\textsc{dat}=\textsc{addit} child \textsc{pl} exist  \\
%     `and have children.' (B060115prs01)
% \z
% \z
%} \\
\section{Idioms}\label{sec:constr:Idioms}
This section contains SLM idioms, constructions whose semantic content has no direct relation to the semantic content of their components.

\subsection{mà-V caape=work hard}\label{sec:constr:mà-Vcaape=workhard}

\xbox{16}{
\ea 
\gll  {\em freedom}=yang   mà-daapath=nang  kithang=nang   bannyak \textbf{caape} aada. \\
      freedomg=\textsc{acc} \textsc{inf}-get=\textsc{dat} \textsc{1pl}=\textsc{dat} much tired exist \\
\z
} \\

\xbox{16}{
\ea 
\gll  giini    duuduk     bannyak  kithang=pe     oorang pada=nang  anà-\textbf{caape}. \\
      like.this stay much \textsc{1pl}=\textsc{poss} man \textsc{pl}=\textsc{dat} \textsc{past}-tired \\
\z
} \\ 

\subsection{luppas thaangang = give up}\label{sec:constr:luppasthaangang=giveup}
\xbox{16}{
\ea 
\gll
 thumpath, gaaga saawa,  inni samma=yang asà-luppas thaangang\\
place land field \textsc{prox} all=\textsc{acc} \textsc{cp}-leave hand\\
\z
}


\subsection{NUM gìnnap = be in the NUMth year on earth}\label{sec:constr:NUMginnap}

\xbox{16}{
\ea
\gll  lima-pulu thaaun arà-gìnnap. \\
      five-ty year \textsc{non.past}-complete \\
    `I am completing my fiftieth year (=I am 49).' (K060108nar01)
\z
} \\ 

\subsection{saakith athi = personal difference}\label{sec:constr:saakithathi=personaldifference}
K060116nar04.trs:ithuka sakith aathi asa jaadi aada
 


\subsection{mapi madhaathang=return trip}\label{sec:constr:mapimadhaathang=returntrip}
%  two mile duuwa kaayu kithang masa pii  ithu kithang=pe muusing kithang jaalang duuwa kaayu  during our time we were walking two mile up and down  four miles


\xbox{16}{
\ea
\gll mà-pii mà-dhaathang empath kaayu. \\
     \textsc{inf}-go \textsc{inf}-come four mile  \\
    `The round trip was four miles then.' (K051213nar03)
\z
} \\

\subsection{kaasi thaangang = lend a helping hand}\label{sec:constr:kaasithaangang=lendahelpinghand}
\xbox{16}{
\ea 
\ea 
\gll luu=le asà-dhaathang kanaapa nya-laaher. \\
     \textsc{2s.familiar}=\textsc{addit} \textsc{cp}-come why \textsc{past}-be.born  \\
\gll luu abbis bìssar lu=ppe umma-baapa=nang \textbf{kaasi} \textbf{thaangang}. \\
    `When you will have finished growing up, lend a hand to your parents.' (K060116sng01)
\z
\z
} \\
\subsection{ambel thriima}\label{sec:constr:ambelthriima}
\xbox{16}{
 \ea 
   \gll  itthu=pe        waanging=yang   ini      luwar   nigiri  oorang pada       baaye  suuka=nang   anà-\textbf{ambel}    \textbf{thriima}. \\
    \textsc{dist}=\textsc{poss} fragrance=\textsc{acc} \textsc{prox} outside country man \textsc{pl}  good like=\textsc{dat} \textsc{past}-take delight \\
\z
}


\subsection{aari mapii mpii= as days went by}\label{sec:constr:aarimapiimpii=asdayswentby}
\xbox{16}{
\ea 
\gll  itthu=nang      blaakang karang \textbf{aari} \textbf{mà-pii}    \textbf{m-pii}      karang kitham=pe      mlaayu arà-mulain. \\
    `As days went by, our Malays developed.' (K051222nar03)
\z
} \\

\xbox{16}{
\ea 
\gll inni     mlaayu pada aari \textbf{mà-pii}   \textbf{mà-pii}  \el{}  inni     samma=yang   asà-luppas   thaangang. \\
     \textsc{prox} Malay \textsc{pl} day \textsc{inf}-go \textsc{inf}-go { } \textsc{prox} all=\textsc{acc} \textsc{cp}-leave hand  \\
\z
} \\ 
\subsection{kaasi bìrrath}\label{sec:constr:kaasibirrath}

% \xbox{16}{
% \ea
% \gll  seeyang {\em university} ka su ambel bìrrath. \\
%        \\
%     `.' (nosource)6.11.08
% \z
% } \\

%
%Yazmin
%V F supuukul
%* F sukenna
%V A Fnang supuukul
%* A Fyang sukenna
%V A Fnang sukennakang
%V Fnang supukul
%V Fnang sukenna
%* F askenna supukul
%V F pukulan sukenna
%V F Adheri pukulan sukenna
%V r s nang mapinang duppand F pukulan suambel
%V F pukulan askenna apa rs nang supi
%V pukulan sekennanang blaakang F rsna supi
%* pukulanna blaakang A Fyang rsna subaa
%V pukulan kennanam blaakang A Fy rsna subaa






%karang itu musing=ka liiwat ruuma pada aada
%karanga itu musing liiwat sampi pada aada
%duppangnang sampi pada attiliiwat
%aanak pada atti liiwat
%aanak pada tama liiwat
%aanak pada liiwat bikkang
%    no increase
%aanak pada tra liiwat
%    no more children
%aanak pada are=liiwat
%    number is going up
%se iniskalli pugiske=llinang liiwat duwit se=caari 
