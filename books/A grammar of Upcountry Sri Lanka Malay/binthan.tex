\section*{Binthan Muumin's story}
This text is about the family of the speaker and the very recent history of the Malays of Peradeniya Road in Kandy. In former times, there were many Malays in this area, but as it develops commercially, the resident Malays sell their property to buy cheaper places in rural areas. This text is given in phonetic transcription with rough indications of intonation, next to the normal orthographic rendering. Especially interesting are the different realizations of schwa, as well as the reduction of some other vowels. In the consonantal domain, the variation in geminated palatal stops can also be seen. Dots indicate parts that are not transcribed, normally the researcher asking a question in order to keep the narrative going.

\let\eachwordone=\sc \let\eachwordtwo=\tipaencoding \let\eachwordthree=\it

\ea \label{K051222nar04.6}
\gllll  ~  h\hspace{1.3cm}l\\
 ki\dentt aN kEn\postalvd ika\J o\\
 kithang  Kandy=ka=jo\\
 \textsc{1pl} Kandy=\textsc{loc}=foc\\
 `We are from Kandy.'
\z


\ea \label{K051222nar04.7}
\gllll ~~~~~~h  \hspace{2cm}l\\
 sEp:E Um:a:ba:pa\\
 se=ppe umma-baapa\\
 \textsc{1s=poss} mother-father\\
 `As for my parents,'
\z


\ea \label{K051222nar04.8}
\gllll ~ ~~~~~~h ~\hspace{1.2cm}l ~h\hspace{1.1cm}l ~h\hspace{0.8cm}l \\
 sEppe ba:pa n:ig2mbO, nig2:mbo gu:mbu\\
 se=ppe baapa {\em Negombo} {\em Negombo} Guu\u mbu \\
 \textsc{1s=poss} father Negombo Negombo Negombo \\
 `my father is from Negombo.'
\z

[...]

\ea \label{K051222nar04.13}
\gllll l~~~~h ~~\hspace{.7cm}l\\
 \s mma k\super han\postalvd i\\
 umma Kandi\\
 mother Kandy\\
`My mother is from Kandy.'
\z


\ea \label{K051222nar04.14}
\gllll \hspace{1.2cm}h ~ ~ ~~~l\hspace{1cm}h ~\hspace{1cm}l\\
  aska:\V IN ba:pa pIkI\J an anag@RI\texttoptiebar{\J\textctz}a p@li:ska \\
 asà-kaaving baapa pukurjan anà-kijja {\em police}=ka\\
 \textsc{cp}-marry father work \textsc{past}-make police=loc\\
 `After having married, father worked in the police.'
\z

\xbox{16}{
\ea \label{K051222nar04.15}
\gllll \hspace{1cm}h \hspace{1cm}l \hspace{1cm}h \hspace{0.9cm}lh ~ h   ~~~~~l  ~~~\hspace{.4cm}h~~~~~l\\
 pOli:ska p@kU\J an asgI\texttoptiebar{\J\textctz}a s:i:ni\J o: asa\dz u:\dz u a:nap pa\dz a s5bRa:naP\\
 {\em police}=ka pukurjan asà-kijja siini=jo asà-duuduk aanak pada su-braanak\\
 police=\textsc{loc} work \textsc{cp}-make here=\textsc{emph} \textsc{cp}-stay  child \textsc{pl} \textsc{past}-birth\\
`Having worked in the police and having come to live right here, the children were born.'
\z
}

\ea \label{K051222nar04.17a}
\gllll ~ ~      ~~~~~~h~~~~l  \\
 a:nak pa:\dz a @sb8s:aR.  \\
 aanak pada asà-bìssar\footnotemark  \\
 child  \textsc{pl} \textsc{cp}-big     \\
`The children grew up.'
\z

\footnotetext{In this sentence, the conjunctive participle \em asà- \em is used in a main clause, as indicated by the intonation. This sentence and the following thus do \em not \em form a clause chain. While the use of \em asà- \em is possible in main clauses, it is not very common; the use in cosubordinate clauses is more common.}

\ea \label{K051222nar04.17b}
\gllll  ~~~~h     \hspace{.8cm}l \\
   s:kulnaN an@pi:\\
   skuul=nang anà-pi\\
     school=\textsc{dat} \textsc{past}-go\\
`They went to school.'
\z

\ea \label{K051222nar04.19a}
\gllll ~~~h \hspace{.9cm}l  \hspace{1cm}h \\
skulnan an@pi:nam bla:k5N  \\
skuul=nang anà-pi=nang blaakang  \\
school=\textsc{dat} \textsc{past}-go=\textsc{dat} after  \\
`After having gone to school'
\z

\ea \label{K051222nar04.19b}
\gllll ~ ~              ~   ~     ~      ~h                ~ ~    \hspace{1.4cm}l\\
 ka:RaN ki\dentt an:5 in:i n:i nigIRi su\textrhoticity{} su suuk su-bija:sa\\
 karang kithang inni inni  nigiri ... ... ...	  su-byaasa\\
now 	\textsc{1pl} 	\textsc{dem.prox}  \textsc{dem.prox} country ... ... ... \textsc{past}-accustomed\\
`we have got accustomed to this country now.'
\z


\ea \label{K051222nar04.20}
\gllll ~h  ~        ~          ~    \hspace{1.4cm}l\\
 lu\V aR nigiRi ki\dentt an:@ m@pi: \dentt@R5su:ka\\
 luvar nigiri kithang=nang mà-pi thàrà-suuka\\
 outside country \textsc{1pl}=\textsc{dat} \textsc{non.past}-go \textsc{neg}-like\\
`We do  not want to go abroad.'
\z


\ea \label{K051222nar04.21}
\gllll ~ ~   ~h            ~              ~    ~      ~          ~~~~h~~~~~~~l\\
 kaR\~5 in:i nigiRika\J O ki\dentt ampE a:nak bu:\V a pa\dz ajaN as:umpaN \\
 karang inni nigiri=ka=jo kithang=pe aanak buuva\footnotemark{} pada=yang asà-simpang \\
 now \textsc{dem.prox} country=\textsc{loc}=\textsc{emph} \textsc{1pl}=\textsc{poss} child fruit  \textsc{pl}=\textsc{acc} \textsc{cp}-keep\\
`It is in this country that we raised our children'
\z

\footnotetext{\em aanak buuva \em is an idiom to refer to one's children.}



\ea \label{K051222nar04.22}
\gllll ~h ~  ~         ~l~~h~~lh\\
in:i sku:ls pa\dz 5naN aski:RiN\\
 inni skuuls pada=nang  asà-kiiring\\
 \textsc{dem.prox} schools \textsc{pl}=\textsc{dat}  \textsc{cp}-send\\
`Having send them to these schools '
\z


\ea \label{K051222nar04.23}
\gllll ~h \hspace{1.2cm}lh  ~h     \hspace{1cm}l~h ~h ~           l~~~~~h ~    \hspace{1cm}l ~~~~~~~~h\hspace{0.5cm}l \\
 sam:a asg@RI\texttoptiebar{\J\textctz}a kaRan sUbla:\J aR p@kU\J an asg@RI\J a ambE ska:Ran si:ni\J o:  aR@\dz u:\dz uP\\
 samma asà-kijja karang asà-blaajar pukurjan asà-kijja ambel skaarang siini=jo arà-duuduk\\
 all \textsc{cp}-make now \textsc{cp}-learn work \textsc{cp}-make take now    here=\textsc{emph} \textsc{non.past}-stay\\
`Having done all that now, having learned, having taken up work, we are now living right here.'\footnotemark
\z

\footnotetext{The agents of the different events (working, learning, living) change reference. Sometimes the agent includes the speaker, sometimes not. The change of participants between elements of a clause chain is normal, see \formref{sec:cls:Conjunctiveparticipleclause}. This is especially clear from the events of `sending to school' \xref{K051222nar04.22} and `learning' \xref{K051222nar04.23}, which clearly have different agents. The person who sends is not the one who is learning, it is the person being sent who learns.}

\ea \label{K051222nar04.24}
\gllll ~ ~  ~             ~h     \hspace{1cm}l\\
 kaRaN in:i ki\dentt ampE nigiRi su\J a:\postalvd i\\
 karang inni kithang=pe nigiri su-jaadi\\
 now \textsc{dem.prox} \textsc{1pl}=\textsc{poss} country \textsc{past}-become\\
`Now this has become our country.'
\z


\ea \label{K051222nar04.25}
\gllll \hspace{1.1cm}h ~   \hspace{1.8cm}h ~         ~      \hspace{1.2cm}l \hspace{.6cm}h ~  \hspace{1.1cm}l           \\
 se:l\~oNlE ki\dentt ampE mla:junal:E ha\dentt u pagijan an5a:\dz a ha\dentt u nigiRi  su\J a:\postalvd iP\\
 Seelon=le [[kithang=pe mlaayu=nang=le hatthu bagiyan anà-aada] hatthu nigiri]\footnotemark{} su-jaadi\\
 Ceylon=\textsc{addit}  \textsc{1pl}=\textsc{poss} Malay=\textsc{dat}=\textsc{addit} \textsc{indef} part exist \textsc{indef} country here=\textsc{emph} \textsc{non.past}-stay\\
`Ceylon also became a country where our Malays also have a part in.'
\z
\footnotetext{Note the preposed relative clause, which is separated from the head word by the indefiniteness marker \em hatthu\em.}


% \ea \label{K051222nar04.26}
% \gllll \\
%  aada Guu\u mbuka\\
%  aada Guu\u mbu=ka\\
%  exist Negombo=loc\\
% 
% \z

\el{}

\ea \label{K051222nar04.27}
\gllll \hspace{0.7cm}h ~      \hspace{0.7cm}l     ~h         ~                    ~   ~~~~l\\
 sam:a \dentt umpa\dentt{}  sE:lON, sam:a \dentt umpa\dentt{} \s mla:ju a:\dz a\\
 samma thumpath Seelon samma thumpath mlaayu aada\\
 all place Ceylon all place Malay exist\\
`All over Ceylon there are Malays.'
\z


\ea \label{K051222nar04.28}
\gllll \hspace{1cm}h l     ~              \hspace{1.2cm}h ~  ~            \hspace{1cm}lh l\hspace{1cm}h ~   \hspace{0.9cm}l \\
 k5\dentt a:ma ki\dentt aN ki\texttoptiebar{c\textctc}Il mu:siNka ini pe:Ra\dentd enije  \J a:laNka sama na:\dz a \s mla:ju\\
 kàthaama kithang kiccil muusing=ka ini Peradeniya jaalang=ka samma anà-aada mlaayu\\
 before \textsc{1pl} small period=\textsc{loc} \textsc{dem.prox} Peradeniya street=\textsc{loc} all \textsc{past}-exist Malay\\
`In former times, when we were small, all the people who lived on the Peradeniya Road were Malays.'
\z


\ea \label{K051222nar04.29}
\gllll ~~~~h  ~~~~l~~h l\hspace{0.5cm}h \hspace{0.9cm}l\\
 ba:\~e \s mla:ju ba\dentt ai \s mla:ju \\
 baae mlaayu bathai mlaayu\\
 good Malay courageous Malay\\
`Good Malays, brave Malays.'
\z


\ea \label{K051222nar04.30}
\gllll \hspace{0.7cm}h ~ \hspace{0.6cm}l \\
 ba\super Q:e \s mla:ju pa\dz 5\\
 bae mlaayu pada\\
 good Malay \textsc{pl} \\
`Good Malays.'
\z


\ea \label{K051222nar04.31}
\gllll ~ \hspace{0.9cm}h ~ lhl ~~~~~h~l ~~~~~h~l\\
 n:i \s mla:ju pa\dz 5 \~aRi m@pi: m@pi: \\
 inni mlaayu pada aari mà-pii mà-pii  \\
 \textsc{dem.prox} Malay \textsc{pl} day \textsc{inf}-go \textsc{inf}-go \\
`As days passed by the Malays'
\z

[...]

\ea \label{K051222nar04.34a}
\gllll h\hspace{1cm}l ~h\hspace{1.2cm}l ~ ~h\hspace{1.2cm}l  ~h\hspace{0.4cm}l ~h\hspace{0.2cm}l ~h\hspace{0.4cm}l ~h\hspace{0.2cm}l  ~ ~~~h\hspace{0.8cm}l \\
 \dentt umpa\dentd{}  gagasa:\V\~a -- gagasa:\V a, ga:ga, \rm  land, sa:\V a, \rm fields,   \rm their \rm property -- \\
 thumpath  gaga-saa\V a -- gaga-saa\V a, ga:ga, land, sa:\V a, fields,  their property --  \\
 place  land-field -- land-field land land field  field their property\footnotemark{}	 -- \\
`places, fields and lands, their property'
\z 

\footnotetext{This is some metalinguistic commentary on the word used.}

\ea \label{K051222nar04.34b}
\gllll ~ \hspace{0.5cm}h ~ \hspace{0.8cm}l  \\
 in:i sam:ajaN aslUp:as \dentt a:NaN\\
  inni samma=yang asà-luppas thaangang\footnotemark\\
  \textsc{dem.prox} all=\textsc{acc} \textsc{cp}-leave hand/arm\\
`gave up all that.'
\z
\footnotetext{\em luppas thaangang \em is an idiom meaning `to give up/to let go'.}

\ea \label{K051222nar04.36}
\gllll \hspace{0.7cm}h ~ ~~~~~l\\
 \V u\postalvd ik sub@lan\~5 s@pi:\\
 uudik subàla=nang su-pii\\
 countryside side=\textsc{dat} \textsc{past}-go\\
`and went to the village areas.'
\z


\ea \label{K051222nar04.37}
\gllll ~ ~h \hspace{0.7cm}l\\
 in:i s\V a:Ra li:\V a\dentt\\
 inni svaara liivath\\
 \textsc{dem.prox} noise more\\
`The noise here is too much.'
\z


\ea \label{K051222nar04.38}
\gllll \hspace{1.4cm}h   ~         ~~~~~~l ~~~h            \hspace{1.1cm}l ~h~~~~l  ~h ~ \hspace{1.6cm}l \\
  i\dentt:usUb:a\dentt{} \dz@\rz am pada \texttoptiebar{\J\textctz}a:laN aR@gI\J:a bu\dentt:ul Ru:ma padajaN aR@picak\~aN\\
 itthusubbath derang pada jaalang arà-kijja butthul ruuma pada=yang arà-picca-king\\
 therefore \textsc{3pl} \textsc{pl} street \textsc{non.past}-make correct house \textsc{pl}=\textsc{acc} \textsc{non.past}-broken-\textsc{caus}\\
`Therefore, they are repairing the roads, they are demolishing the houses.'
\z


\ea \label{K051222nar04.39}
\gllll ~h~~~~l ~~~~~~h~~~~l \hspace{1.3cm}h ~      ~                    \hspace{1.1cm}l \\
 Ru:ma aR@ki\texttoptiebar{c\textctc}il i\dentt usub:a\dentt{} ini \dentt\~umpa\dentt{} pa\dz 5j5m pa:saR pa\dz@naN ...\\
 ruuma arà-kiccil; itthusubbath inni thumpath pada=yang paasar pada=nang ...\footnotemark{} \\
 house \textsc{non.past}-small therefore \textsc{dem.prox} place  \textsc{pl}=\textsc{acc} shop  \textsc{pl}=\textsc{dat} ...\\
`The houses were becoming small, and therefore these places were (turned) into shops.'
\z

\footnotetext{The verb is missing in this sentence because the speaker switches to explaining the kinds of shops established before finishing this sentence.}


\ea \label{K051222nar04.40}
\gllll ~h      ~     \hspace{1.1cm}l \hspace{0.7cm}h\\
 O:ba\dentt{} pa:saR pa\dz@n@n dispEnsERinaN\\
 oobath paasar pada=nang \rm dispensary=\it nang\\
 medicine shop  \textsc{pl}=\textsc{dat} dispensaries=\textsc{dat}       \\
`Dispensaries.'
\z

[...]

\ea \label{K051222nar04.41}
\gllll \hspace{0.8cm}l \hspace{0.8cm}h ~ ~l~~~~~h ~~h~~l ~~h~~l ~~~~~h ~ \hspace{1cm}l ~~~hl\\
 askasi \dentt Ri:ma dER5n sam:a \dzh a:o u:\postalvd iP vilE\texttoptiebar{\J\textctz}  \ae Rija pa\dz5\nz\~a s@pi: \\
  asà-kaasi thriima derang samma jaau uudik village area padanang supii\\
 \textsc{cp}-give thanks \textsc{3pl} all far countryside village area \textsc{pl}=\textsc{dat} \textsc{past}-go\\
`After handing over (the property), they all went to the far away village areas.'
\z

\glossSTDmode