\chapter{Methodology}
\section{Theoretical assumptions}
The presentation in this grammar is intended to be as theory-neutral as possible. No knowledge of any particular theoretical framework is assumed in order to keep the book accessible for  scholars and laymen alike.
% We do not know how long the current formalisms in linguistic theory will survive; it might very well be the case that fifty years from now, people will have no more idea of contemporary theories than we have of tagmemics, en vogue fifty years ago.

Being theory-neutral does not mean that this grammar is atheoretical. There are many decisions in the presentation of the data that were guided by theoretical considerations. Structuring data is impossible without a solid theory that lays the foundation on which the grammatical findings can be grounded. But in a house, it is not necessary to see plans of the foundation in every room, and in a grammar book it is not necessary to cast everything into a powerful yet complicated formalism when it can also be said in simple, theory-neutral words.

One fundamental design principle of this grammatical description is the separation of formal and functional descriptions. In the formal part (part II of this book), I describe linguistic forms (morphemes, constructions) and indicate the functions they can fulfil (indicate recipient, situate event in the past). In the functional part (part III of this book), I take the converse approach and look at functions and how they can be encoded by morphemes and constructions.

The formal part of the grammar is inspired by Radical Construction Grammar \citep{Croft2001rcg} and its separation of linguistic forms along the axes of atomic vs. complex and specific vs. schematic. Atomic schematic constructions are word classes; atomic specific constructions are morphemes; complex schematic constructions are NPs, VPs, clauses, among others; and complex specific constructions are idioms. These different types are discussed in the individual chapters of the formal part. It should be noted that phonology is separate from this approach, and is treated as a precondition for it.

The functional part of the grammar is structured along the levels found in Functional Discourse Grammar \citep{HengeveldEtAl2008fdg}, although some levels have been collapsed for easier na\-vi\-ga\-tion. Propositional content is separated from information structure and speech acts. These three levels are discussed in separate chapters in the functional part.

Two chapters are in an intermediate position between a formal and a functional description. These are the chapters on grammatical relations and valency, which are inspired by Role and Reference Grammar \citep{VanValinEtAl1997rrg}.

Care has been taken to avoid theory-specific jargon and formalisms as much as possible. Normally, standard descriptive nomenclature has been used. Every nomenclature in itself relies on theoretical considerations; the nomenclature here is based on Basic Linguistic Theory \citep{Dixon1997riseandfall, Dryer2006blt}, which should be accessible to every reader with a basic knowledge of typology.

\section{Data collection}
The data for this grammar was collected  in three field trips to Sri Lanka in 2005, 2006 and 2007. Additionally,  two consultants visited the Netherlands in 2008. The first two trips took two and a half months each, while the 2007 trip was two weeks only. The 2008 visit was three weeks. Between the trips, I was in contact with the consultants vie email and telephone. During the trips, the focus was on data gathering, while the periods between them where devoted to analysis. The findings were cross-checked during the following field-trip, or by email. Focus was on natural discourse, monologues as well as conversation. In the first field trip, I did quite a lot of elicitation to get a feeling for the possible constructions in the language so that I could identify them when I heard/transcribed them. The phonological system had  already been analyzed in  its basics \citep{Bichsel}, so it was possible to devise a practical orthography quite quickly and proceed to morphology and syntax. It turned out that my main consultants had complementary strengths: Mr Cassiere provided a lot of narratives and cultural information, while the Hamit and Salim families helped me with grammatical analysis. Text production was stimulated using my basic command of Sri Lanka Malay as far as possible. This was done in order to reduce the possibility of interference from English through code switches. Grammatical analysis was done with English as a meta-language. Occasionally, Sinhala sentences were used to get at meanings which are difficult to render in English, e.g. evidentials. The Malays all have a good command of English, so English was also used in conversation, not only for difficult topics, but also when the speakers just felt like it. In order to maintain a relaxed and casual atmosphere, I did not interfere with that. Maybe some day in the future a researcher will want to do research on code-switching and will be happy to have this material as it is.

Between the first and the second trips, I transcribed the data in the Netherlands and collected interesting grammatical phenomena to check during the second trip. During my second trip, I taught transcription to Mr Izvan Salim, who then transcribed all new recordings, and also the recordings I had transcribed before myself. The great majority of the example sentences in this book are taken from the phonemic transcriptions he made, with minor adjustments for typos and for more complicated aspects of the orthography, especially the use of \graphem{nny} and \graphem{nng}, and spacing. The third, short, trip had as the main goal to restart the transcription process, which had come to a standstill in the meantime. This worked out fine. Due to the worsening security situation in Sri Lanka, the field trip in 2008 was canceled, and two of the main consultants came to the Netherlands instead, where the translations of the texts were perfected and remaining detailed grammatical questions were dealt with.

Besides elicitation, transcription and translation, I visited many Malay households in Sri Lanka, mainly in Kandy, but also in Gampola, Mawanella, Ulapane, Nawalapitiya, Badulla, Colombo, and Galle, and conducted interviews, mainly about family situation and history. The total number of Malays interviewed should be somewhere around 50. Impressionistically, the dialectal differences are slim, and I had no problems understanding the speakers of other towns. Note that I have not been to the South (Hambantota and Kirinda), where the dialect seems to be a bit different (Peter Slomanson, p.c.).

The examples in this book are mainly taken from natural discourse. For every example, its source is indicated. The source file follows certain conventions that permit to identify the text genre. The first letter indicates the town and the following numbers indicate the date (YYMMDD). The three following letters indicate the genre, for instance \texttt{nar} for narrative or \texttt{eli} for elicitation (see Table \ref{tab:genres} for a list). The transcriptions as well as the media files are available on the Internet.\footnote{\url{http://corpus1.mpi.nl/ds/imdi\_browser/}} The primacy of natural discourse over elicitation means that there are not many examples that stem from an \texttt{eli}-file. These are mainly for negative evidence, which never occurs in normal discourse, and marginal phenomena that could not be found in natural discourse but are generally considered of linguistic interest. A third reason to use elicited examples was the desire to keep the context constant, e.g. in the elicitation of paradigms.

\begin{table}
\begin{center}
% use packages: array,tabularx
\begin{tabular}{rl}
\ttfamily nar & narrative \\
\ttfamily cvs & conversation \\
\ttfamily sng & song \\
\ttfamily rec/rcp & recipe \\
\ttfamily wrt & written text \\
\ttfamily prs & presentation \\
\ttfamily eml & email \\
\ttfamily eli & elicitation \\
\ttfamily mix & mix \\
\end{tabular}
\caption{Genre codes used in text examples}
\label{tab:genres}
\end{center}
\end{table}

The recordings are transcribed phonemically. This means that phonological performance factors (e.g. allegro forms) are eliminated.
For instance, the dative marker is transcribed as \em =nang \em most of the time, even if phonetically it might be closer to \phonet{n\~@}. In contradistinction to phonology, the transcriber Mr Salim did not alter morphology or syntax of the recordings. He would transcribe every morpheme he heard, even if he thought that it was incorrect or misplaced. Likewise, he would not add morphemes which were not present in the recording, even if he thought that they would have been necessary. In case of doubt, it was decided to stick with the recorded data, even if it might contain performance errors. Performance errors are unavoidable in natural speech and metalinguistic elicitation alike, as \citet[72]{Birdsong1989} reminds us:\footnote{See \citet{Schuetze1996} for a critical discussion of the reliability of different data gathering procedures in linguistics.}

\begin{quote}
The hypocrisy of rejecting linguistic performance data as too noisy to study, while embracing metalinguistic performance data as proper input to theory, should be apparent to any thoughtful linguist. \citep[72]{Birdsong1989}
\end{quote}

\section{Data gathering techniques}
\subsection{Elicitation}
The Sri Lankan context has some advantages and some drawbacks for elicitation. The advantage is the familiarity of the speakers with the English language, so that the choice of metalanguage is easy, and very often subtle differences can be made clear. Another advantage is the multilingualism of the speakers, which allows switching to another language to exemplify the intended meaning. The major drawback is again multilingualism, which leads, at least in the case of Sri Lanka Malay, to a `diffuse' grammar, which can accommodate a lot of variation. It is common that consultants repeat the word order of the sentence in the metalanguage in their translation. If asked in English, the answer will be SVO, if asked in Sinhala, the answer will be SOV. This has happened more than once, and was in fact used to test informants for consistency. Another drawback (for this task at least) is the social climate, something which could be approximated by a term like `politeness'.  Consultants, like any other person you meet, want the meeting to be enjoyable, and one way of making it enjoyable is giving the linguist the things he wants to hear, or at least not refusing him the things he longs for that much. The elicitor's wits, charm, subterfuge and creativity (or lack thereof) in making up contexts  determine a great deal more the grammaticality judgment of a given sentence than does the consultant's intuition.

One thing that surely triggers politeness and accommodation is the famous sentence \em Can you say X? \em and its variations like \em And you can't say Y, can you? \em and \em So you can also say Y? \em In a diffuse multilingual setting like Sri Lanka, and under the absence of prescriptive schooling in Sri Lanka Malay, speakers are used to a lot of variation in their input and will only reject a sentence if it seriously hinders comprehension.

The approach taken to data collection in this grammar was to elicit grammaticality judgements in the role of a second language learner. These judgments helped shape my idea of the language and permitted me to understand more and more of the transcribed texts. As it happens, most of the elicited phenomena could also be found in the texts, and these naturally occurring examples are then used as illustration rather than the elicited sentences. In the remaining cases, the methods involving less target language production by the linguist were preferred over others. The normal way of prompting informants to produce a certain sentence was to establish a certain context that would logically lead to an utterance with the desired meaning. When checking for identificational constructions, for instance, the following context was presented:

\ea Suppose you come home and you see that someone has broken in and stolen the TV, some money and so on. You do not know who that was. But later you learn that Farook was the thief. How would you then tell a friend that Farook was the thief? \z

Normally consultants would not just give the last sentence, but they would actually repeat the story with the final climax that Farook was found to be the thief. This sentence is \em Farook=jo maaling\em, with the emphatic clitic \em =jo \em indicating the identification. Given that there was a whole story, the consultants would not really know what I was looking for, and accommodation effects should be reduced.

Additionally, the existence of other possibilities was also asked for (\em Could you also say something else?\em). If the construction I was looking for still did not show up, I would try to modify the sentence more and more, trying to hide my intention as much as possible.

\ea \ea Can you start with another word?  
\ex Can you leave out something, and does that change the meaning?  
\ex Could you also start with X?\z\z

This involves an increasing amount of production by the linguist, but it is very limited, and the consultant still has to utter the sentence, instead of just nodding. Consultants easily accept doubtful sentences, but they rarely utter them themselves. The last level would be 

\ea And if I say X Y Z, how does that sound?\z

This last sentence is already pushing the linguist's production quite far and is susceptible to being accepted as correct even if no native speaker would ever utter it.
The second part of the sentence is crucial because the consultant cannot just say \em yes\em. He has to give a qualitative interpretation between something like \em very good \em and \em You can't say that. That does not mean anything! \em If the sentence X Y Z was accepted only at this late stage, I did two things. First, I asked the consultant to repeat the sentence. Many people can say that incomplete sentences are fine, for instance, but the dropped material will invariably show up when they are asked to repeat. The second thing is to ask which one of the sentences is `better', the one that the consultant gave first, or the one she finally accepted after direct prompt. The answer \em Yours is better \em clearly indicates politeness effects. Fortunately, I did not come across this. The answer \em both same \em is OK and indicates that there seems to be no great difference in acceptability between the two sentences. Still, it has to be noted down which sentence came first. The answer \em The second one is a bit odd, no? \em or \em You can say that in fast speech \em indicates that the second sentence would never be said in that way by the consultant and is thus of dubious grammaticality.
Another technique  employed was to start the sentence in a doubtful tone and leaving it suspended in the middle, like looking for a word. Often, the consultant will join in then and finish the sentence.

\subsection{Conversation and narratives}
While I tried not to speak SLM during elicitation sessions in order to not distort the data, I tried to avoid speaking English in conversation and when collecting narratives. People tend to give answers in the languages they are asked in, so asking in English would give stories in English, which was not desired. A very convenient paralinguistic feature is the `Indian Head Wobble', which indicates (roughly) that the hearer agrees with what the speaker is saying. This gesture can then be used to stimulate further linguistic production and normally does not lead to a code-switch to English or other accommodation in the form of foreigner talk.

The inhabitants of Sri Lanka are generally very sociable and love to chat. The Malays are no exception to this. They integrate foreigners very well, and so there is some linguistic production of mine to be found in the \texttt{cvs}-files. As far as I can see, this has not altered the speech of the Malays in general. I do not think that they accommodated their speech beyond reducing speed and articulating better (and even that they did not always do). I thus feel that examples from these conversations can be used to illustrate language use as well as any other file. In fact, the analysis of intonation is based on a conversation among several Malays and myself where they inquire about my place of living etc.

Sometimes my consultants would become involved in a conversation and forget about my pre\-sence. In that case, I often withdrew from the conversation and let them speak among themselves. These data are about the most naturalistic as one can get. Unfortunately, this is not only true for morphosyntax, intonation, discourse structure etc., but also for what is generally known as common ground. Given that I did not participate in the conversations, no need was felt to explain concepts and relations I was not aware of, and I must say that very often I had not the slightest clue what a conversation was about. This is partly due to specialized vocabulary, fast speech and so on, but also to the fact that I did not know the people, places and events that were interesting. Add this ignorance of the common ground to the general South Asian tendency to drop known material and you get a completely lost linguist.



%  It might be interesting to do analyses of syllable structure and stress using songs \citep[cf.][]{Birch} but this has not been done yet.

\subsection{Translation}
The `elicitation' sessions were mostly spontaneous. I had noted down some phenomena I was interested  in and would improvise contexts for variations as we went along. However, sometimes I would use prefabricated English or Sinhala sentences to systematically check some phenomenon.
Normally, I wrote up sentences for three or four different topics which I asked in random order. This was done to prevent entrenchment of the constructions \citep[134]{Schuetze1996}. When people are asked the very same construction twenty times in a row with only marginal variation, they tend to rate the 21st construction like the others. Mixing in some completely different constructions can prevent this feeling of `assembly chain elicitation'.

I also used examples from a learner's grammar of Spoken Sinhala,\footnote{At that point in time I was not aware of \citet{GairEtAl1979textbook}, a textbook for Spoken Sri Lankan Tamil.} \citet{Karunatillake2004}. This was done to see whether some categories that are grammaticalized in Sinhala are grammaticalized in SLM, too. For instance, I had the suspicion that SLM would have a grammaticalized evidential marker because this is an areal feature. However, asking to translate English sentences with some evidential reading like \em It is said/it seems that X \em would always yield \em aràbiilang ... \em, which is the lexical word ``to say''. It was impossible to construct a context in English where speakers were led to mark evidentiality. Using Sinhala primes proved an instant success, though. The Sinhala evidential clitic \em =lu \em in some sentences was translated as the particle \em kiyang\em. \em Kiyang \em is the evidential marker that was impossible to elicit through English. Given this success, I copied all sentences from the learner's grammar and read them to the Malays, a source of big fun (\em How do you say `Oh no, the uncle fell in the well!' \em will always be remembered for that ...). From their translations, I could see whether there was some more grammaticalized material in SLM that was very difficult to elicit through English. Of course, this technique cannot be used as evidence for semantics or word order because of interference effects. Where it can be used successfully is to establish the existence of elusive grammatical morphemes.


\subsection{Word lists}
Word collection was largely done in a spontaneous and associative way. Swadesh lists and extended Swadesh lists for South Asia taken from \citet{Abbi2001} were also used, as were pictures from the MPI Nijmegen field manual \citep{Fieldmanual2001}. Abbi also provides questionnaires for some typical areal phenomena, but it was found that they were too schematic and would not yield good responses.

Every now and then, I checked parts of my dictionary with consultants to see whether I had correctly written down what they had told me. This was especially necessary for the different coronal stops. I found it very hard to hear the difference between dental and retroflex stops, so cross-checking was necessary. The Sinhala alphabet has different graphemes for  these stops, so I could ask the consultants to write down the Malay word in Sinhala script and see which grapheme they would use.\footnote{Sri Lanka has a literacy rate of 95\%, and the Malays are among the best educated ethnic groups so this technique is culturally adequate. Also, the Sinhala alphabet is phonemic and the pronunciation of a given grapheme is always predictable. My informants happened to be more familiar with the Sinhala script, but the Tamil script could also have been used.}
This method relies on writing because, when asked which letter they would write, they would either say \phonet{\dentt aj@n@} or \phonet{\tz aj@n@}, the Sinhala names of the letters. This leaves the initial problem of whether the stop was dental or alveolar. But \phonet{\dentt aj@n@} is represented in Latin script by \graphem{th}, so I found out that I could ask the informants whether a certain word should be spelled with \graphem{t} or \graphem{th}. The answer would indicate whether the stop was dental (th) or alveolar (t).
Other contexts where this asking to write down a word was useful was vowel quantity, the presence or absence of a labial approximant before [u], and the distinction between prenasalized consonants and combinations of nasal+homorganic stop.

\section{Recording}
Most sessions were recorded on audio with permission of the consultants. Sensitive material was deleted on request or not recorded. Recording devices are a bit obtrusive, and speakers were a bit concerned at first. I have the impression that after some time the consultants got used to the recording device and continued to talk  like normal.

Some sessions were also recorded on video with a very small camcorder. This is more obtrusive than audio recording, and the data are less naturalistic. Some people were perfectly fine with being taped, while others seemed to be a bit uneasy. In the latter case, the video camera was not used much. I found that using a tripod improves the technical quality, but decreases the linguistic quality of the data because people tend to stare at the camera. This is why I did not use it much. I would normally sit with the camera on my lap, the LCD display turned 45^{o} so that I could see it. This permitted me  to keep eye contact with the consultant and to use mimics and gesture to signal that I was keeping track of the story. At the same time, I could occasionally shed a glance to the LCD to see how the camera was doing, whether the tape was full or the battery empty etc.

All audio data were recorded as linear pcm at 44,1kHz on flash memory cards using a Mayah Flashman  and a Sony electret condenser microphone ECM-MS907. Video was taken with a Sony DCR-HC90E camcorder and a Sony ECM-HGZ1 gun zoom microphone. The videos seen in the archive have their audio track replaced by the Flashman audio track because the latter's quality is superior.

Every evening, the audio data were copied to the laptop, and video was captured with the program Premiere. Video was converted to mpeg1 and mpeg2 with the program Tsunami,\footnote{It is a somewhat awkward feeling to use a program called Tsunami in Sri Lanka on an everyday basis.} while audio was left as pcm/wav. The audio files were segmented into thematic chunks with the program \texttt{transcriber} and cut into smaller session files with the export function of the same program. Mpeg1, mpeg2, master audio files and session files were stored in a folder that was burned twice on DVD when it reached 4 GB. One DVD was sent by snail mail to Amsterdam while the other DVD served as a local backup.

The data collected during one trip amount to about 50 GB, which is too much for my laptop's hard disk. An external USB-disk was used for extra storing capacity. This disk was also used as an additional backup place as long as space permitted.

The very first narrow phonetic transcriptions were done using \texttt{transcriber}. I wanted to use the program Keyman, which permits easy insertion of IPA symbols into any program. But I found that Keyman and transcriber would not work together. I finally used transcriber as if Keyman worked. This means, for instance, typing e= to get a schwa. In any other program, the conversion is done instantly, but not in transcriber. I stored the file with e= and later search\&replaced the e= by \E. After I had worked out an orthography, this was no longer necessary because the orthography only draws on the standard Latin alphabet.

Transcriptions were also backed up on the same DVDs as the media files. I wrote a small perl script to convert the transcriber files into Toolbox files, keeping the time code information in a way such that ELAN would later be able to import my Toolbox files. Toolbox was used to interlinearize the transcriptions. As a final step, the Toolbox file was imported into ELAN and aligned with video if that was available (the audio file was already aligned through the time code I had kept).


\section{Corpus}
The corpus consists of about 150 texts of diverging nature, complemented by some elicitation sessions. They include narratives, conversations, recipes, emails, and written texts (see Table \ref{tab:genres}). Texts are mainly from Kandy, but can also stem from Badulla, Nawalapitiya, and Gampola. The shortest text is only 10 words, while the longest counts 998. The youngest speaker is around 12, while the oldest speaker is over 70. Table \ref{tab:meth:corpus} gives an overview of the texts used to a greater or lesser extent to illustrate the phenomena discussed in this grammar. Examples are commonly used to illustrate more than one phenomenon. When my analyses are discussed in future work, reference should be made to this grammar. When only the examples are used, reference should be made to the individual text as available in the online corpus. Five representative texts are included in the appendix.

\begin{table}[p]
\begin{tabular}{rrrrrrrr}
141 &\ttfamily K070000wrt04 &11  &\ttfamily  N060113nar05 &6 & \ttfamily K060103rec01 & 2 &\ttfamily K061123sng01\ttfamily \\
 99 &\ttfamily K060108nar02 &11  &\ttfamily  N060113nar03 &6 & \ttfamily B060115prs21 & 2 &\ttfamily K061030mix01\ttfamily \\
 84 &\ttfamily K081105eli02 &11  &\ttfamily  K070000wrt03 &6 & \ttfamily B060115cvs04 & 2 &\ttfamily K061019nar01\ttfamily \\
 80 &\ttfamily K081106eli01 &11  &\ttfamily  K060116nar11 &5 & \ttfamily N060113nar04 & 2 &\ttfamily K060116nar15\ttfamily \\
 59 &\ttfamily K081104eli06 &11  &\ttfamily  K060116nar06 &5 & \ttfamily K081111eli01 & 2 &\ttfamily K060116nar08\ttfamily \\
 54 &\ttfamily K070000wrt01 &11  &\ttfamily  K060116nar04 &5 & \ttfamily K081104eli03 & 2 &\ttfamily K051222nar07\ttfamily \\
 48 &\ttfamily K051220nar01 &11  &\ttfamily  K051201nar02 &5 & \ttfamily K061122nar03 & 2 &\ttfamily K051222nar01\ttfamily \\
 47 &\ttfamily K051206nar02 &11  &\ttfamily  B060115prs10 &5 & \ttfamily K060116nar05 & 2 &\ttfamily K051213nar08\ttfamily \\
 40 &\ttfamily K081103eli02 &10  &\ttfamily  K061120nar01 &5 & \ttfamily K051213nar07 & 2 &\ttfamily K051206nar13\ttfamily \\
 40 &\ttfamily K051213nar06 &10  &\ttfamily  K060116nar07 &5 & \ttfamily K051206nar18 & 2 &\ttfamily K051206nar03\ttfamily \\
 39 &\ttfamily B060115nar04 &10  &\ttfamily  K060103rec02 &5 & \ttfamily K051206nar17 & 2 &\ttfamily B060115prs20\ttfamily \\
 38 &\ttfamily B060115cvs01 &10  &\ttfamily  K051222nar03 &5 & \ttfamily B060115cvs16 & 2 &\ttfamily B060115prs13\ttfamily \\
 34 &\ttfamily K081114eli01 &10  &\ttfamily  K051201nar01 &4 & \ttfamily N060113nar02 & 2 &\ttfamily B060115prs04\ttfamily \\
 34 &\ttfamily K081103eli04 &10  &\ttfamily  B060115cvs08 &4 & \ttfamily K071011eml01 & 1 &\ttfamily K090327eml01\ttfamily \\
 34 &\ttfamily K051205nar05 & 9  &\ttfamily  K061127nar03 &4 & \ttfamily K061019sng01 & 1 &\ttfamily K081201eml01\ttfamily \\
 33 &\ttfamily N061031nar01 & 9  &\ttfamily  K061026rcp01 &4 & \ttfamily K060116nar23 & 1 &\ttfamily K081105eli01\ttfamily \\
 33 &\ttfamily K051222nar06 & 9  &\ttfamily  K061019prs01 &4 & \ttfamily K060116nar09 & 1 &\ttfamily K071203eml01\ttfamily \\
 31 &\ttfamily G051222nar01 & 9  &\ttfamily  K051205nar04 &4 & \ttfamily K060116nar03 & 1 &\ttfamily K071113eml01\ttfamily \\
 30 &\ttfamily K051206nar07 & 9  &\ttfamily  B060115cvs13 &4 & \ttfamily K060116nar01 & 1 &\ttfamily K071029eml01\ttfamily \\
 29 &\ttfamily B060115nar05 & 8  &\ttfamily  N061124sng01 &4 & \ttfamily K060103cvs01 & 1 &\ttfamily K060116nar21\ttfamily \\
 28 &\ttfamily K061026prs01 & 8  &\ttfamily  K061019nar02 &4 & \ttfamily K051206nar14 & 1 &\ttfamily K060116nar13\ttfamily \\
 27 &\ttfamily K070000wrt05 & 8  &\ttfamily  K060116nar10 &4 & \ttfamily K051206nar10 & 1 &\ttfamily K051222nar02\ttfamily \\
 27 &\ttfamily K060103nar01 & 8  &\ttfamily  K060116nar02 &4 & \ttfamily K051206nar05 & 1 &\ttfamily K051220nar02\ttfamily \\
 27 &\ttfamily K051213nar01 & 8  &\ttfamily  K060112nar01 &4 & \ttfamily B060115prs14 & 1 &\ttfamily K051213nar04\ttfamily \\
 26 &\ttfamily B060115nar02 & 8  &\ttfamily  K051206nar20 &4 & \ttfamily B060115prs05 & 1 &\ttfamily B060115prs18\ttfamily \\
 24 &\ttfamily N060113nar01 & 8  &\ttfamily  K051206nar11 &4 & \ttfamily B060115nar03 & 1 &\ttfamily B060115prs12\ttfamily \\
 24 &\ttfamily K061026rcp04 & 8  &\ttfamily  K051205nar02 &4 & \ttfamily B060115nar01 & 1 &\ttfamily B060115prs07\ttfamily \\
 23 &\ttfamily K070000wrt02 & 8  &\ttfamily  G051222nar02 &4 & \ttfamily B060115cvs02 & 1 &\ttfamily B060115prs06\ttfamily \\
 23 &\ttfamily K051222nar04 & 8  &\ttfamily  B060115cvs03 &3 & \ttfamily K081118eli01 & 1 &\ttfamily B060115cvs06\ttfamily \\
 20 &\ttfamily K051206nar12 & 7  &\ttfamily  K051222nar05 &3 & \ttfamily K060103cvs02 & 1 &\ttfamily B060115cvs05\ttfamily \\
 19 &\ttfamily K060108nar01 & 7  &\ttfamily  K051206nar16 &3 & \ttfamily K051222nar08 &   &\ttfamily \\
 18 &\ttfamily K081104eli05 & 7  &\ttfamily  G051222nar04 &3 & \ttfamily K051213nar05 &   &\ttfamily \\
 18 &\ttfamily B060115rcp02 & 7  &\ttfamily  G051222nar03 &3 & \ttfamily K051206nar15 &&\ttfamily  \\
 16 &\ttfamily K051213nar03 & 7  &\ttfamily  B060115rcp01 &3 & \ttfamily K051206nar08 &&\ttfamily  \\
 15 &\ttfamily K061122nar01 & 7  &\ttfamily  B060115prs03 &3 & \ttfamily K051206nar06 &&\ttfamily  \\
 15 &\ttfamily K051213nar02 & 7  &\ttfamily  B060115prs01 &3 & \ttfamily B060115prs17 &&\ttfamily  \\
 14 &\ttfamily K081103eli03 & 6  &\ttfamily  K081104eli01 &3 & \ttfamily B060115cvs17 &&\ttfamily  \\
 14 &\ttfamily K051206nar04 & 6  &\ttfamily  K061125nar01 &3 & \ttfamily B060115cvs09 &&\ttfamily  \\
 14 &\ttfamily B060115prs15 & 6  &\ttfamily  K061026rcp02 &3 & \ttfamily B060115cvs07 &&\ttfamily  \\
 12 &\ttfamily K051206nar19 & 6  &\ttfamily  K060116sng01 &2 & \ttfamily K061123sng03 &&\ttfamily
\end{tabular}
\caption[Number of examples taken from different texts in the corpus]{Number of examples taken from different texts in the corpus and elicitation sessions. Many examples are used at more than one place.}
\label{tab:meth:corpus}
\end{table}
 


The data presented in examples follow a number of conventions which are illustrated below in example \xref{ex:form:intro}.

\xbox{16}{
\ea\label{ex:form:intro}
\gll [Aashik=nang hathu {\em soldier} mà-jaadi suuka=si katha] arà-caanya. \\
     Aashik=\textsc{dat} \textsc{indef} soldier \textsc{inf}-become like=\textsc{interr} \textsc{quot} \textsc{non.past}-ask. \\
    `(He) asks if you want to become a soldier, Ashik [because your father is one, too].' (B060115prs10)
\z
}

\noindent
The conventions are:

\begin{itemize}
 \item Sri Lanka Malay words are in italics (\trs{suuka}{like}).
 \item Words from another language (code-switching or borrowings which are not phonologically integrated), mostly English, are in roman (soldier).
 \item Clitics are marked with the equation mark (\trs{=nang}{\textsc{dat}}).
 \item Affixes are marked with the hyphen (\trs{mà-}{\textsc{inf}}).
 \item Square brackets [] are sometimes used to indicate constituent structure where this facilitates the parsing of the sentence (e.g. heavy NPs, headless relative clauses, subordinates).
 \item Parentheses () are sometimes used for material needed in English but not present in the source, e.g \em He \em in \xref{ex:form:intro}
 \item Background information can be given in brackets [] in the translation ([because your father is one, too]).
 \item The source from which the example is taken is given in parenthesis after the translation (B060115prs10). Most of the sources can be accessed in text and audio in the Dobes archive\\ \url{http://corpus1.mpi.nl/ds/imdi\_browser/} . I would like to thank the Volkswagen foundation and its Dobes project for providing funds for the storage and long-term access of SLM data, next to funding the gathering and analysis of the data in the first place.
\end{itemize}

                                                                                                                                                                                                                                                                                                                                                                                                                                                                                                                                                   