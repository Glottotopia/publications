\chapter{Texts}
This chapter contains five texts. The first and the last texts were provided in writing, while texts two to four are transcriptions of recordings. The texts are orded according to difficulty.

The first text, \em How Andara Ate Sugar \em is a short story of a jester, Andare, who the king wants to fool, but in the end it is the jester who fools the king. Andare is a character of Sinhalese folklore. This text was provided by Mr Izvan Salim.

The second text was told by Gnei Binthan Muumin, a senior English teacher,  and relates the story of her life as a Malay. This text is given in both phonetic and orthographic transcription. Intonation is indicated as well.

The third text covers the history of the Sri Lankan Malays. It is told by Tony Salim, who belongs to the same household as Gnei Binthan Muumin, and is the father or Izvan Salim. It shows more oral features than the preceding ones. All the speakers mentioned above are from Kandy and are native speakers of Sri Lanka Malay. All have command of Sinhala, English and Tamil.

The fourth text was told by Mr Sherifdeen from Mawanella and is about the `Sri Lankan Robin Hood', \em Saradiyel\em, who was captured by a Sri Lankan Malay. This text is very oral in nature, and a bit difficult to follow in the written form. Mr Sherifdeen is a retired soldier and speaks Sri Lanka Malay, Sinhala, Tamil, and English.

The fifth text is a translation of the Brothers Grimm's tale \em Snow-White and Rose-Red\em. The translation was made by Izvan Salim. It is the  most complex text, but might show some features more typical of European writing than the other texts.

Some parts of the transcriptions are not included in the text here. This is the case for questions asked by the researcher, other people intervening, or unrelated content (e.g. offering tea).

This text collection was edited in collaboration with Tony Salim and Izvan Salim. I am very grateful for their help, commitment and patience.
