\section*{Saradiyel}
This story is about the `Sri Lankan Robin Hood', Saradiyel, who robbed the rich to give to the poor and was arrested by a Malay policeman \citep[cf.][52]{OsmanEtAl2008}. Generally, Robin-Hood-types carry the sympathy of the speakers, but in this case, his adversary is Malay, which yields split sympathies. On the one hand, Saradiyel is portrayed as a hero, on the other hand as a terrorist. Especially interesting in this text is the tracking of reference between the policeman and Saradiyel. The policeman is referred to by the polite third person singular pronoun \em incayang\em, while Saradiyel is normally indicated with the impolite form \em dee\em. This suffices to track reference, and other means of disambiguation are normally not employed.

This text shows more features of oral communication than the preceding ones. There are frequent changes of topic, and new starts, as well as a good deal of code switching between SLM and English and some lexical gaps which are filled by Sinhala or Tamil.
Also, the story is told in a somewhat cyclical fashion. In a first run-through, the major events are related, while the details are worked out in subsequent iterations.


\glossSTDmode
\xbox{16}{
\ea
\gll Robin {\em Hood} katha arà-biilang    laama  ...\\
      Robin Hood \textsc{quot} \textsc{non.past}-say old  ... \\
`The person we call Robin Hood is an old ...'
\z
}

\xbox{16}{
\ea
\gll cinggala=dering   arà-biilang    naama Saradiyel \\
      Sinhala=\textsc{abl} \textsc{non.past}-say name Saradiyel   \\
`The name he is called in Sinhala is  ``Saradiyel''.'
\z
}

\xbox{16}{
\ea
\gll de=ppe\footnotemark{}     naama Saradiyel \\
     3\textsc{s.impolite}=\textsc{poss} name Saradiyel    \\
`His name was Saradiyel.'
\z
}
\footnotetext{The first occurrence of the impolite pronoun \em dee \em to refer to Saradiyel indicates that he does not have the sympathy of the speaker.}

 

\xbox{16}{
\ea
\gll diya\footnotemark{} ini      Seelong=ka   anà-duuduk        {\em 18},     {\em 18}   {\em odd}=ka,    {\em 18}   {\em odd}=ka=jo\footnotemark{} \\
      \textsc{3s} \textsc{dem.prox} Ceylon=\textsc{loc} \textsc{past}-exist.\textsc{anim} 18 18 odd=\textsc{loc} 18 odd=\textsc{loc}=\textsc{emph}   \\
`He lived here in Ceylon in 18-odd.'
\z
}
\addtocounter{footnote}{-1}
\footnotetext{\em Diya \em is a neutral third person pronoun, which is used very rarely.}
\addtocounter{footnote}{1}
\footnotetext{This sentence shows many oral features, like repetition and code switching.}

\xbox{16}{
\ea
\gll {\em 18}  {\em odd}=ka=jo       anà-duuduk        siini Seelong=ka \\
     18 odd=\textsc{loc}=\textsc{emph} \textsc{past}-exist.\textsc{anim} here  Ceylon=\textsc{loc}  \\
`He lived here in 18-odd, in Ceylon.'
\z
}
 

\xbox{16}{
\ea
\gll de=ppe     pukurjan: \\
     3\textsc{s.impolite}=\textsc{poss} work    \\
`His work:'
\z
}

\xbox{16}{
\ea
\gll dee maaling hatthu \\
      3\textsc{s.impolite} thief \textsc{indef}   \\
`He was a thief.'
\z
}

\xbox{16}{
\ea
\gll dee arà\footnotemark-cuuri   baaye kaaya      oorang pada=dering \\
      3\textsc{s.impolite} \textsc{non.past}-steal good rich man \textsc{pl}=\textsc{abl}   \\
`He steals from very rich people.'
\z
}
\footnotetext{The tense in the narrative switches from past to the present \em arà \em to render the story more lively.}

\xbox{16}{
\ea
\gll kaaya oorang --     monied  people --     =dering=jo         arà-cuuri \\
     rich man { } { } { } { } =\textsc{abl}=\textsc{emph} \textsc{non.past}-steal    \\
`It is from moneyed people that he steals.'
\z
}

\xbox{16}{
\ea
\gll mà\footnotemark-cuuri    abbis    dee arà-kaasi    miskin pada=nang \\
     \textsc{inf}-steal finish 3\textsc{s.impolite} \textsc{non.past}-give poor \textsc{pl}=\textsc{dat}    \\
`After stealing, he gives the money to the poor.'
\z
}
\footnotetext{Not all speakers would agree with the infinitive marker here.}

\xbox{16}{
\ea
\gll itthunangapa          diya asàdhaathang   inni     {\em government}, {\em government}=pe     {\em forces} pada=nang=susaama      baaye thraa\footnotemark \\
     however \textsc{3s} \textsc{copula} \textsc{dem.prox} government government=\textsc{poss} forces \textsc{pl}=\textsc{dat}=\textsc{comit} good \textsc{neg}    \\
`However, he was not good with the government forces.'
\z
}
\footnotetext{The structure of this sentence seems to  be influenced by English syntax, namely the use of the copula and the comitative.}

\xbox{16}{
\ea
\gll dee mana  vakthu=le      {\em forces}=nang   arà-bukalai \\
      \textsc{3s} which time=\textsc{addit} forces=\textsc{dat} \textsc{non.past}-fight   \\
`He fights with the forces all the time.'
\z
}

\xbox{16}{
\ea
\gll {\em forces} pada=samma      arà-bukalai \\
     forces \textsc{pl}=\textsc{comit} \textsc{non.past}-fight    \\
`He fights with the army.'
\z
}

\xbox{16}{
\ea
\gll inni=yang       mà-peegang=nang, subla      nigiri=dering   suda anà-dhaathang.  \\
      \textsc{dem.prox} \textsc{inf}-catch=\textsc{dat} side country=\textsc{abl} thus \textsc{past}-come   \\
`to catch this man, forces\footnotemark{} from the neighbouring countries came.'
\z
}
\footnotetext{The reference of the helpers is not established but must be inferred from the context. Some unnamed entity came from neighbouring countries. This entity must be in a position to help law enforcement against a thief, so it must also be some kind of force. Also note the switch back to past tense.}

\xbox{16}{
\ea
\gll   inni     {\em forces} pada asà-dhaathang, Seelong=ka   asà-duuduk,     derang anà-peegang    dee=yang\footnotemark{} \\
    \textsc{dem.prox} forces \textsc{pl} \textsc{cp}-come Ceylon=\textsc{loc} \textsc{cp}-stay \textsc{3pl} \textsc{past}-catch 3\textsc{s.impolite}=\textsc{acc}     \\
`These forces came and stayed in  Ceylon and caught him.'
\z
}
\footnotetext{Note the SVO word order.}

\xbox{16}{
\ea
\gll itthunang pàrthaama kithang=pe     mlaayu=jo    anà-peegang\footnotemark{}     \\
     but before \textsc{1pl}=\textsc{poss} Malay=\textsc{emph} \textsc{past}-catch    \\
`But, before that it was our Malay who caught him.'
\z
}
\footnotetext{Note the dropped `object'/patient in this sentence. The agent \em mlaayu \em is mentioned, but not the patient, Saradiyel, which can be inferred from context.}

\xbox{16}{
\ea
\gll  [incayang anà-peegang  \zero{}]\footnotemark{}  asàdhaathang Mavanella \\
      3\textsc{s.polite} \textsc{past}-catch { }  \textsc{copula} Mawanella   \\
`Where he caught him was at Mawanella.'
\z
}
\footnotetext{This is a headless relative clause referring to a location.}

\xbox{16}{
\ea
\gll peegang abbis \\
     catch finish    \\
`After catching him'
\z
}

\xbox{16}{
\ea
\gll peegang=nang   blaakang, incayang\footnotemark{}    thee\u mbak aada \\
     catch=\textsc{dat} after 3\textsc{s.polite} shoot exist    \\
`after catching him,  he shot him.'
\z
}
\footnotetext{It is not clear whether the police man is agent or patient of the shooting here.}

\xbox{16}{
\ea
\gll incayang  thee\u mbak abbis, \zero{}\footnotemark{}    salba  laari aada  thumpath=nang \\
      3\textsc{s.polite} shoot finish { } escape run exist place=\textsc{dat}   \\
`Upon shooting,  he escaped and ran to the place.'
\z
}
\footnotetext{From syntax only, it is not clear whether the shooter or the victim ran away.}

\xbox{16}{
\ea
\gll itthu=jo,    {\em first}, kàthaama {\em police} oorang nya-maathi\footnotemark{}    inni     {\em terrorist} hatthu=dering \\
     \textsc{dem.dist}=\textsc{emph} first first police man \textsc{past}-die \textsc{dem.prox} terrorist \textsc{indef}=\textsc{abl}    \\
`And that is how the first policeman came to die by the hands of a terrorist.'
\z
}
\footnotetext{It is only this sentence that establishes that the police man was shot, and that the thief was the killer, who then ran away.}

\xbox{16}{
\ea\label{ex:text:saradiyel:terrorist}
\gll {\em `Terrorist'}=nang   apa=yang arà-biilang    mlaayu=dering? \\
     terrorist=\textsc{dat} what=\textsc{acc} \textsc{non.past}-say Malay=\textsc{abl}    \\
`How do you say ``terrorist'' in Malay?'
\z
}

\xbox{16}{
\ea
\gll {\em terrorist} hatthu=dering  anà-maathi    kàthaama oorang=jo    incayang,  {\em Kandy}, {\em police} {\em man}, {\em bandit}\footnotemark \\
      terrorist  \textsc{indef}=\textsc{abl} \textsc{past}-die first man=\textsc{emph} 3\textsc{s.polite} Kandy police man, bandit \\
`The first person who died by a terrorist was him.'
\z
}
\footnotetext{The words in this string change reference between the police man and the bandit.}

\xbox{16}{
\ea
\gll {\em bandit}, {\em bandit} {\em bandit}=dering  anà-maathi    {\em first}  kàthaama {\em police} {\em man}=jo    incayang \\
     bandit bandit bandit=\textsc{abl} \textsc{past}-die first first police man=\textsc{emph} 3\textsc{s.polite}    \\
`The first policeman killed by the hands of a bandit was him.'
\z
}

\xbox{16}{
\ea
\gll laile itthu=yang      arà-..inni..-iingath-kang   inni     {\em police}=dering \\
      still \textsc{dem.dist}=\textsc{acc} \textsc{non.past}-..\textsc{hesit}..-think-\textsc{caus} \textsc{dem.prox} police=\textsc{abl}   \\
`Still the police bring into memory this story.'
\z
}


\xbox{16}{
\ea
\gll [{\em police}=ka    anà-aada  \zero]\footnotemark{} \zero{}\footnotemark{}   {\em first} {\em police} {\em man} {\em to}  {\em die} {\em in} {\em Sri} {\em Lanka}\\
     police=\textsc{loc} \textsc{past}-exist { } { } first police man to die  in Sri Lanka  \\
`Of those who were in the police, he was the first man to die in Sri Lanka.'
\z
}

\addtocounter{footnote}{-1}
\footnotetext{A headless relative clause referring to all persons who are in the police.}
\addtocounter{footnote}{1}
\footnotetext{The participant is dropped here.}
 

\xbox{16}{
\ea
\gll {\em by} {\em terrorist} \\
     by terrorist    \\
`from the hands of a terrorist.'
\z
}

\xbox{16}{
\ea
\gll itthu=yang      arà-..  laile         arà-{\em celebrate}-kang,     oorang pada ini {\em memorial}           dìkkath=ka     asà-pii derang pada {\em parade} samma asà-gijja derang=yang   arà-iingath    maana vakthu=le\\
      \textsc{dem.dist}=\textsc{acc} \textsc{non.past}-.. still \textsc{non.past}-celebrate-\textsc{caus} man \textsc{pl} \textsc{dem.prox} memorial vicinity=\textsc{loc} \textsc{cp}-goo \textsc{3pl} \textsc{pl} parade all \textsc{cp}-make \textsc{3pl}=\textsc{acc} \textsc{non.past}-think which time=\textsc{addit}   \\
`Still, this incident is celebrated, people go to the memorial place and hold a parade and think of him all the time.'
\z
}
  

\xbox{16}{
\ea
\gll  Maana maana thaaun=le \\
     which which year=\textsc{addit}    \\
`Every year'
\z
}

\xbox{16}{
\ea
\gll maana thaaun=le      Mavanella=ka,    itthu    asàdhaathang Uthuvan Kandha Paamula=ka. \\
      which year=\textsc{addit} Mawanella=\textsc{loc} \textsc{dem.dist} \textsc{copula} Uthuwan Kandha Paamula=\textsc{loc}   \\
`every year in Mawanella, that is at the Uthawan Kandha Paamula (at the foot of the mountain).'
\z
}

\xbox{16}{
\ea
\gll Uthuvan Kandha=ka=jo       diya anà-duuduk.      \\
     Uthuwan Kandha=\textsc{loc}=\textsc{emph} 3 \textsc{past}-stay    \\
`He stayed at Uthuwan Kandha.'
\z
}


\xbox{16}{
\ea
\gll   Sini=ka             aada,  blaakang=ka,    bìssar guunung  hatthu. \\
       here=\textsc{loc} exist behind=\textsc{loc} big mountain indef\\
    `There is a big mountain here, behind us.'
\z
} \\

\xbox{16}{
\ea
\gll Dee itthu=ka=jo          arà\footnotemark-sbuuni   duuduk \\
     3\textsc{s.impolite} \textsc{dem.dist}=\textsc{loc}=\textsc{emph} \textsc{non.past}-hide stay    \\
`He was hiding there.'
\z
}
\footnotetext{The use of \em arà- \em is surprising here. Time reference is clearly not the present, and relative simultaneous tense seems unlikely as well.}

\xbox{16}{
\ea
\gll Itthu=ka=jo     anà-sbuuni   duuduk.  \\
     \textsc{dem.dist}=\textsc{loc}=\textsc{emph} \textsc{past}-hide stay   \\
`He was hiding right there.'
\z
}

\xbox{16}{
\ea
\gll  Itthu=deri sbuuni, dee maana aari=le      asà-dhaathang, thìnga-ari vakthu=nang   kalthraa    maalang vakthu=nang \\
      \textsc{dem.dist}=\textsc{abl} hide 3\textsc{s.impolite} which day=\textsc{addit} \textsc{cp}-come middle-day time=\textsc{dat} otherwise night time=\textsc{dat}    \\
`He was hiding right there, and from there he came every day, during the afternoon or otherwise during the night and'
\z
}

  

\xbox{16}{
\ea
\gll kàthaama nya-aada     {\em lorry} {\em bus} thraa\footnotemark \\
      before \textsc{past}-exist lorry bus \textsc{neg}   \\
`before, there were no lorries nor busses.'
\z
}

\footnotetext{The cooccurrence of \em aada \em and \em thraa \em in surprising in this sentence. Strictly speaking, \em thraa \em is already a negated existential, so that there is no need to add the existential \em aada\em. One could interpret \em nyaaada lorry bus \em as a relative clause, meaning `the lorries and busses which existed', but the whole meaning `the lorries and busses which existed did not exist in former times' does not sound very good either.}

\xbox{16}{
\ea
\gll {\em lorry} {\em bus} thraa, kreetha kuuda kreetha, ya,   kuuda kreetha \\
     lorry bus \textsc{neg} cart horse cart yes horse cart    \\
`There were no lorries or busses, only horse carts, yeah, horse carts.'
\z
}

\xbox{16}{
\ea  {\em kuuda} {\em kreetha}, horse driven cart\z
}

\xbox{16}{
\ea
\gll itthu    guunung=ka     naayek=jo,   asà-dhaathang arà\footnotemark{}-{\em attack}-kang      itthu    oorang pada=yang \\
     \textsc{dem.dist} mountain=\textsc{loc} climb=\textsc{emph} \textsc{cp}-come \textsc{non.past}-attack \textsc{dem.dist} man \textsc{pl}=\textsc{acc}    \\
`As soon as they climb up the hill, he comes and attacks the people.'
\z
}
\footnotetext{A return to the historical present in this sentence.}

\xbox{16}{
\ea
\gll [Sithu=ka     aada  bìssar oorang pada]=yang   asà-{\em attack}-kang,    {\em mail}=nya\footnotemark{}    asà-cuuri,   \\
      \textsc{dem.dist}=\textsc{loc} exist big man \textsc{pl}=\textsc{acc} \textsc{cp}-attack-\textsc{caus} mail=\textsc{acc} \textsc{cp}-steal   \\
`He attacks the big people who are there and stole the mail.'
\z
}
\footnotetext{The meaning of \em =nya \em is unclear here.}

\xbox{16}{
\ea
\gll  {\em Mail}=ka    duvith arà-baapi,    bukang\\
       mail=\textsc{loc} money \textsc{non.past}-take.away tag\\
    `The mail has money in it, isn't it.'
\z
} \\

\xbox{16}{
\ea
\gll Suurath arà-baapi=ka       duvith aada. Laeng   Kandi=dang,     Kluu\u mbu, Kandi=dering Kluu\u mbu=dang suurath arà-baapi.\footnotemark \\
     letter \textsc{non.past}-take.away=\textsc{loc} money exist Other Kandy=\textsc{dat} Colombo Kandy=\textsc{abl} Colombo=\textsc{dat} letter \textsc{non.past}-take.away    \\
`The brought letters contained money. Some to Kandy, some from Kandy to Colombo the letters go.'
\z
}
\footnotetext{The speaker gets a bit lost in the marking of the spatial semantic roles in this sentence.}

\xbox{16}{
\ea
\gll Inni     samma dee ambel abbis. \\
      \textsc{dem.prox} all 3\textsc{s.impolite} take finish   \\
`He steals all that.'
\z
}

\xbox{16}{
\ea
\gll Dee athi-kaasi    miskiin oorang pada=nang. \\
     3\textsc{s.impolite} \textsc{irr}-give poor man \textsc{pl}=\textsc{dat}    \\
`He would give it to the poor.'
\z
}

\xbox{16}{
\ea
\gll Miskiin oorang, miskiin  {\em means} {\em poor} {\em people} \\
     poor man poor     \\
\z
}

\xbox{16}{
\ea
\gll oorang pada=nang   athi-kaasi. \\
      man \textsc{pl}=\textsc{dat} \textsc{irr}-give   \\
`would give it to the people.'
\z
}

\xbox{16}{
\ea
\gll Itthu=yang=jo         incayang=le       diyath anà-duuduk. \\
     \textsc{dem.dist}=\textsc{acc}=\textsc{emph} 3\textsc{s.polite}=\textsc{addit} look \textsc{past}-sit    \\
`He was always eyeing for that mail.'
\z
}

\xbox{16}{
\ea
\gll [[Dee asà-duuduk]     anà-gijja    samma]=jo    itthu\footnotemark \\
     3\textsc{s.impolite} \textsc{cp}-exist.\textsc{anim} \textsc{past}-make all=\textsc{emph} \textsc{dem.dist}    \\
`That was what he did during his lifetime.'
\z
}
\footnotetext{An equational sentence where the first term is a quantifier modified by a complex relative clause. The relative clause consists of a higher clause containing the verb \trs{gijja}{make} with the participants \trs{dee}{he} and the head noun \trs{samma}{all}. This clause has in turn a cosubordinate clause containing the verb \trs{duuduk}{live}.}

\xbox{16}{
\ea
\gll Itthu=nang      blaakang dee=yang   su-peegang     siini Seelong=ka. \\
     \textsc{dem.dist}=\textsc{dat} after 3\textsc{s.impolite}=\textsc{acc} \textsc{past}-catch here Ceylon=\textsc{loc}    \\
`After that they caught him here in Ceylon.'
\z
}

\xbox{16}{
\ea
\gll Peegang abbis \\
     catch finish    \\
`After being caught'
\z
}

\xbox{16}{
\ea
\gll diya siini thàrà-duuduk. \\
    \textsc{3s} here \textsc{neg.past}-stay     \\
`he didn't stay here.'
\z
}

\xbox{16}{
\ea
\gll Diya laaye   hathu  nigiri=nang    anà-baapi,    buunung-king=nang,     iiya \\
     3 other \textsc{indef} country=\textsc{dat} \textsc{past}-bring kill-\textsc{caus}=\textsc{dat}  yes  \\
`He was taken to another country, to have him executed.'
\z
}

\xbox{16}{
\ea  hero\footnotemark{} not killed in  Sri Lanka so\z
}
\footnotetext{Note that the former `terrorist' is now treated as a `hero'.}

\xbox{16}{
\ea
\gll Diya=yang  anà-baapi London.\footnotemark \\
      \textsc{3s}=\textsc{acc} \textsc{past}-take.away London      \\
`They took him to London.'
\z
}
\footnotetext{Note the absence of case marking on \em London\em.}

\xbox{16}{
\ea
\gll Thraa, {\em England}=nang   nya-baapi,    dee=yang mà-{\em execute}-kang\footnotemark=nang.\\
       no England=\textsc{dat} \textsc{past}-bring 3\textsc{s.impolite}=\textsc{acc} \textsc{inf}-execute-\textsc{caus}=\textsc{dat} \\
`No, they took him to England to execute him.'
\z
}
\footnotetext{\em -kang \em is not used as a causative marker here, but as a loanword integrator, as is clear from the following sentence, which paraphrases this one.}

\xbox{16}{
\ea
\gll Mà-mathi-king,       mà-mathi-king=nang   siithu=jo    anà-baapi. \\
     \textsc{inf}-kill-\textsc{caus} \textsc{inf}-kill-\textsc{caus}=\textsc{dat} there=\textsc{emph} \textsc{past}-take.away    \\
`They took him there to execute him.'
\z
}

\xbox{16}{
\ea
\gll Itthu=nang      kàthaama,  incayang  punnung   dhraapa oorang thee\u mbak=le, incayang=nang thama\footnotemark-kìnna. \\
      \textsc{dem.dist}=\textsc{dat} before 3\textsc{s.polite} full how.many man shoot=\textsc{addit} 3\textsc{s.polite}=\textsc{dat} \textsc{neg.irr}-strike   \\
`Before that, however many people shot at him, they would never hit him.'
\z
}
\footnotetext{Note the use of the irrealis negator \em thama- \em to negate a habitual context in the past. \em Thama- \em is normally used for non-past negation, but habitual contexts also allow the use of \em thama- \em with past reference.}

\xbox{16}{
\ea
\gll {\em Police} pada dhraapa   thee\u mbak=le,      siyang  siyang  thama-kìnna          de=dang \\
     police \textsc{pl} how.many shoot=\textsc{addit} earlier earlier \textsc{neg.irr}-strike 3\textsc{s.impolite=dat}    \\
`How much the policemen would shoot, first they would not strike him.'
\z
}

\xbox{16}{
\ea
\gll Incayang=pe      kaaki=ka    nya-aada     kiyang\footnotemark, hathu  {\em talisman}. \\
     3\textsc{s.impolite}=\textsc{poss} leg=\textsc{loc} \textsc{past}-exist \textsc{evid} \textsc{indef} talisman    \\
`In his leg there was a talisman, it seems.'
\z
}
\footnotetext{In contradistinction to the preceding sentences, which the speaker did not witness either, this sentence contains the evidential marker \em kiyang\em, probably to underscore the unbelievable nature of a talisman, which the speaker cannot vouch for.}

\xbox{16}{
\ea
\gll Surayak\footnotemark,                  {\em talisman},  {\em talisman},        {\em talisman}. \\
     talisman[Sinh.]    \\
\z
}
\footnotetext{This lexical gap is filled with Sinhala.}

\xbox{16}{
\ea  So that he didn't get shot.
\z
}

\xbox{16}{
\ea
\gll  {\em Talisman} hatthu anà-aada,     kiyang.  \\
     talisman \textsc{indef} \textsc{past}-exist \textsc{evid}    \\
`A talisman was there it seems.'
\z
}

\xbox{16}{
\ea
\gll  {\em Talisman} katha arabiilang    hatthu .. inni ..; aapa\footnotemark{} itthu=nang      mlaayu=dering  arà-biilang?     \\
     talisman \textsc{quot} \textsc{non.past}-say \textsc{indef} .. \textsc{dem.prox}  ..  what \textsc{dem.dist}=\textsc{dat} Malay=\textsc{abl} \textsc{non.past}-say\\
`The thing they call a talisman, what do they call this in Malay again?'
\z
}
\footnotetext{Metalinguistic information is requested in this sentences by means of \em aapa\em, whereas when querying for the word for `terrorist' above \xref{ex:text:saradiyel:terrorist}, \em apayang \em was used.}

\xbox{16}{
\ea
\gll Inni kithang arà-thaaro,    iiya, iiya, kithang arà-thaaro,    aapa thaau=si,  `jiimath'! \\
     \textsc{dem.prox} \textsc{1pl} \textsc{non.past}-put yes yes \textsc{1pl} \textsc{non.past}-put  what know=\textsc{interr} talisman    \\
`We put, yes, we put, you know what, a talisman.'
\z
}

\xbox{16}{
\ea  jiimath,  jiimath,  talisman is  jiimath  
\z
}

\xbox{16}{
\ea
\gll Jiimath  hatthu thaaro kal\footnotemark\addtocounter{footnote}{-1}-aada,    jiimath=yang    thaaro kalu,\footnotemark{}    inni     kaaki daalang=ka=jo       bìlla abbis thaaro aada.\footnotemark\\
     talisman \textsc{indef} put if-exist talisman=\textsc{acc} put if \textsc{dem.prox} leg inside=\textsc{loc}=\textsc{emph} split finish put exist    \\
`When you have put a talisman, when you have put the talisman in this leg, he had cut open (the leg) and put it.'
\z
}
\addtocounter{footnote}{-1}
\footnotetext{Note the preverbal and postverbal occurrence of the conditional marker \em kal(u)\em.}
\addtocounter{footnote}{1}
\footnotetext{This string of four verbs (\em bìlla abbis thaaro aada\em) should probably receive some TAM-marking on one of the verbs, although it is difficult to determine on which one.}

\xbox{16}{
\ea
\gll Kaaki poothong, daalang=ka    thaaro=apa,     jaaith aada,  a. talisman \\
     leg cut inside=\textsc{loc} put=after sew exist a. talisman    \\
`Cut the leg and put it inside and sewn it, the talisman.'
\z
}

\xbox{16}{
\ea
\gll talisman thaaro jaaith, a. jaaith\footnotemark \\
     talisman put sew a. sew    \\
`Put the talisman and sew.'
\z
}
\footnotetext{The speaker gets more and more taken away by the fantastic narrative and the syntactic structure of the sentences becomes less important.}

\xbox{16}{
\ea
\gll Suda itthu=dering     de=dang    thama-kìnna           kiyang vatthu. \\
     thus \textsc{dem.dist}=\textsc{abl} \textsc{3s.impolite=dat} \textsc{neg.irr}=strike \textsc{evid}  time   \\
`So because of that he was never hit it seems.'
\z
}

\xbox{16}{
\ea
\gll Itthu    asàdhaathang {\em Buddhist}  {\em priest} hatthu.\\
     \textsc{dem.dist} \textsc{copula} Buddhist priest \textsc{indef}    \\
`That was a Buddhist priest.'
\z
}

\xbox{16}{
\ea
\gll [Anà-gijja    kaasi \zero] kiyang, incayang=nang    giithu   hathu  hathu  inni     {\em power} hatthu, kuvath-ahaan hatthu \\
      \textsc{past}-make give { } \textsc{evid} 3\textsc{s.polite}=\textsc{dat} that.way \textsc{indef} \textsc{indef} \textsc{dem.prox} power \textsc{indef} strong-\textsc{nmlzr} \textsc{indef}  \\
`The one who gave home this power it seems, like this a this this power, this power'
\z
}
 
 

\xbox{16}{
\ea
\gll Itthu     {\em Buddhist}  {\em priest}=jo    gijja kaasi aada. \\
      \textsc{dem.dist} Buddhist priest=\textsc{emph} make give exist   \\
`That Buddhist priest made that for him.'
\z
}

 

\xbox{16}{
\ea
\gll Suda lai     hatthu ma... mosthor arà-biilang \\
      thus more \textsc{indef} ma... manner \textsc{non.past}-say   \\
`Then, another version says'
\z
}

\xbox{16}{
\ea
\gll incayang=nang    inni     Seelong=ka   anà-duuduk        mlaayu pada, itthu    muusing=ka    anà-duuduk  mlaayu pada, {\em 1876}=ka     anà-duudu mlaayu pada, itthu    itthu muusing, {\em those} {\em days} \\
    3\textsc{s.polite}=\textsc{dat} \textsc{dem.prox} Ceylon=\textsc{loc} \textsc{past}-exist.\textsc{anim} Malay \textsc{pl}   \textsc{dem.dist} period=\textsc{loc} \textsc{past}-exist.\textsc{anim} Malay \textsc{pl} 1876=\textsc{loc} \textsc{past}-exist.\textsc{anim} Malay \textsc{pl} \textsc{dem.dist} \textsc{dem.dist} period those days    \\
`that the Malays who lived here in 1876 back then.\footnotemark'
\z
} 
\footnotetext{The complex NP is uttered three times until the speaker is satisfied with its structure.}

\xbox{16}{
\ea
\gll Mlaayu pada dhaathang vakthu \\
     Malay \textsc{pl} come time    \\
`When the Malays came'
\z
}

\xbox{16}{
\ea
\gll incayang  hatthu mlaayu derang=samma kumpulan      baaye. \\
      3\textsc{s.polite} \textsc{indef} Malay \textsc{3pl}=\textsc{comit} association good   \\
`he\footnotemark{} was a Malay, who was a good friend of his.'
\z
}
\footnotetext{At this point in time, the anaphoric pronoun \em incayang \em has no antecedent. It refers to Saradiyel's friend Mammale, who has, however, not been introduced yet.}

\xbox{16}{
\ea
\gll Mammallan        itthu    inni=samma, derang=pe     naama Mammale, itthu   inni    katha, derang pada baaye kumpulan. \\
      Mammallan \textsc{dem.dist} \textsc{dem.prox}=\textsc{comit} \textsc{3pl}=\textsc{poss} name Mammale \textsc{dem.dist} \textsc{dem.prox} \textsc{quot} \textsc{3pl} \textsc{pl} good association  \\
`Mammale was his good friend, his name was Mammale, who was always together with him.'
\z
}

\xbox{16}{
\ea
\gll Suda kapang-duuduk   derang pada\footnotemark{} kaasi kiyang hathu  kirris  hatthu. \\
     thus when-exist.\textsc{anim} \textsc{3pl} \textsc{pl} give \textsc{evid} \textsc{indef} dagger \textsc{indef}    \\
`So when they were there, it seems he gave him a dagger.'
\z
}
\footnotetext{\em Derang pada \em is technically plural, but is used here with singular reference, which is very polite. The use of plural forms for singular reference is also found in Tamil. Furthermore note that there is no overt indication of semantic role; we do not know who was the giver of the dagger and who the recipient.}

\xbox{16}{
\ea
\gll kirris kirris, a. kirris     {\em kinissa}\footnotemark \\
     dagger dagger a. dagger dagger[Sinh.]    \\
\z
}
\footnotetext{The English word \em dagger \em could not be retrieved at this moment, and the Sinhala word \em kinissa \em is used instead.}

\xbox{16}{
\ea  what do you call that, `dagger'!\z
}
 

\xbox{16}{
\ea
\gll itthu    asàdhaathang [baaye=nang   vaasil-kang     aada  {\em dagger}] hatthu \\
      \textsc{dem.dist} \textsc{copula} good=\textsc{dat} blessed-\textsc{caus} exist dagger \textsc{indef}   \\
`That was a well blessed dagger.'
\z
}

\xbox{16}{
\ea  waasil means
\z
}

\xbox{16}{
\ea
\gll aapa dhua, dhua, dhua \\
     what prayer prayer prayer    \\ 
\z
}

\xbox{16}{
\ea {\em dhua}  blessed, blessed
\z
}

\xbox{16}{
\ea  blessed dagger, blessed blessed, blessed dagger
\z
}

\xbox{16}{
\ea  you understand blessed dagger, ya blessed dagger
\z
}

\xbox{16}{
\ea
\gll so  itthu=ka       mà\footnotemark-aada     kiyang, thembak-an       thama-kìnna  hatthu oorang=na=le \\
     so \textsc{dem.prox}=lic \textsc{inf}-exist \textsc{evid} shoot-\textsc{nmlzr} \textsc{neg.irr}-strike \textsc{indef} man=\textsc{dat}=\textsc{addit}    \\
`So when he has that one, no shot will hit him it seems, by anybody.'
\z
}
\footnotetext{The use of the infinitive is unexpected here, a conditional would be  more readily expected.}

\xbox{16}{
\ea
\gll Itthu    nya-aada     katha=le      arà-biilang. \\
     \textsc{dem.dist} \textsc{past}-exist \textsc{quot}=\textsc{addit} \textsc{non.past}-say    \\
`He had that also, they say.'
\z
}

\xbox{16}{
\ea
\gll Arà-biilang,    dee=ka    itthu=jo       anà-aada     katha=jo    arà-biilang. \\
     \textsc{non.past}-say 3=\textsc{loc} \textsc{dem.dist}=\textsc{emph} \textsc{past}-exist \textsc{quot}=\textsc{emph} \textsc{non.past}-say    \\
`They say that he had this with him.'
\z
}

\xbox{16}{
\ea
\gll [Dee arà-sbuuni   duuduk     {\em cave}] asàraathang  sini=ka asàduuduk hathu  {\em three}  {\em miles} cara  jaau=ka. \\
     3\textsc{.impolite} \textsc{non.past}-hide exist.\textsc{anim} cave \textsc{copula} here=\textsc{loc} from \textsc{indef} three miles way far=\textsc{loc}    \\
`The cave where he stayed hidden is  about three miles away from here.'
\z
}

\xbox{16}{
\ea
\gll {\em Three}  {\em miles}  cara  jaau=ka    aada  dee anà-sbuuni   duuduk     {\em cave}=yang.\footnotemark \\
    three miles war far=\textsc{loc} exist 3\textsc{.impolite} \textsc{past}-hide stay cave=\textsc{acc}    \\
`Three miles away from here is the cave where he stayed hidden.'
\z
}
\footnotetext{Note the use of the accusative marker \em =yang \em here for a locational predication, but its absence in the preceding sentence.}

\xbox{16}{
\ea
\gll Itthu    {\em cave}=nang   kithang=le      pii aada,  mà-liyath=nang. \\
     \textsc{dem.dist} cave=\textsc{dat} \textsc{1pl}=\textsc{addit} go exist \textsc{inf}-look=\textsc{dat}    \\
`We have also gone to the cave to have a look.'
\z
}

\xbox{16}{
\ea
\gll Bannyak jaau mà-pii    thàràboole. \\
     much far \textsc{inf}-go cannot    \\
`You can't go very far.'
\z
}

\xbox{16}{
\ea
\gll Itthu=ka       daalang=ka    mà-jaalang    naarath,\footnotemark{}   mlaarath.\ \\
      \textsc{dem.dist}=\textsc{loc} inside=\textsc{loc} \textsc{inf}-walk difficult difficult   \\
`Walking inside there  is difficult.'
\z
}
\footnotetext{\em mnaarath \em or \em naarath \em are occasionally found for \trs{mlaarath}{difficult}.}

\xbox{16}{
\ea
\gll Itthu=ka,       daalang=ka    giini    aada,  {\em sorungum}\footnotemark \\
     \textsc{dem.dist}=\textsc{loc} inside=\textsc{loc} like.this exist dungeon[Tamil]    \\
`Inside the cave, it is like a dungeon.'
\z
}
\footnotetext{Here a Tamil word is used to fill the lexical gap.}

\xbox{16}{
\ea
\gll Giithu=jo      thuu\u nduk abbis=jo       masà-pii. \\
     that.way=\textsc{emph} bend finish=\textsc{emph} must-go    \\
`You must really walk bowed down.'
\z
}

\xbox{16}{
\ea
\gll Kapang-pii,   gìllap   daalang=ka. \\
      when-go dark inside=\textsc{loc}   \\
`When you go, it's dark inside.'
\z
}

\xbox{16}{
\ea
\gll {\em vavvaal}\footnotemark{}  itthu    ni         aada,   {\em vavvaal}  katha binaathan      pada \\
      bat[Tamil] \textsc{dem.dist} \textsc{dem.prox} exist bat \textsc{quot} animal \textsc{pl}   \\
`There are vavvaal, the animals called vavvaal.'
\z
}
\footnotetext{The lexical gap is again filled with a Tamil word.}

\xbox{16}{
\ea different kind of birds what do you call this bats and all this, vavula[Sinh] \z
}

\xbox{16}{
\ea
\gll {\em vavvaal} \\
      bat[Tamil]   \\
\z
}

\xbox{16}{
\ea
\gll Kiccil vavvaal pada daalang=ka    arà-duuduk.\footnotemark{} \\
       small bat \textsc{pl} inside=\textsc{loc} \textsc{non.past}-exist.\textsc{anim}  \\
`There are small bats inside.'
\z
}
\footnotetext{Note the use of \em duuduk \em as an existential marker for [+animate] bats, which do not have the ability to sit, the original meaning of \em duuduk\em. This indicates the semantic bleaching of \em duuduk \em from a full verb to an existential.}

\xbox{16}{
\ea
\gll Suda mà-pii    thàràboole   daalang=ka,   light=le       mà-ambel    baapi thàràboole,  daalang=ka    pii=nang   blaakang   dhraapa   puukul=le      thama-kuthumung. \\
     so \textsc{inf}-go cannot inside=\textsc{loc} light=\textsc{addit} \textsc{inf}-take take.away cannot inside=\textsc{loc} go=\textsc{dat} after how.much hit=\textsc{addit} \textsc{neg.irr}-see    \\
`So you cannot go inside, you cannot take light inside, after going inside, however much he would hit them,\footnotemark{} no one would see.'
\z
}
\footnotetext{This probably refers to the victims of Saradiyel, although the precise meaning is unclear.}

\xbox{16}{
\ea
\gll {\em so} {\em dark} {\em inside}=ka\\
     so dark inside=\textsc{loc}    \\
\z
}

\xbox{16}{
\ea  it is just close by here\z 
}

\xbox{16}{
\ea  it was by this about 3 miles\z
}

\xbox{16}{
\ea
\gll mà-jaalang=jo       masà-pii     itthu=ka,       bukang \\
     \textsc{inf}-walk=\textsc{emph} must-go \textsc{dem.dist}=\textsc{loc} \textsc{tag}    \\
`You have to walk there, you know.'
\z
}

\xbox{16}{
\ea
\gll puunu   mlaarath  guunung  atthas=ka    ...     jaalang kapang-pii   athi-kuthumung    itthu=yang \\
     full difficult mountain top=\textsc{loc} ... walk when-go \textsc{irr}-see \textsc{dem.dist}=\textsc{acc}    \\
`Very difficult on top of the mountain. While walking, you can see all this.'
\z
}

\xbox{16}{
\ea
\gll Mavanella kapang-liyath   athi-kuthumung,   Mavanella {\em bridge}=ka     asàduuduk liyath kalu \\
     Mawanella when-see \textsc{irr}-see Mawanella bridge=\textsc{loc} from see if    \\
`When you look in Mawanella, you will see, if you look from the Mawanella bridge.'
\z
}

\xbox{16}{
\ea
\gll Saradiyel=pe     baathu arà-kuthumung \\
     Saradiyel=\textsc{poss} stone \textsc{non.past}-see    \\
`You see the Saradiyel Rock.'
\z
}
% 
% \xbox{16}{
% \ea
% \gll [incomprehensible],     sure       arà-kuthumung    itthu    baathu=yang \\
%      ......................,   sure \textsc{non.past}-see \textsc{dem.dist} stone=\textsc{acc}    \\
% `You see that rock.'
% \z
% }
% 
% \xbox{16}{
% \ea
% \gll guunung  guunung  guunung \\
%      mountain mountain mountain    \\ 
% \z
% }
% 
% \xbox{16}{
% \ea
% \gll [Dee duuduk\footnotemark{}     sbuuni anà-duuduk        thumpath]=le      asàdhaathang Buhaari \\
%       3\textsc{.impolite} stay hide \textsc{past}-exist.\textsc{anim} place=\textsc{addit} \textsc{copula} Buhaari   \\
% `The place where he stayed hidden is Buhari.'
% \z
% }
% \footnotetext{This is probably an anticipation of the \em duuduk \em which follows.}
% 
% \xbox{16}{
% \ea
% \gll Ithu     {\em bus} arà-binthi=kang,      Kandy {\em bus}, iiya iiya \\
%       \textsc{dem.dist} bus \textsc{non.past}-stop=\textsc{caus} Kandy bus yes yes   \\
% `The place where you\footnotemark{}\addtocounter{footnote}{-1} stop the Kandy bus, yeah.'
% \z
% }
% 
% \xbox{16}{
% \ea
% \gll Inni     {\em curve} arà-thaaro    [incomprehensible] \\
%     \textsc{dem.prox} curve \textsc{non.past}-put .............    \\
% `Where it\footnotemark{} takes that curve.'
% \z
% }
% \footnotetext{The agent has to be inferred from context here.}
% 
% \xbox{16}{
% \ea
% \gll Itthu    dìkkath=ka     aada,  duppang=ka    aada  ruuma pada. \\
%       \textsc{dem.dist} vicinity=\textsc{loc} exist front=\textsc{loc} exist house \textsc{pl}   \\
% `Close to there, in front, there are houses.'
% \z
% }
% 
% \xbox{16}{
% \ea
% \gll Benowan, Benowan, Benuwan is co.., Benuwan, ya, ya \\
%          \\
% \z
% }
% 
% \xbox{16}{
% \ea
% \gll Sithu=ka     aada  duppang=ka    aada  ruuma=pe     atthas=ka=jo       derang asà-duuduk     aada.   \\
%      there=\textsc{loc} exist front=\textsc{loc} exist house top=\textsc{loc}=\textsc{emph}  \textsc{3pl} \textsc{cp}-exist.\textsc{anim} exist   \\
% `There are some, in front are some, above the house some others have stayed.''
% \z
% }
% 
% \xbox{16}{
% \ea
% \gll Laile          ithu     ruuma pada aada, duppang=ka    sajja sdiikith laayeng-kang   aada. \\
%      still \textsc{dem.dist} house \textsc{pl} exist front=\textsc{loc} only few different=\textsc{caus} exist    \\
% `Those houses still remain, only the front portion has been renovated.'
% \z
% }
%  
% 
% \xbox{16}{
% \ea
% \gll Butthul, baaye baaye inni  {\em history} aada,  mà-liyath=nang      siini. \\
%       correct good good \textsc{dem.prox} history exist \textsc{inf}-see=\textsc{dat} here      \\
% `you have good history to see here.'
% \z
% }
% 
% \xbox{16}{
% \ea
% \gll Itthu=kapang,      kithang=pe     {\em first}, inni     {\em area}=pe     {\em police} {\em man} asàdhaathang, {\em police}  {\em station} asàdhaathang siini=jo,    baava=ka=jo        nya-aada \\
%      \textsc{dem.dist}=when \textsc{1pl}=\textsc{poss} first \textsc{dem.prox} area=\textsc{poss} police man \textsc{copula} police station \textsc{copula} here=\textsc{emph} bottom=\textsc{loc}=\textsc{emph} \textsc{past}-exist    \\
% `Then our first, this area's first police man, er, station was here, just below.'
% \z
% }   
% 
% \xbox{16}{
% \ea
% \gll polisiya wattha katha, mh.. \\
%      police[Sinh.] ground[Sinh.] \textsc{quot} mh..     \\
% `it was called Polisiya Wattha.'
% \z
% }
% 
% \xbox{16}{
% \ea
% \gll understand, ........ \\
%          \\ 
% \z
% }
% 
% \xbox{16}{
% \ea
% \gll police station \\
%          \\ 
% \z
% }
% 
% \xbox{16}{
% \ea
% \gll this area Mavanella Uthuvan Kandha area \\
%          \\ 
% \z
% }
% 
% \xbox{16}{
% \ea
% \gll Anà-aada     siini=jo. \\
%       \textsc{past}-exist here=\textsc{emph}   \\
% `It/They\footnotemark{} was here.'
% \z
% }
% \footnotetext{The referent is unclear here and in the following sentences.}
% 
% \xbox{16}{
% \ea
% \gll Inni     dìkkath=ka=jo        anà-aada,     sini=ka asàduuduk hathu  {\em half} {\em a}    {\em mile}=kee \\
%       \textsc{dem.prox} vicinity=\textsc{loc}=\textsc{emph} \textsc{past}-exist here=\textsc{loc} from \textsc{indef} half a mile=\textsc{simil}   \\
% `It close to here, about half a mile from here.'
% \z
% }
% 
% \xbox{16}{
% \ea
% \gll Sini=ka   asà-duuduk aada. \\
%      here=\textsc{loc} \textsc{cp}-exist.\textsc{anim} exist    \\
% `They lived here.'
% \z
% }
% 
% \xbox{16}{
% \ea
% \gll Sithu=ka=jo        aada,  itthu=nam  biilang laile Polisiya Wattha katha \\
%      \textsc{dem.dist}=\textsc{loc}   exist \textsc{dem.dist}=\textsc{dat} say still police[Sinh] ground[Sinh] \textsc{quot} \\
% `They are there, they still call it ``Polisiya Wattha'' there.'
% \z
% }
% 
% \xbox{16}{
% \ea
% \gll Itthu=ka=le            anà-duuduk    siyang    siyang mlaayu=jo. \\
%      \textsc{dem.dist}=\textsc{loc}=\textsc{addit} \textsc{past}-exist.\textsc{anim} before before Malay=\textsc{emph}    \\
% `Before, those who lived even there were the Malays.'
% \z
% }
% 
% \xbox{16}{
% \ea
% \gll Kithang=pe     neene       pada athi-biilang    itthu    muusing. \\
%    \textsc{1pl}=\textsc{poss} grandmother past \textsc{irr}-say \textsc{dem.dist} time      \\
% `Our grandmothers would say that at that time.'
% \z
% }
% 
% \xbox{16}{
% \ea
% \gll neene \\
%        grandmother  \\
% `grandmother'
% \z
% }
% 
% \xbox{16}{
% \ea
% \gll neene is grand mother \\
%          \\ 
% \z
% }
% 
% \xbox{16}{
% \ea
% \gll grand mother grand father they used to tell \\
%          \\ 
% \z
% }
% 
% \xbox{16}{
% \ea
% \gll there the police station is here \\
%          \\ 
% \z
% }
% 
% \xbox{16}{
% \ea
% \gll at Uthuvan Kandha that is the first police station lied here \\
%          \\ 
% \z
% }
% 
% \xbox{16}{
% \ea
% \gll but they also because they were very small \\
%          \\ 
% \z
% }
% 
% \xbox{16}{
% \ea
% \gll Derang pada buthul kiccil, neene       pada. \\
%      \textsc{3pl} \textsc{pl}  correct small grandmother \textsc{pl}    \\
% `They were very small, the grandmothers.'
% \z
% }
% 
% \xbox{16}{
% \ea
% \gll [Neene       pada anà-biilang    pada]\footnotemark=jo    itthu. \\
%      grandmother \textsc{pl}  \textsc{past}-say \textsc{pl}=\textsc{emph} \textsc{dem.dist}    \\
% `What the grandmothers said was this.'
% \z
% }
% \footnotetext{This headless relative clause is the first part of the equational sentence, the second part is \trs{itthu}{that}.}
% 
% \xbox{16}{
% \ea
% \gll Kithang=le arà-biilang    kithang=pe     cuucu      pada=nang   giini    giini=jo       ini      {\em period} mosthor inni=jo   katha. \\
%      \textsc{1pl}=\textsc{addit} \textsc{non.past}-say \textsc{1pl}=\textsc{poss} grandchild \textsc{pl}=\textsc{dat} like.this like.this=\textsc{emph} \textsc{dem.prox} period manner \textsc{dem.prox}=\textsc{emph} \textsc{quot}    \\
% `And we are also continuing with the same story to our grandchildren, just like this.'
% \z
% }
% 
% \xbox{16}{
% \ea
% \gll [Neene karang biilang pada],\footnotemark\addtocounter{footnote}{-1}{} kithang=pe     aanak pada=nang,   kithang arà-biilang \\
%      grandmother now say \textsc{pl} \textsc{1pl}=\textsc{poss} child \textsc{pl}=\textsc{dat} \textsc{1pl} \textsc{non.past}-say    \\
% `Now, what our grandmother tells us, we in turn relate it to our children.'
% \z
% }
% 
% \xbox{16}{
% \ea
% \gll [neene       pada anà-biilang    pada],\footnotemark kithang=pe     aanak pada=nang,   kithang=pe     cuucu      pada=nang   kithang   arà-biilang \\
%     grandmother \textsc{pl} \textsc{past}-say  \textsc{pl}  \textsc{1pl}=\textsc{poss} child \textsc{pl}=\textsc{dat} \textsc{1pl}=\textsc{poss} grandchild \textsc{pl}=\textsc{dat} \textsc{1pl} \textsc{non.past}-say     \\
% `What the grandmothers said, we relate it to our children, to our grandchildren.'
% \z
% }
% \footnotetext{This is a headless relative clause meaning `What the grandmothers told'.}
% 
% \xbox{16}{
% \ea
% \gll how it goes down \\
%          \\ 
% \z
% }
% 
% \xbox{16}{
% \ea
% \gll Lai     aada=si    aapa=ke      mà-caanya? \\
%      more exist=\textsc{irr} what=\textsc{simil} \textsc{inf}-ask    \\
% `What else is there to ask?\footnotemark'
% \z
% }
% \footnotetext{This question indicates that this topic is exhausted, and the speaker suggests to switch to another topic.}
% 
% \xbox{16}{
% \ea
% \gll Aapa=ke      mà-caanya    aada  lai,     lai     aapa? \\
%       what=\textsc{simil} \textsc{inf}-ask exist more more what   \\
% `Something else is there to ask? What else?'
% \z
% }
% 
% \xbox{16}{
% \ea
% \gll Lai     aapa oomong, aapa habaran       mà-caanya    aada,  lai? \\
%      what more talk what news \textsc{inf}-ask exist more    \\
% `What else do I say,  what other news is there to ask?'
% \z
% }
% 
% % \xbox{16}{
% % \ea
% % \gll aapa oomong maau, any        question  you want  to  know, a..ok \\
% %      what talk want     \\
% % `What do you want me to say?'
% % \z
% % }
