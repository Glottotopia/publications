wordclasses.tex:Sri Lankan Malay has a large class of verbs, which most often denote events like \trs{laari}{run}, but  states like \trs{thiidor}{sleep} can also be found in this class. All verbs share the following characteristics:
wordclasses.tex:Sri Lankan Malay has a large open class of nouns, which most often denote objects like \trs{ruuma}{house} or persons like \trs{aanak}{child}, but also some abstract concepts like \trs{watthu}{time}. Many abstract concepts are derived from another word class by means of the nominalizer \em -an\em, like \trs{makanan}{food} from \trs{maakang}{eat}.
wordclasses.tex:\gll Itthusubbath=jo incayang=nang \textbf{maau}, ini {\em Sri} {\em Lankan} {\em Malay} mà-blaajar \textbf{maau}. \\
wordclasses.tex:  therefore=\textsc{emph} 3p.\textsc{polite}=\textsc{dat} want \textsc{prox} Sri Lankan Malay \textsc{inf}-learn want  \\
wordclasses.tex:`This is why he wants to learn this Sri Lankan Malay.' (B060115prs15)
wordclasses.tex:`We want that you learn our [Sri Lankan] Malay, and we learn your [Malaysian] Malay.' (K060116nar02)
wordformation.tex:%``\el plurality in Sri Lankan Malay \el is, however, not indicated by reduplication byt by a plural marker `\em pada\em'  Adelaar (1991:32) states that [SLM] has borrowed this feature from Jakartanese where `\em pada\em' precedes the predicate and indicates plurality of the subject. Jakartanese, however, has borrowed this feature from Javanese.''\citet[]{Jayasuriya2001}
