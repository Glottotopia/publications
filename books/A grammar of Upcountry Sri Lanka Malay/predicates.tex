 \chapter{Predicates}\label{sec:pred}
Having discussed NPs and PPs, we now turn to different types of  predicates, which can take NPs or PPs as arguments.
While English basically has only two different predicate constructions, the verbal one and the non-verbal one supported by the copula \em to be\em, six types can be distinguished in SLM, with some further subdivisions. These are:

\begin{enumerate}
	\item The verbal predicate \formref{sec:pred:Verbalpredicates},
	\item The existential predicate \formref{sec:pred:Existentialpredicate}
	\item The modal predicate \formref{sec:pred:Modalpredicate}
	\item The nominal predicate \formref{sec:pred:Nominalpredicate}
	\item The circumstantial predicate \formref{sec:pred:Circumstantialpredicate}
	\item The adjectival predicate \formref{sec:pred:Adjectivalpredicate}
\end{enumerate}

The different predicate types can be identified by the way they are negated, as shown in Figure \ref{tab:NegationOfPredicatePhrases}. The differentiation between main type and subtypes is discussed in the respective sections themselves.

\begin{table}
	\begin{center}
	\begin{tabular}{l|llll}
						& PAST 			& PERFECT 	& PRESENT & FUTURE \\ 
						\hline
	VERBAL 		&   thàrà-V&  V thraa		&  thama-V &   thama-V\\
	EXISTENTIAL 	&   thraa(da)&   thraa(da)&   thraa  &   thama-aada\\
	MODAL \\
	~~~~boole	&   thàrboole&   thàrboole&   thàrboole&   thàrboole\\
	~~~~maau		&   thussa		&   thussa	&   thussa  	&   thussa\\
	~~~~kamauvan	&   thàrkamauvan &   thàrkamauvan	&   thàrkamauvan  	&   thàrkamauvan\\
	NOMINAL 	&   bukang  &  bukang   &  bukang 	&   thama-jaadi/bukang\\
	~~~~identificational\footnotemark & & & bukang &    \\
	CIRCUMSTANTIAL\footnotemark 	&   bukang	&  bukang	&  bukang 	&  ? \\  
	~~~~LOCATION 	&    thraa &   thraa &   thraa & thama-aada  \\
	ADJECTIVE \\	
	~~~~ADJ1	&   ADJ thraa&   ADJ thraa&   ADJ thraa&  TAM-ADJ \\
	~~~~ADJ2	&   thàrà-ADJ&   thàrà-ADJ&   thàrà-ADJ&  TAM-ADJ\\
	\end{tabular}
	\end{center}
\caption{Negation of different types of predicative phrases}
\label{tab:NegationOfPredicatePhrases}
\end{table}

\footnotetext{Identificational predicates always have generic time reference. It is impossible for my father's brother to be the same person as my uncle at point A and no longer to be the same person at point B.}
\footnotetext{Circumstantial  predicates which are true in the present but will no longer be true in the future are difficult to conceive, and for those, there is no information available.}

\section{Verbal predicates}\label{sec:pred:Verbalpredicates}
Verbal predicates are very frequent in SLM. They consist of a verb (or a converted adjective), typically marked for TAM.\footnote{Although TAM-marking is optional (\citet[cf.][143]{Slomanson2006cll}, \citet[169]{SmithEtAl2006cll}).} This structure is given in \xref{cbx:constr:predicate:verbal}. An illustrative example is \xref{ex:constr:pred:illustration}. More detailed discussion can be found below \formref{sec:pred:Thestandardverbalpredicate}.

\cbx[\label{cbx:constr:predicate:verbal}]{TAM-V}{VPRED}

\xbox{16}{
\ea\label{ex:constr:pred:illustration}
\gll Ini      gaaja    \textbf{su-pii}. \\
      \textsc{prox} elephant \textsc{past}-go \\
    `This elephant left.' (B060115nar05)
\z
}

It is also possible to use two verbs in one verbal predicate. Two cases can be distinguished and will be discussed below: combination of a verb with a vector verb \xref{ex:constr:pred:vectorverb} \formref{sec:pred:Verbalpredicateswithavectorverb} and combinations of two full verbs \xref{ex:constr:pred:serialverb} \formref{sec:pred:Verbalpredicateswithtwofullverbs}.

\cbx{TAM-V $\left\{\begin{array}{l} \rm V\\ \rm VectorV\end{array}\right\}$}{VPRED}

\xbox{16}{
 \ea\label{ex:constr:pred:vectorverb}
   \gll Itthukapang=jo         derang \textbf{nya-thaau}$_{fullverb}$   \textbf{ambel}$_{vectorverb}$ derang pada {\em politic}=nang   suuka katha. \\
    then=\textsc{emph} \textsc{3pl} \textsc{past}-know take \textsc{3pl} \textsc{pl} politic=\textsc{dat} like \textsc{quot}\\
`Only then will they come to know that they like politics' (K051206nar12)
\z
}


\xbox{16}{
\ea\label{ex:constr:pred:serialverb}
\gll Hathu haari, hathu oorang thoppi mà-juval=nang kampong=dering kampong=nang \textbf{su-jaalang}$_{fullverb}$ \textbf{pii}$_{fullverb}$. \\
     \textsc{indef} day \textsc{indef} man hat \textsc{inf}-sell=\textsc{dat} village=\textsc{abl} village=\textsc{dat} \textsc{past}-walk go  \\
    `One day, a man walked from village to village to sell hats.'  (K070000wrt01)
\z      
}


\subsection{Standard verbal predicate}\label{sec:pred:Thestandardverbalpredicate}
In all verbal predicates, the verb can be modified by an adjective \xref{ex:vpred:mod:adj}, a postpositional phrase with \em =nang \em \xref{ex:vpred:mod:nang1}\xref{ex:vpred:mod:nang2} or \em =ka \em  \xref{ex:vpred:mod:ka1}\xref{ex:vpred:mod:ka3}, a comparative with \em =ke \em \xref{ex:vpred:mod:ke} or a secondary predicate with a reduplicated verb \xref{ex:vpred:mod:redup}. This is formalized in \xref{cbx:constr:predicates:standardverbal}.


\cbx[\label{cbx:constr:predicates:standardverbal}]{$\left\{\begin{array}{l} \rm ADJ\\\rm NP=\textit{nang}\\\rm NP=\textit{ka}\\ \rm CLS=ke\\ \rm V_i\sim V_i\\\end{array}\right\}$   TAM-V}{VPRED}


\xbox{16}{
\ea\label{ex:vpred:mod:adj}
\gll Kithang \textbf{baaye} mlaayu arà-oomong katha incayang biilang thraa. \\
      1\textsc{pl} good Malay \textsc{non.past}-speak \textsc{quot} \textsc{3s.polite} say \textsc{neg} \\
    `He has not said that we speak good Malay.'  (B060115prs15)
\z      
}

\xbox{16}{
\ea\label{ex:vpred:mod:nang1}
\gll Bras=iyang \textbf{baaye=nang} \textbf{cuuci}. \\
 raw.rice=\textsc{acc} good=\textsc{dat} wash\\
`Wash the rice well!' (K060103rec01)
\z
}


\xbox{16}{
\ea\label{ex:vpred:mod:nang2}
\ea 
\gll Luu \textbf{baaye=nang} masà-blaajar \textbf{baaye=nang} masà-mnaaji. \\
      \textsc{2s.familiar} good=\textsc{dat} must-learn good=\textsc{dat} must-recite \\
    `You have to learn well and you have to recite well.'
\ex
\gll Lu=ppe umma-baapa=nang \textbf{baaye=nang} masà-kaasi thaangang. \\
      \textsc{2s}=\textsc{poss} mother-father=\textsc{dat} good=\textsc{dat} must-give hand \\
    `You must lend a hand to your parents.' (K060116sng01)
\z
\z
}

% 
% \xbox{16}{
% \ea\label{ex:vpred:mod:nang3}
% \gll Karang see siithu pukurjan arà-jalang-kang, \textbf{hathu} \textbf{{\em engineer}} \textbf{mosthor=nang}. \\
%      now 1w there work \textsc{non.past}-walk-\textsc{caus} \textsc{indef} engineer way=dat\\
%     `Now I run the work over there like an engineer.'  (K061026prs01)
% \z      
% }\\


\xbox{16}{
\ea\label{ex:vpred:mod:ka1}
\gll Itthu    vakthu=ka    hathu  bìssar beecek caaya  Buruan \textbf{mlaarath}=\textbf{ka}     uuthang=dering  luvar=nang     su-dhaathang. \\
      \textsc{dist} time=\textsc{loc} \textsc{indef} big brown colour bear difficulty=\textsc{loc} forest=\textsc{abl} outside=\textsc{dat} \textsc{past}-come \\
    `Then, a big brown bear came out of the woods with difficulties.' (K070000wrt04)
\z
}

% 
% \xbox{14}{
% \ea\label{ex:vpred:mod:ka2}
% \gll Derang anà-duuduk samma vakthu=ka=le derang suuka=\textbf{ka} vakthu anà-empas.  \\
%       \textsc{3pl} \textsc{past}-stay all time=\textsc{loc}=\textsc{addit} \textsc{3pl} like=\textsc{loc} time \textsc{past}-spend \\
%     `All the time when they were together, they spent the time very happily.' (K070000wrt04)
% \z
% }

\xbox{14}{
\ea\label{ex:vpred:mod:ka3}
\gll Aajuth thaakuth=\textbf{ka} su-naangis,  \\
    dwarf fear=\textsc{loc} \textsc{past}-weep   \\
    `The dwarf wept in fear.' (K070000wrt04)
\z
}


\xbox{16}{
\ea\label{ex:vpred:mod:ke}
\gll Itthu pada [sraathus binthan pada arà-kiilap]$_{CLS}$=\textbf{ke} su-kiilap. \\ % bf
     \textsc{dist} \textsc{pl} 100 star \textsc{pl} \textsc{simult}-shine=\textsc{simil} \textsc{past}-shine  \\
    `They shone like a hundred stars.'  (K070000wrt04,K081105eli02)
\z      
}


% B060115nar04.txt: kàthaama asàduudukke            anaduuduk

\xbox{16}{
\ea\label{ex:vpred:mod:redup}
\gll Kancil \textbf{lompath}\~{ }\textbf{lompath} arà-laari. \\
     rabbit jump\~{ }\textsc{red}      \textsc{non.past}-run \\
    `The rabbit runs away jumping.'  (K081104eli06)
\z      
}


The full structure of the verbal predicate phrase, combining \xref{cbx:constr:predicate:verbal} and \xref{cbx:constr:predicates:standardverbal} in then as follows:


\cbx{$\left\{\begin{array}{l} \rm ADJ\\\rm NP=\textit{nang}\\\rm NP=ka\\\rm CLS=ke \\\rm V_i\sim V_i\\\end{array}\right\}$   TAM-V $\left(\begin{array}{l} \rm V\\ \rm VectorV\end{array}\right)$}{VPRED}



Verbal predicates take between 0 and 3 arguments. Zero-place verbs are represented by \trs{uujang}{rain} in \xref{ex:vp:val:0}. A monovalent verb is shown in \xref{ex:vp:val:1}, a bivalent verb in \xref{ex:vp:val:2}. The trivalent verb \trs{kaasi}{give} is shown in \xref{ex:vp:val:3}. \xref{ex:vp:val:4} shows a verb with four participants, but their status as arguments might be debatable. As discussed in \formref{sec:argstr} and \formref{sec:gramrel}, the distinction between arguments and adjuncts is not easy to make in SLM and might not be relevant at all.

\xbox{16}{
\ea\label{ex:vp:val:0}
\gll \zero{} anthi-uujang. \\ % bf
       { } \textsc{irr}-rain\\
`It will rain.' (K081104eli06)
\z
}


\xbox{16}{
\ea\label{ex:vp:val:1}
\gll [Mà-poothong oorang pada]$_{NP1}$ su-dhaathang. \\ % bf
     \textsc{inf}-cut man \textsc{pl} \textsc{past}-come\\
    `The cutters came.'  (K051205nar05)
\z      
}


%\xbox{16}{
%\ea\label{ex:vp:val:2}
%\gll Sdiikith aari=nang blaakang Snow-white Aanak raaja=yang=le, Rose-red incayang=pe sudaara=yang=e su-kaaving. \\
%      few day=\textsc{dat} after Snow-white child king=\textsc{acc}=\textsc{addit} Rose-red 3.polite=\textsc{poss} brother=\textsc{acc}=\textsc{addit} \textsc{past}-marry  \\
%    `Few days later, Snow white married the prince, and Rose red married his brother.'  (K070000wrt04)
%\z      
%}

\xbox{16}{
\ea\label{ex:vp:val:2}
\gll Itthusubbath,   [deram pada]$_{NP1}$    [jaalang]$_{NP2}$ arà-kijja. \\  % bf
	therefore \textsc{3pl} \textsc{pl} street \textsc{non.past}-make \\
    `Therefore, they build the street.'  (K051222nar04)
    \z      
}


\xbox{16}{
\ea\label{ex:vp:val:3}
\gll [Se=ppe baapa]$_{NP1}$ [incayang]$_{NP2}$=nang [ummas]$_{NP3}$ su-kaasi. \\ % bf
      \textsc{1s}=\textsc{poss} father \textsc{3s.polite}=\textsc{dat} gold \textsc{past}-give\\
    `My father gave him gold.'  (K070000wrt04)
\z      
}


\xbox{16}{
\ea\label{ex:vp:val:4}
\gll [Itthu    baathu]$_{NP1}$=yang   [incayang]$_{NP2}$ [Seelong]$_{NP3}$=dering   [laayeng    nigiri]$_{NP4}$=nang asà-baapi. \\ % bf
 \textsc{dist} stone=\textsc{acc} \textsc{3s.polite} Ceylon=\textsc{abl} other country=\textsc{dat} \textsc{cp}-bring\\
`These stones, he brought them from Ceylon to other countries.' (K060103nar01)
\z
}



Verbal predicates are negated by the prefix \em thàrà- \em in the past \xref{ex:vp:neg:past}, postverbal \em thraa \em in the perfect \xref{ex:vp:neg:perf} and the quasi-prefix \em thama- \em in non-past tenses \xref{ex:vp:neg:nonpast}.


\xbox{16}{
\ea\label{ex:vp:neg:past}
\gll Ikkang           Seelon=ka  kitham=nang      gaaji  \textbf{thàrà-}sampe. \\ % bf
     then Ceylon=\textsc{loc} \textsc{1pl}=\textsc{dat} salary \textsc{neg.past}-reach  \\
    `Then, the salary was not enough for us in Sri Lanka.'  (N060113nar04)
\z      
}


\xbox{16}{
\ea\label{ex:vp:neg:perf}
\gll {\em Invitations}      daapath \textbf{thraa}. \\
     invitations get \textsc{neg}  \\ % bf
    `The invitations had not been received.'  (K060116nar11)
\z      
}

\xbox{16}{
\ea\label{ex:vp:neg:nonpast}
\gll Kitham=pe      aanak pada \textbf{thama}-oomong. \\
 \textsc{1pl}=\textsc{poss} child \textsc{pl} \textsc{neg.nonpast}-speak \\ % bf
`Our children do not speak.' (G051222nar01)
\z
}

Verbal predicates can be seen in all major clause types in SLM. All main clause constructions and all subordinate clause constructions can contain a verbal predicate. The most ample research on verbal predications  to date is found in \citep{Slomanson2006cll}.


\subsection{Verbal predicates with a vector verb}\label{sec:pred:Verbalpredicateswithavectorverb}
Verbal predicates with a vector verb consist of a full verb and a vector verb \formref{sec:wc:Vectorverbs}.  The vector verb occurs always after the full verb. Vector verb and full verb are parsed into two different phonological words, which distinguishes them from compound verbs  \formref{sec:wofo:Compoundsinvolvingtwoverbs}. The vector verb modifies the full verb and adds aspectual meaning to it.


\xbox{16}{
\ea\label{ex:constr:pred:vector:ambel}
\gll Kanabisan=ka=jo duva oorang=le \textbf{anà-thaau} \textbf{ambel} [Andare duva oorang=yang=le asà-enco-kang aada] katha. \\
      last=\textsc{loc}=\textsc{emph} two man=\textsc{addit} \textsc{past}-know take Andare two man=\textsc{acc}=\textsc{addit} \textsc{cp}-fool-\textsc{caus} exist \textsc{quot} \\
    `At the very end, both women understood that Andare had fooled both of them.' (K070000wrt05)
\z
}

In example \xref{ex:constr:pred:vector:ambel}, the two women were at first not aware of the trick being played on them. Finally, they discover it, and they change from a state of ignorance to a state of knowledge. This aspectual information is encoded by the vector verb \em ambel\em. Note that \em ambel \em does not have its normal meaning \em take \em in this context.

\subsection{Verbal predicates with two full verbs}\label{sec:pred:Verbalpredicateswithtwofullverbs}
Verbal predicates with two full verbs are serial verb constructions where none of the participating verbs is a vector verb. This is a lot less frequent than the construction involving vector verbs. Just like the preceding construction, the verbs in this construction are parsed into two phonological words, which distinguishes them from compound verbs: \formref{sec:wofo:Compoundsinvolvingtwoverbs}. All the examples of this construction in the corpus deal with some kind of motion.

The first example involves the neutral verb \trs{pii}{go}. In this context, we are dealing with one event of walking, which is coded by two words, \trs{jaalang}{walk}, indicating the manner, and \trs{pii}{go} indicating motion away from the center. 


\xbox{16}{
\ea\label{ex:constr:pred:v:svc:pii}
\gll Hathu haari, hathu oorang thoppi mà-juval=nang kampong=dering kampong=nang su-\textbf{jaalang} \textbf{pii}. \\
     \textsc{indef} day \textsc{indef} man hat \textsc{inf}-sell=\textsc{dat} village=\textsc{abl} village=\textsc{dat} \textsc{past}-walk go  \\
    `One day, a man went and walked from village to village to sell hats.'  (K070000wrt01)
\z      
}

Care must be taken to not confound serial verbs of this type with verbs of motion with a purposive clause, such as \xref{ex:constr:pred:v:svc:contr:inf}, where the purpose of the movement is indicated by the infinitve marker \em mà-\em, and additionally by the dative marker \em =nang\em.


\xbox{14}{
\ea\label{ex:constr:pred:v:svc:contr:inf}
\gll Derang [dìkkath=ka     aada  laapang]=nang   \textbf{mà}-maayeng=\textbf{nang}      su-pii. \\
       \textsc{3pl} vicinity=\textsc{loc} exist ground=\textsc{dat} \textsc{inf}-play=\textsc{dat} \textsc{past}-go. \\
    `They went  to the nearby ground to play.' (K070000wrt04)
\z
}


 

An example of full verb serialization with a more specific verb, \trs{laari}{run} is \xref{ex:constr:pred:v:svc:kluuling}.


\xbox{16}{
\ea\label{ex:constr:pred:v:svc:kluuling}
\gll Aanak su-\textbf{laari} \textbf{kluuling}. \\
       child \textsc{past}-run roam\\
    `The child went astray.' (K061019sng01)
\z
}

% In example \xref{ex:constr:pred:v:svc:laari}, \trs{laari}{run} is the first verb, whereas in the following example, it is the second one. There is thus no fixed order of motion verb and non-motion verb.
% 
% \xbox{16}{
% \ea\label{ex:constr:pred:v:svc:laari}
% \gll Incayang  thee\u mbak abbis,    \textbf{salba}  \textbf{laari} aada  tumpath=nang. \\
%       \textsc{3s.polite} shoot    finished escape run   exist place=\textsc{dat}  \\
%     `After the shooting he escaped to the (aforementioned) place.' (K051206nar02)
% \z
% } \\  

Motion towards the deictic center is also possible in a serial verb construction, as \trs{dhaathang}{come} shows in example \xref{ex:constr:pred:v:svc:dhaathang1} and \xref{ex:constr:pred:v:svc:dhaathang2}.



\xbox{16}{
\ea\label{ex:constr:pred:v:svc:dhaathang1}
\gll Oorang pada kapang-\textbf{laari}   \textbf{dhaathang}, ini daara sgiithu=le  suusu su-jaadi. \\
    man \textsc{pl} when-run come \textsc{prox} blood that.much=\textsc{addit} milk \textsc{past}-become   \\
    `When people came running, the blood had turned into milk.' (K051220nar01)
\z
}


\xbox{16}{
\ea\label{ex:constr:pred:v:svc:dhaathang2}
\gll See=yang asà-\textbf{caari} \textbf{dhaathang}=apa. \\
     \textsc{1s}=\textsc{acc} \textsc{cp}-search come=after  \\
    `he came in search of me and ...' (K051213nar06)
\z
}
% 
% In a combination of the emphatic clitic \em =jo \em and the conjuctive participle prefix \em asà-\em, the two verbs can be split apart by those two elements.
% 
% \xbox{16}{
% \ea\label{ex:constr:pred:v:svc:asajo}
% \gll Itthu    guunung=ka     \textbf{naayek=jo}    \textbf{asà-dhaathang} arà-attack-kang      itthu oorang pada=yang. \\
%     \textsc{dist} mountain=\textsc{loc} climb=\textsc{emph} \textsc{cp}-come \textsc{non.past}-attack-\textsc{caus} \textsc{dist} man \textsc{pl}=\textsc{acc}   \\
%     `He came climbing up the mountain and attacked those men.' (K051206nar02)
% \z
% } \\


An example with the motion verb hidden in more elaborate semantics are found in \xref{ex:constr:pred:v:svc:baapi}\xref{ex:constr:pred:v:svc:angkath}, where the word \trs{baapi}{take.to.a.place} implies motion.

\xbox{16}{
\ea\label{ex:constr:pred:v:svc:baapi}
   \gll Daalang=ka  {\em light}=le    mà-\textbf{ambel} \textbf{baapi} thàràboole. \\
     inside=\textsc{loc} light=\textsc{addit} \textsc{inf}-take bring cannot \\
`You cannot take a light either.' (K051206nar02)
\z
}


\xbox{16}{
\ea\label{ex:constr:pred:v:svc:angkath}
\gll Aajuth=yang buurung mà-angkath baapi su-diyath.\footnotemark{} \\
      dwarf=\textsc{acc} bird \textsc{inf}-lift take.away \textsc{past}-try \\
    `The bird tried to carry the dwarf away.' (K070000wrt04)
\z
}
\footnotetext{It can actually be argued that the \em baapi diiyath \em in \xref{ex:constr:pred:v:svc:angkath} involves three verbs because \trs{baapi}{bring/take} can be segmented in to \trs{baa(wa)}{bring} and \trs{pii}{go}.}

In all the examples given in this section but one, the left border of the serial verb is indicated by a preceding TAM-prefix.
The right border is indicated by
\begin{itemize}
 \item a postposition \xref{ex:constr:pred:v:svc:dhaathang2},
 \item a modal \xref{ex:constr:pred:v:svc:baapi},
 \item or the end of the clause \xref{ex:constr:pred:v:svc:pii}, \xref{ex:constr:pred:v:svc:kluuling}, \xref{ex:constr:pred:v:svc:dhaathang1}, \xref{ex:constr:pred:v:svc:angkath}.
\end{itemize}

There are other examples where the right border is an argument. These have not been selected as examples here because of the difficulty to ascertain whether the argument forms part of the serial verb construction or not. With postpositions, modals, or the end of the clause, it is clear that they are  not part of the serial verb constructions.  

Between the left and the right boundary, only verbs are found. There is  no intervening material. This distinguishes  serial verbs from clause chains involving \em asà- \em \formref{sec:cls:Conjunctiveparticipleclause}.


The meaning of the serial verb constructions involving two full verbs is transparent and compositional as can be seen from the examples above.  Three of the examples above are intransitive \xref{ex:constr:pred:v:svc:pii}\xref{ex:constr:pred:v:svc:kluuling}\xref{ex:constr:pred:v:svc:dhaathang1}, while the others are transitive. For the transitives, both component verbs assign agent to the same referent, but the patient assigned by the transitive component verb is not present in \trs{dhaathang}{come} in \xref{ex:constr:pred:v:svc:dhaathang2}. In \xref{ex:constr:pred:v:svc:baapi}, both verbs \em ambel \em and \em baapi \em agree in assigning the agent to a generic entity and the patient to a lamp. There are thus no mismatches between the assignments of the semantic roles to the component verbs. An analogous analysis is possible for \trs{angkath}{lift} and \trs{baapi}{take.away} in \xref{ex:constr:pred:v:svc:angkath}.

Given these facts, full verb serialization in SLM can be analyzed as an instance of `nuclear juncture' \citep{FoleyEtAl1985, VanValinEtAl1997rrg}. In nuclear serialization, the conjoined verbs take a single set of actor and undergoer arguments and cannot be modified or negated independently, whereas in `core juncture' (not found in SLM), each verb selects its arguments independently, and the verbs can individually be negated or modified. A final type, `clause juncture', is distinct from the two types discussed in that identity of core arguments is not required. This type is instantiated by clause chains in SLM \formref{sec:cls:Conjunctiveparticipleclause}.

% \xbox{16}{
% \ea\label{ex:constr:pred:NP:prenom:}
% \gll [Hathyang]$_{ADJ}$ [thiiga]$_{NUM}$ [oorang]$_{N}$=yang {\em Malaysia}=nang su-ambel baapi. \\
%      next three man=\textsc{acc} Malaysia=\textsc{dat} \textsc{past}-take bring \\
%     `They took three more men to Malaysia.'  (K060108nar02)
% \z      
% }



\section{Existential predicate}\label{sec:pred:Existentialpredicate}
\subsection{Standard existential predicate}
The second type of predicate is the existential predicate, in which one of the existential verbs \formref{sec:wc:Existentialverbs:intro} \em aada \em or \em duuduk \em is used.

\cbx{
	$TAM-
	\left\{
		\begin{array}{l}
			 aada\\
			 duuduk
		\end{array}
	\right\}
	$
}{VPRED_{exist}}

Example \xref{ex:constr:pred:exist:intro} illustrates this pattern.

\xbox{16}{
\ea\label{ex:constr:pred:exist:intro}
\gll {\em Problem}  pada=le      \textbf{aada}. \\
      problem \textsc{pl}=\textsc{addit} exist\\
    `There are also problems.'  (N060113nar04)
\z
}




\em Aada \em cannot take the progressive prefix \em arà-\em, while \em duuduk \em can and normally does. Both can be inflected for the remaining tenses. The negation of both is normally simply \em thraa \em \xref{ex:pred:exist:thraa1}\xref{ex:pred:exist:thraa2}\xref{ex:pred:exist:thraa3}.
%, but the negation of \em aada \em can also be \em thraada \em \xref{ex:pred:exist:thraada}\citep[cf.][]{Slomanson2008lingua}.



\xbox{16}{
\ea\label{ex:pred:exist:thraa1}
\gll {\em Malay}  {\em political}  {\em party}  atthu  \textbf{thraa}. \\
    Malay  political  party  \textsc{indef} \textsc{neg}  \\
    `There is no Malay political party.'  (K051206nar12)
\z      
}

\xbox{16}{
\ea\label{ex:pred:exist:thraa2}
\gll Se=ppe umma \textbf{thraa}. \\
 \textsc{1s}=\textsc{poss} mother  \textsc{neg}\\
`My mother has passed away.' (B060115prs03)
\z
}


\xbox{16}{
\ea\label{ex:pred:exist:thraa3}
   \gll kàthaama su-aada,  karang \textbf{thraa}. \\
   before \textsc{past}-exist, now  \textsc{neg}\\
`Before there was rugby,  now there is no rugby' (B060115cvs01)
\z
}

% \xbox{16}{
% \ea\label{ex:pred:exist:thraada}
% \gll Giithu \textbf{thraada} kalu. \\
%      that.way \textsc{neg}.exist if  \\
%     `If that is not available'  (K060103rec01)
% \z      
% }

\em Duuduk \em in its existential reading, on the other hand, cannot be negated by \em thàrà-duuduk\em, because this forces a reading as `stay' or  `sit'. Simple \em thraa \em has to be used instead (cf. \xref{ex:pred:exist:thraa2}). So, the interrogative existential sentence in \xref{pred:exist:duuduk:affirm} can be formed with \em duuduk\em, but a negative sentence with \em duuduk\em, like \xref{pred:exist:duuduk:neg}, does not express non-existence, but non-location, i.e. the addressee did indeed happen to exist at that point in time, during the fasting period, but he did not stay at the relevant place. If non-existence has to be expressed, \em thraa \em has to be used, as in \xref{ex:pred:exist:thraa2}, rather than a negation of \em duuduk\em.

\xbox{16}{
\ea\label{pred:exist:duuduk:affirm}
\gll Sudaara sudaari pada arà-\textbf{duuduk}=si? \\
     brother sister \textsc{pl} \textsc{non.past}-exist.\textsc{anim}=\textsc{interr}  \\
    `Do there exist (=Do you have) any brothers or sisters?'  (B060115cvs03)
\z      
}


\xbox{16}{
\ea\label{pred:exist:duuduk:neg}
\gll Puaasa  muusing thàrà-\textbf{duuduk}=si,       {\em fasting}      {\em period}=ka? \\
     fasting season \textsc{neg.past}=\textsc{interr}  fasting period=\textsc{loc} \\
    `You were not here during the fasting period, were you?'  (B060115cvs03)
\z      
}

This restriction only holds for the existential reading. For the locational reading, both negations (\em thàraduuduk \em and \em thraa \em are possible).


% \xbox{16}{
% \ea\label{ex:constr:pred:unreferenced}
% \gll Mlaayu=ka=jo bannyak avuliya Seelon=ka aada. \\
%      Malay=\textsc{loc}=\textsc{emph} many saint Ceylon=\textsc{loc} exist  \\
%     `Among the Malays there are many saints in Sri Lanka.'  (K060108nar02)
% \z      
% }

The existential predicate is used for five different, but interrelated functions: 

\begin{itemize}
	\item presentationals  \xref{ex:pred:mod:presentationals}
	\item existence  \xref{ex:pred:mod:existence}
	\item availability \xref{ex:pred:mod:availability}
	\item location as a special type of existence \xref{ex:pred:mod:location}
	\item possession as a special type of existence \xref{ex:pred:mod:possession}
\end{itemize}


\xbox{16}{
\ea\label{ex:pred:mod:presentationals}
\gll Hathu muusing=ka ...  hathu  kiccil ruuma su-\textbf{aada} \\
     \textsc{indef} time=\textsc{loc} ... \textsc{indef} small house \textsc{past}-exist  \\
    `Once upon a time, there was a small house.'  (K07000wrt04)
\z      
}
 
 
\xbox{16}{
\ea\label{ex:pred:mod:existence}
\gll Karang Kluu\u mbu=ka    mlaayu pada \textbf{aada}. \\
     now Colombo=\textsc{loc} Malay \textsc{pl} exist.\textsc{anim}  \\
    `Now, there are many Malays in Colombo.'  (N060113nar01)
\z      
}

\xbox{16}{
\ea\label{ex:pred:mod:availability}
\gll Itthu blaakang ini karang \textbf{santham} \textbf{aada}, bukang. \\
     \textsc{dist} after \textsc{prox} now coconut.milk exist \textsc{tag}  \\
    `Now there is this coconut milk, you know.'  (B060115rcp02)
\z      
}

\xbox{16}{
\ea\label{ex:pred:mod:location}
\gll Se=ppe      dhaatha=pe           thiiga aanak=le      \textbf{Dubai=ka}     arà-\textbf{duuduk}. \\
 \textsc{1s}=\textsc{poss} elder.sister=\textsc{poss} three child=\textsc{addit} Dubai=\textsc{loc} \textsc{non.past}-exist.\textsc{anim} \\
`My sister's three children also live in Dubai.' (B060115prs21)
\z
}

\xbox{16}{
\ea\label{ex:pred:mod:possession}
\gll Thiiga klaaki aade=le, atthu  pompang  aade=le \textbf{se=dang}     \textbf{arà-duuduk}. \\
three male younger.sibling one female younger.sibling=\textsc{addit} \textsc{1s.dat} \textsc{non.past}-exist.\textsc{anim}\\
`I have three younger brothers and one younger sister, too.' (K060108nar01)
\z
}



%  
% \xbox{16}{
% \ea\label{ex:constr:pred:exist:intro}
% \gll Kithang=pe nigiri=ka bedahan aada.  \\
%      \textsc{1pl}=\textsc{poss} country=\textsc{loc} difference exist.   \\
%     `There are differences in our country.' (K061127nar03)
% \z
% } \\


\subsection{Possessive predicate}\label{sec:pred:Possessivepredicate}
The possessive predicate construction consists of an existential predicate with additional indication `to whom' the entity referred to exists, i.e. who possesses it \funcref{sec:func:Possession}. Two types of possessive relationship are distinguished: permanent and temporary. In case of permanent possession, the possessor is marked with \em =nang \em \xref{ex:pred:poss:perm} \citep[26]{Ansaldo2005ms}, in case of temporary possession, the possessor is marked with \em =ka \em \xref{ex:pred:poss:temp}. The choice of the existential verb (\em aada/duuduk\em) is the same as for the general existential construction.


\xbox{16}{
\ea\label{ex:pred:poss:perm}
\gll \textbf{Se=dang} liima anak  klaaki pada \textbf{aada}. \\
      \textsc{1s.dat} five child male \textsc{pl} exist \\
    `I have five sons.' (permanent) (K060108nar02)
\z
}


\xbox{16}{
\ea\label{ex:pred:poss:temp}
\gll \textbf{Incayang=ka} ... bìssar beecek caaya hathu {\em bag} \textbf{su-aada}. \\
     \textsc{3s.polite}=\textsc{loc} ... big mud colour \textsc{indef} bag \textsc{past}-exist  \\
    `He had a big brown bag with him.' (temporary) (K070000wrt04)
\z
}



This predicate type can be used in both main clauses and subordinate clauses, as the following example shows, where the location of the money in the bank is expressed in a subordinate clause, and the possession of the money by Mr Yusuf is expressed in the main clause, both clauses making use of \em aada\em.

\xbox{16}{
\ea\label{ex:pred:mod:possession:mainsubordcls}
\gll Mr. Yusuf karang incayang=nang$_{i}$ ini [[[{\em private} {\em bank}]=ka$_{j}$ aada$_{j}$] duvith]$_{i,j}$ pada aada$_{i}$, bukang \\
     Mr Yusuf  now \textsc{3s.polite}=\textsc{dat} \textsc{prox} private bank=\textsc{loc} exist money \textsc{pl} exist, tag\\
    `Now, Mr Yusuf, he had the money that was at that private bank, isn't it?'  (K060116nar09)
\z      
}


\section{Modal predicate}\label{sec:pred:Modalpredicate}
A third predicate type is the modal predicate. It consists of a nominal argument or a clause in the infinitive followed by one of the four modal particles \trs{boole}{can}, \trs{thàrboole}{cannot}, \trs{(ka)mau(van)}{want} and \trs{thussa}{\textsc{neg}.want}. The entity for which the modal predicate holds is in the dative case, marked by \em =nang\em.

\cbx{	
			$	
			\left\{
			\begin{array}{c}
			\rm NP\\
			\rm \NP* \textit{mà}-V\\
			\end{array}
		\right\}
		\left\{
		\begin{array}{c}
        boole\\th\grave{a}rboole\\(th\grave{a}r)(ka)mau(van)\\thussa\\
		\end{array}
		\right\}
		$		
}{MODPRED}

\xref{ex:pred:modal:nominal} shows the use of this predicate type with a nominal complement, \xref{ex:pred:modal:clausal} with a clausal complement.

\xbox{16}{
\ea\label{ex:pred:modal:nominal}
\gll Deran=\textbf{nang}    [thumpath]$_{N}$ \textbf{maau}. \\ % bf
     \textsc{3pl=dat} place want  \\
    `They wanted land.'  (N060113nar01)
\z      
}

\xbox{16}{
\ea\label{ex:pred:modal:clausal}
\gll Derang pada=\textbf{nang} [atthu=le \textbf{mà}-kijja]$_{\textsc{cls}}$=nang  \textbf{thàràboole}. \\ % bf
3pl \textsc{pl}=\textsc{dat} one=\textsc{addit} \textsc{inf}-make=\textsc{dat} cannot  \\
`They could not do a single thing.' (N060113nar01)
\z
} 
 
This predicate type can be used in main clauses as above, but also in relative clauses as in \xref{ex:pred:modal:relc}.

 
\xbox{16}{
\ea\label{ex:pred:modal:relc}
\gll [Kitha=nam       boole] mosthor=jo. \\ % bf
      \textsc{1pl}=\textsc{dat} can way=\textsc{emph} \\
    `The way we can do it'  (B060115prs17)
\z      
}

The proclitic variants of the modal particles (\formref{sec:wc:Modalparticles})  are not modal predicates; they are treated as verbal inflection and are discussed with the verbal predicates \formref{sec:pred:Verbalpredicates}.\footnote{See \citet[128]{Noonan1985} for a discussion of general difficulties in establishing the predicate status of modals.}


\section{Nominal predicates}\label{sec:pred:Nominalpredicate}

The fourth type of predicate is the nominal predicate. Two subtypes can be distinguished, which differ in the information structure context in which they are used. These are the ascriptive nominal predicate \formref{sec:pred:Nominalpredicate,ascriptive} and the identificational/equational nominal predicate \formref{sec:pred:Equationalpredicate}  \citep[cf.][104f]{Hengeveld1992nvpttd}.\footnote{I use ``equational'' here instead of ``equative'' used by Hengeveld, which I reserve for comparative contexts.}

\subsection{Ascriptive nominal predicate}\label{sec:pred:Nominalpredicate,ascriptive}
The ascriptive nominal predicate (ANP) consist of a predicate NP and an argument NP. It ascribes membership in the class of the predicate NPs to the argument NP. Normally, argument and predicate are simply juxtaposed, but it is also possible to use the copula \em asdhaathang(pa)\em.

If the argument NP is in the singular, the use of the indefiniteness marker \em atthu \em is obligatory for the predicate NP \xref{ex:constr:pred:nom:intro1}-\xref{ex:constr:pred:nom:intro3}. This is not the case for the plural.

\cbx{(COPULA) (\textit{hatthu}) NP_{pred}}{NOMPRED_{ascr}}



\xbox{16}{
\ea\label{ex:constr:pred:nom:intro1}
\gll [Andare katha arà-biilang  \zero{}]  raaja mliiga=ka    [\textbf{hathu}  \textbf{oorang} \textbf{koocak}]$_{pred}$. \\
     Andare \textsc{quot} \textsc{non.past}-say { }] king palace=\textsc{loc} \textsc{indef} man joke \\
    `The man called Andare was king's jester at the royal palace.'  (K070000wrt05)
\z
}


\xbox{16}{
\ea\label{ex:constr:pred:nom:intro2}
\gll Itthu [\textbf{bannyak} \textbf{laama} \textbf{hathu} \textbf{ruuma}]$_{pred}$. \\
      \textsc{dist} very old \textsc{indef} house \\
    `That one was a very old house.'  (K070000wrt04)
\z      
}


\xbox{16}{
\ea\label{ex:constr:pred:nom:intro3}
\gll Se asdhaathang [\textbf{hatthu} \textbf{butthul} {\em \textbf{moderate}} \textbf{Muslim} \textbf{atthu}]$_{pred}$. \\
 \textsc{1s} \textsc{copula} one very moderate Muslim one\\
`As for me, I am a very moderate Muslim.' (K051206nar18)
\z
}

In the plural, nominal predicates do not carry markers of indefiniteness.


 \xbox{16}{
\ea\label{ex:constr:pred:nom:pl}
\gll Itthukapang se=ppe baapa, se=ppe kaake, kaake=pe baapa, kithang samma oorang [Seelong=pe oorang pada]$_{pred}$. \\ % bf
   then \textsc{1s}=\textsc{poss} father \textsc{1s}=\textsc{poss} grandfather grandfather=\textsc{poss} father \textsc{1pl}  all man  Ceylon=\textsc{poss} man \textsc{pl} \\
`Then my father, my grandfather, grandfather's father, all of us are Ceylonese' (K060108nar02,K081105eli02)
\z
}

Nominal predicates are negated by \em bukang \em in all tenses  \xref{ex:pred:nom:neg:constr}. The use of \em hatthu \em is optional in negative contexts. Where the affirmative predicate \em Muslim \em is marked with \em hatthu \em in \xref{ex:pred:nom:neg:natural}, the negated predicate \trs{mlaayu}{Malay} is not.

\xbox{16}{
\ea\label{ex:pred:nom:neg:constr}
\gll Incayang (hatthu) doktor bukang. \\
     \textsc{3s.polite} \textsc{indef} doctor \textsc{neg.nonv}\\
    `He is/was not a doctor.'  (K081105eli02)
\z
}


% \xbox{16}{
% \ea
% \gll Incayang doktor, bukang. \\
%        \\
%     `.' (nosource)
% \z
% } \\

\xbox{16}{
\ea\label{ex:pred:nom:neg:natural}
\gll Sindbad  {\em the}  {\em sailor}     hatthu Muslim, mlaayu bukang. \\
Sindbad  the  sailor  \textsc{indef} Muslim, Malay \textsc{neg.nonv}\\
`Sindbad the Sailor was a Moor, he was not a Malay.' (K060103nar01)
\z
}

There is no grammaticalized way to mark TAM on nominal predicates \funcref{sec:func:Time}. Either a lexical solution must be employed, as in \xref{ex:pred:nom:tam:lexical}, or a construction involving \trs{jaadi}{become} \xref{ex:pred:nom:tam:jaadi}.

\xbox{16}{
\ea\label{ex:pred:nom:tam:lexical}
\gll Itthu duuva bergaada=jo \textbf{kàthaama} oorang ikkang. \\
     \textsc{dist} two group=\textsc{emph} earlier man fish  \\
    `These two groups were fishermen.'  (K060108nar02)
\z      
}

\xbox{16}{
\ea\label{ex:pred:nom:tam:jaadi}
\gll Ini kittham=pe nigiri su-\textbf{jaadi}. \\ % bf
 \textsc{prox} \textsc{1pl}=\textsc{poss} country \textsc{past}-become\\
`This (country) became our country.' (K0512222nar04)
\z
}

% \xbox{16}{
% \ea\label{ex:constr:pred:unreferenced}
% \gll Laayeng kithang=nang   aada  laayeng  makanan        pada saathe. \\
%       other \textsc{1pl}=\textsc{dat} exist other food \textsc{pl} sate \\
%     `Another dish we have is sate.' (K061026rcp03)
% \z
% } \\


% 
% \subsection{Modifications of the nominal predicate}\label{sec:pred:Modificationsofthenominalpredicate}
% The nominal predicate can receive a modifier marked by \em =ke\em.
% 
% 
% \xbox{16}{
% \ea\label{ex:constr:pred:unreferenced}
% \gll Se=dang baapa=ke {\em soldier} mà-jaadi suuka. \\
%      \textsc{1s.dat} father=\textsc{simil} soldier \textsc{inf}-become like  \\
%     `I want to become a soldier like daddy.' (B060115prs10)
% \z
% } \\
% 
% comparison

Negated nominal predicates can express temporal reference lexically, like \trs{innam blaakang}{from now onwards} in \xref{ex:pred:nom:tam:bukang}. The negator is still \em bukang \em and does not change with regard to the normal negation of nominal predicates.

\xbox{16}{
\ea\label{ex:pred:nom:tam:bukang}
\gll See innam blaakang, hatthu aanak bukang. \\
     \textsc{1s} \textsc{prox.dat} after \textsc{indef} child \textsc{neg.nonv} \\
    `I will never be a child again.' (K081106eli01) 
\z
}

\subsection{Equational predicate}\label{sec:pred:Equationalpredicate}
The equational predicate   asserts the identity of the referents designated by two NPs (X = Y). Both NPs are  definite. This means that the indefiniteness marker \em hatthu \em is never employed in this predicate type. In return, one of the referents is very often marked with the emphatic clitic \em =jo \em \xref{ex:constr:pred:jo}. Also, the copula \em asàdhaathang \em is often present \xref{ex:constr:pred:cop}. This predicate type is often found with kin.

 

\xbox{16}{
\ea\label{ex:constr:pred:jo}
\gll Suda [itthu    kaake=pe aade=pe                aanak]$_{X}$=\textbf{jo}    [baapa]$_{Y}$. \\ % bf
      thus \textsc{dist} grandfather=\textsc{poss} younger.sibling=\textsc{poss} child=\textsc{emph} father \\
    `So that grandfather's younger sister's child is my father.' (K051205nar05)
\z
} 

\xbox{16}{
\ea\label{ex:constr:pred:cop}
\gll [Baapa=pe      umma]$_{X}$   \textbf{asàdhaathang} [kaake=pe           aade]$_{Y}$. \\
    father=\textsc{poss} mother \textsc{copula} grandfather=\textsc{poss} younger.sibling  \\
    `My paternal grandmother was my grandfather's younger sister.' (K051205nar05)
\z
}

\section{Circumstantial predicate}\label{sec:pred:Circumstantialpredicate}
\subsection{Standard circumstantial predicate}
The fifth predicate type is the circumstantial predicate. It consists of a postpositional phrase.

\cbx{PP}{CIRCPRED}

It asserts that the two referents are in a relation expressed by the postposition. Very frequent are local postpositional phrases with \trs{=dering}{from} \xref{ex:constr:pred:circ:deri},  \trs{=ka}{at} \xref{ex:constr:pred:circ:ka}, and \trs{=nang}{to}{} \xref{ex:constr:pred:circ:nang}, but other circumstantials are also possible. The locational type with \em =ka \em will be discussed more extensively below.

\xbox{16}{
\ea\label{ex:constr:pred:circ:deri}
\gll Duuva thiiga Kluu\u mbu=deri. \\
     two three Colombo=\textsc{abl}  \\
    `Two or three are from Colombo.'  (K051206nar13)
\z      
}


\xbox{16}{
\ea\label{ex:constr:pred:circ:ka}
\ea
\gll Hatthu aade  luvar nigiri=\textbf{ka}. \\
     one younger.sibling outside country=\textsc{loc}  \\
    `One younger sibling is abroad,' 
\ex
\gll Hatthu aade {\em Suisse}=\textbf{ka}. \\
     one younger.sibiling Suisse=\textsc{loc}  \\
    `one younger sibling is at Hotel Suisse,' 
\ex
\gll Hatthu aade {\em HNB}=ka arà-bagijja. \\
      one younger.sibling HNB=\textsc{loc} \textsc{non.past}-work \\
    `one younger sibling works at Hatton National Bank.' (K061019prs01)
\z
\z
}

\citet[27]{Ansaldo2008genesis} has a nice example of this type (his orthography).
 
\xbox{16}{
\ea
\gll Ini buk go-ri\ng{} lu-da\ng{}. \\
      this book I-instr you-\textsc{dat} \\
    `This book is from me to you.' \citep[27(14)]{Ansaldo2008genesis}
\z
}

While there is no instance of a circumstantial predicate with the dative in the corpus, Ansaldo's example would be perfectly fine in the Upcountry dialect.

\xbox{16}{
\ea\label{ex:constr:pred:circ:nang}
\gll Ini buk see=dering lorang=nang. \\
      \textsc{prox} book \textsc{1s}=\textsc{abl} \textsc{2pl}=\textsc{dat} \\
    `This book is from me to you.' (K081118eli01)
\z
}

Besides the circumstantial predicates with a spatial meaning, other postpositions can also be used, like the possessive \em =pe \em in \xref{ex:constr:pred:circ:pe}. 

\xbox{16}{
\ea\label{ex:constr:pred:circ:pe}
\gll Itthu    muusing bannyak {\em teacher} pada [\textbf{Jaapna=pe}]$_{pred}$. \\
      \textsc{dist} time many teacher \textsc{pl} Jaffna=\textsc{poss} \\
    `Back then, many teachers were from Jaffna.' (K051213nar03)
\z
}
 
The position of NP and PP can be inverted, as the following example shows. This has consequences for information structure in that \xref{ex:constr:pred:circ:inv} is about the town of Gampola and provides new information about it; it is not about Tamils. This contrasts with \xref{ex:constr:pred:circ:deri}, which is not about the town, but about the people.

\xbox{16}{
\ea\label{ex:constr:pred:circ:inv}
\gll Gampola=ka    bannyak mulbar. \\ % bf
     Gampola=\textsc{loc} much Tamil  \\
    `There are a lot of Hindus in Gampola.' (G051222nar04)
\z
}


A special case is the use of a relator noun in a circumstantial predicate, as given in \xref{ex:constr:pred:circ:reln}.

\xbox{16}{
 \ea\label{ex:constr:pred:circ:reln}
   \gll Spaaman anà-nii\u n\u ggal thumpath=\textbf{nang}=le        Passara   katha arà-biilang    nigiri=\textbf{nang}=le \textbf{dìkkath}. \\
         \textsc{3s} \textsc{past}-die place=\textsc{dat}=\textsc{addit} Passara \textsc{quot} \textsc{non.past}-say country=\textsc{dat}=\textsc{addit} vicinity\\
 	`The place where he died is close to the village called Passara.' (B060115nar05,K081105eli02)
\z
} 

\subsection{Locational predicate}\label{sec:pred:Locationalpredicate}
The locational predicate is the most frequent subtype of the circumstantial predicate. It indicates that the argument is located at the place designated by the predicate. The predicate can be either  a noun marked with the locative postposition \em =ka \em or an adverb, which may or may not bear \em =ka\em.

\cbx{
NP=\textit{ka}\\
}{LOCPRED}

\cbx{
DEIC(=\textit{ka})\\
}{LOCPRED}

Example \xref{ex:pred:locational:NPka1}  and \xref{ex:pred:locational:NPka2} show a locational predicate with an NP and the locative marker \em =ka\em. \xref{ex:pred:locational:siinika}  shows the use of an adverb \em siini \em with and without \em =ka\em.

\xbox{16}{
\ea\label{ex:pred:locational:NPka1}
\gll Kithang \textbf{{\em Kandy}=ka}=jo. \\
 \textsc{1pl} Kandy=\textsc{loc}=foc\\
`We are from Kandy.' (K051222nar04)
\z
}



\xbox{16}{
\ea\label{ex:pred:locational:NPka2}
\gll Se=ppe    kaake       \textbf{hathu}  \textbf{{\em estate}=ka}. \\
     \textsc{1s=poss} grandfather \textsc{indef} estate=\textsc{loc}  \\
    `My grandfather was on an estate.' (K051205nar05)
\z
}



\xbox{16}{
\ea\label{ex:pred:locational:siinika}
\gll Kitham pada \textbf{siini}(=\textbf{ka}) (arà-duuduk). \\
     \textsc{1pl} \textsc{pl} here=\textsc{loc} \textsc{non.past}-exist.\textsc{anim}  \\
    `We are here.'  (K081104eli06)
\z
}

% \xbox{16}{
% \ea\label{ex:pred:locational:sininoka}
% \gll NP siini. \\
%        \\
%     `.'  (test)
% \z      
% }

Temporal reference need not be expressed on the locational predicate. The following example refers to the past, but this is not coded morphosyntactically.


\xbox{16}{
 \ea\label{ex:constr:pred:loc:tense}
   \gll Kithang=pe     oorang thuuva pada samma \textbf{Seelong=ka}. \\
   \textsc{1pl}=\textsc{poss} man old \textsc{pl} all Ceylon=\textsc{loc} \\
`Our forefathers were all in Ceylon' (K060108nar02)
\z
}

This construction cannot only be used for space \funcref{sec:func:Givingthenon-deicticreferencespace} but also for time \funcref{sec:func:Givingnon-deicticreferencetime}.

\xbox{16}{
\ea\label{ex:pred:locational:time}
\gll Inni     dhaathampa  \textbf{{\em 1987}=ka}. \\
      \textsc{prox} \textsc{copula} 1987=\textsc{loc} \\
    `That happened in 1987.'  (K060116nar06)
\z      
}


The negation of this predicate is done just like the negation of the existential predicate, but there are slight differences in meaning which can lead to confusion. In \xref{ex:pred:locational:neg:contr}, the existential predicate is negated, leading to an interpretation of non-existence, while in \xref{ex:pred:locational:neg}, the locational predicate is negated, which does not imply that the person is not alive, but rather that she happened to be absent.


\xbox{16}{
\ea\label{ex:pred:locational:neg:contr}
\gll Se=ppe umma \textbf{thraa}. \\
 \textsc{1s}=\textsc{poss} mother  \textsc{neg}\\
`My mother has passed away.' (B060115prs03)
\z
}


\xbox{16}{
\ea\label{ex:pred:locational:neg}
\gll Ithu=kapang {\em wife}=le \textbf{thraa}. \\
     \textsc{dist}=when wife=\textsc{addit} \textsc{neg}  \\
    `At that point in time, my wife was not at home.' (K060116nar04)
\z
} 


\section{Adjectival predicate}\label{sec:pred:Adjectivalpredicate}
The sixth and final type of predicate is the adjectival predicate. It consists of an adjective, which can be modified by an adverb or a comparative.


 
\cbx{
	$\left[
		STD=\textit{nang}
		\left\{
			\begin{array}{l}
				\textit{libbi}\\
				\textit{liivath}
			\end{array}
		\right\}
	\right]_{comp}~
	(ADV)~
	ADJ$
}{ADJPRED}

 
This type has to be distinguished from the use of adjectives as verbal predicates and from the use of adjectives as nominal predicates \formref{sec:wofo:Conversion}. Verbal predicates can be distinguished by their TAM-morphology, while nominal predicates carry \em hatthu \em in the singular. Nominal predicates in the plural cannot be distinguished from adjectival predicates.


Examples \xref{ex:pred:adj:typical1}-\xref{ex:pred:adj:typical4} show  typical uses of the adjectival predicate.


\xbox{16}{
\ea\label{ex:pred:adj:typical1}
\gll Dee buthul \textbf{jahhath}. \\
      3 very wicked \\
    `He was very wicked.' (K051205nar02)
\z
} 

\xbox{16}{
\ea\label{ex:pred:adj:typical2}
\gll Samma oorang \textbf{baaye}. \\
all man good \\
`All men are good.' (B060115cvs13)
\z
}

\xbox{16}{
\ea\label{ex:pred:adj:typical3}
\gll Goppe     aanak pada samma \textbf{baaye}. \\
     \textsc{1s.familiar}=\textsc{poss} child \textsc{pl} all good  \\
    `My children are all good.' (B060115cvs13)
\z
} 

 \xbox{16}{
\ea\label{ex:pred:adj:typical4}
\gll Skarang biini arà-iingath  puthri \textbf{thuuli} katha; Puthri arà-iingath biini \textbf{thuuli} katha.\\
  now wife \textsc{non.past}-think queen deaf \textsc{quot} queen \textsc{non.past}-think wife deaf \textsc{quot} \\
    `Now the wife thought the queen was deaf, and the queen thought the wife was deaf.'  (K070000wrt05)
\z      
}

% Example \xref{ex:constr:pred:adj:mod:adv} shows the used of an adjectival predicate \trs{panthas}{beautiful}, which is modified by the adverb \trs{bannya}{very}. The next example shows the use of an adjectival predicate modified by a comparative. \xref{ex:constr:pred:adj:mod:PP} shows the use of a PP modifying the adjective.

\xbox{16}{
\ea\label{ex:constr:pred:adj:mod:adv}
\gll Suda, inni kaving \textbf{bannyak}_{ADV} \textbf{panthas}_{ADJ}. \\
 thus \textsc{prox} wedding much beautiful\\
`So this wedding was very beautiful.' (K060116nar04)
\z
}


\xbox{16}{
\ea\label{ex:constr:pred:adj:mod:comp}
\gll Seppe kaaka se=dang libbi thii\u n\u ggi. \\
     \textsc{1s}=\textsc{poss} elder.brother \textsc{1s.dat} remain high  \\
    `My brother is taller than me.'  (K081104eli06)
\z      
}


% \xbox{16}{
% \ea\label{ex:constr:pred:adj:mod:PP}
% \gll Lorang pada \textbf{baaye} \textbf{piddang=dering}. \\
%       \textsc{2pl} \textsc{pl} good sword=\textsc{abl} \\
%     `You are good with the sword.' (K051213nar06)
% \z
% }


% In example \xref{ex:pred:adj:adjnom}, it is impossible to decide whether \em mlaayu \em is an adjectival predicate or a converted nominal predicate. If the argument were in the singular  (one soldier instead of several), the \em atthu \em would have to be used for a nominal predicate, distinguishing it from the adjectival predicate, which would lack \em atthu\em.
% 
%  
% \xbox{16}{
% \ea\label{ex:pred:adj:adjnom}
% \gll Itthu pìrrang=ka siini wickrama rajasingha=pe raaja=pe soldier pada=le {\em British}=pe {\em soldier} pada samma \textbf{mlaayu}. \\
%  \textsc{dist} war=\textsc{loc} here Wickrama Rajasinghe=\textsc{poss} king=\textsc{poss} soldier \textsc{pl}=\textsc{poss} British=\textsc{poss} soldier \textsc{pl} all Malay\\
% `In that war here, Wickramarajasingha's soldiers and the British's soldiers were all Malays.' (K051213nar07)
% \z
% }

% Example \xref{ex:constr:pred:adj:katha} shows the use of an adjectival predicate in an embedded sentence.
% 
% 
% \xbox{16}{
% \ea\label{ex:constr:pred:adj:katha}
% \gll [[Incayang nya-biilang]=jo [buthul]$_{ADJ}$ katha] anà-biilang. \\ % bf
%        \textsc{3s.polite} \textsc{past}-say==\textsc{emph} correct \textsc{quot} \textsc{past}-say \\
%     `He said: ``what he said is the correct thing''.' (K061127nar03)
% \z
% } \\


%\xbox{16}{
%\ea\label{ex:constr:pred:unreferenced}
%\gll Dokter pada \em {\em anesthesia} as-kaasi, doktor pada arà-kijja, itthu \textbf{gampang}. \\
%     doctor \textsc{pl}    anesthesia            \textsc{cp}-give, doctor \textsc{pl} \textsc{non.past}-do, \textsc{dist} easy  \\
%    `The doctors give anaesthesia and do it, that's easy.'  (K061122nar01)
%\z      
%}
%
%
%
%
%\xbox{16}{
%\ea\label{ex:constr:pred:unreferenced}
% \ea\label{ex:constr:pred:unreferenced}
% \gll Baava nigiri      cinggala   guunu nigiri        cinggala   katha. \\
%			 down country Sinhala mountain country Sinhala \textsc{quot} \\
%			`With regard to the lowcountry Sinhala and the Upcountry Sinhala.'
%	\ex
% \gll	 Bedha-han=le         baasa    baaye {\em sdikkith}. \\
%	     different-\textsc{nmlzr}=\textsc{addit} language good small   \\
%	    `the linguistic differences (between them) are quite small.'  
%	\ex
% \gll Kitham=pe mlaayu baasa banya beddha. \\
%       \textsc{1pl}=\textsc{poss} Malay language much different \\
%	    `Our language is very different.'  (K060116nar01.txt)
%	\z
%\z      
%}

 
Negation of adjectival predicates depends on the lexeme. Most adjectives are negated by postposed \em thraa\em, regardless of time reference \xref{ex:constr:pred:adj:neg:thraa:pres}\xref{ex:constr:pred:adj:neg:thraa:past}. Some other adjectives are negated by preposed \em thàrà-\em, again regardless of time reference \xref{ex:constr:pred:adj:neg:thara1}\xref{ex:constr:pred:adj:neg:thara2}.\footnote{As for frequencies, this resemble the position of adjectives in the French NP, where most adjectives are postnominal, while there are a few exceptions which are prenominal.}

\xbox{16}{
\ea\label{ex:constr:pred:adj:neg:thraa:pres}
\gll Se kaaya thraa. \\
     \textsc{1s} rich \textsc{neg}  \\
    `I am not rich.'  (K081104eli06)
\z      
}

\xbox{16}{
\ea\label{ex:constr:pred:adj:neg:thraa:past}
\gll Se pàrthaama kaaya thraa. \\
      \textsc{1s} earlier rich \textsc{neg} \\
    `I was not rich.'  (K081104eli06)
\z      
}


\xbox{16}{
\ea\label{ex:constr:pred:adj:neg:thara1}
\gll Se=dang \textbf{thàrà-sìggar}. \\
     \textsc{1s.dat} \textsc{neg}[adj]-healthy  \\
    `I am/was not healthy.'  (K081104eli06)
\z
}

\xbox{16}{
\ea\label{ex:constr:pred:adj:neg:thara2}
\gll Incayang=pe dudukan hathiyan thaaun=ka=le laile \textbf{thàrà-baae}. \\
     \textsc{3s.polite}=\textsc{poss} behaviour other year=\textsc{loc}=\textsc{addit} still \textsc{neg}[adj]-good  \\
    `His behaviour will still be bad even next year.'  (K081104eli06)
\z
}

Adjectives of the second class can only be negated by \em thàrà-\em. Adjectives of the first class or normally negated by \em thraa\em, but can be negated by \em thàrà- \em if they are converted to verbs. In that case, they get dynamic semantics, and do no longer denote a state, but an achievement or an accomplishment.
When adjectives of the first class (\em thraa\em-class) are negated with \em thàrà-\em, they do get past time reference, and additionally a dynamic reading, which indicates that they are adjectives converted to verbs. Example \xref{ex:constr:pred:adj:neg:thraa:pres} shows the normal negation of \trs{kaaya}{rich}, with \em thraa\em, negating a state. In \xref{ex:constr:pred:adj:neg:tharakaaya}, \em thàrà- \em is used, and it is the event of becoming rich which is negated, not the state (pragmatic implicatures notwithstanding). The event has necessarily temporal reference to the past.


\xbox{16}{
\ea\label{ex:constr:pred:adj:neg:tharakaaya}
\gll Se thàrà-kaaya. \\
     \textsc{1s} \textsc{neg.past}[verbal]-rich  \\
    `I did not become rich?*I was not rich/*I am not rich.'  (K081104eli06)
\z      
}

The sentence in \xref{ex:constr:pred:adj:neg:tharakaaya} is thus parallel to \xref{ex:constr:pred:adj:neg:tharadhaathang}, with a negated verbal predicate.

\xbox{16}{
\ea\label{ex:constr:pred:adj:neg:tharadhaathang}
\gll Incayang nyaari thàrà-dhaathang. \\
     \textsc{3s.polite} today \textsc{neg.past}[verbal]-come  \\
    `He didn't come today/*He is not coming today.'  (K081104eli06)
\z
}

For negation of adjectival predications referring to the future, the adjective is converted to a verb, and the irrealis verbal negator \em thama- \em is used.


\xbox{14}{
\ea
\gll Inni pukuran=yang mà-gijja thamau-gampang \\
     \textsc{prox} work=\textsc{acc} \textsc{inf}-make \textsc{neg.irr}-easy  \\
    `To do that kind of work will never be(come) easy.' (K081106eli01)
\z
}

% *thàràgampang
% gampang thraa
% inni pukuranyang ma gijja thamau gampang
% *inni pukuranyang ma gijja gampang thama pii

% 
% \xbox{16}{
% \ea
% \gll Se spuulu thaaunnang duppang kaaya thraa. \\
%        \\
%     `.' (nosource)4.11.08
% \z
% } \\

% \xbox{16}{
% \ea\label{ex:constr:pred:unreferenced}
% \gll Inni nang duppangle  se thamau kaaya. \\
%       \textsc{prox}=front=\textsc{addit}  \\
%     `In the future I will not become rich.'  (K081104eli06)
% \z      
% }


% \xbox{16}{
% \ea\label{ex:constr:pred:unreferenced}
% \gll Se kaaya bukang, incayangjo kaaya. \\
%        \\
%     `.'  (K081104eli06)
% \z      
% }
% 
% \xbox{16}{
% \ea\label{ex:constr:pred:unreferenced}
% \gll Se kaaya thraa, incayango kaaya. \\
%        \\
%     `.'  (K081104eli06)
% \z      
% }


% \xbox{16}{
% \ea\label{ex:constr:pred:unreferenced}
% \gll Se thàrà-bìssar. \\
%        \\
%     `I did not grow up.'  (K081105eli02)
% \z      
% }
% 




% 
% \xbox{16}{
% \ea\label{ex:constr:pred:unreferenced}
% \ea\
%    \gll Blaakang, see Kandi=nang   anà-aajibaa. \\
%     after,    \textsc{1s}  Kandy=\textsc{dat} \textsc{past}-bring \\
% \ex
%    \gll See Kandi=nang   su-baa, aaji  baa,   se=dang. \\
%       \textsc{1s}  Kandy=\textsc{dat} \textsc{past}-bring bring bring \textsc{1s.dat} \\
% `Then, I was brought to Kandy, I was brought to Kandy' (K060108nar01)
% \z
% \z
% } \\

% 
% \xbox{16}{
% \ea\label{ex:constr:pred:unreferenced}
% \ea
% \gll Giithu   nigiri pada=nang       asà-pii   anà-libbi. \\
%       like.that country \textsc{pl}=\textsc{dat} \textsc{cp}-go \textsc{past}-more \\
%     `Most went to those countries.'  
% \ex
% \gll Konnyong  mlaayu=joo   Seelong=ka  thii\u n\u ggal aada. \\
%       few Malay=\textsc{emph} Ceylon=\textsc{loc} settle exist \\
%     `Few settled in Sri Lanka.' (K051222nar06)
% \z
% \z
% } \\

 
