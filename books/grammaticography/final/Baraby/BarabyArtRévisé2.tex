% This file was converted to LaTeX by Writer2LaTeX ver. 1.0.2
% see http://writer2latex.sourceforge.net for more info
\documentclass[letterpaper]{article}
\usepackage[ascii]{inputenc}
\usepackage[T3,T1]{fontenc}
\usepackage[english]{babel}
\usepackage[noenc]{tipa}
\usepackage{tipx}
\usepackage{amsmath}
\usepackage{amssymb,amsfonts,textcomp}
\usepackage[top=0.9839in,bottom=0.4917in,left=0.9839in,right=0.9839in,nohead,includefoot,foot=0.4925in,footskip=0.946in]{geometry}
\usepackage{array}
\usepackage{hhline}
\usepackage{hyperref}
\hypersetup{colorlinks=true, linkcolor=blue, citecolor=blue, filecolor=blue, urlcolor=blue}
% Text styles
\newcommand\textstyleInternetlink[1]{#1}
% Footnote rule
\setlength{\skip\footins}{0.0469in}
\renewcommand\footnoterule{\vspace*{-0.0071in}\setlength\leftskip{0pt}\setlength\rightskip{0pt plus 1fil}\noindent\rule{0.25\columnwidth}{0.0071in}\vspace*{0.0398in}}
% Pages styles
\makeatletter
\newcommand\ps@Standard{
  \renewcommand\@oddhead{}
  \renewcommand\@evenhead{}
  \renewcommand\@oddfoot{[Warning: Draw object ignored]}
  \renewcommand\@evenfoot{\@oddfoot}
  \renewcommand\thepage{\arabic{page}}
}
\makeatother
\pagestyle{Standard}
\title{Reference grammars addressed to the speakers of minority languages}
\author{}
\date{2012-01-13}
\begin{document}
\clearpage\setcounter{page}{1}\pagestyle{Standard}
{\centering\bfseries
Reference grammars for speakers of minority languages
\par}

{\centering
Anne-Marie Baraby
\par}

{\centering
Universit\'e du Qu\'ebec \`a Montr\'eal
\par}

\clearpage
Most of the work done in grammaticography focuses on the writing of grammars for an audience of linguists, and more specifically, typologists. In this paper, we present a grammati\-cographic model designed mainly to take into account the needs of minority language speakers, because they play a central role in the preservation of their language. However, since in minority language situations it is not possible to generate as many grammars as there are different potential end users, we propose a multilevel grammar, based on our experience as grammarian of Innu, a First Nation language spoken in Quebec (Canada). In this type of grammatical description, the first (main) level is addressed to non-specialist users, the speakers of the language being described, whereas grammatical material aimed at other users (such as linguists) is presented in secondary levels and is limited to core information. Our grammaticographic model was initially conceived for paper (printed) grammars, but we believe that electronic publication offers interesting solutions for multilevel grammars, while  paper (printed) grammatical descriptions have greater limitations.

\clearpage\subsection[INTRODUCTION. Grammaticography as a new branch of linguistics was developed almost at the same time as documentary linguistics (Gippert, Himmelmann \& Mosel, 2006). The development of these new domains of linguistics in the recent past is not a coincidence. In fact, there is a link between both domains, since the development of a grammar is theoretically part of a documentation program for endangered languages. Furthermore, these fields of research were proposed as solutions, among others, in preventing language extinction. But, without denying merit to those who drew up the basics of grammaticography, we believe it maintains an important weakness: most of the work done in grammaticography, i.e. the business of writing grammars, aims the grammatical descriptions primarily at linguists, usually ignoring the minority language speakers, who have their own specific needs. However, we believe that the speakers of an endangered language play a central role in documenting their language, even though they are not specialized in linguistics. Therefore, in our PhD thesis (Baraby, 2011), we propose a model of grammaticography which takes into account speakers{\textquoteright} needs. More precisely, from our experience as a grammarian of Innu, an indigenous language spoken in northern Quebec, we have developed a set of principles which may help in constructing a model of a reference grammar intended particularly for Innu speakers. ]{INTRODUCTION.\footnotemark{}\textsc{ }\textmd{Grammaticography as a new branch of linguistics was developed }\textmd{almost }\textmd{at the same time }\textmd{as}\textmd{ documenta}\textmd{ry}\textmd{ linguistics}\textmd{ (Gippert, Himmelmann \& Mosel}\textmd{,}\textmd{ 2006)}\textmd{. The development of these new domains of linguistics in the recent past }\textmd{is }\textmd{not a coincidence. In fact, there is a link between both domains, since the development of a grammar }\textmd{is theoretically }\textmd{part of }\textmd{a }\textmd{documentation program for endangered languages. Furthermore, these fields of research were proposed as solutions, among others, }\textmd{in preventing }\textmd{language extinction. But, without denying merit to those who drew up the basics of grammaticography, we}\textmd{ believe }\textmd{it }\textmd{maintains }\textmd{an important weakness}\textmd{: most of the work done }\textmd{in }\textmd{grammaticography}\textmd{, i.e. the business }\textmd{of }\textmd{writing }\textmd{grammar}\textmd{s}\textmd{,}\textmd{ }\textmd{aims }\textmd{the }\textmd{grammatical }\textmd{descriptions }\textmd{primarily at}\textmd{ linguists, }\textmd{usually ignoring the }\textmd{minority language speakers, wh}\textmd{o}\textmd{ }\textmd{have their own }\textmd{specific }\textmd{needs. However, we believe that the speakers of an endangered language play a central role in documenting their language, even though they are not specialized in linguistics. Therefore, }\textmd{in our PhD thesis (Baraby, 2011), }\textmd{we propose a model of grammaticography }\textmd{which }\textmd{take}\textmd{s}\textmd{ into account speakers{\textquoteright} needs. More precisely, from our experience }\textmd{as }\textmd{a }\textmd{grammarian of }\textmd{Innu, }\textmd{an indigenous language spoken in northern Quebec, we }\textmd{have }\textmd{developed a set of }\textmd{principles }\textmd{which may help}\textmd{ in}\textmd{ }\textmd{constructing }\textmd{a model of }\textmd{a }\textmd{reference grammar intended}\textmd{ }\textmd{particularly }\textmd{for Innu speakers. }}
\footnotetext{ Thanks to Sebastian Nordhoff and to an anonymous reviewer for their comments on an earlier version of this paper, and thanks to Robert Papen not only for his comments, but especially for the revision of the English text.}
Developing a reference grammar for a minority language, often under{}-documented, usually unwritten, and probably endangered, is quite different from writing a grammar for languages of wider communication such as the main Indo-European languages. In the latter case, it is possible to generate as many grammars as there are different theoretical approaches, different end users, and different objectives. This situation is highly unlikely for most minority languages, where it is generally impossible to develop such a multiplicity of different grammars. In short, writing a grammar for a minority language raises the following question: is it possible to develop a reference grammar aimed at both types of audiences, linguists and non-specialized users?

Actually, writing grammars for non-specialized speakers brings its own challenging issues, which are different, in many regards, from grammaticography conceived for linguists. In the following sections, we will deal mostly with the question of the end users of a particular grammar and the solutions we propose in achieving the task of documenting a language mainly for the speakers of that language, but also for any other interested public, such as linguists who want to learn something about the inner workings of the language. Among the means we propose for such a grammar, based on our experience in the Innu language grammar project, is a multilevel grammar, which we believe can be achieved both in printed and electronic versions. Would an electronic grammar be a better medium for the model of grammar we are proposing? At first glance, it may seem so. However, we do not see the two media types as being opposed, but as complementary options. In our view, developing a grammar on Internet may pose specific challenges, but it also shares problems with writing a grammar for a printed book version.

\subsection[2. THE PROBLEM. Developing a grammaticography for users without specialized training in linguistics raises a number of issues, including the selection of the eventual end users and specific objectives. Among others, these issues concern different choices regarding the theoretical approach, the language used in writing the grammar itself, including the grammatical terminology and metalanguage, and the depth of the grammatical description envisaged. As well, there are questions referring to the content itself, for instance which phenomena to describe, the scope of the description, the organization of the content, etc. And finally, the issue of the type of grammar planned is also often raised as to whether it is to be a pedagogical grammar or a reference grammar. ]{2. THE PROBLEM. \textmd{Developing a grammaticography for users without specialized training in linguistics }\textmd{raises a number of }\textmd{issues, }\textmd{including the }\textmd{selection of}\textmd{ the eventual end users}\textmd{ and }\textmd{specific }\textmd{objectives. Among others}\textmd{,}\footnotemark{}\textmd{ these issues concern different choices regarding }\textmd{the }\textmd{theoretical approach, }\textmd{the }\textmd{language used }\textmd{in }\textmd{writ}\textmd{ing}\textmd{ the grammar}\textmd{ itself, }\textmd{including }\textmd{the }\textmd{grammatical terminology and metalanguage,}\textmd{ and}\textmd{ }\textmd{th}\textmd{e depth}\textmd{ of the grammatical }\textmd{description}\textmd{ envisaged}\textmd{.}\textmd{ }\textmd{As well, there are }\textmd{questions refer}\textmd{r}\textmd{ing}\textmd{ to the content itself, for instance }\textmd{which }\textmd{phenomena to describe, }\textmd{the }\textmd{scope }\textmd{of the description, }\textmd{the }\textmd{organization of the content, etc. }\textmd{And finally, the}\textmd{ issue of the}\textmd{ t}\textmd{ype of grammar}\textmd{ planned }\textmd{is also often}\textmd{ raised}\textmd{ as to whether it is to be }\textmd{a pedagogical}\textmd{ grammar}\textmd{ or a reference grammar}\textmd{.}\textmd{ }}
\footnotetext{ In our thesis, we also discuss the following issues: possible types of grammar,  language register, compari\-son with other languages (related or not), language variation and linguistic norms, description of an oral and/or written language, use of the orthography  (if there is one), use of parts of speech,  presentation of examples.}
Since we believe that the choice of who the eventual end users are to be determines all the other choices made by the grammarian, the following discussion mainly concentrates on this particular issue. 

In our grammatical model, we place the speakers of the language being described in the foreground. But making this basic choice is not as easy as it appears because, as previously stated, in the case of minority languages, the possibility of being able to produce more than one reference grammar is very low to non-existent. Therefore, even if we choose to produce a grammar mainly for speakers who are not specialized in linguistics, we are aware that it is also important to document the language for other users, including linguists, but only as secondary end users. However, such a decision, which takes into account the needs of different types of users for the same product, raises a big problem: \textit{I}\textit{s it possible to write a reference grammar for different users having different expectations}? This is a complex issue in a grammaticography mainly developed with the objective of documenting minority languages. Indeed, aiming at heterogeneous users could bring about dissatisfaction with the content of the grammar on the part of all users.

As a solution to this problem, we propose a multilevel grammar, where most of the content is addressed to a non-specialized audience, speakers of the described language (it constitutes the \textit{first }\textit{or main }\textit{level}), but with some additional grammatical information (the \textit{secondary level}) aimed at specialists, such as linguists. More details about this type of reference grammar are given in Section~4.

Once the end users and the objectives of the reference grammar have been established, the grammarian must decide if the grammatical product is to be a printed book or an electronic document. Both media have advantages and inconveniences. In our thesis, we primarily discuss solutions to produce a good grammar for Innu speakers, with secondary attention paid to a specialized audience.\footnote{ As specialised audience, we think of linguists working in typology, Algonquian linguistics, historical linguistics, anthropological linguistics, etc. } Here, we also intend to look at possibilities of electronic grammars, especially for the kind of multilevel grammar we are proposing for non-specialist end users. Is an electronic edition a better choice for such a grammar? It certainly offers good resources to produce the kind of multilevel grammar we are envisaging. However, since electronic tools are not available everywhere, and also because some users are not yet ready to use these tools, we believe that printed grammars will be maintained, according to users{\textquoteright} needs or wishes. And even where electronic tools are available, both grammars can be seen as complementary ways to reach the main objective: documenting an under{}-described language, and giving speakers a tool to develop a good formal knowledge of their language.

\textbf{3.}\textbf{ }\textbf{THE INNU PEOPLE AND THEIR LANGUAGE}\textbf{.}\footnote{ The Innu were previously called Montagnais by French speakers. They always called themselves Innu ({\textquoteleft}human being{\textquoteright}, now {\textquoteleft}Amerindian{\textquoteright} and {\textquoteleft}Innu First Nation{\textquoteright}) and it is this term that is now officially used not only by the Innu organizations, but also by federal (Canadian) and provincial governments, and in the media.} Before discussing our grammati\-cographic model further, below we give a short description of Innu communities and their language, because we think their linguistic situation may be comparable to many other minority linguistic groups, and above all, because it is their linguistic situation that convinced us to work on a reference grammar of their language. Even if we are well aware that all sociolinguistic situations are not identical, we believe that the grammaticographic principles and solutions we are proposing may be useful elsewhere. 

\subsection[3.1. THE LINGUISTIC, GEOGRAPHIC AND DEMOGRAPHIC SITUATION OF INNU COMMUNITIES. Innu is part of the Algonquian language family and is spoken in Quebec and Labrador. It is closely related to Eastern Cree, also spoken in Quebec, and to Naskapi, spoken in Labrador. The Innu live in ten isolated villages spread out over the immense northern territory of Quebec and Labrador, Canada. Some of these communities still cannot be reached by road. Around 10,000 people use Innu as a first and every{}-day language. The rate of retention of the language varies from one community to another: extinct in one, spoken by one third of the population in another, majority language of nearly 75~\% to 95~\% \ of the population elsewhere. The Innu language therefore is still very vigorous in a majority of Innu communities, where it is the first language learned by children and is used at home and in the community at large. Except for two communities (Mashteuiatsh and Essipit), Innu is also used in religious ceremonies, local administration, community radio, and, with the intervention of interpreters, in health, social and legal services.]{3.1. THE LINGUISTIC, GEOGRAPHIC AND DEMOGRAPHIC SITUATION OF INNU COMMUNITIES. \textmd{Innu}\textmd{ }\textmd{is part of the}\textmd{ Algonquian }\textmd{language }\textmd{family and is spoken in Quebec and Labrador. It is closely related to }\textmd{Eastern Cree}\textmd{, also spoken in Quebec,}\textmd{ and t}\textmd{o}\textmd{ Naskapi}\textmd{, spoken in Labrador}\textmd{. The Innu live in ten isolated villages spread out over the immense }\textmd{northern }\textmd{territory of Quebec and Labrador}\textmd{, Canada}\textmd{. Some of these communities still cannot be reached by road. Around 10,000 people use Innu as a }\textmd{first }\textmd{and every-day language. The rate of retention of the language varies from one community to }\textmd{another}\textmd{: extinct in one, spoken by one third of the population in another, majority language of nearly 75~\% to 95~\% }\textmd{ of the population }\textmd{elsewhere. The Innu language therefore is still very vigorous in }\textmd{a }\textmd{majority of }\textmd{Innu }\textmd{communities}\textmd{, where it is the first language learned by children}\textmd{ and }\textmd{is used }\textmd{at home }\textmd{and }\textmd{in }\textmd{the communit}\textmd{y}\textmd{ at large}\textmd{. }\textmd{Except for two communities (Mashteuiatsh and Essipit), Innu is }\textmd{also used in}\textmd{ religious ceremonies}\textmd{, }\textmd{local administration, community radio, and}\textmd{, with the intervention of interpret}\textmd{er}\textmd{s,}\textmd{ in health}\textmd{,}\textmd{ socia}\textmd{l and }\textmd{legal services}\textmd{.}}
Nevertheless, pressures from the dominant language are very strong, since virtually all Innu are bilingual today, with French as their second language in most Quebec communities, and English in the Labrador community (and partly in Pakuashipu, Quebec). If Innu is not considered endangered enough to disappear in the short term, its survival is not guaranteed in the long term, because of the relatively low number of speakers. 

The Innu language constitutes a continuum of dialects, geographically spread out from the most western one, Mashteuiatsh (Lac Saint-Jean), the {\textquotedblleft}central{\textquotedblright} dialects, with Pessamit, Uashat-Maliotenam on the upper north shore of the Saint Lawrence and Matimekush (Schefferville, Northern Quebec), and Mamit (on the lower north shore of the Saint Lawrence), as well as Ekuantshit, Natashquan, Unamen-Shipu, Pakua-shipu; the dialect of Sheshatshiu (Labrador) is somewhere between the central and Mamit dialects. These dialects are quite different, morphologically or phonologically speaking, but speakers of different dialects still readily understand each other. 

Much as for most Amerindian languages, Innu was until lately an oral tradition language, but a standardized writing system has been developed, except for Mashteuiatsh (Lac Saint-Jean, Quebec).\footnote{ Because the Mashteuiatsh dialect is different from other dialects (it is more conservative) and it is learned by children as a second language, the community did not adopt the standardized orthography. We are now working with the community to develop their standard writing.} However, this standard orthography is not necessarily mastered by all speakers, who more often turn to French (or English) for their written communications. In fact, oral and writing habits are diglossic: Innu for oral communication within the community, French elsewhere.\footnote{ To know more about the development of a standard orthography and the role of writing in Innu communities, see Baraby (2000, 2002 and 2011b).}

\subsection[3.2. THE INNU LANGUAGE IN SCHOOLS. Locally, Band councils run all Innu schools, from kindergarten to high school. More and more Innu teachers are teaching in these schools, however most of these certified teachers are not involved in Innu language teaching: they teach the various subject matters determined by the regular provincial programs, using French or English in the classroom. In fact, these teachers prefer to teach these subjects since they are properly supported with well designed curricula and pedagogical material.]{3.2. THE INNU LANGUAGE IN SCHOOLS. \textmd{Locally}\textmd{,}\textmd{ }\textmd{Band councils run all Innu schools}\textmd{, }\textmd{from }\textmd{kindergarten}\textmd{ }\textmd{to }\textmd{high school}\textmd{. }\textmd{M}\textmd{ore and more Innu teachers }\textmd{are teaching }\textmd{in these schools}\textmd{, however }\textmd{most of these}\textmd{ certified}\textmd{ teachers are not }\textmd{involved}\textmd{ in}\textmd{ Innu language teaching: they teach}\textmd{ the various subject matters determined by the }\textmd{regular }\textmd{provincial programs}\textmd{,}\textmd{ using French or English }\textmd{in }\textmd{the }\textmd{classroom}\textmd{.}\textmd{ }\textmd{In fact, these teachers }\textmd{prefer}\textmd{ }\textmd{to teach }\textmd{these }\textmd{subjects}\textmd{ }\textmd{since they }\textmd{are }\textmd{properly}\textmd{ supported }\textmd{with}\textmd{ }\textmd{well designed }\textmd{curricula and }\textmd{pedagogical material.}}
If Innu is indeed part of the school curriculum, usually taught once or twice a week (1 or 2 hours/week), the working languages of the school, including the languages in which academic subjects are taught, are French in Quebec and English in Labrador. Also, except for help from the \textit{Institut }\textit{Tshakapesh} (see below), Innu language teachers, all competent speakers but without adequate training in linguistics, Innu grammar, or even language teaching pedagogy, are often left quite isolated in their specific teaching tasks ~In fact, teaching Innu is not perceived as being as attractive as teaching regular subject matters. Nevertheless, in spite of all these difficulties, the Innu language teachers are highly motivated in transmitting their language, and in learning more about it.

 Fortunately, a university program for the teaching of Innu  has recently been developed. Also, tailor-made courses in linguistics for Innu teachers are now offered two or three times a year,\footnote{ The Universit\'e du Qu\'ebec \`a Chicoutimi is providing two undergraduate certificates (10 courses each) specially designed for First Nations~: \textit{Technolinguistique autochtone}, \textit{Transmission d{\textquoteright}une langue autochtone}.}  as well as workshops on language teaching. The \textit{Institut Tshakapesh}\footnote{ http://www.icem.ca/icem/} (a cultural and educational institute for the Innu people, based in Sept-\^Iles, Quebec) has taken leadership to promote Innu language preservation and development. The institute supports Innu teachers in different ways, including the organization of meetings, workshops and courses, funding the production of curricula, pedagogical materials, reference materials, hiring specialists in linguistics or in language pedagogy, etc.

Except for the two communities mentioned above,\footnote{ In these two communities (Mashteuiatsh and Essipit), Innu is taught as a second language.} where the language is not spoken by children, Innu is taught as a first language to students who are already fluent in it. The students are expected to improve their oral skills and acquire literacy in Innu. Since the traditional way of life has changed a lot, language transmission has also changed. Parts of traditional vocabulary are being lost, and the language of the youth now includes more and more loanwords, systematic code switching and code mixing (with French and more rarely English). At this point in time, we do not know if pressures from the linguistic environment have already altered certain grammatical structures of Innu, but it is quite possible. Thus, Innu language courses in the school curriculum have an important role to play in preventing language erosion or loss, as do the families and communities themselves. 

\subsection[3.3. DOCUMENTATION OF INNU. Innu is probably one of the most documented languages of all of the native languages of Canada, but except for dictionaries published since the 70s, the majority of linguistic descriptions was intended for linguists and published in academic journals.]{3.3. DOCUMENTATION OF INNU. \textmd{Innu }\textmd{is}\textmd{ probably one of the most documented languages }\textmd{of}\textmd{ all of}\textmd{ }\textmd{the }\textmd{native languages of }\textmd{Canad}\textmd{a}\textmd{, but except for dictionaries published since the 70s, }\textmd{the majority }\textmd{of linguistic description}\textmd{s}\textmd{ was intended }\textmd{for }\textmd{linguists and published in }\textmd{academic}\textmd{ journals.}}
At present, there exists no comprehensive reference grammar for Innu, but we are now working on such a grammar, specifically aimed at Innu speakers (Baraby \& Drapeau, forthcoming). A conjugational guide (Baraby 2004) is however available, and some electronic learning material is also presently being developed.\footnote{ Marie-Odile Junker, a linguist from Carleton University (Ottawa) and specialist of Eastern Cree, has developed electronic material for the Cree in Quebec (\href{http://www.eastcree.org/}{\textstyleInternetlink{www.eastcree.org}}), and she is now collaborating with the Institut Tshakapesh to elaborate similar kinds of material for Innu, mostly for pedagogical use.}

The completion of a reference grammar will answer a pressing request which comes from the Innu themselves, particularly from Innu language teachers. This reference work is necessary to help Innu speakers develop metalinguistic knowledge about their language. As well, it will be a good tool in supporting the development of pedagogical material and literacy.

In sum, producing a good Innu reference grammar, accessible to non-specialists, will give a good opportunity for linguists to transmit to non-specialized speakers the grammatical knowledge they have acquired over time. We believe it is an interesting way of assuring that specialized knowledge does not remain ensconced in academia, but returns to the most prominent actors in the maintenance of an endangered language, the speakers themselves.

We sometimes hear linguists say that minority languages speakers are not really interested in having a (written) grammar of their language. We understand that this is not a priority everywhere. However, the facts from the Innu language situation demonstrate quite the opposite: many Innu speakers are highly motivated to learn more about the grammar of their language, as long as they are given a reasonably easy access to it. 

\subsection[4. A GRAMMATICOGRAPHY INTENDED FOR MINORITY LANGUAGE SPEAKERS. Our work on Innu grammar has raised a number of issues that led us to develop a grammaticographic model, conceived for non{}-specialists, i.e. the speakers of the language being described. Besides the targeted end users which we discuss in the next section ({\S}4.1), the main characteristics of this model are the following:]{4. A GRAMMATICOGRAPHY INTENDED FOR MINORITY LANGUAGE SPEAKERS. \textmd{Our work on Innu grammar }\textmd{has }\textmd{raised }\textmd{a number of }\textmd{issues that led us to develop a }\textmd{g}\textmd{rammatico}\textmd{graphic model, }\textmd{conceived }\textmd{for non}\textmd{{}-}\textmd{specialists}\textmd{, }\textmd{i.e. }\textmd{the speakers of the language }\textmd{being }\textmd{described}\textmd{. }\textmd{Besides the targeted }\textmd{end users }\textmd{which}\textmd{ we }\textmd{discuss}\textmd{ in the }\textmd{next}\textmd{ section}\textmd{ ({\S}}\textmd{4}\textmd{.1)}\textmd{, the main characteristics of }\textmd{this}\textmd{ model }\textmd{are the following:}}
\begin{itemize}
\item The grammar is a \textit{reference grammar} rather than a \textit{pedagogical} grammar. This follows from the fact that the main objective is to document the language in a comprehensive way.\footnote{ Pedagogical grammars are usually less comprehensive than reference grammars. The former usually have as objective the learning of a language as a foreign or second language, or the learning of writing rules based on grammatical structures. The organization (or progression) is also different in both types of grammar. Pedagogical grammars usually include exercises, while reference grammars do not. But the boundaries between both types are sometimes loose, since both types may have pedagogical purposes. In fact, the different types of grammatical descriptions are on a continuum, and a reference grammar for non-specialists is closer to pedagogical grammars than is a grammatical description specifically intended for linguists (see Germain \& S\'eguin, 1985: 46-56; Dirven, 1990: 1-2; Baraby, 2011a: 210-236 for further details).}
\item French is used to write the grammar as well as for the metalanguage and grammatical terminology, because the Innu are familiar with French from their schooling. Actually, they are more used to read in French than in Innu. Moreover, writing the grammar in a relatively well-known language such as French allows most users who are not speakers of Innu to read the grammar, including linguists or other interested users (see {\S}~4.3). 
\item The writing style is rather formal, but with a simple and precise vocabulary.
\item Comparisons with French, English and other Algonquian languages are made, whenever they help users transfer their metalinguistic knowledge from one language to the other, for instance, from school knowledge (French grammar) to mother tongue. It is also interesting for Innu speakers to see the links between their language and other related languages such Cree, Naskapi or other Algonquian languages.
\item Whenever possible, the grammatical terminology used to describe the language is the one used in traditional French grammars, because it is already familiar to most Innu speakers. However, terminology may also be innovative, in order to fill terminological gaps in traditional French grammar. Even if some typical Algonquian linguistic terminology is used, it is sometimes abandoned in light of the current knowledge of the language and in order to meet the particular needs of non-specialist users.\footnote{ For pedagogical reasons, we may decide to abandon terms whose meanings are opaque to laymen. For instance, in Innu there is a mode with a counterfactual meaning, but the term {\textquoteleft}\textit{contrefa}\textit{c}\textit{t}\textit{u}\textit{e}\textit{l}{\textquoteright} (counterfactual) is not well understood by Innu speakers. Also, other terms create confusion, for instance, \textit{subjonctif} (subjunctive) and \textit{subjectif} (subjective); in this case, we replace the latter with another term, \textit{percepti}\textit{f }(perceptive).}
\item Innu is an oral tradition language, so the grammatical description focuses on this aspect, but it also takes into account standard orthography. For instance, examples are given in this orthography, instead of being transcribed in a phonetic alphabet. Phonetic transcriptions are minimally used, mostly in sections specifically addressed to lin\-guists.
\item Dialect variation is considered, but only to a certain point. The use of a standard\-ized orthography is a good solution in giving a more synthetic description that is accepted by all speakers.\footnote{ The Innu orthography is not based on one particular dialect, but on principles such as the following : the Eastern (Mamit) dialect serves as a reference for grammatical spelling, while when variations are more phonological, the Western dialect become the reference. In sum, speakers from all communities have had to compromise in arriving at an agreement on a common spelling system (Baraby, 2000, 2004).} 
\item Even if the language spoken by elders is generally seen as the norm, i.e. {\textquotedblleft}good Innu language{\textquotedblright}, the speech patterns of all generations are considered in the grammatical description.
\item The general organization of the content is {\textquotedblleft}bottom up{\textquotedblright}, starting with the simpler notions, for instance the word before the sentence, the noun before the verb, etc. It goes from structures to functions or from functions to structures.
\item The theoretical approach is a {\textquotedblleft}traditional{\textquotedblright} one, close to traditional French grammar, with additions to reflect particularities of Innu grammar; the latter being based on recent research by Lynn Drapeau, co-author of the grammar and specialist of Innu linguistics. In fact, this comes close to what Dixon (1997, 2010) proposes in his \textit{Basic linguistic theory}: to describe languages in the perspective of language documentation.
\end{itemize}
Obviously, each of these characteristics could be discussed in more detail, but we will now focus on the issue of multiple end users for a grammatical description and the solution we propose to achieve this objective, a multilevel grammar.

\subsection[4.1. TARGETED END USERS FOR A MINORITY LANGUAGE REFERENCE GRAMMAR. The status of the main readership of a grammar, i.e. the targeted end users, needs clarification, because each type of grammar users is in fact not homogenous, some being laymen, other being linguists. Basically, we distinguish two main target groups: non{}-specialists, for instance language teachers (the primary group); specialists, for instance linguists (the secondary group). Each of these two broad groups may be heterogeneous.]{4.1. TARGETED END USERS FOR A MINORITY LANGUAGE REFERENCE GRAMMAR. \textmd{The}\textmd{ status of }\textmd{the }\textmd{main }\textmd{readership }\textmd{of a grammar, }\textmd{i.e. the }\textmd{targeted}\textmd{ end users}\textmd{,}\textmd{ needs clarification, because each }\textmd{type }\textmd{of grammar users is in fact not homogenous, }\textmd{some }\textmd{being layme}\textmd{n, other being}\textmd{ linguists.}\textmd{ }\textmd{Basically, w}\textmd{e distinguish two main target groups: }\textmd{non-specialists, for instance }\textmd{language teachers (}\textmd{the }\textmd{primary group); specialists, for instance linguists (}\textmd{the }\textmd{secondary group).}\textmd{ Each of these two broad groups may }\textmd{be }\textmd{heterogeneous}\textmd{.}}
\subsection[4.1.1. PRINCIPAL END USER : LANGUAGE TEACHERS. As Mithun (2006: 282) points out, non{}-specialized users are not all the same, depending on specific linguistic situations. Actually, a layman readership may include anyone interested in knowing more about the described language: speakers or non{}-speakers, teachers or students, advanced learners or beginners, first or second language learners, having basic, intermediate or advanced knowledge in the grammar or even none at all, literate in the language or not, etc. For Mithun, it is also important to think about the future needs of the members a language community: those who are not interested in or able to use the grammar at present could become users later on, for instance, after some training. ]{4.1.1. PRINCIPAL END USER : LANGUAGE TEACHERS. \textmd{As Mithun (2006}\textmd{: }\textmd{282) point}\textmd{s}\textmd{ out, non-specialized users are not all the same, depend}\textmd{ing}\textmd{ on }\textmd{specific }\textmd{linguistic situation}\textmd{s}\textmd{. Actually, }\textmd{a }\textmd{layman }\textmd{readership }\textmd{may include anyone interested }\textmd{in}\textmd{ know}\textmd{ing}\textmd{ more about the described language: speakers or non-speakers, teachers or students, advanced}\textmd{ learners}\textmd{ or beginners, first or second language learners, }\textmd{having }\textmd{basic}\textmd{, intermediate or }\textmd{advanced }\textmd{knowledge in }\textmd{the }\textmd{grammar or }\textmd{even }\textmd{no}\textmd{ne at all}\textmd{, literate in the language or not, etc. For Mithun, it is also important to think about }\textmd{the }\textmd{future needs of}\textmd{ the members}\textmd{ a language community: those who are not interested}\textmd{ in}\textmd{ or able to use the grammar }\textmd{at present }\textmd{could become users later on, for instance}\textmd{,}\textmd{ after some training. }}
For the grammar of Innu, we propose, as \textit{main}\textit{ end user}, those speakers who have some basics in grammar, if not in Innu grammar, at least in school grammar, that is in French grammar,\footnote{ The complexity of written French grammar makes possible some transfer of grammatical knowledge from French to Innu; for instance, agreement rules or the complexity of verbal inflections. Since most Innu speakers learn French grammar in school, we want to take advantage of this fact. Of course, this advantage does not hold for those who have been schooled in English, as English grammar (at least morphologically speaking) is not as complex as French grammar.} for Quebec Innu speakers. More precisely, we choose Innu language teachers as targeted end users, for various reasons. First, because they are competent speakers of the language. Secondly, because they have had some grammatical basics in their language of schooling (French) and they also have a university degree in teaching, or they have had some training in pedagogy or in Innu grammar.\footnote{ The older Innu teachers may not have a university degree, but they have taken a certain amount of tailor-made courses over the years. We hope that younger teachers will graduate in education, since the expectations from school  administrators are getting higher for their teachers.} Finally and above all, because they are very motivated to learn more about the language they have to transmit to their students. Some of them, but not all, have had some basic training in Innu linguistics.\footnote{ University Certificate programs in Amerindian language teaching and in {\textquotedblleft}technolinguistics{\textquotedblright} (\textit{technolinguistique}), designed by the University of Qu\'ebec at Chicoutimi, are now available. Innu teachers may enrol in these programs, and one or two courses are available every year. } As main end users, we would also add any other professionals involved in different areas related to the Innu language: language curriculum designers and developers of pedagogical material, translators, authors and writers.

In our grammaticographic model, we claim that minority languages require reference grammars rather than pedagogical grammars, in order for them to be well documented, and in a comprehensive way, and such a grammar is probably not for beginners, users without any skill in formal grammar. In other words, in the reference grammar model we propose, the level of difficulty is \textit{intermediate}, which means that main users are not specialists, such as linguists, but they have basic grammatical knowledge, if not in their first language, then in their second language. Of course, as mentioned above, this intermediate level may include other users than teachers such as language professionals, and, even advanced learners or any other person with basic grammatical competence.

\subsection[4.1.2. SECONDARY END USERS. Linguists will also get something out of the kind of grammar we are proposing; either a starting point to their curiosity about the language, or an overall view of it, which could be completed with more specialized publications. For instance, Innu has been the subject of many academic publications in linguistics, but even specialists may find it useful to get all the information in a single place, instead of having to search for information scattered in different journals and books. Therefore, obtaining grammatical knowledge about a minority language in one single document, that is in a reference grammar for non{}-specialized speakers, is a good way of documenting a language, as much for linguists as for non{}-specialist speakers.]{4.1.2. SECONDARY END USERS. \textmd{L}\textmd{inguists will also get something out }\textmd{of }\textmd{the kind of grammar we }\textmd{are }\textmd{propos}\textmd{ing;}\textmd{ }\textmd{either}\textmd{ a starting point to their curiosity about the language, or an overall view of it, which could be completed with more specialized publications. For instance, Innu }\textmd{has been}\textmd{ the subject of many }\textmd{academic }\textmd{publications in linguistics, but even specialists may }\textmd{find it }\textmd{useful to get all }\textmd{the }\textmd{information in }\textmd{a single }\textmd{place, instead of having to }\textmd{search }\textmd{for information}\textmd{ scattered}\textmd{ in different journals and books}\textmd{. Therefore, }\textmd{obtaining }\textmd{grammatical knowledge about a minority language in }\textmd{one single document}\textmd{, that is in a reference grammar for }\textmd{non-specialized }\textmd{speakers, is a good way }\textmd{of}\textmd{ document}\textmd{ing}\textmd{ }\textmd{a }\textmd{language, as much for linguists }\textmd{as }\textmd{for non-specialist speakers.}}
We now raise an important issue: Is the objective of documenting a language -- meaning that the grammar has to be as comprehensive or complete as possible -- compatible with the aim of producing a user-friendly grammatical description? On the one hand, comprehensiveness implies adding more complex information (usually addressed to linguists), such as information that may be necessary for better comprehension of the structure of the language and to the eventual development of linguistic typology. On the other hand, integrating this kind of material may be confusing to non-specialized users, especially if it is inserted in the main part of the description, that is, if it is included in the main grammatical text.

Considering the needs of different potential users in one single reference grammar may seem conflicting, especially with end users as different as non-specialized speakers and linguists. To achieve our objective of an accessible grammar, designed as much to document a language than to train and inform native speakers of it, we propose a document having more than one level (or layer) of reading, the main (first-level) text being user-friendly, with more specialized or complex information intended for linguists being presented at another level. In the next section, we will outline what we mean by a multilevel grammar. 

\subsection[4.2. A MULTILEVEL REFERENCE GRAMMAR. Is it possible to write a reference grammar for different users, with divergent expectations? For under{}-described languages, the choice is limited, since it would be utopian to expect different kinds of grammatical descriptions for each potential audience. As a solution to this issue, we propose our multilevel (or multilayered grammar), that is with each level being aimed at a specific audience.]{4.2. A MULTILEVEL REFERENCE GRAMMAR. \textmd{Is it possible to write a reference grammar for different users, with divergent expectations? For under}\textmd{{}-}\textmd{described languages, the choice is limited, since it would be utopi}\textmd{an}\textmd{ to expect different kinds of grammatical description}\textmd{s}\textmd{ for each potential audience. As a solution to th}\textmd{is}\textmd{ issue, we p}\textmd{ropose }\textmd{our multilevel (or multilayered grammar), that is with each level being aimed at a specific audience.}}
\subsection[4.2.1. FIRST LEVEL: MAIN LEVEL. The first level, intended for specific users, speakers of the language but without specialized training in linguistics or in grammar, is the main level. This means that these end users have priority over all others, and that most of the grammatical content (explanations and descriptions) is found at this level. For that matter, this level is mostly visually unmarked, and this is possible in printed as well as in electronic grammars. It includes the essentials of the language structures and functions, in other words, what the speaker needs to know about his or her language. Besides descriptions and explanations, there must be lots of \ \ examples. These have two purposes: to support the description given and to document the language. Also, tables, diagrams and figures are useful. Moreover, it is very important to give good, clear definitions of grammatical notions and of the terminology used to describe the language. Since describing a language implies the use of some metalanguage, readers of the grammar must become familiar with it. However this also means the metalanguage must be well defined and described. Again, the Innu experience has shown us that speakers can deal with grammatical metalinguistic terms, once they understand what they refer to in their own language. ]{4.2.1. FIRST LEVEL: MAIN LEVEL. \textmd{The}\textmd{ }\textmd{\textit{first level}}\textmd{, intended for specific users, speakers of the language }\textmd{but }\textmd{without specialized training in linguistics or in grammar, is the main level. }\textmd{This}\textmd{ means that these }\textmd{end }\textmd{users have priority over }\textmd{all }\textmd{others, and that most of the grammatical content (explanations and description}\textmd{s}\textmd{)}\textmd{ is found at this level}\textmd{. For that matter, this level }\textmd{is}\textmd{ mostly visually }\textmd{\textit{unmarked}}\textmd{, and }\textmd{this is possible in printed }\textmd{as well as }\textmd{in }\textmd{electronic grammars. It }\textmd{include}\textmd{s}\textmd{ the essentials of the language structures and functions, }\textmd{in other words, }\textmd{what the speaker needs to know about his}\textmd{ or her}\textmd{ language. Besides description}\textmd{s}\textmd{ and explanations, there }\textmd{must }\textmd{be }\textmd{lots of }\textmd{ }\textmd{ examples}\textmd{. These}\textmd{ have two purposes: to }\textmd{support }\textmd{the description}\textmd{ given}\textmd{ and to document the language. Also, tables, diagrams and figures }\textmd{are}\textmd{ useful}\textmd{. Moreover, it}\textmd{ is}\textmd{ very important to give good}\textmd{,}\textmd{ }\textmd{clear }\textmd{definitions of grammatical notions and }\textmd{of }\textmd{the }\textmd{terminology used to describe the language. }\textmd{Since d}\textmd{escribing a language implies}\textmd{ the use of}\textmd{ some metalanguage, }\textmd{readers of the grammar}\textmd{ }\textmd{must }\textmd{become}\textmd{ }\textmd{familiar}\textmd{ with it}\textmd{. However this also means the}\textmd{ metalanguage}\textmd{ must be }\textmd{well defined and described. Again, the Innu experience }\textmd{has }\textmd{show}\textmd{n}\textmd{ us that speakers c}\textmd{an}\textmd{ deal with grammatical }\textmd{metalinguistic }\textmd{terms, once they understand what }\textmd{they}\textmd{ refer to in their }\textmd{own }\textmd{language. }}
Another point to stress is the question of the layout of a grammar for a non-specialized audience. Even if the visual aspect of the grammar may not be as important as the text itself, it is central to this kind of work since it is aimed at users who are not necessarily familiar with grammatical descriptions. In this case, the grammatical product must be attractive, using different typographical means such as different colors, fonts, the use of framed texts, etc. In a way, a reference grammar intended for laymen may resemble a pedagogical grammar in its presentation, the objective being to help the user to easily find what he or she is looking for, providing the reader with certain types of information such as indices, tables, etc.\footnote{ Indexes, tables of content and cross referencing are among the ways of helping to find information in a grammar, but this is true for any kind of grammar, aimed at specialists or not.} Even in a printed grammar, it is possible to provide second{}-level information, clearly distinct from the first-level text, in using typographical treatment, such as different fonts or font sizes, frames, screens, etc. Otherwise, the non-specialist user may feel overwhelmed and be discouraged in going on. In fact, without being a pedagogical grammar, a minority language reference grammar has pedagogical aims, and this is true even for majority language grammars.\footnote{ In a recent meeting with colleagues about choosing a good reference grammar for French courses at the university level, an excellent French reference grammar was rejected, because its presentation was not judged user-friendly enough.}

In the introduction to his grammar of Ojibwa (\textit{nishnaabemwin}), Valentine (2001: xxxi) mentions: {\textquotedblleft}One reviewer pointed out that this grammar is actually a compound work, consisting of an introduction to linguistics as well as a grammar{\textquotedblright}. Valentine explains his choice: {\textquotedblleft}This I have done, again, to accommodate \textit{my intended primary audience}, \textit{those interested in teaching the language}, who typically lack extensive linguistic training{\textquotedblright}. But the problem with Valentine{\textquoteright}s grammar is that all information is given in the same way, i.e. put on the same layer or level, the result being very dense text. This may very well discourage non-specialized users. Valentine (2001) is a good grammar, but it is not very user-friendly. Valentine probably wanted to document Ojibwa in a comprehensive way, and that is a legitimate objective we share; however comprehensiveness sometimes goes against the readability of the whole. To prevent this pitfall, we propose separating grammatical information on distinct levels. On the one hand, we propose different levels intended for different users, as discussed above. On the other hand, we suggest another type of hierarchical organization, even at the first level. For instance, we present definitions or important remarks, often fundamental information, in box frames instead of in plain text, as in example (1).\footnote{ Layouts of the examples we present here are not definitive, but indicative. Later on, we would like to work with a book designer, to find the best ways to format the book. In the meantime, we use simple word processing, to prioritize the grammatical information, the levels and the sublevels. The final product will have a better appearance than what we show here, and the distinction between each kind of rubric will be more salient.} 

(1)\ \ Examples of definitions of linguistic notions in Innu grammar, at \textit{Level 1}

\textbf{\textsc{MODALIT\'ES}}\textbf{\textsc{~}}\textsc{: E}nsemble de faits linguistiques qui traduisent l{\textquoteright}attitude du locuteur par rapport \`a ce qu{\textquoteright}il dit; les modalit\'es peuvent prendre la forme de modes (conjugaisons), de types de phrases (phrases affirmatives, interrogatives, de commandements), d{\textquoteright}adverbes ou d{\textquoteright}autres auxiliaires modaux, selon les langues. En innu, on a surtout recours aux modes (suffixes modaux), mais \'egalement aux pr\'everbes modaux et aux adverbes. 

En innu, les informations v\'ehicul\'ees par les modalit\'es portent, entre autres, sur le degr\'e de certitude, de fiabilit\'e ou de subjectivit\'e de ce qui est \'enonc\'e ou encore sur la possibilit\'e ou non de r\'ealisation de l{\textquoteright}\'ev\'enement dont il est question)\footnote{ Translation: \textit{Modality : Linguistic facts that express the attitude of the speaker towards what he is saying; modalities may take the form of modes (conjugations), clause types (affirmative, interrogative, imperative), adverbs or other modal auxiliaries, according to the specific language. In Innu, recourse is typically to modals (modal suffixes), but also to modal preverbs and to adverbs}.}

(source: Baraby et Drapeau, forthcoming, chapter on modes and modalities)

The definitions in (1) may seem somewhat complex, especially for non-linguists, but they occur after a number of {\textquotedblleft}easier{\textquotedblright} chapters. For instance, Chapter 2 presents elementary concepts.  (2) is an example of this kind of basic definitions and (3) is an example of basic remarks that may accompany plain text or definitions:

(2)\ \ Examples of definitions of linguistic notions or remarks in Innu grammar, \textit{Level 1}

Le VERBE constitue g\'en\'eralement le c{\oe}ur de la phrase. C{\textquoteright}est un mot qui sert \`a exprimer une \textit{action} accomplie ou subie par le \textit{sujet}; ou encore qui sert \`a d\'ecrire un sujet, un \'etat ou un \'ev\'enement.

Le \textit{verbe} innu varie en \textit{genre}, en \textit{nombre}, en \textit{personne }et en \textit{obviation}, comme le \textit{nom.} Plus particuli\`ere\-ment, il varie aussi en \textit{temps}, en \textit{mode} et en \textit{ordre}, formant ainsi des \textit{conjugaisons.}\footnote{ Translation: \textit{The VERB generally constitutes the very heart of the sentence (or clause). It is a word used to express an action accomplished by the subject, or which serves to describe a subject, a state or an event.}\par \textit{The verb in Innu varies in gender, number, person and in obviation, as does the noun. More specifically, it also varies in tense, mode and in order, thus creating conjugations}\textit{.}}

(source: Baraby et Drapeau, forthcoming, chapter on basic notions, section \textit{Les verbes})

 (3)\ \ Examples of fundamental remarks in Innu grammar, \textit{Level 1}

\textsc{remarque}\textsc{ }

Du point de vue de la syntaxe, le verbe s{\textquoteright}\textbf{accorde} habituellement \textbf{avec un} \textbf{sujet}, et parfois \'egalement \textbf{avec un compl\'ement}. Cet accord en \textit{genre}, en \textit{nombre}, en \textit{personne} et en \textit{obviation} est indiqu\'e par des marques grammaticales ajout\'ees au verbe. De plus, le verbe peut varier de fa\c{c}on \`a indiquer le \textit{temps} de l{\textquoteright}action ou de l{\textquoteright}\'ev\'enement d\'ecrit, ainsi que la \textit{modalit\'e} (jugement que le locuteur porte sur son \'enonc\'e). \footnote{ Translation: \textit{From a syntactic point of view, the verb usually agrees with its subject and sometimes with its object. This agreement in gender, number, person and obviation is indicated by grammatical material added to the verb. Moreover, the verb may vary in order to indicate the tense of the action or even being described, as well as the modality (judgement that the speaker makes concerning what is being said}).}

(source: Baraby et Drapeau, forthcoming, chapter on basic notions, section \textit{Les verbes})

As well, we recently decided to include in Level 1 information that is not essential for all non-specialist users, but that may interest some of them. This information is intitled \textit{Grammaire avanc\'ee} (advanced grammar),\footnote{ We got the idea of introducing more complex grammatical information aimed at non-specialist users from our experience in teaching French grammar to native speakers and teaching basic course in linguistics to Innu teachers. In both cases, there were always a number of students who wanted to go beyond the course matter. These kinds of remarks belong at Level 1, because they deal with grammatical information usually readily known by linguists.} as in (4), also from the chapter on elementary notions; it is in the section entitled \textit{Les classes de mots} (parts of speech), and it comes after more basic explanations about kinds of word in Innu:

(4)\ \ Examples of more advanced information, at \textit{Level 1}

\textsc{grammaire avanc\'ee~}\textsc{: }On parle aussi, pour les \textit{classes de mots}, de \textit{cat\'egories majeures} et de \textit{cat\'egories mineures}. Les cat\'egories majeures sont celles qui regroupent les verbes, les noms et les pr\'epositions; les cat\'egories mineures regroupent les autres classes de mots. Les cat\'egories majeures servent \`a exprimer le message du locuteur et elles ont un contenu lexical; les cat\'egories mineures ont un contenu d{\textquoteright}abord grammatical.

On parle \'egalement de \textit{classes ouvertes} pour les verbes, les noms et les adverbes, parce qu{\textquoteright}on peut leur ajouter de nouveaux mots. Les autres classes sont \textit{ferm\'ees}, parce qu{\textquoteright}on peut plus difficilement leur ajouter de nouveaux mots.\footnote{ Translation: \textit{Advance grammar: Word classes can be divided into Major categories and Minor categories. Major categories include verbs, nouns and prepositions; minor categories include all other word classes. Major categories are used to express the message of the speaker and they contain lexical material; minor categories have mainly grammatical content.}\par \textit{Verbs, nouns and adverbs are considered to be Open classes since one can add new words to them while all other word classes are considered Closed because it is much more difficult to add new words to them}\textit{.}}

(source : Baraby et Drapeau, forthcoming, chapter on basic notions)

In these cases, the remarks are clearly identified, giving the user the choice of reading it or not.

At Level 1, there are also comparisons with other languages, when we judge it can help users to understand explanations about a given concept. It is of two types: comparison with languages like French and English, or comparison with other Algonquian languages. Another kind of {\textquotedblleft}special{\textquotedblright} information that belongs at Level 1 concerns orthographical remarks.

The different kinds of information at Level 1 are all presented in box frames, with different layouts or settings, depending on each rubric. Actually, in a printed grammar, we have to employ this kind of typography, because all information is given on the same plane. This is a situation where an electronic version has considerable advantages over a printed version, since it can present more than one plane or versions. However, despite difficulties inherent in a book version, we think it is possible to have a printed multilevel grammar, but it means using a great number of formatting tools and techniques. In the Innu grammar, we have designed a sort of key corresponding to different types of information or rubrics, which we systematically use.

As for the rest of the grammatical description, aimed at the non-specialist user, it is given in plain text, without any special formatting, and it is written using a higher size of font.

As for most printed grammars, the content of Innu grammar is organized in chapters, sections, sub-sections, etc. Nevertheless, there is another question linked to the issue of the organization of grammatical content that may arise; it concerns both the organization of the grammar and the theoretical approach. Traditional grammars, and most grammars generally, are \textit{structural}, meaning that descriptions are based on structures or forms, and not on functions. A grammar  organized on the basis of parts of speech is a good example of a structural organization, just as is a description of verbs based on paradigms and inflections. On the contrary, a grammar that gives more importance to functions, to what one does with structures, how one constructs meanings in a language is a \textit{functional grammar}; for instance: concept identification, message building, making up a message,\footnote{ These first two examples come from \textit{Collins Cobuild English} (Sinclair, 2004), which is a professed functional grammar: {\textquotedblleft}A grammar which puts together the patterns of the language and the things you can do with them is called a functional grammar. This is a functional grammar ({\dots}){\textquotedblright} (Sinclair, 2004: v).\par } marking time, concepts of space and location, command strategies, etc. These two types of grammars are often seen as opposite theoretical approaches in describing language grammars. In our grammaticographic model, it is more a matter of perspectives according to which the grammatical content is organized, than a matter of opposing theoretical approaches. Furthermore, structural and functional approaches (or perspectives) may be quite complementary. Thus, for Innu grammar, we conceive of a mixed perspective, structural and functional, according to descriptive needs. Actually, our Innu grammar starts with chapters based on parts of speech, but other chapters are more functional (for example, a chapter about the meaning of modes and modalities). Some chapters may be more structural (noun and verb morphology), others more functional (semantics of modalities), still others both structural and functional at the same time. We refuse to be confined to one given theoretical approach, and we prefer to make use of what seems to be the best way to describe or explain what is going on in the language. After all, the main objective of a grammar written for speakers is the description of the language, not the defense of a particular linguistic or grammatical theory. In our Innu grammar, we adopt a more or less traditional approach, because it is what Innu speakers are familiar with, since they were schooled in French and were taught French grammar. It is supplemented with information coming from research carried out in Algonquian linguistics and adapted to a non-specialized audience. This point of view is not Eurocentric, but pragmatic: it is a question of building on what speakers already know and which is comparable in French (or English) and in Innu, before introducing new material, more features of Innu language structures. Since most Innu speakers, even Innu language teachers, have never been trained in the grammar of their language, they only have intuitive but no metalinguistic knowledge of it, which is why they have to {\textquotedblleft}learn{\textquotedblright} about the functioning of their own language, to acquire how to think about it and how to talk about it.

Writing a grammar for non-specialists does not mean to oversimplify. Actually, it consists in

vulgarizing specialized matter, to give access to it to those who are not familiar with descriptions aimed at linguists, and doing this is no easy task. It is often easier to use precise terminology, designed for specialists. Example (5) is an extract of Innu grammar introducing fundamentals of morphology:

(5)\ \ Extract of Innu grammar, chapter on elementary concepts, section on word formation

\textit{Les langues du monde ne forment pas toutes leurs mots de la m\^eme fa\c{c}on. On appelle }morphologie \textit{ }\textit{l{\textquotesingle}analyse de la formation des mots. Comme toutes les langues algonquiennes, et contrairement au fran\c{c}ais, l{\textquoteright}innu a une morphologie tr\`es complexe.}

\textit{Par l{\textquotesingle}analyse de la formation des mots, ou l{\textquotesingle}}analyse\textit{ }morphologique\textit{, on parvient \`a isoler les diff\'erentes parties d{\textquotesingle}un mot, chacune ayant une signification propre. On nomme }morph\`eme\textit{ chaque }\textit{partie ind\'ecomposable de mot dont on peut identifier le sens. Le morph\`eme est ainsi dit la }plus petite unit\textit{\'e (ou }unit\'e minimale\textit{) }porteuse de sens\textit{. Cette notion de }sens\textit{ rattach\'ee au morph\`eme}\textit{ }\textit{est tr\`es importante : un mot peut en effet \^etre d\'ecoup\'e en parties, c{\textquotesingle}est-\`a-dire en morph\`emes, en autant que chacune de ces parties signifie quelque chose. La signification d{\textquotesingle}un morph\`eme peut aussi n{\textquotesingle}\^etre que grammaticale : par exemple, dans }\textbf{ashamat}\textit{ }\textit{{\textquoteleft}}\textit{les raquettes}\textit{{\textquoteright}}\textit{ et dans }\textbf{atusseuat}\textit{ }\textit{{\textquoteleft}}\textit{ils travaillent}\textit{{\textquoteright}}\textit{, }\textbf{{}-}\textbf{at}\textit{ marque le pluriel alors que dans }\textbf{atussepan}\textit{ }\textit{{\textquoteleft}}\textit{il travaillait}\textit{{\textquoteright}}\textit{, }\textbf{\textit{{}-pan}}\textit{ marque le pass\'e; les morph\`emes }\textbf{asham}\textit{ et }\textbf{atusse}\textbf{\textit{{}-}}\textit{ portent respectivement les sens de }\textit{{\textquoteleft}}\textit{raquette}\textit{{\textquoteright}}\textit{ et }\textit{{\textquoteleft}}\textit{travailler}\textit{{\textquoteright}}\textit{.}

(source : Baraby et Drapeau, forthcoming, chapter on basic notions)

In this type of reference grammar, we have to define notions such as \textit{word}, \textit{noun}, \textit{verb}, \textit{prefix}, \textit{suffix}, etc. More challenging is the definition of other notions such as \textit{transitivity}, important for the classification of verbs in Innu since in Algonquian languages, there are four classes of verbs based on animacy and transitivity. Of course, when addressing linguists, it is not necessary to explain what a transitive verb is, but the concept is not easily explained to non-specialists.\footnote{ We know this from our own experience in teaching French and Innu grammar, and from what others have told us about their endeavours to teach this concept to Innu speakers.}

Describing for laymen necessitates adaptation in matters of terminology, concepts, and also definitions. Thus, writing for speakers of a minority language implies taking them at a starting point, and in bringing them as far as they want to go, giving them some theoretical tools to better understand their language structures, so they can transfer this knowledge in their teaching. For minority languages, vulgarizing also means helping speakers develop metalinguistic knowledge they did not have the opportunity to learn while at school. Our experience with Innu teachers has shown that they are quite motivated to learn more about their language, as long as we take the time to explain what they need to learn in order to move forward. A good grammar aimed at speakers has to be written in a simple, clear and precise style, but simplifying does not mean less rigor in the description. 

\subsection[4.2.2. SECONDARY LEVELS. All other levels of the grammar, matter mostly intended for specialists or any users other than primary users, are less substantial, and are clearly identified by different kinds of formatting, such as letter{}-press, fonts, frames, colors, \ font size and indentation, headers, etc. In so doing, secondary level will be kept in the background, in such a way that non{}-specialized reader will be able to skip non{}-essential material and focus on the main content. At the same time, those really interested in finding out more about specialized information will know where to find it.]{4.2.2. SECONDARY LEVELS. \textmd{All other levels of the grammar, matter mostly intended for specialists or any users }\textmd{other }\textmd{than primary users, }\textmd{are}\textmd{ less substantial, and }\textmd{are }\textmd{clearly }\textmd{identified }\textmd{by }\textmd{different kinds of}\textmd{ formatting}\textmd{, }\textmd{such as }\textmd{letter-press, fonts, }\textmd{frames}\textmd{, colors, }\textmd{ font size }\textmd{and indentation, headers, etc. In so doing, secondary level will be kept in the background, in such a way that non-specialized reader will be able to skip non-essential }\textmd{material }\textmd{and focus on }\textmd{the }\textmd{main content. }\textmd{At}\textmd{ the same time, those really interested }\textmd{in }\textmd{find}\textmd{ing}\textmd{ out}\textmd{ more }\textmd{about }\textmd{specialized information will know where to find it.}}
To illustrate the kind of information aimed at linguists, we give in (6) and (7) extracts from our Innu grammar (chapter on modes and modalities). In (6), the information is intended to justify the use of a different term for a mode than what was traditionally used in Algonquian linguistics.

 (6)\ \ Example of information pertaining to Level 2, addressed to linguists

\textsc{linguistique : }Dans un syst\`eme de modalit\'es \'epist\'emiques, la terminologie des modes doit pouvoir tenir compte des contextes pr\'ecis d{\textquoteright}utilisation de ces modes. Palmer (2001, p.~24-25) rejette le terme {\textquoteleft}dubitatif{\textquoteright} dans le cas d{\textquoteright}une affirmation qui s{\textquoteright}appuie sur l{\textquoteright}observation. Pour des formes qui, comme celles de l{\textquoteright}innu en \textit{{}-tshe }et \textit{{}-kupan}, n{\textquoteright}indiquent pas un doute formel, il propose plut\^ot le terme {\guillemotleft}~\textit{deductive~}{\guillemotright}. Ainsi, dans l{\textquoteright}exemple anglais \textit{John }\textbf{\textit{must}}\textit{ be in his office }{\textquoteleft}John doit \^etre \`a son bureau{\textquoteright}, Palmer souligne que le locuteur porte un jugement ferme d\'ecoulant d{\textquoteright}une preuve \textbf{observable} : par exemple, \textit{parce que les lumi\`eres du bureau sont allum\'ees, parce que John n{\textquoteright}est pas chez lui, etc}. Ce jugement bas\'e sur la d\'eduction est diff\'erent de celui qui implique l{\textquoteright}emploi de \textit{may} ({\textquoteleft}peut, peut-\^etre{\textquoteright}) \textit{John may be in his office} {\textquoteleft}John est peut-\^etre dans son bureau{\textquoteright} (sp\'eculation) ou encore, l{\textquoteright}emploi de \textit{will} (conclusion raisonnable bas\'ee sur une connaissance partag\'ee) \textit{John{\textquoteright}ll be in his office} {\textquoteleft}Jean est s\^urement dans son bureau{\textquoteright} (\textit{parce qu{\textquoteright}il commence toujours \`a huit heures, parce qu{\textquoteright}il est un travailleur acharn\'e, etc.}) Palmer (2001, p.~25).

Dans une analyse logique des modalit\'es, le mode \textsc{d\'eductif} de l{\textquoteright}innu correspond \`a la \textit{n\'ecessit\'e} \'epist\'emique, qui est bas\'ee sur la d\'eduction.

(source : Baraby et Drapeau, forthcoming, chapter on modes and modalities)

In (7), the information is given to keep track of historical data for a special form of imperative in Innu that is not well known in Algonquian linguistics.

(7)\ \ Example of information pertaining to Level 2, addressed to linguists

\textsc{historique : }L{\textquoteright}imp\'eratif en \textit{{}--me} est rapport\'e par Goddard (1979, p.~90) pour l{\textquoteright}ancien Unami et dans Lemoine (1901, p.~15ff; cit\'e dans Goddard) pour l{\textquoteright}innu. Goddard l{\textquoteright}interpr\`ete comme un imp\'eratif futur et il croit que l{\textquoteright}imp\'eratif en \textit{{}--me} est plus ancien que celui en \textit{{}--hk}. En 1988, Goddard mentionne la pr\'esence de l{\textquoteright}imp\'eratif en --\textit{me} dans la grammaire de l{\textquoteright}algonquin du \textsc{xvii}\textsuperscript{e} si\`ecle du p\`ere Nicolas (Goddard, 1988, p.~11). 

(source : Baraby et Drapeau, forthcoming, chapter on modes and modalities)

Most of the information addressed to linguists has the same format, except for subtitles.

In sum, each type of information must have the same layout over the whole grammar, in order for the reader to be able to recognize it easily and decide if he or she needs to read it or not.

\subsection[4.3. LANGUAGE USED FOR DESCRIPTION AND METALANGUAGE. There is one point that is not often discussed in grammaticography. It concerns the language in which the grammar is written, as well as the language used for grammatical terminology or for metalanguage, what Lehman (1989: 134) prefers to call {\textquotedblleft}background language{\textquotedblright}. For grammars intended for linguists, it may not be an issue; since the language to be described is not mastered by most potential users, a grammarian generally prefers to use a more widespread language such as English. The problem is posed differently in case of grammars mainly written for non{}-specialized speakers of a language. Should the grammarian use the language that is the object of the description as background language or should he or she use a widespread language such as English or French? Both solutions are acceptable, depending on the specific context. If the minority language is used, speakers will need to learn a specialized grammatical lexicon in the language, and this may prove to be a daunting task. Moreover, writing a minority language grammar using the minority language as background language limits the accessibility of the description to speakers {}-- in fact to readers {}-- of the language.]{4.3. LANGUAGE USED FOR DESCRIPTION AND METALANGUAGE. \textmd{There is one point that is not often discussed in grammaticography. It concerns the language in which }\textmd{the}\textmd{ grammar}\textmd{ is written}\textmd{,}\textmd{ as well as }\textmd{the language }\textmd{used for }\textmd{grammatical terminology }\textmd{or for }\textmd{metalanguage, what Lehman (1989: 134) prefers to call {\textquotedblleft}background language{\textquotedblright}. For grammars intended for linguists, it may not be an issue}\textmd{;}\textmd{ since the language to be described is not mastered by most potential users, a grammarian generally prefer}\textmd{s}\textmd{ to use a more widespread language}\textmd{ such as}\textmd{ English}\textmd{.}\textmd{ The problem is posed differently in case of grammar}\textmd{s}\textmd{ mainly written for }\textmd{non-specialized }\textmd{speakers of a language. Should the grammarian use the language that is the object of the description as background language}\textmd{ or should he or she use a widespread language such as English or French}\textmd{? Both solutions }\textmd{are }\textmd{acceptable, depending}\textmd{ on the specific }\textmd{context}\textmd{.}\textmd{ I}\textmd{f }\textmd{the minority }\textmd{language is used}\textmd{, speakers }\textmd{will }\textmd{need to }\textmd{learn}\textmd{ }\textmd{a }\textmd{specialized grammatical}\textmd{ lexicon}\textmd{ in the}\textmd{ l}\textmd{anguage, and this }\textmd{may prove to be }\textmd{a}\textmd{ daunting}\textmd{ task}\textmd{.}\textmd{ }\textmd{M}\textmd{oreover, writing a }\textmd{minority language }\textmd{grammar using }\textmd{the}\textmd{ minority language}\textmd{ as background language}\textmd{ limit}\textmd{s}\textmd{ the accessibility of the description }\textmd{to }\textmd{speakers}\textmd{ }\textmd{{}-- in fact to }\textmd{readers}\textmd{ -- }\textmd{of the language}\textmd{.}}
For our grammar of Innu, the background language used is French, because speakers are bilingual (mostly in Innu and French); they are schooled in French, where they learn formal French grammar. As for grammatical terminology, we utilize traditional French grammatical terminology, in so far as it corresponds adequately to Innu grammar. Traditional French grammatical terminology is useful for quasi{}-universal linguistic or grammatical concepts, but it is insufficient in a number of cases, since Algonquian languages are quite distant from European languages, genetically and typologically. To solve this issue, we tend to search for more suitable terminology in either Algonquian linguistics or in linguistic typology. But since most of the linguistic documents are published in English, once we have found a suitable term, we have to find a good French equivalent. In fact, finding an adequate terminology, especially in French, to describe a language for non-specialist speakers  is quite challenging.

\subsection[5. A PRINTED OR AN ELECTRONIC GRAMMAR? In the preceding sections, we presented the main characteristics or our grammaticographic model, that is, who the end users of the type of grammar we are developing are going to be {}-- speakers of minority languages, as well as linguists {}-- and how it is possible to meet such different users{\textquoteright} needs, our key proposition being to produce a multilevel grammar, with the first and basic level aimed at non{}-specialist speakers, other levels aiming other users, including linguists. Originally, we started developing our grammaticographic model within the scope of the Innu grammar project, whose primary objective was to produce a reference grammar book, and we therefore came up primarily with solutions for printed grammars. In the meantime, we started contemplating what electronic media could bring to minority language grammar projects, inspired by a project of an online grammar that is being elaborated for Eastern Cree, a related language to Innu. Indeed, we still believe both kinds of project {}-- printed and electronic grammars {}-- are worthwhile for minority language speakers, having positive and negative aspects, according to each situation or to the objective of the grammatical description. Because we believed the needs in this matter were important, we pursued our initial project, developing a grammaticography for printed grammars. We know, in fact, from the specific Innu situation, as well as from other contexts with which we are familiar, that many minority language members prefer to have access to grammar books, rather than online grammars. But in this process, we have always kept in mind the possibility of applying to electronic grammars some of the grammaticographic principles and solutions we were developing for printed grammars.]{5. A PRINTED OR AN ELECTRONIC GRAMMAR? \textmd{In the preceding sections, we presented the main characteristics or our grammaticographic model, that is, }\textmd{who the end users of }\textmd{the type of grammar we }\textmd{are developing are going to be }\textmd{{}-- speakers of minority languages, as well as linguists -- and how it is possible to meet such }\textmd{different }\textmd{users{\textquoteright} needs, our key proposition being to produce a multilevel grammar, with the first and basic level aimed at non-specialist speakers, other levels }\textmd{aiming}\textmd{ other users, including linguists. }\textmd{Originally, we started developing our grammaticographic model within the scope of the Innu grammar project, whose primary objective was to produce a reference grammar book, and we therefore}\textmd{ }\textmd{came up primarily with solutions for printed grammars. In the meantime, we} \textmd{started}\textmd{ }\textmd{contemplat}\textmd{ing}\textmd{ what electronic media could bring to minority language grammar projects, inspired by }\textmd{a project of an }\textmd{online grammar that is being elaborated for Eastern Cree}\textmd{,}\footnotemark{}\textmd{ a related language to Innu. Indeed, we }\textmd{still }\textmd{believe both kinds of project -- printed }\textmd{\textit{and}}\textmd{ electronic grammars -- are worthwhile for minority language speakers, having positive and negative aspects, according to each situation or to the }\textmd{objective }\textmd{of the grammatical description. Because we }\textmd{believed }\textmd{the needs }\textmd{in this }\textmd{matter }\textmd{we}\textmd{re important, we }\textmd{pursued }\textmd{our initial project, }\textmd{developing }\textmd{a grammaticography for printed grammars. We know, in }\textmd{fac}\textmd{t, from the}\textmd{ specific}\textmd{ Innu situation, }\textmd{as well as }\textmd{from other context}\textmd{s with which we are familiar}\textmd{, that many minority language members prefer to have access to grammar books, }\textmd{rather than online }\textmd{grammar}\textmd{s}\textmd{. But in }\textmd{this }\textmd{process, we }\textmd{have }\textmd{always kept in mind the }\textmd{possibility }\textmd{of applying to electronic grammars some of the }\textmd{grammaticographic }\textmd{principles and solutions }\textmd{we }\textmd{were }\textmd{develop}\textmd{ing}\textmd{ for printed gramma}\textmd{rs}\textmd{.}}
\footnotetext{ Eastern Cree is spoken in Quebec, on James Bay and inland; it is part of the Cree-Innu-Naskapi continuum, and is very close to Innu; speakers of both language living in contiguous territories are able to understand each other. For more information about the online grammar project of Cree language, see \href{http://www.eastcree.org/}{\textstyleInternetlink{www.eastcree.org}}. }
In the  following sections, we first examine both positive and negative aspects which we have raised for both types of grammars, in particular in the situation of minority languages. In fact, we ask the following question: are these two products really opposed or are they in fact complementary? As well, we consider how electronic (or online) grammatical media might be interesting for a multilevel model of grammar.

\subsection[5.1. POSITIVE AND NEGATIVE ASPECTS OF PRINTED AND ELECTRONIC GRAMMARS. In a grammaticograhy aimed at linguists, the advantages of electronic grammars may be obvious, though not without problems. We do not intend to repeat here what has already been discussed elsewhere, except for the context of minority language grammars for non{}-specialists. In this particular case, criteria in deciding to develop an electronic grammar are not exactly the same as developing one aimed squarely at linguists.]{5.1. POSITIVE AND NEGATIVE ASPECTS OF PRINTED AND ELECTRONIC GRAMMARS. \textmd{In a grammaticograhy aimed at linguists, }\textmd{the advantages }\textmd{of electronic grammars may be obvious, though }\textmd{not without }\textmd{problems. We do not intend to repeat here what ha}\textmd{s}\textmd{ already been discussed elsewhere}\textmd{,}\footnotemark{}\textmd{ except for the context of minority language grammars for non-specialists. In }\textmd{this particular }\textmd{case, criteria }\textmd{in }\textmd{decid}\textmd{ing}\textmd{ to }\textmd{develop an }\textmd{electronic grammar}\textmd{ }\textmd{are not exactly the same }\textmd{as developing one aimed squarely at }\textmd{linguists.}}
\footnotetext{ See Weber (2000; 2006a: 418; 2006b: 458-459), Noonan (2006: 364), Evans \& Dench (2006: 28-30), Nordhoff (2008). In his papers, Weber discusses the limitations of printed reference grammars and the advantages of online (computational) reference grammars, without providing any negative aspects for them. He obviously favours electronic grammars, but he mostly considers reference grammars made by and for linguists.}
\subsection[5.1.1 EVALUATION OF ELECTRONIC GRAMMARS IN GENERAL. To summarize the positive and negative aspects of the online publication of reference grammars, we will take as a starting point some of Noonan{\textquoteright}s (2006) arguments, who evaluates this possibility, but with a linguistic audience in mind: ]{5.1.1 EVALUATION OF ELECTRONIC GRAMMARS IN GENERAL. \textmd{To summarize }\textmd{the }\textmd{positive and negative aspects of}\textmd{ the }\textmd{online publication of reference grammars, we will take as a starting point some of Noonan{\textquoteright}s}\textmd{ (2006)}\textmd{ arguments, who }\textmd{evaluates this possibility,}\textmd{ but with }\textmd{a }\textmd{linguist}\textmd{ic}\textmd{ audience in mind: }}
[{\dots}] online publication of grammars and dictionaries has a number of advantages over paper publication: online grammars and dictionaries can easily be updated and revised [{\dots}]. They can also be made available to a wider audience (especially if access is free) than is possible with paper publication. And lastly, online, or at least electronic, publication can facilitate the addition of audio and visual materials to the written text of the grammar. 

There are two problems with online publication. The first is that, in many cases, it is not evaluated as highly as paper publication for purposes of hiring, tenure, and promotion. [{\dots}]The second problem relates to the relative impermanence of electronic and online publication media (Noonan, 2006: 364).

\subsection[5.1.2. NEGATIVE ASPECTS OF ELECTRONIC GRAMMAR FOR NON{}-SPECIALIST USERS. \ It turns out that the advantages discussed by Noonan (2006) are also relevant for grammars intended for speakers, but before considering these, we wish to examine a number of problems with online grammars for the principal audience we have in mind, starting with those identified in Noonan (2006), as well as a few of our own.]{5.1.2. NEGATIVE ASPECTS OF ELECTRONIC GRAMMAR FOR NON-SPECIALIST USERS.  \textmd{It turns out that the advantages }\textmd{discussed }\textmd{by Noonan}\textmd{ (2006)}\textmd{ are also relevant for grammars}\textmd{ intended for speakers}\textmd{, but before considering these, we }\textmd{wish to }\textmd{examine }\textmd{a number of }\textmd{problems }\textmd{with }\textmd{online grammars for the principal audience we }\textmd{have }\textmd{in mind, starting with }\textmd{those identified in }\textmd{Noonan}\textmd{ (2006)}\textmd{, as well as a few}\textmd{ of our own}\textmd{.}}
The first issue brought up by Noonan (2006) is the potential hesitancy of some linguists to publish online grammatical descriptions, because this kind of work is less valued (for hiring, tenure, promotions, etc.) than printed publication. We should add to this the fact that publications addressed to non-linguist, printed or online, are also much less valued than are more specialized works. Minority language grammarians usually have to go beyond these considerations; otherwise, no under{}-described language would ever be documented or described. 

As for the second of Noonan{\textquoteright}s (2006) disadvantages, we think specialists of electronic grammaticography are well aware of the issue and are working to develop more enduring formats. Here, we must take into account the fact that not all linguists have the technical training and skills to elaborate electronic grammatical tools, as Weber (2006b: 459) admits. As he points out: {\textquotedblleft}Grammar writers need hospitable authoring environments, with tools that are powerful and flexible, yet reasonably easy to learn and use. Until these are available we labor under the limitations of ink-on-paper.{\textquotedblright} Actually, in the particular context of minority languages, where financial resources may be limited -- even for printed grammars -- an online grammar, with the complex infrastructure it requires, is a big challenge, as much for grammarians as for the speakers of these languages.

More specifically, we believe developing electronic grammars for minority languages poses a number of specific problems and difficulties that do not necessarily occur for grammars aimed at linguists, or for grammars of widely spoken languages. 

First of all, producing electronic grammars is not within every linguistic community{\textquoteright}s means, since it requires human and material (or financial) resources that are not available everywhere. As a matter of fact, members of these communities may not be familiar with new technologies, so they prefer something more traditional, such as grammar books.\footnote{ Here, we are not talking of \textit{producing }such a book, but of \textit{using} or \textit{reading} the document once it is published. However, we are well aware that perception about electronic products is evolving rapidly, mostly among young generations, and that present reservations may change faster than what was first thought.} Moreover, experts in technology may be lacking in these communities. As well, new technologies or access to the Internet may be inadequate (for example, there may be no access to high-speed transmission lines). These deficiencies might be temporary, being only a question of time or of one generation. For instance, Innu language professionals are still more familiar with printed material than with electronic material, but Innu youngsters are good users of all new technologies, including Internet: they like chat rooms, e{}-mails, etc. So, we expect they will be quite interested in reference material using Internet or other electronic technologies, in a more or less near future.\footnote{As a matter of fact, the Institut Tshakapesh is now collaborating with Marie-Odile Junker (Carleton University and \href{http://www.eastcree.org/}{\textstyleInternetlink{\textit{www.eastcree.org}}}) to develop different kinds of electronic grammatical tools for Innu: short grammatical explanations (\textit{capsules grammaticales}), a grammatical blog, grammatical exercises, use of Facebook, etc. We will also participate in this project. It will be a good example of complementariness between printed and electronic material to describe the grammar of a language.} 

Electronic grammars do not necessarily alleviate the grammarian{\textquoteright}s task; on the contrary, it  probably increases it, since in electronic grammars, there are no page limits, and because it is tempting to add information in various ways. There is therefore the danger of going too far, and in never completing the grammar. A good solution to avoid this pitfall is to make accessible an alpha version of the grammar, as work in progress, even if the description is not completed. Or to plan publications of parts of the grammar, before completion of the whole, as discussed in Nordhoff (2008).

As a matter of fact, it should be kept in mind that any good grammatical description, whether a printed or an electronic one, is based on same prerequisites: clear choices concerning end users and objectives, good access to linguistic data and examples, accuracy and soundness of the description and analysis of the language.

In spite of these inconveniences, we foresee that electronic grammars will become more important in grammaticography, even for grammars addressed to non-specialized speakers, at least where computers and Internet are available. And thanks to the joint efforts of many specialists in the domain of threatened languages, these tools will become accessible in a larger number of contexts.

\subsection[\ 5.1.3. POSITIVES ASPECT OF ELECTRONIC MEDIA FOR NON{}-SPECIALIST GRAMMAR USERS. Proponents of electronic tools for grammatical description especially underline the flexibility and the accessibility provided by these tools, and these points are certainly of great importance, for all types of grammars, intended as much for linguists as for non{}-specialists. Flexibility will be the main advantage of such a technology for our multilevel grammar, which we will explain in greater detail below. But first, we focus on flexibility and accessibility, in a more general way.]{ 5.1.3. POSITIVES ASPECT OF ELECTRONIC MEDIA FOR NON-SPECIALIST GRAMMAR USERS. \textmd{Proponents of electronic tools for grammatical description }\textmd{especially }\textmd{underline the flexibility and the accessibility }\textmd{provided }\textmd{by these tools, and these points are certainly of }\textmd{great }\textmd{importance, for all }\textmd{types }\textmd{of grammars, intended as much for linguists as for non-specialists. Flexibility will be the main advantage of such a technology for our multilevel grammar, }\textmd{which }\textmd{we will explain in }\textmd{greater }\textmd{detail}\textmd{ }\textmd{b}\textmd{elow. But firs}\textmd{t}\textmd{, }\textmd{we focus on flexibility and accessibility, in a more general way.}}
\subsection[5.1.3.1. FLEXIBILITY OF ELECTRONIC GRAMMARS FOR SPEAKERS. With electronic reference grammars, flexibility might be seen from two points of view, that of the authors, and that of the users.]{5.1.3.1. FLEXIBILITY OF ELECTRONIC GRAMMARS FOR SPEAKERS. \textmd{With electronic reference grammar}\textmd{s}\textmd{, flexibility might be seen from }\textmd{two }\textmd{points of view, that of the authors, and that of the users.}}
From the grammarian{\textquoteright}s perspective, electronic publishing gives better opportunities of revising and updating the grammatical document, as new information or knowledge about the language is made available, and this is quite important for the purpose of language documentation. As an interesting conse\-quence, there is the possibility of making available the grammar before it is completed. In doing so, the grammarian is able to validate his work: first, with the speakers, allowing him to verify the appropriateness of his description or analysis, or the relevance of the examples or linguistic data used; secondly, with the aimed{}-at users, to verify the readability of the grammatical text itself. 

Besides the possibility of up-dating a grammatical description, online grammars offer much more: a whole range of potential interactions between authors and users. These interactions can take different forms, such as the social media of Web 2.0;\footnote{ Supervising the project of the Cree on-line grammar, but also collaborating with Innu speakers, Marie-Odile Junker, is now working with social media and devising this kind of interactive on-line grammatical material aimed at non-specialists. } it can be integrated in the interface of an online grammar, or linked to it. Of course, interactions between grammarians and grammar users will depend on each linguis\-tic situation, and it must be well organized and supervised, to avoid any loss of control over the grammatical content.

Printed grammars are linear, meaning that each document is organized in  a single way, as each author has decided to present his work, for instance, from chapters to chapters, sections to sections, etc. The possibilities in elaborating the organization of electronic grammars are more varied, since they can provide different perspectives or different ways of navigating through the text.

From the user{\textquoteright}s point of view, online grammars might offer a flexible way to get to the required grammatical information; in other words, the user can adapt the grammar to his or her own needs, without getting lost in a profusion of grammatical information. The possibility of easily navigating through an electronic grammar is also an advantage over traditional printed grammars; this has to do with the next point, accessibility of the grammatical description.

\subsection[5.1.3.2. ACCESSIBILITY OF ELECTRONIC GRAMMARS FOR SPEAKERS. The fact that electronic grammars, especially online grammars, are more easily accessible than printed grammars seems evident. But this aspect may also be looked at from different vantage points.]{5.1.3.2. ACCESSIBILITY OF ELECTRONIC GRAMMARS FOR SPEAKERS. \textmd{The fact that electronic grammars, especially online grammars, are more easily accessible}\textmd{ than printed grammars}\textmd{ seems evident. But this aspect may also be looked at from}\textmd{ different vantage points}\textmd{.}}
First, we must take into account the fact that young people in minority language communities are quite attracted by the new media. Therefore, even if persons who are now working on a language -- teaching or describing it {}-- are not at ease with these new media, they will eventually have to take a stand on the matter. They will have to think not only about the future of their language, but also about the future of the young generations to whom they have to transmit it. Those who are not familiar with new technologies might not see their importance in youngsters{\textquoteright} lives but staking on new technologies in a language preservation program is a good investment, since it could meet young speakers{\textquoteright} interests.

Secondly, as mentioned previously, electronic grammars give an easy access to grammatical content; more precisely, it permits easy navigation through the grammatical text. In traditional printed grammars, one needs good tables of content, indices, cross-references, etc. In electronic grammars, such means are easier to use. As well, other kinds of links may be added; for example, to more examples, to texts illustrating the description, to a lexicon or a dictionary, or to a conjugation guide, to name but a few. 

\subsection[5.2. A MULTILEVEL GRAMMAR FOR INNU SPEAKERS. We will now explain how we see this type of grammar for the grammaticographic model we propose, based on our experience \ with Innu. ]{5.2. A MULTILEVEL GRAMMAR FOR INNU SPEAKERS. \textmd{We will now explain how we see th}\textmd{is }\textmd{type of grammar for the grammaticographic model we propose, based on our experience }\textmd{ with }\textmd{Innu. }}
In the case of Innu, when we started to work on the grammar intended for the speakers, we did not even think of the possibility of an electronic or online grammar since the only possibility at the time was a printed grammar. We now have to also consider new technologies, if only to take into account the needs and interests of younger speakers. 

\subsection[5.2.1. A PRINTED OR AN ON{}-LINE INNU GRAMMAR? Developing a grammar for a language that is under{}-described is a long{}-term task: often one starts from scratch, and, as the description progresses, \ it becomes {\textquotedblleft}larger and larger as time goes on, as it is a task for which there is no logical endpoint{\textquotedblright} (Rice, 2006: 400). For this reason, the grammar of Innu is not yet completed. And because the project was, from the beginning, to produce a printed grammar, we will achieve this objective, at least partly, to meet actual users{\textquoteright} expectations. In fact, our principal end users, Innu language teachers, are currently more at ease with traditional grammatical tools, \ i.e. books. But, we are also thinking about future users, who might be more familiar with new technologies, and probably would prefer such media. Therefore, we are contemplating a compromise, which consists in publishing, as soon as possible, a first volume of the Innu grammar, which would be mostly a description of basic structures, basic parts of speech (nouns, pronouns, verbs), as well as a description of inflections, since inflectional morphology is quite complex. And subsequently, we would pursue the grammatical description online.]{5.2.1. A PRINTED OR AN ON-LINE INNU GRAMMAR? \textmd{Developing }\textmd{a grammar for a language that is under}\textmd{{}-}\textmd{described is a long-term task}\textmd{:}\textmd{ }\textmd{often}\textmd{ one}\textmd{ }\textmd{start}\textmd{s}\textmd{ from scratch, and, as }\textmd{the description}\textmd{ progress}\textmd{es}\textmd{, }\textmd{ it becomes }\textmd{{\textquotedblleft}larger and larger as time goes on, as it is a task for which there is no logical endpoint{\textquotedblright} (Rice, 2006: 400). For th}\textmd{is}\textmd{ reason, }\textmd{the grammar of }\textmd{Innu is not yet completed. And because the project was, from the beginning, to produce a printed grammar, we will achieve th}\textmd{is}\textmd{ objective, at least partly, to meet actual users}\textmd{{\textquoteright}}\textmd{ expectations. }\textmd{In fact, }\textmd{our principal}\textmd{ end users}\textmd{, Innu language teachers, are}\textmd{ currently }\textmd{more at ease with traditional grammatical tools, }\textmd{ i.e. }\textmd{books. But, we }\textmd{are }\textmd{also }\textmd{thinking }\textmd{about future users, wh}\textmd{o}\textmd{ might be more familiar with new technologies, and probably would prefer such media. }\textmd{Therefore}\textmd{, we }\textmd{are }\textmd{contemplat}\textmd{ing}\textmd{ a compromise, which consists }\textmd{in}\textmd{ publish}\textmd{ing}\textmd{, as soon as possible, a first volume of the Innu grammar,}\textmd{ which would be }\textmd{mostly a description of basic structures, }\textmd{basic }\textmd{parts of speech (nouns, pronouns, verbs), }\textmd{as well as }\textmd{a description of inflections, since inflect}\textmd{ional}\textmd{ morphology is }\textmd{quite }\textmd{complex. }\textmd{And s}\textmd{ubsequently, we w}\textmd{ould}\textmd{ pursue the grammatical description online.}}
Therefore, we see printed and online projects as complementary, rather than opposed. This way may constitute a good transition between both kinds of production. Besides writing an online grammar to complement a printed volume, we may think about other ways to see future grammatical products, for instance, interactive tools, or grammatical sketches.

Building up an electronic infrastructure for Innu grammar is possible because the resources are available: Innu speakers have access to computers, at home or in schools; Innu language specialists are working closely with those who are developing a Website for the Cree language, which includes a dictionary and a grammar, and that makes it possible to benefit from what was developed for Cree, which is close to Innu.

\subsection[5.2.2. AN ELECTRONIC GRAMMAR OF INNU LANGUAGE. Even if we are not an expert in new technologies, particularly in the conception of electronic grammatical infrastructures, we are well aware that these technologies provide a very interesting option for our model of multilevel grammar. We will not discuss here which technologies could be used to make this kind of project achievable, but we will try to illustrate the possibilities we envisage with examples of Innu grammar.]{5.2.2. AN ELECTRONIC GRAMMAR OF INNU LANGUAGE. \textmd{Even if we are not an expert in new technologies, particularly in the conception of electronic grammatical infrastructures, we are well aware that these technologies provide a }\textmd{very }\textmd{interesting option for our model of multilevel grammar. }\textmd{W}\textmd{e will }\textmd{not }\textmd{discuss here wh}\textmd{ich}\textmd{ }\textmd{technologies }\textmd{could be }\textmd{used to make th}\textmd{is}\textmd{ kind of project}\textmd{ }\textmd{achievable}\textmd{, but}\textmd{ }\textmd{we will }\textmd{try to illustrate }\textmd{the }\textmd{possibilities }\textmd{we }\textmd{envis}\textmd{age}\textmd{ }\textmd{with examples of Innu grammar.}}
\subsection[5.2.2.1. AN ELECTRONIC MULTILEVEL OR MULTILAYER GRAMMAR. The main characteris\-tics of our grammaticographic model follow from the objective which consists in meeting as much as possible the needs of non{}-specialized users while adequately documenting the grammar of their language. To achieve this objective, we have proposed a grammar with different levels of reading or use, for different types of audience. Applying such a model to printed grammars means employing various typographical processes to differentiate each level. In fact, there are not many ways to reflect in a printed grammar the layered organization we wish for, since printed documents are basically linear. Furthermore, the various techniques that we can imagine are expensive, for example the use of colors, or the fact of requiring specialists such as book designers.]{5.2.2.1. AN ELECTRONIC MULTILEVEL OR MULTILAYER GRAMMAR. \textmd{The main characteris}\textmd{\-}\textmd{tics of our grammaticographic model follow from the objective which consists in meeting as much as possible }\textmd{the }\textmd{needs of }\textmd{non-}\textmd{specialized users while adequately documenting the grammar of their language}\textmd{.}\textmd{ }\textmd{T}\textmd{o achieve this objective, we }\textmd{have }\textmd{proposed a grammar with different levels of reading or use, for different types of audience. Applying such a model to printed grammars means employ}\textmd{ing}\textmd{ }\textmd{various }\textmd{typographical processes to differentiate each level. In fact, there are not many ways to reflect in a printed grammar the layered organization we}\textmd{ wish for}\textmd{, since printed document}\textmd{s}\textmd{ are basically linear. }\textmd{Furthermore, the various techniques that we c}\textmd{an}\textmd{ imagine are expensive, for example the use of colors, }\textmd{or the fact of}\textmd{ requiring specialists such as book designers.}}
Organizing a multilayered grammar is much easier with electronic media, once the infrastructure for a grammar is available, since each level (or layer) of information can be provided on different pages, with links between each level. In this way, the main text is not encumbered with unneeded information. In fact, all information aimed at other users than the principal end user {}-- advanced learner or speaker, linguist, second language learner, etc. -- is found in other layers, accessible by simply clicking on special tabs.  In this way, first level users will not be diverted or confused with a profusion of information. Moreover, it becomes possible for the user to {\textquotedblleft}follow his or her own path to explore{\textquotedblright} the grammatical description (Nordhoff 2008: 315). As for specialists, an electronic document may provide links to other publications, such as  academic articles, on particular linguistic structures described in the grammar.

For Innu, for example, there exists a  large lexical database  as well as a conjugational guide, with links between both. We imagine that associating a grammar with the above tools is undoubtedly feasible.

\subsection[5.2.2.2. ACCESSIBLE AND FLEXIBLE USE OF AN ONLINE GRAMMAR. Besides allowing navigating in the grammar from one level to another, electronic grammars are easier to use even with a single level, permitting accessible cross{}-referencing, links to a glossary of terminological terms, to a lexicon or to verb paradigms, etc. ]{5.2.2.2. ACCESSIBLE AND FLEXIBLE USE OF AN ONLINE GRAMMAR. \textmd{Besides }\textmd{allowing }\textmd{navigati}\textmd{ng}\textmd{ }\textmd{in}\textmd{ the grammar from one level to}\textmd{ }\textmd{an}\textmd{other, electronic grammars are }\textmd{easier }\textmd{to use even }\textmd{with a single}\textmd{ level}\textmd{, permitting accessible cross-referencing, links to a glossary of terminological terms, to a lexicon or }\textmd{to }\textmd{verb paradigms, etc. }}
Moreover, the presentation of the content of the grammar following both structural and functional perspectives is facilitated in an electronic grammar, with the possibility of links between both perspectives.  To give a concrete example, in Innu, 80 \% of words are verbs. As well, the verbal system is quite complex, with a rich derivational and inflectional morphology. Each verb has many conjugations, belonging to one of four verb classes, three orders of conjugations, many modes and tenses. For example, our \textit{Guide }\textit{de}\textit{s}\textit{ }\textit{conjuga}\textit{is}\textit{ons}, which we have developed, provides only verb paradigms, without any grammatical explanations, and yet is about 80 pages long. Also, to describe the verbs, we cannot simply present the morphology, but we must also describe the context of use of some features of the verbal system: orders refer to the syntax, the semantics and the pragmatics; modes and modalities refer to semantics, and so on. Making choices about the organization of the grammar, the ordering of the chapters, etc., is not easy, since there  are different options. Weber (2000: 2) observed:

The linear organization of grammars in no way reflects the structure of language itself. Language is an \textit{organic }whole, a complex of subsystems so tightly interwoven that change in one part generally has consequences in many other parts. Forcing a grammar into an outline is, in itself, a misrepresentation of its structure (one that I suspect has led to considerable frustration for most grammar writers).

In the specific case of Innu verbs, an electronic grammar could be more flexible than a printed grammar. We know that Innu teachers are not at ease with the \textit{Guide }\textit{des}\textit{ }\textit{conjug}\textit{ais}\textit{ons}, as it exists now. So we have to find a better way to present conjugations.

Another point to consider is the examples that illustrate the description or help in understanding the explanations. In a grammar addressed to non-specialists, it is essential to provide a good set of examples. And this is even more important for under{}-described or under{}-documented languages. The number of examples and the way they are presented are problematic in printed grammars. For instance, in a grammar written for a large audience, one would not find linguistic annotations such as are usually found in descriptions aimed primarily at linguists. In electronic grammars, there is more latitude in the matter, and there is the opportunity to link examples or explanations to other corpora or texts.

As another option, in the Innu grammar, we envisage adding links to grammatical exercises, and  perhaps different kinds of interaction between the grammarians and the users,  such as blogs, and on-line discussion groups.

\subsection[5.2.2.3. SOME FINAL REMARKS. \ At present, Innu speakers are expecting a printed grammar of their language, mostly because they are more used to this type of work. Therefore, we want to meet their needs in producing, as soon as possible, a first volume of the Innu grammar. But we know that it would require more time and work to achieve a more complete description of the grammar. Thus we are now contemplating the idea of publishing other parts of the grammar using electronic media, probably on Internet. As we have said previously, printed and electronic grammars should be seen as complementary rather than opposing tools. In some linguistic contexts, it is more realistic to start with a printed grammar book before having the resources to develop a grammar using new technologies. ]{5.2.2.3. SOME FINAL REMARKS.  \textmd{At present, Innu speakers are expecting a printed grammar of their language}\textmd{,}\textmd{ }\textmd{m}\textmd{ostly because they are more used to th}\textmd{is}\textmd{ }\textmd{type }\textmd{of work. }\textmd{Therefore, }\textmd{we want to meet their needs in producing, as soon as possible, a first volume of the Innu grammar. But we know that it would require more time and work to achieve a more complete description of the grammar.}\textmd{ Thus}\textmd{ we }\textmd{are }\textmd{now contemplat}\textmd{ing}\textmd{ the idea }\textmd{of }\textmd{publish}\textmd{ing}\textmd{ other parts of the grammar using electronic media, probably on Internet.}\textmd{ }\textmd{As we }\textmd{have }\textmd{said previously, printed and electronic grammars }\textmd{should }\textmd{be seen as complementary}\textmd{ rather than}\textmd{ opposi}\textmd{ng}\textmd{ tools. In some linguistic context}\textmd{s}\textmd{, it is more realistic to start with a }\textmd{printed }\textmd{grammar book before having the resources to }\textmd{develop a grammar using }\textmd{new technologies. }}
If developing an electronic grammar provides a number of solutions to various issues of grammar publication, it is also a real challenge. It will not make the writing of descriptions less burdensome. Moreover, it requires various resources,  building the infrastructure with the new media (electronic or online), in other words, people with good technical skills, as well as software to manage the elaboration of the grammar. And for non-linguists as eventual readership it is important to have a well-designed format, with attractive presentations, as we have proposed for a printed version, and this may require other kinds of expertise.

\subsection[6. CONCLUSION. Minority language situations are not all the same. Therefore we do not propose solutions for grammar writing that suit every under{}-described language. But we think it is vitally important to take into account the role of speakers in grammaticography, because they are the main actors in language maintenance and transmission. From our experience as a grammarian of Innu, we have elaborated a model of a multilevel grammar, which places the speakers of the language in the foreground, as well as considering other users, including linguists. Even if our grammaticographic model was first conceived for printed grammars, we have considered the possibility of applying it to electronic or online grammars; in other words, to apply electronic solutions to this model.]{6. CONCLUSION. \textmd{Minority language situations are not all the same. }\textmd{Therefore }\textmd{we do not }\textmd{propose }\textmd{solutions for grammar writing that suit every under}\textmd{{}-de}\textmd{scribed language. But we think it is }\textmd{vitally }\textmd{important to take into account the role of speakers in grammaticography, because they are the main actors in language maintenance}\textmd{ }\textmd{and transmission. From our experience as }\textmd{a }\textmd{grammarian of Innu, we have elaborated a model of }\textmd{a }\textmd{multilevel grammar, which places the speakers}\textmd{ of the language}\textmd{ in the foreground, }\textmd{as well as }\textmd{considering other users, including linguists. Even if our grammaticographic model was first }\textmd{conceived }\textmd{for printed grammars, we }\textmd{have }\textmd{considered the possibility }\textmd{of }\textmd{apply}\textmd{ing}\textmd{ it to electronic or online grammars}\textmd{; in other words}\textmd{, to apply electronic solutions to this model.}}
Is an electronic grammar a better medium for a multilevel grammar than a printed grammar? There is no simple answer to this question. Developing a grammar on the web may pose specific challenges, but it also shares problems with writing a printed grammar book. Instead of seeing both as being opposed, we believe they are complementary. In some linguistic communities, even printing a grammar book is a complex task, whereas others are already on the way of producing online grammars. The most important is to keep in mind the objective of giving the speakers a good grammatical description, on paper or online.

We believe that an online grammar could be a good solution to carry out a multilevel (or multilayered) model of grammar. But it involves resources, i.e. experts, software and hardware, etc., that are not necessarily within the reach of all grammarians or minority language communities.

Looking to the future, and taking in consideration the rapid progression of technological tools,  we can anticipate that various new technologies will become more and more accessible. Also, in view of the interest of younger generations in new technologies, we think online grammars aimed at non-specialists will have a bright future.

In our view, writing a high-quality reference grammar, whether electronic or printed, may be a good opportunity to transmit grammatical knowledge from linguists to speakers, a way to make sure that such knowledge will not remain ensconced in academia.

\subsection{References}
Baraby, Anne-Marie.~2000. Developing a standard orthography or an oral language : the Innu (Montagnais) experiment. In Nicholas Ostler \& Blair Rudes (eds.), \textit{Endangered }\textit{l}\textit{anguages and }\textit{l}\textit{iteracy. Proceedings of the Fourth FEL Conference}, 78-84. Bath (UK): Foundation for Endangered Languages.

Baraby, Anne-Marie. 2002. The process of spelling standardization of Innu-Aimun (Montagnais). In Barbara Burnaby \& Jon Allan Reyhner (eds.), \textit{Indigenous languages across the community}, 197-212. Flagstaff: Northern Arizona University Press.

Baraby, Anne-Marie.~2004. \textit{Guide de conjugaisons de la langue innue}. 2\textsuperscript{nd}~ed. Sept-\^Iles (QC) : Institut culturel et \'educatif montagnais.

Baraby, Anne{}-Marie. 2011a. \textit{Grammaticographie des langues minoritaires. Le cas de l{\textquoteright}innu}. Dissertation. Quebec: Universit\'e Laval.

Baraby, Anne-Marie. 2011b. Chapitre 3. L{\textquoteright}\'ecrit dans une langue de tradition orale: le cas de l{\textquoteright}innu. In Lynn Drapeau (ed.), \textit{Les langues autochtones du Qu\'ebec: Un patrimoine en danger}, 47-66. Quebec: Presses de l{\textquoteright}Universit\'e du Qu\'ebec.

Baraby, Anne-Marie \& Lynn Drapeau. Forthcoming. \textit{Grammaire de r\'ef\'erence de l{\textquoteright}innu.}

Dirven, Ren\'e.~1990. Pedagogical grammar. \textit{Language Teaching}, 23. 1-18.

Dixon, Robert M.~W. 1997. \textit{The rise and fall of languages}. Cambridge (UK): Cambdrige University Press.

Dixon, Robert M.~W. 2010. \textit{Basic linguistic theory}. \textit{V}\textit{ol.1.} \textit{Methodology}. New York: Oxford University Press.

Evans, Nicholas \& Alan Dench. 2006. Introduction: Catching language. In Felix~K. Ameka, Alan Dench \& Nick Evans (eds.), \textit{Catching language}\textit{: }\textit{The standing challenge of grammar writing}, 1-39. Berlin: Mouton de Gruyter.

Gippert, Jost, Nikolaus Himmelmann \& Ulrike Mosel (eds.). 2006. \textit{Essentials of language documentation}. Berlin: Mouton de Gruyter.

Lehmann, Christian. 1989. Language description and general comparative grammar. In\textit{ }Gottfried Graustein \& Gerhard Leitner (eds.), \textit{Reference grammars }\textit{and modern linguistic theory}, 133-162. T\"ubingen: Niemeyer.

Mithun, M.~2006. Grammars and the community. In Thomas Payne \& David Weber (eds.), \textit{Perspectives on grammar writing. Studies in language}, 30 (2), 281-306.

Nordhoff, Sebastian. 2008. Electronic reference grammars for typology: Challenges and solutions. \textit{Language documentation and conservation} 2 (2), 296-324, http://nflrc.hawaii.edu/ldc/.

Noonan, Michael. 2006. Grammar writing for a grammar-reading audience. In Thomas Payne \& David Weber (eds.), \textit{Perspectives on grammar writing. Studies in language}, 30 (2), 351-365.

Payne, Thomas E. \& David J. Weber (eds.). 2006. \textit{Perspectives on grammar writing. Studies in language}, 30 (2).

Rice, Keren 2006. A typology of good grammar. In Thomas E. Payne \& David J. Weber (eds.), \textit{Perspectives on grammar writing. }\textit{Studies in language}, 30 (2), 385-415.

Germain, Claude \& Hubert S\'eguin. 1995. \textit{Le point sur la grammaire en didactique des langues}. Anjou (QC): CEC.

Sinclair, John. (ed.). 2004. \textit{Collins Cobuild English Grammar}. Glasgow: Harper Collins Publishers.

Valentine, J. Randolph. 2001. \textit{Nishnaabemwin reference grammar}. Toronto: University of Toronto Press.

Weber, David J. 2000. Reference~grammar for the computational age: From Gleason files to Sci-Fi grammar, \textit{Linguistic exploration. New methods for creating, exploring and }\textit{disseminating linguistic field data}. http://\href{http://www.ldc.upenn.edu/exploration/LSA/weber}{\textstyleInternetlink{www.ldc.upenn.edu/exploration/LSA/weber}}. htm/.

Weber, David J.~2006a. Thoughts on growing a grammar. In Thomas Payne \& David Weber (eds.), \textit{Perspectives on grammar writing. Studies in language}, 30 (2), 415-444.

Weber, David J.~2006b. The linguistic example. In Thomas Payne \& David Weber (eds.), \textit{Perspectives on grammar writing. Studies in language}, 30 (2), 445-460.

\clearpage
baraby.anne-marie@uqam.ca
\end{document}
