\renewcommand\chapname{Contributors}	
\renewcommand\longchapname{Contributors}
\renewcommand\shortauthor{}
\renewcommand\longauthor{}
\chapter*{\longchapname} 
\mytoc{}  

\textbf{Tim Baldwin} completed a BSc(CS/Maths) and BA(Linguistics/Japanese)
at the University of Melbourne in 1995, and an MEng(CS) and PhD(CS) at
the Tokyo Institute of Technology in 1998 and 2001, respectively. He is
currently an Associate Professor and Deputy Head of the Department of
Computing and Information Systems, The University of Melbourne, and a
contributed research staff member of the NICTA Victoria Research
Laboratories. His research interests cover topics including deep linguistic
processing, multiword expressions, text mining of social media, computer-
assisted language learning, information extraction and web mining, with
a particular interest in the interface between computational and theoretical
linguistics.

\textbf{Anne-Marie Baraby} has been working on Innu language for the past thirty years, after having completed her studies in linguistics in the fields of Native American language description and of grammaticography of minority languages. Also working as instructor in linguistics among Innu language teachers, she is presently a part-time teacher in French grammar in the Département de linguistique at the Université du Québec à Montréal 

\textbf{Emily M. Bender} received her PhD in Linguistics from Stanford
University in 2001 and is presently an Associate Professor in the
Department of Linguistics at the University of Washington.  Her
primary research interests lie in grammar engineering.  She is the PI of
the Grammar Matrix project.

\textbf{Cheryl} and \textbf{ H. Andrew Black} are linguistic consultants with SIL Mexico (http://www.sil.org/mexico/00i-index.htm). They previously served with SIL in Peru.  They are adjunct faculty with the Summer Institute of Linguistics at the University of North Dakota (http://arts-sciences.und.edu/summer-institute-of-linguistics/). Andrew earned his PhD in Linguistics from the University of California, Santa Cruz in 1993 and Cheryl earned hers in Linguistics from the University of California, Santa Cruz in 1994.

\textbf{Peter Bouda} finished his M.A. in 2007 at the Institute of General Linguistics
and Language Typology at the Ludwig-Maximilian-University in Munich. He
then worked as a software developer for Linguatec GmbH in Munich and 
later as a freelancer in software development for mobile phones. He is now a
researcher within the project "Quantitative Historical Linguistics" at the University of Munich and is 
responsible for the development of the web application and the database 
design. His research focus is the design and usability of software used 
in linguistic research. He develops Python modules and applications that
 allow linguists to annotate and analyze their data


\textbf{Rebecca Dridan} received her PhD in Computational Linguistics from
Saarland University, Germany in 2009. She is currently employed as a
Postdoctoral Fellow in the Language Technology group at the University
of Oslo, where she is part of the WeSearch project. Her primary
research focus is on combining statistical and linguistic information
to extract meaning from text.


\textbf{Sebastian Drude} is the Scientific Coordinator of The Language Archive (TLA) at the Max-Planck-Institute for Psycholinguistics. He is a documentary / anthropological linguist interested in language technology and infrastructure.  Since 1998, he has conducted fieldwork among the Awetí indigenous group in Central Brazil, participating in the DOBES (Documentation of Endangered Languages) research program from 2000 on.  From 2008 on he was a Dilthey fellow at University Frankfurt, before in November 2011 he went to the MPI Nijmegen joining the leading group of TLA, which hosts the central DOBES language archive and develops tools and infrastructure for linguistics and the digital humanities.

\textbf{Sumukh Ghodke} is pursuing his PhD in the Language Technology Group,
University of Melbourne and is being advised by Assoc. Prof. Steven Bird.
His primary research interest is in database systems for managing large
collections of semi-structured data.
 
 

\textbf{Jeff Good} is Assistant Professor of Linguistics at the University at Buffalo. His research areas include examining the impact of new digital technologies on the practice of linguistics, documentation of languages of Northwest Cameroon, comparative Benue-Congo linguistics, and morphosyntactic typology.



\textbf{Johannes Helmbrecht} studied General and Comparative Linguistics, Philosophy,
and Psychology at the University of Bonn and the University of Cologne. He
received his PhD from the University of Bonn in 1994 with a thesis on the
concept of semantic roles. Areas of research later on were the morphosyntax of
East Caucasian languages, in particular Lak, and personal pronouns and person
marking in general and in North American Indian languages. He finished the
``Habilitation'' with a
thesis on the typology of personal pronouns at the University of Erfurt. He
conducted extensive fieldwork in Daghestan (Russia) and on Hocank, a North
American Indian language of the Siouan family in Wisconsin. He was principal
investigator together with Christian Lehmann of the DOBES project on the
documentation of the Hocank language. Since 2006, he  holds a chair in General
and Comparative Linguistics at the University of Regensburg.


\textbf{Mike Maxwell} is a researcher in grammar description and other computational
resources for low density languages, at the Center for Advanced Study of
Language at the University of Maryland. He has also worked on endangered
languages of Ecuador and Colombia, with the Summer Institute of
Linguistics, and on low density languages with the Linguistic Data
Consortium (LDC) of the University of Pennsylvania.


\textbf{Ulrike Mosel} is professor emerita of General Linguistics at the University of Kiel. After gaining her PhD in Semitic languages at the University of Munich (1974), she started researching South Pacific languages and became an expert in collaborative fieldwork. Her books include /Tolai Syntax /(1984), /Samoan Reference Grammar /(1992, with Even Hovdhaugen), /Say it in Samoan /(1997, with Ainslie So'o). Currently she is working on the documentation of the Teop language of Bougainville, Papua New Guinea. Together with Christian Lehmann, Hans-Jürgen Sasse and Jan Wirrer she initiated the DoBeS language documentation programme funded by the Volkswagen Foundation since 2000.
 


\textbf{Simon Musgrave} is a lecturer in the School of Languages, Cultures and
Linguistics at Monash University. He completed his doctorate at the
University of Melbourne in 2002, and was then a post-doctoral researcher at
Leiden University and an Australian Research Council post-doctoral fellow
at Monash.  His research interests include Austronesian languages, language
documentation and language endangerment, African languages in Australia,
communication in medical interactions, and the use of technology in
linguistic research. Major publications include the edited volumes *Voice
and Grammatical Relations in Austronesian* (2008) and *The Use of Databases
in Cross-linguistic Research *(2009). Simon has also been closely involved
in the Australian National Corpus project from an early stage, serving on
the steering committee for the first stage of the project as well as being
the treasurer of Australian National Corpus Inc.

\textbf{Sebastian Nordhoff}  is a postdoctoral researcher at the Max Planck Institute for Evolutionary Anthropology in Leipzig. He specializes in language contact and language change and the interface of language description and documentation on the one hand and electronic publication on the other. He is a member of the working group on Open Data in Linguistics of the Open Knowledge Foundation, where he works on integrating typological data into the Linguistic Linked Open Data Cloud. 


\textbf{Nicholas Thieberger} wrote a grammar of South Efate, a language from
central Vanuatu and is project manager for the digital archive
PARADISEC. He is interested in developments in e-humanities methods
and their potential to improve research practice and he is now
developing methods for creation of reusable data sets from fieldwork
on previously unrecorded languages. He is an Australian Research
Council QEII Fellow at the University of Melbourne.


