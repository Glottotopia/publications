\appendix
\renewcommand\chapname{Appendix}	
\renewcommand\longchapname{Appendix}
\renewcommand\shortauthor{}
\renewcommand\longauthor{}
\chapter*{\longchapname} 
\mytoc{} 

\section*{Nordhoff's maxims}
The following is a list of Nordhoff's (2008) maxims:

\begin{enumerate}

\item[1.] Data quality.
\item[1.1.] Accountability. We value application of the scientific method.

\ea  Every step of the linguistic analysis should be traceable to a preceding step, until the original utterance of a speaker is reached. \z

\ea  Every phenomenon described should be sourced using an actual utterance. \z

\ea  More sources for a phenomenon are better than fewer sources.  \z

\ea  The context of the utterance should be retrievable.  \z

\item[1.2.] Actuality. We value scientific progress.

\ea  A GD should incorporate provisions to incorporate scientific progress. \z

\ea  The GD should present state-of-the-art analyses. \z

\item[1.3.] History. We value the recognition of the historic evolution of ideas.

\ea  The GD should present both historical and contemporary analyses  \z

\item[2.] Creation
\item[2.1.] Layout assistance and templates. We value speed of creation and comparability.

\ea  Layout should be automatic as far as possible. \z

\ea  A GAP which provides templates is better (Weber 2006a:430, 434). \z

\item[2.2.] Creativity.We value the individual mind's expressive abilities.

\ea  A GAP that does not interfere with the creativity of the author is better  \z

\item[2.3.] Collaboration. 

\ea  A GAP that does not require the writers to be present at the same place is better \z

\ea  A GAP should show which collaborator has contributed what. \z

\ea  A GAP which can be used both online and offline is better  \z

\item[2.4.] Backup. We value safety of the data.

\ea  A GAP should provide the author with regular automated backups \z

\item[3.] Exploration.

\item[3.1.] Ease of finding. We value ease and speed of retrieving the information needed.

\ea  A GD which has a table of contents, an index, and full text search is preferable \z

\ea  A GD that does not require internet access is preferable \z

\item[3.2] Individual reading habits. We value the individual linguist's decisions as to what research questions could be interesting

\ea  A GD should permit the reader to follow his or her own path to explore it. \z

\ea  A short path between two related phenomena is better. \z

\item[3.3.] Familiarity. We value ease of access.

\ea  A GD that is similar to other GDs known to the reader is better \z

\item[3.4.] Guiding. We value an informed presentation of the data.

\ea  The GD should present the data in a didactically preferred way \z

\item[3.5.] Ease of exhaustive perception. We value the quest for comprehensive knowledge of a language. 

\ea  The readers should be able to know that they have read every page of the grammar. \z

\item[3.6] Relative importance. We value the allocation of scarce resources of time to primary areas of interest.

\ea  The relative importance of a phenomenon for (a) the language and (b) language typology should be retrievable. \z

\item[3.7.] Quality Assessment. We value indication of the reliability of analyses.

\ea  The quality of a linguistic description should be indicated \z

\item[3.8.] Persistence. We value citability.

\ea  In order to facilitate longterm reference, a grammatical description should not change over time. \z

\item[3.9.] Multilingualization. We value the interest of every human in a given language, especially interest from the speakers of the language in question.

\ea  A GD should be available in several languages, among others the language of wider communication of the region where the language is spoken. \z

\item[3.10] Manipulation. We value portability and reusability of the data.

\ea  The data presented in a GD should be easy to extract and manipulate \z

\item[3.11.] Tangibility. We value the appreciation of a grammatical description as a comprehensive aesthetic achievement.

\ea  A GD that can be held in the hand is better. \z
\end{enumerate}
