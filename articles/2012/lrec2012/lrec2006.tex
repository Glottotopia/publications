%
% lrec2006.sty
%
\typeout{}
\typeout{LREC 2006 Proceedings style -- Februar 2006}
\typeout{}
%
% ----------------------------------------------------------------------
% INTRO
%
% This is the LaTeX style file for LREC 2006. It is a
% modification of the lrec2000.sty which was based on
% several similar files like colacl.sty for COLING-ACL, and
% lrec98.cls, by A.Rubio-P.Garcia, for the Granada LREC.
%
% lrec2006.sty is designed for use as a package or option with the
% standard LaTeX article class, and the BibTeX style lrec2006.bst. 
%
% Author/title and citation formatting differs slightly from standard
% LaTeX; see AUTHOR FORMATS and CITATION FORMATS below for more
% information.
% 
% This file is supplied as a hopefully convenient implementation of 
% some of the "instructions for authors" repeated below. It is not
% guaranteed to work in any given LaTeX installation or in conjunction
% with any given class, package or style, and it is not intended as
% a LaTeX tutorial.
%
% ----------------------------------------------------------------------
% INSTRUCTIONS TO AUTHORS
% 
%   (i) Paper size: A4. Use 10pt should be the normal font size.  
%	The .tex preamble should include the following line:
%       \documentclass[10pt, a4paper]{article} 
%
%  (ii) Left, right, and bottom margins: 1.9 cm
%       Top margin: 2.5 cm 
%
%  (iii) Body of text to be set in two columns.  Full-width figures
%       (i.e. using \begin{figure*}) and tables may be used if
%       necessary. 
%
%  (iv) Times Roman is the default font. 
%
%  (v) No page numbering (pages should be numbered in pencil, on the
%       reverse side).
% 
% Items (ii), (iii), (iv) and (v) are handled by this file, and should
% therefore not be overridden by resetting \textwidth, \textheight,
% \pagestyle etc. in your document, or by using styles or packages
% which have the same effect.
%
% -------------------------------------------------------------------
% SAMPLE PAPER
% A sample paper template can be downloaded from the LREC site
%
%
% ----------------------------------------------------------------------
% AUTHOR FORMATS
% 
% ------------
% For several authors from the same institution:
%
% \name{Firstname1 Surname1, Firstname2 Surname2, ..., FirstnameN SurnameN}
%
% \address{Institution \\ Address \\ {mail1, mail2, ...,  mailN}@institution}
%
% If the names do not fit well on one line use
%
%         \name{Firstname1 Surname1 \\ {\bf \large Firstname2 Surname2} \\
%         ... \\ {\bf \large FirstnameN SurnameN }}
%
% ------------
%
% For authors from different institutions:
%
% \name{Firstname1 Surname1$^{\ast}$, Firstname2 Surname2$^{\ast}$, Firstname3 Surname3$^{\dagger}$, Firstname4 Surname4$^{\ast}$}
%
% \address{$^{\ast}$Institution1 \\ Address1 \\ 
%          {mail1, mail2, mail4}@Institution1} \\ \\
%          $^{\dagger}$Institution2 \\ Address3 \\ 
%          mail3@Institution2 }
%
% ------------
%
% If the title and author information does not fit in the area allocated,
% place \setlength\titlebox{<new height>} after \usepackage{lrec2006},
% where <new height> can be something larger than 2.0in
%
% ----------------------------------------------------------------------
% CITATION FORMATS
% 
% Three possible citation formats:
% "\cite{...}"      produces a citation like "(Author, 1999)"
% "\shortcite{...}" produces a citation like "(1999)"
% "\newcite{...}"   produces a citation like "Author (1999)"
%
% All three take an optional argument which can be used to add page
% references, etc.:
% "\newcite[1--6]{...}" produces a citation like "Author (1999, 1--6)"
%
% ----------------------------------------------------------------------
% IF IT DOESN'T WORK
%
% The error message "File `lrec2006.sty' not found." indicates that this
% file has not been installed in a location which is visible to your 
% LaTeX.  Try putting it in the same directory as your paper, and
% running LaTeX there.  Consult your `Local Guide' documentation or
% your system administrator to find out how LaTeX searches for input
% files.
%
% "\documentclass..." is a LaTeX2e declaration.  An error message
% "Undefined control sequence." followed by a line ending in
% "\documentclass" indicates that you have used this with an obsolete
% LaTeX installation.

% Some interactions with other packages may still occur.  In order to
% remove the page number from the first page, you may have to place the
% "\thispagestyle{empty}" command immediately after "\maketitle".
%
% ----------------------------------------------------------------------
% NOTE:  Some laser printers have a serious problem printing TeX output.
% These printing devices, commonly known as "write-white" laser
% printers, tend to make characters too light.  To get around this
% problem, a darker set of fonts must be created for these devices.
%
% ----------------------------------------------------------------------



% ----------------------------------------------------------------------
% Margins, columns, etc

\setlength\topmargin{-0.1cm}
\setlength\oddsidemargin{-0.7cm}
\setlength\evensidemargin{-0.7cm}
\setlength\textheight{25.4cm}
\setlength\textwidth{17.2cm}
\setlength\columnsep{0.63cm}
\setlength\columnwidth{8.25cm}
\newlength\titlebox
\setlength\titlebox{2.0in}
\setlength\headheight{0pt}
\setlength\headsep{0pt}
\setlength\footskip{0pt}                % irrelevant when no footers.
\pagestyle{empty}			% no page numbers
\thispagestyle{empty}			% no page numbers

\setlength{\baselineskip}{11pt} %LREC 98 

\flushbottom
\twocolumn
\sloppy

% ----------------------------------------------------------------------
% Times Roman will be the default font. 
% The following is copied from times.sty
\renewcommand{\sfdefault}{phv}
\renewcommand{\rmdefault}{ptm}
\renewcommand{\ttdefault}{pcr}

% ----------------------------------------------------------------------
% We're never going to need a table of contents, so just flush it to
% save space --- from colacl.sty
\def\addcontentsline#1#2#3{}

% ----------------------------------------------------------------------
% Title and abstract, from lrec98.cls

\def\maketitleabstract{\par
 \begingroup
 \def\thefootnote{}
 \def\@makefnmark{\hbox
 to 0pt{$^{\@thefnmark}$\hss}}
 \if@twocolumn
 \twocolumn[\@maketitleabstract]
 \else \newpage
 \global\@topnum\z@ \@maketitleabstract \fi\@thanks
 \endgroup
 \setcounter{footnote}{0}
 \let\maketitleabstract\relax
 \let\@maketitleabstract\relax
 \gdef\thefootnote{\arabic{footnote}}\gdef\@@savethanks{}%
 \gdef\@thanks{}\gdef\@author{}\gdef\@title{}\let\thanks\relax}

\def\@maketitleabstract{\newpage
 \null
 \vskip 2em \begin{center}
 \vspace*{-.25cm}{\Large \bf \@title \par} \vskip 1.5em \lineskip .5em
 \begin{tabular}[t]{c} \large\bf \@name\\ \normalsize \@address \end{tabular}\par 
 \@abstract
 \end{center}
 \par
 \vspace*{0.2cm}
}


\def\title#1{\gdef\@title{#1}}
\def\name#1{\gdef\@name{#1\\}}
\def\address#1{\gdef\@address{#1\\}}
\def\abstract#1{\gdef\@abstract{
  {\bf \normalsize \abstractname\vspace{-.5em}\vspace{0.2cm}}
  \setlength{\baselineskip}{10pt}\parbox{17.2cm}{\small #1 \par}
}}

%\def\endabstract{\if@twocolumn\else\endquotation\fi}
\gdef\@title{\uppercase{title of paper}}
\gdef\@name{{\em Name of author}\\}
\gdef\@address{Address - Line 1 \\
               Address - Line 2 \\
               Address - Line 3}
\gdef\@abstract{{\em Abstract content}}

\let\@@savethanks\thanks
\def\thanks#1{\gdef\thefootnote{}\@@savethanks{#1}}
\def\sthanks#1{\gdef\thefootnote{\fnsymbol{footnote}}\@@savethanks{#1}}


% ----------------------------------------------------------------------
% bibliography and citations
% From colacl.sty

% most of cite format is from aclsub.sty by SMS

% don't box citations, separate with ; and a space
% Replaced for multiple citations (pj) 
% don't box citations and also add space, semicolon between multiple
% citations
%
\def\@citex[#1]#2{\if@filesw\immediate\write\@auxout{\string\citation{#2}}\fi
  \def\@citea{}\@cite{\@for\@citeb:=#2\do
     {\@citea\def\@citea{; }\@ifundefined
       {b@\@citeb}{{\bf ?}\@warning
        {Citation `\@citeb' on page \thepage \space undefined}}%
 {\csname b@\@citeb\endcsname}}}{#1}}

% Allow short (name-less) citations, when used in
% conjunction with a bibliography style that creates labels like
%	\citename{<names>, }<year>
% 
\let\@internalcite\cite
\def\cite{\def\citename##1{##1, }\@internalcite}
\def\shortcite{\def\citename##1{}\@internalcite}
\def\newcite{\leavevmode\def\citename##1{{##1} (}\@internalciteb}

% Macros for \newcite, which leaves name in running text, and is
% otherwise like \shortcite.
%
\def\@citexb[#1]#2{\if@filesw\immediate\write\@auxout{\string\citation{#2}}\fi
  \def\@citea{}\@newcite{\@for\@citeb:=#2\do
    {\@citea\def\@citea{;\penalty\@m\ }\@ifundefined
       {b@\@citeb}{{\bf ?}\@warning
       {Citation `\@citeb' on page \thepage \space undefined}}%
% gjr: hbox causes too many bad linebreaks
%\hbox{\csname b@\@citeb\endcsname}}}{#1}}
     {\csname b@\@citeb\endcsname}}}{#1}}

\def\@internalciteb{%
  \@ifnextchar [{\@tempswatrue\@citexb}{\@tempswafalse\@citexb[]}}

\def\@newcite#1#2{{#1\if@tempswa, #2\fi)}}

% gjr: no labels in this bibliography style
%\def\@biblabel#1{\def\citename##1{##1}[#1]\hfill}
\def\@biblabel#1{}

%%% More changes made by SMS (originals in latex.tex)
% Use parentheses instead of square brackets in the text.
\def\@cite#1#2{({#1\if@tempswa , #2\fi})}

% Don't put a label in the bibliography at all.  Just use the unlabeled format
% instead.
% gjr: removed \@mkboth -- no headers here.
% gjr: reduced vertical space between entries (plus was .33em)
%
\def\thebibliography#1{%
  \section{References}    % References section will be numbered in lrec2006.sty
  \list{}{\setlength{\labelwidth}{0pt}
          \setlength{\leftmargin}{0.35cm}
          \setlength{\itemsep}{0.11ex plus 0.11ex}
          \setlength{\parsep}{0ex}
          \setlength{\itemindent}{-0.35cm}}
  \def\newblock{\hskip .11em plus .11em minus -.07em}
  \sloppy\clubpenalty4000\widowpenalty4000
  \sfcode`\.=1000\relax}
\let\endthebibliography=\endlist

\def\@lbibitem[#1]#2{\item[]\if@filesw 
      { \def\protect##1{\string ##1\space}\immediate
        \write\@auxout{\string\bibcite{#2}{#1}}\fi\ignorespaces}}

\def\@bibitem#1{\item\if@filesw \immediate\write\@auxout
       {\string\bibcite{#1}{\the\c@enumi}}\fi\ignorespaces}

% ----------------------------------------------------------------------
% Section headings with less space

% Instead of the default section headings, we have numbers
% followed by dots
% 1. Section
% 1.1. Subsection
% 1.1.1. Subsection

\renewcommand{\thesection}{\arabic{section}.}
\renewcommand{\thesubsection}{\thesection\arabic{subsection}.}
\renewcommand{\thesubsubsection}{\thesubsection\arabic{subsubsection}.}

\def\Keywords{
{\bf{Keywords}:\,\relax%
}}


\def\section{%
    \@startsection{section}{1}{\z@}%
                  {-2.0ex plus -0.5ex minus -0.3ex}%
                  {0.8ex plus 0.3ex minus 0.2ex}%
                  {\large\bf\center}}
\def\subsection{%
    \@startsection{subsection}{2}{\z@}%
                  {-2.0ex plus -0.5ex minus -0.3ex}%
                  %{-1.4ex plus -0.4ex minus -0.2ex}%
                  {0.6ex plus 0.2ex minus 0.1ex}%
                  {\normalsize\bf\raggedright}}
\def\subsubsection{%
    \@startsection{subsubsection}{3}{\z@}%
                  {-2.0ex plus -0.5ex minus -0.3ex}%
                  %{-0.8ex plus -0.3ex minus -0.1ex}%
                  {0.001ex}%
                  {\normalsize\bf\raggedright}}
\def\paragraph{%
    \@startsection{paragraph}{4}{\z@}%
                  {-0.8ex plus -0.3ex minus -0.1ex}%
                  {-1em}%
                  {\normalsize\bf}}
\def\subparagraph{%
    \@startsection{subparagraph}{5}{\parindent}%
                  {0.4ex plus 0.3ex minus 0.1ex}%
                  {-1em}%
                  {\normalsize\bf}}


% ----------------------------------------------------------------------
% Footnotes
% From lrec98.cls

\let\footnotesize=\small
\def\footnoterule{
   \vspace*{0.1cm}\noindent\rule{5cm}{0.001cm}\vspace*{0.1cm}
}

% ----------------------------------------------------------------------
% Lists and paragraphs

% From lrec98.cls

\setlength{\parindent}{0cm}

% The first paragraph is indented as well.
% From package indentfirst.sty

\let\@afterindentfalse\@afterindenttrue
\@afterindenttrue


\leftmargin 2em \leftmargini\leftmargin \leftmarginii 2em
\leftmarginiii 1.5em \leftmarginiv 1.0em \leftmarginv .5em \leftmarginvi .5em
\labelwidth\leftmargini\advance\labelwidth-\labelsep \labelsep 5pt

% ----------------------------------------------------------------------
% Floats (figures, tables etc.)
% From colacl.sty
%
% Allow a larger proportion of the column/page to be taken up with
% floats than the standard classes.  Also discourage the creation of
% columns/pages containing only floats.

% Maximum fraction of the page that can be occupied by floats:
%
\renewcommand\topfraction{.9}
\renewcommand\bottomfraction{.5}
\renewcommand\dbltopfraction{.9}	% 2-column floats

% Minimum fraction of page that can be occupied by text:
%
\renewcommand\textfraction{.1}

% Maximum fraction of a page that can be occupied by floats before a
% separate float page is produced:
%
\renewcommand\floatpagefraction{0.9}
\renewcommand\dblfloatpagefraction{.9}	% 2-column floats


% ----------------------------------------------------------------------
% End of lrec2006.sty
% ----------------------------------------------------------------------


