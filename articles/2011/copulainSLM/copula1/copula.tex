\documentclass[a4paper,12pt]{article}
%-------------general packages-----------
\usepackage[utf8]{inputenc}
\usepackage[T1]{fontenc}
\newcommand{\xbox}[2]{\noindent\parbox[t]{#1}{#2}\noindent}
\newcommand{\xref}[1]{(\ref{#1})}     

\usepackage{gb4e,tipa,lingsty,ipashortcuts}
\usepackage[authoryear]{natbib}
\bibpunct[:]{(}{)}{,}{a}{}{,}
\setlength{\bibsep}{0.05cm}

\newcommand{\trs}[2]{{\em #1\em} `#2'}
\newcommand{\ea}{\\\\}
\newcommand{\z}{\\\\}
%opening
\title{Having come to be a copula in Sri Lanka Malay -- an unusual grammaticalization path}
\author{Sebastian Nordhoff}
\begin{document}

\maketitle

\begin{abstract}
Sri Lanka Malay has innovated a prominent and productive copula, which sets it apart from other descendents of colloquial Malay varieties. This copula has developed from the verb \trs{dhaathang}{to come}, which is a grammaticalization path not attested in the literature so far. This paper describes the forms and functions of this copula and shows that it cannot be traced to any of the main input languages of Sri Lanka Malay (Trade Malay, Tamil, Sinhala). Comparing the Sri Lanka Malay case to attested grammaticalization paths, the paper concludes that the grammaticalization of COME to a copula is less surprising when assuming intermediate stages of `resultative' and `stative'. These subpaths are illustrated by a variety of Creole and non-Creole languages.
\end{abstract}

\section{Introduction}
\thanks{I would like to thank Eva van Lier and Benard Comrie for comments on earlier version of this paper. The usual disclaimers apply.}
Languages with copulas are widely known in Europe and beyond, but languages without a copula are also frequently encountered. If languages without a copula develop one over time, the source is normally either a former locational verb or a deictic \citep[91]{Stassen1997}. In this paper, I want to highlight yet another source for a copula, namely the participle of the verb COME, which has grammaticalized into a copula in Sri Lanka Malay.
 
\citet{Slomanson2006cll} and \citet{Ansaldo2005ms} have shown this and that

I will first give some information on Sri Lanka Malay (Section \ref{sec:introslm})
and the types of copula sentences we find in that language (Section \ref{sec:slmcopula}).
I will then show that neither other varieties of Malay nor the contact languages Sinhala and Tamil provide models for the development of this particular copula in Sri Lanka Malay (Section \ref{sec:contactlgs}). I then discuss common sources for the grammaticalization of copulas in Section \ref{sec:sources} and common targets of grammaticalization for the verb COME in Section \ref{sec:targets}, showing that the particular combination COME$\to$copula is not attested. However, three subpaths, namely  COME$\to$resultative, resultative$\to$stative and stative$\to$copula are attested, which, when chained, can explain the Sri Lanka Malay facts.

\section{Sri Lanka Malay}\label{sec:introslm}
Sri Lanka Malay is the language of the descendents of soldiers, exiles, convicts and slaves who were brought to the island by the colonial powers of the Dutch and the British between roughly 1650 and 1850 \citep{Hussainmiya1990}. The grammar of the language has changed dramatically ever since \citep{Adelaar1991,SmithEtAl2004,Ansaldo2008genesis,Nordhoff2009phd} and has shed most of its Austronesian origins to align with the Sri Lankan sprachbund \citep{Bakker2006}, although some traces of Malay grammar still remain \citep{Slomanson2006cll}. For instance, the language has changed from SVO to SOV and from prepositional to postpositional. The language has developed more than 20 new bound morphemes, mainly for the marking of TAM and case, which is virtually unheard of in other parts of the Malay world.  Most of the changes the language has undergone since its arrival on the island 350 years ago were triggered by language contact. The copula, however, does not seem to have a clear origin in any contact language or any historical variety of Malay.

\section{The copula in Sri Lanka Malay}\label{sec:slmcopula}
The copula in Sri Lanka Malay is homophonous to the conjunctive participle of the verb \trs{dhaathang}{come}.
The Sri Lanka Malay verb can take a number of prefixes, one of them being the conjunctive participle. This form is used to indicate that the event depicted in the clause thus marked is prior to the event in the main clause. In this sense, it is similar to an English construction like \em Having done X, having done Y, finally Z happened\em. An illustration of this form is given in \xref{ex:intro:cp:pirrang}.

\xbox{\textwidth}{
\ea\label{ex:intro:cp:pirrang}
 \ea 
 \gll Oorang pada \textbf{asà-}pìrrang, \\
	 man \textsc{pl} \textsc{cp}-wage.war  \\
	`After having waged war'
	\ex
 \gll derang=nang \textbf{asà-}banthu, \\
	\textsc{3pl}=\textsc{dat} \textsc{cp}-help\\
	`and after having helped them'
	\ex
	\gll siini=jo su-cii\u n\u ggal. \\ % bf
	here=\textsc{emph} \textsc{past}-settle\\
	`the people settled down right here.' (K051222nar03)
	\z
\z
}


% K060108nar02.txt:\tx Kandinang   asadhaathang Kandika    asakaavingapa       itthunang
% K060108nar02.txt:\tx blaakangjo    kithang pada anabissar 

Alternatives to the form with \em asà- \em shown in \xref{ex:intro:cp:pirrang} are the form with \em =apa \em shown in \xref{ex:intro:cp:apa} and the form with both \em asà- \em and \em =apa\em, shown in \xref{ex:intro:cp:asaapa}. These three forms seem to be functionally equivalent.

\xbox{\textwidth}{
\ea\label{ex:intro:cp:apa}
\ea
\gll Oorang pada  thiikam=\textbf{apa}\\
man \textsc{pl}=\textsc{dat} shoot=\textsc{cp}\\
`People having been stabbed,/People were stabbed and'
\ex
\gll oorang pada=nang thee\u mbak=\textbf{apa}\\
man \textsc{pl} stab=\textsc{cp}\\
`people having been shot,/people were shot and'
\ex
\gll se=dang bannyak  creeveth pada su-aada. \\
        \textsc{1s=dat} much trouble \textsc{pl} \textsc{past}-exist \\
    `I had a lot of problems.' (K051213nar01)
\z
\z
}

\xbox{\textwidth}{
\ea\label{ex:intro:cp:asaapa}
  \ea 
  \gll Siithu \textbf{asà}-blaajar=\textbf{apa}, \\
  there \textsc{cp}-learn=after\\
  `After having learned there,'
 \ex
  \gll thaaun nnamblas=ka         se   skuul \textbf{asà}-luppas=\textbf{apa}, \\
  year sixteen=\textsc{loc} \textsc{1s} school \textsc{cp}-leave=after\\
  `after having left the school at 16,'
	\ex 
  \gll pukuran asà-caari        anà-pii. \\ % bf
  work \textsc{cp}-find \textsc{past}-go\\
  `I looked for work and went (away).' (K060108nar01)
  \z
  \z
}
 
The Sri Lanka Malay copula is built around the lexical verb \trs{dhaathang}{come} in its conjunctive participle form. All three forms mentioned above are possible for the copular use. The following three examples illustrate this.\footnote{Note that the word order in copula sentences is NP Cop NP, while in verbal sentences, it is typically NP NP V.}

\xbox{\textwidth}{
\ea\label{ex:intro:copula:asadhaathang}
\gll Se=ppe    naama \textbf{asà}dhaathang  Cintha Sinthani. \\
     \textsc{1s}=\textsc{poss} name \textsc{copula} Chintha Sinthani. \\
    `My name is Chintha Sinthani.' (B060115prs04)
\z
}

\xbox{\textwidth}{
\ea\label{ex:intro:copula:dhaathangapa}
\gll Se=ppe    baapa  dhaathang\textbf{apa}  Jinaan Samath. \\
     \textsc{1s}=\textsc{poss} father \textsc{copula} Jinaan Samath. \\
    `My father was Jinaan Samath.' (N060113nar03)
\z
}


\xbox{\textwidth}{
\ea\label{ex:intro:copula:asadhaathangapa}
\gll {\em Estate}=pe {\em field} {\em officer} \textbf{asà}dhaathang\textbf{apa}  kithang=pe     kaake\\
     estate=\textsc{poss} field officer \textsc{copula} \textsc{1pl}=\textsc{poss} grandfather. \\
    `The estate field officer was our grandfather.' (N060113nar03)
\z
}



The three examples above illustrate the use of \em (asà)dhaathang(apa) \em as a copula. Any of the three forms mentioned could also be used as a conjunctive participle proper when combined with \trs{dhaathang}{come}, as shown in the following three examples.


\xbox{\textwidth}{
\ea\label{ex:intro:cp:asdhaathang}
\ea
\gll Moonyeth pada=le \textbf{asà-dhaathang}\\
  monkey \textsc{pl}=\textsc{addit} \textsc{cp}-come \\
`The monkeys would go and'
\ex
\gll creeveth  athi-kaasi. \\
       trouble \textsc{irr}-give\\
    `cause (some other) trouble.'  (K070000wrt01)
\z
\z    
}
% 
% \xbox{\textwidth}{
% \ea\label{ex:intro:cp:asdhaathang} 
% \gll [Itthu nigiri=deri     \textbf{as-dhaathang}    anà-thii\u n\u ggal  oorang pada]=jo       kithang. \\
%      \textsc{dist} country=\textsc{abl} \textsc{cp}-come \textsc{past}-stay man \textsc{pl}=\textsc{emph} 1\textsc{pl}. \\
%     `The people who have come from those countries and stayed (here) are we.' (K051222nar03)
% \z
% }



\xbox{\textwidth}{
\ea\label{ex:intro:cp:dhaathangapa} 
\ea
\gll oorang mlaayu siithu=dering  \textbf{dhaathang=apa}\\
     man malay there=\textsc{abl} come=\textsc{cp} \\
    `The Malay men came from there and'  
\ex
\gll  cinggala  raaja=nang=le  anà-banthu\\
      Sinhala king=\textsc{dat}=\textsc{addit} past-help \\
    `helped the Sinhalese king.' (K051206nar04)

\z
\z
} 


 

% 
% 
% \xbox{\textwidth}{
% \ea\label{ex:intro:cp:dhaathangapa} 
% \ea
% \gll Malay {\em regiment} hatthu \textbf{dhaathangapa}    \\
%       Malay regiment \textsc{indef} come=\textsc{cp} \\
%     `A Malay regiment came and'
% \ex
% \gll  sini=ka {\em settle}=\textbf{apa} 
%       here=\textsc{loc} settle=\textsc{cp} \\
%     `settled down here and'
% \ex
% \gll itthu=nang      blaakang bannyak oorang pada siini duuduk=\textbf{apa} \\
%      dist=\textsc{dat} after many man \textsc{pl} here stay=\textsc{cp}  \\
%     `after that many people stayed here and ' (nosource)
% \ex
% \gll {\em forces} pada=nang   su-ambel \\
% forces \textsc{pl}=\textsc{dat} \textsc{past}-take\\
% `the army took them (in).'(G051222nar03)
% \z
% \z
% } 
 


\xbox{\textwidth}{
\ea\label{ex:intro:cp:asadhaathangapa} 
\ea
\gll incayang  islaam=nang   \textbf{asà-dhaathang=apa}  \\
     \textsc{3s}.\textsc{polite} Islam=\textsc{dat} \textsc{cp}-come=\textsc{cp} \\
    `He came to Islam'
\ex
\gll inni     siigith=nang    asà-dhaathang \\
     \textsc{prox} mosque=\textsc{dat} \textsc{cp}-come  \\
    `and came to the mosque'
\ex
\gll giini    girja \\
     like.this make  \\
    `and does like this (=observes Islamic rites).' (K051220nar01)
\z
\z
} 
 

The interpretation as involving an event of coming is mandatory in examples \xref{ex:intro:cp:asdhaathang}-\xref{ex:intro:cp:asadhaathangapa}, but impossible in examples \xref{ex:intro:copula:asadhaathang}-\xref{ex:intro:copula:asadhaathangapa}. We are thus dealing with polysemy of the form \em (asà)dhaathang(apa)\em.
 
% \subsection{Functional properties of the copula}
In the languages of the world which have a copula, it is normally used for one or more types of non-verbal predicates \citep{Stassen1997}. The predication types which are often found to require support by a copula are property assignment (\em John is tall\em), class inclusion (\em John is a doctor\em), existence (\em There is a solution\em) and location (\em John is in Dubai\em). Additionally, identification (\em The murderer was the butler; I am John\em) and specification (\em The president is Obama\em) also often require the use of a copula \citep{Hengeveld1992nvpttd}. This is actually exactly what we find in English, where the copula \em to be \em is present in all these cases.

In Sri Lanka Malay, the copula is mainly used for naming people and for indications of group membership. Property assignment or location are not areas which trigger the use of the copula. Among these uses, naming   is the most common one and has been illustrated in \xref{ex:intro:copula:asadhaathang} and \xref{ex:intro:copula:dhaathangapa} above.
Apart from naming, the copula is most frequently used for indicating membership in ethnic,  professional or gender classes or groups. These three uses are given in \xref{ex:func:ethnicgroup}-\xref{ex:func:gender}. The three forms with \em asà-, =apa \em and \em asà-...=apa \em appear to be used indiscriminately for these functions.


\xbox{\textwidth}{
\ea\label{ex:func:ethnicgroup}
\ea
\gll Se=ppe    {\em daughter-in-law}=pe {\em mother}  asàdhaathang \textbf{bìnggaali}$_{ethnic group}$. \\
      \textsc{1s}=\textsc{poss} daughter-in-law=\textsc{poss} mother \textsc{copula} Bengali \\
    `My daughter-in-law's mother is Bengali.'
\ex
\gll Ithukapang       {\em daughter-in-law}=pe     {\em father}  asàdhaathang \textbf{mlaayu}$_{ethnic group}$. \\
      then daughter-in-law=\textsc{poss} father \textsc{copula} Malay \\
    `Then my daughter-in-law's father is Malay.' (K051206nar08)
\z
\z
}

\xbox{\textwidth}{
\ea\label{ex:func:profession}
\gll [Seelong=nang  duppang duppang anà-dhaathang  mlaayu] asàdhaathang \textbf{oorang} \textbf{ikkang}$_{profession}$. \\
      Ceylon=\textsc{dat} before before \textsc{past}-come Malay] \textsc{copula} man fish \\
    `The Malays who came to Ceylon very early were fishermen.' (K060108nar02)
\z
}

\xbox{\textwidth}{
\ea\label{ex:func:gender}
\gll Kàthama aanak dhaathangapa \textbf{klaaki}$_{gender}$. \\
      first child \textsc{copula} male \\
    `My oldest child is a boy.' (G051222nar01)
\z
}


Another use frequently found use is the indication of kinship relations, another type of class membership. Example  \xref{ex:func:kin} illustrates this.

\xbox{\textwidth}{
\ea\label{ex:func:kin} 
\gll Baapa=pe      umma   asàdhaathang  \textbf{kaake=pe}           \textbf{aade}$_{kin}$. \\
    father=\textsc{poss} mother \textsc{copula} grandfather=\textsc{poss} younger.sibling. \\
    `My paternal grandmother was my grandfather's younger sister.' (K051205nar05)
\z
}






All the functions mentioned above, with the exception of naming, have in common that they indicate group membership. Another function is the use of the copula in identificational or specificational sentences. The following two examples show the use of the copula to further specify or identify the nature of a problem.

\xbox{\textwidth}{
\ea \label{ex:ident:JVP}
\gll [Itthu    vakthu kithang=nang nya-aada]     \textbf{asàdhaathang} ini      JVP katha hathu  {\em problem}. \\
     \textsc{dist} time \textsc{1pl}=\textsc{dat} \textsc{past}-exist \textsc{copula} \textsc{prox} JVP \textsc{quot} \textsc{indef} problem. \\
    `What we had at that time was the so-called JVP-problem.' (K051206nar10)
\z
}

\xbox{\textwidth}{
\ea\label{ex:ident:thamaoomong}
\gll suda karang [kithang=nang   aada  {\em problem}] \textbf{dhaathangapa} kithang=pe     aanak pada mlaayu thama-oomong \\
       thus now \textsc{1pl}=\textsc{dat} exist problem \textsc{copula} \textsc{1pl}=\textsc{poss} child \textsc{pl} Malay \textsc{neg.nonpast}-speak\\
    `So, the problem we are having now is that our children do not speak Malay.' (G051222nar01)
\z
} 

To sum up, of the functions commonly associated with the copula, `class membership' seems to be central for the Sri Lanka Malay case, while identification is another use. Identificational uses of the copula are very common as far as naming is concerned, but far less common for the types of identification and specification illustrated with examples \xref{ex:ident:JVP}-\xref{ex:ident:thamaoomong}.

The domains of property assignment and location do not seem to trigger the use of the copula in Sri Lanka Malay, as illustrated in \xref{ex:contr:property} and \xref{ex:contr:location}.

\xbox{\textwidth}{
\ea\label{ex:contr:property}
\gll Thee\umb ok (*asàdhaathang) puuthi \\
     wall \textsc{copula} white  \\
    (intended reading `The wall is white.')
\z
} 


\xbox{\textwidth}{
\ea\label{ex:contr:location}
\gll Tony (*asàdhaathang) Kluu\umb u=ka \\
      Tony \textsc{copula} Colombo=\textsc{loc} \\
    (intended reading `*Tony is in Colombo.', possible reading `Tony has arrived in Colombo.') 
\z
} 


While class inclusion as described above often does show the copula, it is by no means obligatory. The copula can be left out in all the sentences above without affecting truth conditions. This means that together with the sentence in  \xref{ex:func:profession}, \xref{ex:func:profession:contr} is also grammatical.

\xbox{\textwidth}{
\ea\label{ex:func:profession:contr}
\gll [Seelong=nang  duppang duppang anà-dhaathang  mlaayu] \zero{}  \textbf{oorang} \textbf{ikkang}. \\
      Ceylon=\textsc{dat} before before \textsc{past}-come Malay] {  }  man fish \\
    `The Malays who came to Ceylon very early were fishermen.'  
\z
}


It appears that the use of the copula is conditioned more by considerations of information structure than by syntax or semantics. Especially topic-comment structures seem to favour the use of the copula \citep[cf.][420]{LiEtAl1977cop}. The stretch preceding the \em (asà)dhaathang(apa) \em is the topic, while the following portion is the comment. In that sense, \xref{ex:func:profession:contr} without the copula is less felicitous than \xref{ex:func:profession}, where the topic-comment-structure is overtly indicated by \em (asà)dhaathang(apa)\em.


It is not necessary for the topic to be nominal. In \xref{ex:topic:dovulu}, we see an adverbial topic \trs{dovulu}{in former times}, the comment on which is preceded by \em abbisdhaathang\em, an idiolectal variant of \em asàdhaathang\em. In \xref{ex:topic:siini}, we have a spatial topic \trs{siini}{here} instead of a temporal topic.


\xbox{\textwidth}{
\ea\label{ex:topic:dovulu}
\gll \textbf{dovulu} abbisdhaathang       muuka thama-thuuthup \\
     earlier \textsc{copula} face \textsc{neg.irr}-close  \\
    `Earlier, the women would not cover their faces.' (K061026prs01)
\z
} 

\xbox{\textwidth}{
\ea\label{ex:topic:siini}
\gll \textbf{siini} dhaathangapa mixed, mixed community \\
     here copula mixed mixed community  \\
    `What we get here is a mixed, a mixed community.' (G051222nar04)
\z
} 


Occasionally, the comment can be quite elaborate as in the following two examples.

\xbox{\textwidth}{
\ea\label{ex:form:copula:special:missingargument2}
\gll Itthu abbisdhaathang {\em custard} {\em powder}=dering=jo arà-kirja. \\ % bf
      \textsc{dist} \textsc{copula} custard powder=\textsc{abl}=\textsc{emph} \textsc{non.past}-make \\
    `As for this one, it is such that it is made with custard powder.' (K061026rcp02,K081105eli02)
\z
}



\xbox{\textwidth}{
\ea
\ea
\gll Sepakthakrow=pe     {\em rules} dhaathangapa \\
     sepaktakrow=\textsc{poss} rules copula  \\
    `The Sepaktakrow rules are as follows:' 
\ex
\gll  inni     hathu  {\em badminton} {\em court}=ka arà-maayeng\\
      \textsc{prox} \textsc{indef} badminton court=\textsc{loc} \textsc{non.past}-play \\
    `You play it on a badminton court;'  
\ex
\gll  {\em game} hatthu itthe  {\em same} {\em measurement} ambel=apa;     {\em height}=le      {\em same}=jo\\
      game \textsc{indef} \textsc{dist} same measurement take=after height=\textsc{addit} same=\textsc{emph} \\
    `(For) a game, you take the same measurements, and then the height (of the net) is also the same.' (N060113nar05)
\z
\z
} 



Furthermore, use of the copula seems to be strengthened by the presence of loanwords in the sentence, as is already evident from the preceding example. Most of the Malays have an above average command of English for the Sri Lankan context, and it cannot be excluded that the use of lexical material from English (loanwords) activates grammatical structures of English as well, in this case the requirement for a copula in predications of class membership.
% The following three examples show very typical sentences, where the copula is found together with a loanword.
%  In my corpus of about 22,000 words, the copula occurs 26 times. In  19 of the cases it cooccurs with a loanword from English, while only 7 are in sentences without an English word. Overall figure for the prevalence of English words in the corpus are much lower.
% There are 3357 sentences without a loanword in the GD, and 818 include at least one loanword, mostly from English

\section{Where does the copula come from?}\label{sec:contactlgs}
The copula is both synchronically and diachronically analyzable as conjunctive participle of the lexical verb \trs{dhaathang}{come}. The first Malays who arrived in Ceylon in the 17$^{th}$ century communicated in Vehicular Malay, a trade language used in and around the Indonesian archipelago \citep{Smith2003timing, SmithEtAl2004}. The language has changed considerably in the last three centuries and has typologically converged towards the local languages Tamil and Sinhala \citep{SmithEtAl2004, SmithEtAl2006cll, Ansaldo2005ms, Ansaldo2008genesis,Nordhoff2009phd}. Given these historical facts, three possible origins for the copula can be suggested: Indonesian varieties of Malay, and Sinhala or Tamil. In this section, I will discuss the copula-like structures we find in these languages and the possibility of their influence om the development of Sri Lanka Malay.

\subsection{Copulas in Indonesian varieties}\label{sec:copulaCJI}
In Indonesian there exist two copulas, \em adalah \em and \em ialah\em, which are restricted to very formal varieties
% \citep[235]{Ewing2005}.\footnote{David Gil (p.c.) informs me that all the colloquial varieties that he is familiar with do not have this form, or any other copula for that matter.} 
 \citet[98f]{Sneddon2006CJI} informs us that

\begin{quote}
[t]he copulas \em adalah \em and \em ialah \em optionally occur in F[ormal ]I[ndonesian] to link a subject and nominal predicate, and sometimes an adjectival predicate [...] with \em adalah \em being considerably more common than \em ialah.\em\\
\em Adalah \em also occurs in C[olloquial ]J[akarta ]I[ndonesian], although it is very rare; in about half the texts it did not occur. In most of the others it occurred with very limited frequency. It was only in the meetings and in the two most formal interviews [...] that it occurred frequently. [...] It can be said that \em adalah \em is marked for formality while \em ialah, \em being entirely absent form CJI, is very highly marked for formality.
\end{quote}

The immigrants to Sri Lanka were mostly soldiers temporarily garrisoned in Batavia/Jakarta and spoke a very colloquial variety of Malay, remote from any formal standard \citep{Adelaar1991,SmithEtAl2004,Nordhoff2009phd}. It is unlikely that Formal Indonesian (or the equivalent historical variety) has had a significant impact on their speech. The sociolinguistic profile of Colloquial Jakarta Indonesian is closer to what would have been appropriate for the immigrants, although the relations between the colloquial varieties of the 17th century and the contemporary varieties are not exactly straightforward \citep{Grijns1991, AdelaarEtAl1996}. The copula \em adalah \em can be found in CJI although it is extremely rare. The following is an example of the copula in informal speech.

 

\xbox{\textwidth}{
\ea\label{ex:indonesian:adalah}
\gll Yang pasti bakal ngebikin gua terpesona banget \textbf{adalah} cewek dengan rambut panjang yang tergerai, terus anggun \textsc{CJI} \\
      which certain will make me enchanted very \textsc{copula} girl with hair long which flowing then elegant \\
    `What would certainly very much enchant me is an elegant girl with long flowing hair.' \citep[79]{Sneddon2006CJI}
\z
}

This specificational use of the copula in \xref{ex:indonesian:adalah} is similar to example \xref{ex:ident:JVP}, repeated as \xref{ex:indonesian:adalah:contr} for convenience.


\xbox{\textwidth}{
\ea \label{ex:indonesian:adalah:contr}
\gll [Itthu    vakthu kithang=nang nya-aada]     \textbf{asàdhaathang} ini      JVP katha hathu  {\em problem}. \\
     \textsc{dist} time \textsc{1pl}=\textsc{dat} \textsc{past}-exist \textsc{copula} \textsc{prox} JVP \textsc{quot} \textsc{indef} problem. \\
    `What we had at that time was the so-called JVP-problem.' (K051206nar10)
\z
}


While the functional properties of the copulas in CJI and SLM are similar, they are morphologically very different. The Indonesian form is composed of the existential plus the imperative/emphatic clitic, but the Sri Lankan form is composed of the verb `to come' and the conjunctive participle prefix \em asà-\em. It cannot be excluded that a construction involving a copula crossed the Bay of Bengal with the immigrants, but given the formal register with which this construction occurs in Indonesia, this is not very likely. An additional argument against a historical relatedness of the SLM and CJI constructions is the use of \trs{dhaathang}{come} in SLM. \em Dhaathang \em   would have had to replace the existential \em ada \em without any particular motivation. One reason for the replacement of the morphological composition of the copula could be influence from the adstrates. This will be discussed in the following section.

\subsection{Copulas in Sinhala and Tamil}
Colloquial Sinhala has no copula as such, but uses a predicative suffix \em -yi \em on adjectives ending with a vowel \xref{ex:sinh:yi:adj:v}. Adjectives which end in a consonant  \xref{ex:sinh:yi:adj:c}, and other word classes ending in any sound do not take this suffix \xref{ex:sinh:yi:n:c}\xref{ex:sinh:yi:n:v}.


\xbox{\textwidth}{
\ea\label{ex:sinh:yi:adj:v}
\gll mee amu miris hari s\ae ra-yi \textsc{Colloquial Sinhala}  \\
     prox raw chillis really strong-\textsc{yi}  \\
    `These green chillies are really strong.' \citep[794]{Gair2003}
\z
} 


\xbox{\textwidth}{
\ea\label{ex:sinh:yi:adj:c}
\gll mee dawasw\E l\E{} haal bohom\E{} gana\ng(*-yi) \textsc{Colloquial Sinhala} \\
     prox days rice very expensive(-\textsc{yi})  \\
    `These days (uncooked) rice is very expensive.' \citep[795]{Gair2003} 
\z
} 


\xbox{\textwidth}{
\ea\label{ex:sinh:yi:n:c}
\gll mam\E{} govi-yek(*-yi)  \textsc{Colloquial Sinhala}\\
     1\textsc{s} farmer-\textsc{indef.anim}(-\textsc{yi})  \\
    `I am a farmer.' \citep[241]{Gair1998}
\z
}


\xbox{\textwidth}{
\ea\label{ex:sinh:yi:n:v}
\gll ma-\dott\E{} niwaa\dotd u*(-yi) \textsc{Colloquial Sinhala}\\
     1s-\textsc{dat} vacation(-\textsc{yi})  \\
    `I'm on vacation.' \citep[794]{Gair2003}
\z
}
 
In Literary Sinhala, a similar suffix \em -mi \em is also used for nominal predications, as in \xref{ex:sinh:yi:classical}.


\xbox{\textwidth}{
\ea\label{ex:sinh:yi:classical}
\gll mama  goviyekmi \textsc{Literary Sinhala} \\
     1\textsc{s}-\textsc{nom} farmer.\textsc{nom}.\textsc{1sg} \\
    `I am a farmer.' \citep[242]{Gair1998}
\z
} 

There is some discussion of whether \em mi \em in should indeed be analyzed as a copula, or rather as an agreement suffix \citep[242]{Gair1998}. Whatever the right analysis of \em -mi \em and similar forms, it is clear that the variety of Sinhala the Malays were exposed to did not feature this form. The most copula-like element they are likely to have encountered is \em -yi\em, which does not suggest itself as possible model for the Sri Lanka Malay copula. Its form as a suffix is very different from the verbal form employed by Sri Lanka Malay, and its use for adjectival predications is precisely one where the use of the Sri Lanka Malay copula is not found.

In Tamil, the existential \em iru \em is sometimes analyzed as a copula. This verb is used for predications of existence as in \xref{ex:tamil:irukkiratu:exist} but also for locational predications as in \xref{ex:tamil:irukkiratu:loc}.


\xbox{\textwidth}{
\ea\label{ex:tamil:irukkiratu:exist}
\gll kumaar-ukku oru paiya\textipa{\textsubbar{n}} iru-kki\textipa{\textsubbar{r}}-aa\textipa{\textsubbar{n}} \textsc{Tamil} \\
     Kumar-\textsc{dat} one boy \textsc{iru}-\textsc{pres}-\textsc{3sm}  \\
    `Kumar has one boy/There exists one boy to Kumar.' \citep[188]{Lehmann1989tamil}
\z
} 

\xbox{\textwidth}{
\ea\label{ex:tamil:irukkiratu:loc}
\gll inta$\cdot$k\footnotemark{} kiraama-tt-il muu\textipa{\textsubbar{n}}\textipa{\textsubbar{r}}u koovil (iru-kki\textipa{\textsubbar{r}}-atu) \textsc{Tamil} \\
      this village-\textsc{obl}-\textsc{loc} three temple \textsc{iru}-\textsc{pres}-\textsc{3sn} \\
    `There are three temples in this village.' \citep[173]{Lehmann1989tamil}
\z
} 
\footnotetext{The dot $\cdot$ separates the grammatical word from the first part of a long consonant which is the result of certain morphosyntactic environments. The words in isolation would be \em inta \em and \em kiraama\em. One could also choose to write \em inta kkiraama \em to correctly represent pronunciation and orthography, but Lehmann chooses to leave the first part of the long consonant as the coda of the preceding syllable, although he separates it with a dot to indicate the special status.}

Uses for property word predications \xref{ex:tamil:irukkiratu:adj} and nominal predications \xref{ex:tamil:irukkiratu:n} are  ungrammatical.


\xbox{\textwidth}{
\ea\label{ex:tamil:irukkiratu:adj}
\gll pa\dott ippu mukkiyam (*irukki\textipa{\textsubbar{r}}atu) \textsc{Tamil} \\
     education importance (\textsc{iru}.\textsc{3sn}) \\
    `Education is important.' \citep[172]{Lehmann1989tamil}
\z
} 


\xbox{\textwidth}{
\ea\label{ex:tamil:irukkiratu:n}
\gll kumaar vakkiil (*irukki\textipa{\textsubbar{r}}atu/*irukki\textipa{\textsubbar{r}}aa\textipa{\textsubbar{n}}) \textsc{Tamil} \\
      Kumar lawyer (\textsc{iru}.\textsc{3sn}/\textsc{iru}.\textsc{3sm})\\
    `Kumar is a lawyer.' \citep[171]{Lehmann1989tamil}
\z
} 

While in \xref{ex:tamil:irukkiratu:adj} and \xref{ex:tamil:irukkiratu:n}  the English translational equivalent has a copula, it does not seem pertinent to grant the same status to the Tamil form. An analysis as an existential can cover both the meanings given in \xref{ex:tamil:irukkiratu:exist} and \xref{ex:tamil:irukkiratu:loc}, namely `A boy exists (for Kumar)' and `Temple exist in this village', while excluding the readings `Education exists importance' and `Kumar exists a lawyer'. This is precisely what we find in \xref{ex:tamil:irukkiratu:adj} and \xref{ex:tamil:irukkiratu:n}. It is true that \em iru \em is sometimes found in nominal predications as in \xref{ex:tamil:irukkiratu:adv}, but in that case, the noun is adverbialized by \em aakalaay\em, and the construction can actually be shown to have a locational reading as well \citep[cf.][174]{Lehmann1989tamil}.


\xbox{\textwidth}{
\ea\label{ex:tamil:irukkiratu:adv}
\gll kumaar vakkiil-aakalaay iru-nt-aa\textipa{\textsubbar{n}} \textsc{Tamil} \\
      Kumar lawyer-\textsc{adv} \textsc{iru}-\textsc{past}-\textsc{3sm}\\
    `Kumar was a lawyer.' \citep[174]{Lehmann1989tamil}
\z
} 

Whatever analysis one prefers for \em iru\em, this Tamil form seems to be closer related to the Sri Lanka Malay existential \em aada\em, which would be used for translations of \xref{ex:tamil:irukkiratu:exist} and \xref{ex:tamil:irukkiratu:loc}, given in \xref{ex:tamil:irukkiratu:slm:loc} and \xref{ex:tamil:irukkiratu:slm:exist}.\footnote{Note that \em aada \em in SLM is only an existential, whereas the historically related form in Colloquial Jakarta Indonesian \em adalah \em (Section \ref{sec:copulaCJI}) is a copula which can also be used for non-existential predications, e.g. as in \xref{ex:indonesian:adalah}.}


\xbox{\textwidth}{
\ea\label{ex:tamil:irukkiratu:slm:loc}
\gll Ini nigiri=ka thiiga koovil aada \textsc{SLM} \\
     \textsc{prox} village=\textsc{loc} three temple exist  \\
    `There are three (Hindu) temples in this village.'
\z
} 


\xbox{\textwidth}{
\ea\label{ex:tamil:irukkiratu:slm:exist}
\gll  Kumaar=nang hatthu aanak klaaki aada \textsc{SLM} \\
      Kumar=\textsc{dat} one child male exist \\
    `Kumar has one boy.' 
\z
} 

While \em iru \em and \em aada \em are quite similar, the parallels between \em iru \em and \em (asà)dhaathang(apa) \em are less striking, and it seems unlikely that Sri Lanka Malay emulated the Tamil form \em iru \em with \em (asà)dhaa\-thang(apa).\em

Sinhala and Tamil both have conjunctive participles, too. The conjunctive participle forms of the verb meaning `to come' in Sinhala (\em \ae villaa\em) and Tamil (\em vantu\em) are not used in a copular sense in these languages; they can only be used in the original reading of temporal anteriority.


\xbox{\textwidth}{
\ea
\ea
\gll  lamayaa pantiya=\dott a \textbf{\ae villaa},\\
 child class=\textsc{dat}  come.\textsc{cp}\\
`The child came to the class and/The child having come to class,'
\ex 
\gll liyuma livvaa \textsc{Sinhala}\\
      letter write\\
     `(he) wrote the letter.' \citep[161]{Karunatillake2004}
\z
\z
} 

\xbox{\textwidth}{
\ea
\ea
\gll naa\dotl ai kumaar \textbf{va-ntu},\\
 tomorrow Kumar come-\textsc{cp}\\
 `Tomorrow Kumar will come and/Kumar having come tomorrow,'
\ex
\gll e\.nka\dotl-ai$\cdot$c canti-pp-aa\textipa{\textsubbar{n}}  \textsc{Tamil} \\
     \textsc{1pl}-\textsc{acc} meet-\textsc{fut}-\textsc{3sm} \\
    `(he) will meet us.' \citep[267]{Lehmann1989tamil}
\z
\z
} 

To sum up, while the Malay ancestor language, Sinhala, and Tamil all have structures which come at least close to a copula, none of these structures is similar enough to the SLM structure as to suggest itself as the driving force behind the development of the copula in Sri Lanka Malay. It thus seems likely that the copula in Sri Lanka Malay is an independent development.

\section{The emergence of copulas}\label{sec:sources}
The most common etymological source for copulas is an existential \citep{Munro1977} or a positional verb \citep[94]{Stassen1997}. Stassen illustrates this with data from Kiowa gathered by Watkins, which are repeated here.\footnote{Stassen uses the 1980 PhD-thesis; in this article, I use the newer version \citep{Watkins1984} and add stripped diacritics   back in place.} In \xref{ex:intro:kiowa:cop:loc} we see the last reflexes of the original positional/locational use of \em do:\em. This locational use has given way to the general use of \em do: \em as a copula, which can be seen in \xref{ex:intro:kiowa:cop:cop}. In return, most of the original uses of \em do:  \em for locational purposes are now expressed by other morphemes, e.g.  \em -cél \em in \xref{ex:intro:kiowa:loc}. \em Do: \em has thus nearly fully completed the grammaticalization from locational verb to copula.


\xbox{\textwidth}{
\ea\label{ex:intro:kiowa:cop:loc}
\gll p'\textipa{\'O}$\cdot$ h\textipa{\textpolhook{\'e}}$\cdot$ gyà-d\textipa{\'O}$\cdot$ \textsc{Kiowa} \\
     moon gone \textsc{pl}-be  \\
    `There is (temporarily)  no moon.' \citep[216]{Watkins1984}
\z
}

\xbox{\textwidth}{
\ea\label{ex:intro:kiowa:cop:cop}
\gll té$\cdot$ k\textipa{\'O}ygú bà-d\textipa{\'O}$\cdot$ \textsc{Kiowa}  \\
     all Kiowa \textsc{2pl}-be \\
    `You are all Kiowas.' \citep[227]{Watkins1984}
\z
} 


\xbox{\textwidth}{
\ea\label{ex:intro:kiowa:loc}
\gll \textipa{\textsubdot{\'e}}$\cdot$g\textipa{\`O}  yí$\cdot$ \textipa{\'O}l \textipa{\textpolhook{\`e}}-cél kí$\cdot$còy-kyà  \textsc{Kiowa} \\
     here two hair \textsc{3dual}-be.in soup-in   \\
    `There are two pieces of hair in the soup.' \citep[211]{Watkins1984}
\z
}



The other common source for copulas are pronouns. This can be seen in the famous example from Chinese \citep{LiEtAl1977cop}.

\xbox{15cm}{
\ea
\gll jì yù qí sh\=eng yòu yù qi s\v\i{}, \textbf{shì} huò y\v e  \textsc{Archaic Chinese}\\
     already wish \textsc{3sg} live also wish \textsc{3sg} die this indecision \textsc{dcl}  \\
    `Wishing him to live and whishing him to die, this is indecision.' \citep[424]{LiEtAl1977cop}
\z
} 

The demonstrative \em shì \em in this  construction type was reanalyzed as a copula \em shì \em \citep[424]{LiEtAl1977cop}.



\xbox{\textwidth}{
\ea
\gll nèi-ge rén   \textbf{shì} xuésh\=eng \textsc{Modern Mandarin} \\
     that-\textsc{clf} man \textsc{cop} student \\
    `That man is a student' \citep[422]{LiEtAl1977cop}
\z
} 

Next to Chinese, this pattern is for instance found in Hebrew, Arabic and furthermore in a number of North American and African languages (\citet[77--91]{Stassen1997}).

With regard to Sri Lanka Malay, we can say that the existential \em aada \em is different from \em asàdhaathangapa \em and does not seem to be very much grammaticalized towards a copula. The deictics \em ini \em and \em itthu \em are not recruited for copular purposes either.
Changes from lexical meanings like `coming' to more grammatical meanings like `\textsc{copula}' have been investigated under the umbrella of grammaticalization. The verb `come' is a frequent source of a number of grammatical functions, so that an investigation
of the directions of development of this verb might shed light on what we find in Sri Lanka Malay.  \citet[92]{Stassen1997} notes that a minor class of sources for the grammaticalization of copulas includes motion verbs like `go' and `come', but unfortunately he does not cite languages illustrating the `come' case, so that it is not possible to compare the Sri Lanka Malay case to other purported instances of this path.

\section{Grammaticalization paths of COME}\label{sec:targets}
In the languages of the world, verbs meaning `to come' often undergo grammaticalization and can be used for a variety of grammatical functions. In French, \em venir de \em is used to express immediate past, but the venitive meaning is bleached: it is possible to say \em Jean vient de partir\em, literally `John comes from leaving', meaning that John has just left (and not that he has just come back). \citet[68--78]{HeineEtAl2002} list a number of grammaticalization targets for COME: consecutive, continuous, hortative, venitive, ablative, near past, benefactive, change-of-state, future, proximative, purpose. The copula is not found among them. The closest example to a copular use of COME I am aware of is the use as an auxiliary in passive constructions, as found in Italian. In Italian, the verb \em venire \em  can be used to form passives as in \xref{ex:italian}. Those passives always have a dynamic interpretation.\footnote{For some diachronic information on this development, see \citet{Michaelis1997venire} and \citet{Giacalone2000venire}}


\xbox{\textwidth}{
\ea\label{ex:italian}
\gll Era chiaro che i ragazzi veni-vano maltratta-t-i \textsc{Italian} \\
     was clear that the children come-\textsc{3pl.impft} illtreat-\textsc{ppl}-\textsc{pl}  \\
    `It was obvious [that] the children were being ill-treated.' \citep[282]{MaidenEtAl2000}
\z
} \\

As in the Sri Lanka Malay case, the verb meaning `to come' has lost its original venitive meaning. There is no motion towards the deictic center involved in \xref{ex:italian}. In Italian as in Sri Lanka Malay, the verb meaning `to come' is completely bleached and serves a purely grammatical function, which would in both cases be fulfilled in English by the copula `to be'. However, the similarities end here. In Italian, \em venire \em is used as an auxiliary for diathesis, which is part of a larger construction involving a past participle. Both the auxiliary and the participle agree in number with the subject of the sentence. In the Sri Lanka Malay case, the absence of other verbs in the sentence suggests that  \em asàdhaathang \em is not an auxiliary. There is no particple either, let alone agreement.  Furthermore, the Sri Lanka Malay copula typically surfaces in intransitive, non-verbal predications, while the Italian passive construction with \em venire \em requires at least a bivalent verb. The Italian construction has a dynamic reading, while in the Sri Lanka Malay cases, a stative reading is the norm. For naming uses for instance, a dynamic reading is completely excluded. This all suggests that the grammaticalization path of Sri Lanka Malay \trs{dhaathang}{come} is not the comparable to  Italian \em venire\em, nor to any other of the cases discussed in \citet{HeineEtAl2002}. With the knowledge we have today, it seems that this grammaticalization from COME to a copula is unique to Sri Lanka Malay.

\section{Possible grammaticalization paths}
In the preceding sections, I have ruled out inheritance of the copula construction, or emergence through language contact. I have also shown that the existing literature on grammaticalization paths  does not cover the phenomenon encountered in Sri Lanka Malay. Given this paucity of literature to hook on to, some speculation as to the possible development may be warranted. One possible path from COME to the copula would lead over an intermediate resultative stage. This resultative reading of come is found for instance  in English constructions like \xref{ex:dream}.

\ea\label{ex:dream} A dream come true\z

In this case, the original motion meaning of \em to come \em has been lost. There is no spatial motion involved. What remains is the change of state from A to A', which is also what we observe in motion events like \em John came home from work\em. In \xref{ex:dream}, the state of the dream changes from unrealized to realized/true.\footnote{Note also that the string \em come \em is present in English \em become\em, and \trs{venir}{come} is found in French \trs{devenir}{become} and several Romance cognates.} While in English, this resultative reading is restricted to a small set of words (\em *A man come tall, *A T-shirt come grey\em), in other languages, it has a wider application. In To'aba'ita, the verb \trs{mai}{come} has an ingressive/resultative meaning when used with a property concept as in \xref{ex:toabaita}.

\xbox{\textwidth}{
\ea\label{ex:toabaita}
\gll fanua-'e rodo na-mai \textsc{To'aba'ita} \\
     place-it:\textsc{pfv} be.dark \textsc{perf}-come \\
`It has become dark.' \citep[487]{Lichtenberk1991hetero}
\z
}

\citet[74]{HeineEtAl2002} note that ``[t]his grammaticalization appears to be particularly common in pidgin and creole languages.'' While it is not entirely clear whether Sri Lanka  Malay should be seen as a Creole \citep{SmithEtAl2006cll} or not \citep{Ansaldo2008genesis,Nordhoff2009phd} it is clear that the immigrants to Sri Lanka spoke Vehicular Malay, a trade language for inter-ethnic communication which has been likened to a pidgin \citep{AdelaarEtAl1996}. As such, it is interesting to note that Fa d'Ambu Creole Portuguese \xref{ex:FdACP}, Guayanese Creole French \xref{ex:GCF}, and Seychelles Creole French \xref{ex:SCF}   show the resultative reading of the verb meaning `to come' as well.\footnote{Furthermore, this is also found in Ghanaian Pidgin English \citep{Huber1996} and Chinook Jargon \citep[236]{Grant1996}.}

\xbox{\textwidth}{
\ea\label{ex:FdACP}
\gll tyipa bi sxa dual eli kumu pasa \textsc{Fa d'Ambu} \\
     stomach come \textsc{part} hurt 3s east surpass \\
`His stomach hurt; he had eaten to much.' \citep[159]{Post1992fadambu} 
\z
}
\xbox{\textwidth}{
\ea\label{ex:GCF}
\gll i vini malad \textsc{Guayanese Creole French}\\
     3s come sick \\
`He has become sick.' \citep[90]{Corne1971gcf}
\z
}

\xbox{\textwidth}{
\ea\label{ex:SCF}
\gll i n vin larpâter \textsc{Seychelles Creole French}\\
     he \textsc{cpl} come surveyor\\
`He became a surveyor.' \citep[80]{Corne1977scf}
\z
}

The grammaticalization from `to come' to a resultative marker is the first step in a grammaticalization chain I want to propose. The second case involves the change from resultative to stative. For this case, especially the Seychelles Creole French example in \xref{ex:SCF} is interesting. It shows a great resemblance to one of the main uses of the SLM copula, namely membership in a (professional class) as in \xref{ex:func:profession} or also in \xref{ex:SCF:SLM} below.

\xbox{\textwidth}{
\ea\label{ex:SCF:SLM}
\gll Umma=pe       baapa  dhaathangapa  hathu  {\em inspector}          {\em of} {\em police}. \textsc{SLM} \\
     mother=\textsc{poss} father \textsc{copula} \textsc{indef} inspector of police. \\
    `My mother's father was an inspector of police.' (N060113nar03)
\z
}

The difference between   the Seychelles Creole French example in \xref{ex:SCF} and the Sri Lanka Malay example in \xref{ex:SCF:SLM} is the change from resultative to stative. The Seychellois sentence is necessarily resultative, while the SLM sentence is necessarily stative. In the Sri Lanka Malay sentence, the information is not about the grandfather having passed an exam, thereby becoming an inspector of police (unlike the Seychellois surveyor). Rather, it is the first mentioning of the grandfather in this text, indicating his profession as a stative predicate without forcing a resultative interpretation.

However, it is logically necessary that the grandfather must have made the step from prospective inspector of police to certified inspector of police at one point in time since it is unlikely that he was born inspector. With many predicates there is a reciprocal entailment between the stative reading and a prior ingressive/resultative one. This is especially true for predicates of profession: If you become a professor, you are then a professor; if you are a priest, you must have become a priest before.\footnote{A similar case of resultative entailing a subsequent stative reading is Latin \em cognovi \em in the perfect, literally `I have learnt', which very often means `I know now' with a present reading.} The second step of the grammaticalization chain is then the change from a resultative reading entailing stative to a pure stative reading. This is likely to have been pioneered by predicates of profession, which are still one of the most frequent predicates found with the copula in Sri Lanka Malay. Only later would other predicates where a resultative meaning is impossible (e.g. sex) have followed suit. The proposed development can then be summarized as follows:

\small
\ea come$>$resultative$>$resultative entailing stative$>$copula (professions)$>$copula (other predicates)\z
\normalsize
 

\section{Conclusion}
Sri Lanka Malay has a copula, which is mainly used for naming and predications of class membership, although some information structure use especially with regard to topic and comment is also found. This copula is derived from the conjunctive participle form of the verb \trs{dhaathang}{come}. Neither the formal properties of the copula, nor its functional uses show parallels to any of the possible languages of origin, i.e. Indonesian varieties, Sinhala, or Tamil. This suggests an independent development. This independent development, however, is not of a type attested in the literature on grammaticalization and seems to be a new grammaticalization path not found up to now. While further research is needed, a promising hypothesis is the development through an intermediate stage of a resultative marker, as found in other contact languages like Seychelles Creole French. This marker would then lose its resultative reading for predicates of profession and become a semantically void copula, which then spreads to other types of intransitive predication.
 

\bibliographystyle{natuva}
\bibliography{ansaldo,asw,creole,india,phon,malay,sinhala,tamil,nordhoff,lankahist,wortarten,grammars}


\end{document}
