\documentclass[a4paper,12pt]{article}
%-------------general packages-----------
\usepackage[utf8]{inputenc}
\usepackage[T1]{fontenc}
% \newcommand{\changefont}[3]{\fontfamily{#1} \fontseries{#2} \fontshape{#3} \selectfont}
% \usepackage{anysize}
% \marginsize{4.5cm}{4.5cm}{4.8cm}{4.8cm}
% \usepackage[british]{babel}
%------------frontmatter packages--- 

%-----------mainmatter packages-----
\usepackage{gb4e,tipa,lingsty,ipashortcuts}
\usepackage[authoryear]{natbib}
\bibpunct[:]{(}{)}{,}{a}{}{,}
\setlength{\bibsep}{0.05cm}

%opening
\title{Having come to be a copula in Sri Lanka Malay -- un unusual grammaticalization path}
\author{Sebastian Nordhoff}

\begin{document}

\maketitle

\begin{abstract}

\end{abstract}

\section{Introduction}
Languages with copulas are widely known in Europe and beyond, but languages without a copula are also frequently encountered. If languages without a copula develop one over time, the source is normally either a former existential or a deictic. In this paper, I want to highlight yet another source for a copula, namely the participle of the verb COME, which has grammaticalized into a copula in Sri Lanka Malay.


\section{The emergence of copulas}
stassen 1997:92f
\subsection{Copulas from existentials}

hengeveld 1992:237ff

\subsection{Copulas from pronons}

\xbox{\textwidth}{
\ea
\gll jì yú qí sh\=eng yoù yù qi s\v\i{} \textbf{shí} huò y\v e \\
     already wish \textsc{3sg} live also wish \textsc{3sg} die this indecision \textsc{dcl}  \\
    `Wishing him to live and whishing him to die, this is indecision.' \citep[425]{LiEtAl1977cop}
\z
} 

The deictic \em shí \em in this  construction type was reanalyzed as a copula \em shì\em.



\xbox{\textwidth}{
\ea
\gll c\v\i{} bái wù \textbf{shì} hé d\v eng\\
     this white thing \textsc{cop} what kind \\
    `What kind of stuff is this white thing?' \citep[425]{LiEtAl1977cop}
\z
} 

Hebrew Arabic (Li), furthermore a number of Farican lgs \citep[56]{Pustet}

\section{Sri Lanka Malay}
\section{The copula in Sri Lanka Malay}
\subsection{The conjuctive participle and the copula}
The Sri Lanka Malay verb can take a number of tense prefixes, one of them being the conjunctive participle. This form is used to indicate that the event depicted in the clause thus marked is prior to the event in the main clause. In this sense, it is similar to an English construction like \em Having done X, having done Y, finally Z happened\em. An illustration of this form is given in \xref{ex:intro:cp:pirrang}.

\xbox{\textwidth}{
\ea\label{ex:intro:cp:pirrang}
 \ea 
 \gll Oorang pada \textbf{asà-}pìrrang, \\
	 man \textsc{pl} \textsc{cp}-wage.war  \\
	`After having waged war'
	\ex
 \gll derang=nang \textbf{asà-}banthu, \\
	\textsc{3pl}=\textsc{dat} \textsc{cp}-help\\
	`and after having helped them'
	\ex
	\gll siini=jo su-cii\u n\u ggal. \\ % bf
	here=\textsc{emph} \textsc{past}-settle\\
	`the people settled down right here.' (K051222nar03)
	\z
\z
}


% K060108nar02.txt:\tx Kandinang   asadhaathang Kandika    asakaavingapa       itthunang
% K060108nar02.txt:\tx blaakangjo    kithang pada anabissar 

Alternatives to the form with \em asà- \em shown in \xref{ex:intro:cp:pirrang} are the form with \em =apa \em shown in \xref{ex:intro:cp:apa} and the form with both \em asà- \em and \em =apa\em, shown in \xref{ex:intro:cp:asaapa}. These three forms seem to be functionally equivalent.

\xbox{\textwidth}{
\ea\label{ex:intro:cp:apa}
\gll Oorang pada  thiikam=\textbf{apa}  oorang pada=nang thee\u mbak=\textbf{apa}  se=dang bannyak  creeveth pada su-aada. \\
      man \textsc{pl} stab=after man \textsc{pl}=\textsc{dat} shoot=after \textsc{1s=dat} much trouble \textsc{pl} \textsc{past}-exist \\
    `People were stabbed, people were shot, I had a lot of problems.' (K051213nar01)
\z
}

\xbox{\textwidth}{
\ea\label{ex:intro:cp:asaapa}
  \ea 
  \gll Siithu \textbf{asà}-blaajar=\textbf{apa}, \\
  there \textsc{cp}-learn=after\\
  `After having learned there,'
 \ex
  \gll thaaun nnamblas=ka         se   skuul \textbf{asà}-luppas=\textbf{apa}, \\
  year sixteen=\textsc{loc} \textsc{1s} school \textsc{cp}-leave=after\\
  `after having left the school at 16,'
	\ex 
  \gll pukuran asà-caari        anà-pii. \\ % bf
  work \textsc{cp}-find \textsc{past}-go\\
  `I looked for work and went (away).' (K060108nar01)
  \z
  \z
}

The Sri Lanka Malay copula is built around the lexical verb \trs{dhaathang}{come} in its conjunctive participle form. All three forms mentioned above are possible for the copular use. The following three examples illustrate this.

\xbox{\textwidth}{
\ea\label{ex:intro:copula:asadhaathang}
\gll Se=ppe    naama \textbf{asà}dhaathang  Cintha Sinthani. \\
     \textsc{1s}=\textsc{poss} name \textsc{copula} Chintha Sinthani. \\
    `My name is Chintha Sinthani.' (B060115prs04)
\z
}

\xbox{\textwidth}{
\ea\label{ex:intro:copula:dhaathangapa}
\gll Se=ppe    baapa  dhaathang\textbf{apa}  Jinaan Samath. \\
     \textsc{1s}=\textsc{poss} father \textsc{copula} Jinaan Samath. \\
    `My father was Jinaan Samath.' (N060113nar03)
\z
}


\xbox{\textwidth}{
\ea\label{ex:intro:copula:asadhaathangapa}
\gll {\em Estate}=pe {\em field} {\em officer} \textbf{asà}dhaathang\textbf{apa}  kithang=pe     kaake\\
     estate=\textsc{poss} field officer \textsc{copula} \textsc{1pl}=\textsc{poss} grandfather. \\
    `The estate field officer was our grandfather.' (N060113nar03)
\z
}



The three examples above illustrate the use of \em (asà)dhaathang(apa) \em as a copula. Any of the three forms mentioned could also be used as a conjunctive particple proper when combined with \trs{dhaathang}{come}, as shown in the following three examples.


\xbox{\textwidth}{
\ea\label{ex:intro:cp:asdhaathang} 
\gll Moonyeth pada=le \textbf{asà}-dhaathang creeveth  athi-kaasi. \\
       monkey \textsc{pl}=\textsc{addit} \textsc{cp}-come trouble \textsc{irr}$\div$give\\
    `The monkeys would certainly go and cause (some other) trouble.'  (K070000wrt01)
\z    
}
% 
% \xbox{\textwidth}{
% \ea\label{ex:intro:cp:asdhaathang} 
% \gll [Itthu nigiri=deri     \textbf{as-dhaathang}    anà-thii\u n\u ggal  oorang pada]=jo       kithang. \\
%      \textsc{dist} country=\textsc{abl} \textsc{cp}-come \textsc{past}-stay man \textsc{pl}=\textsc{emph} 1\textsc{pl}. \\
%     `The people who have come from those countries and stayed (here) are we.' (K051222nar03)
% \z
% }



\xbox{\textwidth}{
\ea\label{ex:intro:cp:dhaathangapa} 
\ea
\gll oorang mlaayu siithu=dering  dhaathang=apa\\
     man malay there=\textsc{abl} come=after \\
    `The Malay men came from there and'  
\ex
\gll  cinggala  raaja=nang=le  anà-banthu\\
      Sinhala king=dat=addti past-help \\
    `helped the Sinhalese king.' (K051206nar04)

\z
\z
} 


 

% 
% 
% \xbox{\textwidth}{
% \ea\label{ex:intro:cp:dhaathangapa} 
% \ea
% \gll Malay {\em regiment} hatthu \textbf{dhaathangapa}    \\
%       Malay regiment \textsc{indef} come=\textsc{cp} \\
%     `A Malay regiment came and'
% \ex
% \gll  sini=ka {\em settle}=\textbf{apa} 
%       here=\textsc{loc} settle=\textsc{cp} \\
%     `settled down here and'
% \ex
% \gll itthu=nang      blaakang bannyak oorang pada siini duuduk=\textbf{apa} \\
%      dist=\textsc{dat} after many man \textsc{pl} here stay=\textsc{cp}  \\
%     `after that many people stayed here and ' (nosource)
% \ex
% \gll {\em forces} pada=nang   su-ambel \\
% forces \textsc{pl}=\textsc{dat} \textsc{past}-take\\
% `the army took them (in).'(G051222nar03)
% \z
% \z
% } 
 


\xbox{\textwidth}{
\ea\label{ex:intro:cp:asadhaathangapa} 
\ea
\gll incayang  islaam=nang   \textbf{asà-dhaathang=apa}  \\
     \textsc{3s}.\textsc{polite} Islam=\textsc{dat} \textsc{cp}-come=\textsc{cp} \\
    `He came to Islam.'
\ex
\gll inni     siigith=nang    asà-dhaathang \\
     \textsc{prox} mosque=\textsc{dat} \textsc{cp}-come  \\
    `and came to the mosque.'
\ex
\gll giini    girja \\
     like.this make  \\
    `and does like this (=observes Islamic rites).' (K051220nar01)
\z
\z
} 
 

The interpretation as involving an event of coming is mandatory in examples \xref{ex:intro:cp:asdhaathang}-\xref{ex:intro:cp:asadhaathangapa}, but impossible in examples \xref{ex:intro:copula:asadhaathang}-\xref{ex:intro:copula:asadhaathangapa}. We are thus dealing with polysemy of the forms \em (asà)dhaathang(apa)\em.
 
\subsection{Functional properties of the copula}
adj, nom, ident, spec,  exist, quantm 

halliday1994:119 hengeveld1992:73ff 

eq
loc
exist



Copulas can have different functions in the languages of the world. \citet{Pustet} mentions X, Y and Z. \citet{Hengeveld1992nvp} adds G, H, and I. In Sri Lanka Malay, the copula is mainly used for naming people and for indications of group membership. The naming use is the most common one and illustrated in \xref{ex:intro:copula:asadhaathang} and \xref{ex:intro:copula:dhaathangapa} above.

Apart from naming, the copula is most frequently used for indicating membership in ethnic,  professional or gender groups. These three uses are given in \xref{ex:func:ethnicgroup}-\xref{ex:func:gender}

\xbox{\textwidth}{
\ea\label{ex:func:ethnicgroup}
\ea
\gll Se=ppe    {\em daughter-in-law}=pe {\em mother} \textbf{asàdhaathang} \textbf{bìnggaali}$_{ethnic group}$. \\
      \textsc{1s}=\textsc{poss} daughter-in-law=\textsc{poss} mother \textsc{copula} Bengali \\
    `My daughter-in-law's mother is Bengali.'
\ex
\gll Ithukapang       {\em daughter-in-law}=pe     father \textbf{asàdhaathang} \textbf{mlaayu}$_{ethnic group}$. \\
      then daughter-in-law=\textsc{poss} father \textsc{copula} Malay \\
    `Then my daughter-in-law's father is Malay.' (K051206nar08)
\z
\z
}

\xbox{\textwidth}{
\ea\label{ex:func:profession}
\gll [Seelong=nang  duppang duppang anà-dhaathang  mlaayu] \textbf{asà}dhaathang \textbf{oorang} \textbf{ikkang}$_{profession}$. \\
      Ceylon=\textsc{dat} before before \textsc{past}-come Malay] \textsc{copula} man fish \\
    `The Malays who came to Ceylon very early were fishermen.' (K060108nar02)
\z
}

\xbox{\textwidth}{
\ea\label{ex:func:gender}
\gll Kàthama aanak dhaathangapa \textbf{klaaki}$_{sex}$. \\
      first child \textsc{copula} male \\
    `My oldest child is a boy.' (G051222nar01)
\z
}


Another frequently found use is for indicating kin relations. Example  \xref{ex:func:kin} illustrates this.

\xbox{\textwidth}{
\ea\label{ex:func:kin} 
\gll Baapa=pe      umma   \textbf{asà}dhaathang  kaake=pe           \textbf{aade}$_{kin}$. \\
    father=\textsc{poss} mother \textsc{copula} grandfather=\textsc{poss} younger.sibling. \\
    `My paternal grandmother was my grandfather's younger sister.' (K051205nar05)
\z
}


The three forms with \em asà-, =apa \em and \em asà- ... =apa \em appear to be used indiscriminately for the functions indicated above.

All the functions mentioned above have in common that they indicate group membership. A more remotely related function is the use of the copula in identificational or specificational sentences. The following two examples show the use of the copula to further specify or identify the nature of a problem.

\xbox{\textwidth}{
\ea 
\gll [Itthu    vakthu kithang=nang nya-aada]     \textbf{asàdhaathang} ini      JVP katha hathu  {\em problem}. \\
     \textsc{dist} time \textsc{1pl}=\textsc{dat} \textsc{past}-exist \textsc{copula} \textsc{prox} JVP \textsc{quot} \textsc{indef} problem. \\
    `What we had at that time was the so-called JVP-problem.' (K051206nar10)
\z
}

\xbox{\textwidth}{
\ea
\gll suda karang [kithang=nang   aada  problem] dhaathangapa kithang=pe     aanak pada mlaayu thama-oomong \\
       thus now \textsc{1pl}=\textsc{dat} exist problem \textsc{copula} \textsc{1pl}=\textsc{poss} child \textsc{pl} Malay \textsc{neg.nonpast}-speak\\
    `So, the problem we are having now is that our children do not speak Malay.' (G051222nar01)
\z
} 

\subsection{Differences to similar functions}
There are some functions fulfilled by copulas in other languages which are not expressed by the SLM copula. These include the use of the copula with adjectives or locations.

\citet{Stassen1997} distinguishes the following types of intransitive predications: actions, class membership, states, locations. The last three domains are relatively common domains for the use of the copula. In Sri Lanka Malay, only predications of class membership frequently take the copula, states and locations are normally not triggers for the use of the copula. The following examples illustrate this.


\xbox{\textwidth}{
\ea
\gll Theembok (*asdhaathang) puuthi \\
     wall copula white  \\
    (intended reading `The wall is white.')
\z
} 


\xbox{\textwidth}{
\ea
\gll Tony (*asdhaathang) Kluu\umb u=ka \\
      Tony \textsc{copula} Colombo=\textsc{loc} \\
    (intended reading `*Tony is in Colombo.', possible reading `Tony has arrived in Colombo.') 
\z
} 


Class inclusion as described above can show the copula, but it is by no means obligatory. The copula can be left out in all the sentences above without affecting truth conditions. This means that instead of the sentence in  \xref{ex:func:profession}, \xref{ex:func:profession:contr} is also grammatical.

\xbox{\textwidth}{
\ea\label{ex:func:profession:contr}
\gll [Seelong=nang  duppang duppang anà-dhaathang  mlaayu]   \textbf{oorang} \textbf{ikkang}. \\
      Ceylon=\textsc{dat} before before \textsc{past}-come Malay]   man fish \\
    `The Malays who came to Ceylon very early were fishermen.'  
\z
}


It appears that the use of the copula is conditioned more by considerations of information structure than by syntax or semantics. Especially topic-comment structures seem to trigger the use of the copula. The stretch preceding the \em (asà)dhaathang(apa) \em is the topic, while the following portion is the comment.


It is not necessary for the topic to be nominal. In \xref{ex:topic:dovulu}, we see an adverbial topic \trs{dovulu}{in former times}, the comment on which is preceded by \em abbisdhaathang\em. In \xref{ex:topic:siini}, we have a spatial topic \trs{siini}{here} instead of a temporal topic.


\xbox{\textwidth}{
\ea\label{ex:topic:dovulu}
\gll dovulu abbisdhaathang       muuka thama-thuuthup \\
     earlier \textsc{copula} face \textsc{neg.irr}-close  \\
    `Earlier, the women would not cover their faces.' (K061026prs01)
\z
} 

 


\xbox{\textwidth}{
\ea\label{ex:topic:siini}
\gll siini dhaathangapa mixed, mixed community \\
     here copula mixed mixed community  \\
    `What we get here is a mixed, a mixed community.' (G051222nar04)
\z
} 


Occasionally, the comment can be quite elaborate as in the following two examples.

\xbox{\textwidth}{
\ea\label{ex:form:copula:special:missingargument2}
\gll Itthu abbisdhaathang {\em custard} {\em powder}=dering=jo arà-kirja. \\ % bf
      \textsc{dist} \textsc{copula} custard powder=\textsc{abl}=\textsc{emph} \textsc{non.past}-make \\
    `As for this one, it is such that it is made with custard powder.' (K061026rcp02,K081105eli02)
\z
}



\xbox{\textwidth}{
\ea
\ea
\gll Sepakthakrowpe     rules dhaathangapa \\
     sepaktakrow=poss rules copula  \\
    `The Sepaktakrow rules are as follows:' 
\ex
\gll  inni     hathu  badminton courtka arà-maayeng\\
      \textsc{prox} \textsc{indef} badminton court=\textsc{loc} \textsc{non.past}-play \\
    `You play it on a badminton court;'  
\ex
\gll  game hatthu itthe  same measurement ambel=apa;     height=le      same=jo\\
      game \textsc{indef} \textsc{dist} same measurement take=after height=\textsc{addit} same=\textsc{emph} \\
    `(For) a game, you take the same measurements, and then the height (of the net) is also the same.' (N060113nar05)
\z
\z
} 



Furthermore, use of the copula seems to be strengthened by the presence of loanwords in the sentence, as is already evident from the preceding example. Most of the Malays have an above average command of English for the Sri Lankan context, and it cannot be excluded that the use of lexical material from English (loanwords) activates grammatical structures of English as well, in this case the requirement for a copula in predications of class membership. The following three examples show very typical sentences, where the copula is found together with a loanword.
%  In my corpus of about 22,000 words, the copula occurs 26 times. In  19 of the cases it cooccurs with a loanword from English, while only 7 are in sentences without an English word. Overall figure for the prevalence of English words in the corpus are much lower.
% There are 3357 sentences without a loanword in the GD, and 818 include at least one loanword, mostly from English

\section{Where does the copula come from?}
The copula is both synchronically and diachronically analyzable as conjunctive participle of the lexical verb \trs{dhaathang}{come}. Sri Lanka Malay is the language spoken by the descendents of soldiers, exiles, convicts and slaves brought by the colonial powers of the Dutch and the British between 1650 and 1850 \citep{Bichsel,Hussainmiya1990,Nordhoff2009phd}. These immigrants communicated in Vehicular Malay, a trade language used in and around the Indonesian archipelago \citep{Smith2003timing, SmithEtAl2004}. The language has changed considerably in the last 3 centuries and has typologically converged towards the local languages Tamil and Sinhala \citep{SmithEtAl2004, SmithEtAl2006cll, Ansaldo2005ms, Ansaldo2008genesis,Nordhoff2009phd}, becoming a member of the Sri Lankan Sprachbund \citep{Bakker2006}. Given this origin, three possible origins for the copula can be suggested: Indonesian varieties of Malay, and Sinhala or Tamil.

\subsection{Copulas in Indonesian varieties}
In Indonesian there exists a copula \em adalah\em, which is restricted to very formal varieties \citep[235]{Ewing2005}.

All in all, the Sri Lanka Malay copula is quite distinct both in its formal as well as in its functional properties. While the Indonesian form is composed of the existential plus the imperative clitic, the Sri Lankan form is composed of the verb `to come' and the conjunctive participle prefix \em asà-\em.  The Indonesian form is used for X, Y, and Z in formal contexts, but the Sri Lankan form is used for A, B, and C, and does not seem to be tied to a particular register. It does thus not seem likely that the Sri Lankan form is a direct development of the Indonesian form.

\subsection{Copulas in Sinhala and Tamil}
Sinhala has no copula as such, but uses a predicative suffix \em -yi \em on adjectives ending with a vowel \xref{ex:sinh:yi:adj:v}. Adjectives which end in a consonant  \xref{ex:sinh:yi:adj:c}, and other word classes ending in any sound do not take this suffix \xref{ex:sinh:yi:n}.


\xbox{\textwidth}{
\ea\label{ex:sinh:yi:adj:v}
\gll mee amu miris hari s\ae ra-yi  \\
     prox raw chillis really strong-\textsc{yi}  \\
    `These green chillies are really strong.' \citep[794]{Gair2003}
\z
} 


\xbox{\textwidth}{
\ea\label{ex:sinh:yi:adj:c}
\gll mee dawasw\E l\E{} haal bohom\E{} gana\ng(*-yi)  \\
     prox days rice very expensive(-\textsc{yi})  \\
    `These days (uncooked) rice is very expensive.' \citep[795]{Gair2003} 
\z
} 


\xbox{\textwidth}{
\ea\label{ex:sinh:yi:n}
\gll gun\E siri mahatt\E ya apee iskolee  mul guruw\E r\E ya(*-yi)  \\
     Gunasiri gentleman our school.\textsc{loc} head teacher(-\textsc{yi})  \\
    `Mr Gunasiri is the head teacher of our school.' \citep[793]{Gair2003}
\z
} 

In classical Sinhala, this suffix was also used for nominal predications, as in \xref{ex:sinh:yi:classical}.


\xbox{\textwidth}{
\ea\label{ex:sinh:yi:classical}
\gll  \\
       \\
    `.' (nosource)
\z
} 

However, the period of literary Sinhala ended before the Malay arrived.   Sinhala \em -yi \em does not suggest itself as possible model for the SLM copula. Its form as a suffix is very different from the verbal form employed by Sri Lanka Malay, and its use for adjectival predications is precisely one where the use of the Sri Lanka Malay copula is excluded.

In Tamil, the existential \em iru \em is sometimes analyzed as a copula. This verb is used for predications of existence as in \xref{ex:tamil:irukkiratu:exist} but also for locational predications as in \xref{ex:tamil:irukkiratu:loc}.


\xbox{\textwidth}{
\ea\label{ex:tamil:irukkiratu:exist}
\gll kumaar-ukku oru paiya\textipa{\textsubbar{n}} iru-kki\textipa{\textsubbar{r}}-aa\textipa{\textsubbar{n}}\\
     Kumar-\textsc{dat} one boy \textsc{iru}-\textsc{pres}-\textsc{3sm}  \\
    `Kumar has one boy/There exists one boy to Kumar.' \citep[]{Lehmann1989tamil}
\z
} 

\xbox{\textwidth}{
\ea\label{ex:tamil:irukkiratu:loc}
\gll inta$\cdot$k\footnotemark{} kiraama-tt-il muu\textipa{\textsubbar{n}}\textipa{\textsubbar{r}}u koovil (iru-kki\textipa{\textsubbar{r}}-atu)  \\
      this village-\textsc{obl}-\textsc{loc} three temple \textsc{iru}-\textsc{pres}-\textsc{3sn} \\
    `There are three temples in this village.' \citep[173]{Lehmann1989tamil}
\z
} 
\footnotetext{The dot $\cdot$ separates the grammatical word from the first part of a long consonant which is the result of certain morphosyntactic environments. The words in isolation would be \em inta \em and \em kiraama\em. One could also choose to write \em inta kkiraama \em to correctly represent pronunciation and orthography, but Lehmann chooses to leave the first part of the long consonant as the coda of the preceding syllable, although he separates it with a dot to indicate the special status.}

Uses for property word predications \xref{ex:tamil:irukkiratu:adj} and nominal predications \xref{ex:tamil:irukkiratu:n} are  ungrammatical.


\xbox{\textwidth}{
\ea\label{ex:tamil:irukkiratu:adj}
\gll pa\dott ippu mukkiyam (*irukki\textipa{\textsubbar{r}}atu)\\
     education importance (\textsc{iru}.\textsc{3sn}) \\
    `Education is important.' \citep[172]{Lehmann1989tamil}
\z
} 


\xbox{\textwidth}{
\ea\label{ex:tamil:irukkiratu:n}
\gll kumaar vakkiil (*irukki\textipa{\textsubbar{r}}atu/irukki\textipa{\textsubbar{r}}aa\textipa{\textsubbar{n}})\\
      Kumar lawyer (\textsc{iru}.\textsc{3sn}/\textsc{iru}.\textsc{3sm})\\
    `Kumar is a lawyer.' \citep[171]{Lehmann1989tamil}
\z
} 

While in \xref{ex:tamil:irukkiratu:adj} and \xref{ex:tamil:irukkiratu:n}  the English translational equivalent has a copula, it does not seem pertinent to grant the same status to the Tamil form. An analysis as an existential can cover both the meanings given in \xref{ex:tamil:irukkiratu:exist} and \xref{ex:tamil:irukkiratu:loc}, namely `A boy exists (for Kumar)' and `Temple exist in this village', while excluding the readings `Education exists importance' and `Kumar exists a lawyer'. This is precisely what we find in \xref{ex:tamil:irukkiratu:adj} and \xref{ex:tamil:irukkiratu:n}. It is true that \em iru \em is sometimes found in nominal predications as in \xref{ex:tamil:irukkiratu:adv}, but in that case, the noun is adverbialized by \em aakalaay\em, and the construction can actually be shown to have a locational reading as well \citep[174]{Lehmann1989tamil}.


\xbox{\textwidth}{
\ea\label{ex:tamil:irukkiratu:adv}
\gll kumaar vakkiil-aakalaay iru-nt-aa\textipa{\textsubbar{n}}  \\
      Kumar lawyer-\textsc{adv} \textsc{iru}-\textsc{past}-\textsc{3sm}\\
    `Kumar was a lawyer.' \citep[174]{Lehmann1989tamil}
\z
} 



Whatever analysis one prefers for \em iru\em, this Tamil form seems to be closer related to the Sri Lanka Malay form \em aada\em, which would be used for translations of \xref{ex:tamil:irukkiratu:exist} and \xref{ex:tamil:irukkiratu:loc}, given in \xref{ex:tamil:irukkiratu:slm:loc} and \xref{ex:tamil:irukkiratu:slm:exist}.


\xbox{\textwidth}{
\ea\label{ex:tamil:irukkiratu:slm:loc}
\gll Ini nigiri=ka thiiga koovil aada \\
     \textsc{prox} village=\textsc{loc} three temple exist  \\
    `There are three (Hindu) temples in this village.'
\z
} 


\xbox{\textwidth}{
\ea\label{ex:tamil:irukkiratu:slm:exist}
\gll  Kumaar=nang hatthu aanak klaaki aada\\
      Kumar=dat one child male exist \\
    `Kumar has one boy.' 
\z
} 

The parallels between \em iru \em and \em (asà)dhaathang(apa) \em are less striking, and it seems unlikely that Sri Lanka Malay emulated the Tamil form \em iru \em with \em (asà)dhaathang(apa).\em

Sinhala and Tamil both have conjunctive participles, too. The conjunctive particple forms of the verb meaning `to come' in Sinhala (\ae villaa) and Tamil (vantu) are not used in a copular sense in these languages; they can only be used in the original reading of temporal anteriority.


\xbox{\textwidth}{
\ea
\gll  lamayaa pantiya=\dott a \ae villaa, liyuma livvaa \\
      child class=dat  come.cp letter write\\
    `The child came to the class and wrote the letter/The child having come to class, he wrote a letter.' \citep[161]{Karunatillake2004}
\z
} 

\xbox{\textwidth}{
\ea
\gll naa\dotl ai kumaar va-ntu e\.nka\dotl-ai$\cdot$c canti-pp-aa\textipa{\textsubbar{n}}  \\
     tomorrow Kumar come-\textsc{cp} \textsc{1pl}-\textsc{acc} meet-\textsc{fut}-\textsc{3sm}  \\
    `Tomorrow Kumar will come and meet us/Having come tomorrow, Kumar will meet us.' \citep[267]{Lehmann1989tamil}
\z
} 

To sum up, while the Malay ancestor language, Sinhala, and Tamil all have structures which come at least close to a copula, none of these structures is similar enough to the SLM structure as to suggest itself as the driving force behind the development of the copula in Sri Lanka Malay. It thus seems likely that the copula in Sri Lanka Malay is an independent development.


\section{Grammaticalization paths of COME}
In the languages of the world, verbs meaning `to come' often undergo grammaticalization and can be used for a variety of grammatical functions. In French, \em venir de \em is used to express immediate past, but the venitive meaning is bleached: it is possible to say \em Jean vient de partir\em, literally `John comes from leaving', meaning that John has just left (and not that he has just come back). In Italian legalese, the verb \em venire \em in the future form \em verrà \em can be used to form impersonal passives as in \trs{ogni abuso verrà punito}{Every abuse will be punished.} This suggests a possible semantic connection between COME and auxiliaries. Even in English, the sentence \em Every abuse will come to be punished \em could be uttered in such a context. 

Heine and Kuteva


\section{Conclusion}

\bibliographystyle{natuva}
\bibliography{ansaldo,asw,creole,india,phon,malay,sinhala,tamil,nordhoff,lankahist,wortart}


\end{document}
