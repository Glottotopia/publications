\documentclass[12pt]{article}
\usepackage{libertine}
\usepackage{booktabs}
\usepackage{url}
\urlstyle{same}
\usepackage[
	bookmarks=true,bookmarksopen=true,bookmarksopenlevel=1,%
	bookmarksdepth=5,
	bookmarksnumbered=true,
	hyperindex=true,%
	breaklinks=true,
	draft=false,
	plainpages=false,
	pdfusetitle=true,  % puts author and title in automatically, maybe only in final mode?
	pdfkeywords={}
	%ps2pdf=true
	]{hyperref}
	
\hypersetup{colorlinks=false, pdfborder={0 0 0}}  
\usepackage[margin=25mm]{geometry}
\usepackage{todonotes}
% \usepackage{natbib}
\usepackage{pdfpages}
\usepackage[
	natbib=true,
	style=apa, 
	backend=biber	
	]{biblatex}
\title{Outcompeting Gold: sharing basics for collaborative approaches to publishing}
\author{Sebastian Nordhoff, Language Science Press}
\date{} 
\bibliography{roa}

\usepackage{moreverb}
\immediate\write18{texcount -inc -incbib -sum main.tex|grep "Sum count"|sed s/Sum/Word/ > /tmp/wordcount.tex}
\newcommand\wordcount{\verbatiminput{/tmp/wordcount.tex}}
\begin{document}

\maketitle

\begin{abstract}
Traditional reader-pays model have shielded book-based sciences from the onslaught of commodification because of the completely opaque financing and cost structure. 
The advent of author-pays models will associate costs more closely with individual titles. This poses two threats to book-based disciplines:
1) high unit prices and  
2) standardization. In order for book-based disciplines to stand up to an article-based system inspired by STEM, it is necessary to set up alternative publishing platforms which are closer in spirit to the humanities. Nevertheless, these platforms have to have solid calculations if they are to succeed. Sharing business data can help these alternative platforms to get a clearer picture of their respective cost structure. 
\end{abstract}

\section{Introduction}
Books are very heterogeneous

\section{Flow of money in the traditional system}
Under the traditional setup modelled on the distribution of printed copies, the funding of book publishing is completely intransparent and opaque. Different governments, foundations or companies give money to their universities, which give it to their libraries, which buy printed or digital copies. The number of total copies bought by libraries worldwide is typically not disclosed by publishers; the same is true of discounts or package deals. This means that there is no way for an interested researcher, accountant, or other person to get to know how much money society at large has transferred to the publisher in total for making a given title available.  This informational asymmetry allows the publisher to realize to make profits on the one hand. They can easily justify the higher prices of their titles as compared to the competition by alleging that they are catering to a more specialized audience and sell less copies. That assertion cannot be questioned without access to their data, which they do not provide. On the other hand, it also allows for cross-subsidies between different titles. The recent textbook by a leading scholar can generate a surplus which can help establish that experimental series on a burgeoning new subject, which has an uncertain future. 

\section{The Golden Threat}
The obscene profit margins and sales practices in the journal market have led to calls for author-pays models instead of reader-pays models. Next to additional advantages like world-wide royalty-free access to scientific literature, this will also allow an easier analysis of how much is being paid for the publication of an actual written piece of research. Instead of 235 libraries world-wide sharing the cost, it will be exactly one funder who will pay processing fees for exactly one publication. That funder will know precisely what they have funded, and they can check whether that item conforms to their specifications and requirements. Given that they are the unique ``customer'' for that work's publication, they can also easily compare its kind, quality, and price to other items they have funded. Article-based metrics\footnote{``Article-based metrics''
 is a term coined in opposition to the Journal Impact Factor and as such does not apply naturally to books. However, it is getting used in a wider sense meaning ``non-journal-based metrics'', including works which are not articles in the strict sense, e.g. contributions to edited volumes, or even books.
 } furthermore allow funders to see the uptake of the publications they have financed. This leads to a ``processing charge market'', where funders can compare offers and prices. In the recent years, we have seen surveys and analyses of costs and quality of different publishers. Lists are prepared to signal low-quality (``predatory'') publishers, whose fees will not be taken over by funders,  and high-quality (vetted) publishers, for which processing charges can be recovered. Research associations are monitoring the distribution and development of fees (mean, median, minimum, maximum, rise) %TODO Marco
 and are extrapolating future costs. A natural follow-up would be service providers who help you select the ``right'' offer depending on your walllet, funding requirements and desired reach, similar to specialised search engines for holiday trips for instance. This suggests that we are actually moving towards a developed processing charges market, where different players are catering to different customer segments. In the top tier, demand will be unelastic. This means that an increase in prices of the top brands of a field will not lead to decrease in demand.  In the lower ties, demand should be more elastic. If the fees for a publication with ``Studies in X'' become too high, authors could substitute by switching to another outlet, say, ``Advances in X'' instead, for a lower fee.\footnote{I do not want to claim that the market for journal articles is perfect in any way, or that it is actually a desirable setup to begin with. My goal is to contrast journal articles with books, and my review of the ``functioning market'' for articles has to be interpreted in that context.} 
 
 One necessary precondition for this to work is that the items are substitutable on the publishers' side as well. The author from Mumbai withdrew her manuscript and went with someone else? Well, follow-up with the one from Stockholm. Journal article production is done at an industrial scale, with specified workflows, and you can simply plug in some other manuscript and expect about the same effort required. 
 
 This is where books differ, and this is where the humanities differ. The shortest book with Language Science Press has 97 pages, the longest has over 800 pages. Obviously, the former took less time on our side than the latter, and we could not simply substitute one for the other. There are argumentative books with a lot of prose text and only examples from languages with a Latin alphabet (less demanding), and there are books with complex tables, charts, and lots of examples from languages with special typographic requirements, e.g. special characters, stacked diacritics, non-Latin scripts, or a mix of left-to-right and righ-to-left passages (more demanding). In that sense, book publishers are operating more on an artisanal scale than on an industrial scale, and every book is made-to-order. This heterogeneity of books implies that they cannot be easily substituted for each other (in the technical economic sense).
 
 
 
 
 
 
 
 
 
 
 
 
 
 

\subsection{The Golden Threat}
            Kleine Fächer
            unintended consequences
\subsubsection{metrics}
                    maximize IF per hour
                        publion
\subsubsection{green}
                    STEM
\subsubsection{Fehlanreize}
            bang for the buck
\subsubsection{only fund medical research}
\subsubsection{short sighted}
\subsubsection{commodification}
                    difference to neoliberalism
\subsubsection{strategy}
                    reduce costs
                    standardize
                    same as for articles, no novel research
\subsubsection{competition between resarch communities}
            evolutionary stability
\subsubsection{all nice people will not work}
\section{collaboration}
\subsection{general}
            Tragic of the anticommons
            Wikipedia
            Apache
            rival goods
\subsection{langsci}
            community proofreading 
            book stands
            open source
            tex-code
\subsection{open accountancy}
            accounting != capitalism
\subsubsection{book stands}
\subsubsection{oapen ch}
\subsubsection{wikipedia}
            contribution margin
\subsection{best practices}
\subsection{criteria promoting cooperation}
\subsection{Platinum road}
            Glossa
            LangSci
            SciPost
\section{fair OA principles}
\section{conclusion}
\subsection{brands}
\subsection{self-exploitation is not a business model}
\subsection{make yourself replaceable}


\sloppy
\printbibliography


\wordcount
\end{document}
