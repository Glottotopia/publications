\documentclass[12pt]{article}
\usepackage{libertine}
\usepackage{booktabs}
\usepackage{url}
\urlstyle{same}
\usepackage[
	bookmarks=true,bookmarksopen=true,bookmarksopenlevel=1,%
	bookmarksdepth=5,
	bookmarksnumbered=true,
	hyperindex=true,%
	breaklinks=true,
	draft=false,
	plainpages=false,
	pdfusetitle=true,  % puts author and title in automatically, maybe only in final mode?
	pdfkeywords={}
	%ps2pdf=true
	]{hyperref}
	
\hypersetup{colorlinks=false, pdfborder={0 0 0}}  
\usepackage[margin=25mm]{geometry}
\usepackage{todonotes}
% \usepackage{natbib}
\usepackage{pdfpages}
\usepackage[
	natbib=true,
	style=apa, 
	backend=biber	
	]{biblatex}
\title{Outcompeting Gold: sharing basics for collaborative approaches to publishing}
\author{Sebastian Nordhoff, Language Science Press}
\date{} 
\bibliography{roa}

\usepackage{moreverb}
\immediate\write18{texcount -inc -incbib -sum main.tex|grep "Sum count"|sed s/Sum/Word/ > /tmp/wordcount.tex}
\newcommand\wordcount{\verbatiminput{/tmp/wordcount.tex}}
\begin{document}

\maketitle

\begin{abstract}
Traditional reader-pays model have shielded book-based sciences from the onslaught of commodification because of the completely opaque financing and cost structure. 
The advent of author-pays models will associate costs more closely with individual titles. This poses two threats to book-based disciplines:
1) high unit prices and  
2) standardization. In order for book-based disciplines to stand up to an article-based system inspired by STEM, it is necessary to set up alternative publishing platforms which are closer in spirit to the humanities. Nevertheless, these platforms have to have solid calculations if they are to succeed. Sharing business data can help these alternative platforms to get a clearer picture of their respective cost structure. 
\end{abstract}

\section{Introduction}
The Open Access movement started in the sciences, in disciplines with a more or less standardized article format as the main type of publication. The proposed target models for future publication platforms (Green Road, Golden Road) are shaped by the requirements and usages of these pioneer communities. However, the demands of the humanities, and other disciplines where the book is a predominant format, are different, and the models crafted in the sciences cannot be transferred without further ado. The Green Road has not seen significant followership in the humanities, but also the uptake of the Golden Road is slow. In this paper, I will discuss in how far the competition introduced by Gold models is actually detrimental to research. I then show how collaborative approaches have advantages over the Golden Road, and argue that sharing business figures between collaborative approaches should become customary. 

\section{Flow of money in the traditional system}
Under the traditional setup modelled on the distribution of printed copies, the funding of book publishing is completely intransparent and opaque. Different governments, foundations or companies give money to their universities, which give it to their libraries, which buy printed or digital copies. The number of total copies bought by libraries worldwide is typically not disclosed by publishers; the same is true of discounts or package deals. This means that there is no way for an interested researcher, accountant, or other person to get to know how much money society at large has transferred to the publisher in total for making a given title available.  This informational asymmetry allows the publisher to realize to make profits on the one hand. They can easily justify the higher prices of their titles as compared to the competition by alleging that they are catering to a more specialized audience and sell less copies. That assertion cannot be questioned without access to their data, which they do not provide. On the other hand, it also allows for cross-subsidies between different titles. The recent textbook by a leading scholar can generate a surplus which can help establish that experimental series on a burgeoning new subject, which has an uncertain future. 

\section{Competition and The Golden Threat}\label{sec:goldenthreat}
The obscene profit margins and sales practices in the journal market have led to calls for author-pays models instead of reader-pays models. Next to additional advantages like world-wide royalty-free access to scientific literature, this will also allow an easier analysis of how much is being paid for the publication of an actual written piece of research. Instead of 235 libraries world-wide sharing the cost, it will be exactly one funder who will pay processing fees for exactly one publication. That funder will know precisely what they have funded, and they can check whether that item conforms to their specifications and requirements. Given that they are the unique ``customer'' for that work's publication, they can also easily compare its kind, quality, and price to other items they have funded. Article-based metrics\footnote{``Article-based metrics''
 is a term coined in opposition to the Journal Impact Factor and as such does not apply naturally to books. However, it is getting used in a wider sense meaning ``non-journal-based metrics'', including works which are not articles in the strict sense, e.g. contributions to edited volumes, or even books.
 } furthermore allow funders to see the uptake of the publications they have financed. This leads to a ``processing charge market'', where funders can compare offers and prices. In the recent years, we have seen surveys and analyses of costs and quality of different publishers. Lists are prepared to signal low-quality (``predatory'') publishers, whose fees will not be taken over by funders,  and high-quality (vetted) publishers, for which processing charges can be recovered. Research associations are monitoring the distribution and development of fees (mean, median, minimum, maximum, rise) %TODO Marco
 and are extrapolating future costs. A natural follow-up would be service providers who help you select the ``right'' offer depending on your walllet, funding requirements and desired reach, similar to specialised search engines for holiday trips for instance. This suggests that we are actually moving towards a developed processing charges market, where different players are catering to different customer segments. In the top tier, demand will be unelastic. This means that an increase in prices of the top brands of a field will not lead to decrease in demand.  In the lower tiers, demand should be more elastic. If the fees for a publication with ``Studies in X'' become too high, authors could substitute by switching to another outlet, say, ``Advances in X'' instead, for a lower fee.\footnote{I do not want to claim that the market for journal articles is perfect in any way, or that it is actually a desirable setup to begin with. My goal is to contrast journal articles with books, and my review of the ``functioning market'' for articles has to be interpreted in that context.} 
 
 One necessary precondition for this to work is that the items are substitutable on the publishers' side as well. The author from Mumbai withdrew her manuscript and went with someone else? Well, follow-up with the one from Stockholm. Journal article production is done at an industrial scale, with specified workflows, and you can simply plug in some other manuscript and expect about the same effort required. 
 
 This is where books differ, and this is where the humanities differ. The shortest book with Language Science Press has 97 pages, the longest has over 800 pages. Obviously, the former took less time on our side than the latter, and we could not simply substitute one for the other. There are argumentative books with a lot of prose text and only examples from languages with a Latin alphabet (less demanding), and there are books with complex tables, charts, and lots of examples from languages with special typographic requirements, e.g. special characters, stacked diacritics, non-Latin scripts, or a mix of left-to-right and righ-to-left passages (more demanding). In that sense, book publishers are operating more on an artisanal scale than on an industrial scale, and every book is made-to-order. This heterogeneity of books implies that they cannot be easily substituted for each other (in the technical economic sense).
 
 If ever there was price-based competition on the book market with a flat fee and a fixed cap, what would the consequences be? For one, publishers would opt for the run-of-the mill books. No more Old Syriac script,%TODO
 no more hand-crafted syntactic representations, no more adaptation of Hebrew script to the needs of Yiddish, no more labour of love. A fixed page count of 300. Fit your stuff in there, even if this means distortion or simplification. And, two: more books! Rather than an exhaustive biography of Humboldt, have one about his youth, one about his early years, and one about his old age. Same work, treble the revenue! The result will be non-distinctive books of equal appearance, all of about the same size, all equally mainstream, and all equally dull.\footnote{I recognise the merit of standardized presentation of methods and findings in the experimental sciences, which make the appreciation of a novel experiment much easier. But this approach cannot be transferred to interpretive sciences.}
 
 After the nefarious effects of the Impact Factor, article processing charges will lead to journal articles becoming even more a commodity.\footnote{Commodities are defined as  goods which are essentially interchangeable (``fungible'').} While this streamlining is already bad for articles and article-based research, it will be disastrous for books and might as well spell doom on certain disciplines as a whole, which happen to have modes of publishing research which cannot be made to fit under one figure, the quote for the book processing charges. 
 
 Another aspect, psychological this time, is that books require more effort than an article, and hence higher fees which can easily end up in the five-digit range \citep{abc}. This might be the same cost as for 7 articles, but since the fee is to be paid in one go, it appears enormous and is generally beyond any financing cap of open access funds. Articles can fly under the radar and go unnoticed, but the large body of research which is a book, will draw attention, and people will ask whether this expenditure is really necessary. 
 
\section{Collaboration}
Every couple of years, German newspapers feature stories about how students of law and students of theology are found to be the top culprits of displacing or hiding books in research libraries so that their fellow students cannot access them, giving the hiders a comparative advantage when it comes to exams. If your reaction to this is bewilderment or repulsion, this obviously shows that your moral compass is well calibrated, but it also shows that our expectations for researchers are actually not to compete, but to collaborate. At the beginning of their career, most researches should be led by their individual quest for knowledge, not by their desire to surpass their competitors. It is only within the university that that initial goal becomes perverted and competition rears its ugly head. This is partly due to perverted incentive structures, which make competition a more promising individual strategy even if collaboration would be more advantageous for society. In analogy to the well-known Tragedy of the Commons, this has been called the Tragedy of the Anti-Commons \citep{Heller1998}. Software development has been plagued by the same problem for a long time,  but the success of Open Source as witnessed by Apache or Firefox, and the transfer of its principles to other domains such as Wikipedia or OpenStreetMaps shows that collaboration as a quality of the human being is actually more powerful than reductionist models based on homo oeconomicus would predict.

The question is now how we can set up a system which will help make flourish that human instinct of collaboration and cooperation. 
I propose three main aspects.

\subsection{Attribution and recognition}
Every small contribution towards a publication should be acknowledged and recognised. A book is a complex achievement. There are very complex issues, such as typesetting non-Latin scripts, and there are tasks requiring less erudition, e.g. checking whether each table is referenced in the text or not. Researchers at different stages of their career can contribute to the areas most suitable for them, the junior ones to easier tasks, the senior ones for the advanced ones. This works if the book is seen as not the work of one author alone, but rather as a joint project of a research community. At Language Science Press, we crowd-source proofreading to the community. There have been books with more than 30 community proofreaders, who are all acknowledged in the book and in our Hall of Fame. Instead of the book being a sword which a sole author wields in competing with his rival authors, the book becomes more of a house where different people contribute in building something for the whole discipline. This only works because everybody believes they are not doing it for themselves, but for the greater good, and they want to be recognised for abiding to this. 

\subsection{Brands}
After this rather poetic depiction, I now turn to a more strategic viewpoint: measures have to be taken that the work of the community remain the community's and stay out of the commercial sphere. This means that the project has to recognised as having a name giving symbolic capital: a brand. That brand has to be protected by registering it, and this registration has to be done by a non-commercial entity.\footnote{See \# of the Fair Open Access Principles, \url{https://www.fairopenaccess.org}.}


\subsection{Calculations}
Going even further down the rabbit hole of business administration, in order to stand up to the Golden Threat described in Section~\ref{sec:goldenthreat}, collaborative projects have to make sure they live to tell the tale. The main risks to confront here are commercial bankruptcy due to sloppy book-keeping or planning (or none at all!) and emotional burnout due to self-exploitation. Burnout can actually be said to be directly related to commercial bankruptcy: if your business is healthy, there is no need for you to work overtime. If your calculations and expectations are unrealistic and muddled, on the other hand, the desire to compensate by working late hours might arise, leading to burn out. For your own emotional and social well-being, it is good to know that your figures are in order. 

I am belabouring the obvious here because very often, calculations are put in one bag together with commodification, neoliberalism, and capitalism as a whole, and then, that bag is being beaten with the biggest clubs you have. This blurs important distinctions. The ownership of the means of production (capitalism) is logically independent from knowing your business figures. You could have only state-owned businesses and still need business figures. Minimising intervention of the state in the economy (neoliberalism) is also logically indepedent of business figures. A heavily socialist state can still have business figures. And commodification, finally, is also different from business figures, although less so than the former two, although in a more subtle way: business figures relate to a business; commodification is a process at the level of society.

I will give one example to illustrate why calculations are important for non-profits as well: suppose you want to finance your project by a mix of donations, paid access to pdf at the rate of 2 EUR/pdf and print sales. Your figures show you that the median donation is 2 EUR. But for that, you have to maintain an infrastructure for book keeping, promotion, and ``donor support'' which costs more than the donations contribute. Second, your figures show that paid access to the pdf return 10000 EUR a year, but you have to hire IT staff to get your access control, authentication and billing in order, which costs more than those 10000 EUR. Your print margin is OK. Your project will be better off if you simply disable donations and make the pdf completely free without a paywall. Those two sources of revenue cost more than they contribute. 

In the context of the OpeAire project ``Full disclosure: replicable strategies for book publications supplemented with empirical data'' releases its business model, it's business figures, a detailed editable spreadsheet, and a ``cook book''.\footnote{\url{https://github.com/langsci/opendata}} The idea is to show calculations and figures for a sample publisher and thus share insights with other non-profit publishing platforms. I term this approach ``Open Accountancy''. According to \citet{Caux2017}, this makes the difference between Gold publishers (with APCs), platinum publishers (without APCs) and Palladium publishers (no APCs + disclosure of business data). 

Our calculations based on 4 years of operation between 2014 and 2018 suggest that the average cost of a book with Language Science Press is between 3500 and 5000 EUR. This is considerably less than the information on cost prices of books we get from other sources. Ithaca, OAPEN %TODO
even astonishingly so. There might be a reason for this: Language Science Press does not spend money on measures to prevent people from accessing the content (paywalls, logins, etc), which come with an overhead of authentication, billing, marketing, and user support. This actually makes the operations much leaner, and thus more cost-effective, proving Björn Brembs conjecture ``Are we paying US\$ 200 per article just for paywalls?''\footnote{\url{http://bjoern.brembs.net/2014/07/are-we-paying-us3000-per-article-just-for-paywalls}}

 
\section{Outcompeting Gold}            
In the current research landscape, scientists have a choice between different outlets for their publications. They could go for a traditional reader-pays publishing house; they could try to find funds for Gold processing charges; or they could opt for a community based publisher such as Language Science Press, the Open Library of Humanities\footnote{\url{https://www.openlibhums.org}} or SciPost.\footnote{\url{https://scipost.org/}}  The former two types are set up in a competitive mode; the latter type is set up in a collaborative mode. I have argued that when we pitch the competitive mode against the collaborative mode, bizarrely enough, the collaborative mode will be more cost-effective, and paradoxically, collaboration as a principle outcompetes competition. 
 

\sloppy
\printbibliography


\wordcount
\end{document}
